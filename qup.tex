\chapter{Al-Fatihah (The Opening)}
\begin{Arabic}
\Huge{\centerline{\basmalah}}\end{Arabic}
\flushright{\begin{Arabic}
\quranayah[1][1]
\end{Arabic}}
\flushleft{\begin{hindi}
अल्लाह के नाम से जो रहमान व रहीम है।
\end{hindi}}
\flushright{\begin{Arabic}
\quranayah[1][2]
\end{Arabic}}
\flushleft{\begin{hindi}
तारीफ़ अल्लाह ही के लिये है जो तमाम क़ायनात का रब है।
\end{hindi}}
\flushright{\begin{Arabic}
\quranayah[1][3]
\end{Arabic}}
\flushleft{\begin{hindi}
रहमान और रहीम है।
\end{hindi}}
\flushright{\begin{Arabic}
\quranayah[1][4]
\end{Arabic}}
\flushleft{\begin{hindi}
रोज़े जज़ा का मालिक है।
\end{hindi}}
\flushright{\begin{Arabic}
\quranayah[1][5]
\end{Arabic}}
\flushleft{\begin{hindi}
हम तेरी ही इबादत करते हैं, और तुझ ही से मदद मांगते है।
\end{hindi}}
\flushright{\begin{Arabic}
\quranayah[1][6]
\end{Arabic}}
\flushleft{\begin{hindi}
हमें सीधा रास्ता दिखा।
\end{hindi}}
\flushright{\begin{Arabic}
\quranayah[1][7]
\end{Arabic}}
\flushleft{\begin{hindi}
उन लोगों का रास्ता जिन पर तूने इनाम फ़रमाया, जो माअतूब नहीं हुए, जो भटके हुए नहीं है।
\end{hindi}}
\chapter{Al-Baqarah (The Cow)}
\begin{Arabic}
\Huge{\centerline{\basmalah}}\end{Arabic}
\flushright{\begin{Arabic}
\quranayah[2][1]
\end{Arabic}}
\flushleft{\begin{hindi}
अलीफ़ लाम मीम
\end{hindi}}
\flushright{\begin{Arabic}
\quranayah[2][2]
\end{Arabic}}
\flushleft{\begin{hindi}
(ये) वह किताब है। जिस (के किताबे खुदा होने) में कुछ भी शक नहीं (ये) परहेज़गारों की रहनुमा है
\end{hindi}}
\flushright{\begin{Arabic}
\quranayah[2][3]
\end{Arabic}}
\flushleft{\begin{hindi}
जो ग़ैब पर ईमान लाते हैं और (पाबन्दी से) नमाज़ अदा करते हैं और जो कुछ हमने उनको दिया है उसमें से (राहे खुदा में) ख़र्च करते हैं
\end{hindi}}
\flushright{\begin{Arabic}
\quranayah[2][4]
\end{Arabic}}
\flushleft{\begin{hindi}
और जो कुछ तुम पर (ऐ रसूल) और तुम से पहले नाज़िल किया गया है उस पर ईमान लाते हैं और वही आख़िरत का यक़ीन भी रखते हैं
\end{hindi}}
\flushright{\begin{Arabic}
\quranayah[2][5]
\end{Arabic}}
\flushleft{\begin{hindi}
यही लोग अपने परवरदिगार की हिदायत पर (आमिल) हैं और यही लोग अपनी दिली मुरादें पाएँगे
\end{hindi}}
\flushright{\begin{Arabic}
\quranayah[2][6]
\end{Arabic}}
\flushleft{\begin{hindi}
बेशक जिन लोगों ने कुफ़्र इख़तेयार किया उनके लिए बराबर है (ऐ रसूल) ख्वाह (चाहे) तुम उन्हें डराओ या न डराओ वह ईमान न लाएँगे
\end{hindi}}
\flushright{\begin{Arabic}
\quranayah[2][7]
\end{Arabic}}
\flushleft{\begin{hindi}
उनके दिलों पर और उनके कानों पर (नज़र करके) खुदा ने तसदीक़ कर दी है (कि ये ईमान न लाएँगे) और उनकी ऑंखों पर परदा (पड़ा हुआ) है और उन्हीं के लिए (बहुत) बड़ा अज़ाब है
\end{hindi}}
\flushright{\begin{Arabic}
\quranayah[2][8]
\end{Arabic}}
\flushleft{\begin{hindi}
और बाज़ लोग ऐसे भी हैं जो (ज़बान से तो) कहते हैं कि हम खुदा पर और क़यामत पर ईमान लाए हालाँकि वह दिल से ईमान नहीं लाए
\end{hindi}}
\flushright{\begin{Arabic}
\quranayah[2][9]
\end{Arabic}}
\flushleft{\begin{hindi}
खुदा को और उन लोगों को जो ईमान लाए धोखा देते हैं हालाँकि वह अपने आपको धोखा देते हैं और कुछ शऊर नहीं रखते हैं
\end{hindi}}
\flushright{\begin{Arabic}
\quranayah[2][10]
\end{Arabic}}
\flushleft{\begin{hindi}
उनके दिलों में मर्ज़ था ही अब खुदा ने उनके मर्ज़ को और बढ़ा दिया और चूँकि वह लोग झूठ बोला करते थे इसलिए उन पर तकलीफ देह अज़ाब है
\end{hindi}}
\flushright{\begin{Arabic}
\quranayah[2][11]
\end{Arabic}}
\flushleft{\begin{hindi}
और जब उनसे कहा जाता है कि मुल्क में फसाद न करते फिरो (तो) कहते हैं कि हम तो सिर्फ इसलाह करते हैं
\end{hindi}}
\flushright{\begin{Arabic}
\quranayah[2][12]
\end{Arabic}}
\flushleft{\begin{hindi}
ख़बरदार हो जाओ बेशक यही लोग फसादी हैं लेकिन समझते नहीं
\end{hindi}}
\flushright{\begin{Arabic}
\quranayah[2][13]
\end{Arabic}}
\flushleft{\begin{hindi}
और जब उनसे कहा जाता है कि जिस तरह और लोग ईमान लाए हैं तुम भी ईमान लाओ तो कहते हैं क्या हम भी उसी तरह ईमान लाएँ जिस तरह और बेवकूफ़ लोग ईमान लाएँ, ख़बरदार हो जाओ लोग बेवक़ूफ़ हैं लेकिन नहीं जानते
\end{hindi}}
\flushright{\begin{Arabic}
\quranayah[2][14]
\end{Arabic}}
\flushleft{\begin{hindi}
और जब उन लोगों से मिलते हैं जो ईमान ला चुके तो कहते हैं हम तो ईमान ला चुके और जब अपने शैतानों के साथ तनहा रह जाते हैं तो कहते हैं हम तुम्हारे साथ हैं हम तो (मुसलमानों को) बनाते हैं
\end{hindi}}
\flushright{\begin{Arabic}
\quranayah[2][15]
\end{Arabic}}
\flushleft{\begin{hindi}
(वह क्या बनाएँगे) खुदा उनको बनाता है और उनको ढील देता है कि वह अपनी सरकशी में ग़लत पेचाँ (उलझे) रहें
\end{hindi}}
\flushright{\begin{Arabic}
\quranayah[2][16]
\end{Arabic}}
\flushleft{\begin{hindi}
यही वह लोग हैं जिन्होंने हिदायत के बदले गुमराही ख़रीद ली, फिर न उनकी तिजारत ही ने कुछ नफ़ा दिया और न उन लोगों ने हिदायत ही पाई
\end{hindi}}
\flushright{\begin{Arabic}
\quranayah[2][17]
\end{Arabic}}
\flushleft{\begin{hindi}
उन लोगों की मिसाल (तो) उस शख्स की सी है जिसने (रात के वक्त मजमे में) भड़कती हुईआग रौशन की फिर जब आग (के शोले) ने उसके गिर्दों पेश (चारों ओर) खूब उजाला कर दिया तो खुदा ने उनकी रौशनी ले ली और उनको घटाटोप अंधेरे में छोड़ दिया
\end{hindi}}
\flushright{\begin{Arabic}
\quranayah[2][18]
\end{Arabic}}
\flushleft{\begin{hindi}
कि अब उन्हें कुछ सुझाई नहीं देता ये लोग बहरे गूँगे अन्धे हैं कि फिर अपनी गुमराही से बाज़ नहीं आ सकते
\end{hindi}}
\flushright{\begin{Arabic}
\quranayah[2][19]
\end{Arabic}}
\flushleft{\begin{hindi}
या उनकी मिसाल ऐसी है जैसे आसमानी बारिश जिसमें तारिकियाँ ग़र्ज़ बिजली हो मौत के खौफ से कड़क के मारे अपने कानों में ऊँगलियाँ दे लेते हैं हालाँकि खुदा काफ़िरों को (इस तरह) घेरे हुए है (कि उसक हिल नहीं सकते)
\end{hindi}}
\flushright{\begin{Arabic}
\quranayah[2][20]
\end{Arabic}}
\flushleft{\begin{hindi}
क़रीब है कि बिजली उनकी ऑंखों को चौन्धिया दे जब उनके आगे बिजली चमकी तो उस रौशनी में चल खड़े हुए और जब उन पर अंधेरा छा गया तो (ठिठके के) खड़े हो गए और खुदा चाहता तो यूँ भी उनके देखने और सुनने की कूवतें छीन लेता बेशक खुदा हर चीज़ पर क़ादिर है
\end{hindi}}
\flushright{\begin{Arabic}
\quranayah[2][21]
\end{Arabic}}
\flushleft{\begin{hindi}
ऐ लोगों अपने परवरदिगार की इबादत करो जिसने तुमको और उन लोगों को जो तुम से पहले थे पैदा किया है अजब नहीं तुम परहेज़गार बन जाओ
\end{hindi}}
\flushright{\begin{Arabic}
\quranayah[2][22]
\end{Arabic}}
\flushleft{\begin{hindi}
जिसने तुम्हारे लिए ज़मीन का बिछौना और आसमान को छत बनाया और आसमान से पानी बरसाया फिर उसी ने तुम्हारे खाने के लिए बाज़ फल पैदा किए पस किसी को खुदा का हमसर न बनाओ हालाँकि तुम खूब जानते हो
\end{hindi}}
\flushright{\begin{Arabic}
\quranayah[2][23]
\end{Arabic}}
\flushleft{\begin{hindi}
और अगर तुम लोग इस कलाम से जो हमने अपने बन्दे (मोहम्मद) पर नाज़िल किया है शक में पड़े हो पस अगर तुम सच्चे हो तो तुम (भी) एक सूरा बना लाओ और खुदा के सिवा जो भी तुम्हारे मददगार हों उनको भी बुला लो
\end{hindi}}
\flushright{\begin{Arabic}
\quranayah[2][24]
\end{Arabic}}
\flushleft{\begin{hindi}
पस अगर तुम ये नहीं कर सकते हो और हरगिज़ नहीं कर सकोगे तो उस आग से डरो सिके ईधन आदमी और पत्थर होंगे और काफ़िरों के लिए तैयार की गई है
\end{hindi}}
\flushright{\begin{Arabic}
\quranayah[2][25]
\end{Arabic}}
\flushleft{\begin{hindi}
और जो लोग ईमान लाए और उन्होंने नेक काम किए उनको (ऐ पैग़म्बर) खुशख़बरी दे दो कि उनके लिए (बेहिश्त के) वह बाग़ात हैं जिनके नीचे नहरे जारी हैं जब उन्हें इन बाग़ात का कोई मेवा खाने को मिलेगा तो कहेंगे ये तो वही (मेवा है जो पहले भी हमें खाने को मिल चुका है) (क्योंकि) उन्हें मिलती जुलती सूरत व रंग के (मेवे) मिला करेंगे और बेहिश्त में उनके लिए साफ सुथरी बीवियाँ होगी और ये लोग उस बाग़ में हमेशा रहेंगे
\end{hindi}}
\flushright{\begin{Arabic}
\quranayah[2][26]
\end{Arabic}}
\flushleft{\begin{hindi}
बेशक खुदा मच्छर या उससे भी बढ़कर (हक़ीर चीज़) की कोई मिसाल बयान करने में नहीं झेंपता पस जो लोग ईमान ला चुके हैं वह तो ये यक़ीन जानते हैं कि ये (मिसाल) बिल्कुल ठीक है और ये परवरदिगार की तरफ़ से है (अब रहे) वह लोग जो काफ़िर है पस वह बोल उठते हैं कि खुदा का उस मिसाल से क्या मतलब है, ऐसी मिसाल से ख़ुदा बहुतेरों की हिदायत करता है मगर गुमराही में छोड़ता भी है तो ऐसे बदकारों को
\end{hindi}}
\flushright{\begin{Arabic}
\quranayah[2][27]
\end{Arabic}}
\flushleft{\begin{hindi}
जो लोग खुदा के एहदो पैमान को मज़बूत हो जाने के बाद तोड़ डालते हैं और जिन (ताल्लुक़ात) का खुदा ने हुक्म दिया है उनको क़ताआ कर देते हैं और मुल्क में फसाद करते फिरते हैं, यही लोग घाटा उठाने वाले हैं
\end{hindi}}
\flushright{\begin{Arabic}
\quranayah[2][28]
\end{Arabic}}
\flushleft{\begin{hindi}
(हाँए) क्यों कर तुम खुदा का इन्कार कर सकते हो हालाँकि तुम (माओं के पेट में) बेजान थे तो उसी ने तुमको ज़िन्दा किया फिर वही तुमको मार डालेगा, फिर वही तुमको (दोबारा क़यामत में) ज़िन्दा करेगा फिर उसी की तरफ लौटाए जाओगे
\end{hindi}}
\flushright{\begin{Arabic}
\quranayah[2][29]
\end{Arabic}}
\flushleft{\begin{hindi}
वही तो वह (खुदा) है जिसने तुम्हारे (नफ़े) के ज़मीन की कुल चीज़ों को पैदा किया फिर आसमान (के बनाने) की तरफ़ मुतावज्जेह हुआ तो सात आसमान हमवार (व मुसतहकम) बना दिए और वह (खुदा) हर चीज़ से (खूब) वाक़िफ है
\end{hindi}}
\flushright{\begin{Arabic}
\quranayah[2][30]
\end{Arabic}}
\flushleft{\begin{hindi}
और (ऐ रसूल) उस वक्त क़ो याद करो जब तुम्हारे परवरदिगार ने फ़रिश्तों से कहा कि मैं (अपना) एक नाएब ज़मीन में बनानेवाला हूँ (फरिश्ते ताज्जुब से) कहने लगे क्या तू ज़मीन ऐसे शख्स को पैदा करेगा जो ज़मीन में फ़साद और खूँरेज़ियाँ करता फिरे हालाँकि (अगर) ख़लीफा बनाना है (तो हमारा ज्यादा हक़ है) क्योंकि हम तेरी तारीफ व तसबीह करते हैं और तेरी पाकीज़गी साबित करते हैं तब खुदा ने फरमाया इसमें तो शक ही नहीं कि जो मैं जानता हूँ तुम नहीं जानते
\end{hindi}}
\flushright{\begin{Arabic}
\quranayah[2][31]
\end{Arabic}}
\flushleft{\begin{hindi}
और (आदम की हक़ीक़म ज़ाहिर करने की ग़रज़ से) आदम को सब चीज़ों के नाम सिखा दिए फिर उनको फरिश्तों के सामने पेश किया और फ़रमाया कि अगर तुम अपने दावे में कि हम मुस्तहके ख़िलाफ़त हैं। सच्चे हो तो मुझे इन चीज़ों के नाम बताओ
\end{hindi}}
\flushright{\begin{Arabic}
\quranayah[2][32]
\end{Arabic}}
\flushleft{\begin{hindi}
तब फ़रिश्तों ने (आजिज़ी से) अर्ज़ की तू (हर ऐब से) पाक व पाकीज़ा है हम तो जो कुछ तूने बताया है उसके सिवा कुछ नहीं जानते तू बड़ा जानने वाला, मसलहतों का पहचानने वाला है
\end{hindi}}
\flushright{\begin{Arabic}
\quranayah[2][33]
\end{Arabic}}
\flushleft{\begin{hindi}
(उस वक्त ख़ुदा ने आदम को) हुक्म दिया कि ऐ आदम तुम इन फ़रिश्तों को उन सब चीज़ों के नाम बता दो बस जब आदम ने फ़रिश्तों को उन चीज़ों के नाम बता दिए तो खुदा ने फरिश्तों की तरफ ख़िताब करके फरमाया क्यों, मैं तुमसे न कहता था कि मैं आसमानों और ज़मीनों के छिपे हुए राज़ को जानता हूँ, और जो कुछ तुम अब ज़ाहिर करते हो और जो कुछ तुम छिपाते थे (वह सब) जानता हूँ
\end{hindi}}
\flushright{\begin{Arabic}
\quranayah[2][34]
\end{Arabic}}
\flushleft{\begin{hindi}
और (उस वक्त क़ो याद करो) जब हमने फ़रिश्तों से कहा कि आदम को सजदा करो तो सब के सब झुक गए मगर शैतान ने इन्कार किया और ग़ुरूर में आ गया और काफ़िर हो गया
\end{hindi}}
\flushright{\begin{Arabic}
\quranayah[2][35]
\end{Arabic}}
\flushleft{\begin{hindi}
और हमने आदम से कहा ऐ आदम तुम अपनी बीवी समैत बेहिश्त में रहा सहा करो और जहाँ से तुम्हारा जी चाहे उसमें से ब फराग़त खाओ (पियो) मगर उस दरख्त के पास भी न जाना (वरना) फिर तुम अपना आप नुक़सान करोगे
\end{hindi}}
\flushright{\begin{Arabic}
\quranayah[2][36]
\end{Arabic}}
\flushleft{\begin{hindi}
तब शैतान ने आदम व हौव्वा को (धोखा देकर) वहाँ से डगमगाया और आख़िर कार उनको जिस (ऐश व राहत) में थे उनसे निकाल फेंका और हमने कहा (ऐ आदम व हौव्वा) तुम (ज़मीन पर) उतर पड़ो तुममें से एक का एक दुशमन होगा और ज़मीन में तुम्हारे लिए एक ख़ास वक्त (क़यामत) तक ठहराव और ठिकाना है
\end{hindi}}
\flushright{\begin{Arabic}
\quranayah[2][37]
\end{Arabic}}
\flushleft{\begin{hindi}
फिर आदम ने अपने परवरदिगार से (माज़रत के चन्द अल्फाज़) सीखे पस खुदा ने उन अल्फाज़ की बरकत से आदम की तौबा कुबूल कर ली बेशक वह बड़ा माफ़ करने वाला मेहरबान है
\end{hindi}}
\flushright{\begin{Arabic}
\quranayah[2][38]
\end{Arabic}}
\flushleft{\begin{hindi}
(और जब आदम को) ये हुक्म दिया था कि यहाँ से उतर पड़ो (तो भी कह दिया था कि) अगर तुम्हारे पास मेरी तरफ़ से हिदायत आए तो (उसकी पैरवी करना क्योंकि) जो लोग मेरी हिदायत पर चलेंगे उन पर (क़यामत) में न कोई ख़ौफ होगा
\end{hindi}}
\flushright{\begin{Arabic}
\quranayah[2][39]
\end{Arabic}}
\flushleft{\begin{hindi}
और न वह रंजीदा होगे और (ये भी याद रखो) जिन लोगों ने कुफ्र इख़तेयार किया और हमारी आयतों को झुठलाया तो वही जहन्नुमी हैं और हमेशा दोज़ख़ में पड़े रहेगे
\end{hindi}}
\flushright{\begin{Arabic}
\quranayah[2][40]
\end{Arabic}}
\flushleft{\begin{hindi}
ऐ बनी इसराईल (याक़ूब की औलाद) मेरे उन एहसानात को याद करो जो तुम पर पहले कर चुके हैं और तुम मेरे एहद व इक़रार (ईमान) को पूरा करो तो मैं तुम्हारे एहद (सवाब) को पूरा करूँगा, और मुझ ही से डरते रहो
\end{hindi}}
\flushright{\begin{Arabic}
\quranayah[2][41]
\end{Arabic}}
\flushleft{\begin{hindi}
और जो (कुरान) मैंने नाज़िल किया वह उस किताब (तौरेत) की (भी) तसदीक़ करता हूँ जो तुम्हारे पास है और तुम सबसे चले उसके इन्कार पर मौजूद न हो जाओ और मेरी आयतों के बदले थोड़ी क़ीमत (दुनयावी फायदा) न लो और मुझ ही से डरते रहो
\end{hindi}}
\flushright{\begin{Arabic}
\quranayah[2][42]
\end{Arabic}}
\flushleft{\begin{hindi}
और हक़ को बातिल के साथ न मिलाओ और हक़ बात को न छिपाओ हालाँकि तुम जानते हो और पाबन्दी से नमाज़ अदा करो
\end{hindi}}
\flushright{\begin{Arabic}
\quranayah[2][43]
\end{Arabic}}
\flushleft{\begin{hindi}
और ज़कात दिया करो और जो लोग (हमारे सामने) इबादत के लिए झुकते हैं उनके साथ तुम भी झुका करो
\end{hindi}}
\flushright{\begin{Arabic}
\quranayah[2][44]
\end{Arabic}}
\flushleft{\begin{hindi}
और तुम लोगों से नेकी करने को कहते हो और अपनी ख़बर नहीं लेते हालाँकि तुम किताबे खुदा को (बराबर) रटा करते हो तो तुम क्या इतना भी नहीं समझते
\end{hindi}}
\flushright{\begin{Arabic}
\quranayah[2][45]
\end{Arabic}}
\flushleft{\begin{hindi}
और (मुसीबत के वक्त) सब्र और नमाज़ का सहारा पकड़ो और अलबत्ता नमाज़ दूभर तो है मगर उन ख़ाक़सारों पर (नहीं) जो बख़ूबी जानते हैं
\end{hindi}}
\flushright{\begin{Arabic}
\quranayah[2][46]
\end{Arabic}}
\flushleft{\begin{hindi}
कि वह अपने परवरदिगार की बारगाह में हाज़िर होंगे और ज़रूर उसकी तरफ लौट जाएँगे
\end{hindi}}
\flushright{\begin{Arabic}
\quranayah[2][47]
\end{Arabic}}
\flushleft{\begin{hindi}
ऐ बनी इसराइल मेरी उन नेअमतों को याद करो जो मैंने पहले तुम्हें दी और ये (भी तो सोचो) कि हमने तुमको सारे जहान के लोगों से बढ़ा दिया
\end{hindi}}
\flushright{\begin{Arabic}
\quranayah[2][48]
\end{Arabic}}
\flushleft{\begin{hindi}
और उस दिन से डरो (जिस दिन) कोई शख्स किसी की तरफ से न फिदिया हो सकेगा और न उसकी तरफ से कोई सिफारिश मानी जाएगी और न उसका कोई मुआवज़ा लिया जाएगा और न वह मदद पहुँचाए जाएँगे
\end{hindi}}
\flushright{\begin{Arabic}
\quranayah[2][49]
\end{Arabic}}
\flushleft{\begin{hindi}
और (उस वक्त क़ो याद करो) जब हमने तुम्हें (तुम्हारे बुर्ज़गो को) फिरऔन (के पन्जे) से छुड़ाया जो तुम्हें बड़े-बड़े दुख दे के सताते थे तुम्हारे लड़कों पर छुरी फेरते थे और तुम्हारी औरतों को (अपनी ख़िदमत के लिए) ज़िन्दा रहने देते थे और उसमें तुम्हारे परवरदिगार की तरफ से (तुम्हारे सब्र की) सख्त आज़माइश थी
\end{hindi}}
\flushright{\begin{Arabic}
\quranayah[2][50]
\end{Arabic}}
\flushleft{\begin{hindi}
और (वह वक्त भी याद करो) जब हमने तुम्हारे लिए दरिया को टुकड़े-टुकड़े किया फिर हमने तुमको छुटकारा दिया
\end{hindi}}
\flushright{\begin{Arabic}
\quranayah[2][51]
\end{Arabic}}
\flushleft{\begin{hindi}
और फिरऔन के आदमियों को तुम्हारे देखते-देखते डुबो दिया और (वह वक्त भी याद करो) जब हमने मूसा से चालीस रातों का वायदा किया था और तुम लोगों ने उनके जाने के बाद एक बछड़े को (परसतिश के लिए खुदा) बना लिया
\end{hindi}}
\flushright{\begin{Arabic}
\quranayah[2][52]
\end{Arabic}}
\flushleft{\begin{hindi}
हालाँकि तुम अपने ऊपर ज़ुल्म जोत रहे थे फिर हमने उसके बाद भी दरगुज़र की ताकि तुम शुक्र करो
\end{hindi}}
\flushright{\begin{Arabic}
\quranayah[2][53]
\end{Arabic}}
\flushleft{\begin{hindi}
और (वह वक्त भी याद करो) जब मूसा को (तौरेत) अता की और हक़ और बातिल को जुदा करनेवाला क़ानून (इनायत किया) ताके तुम हिदायत पाओ
\end{hindi}}
\flushright{\begin{Arabic}
\quranayah[2][54]
\end{Arabic}}
\flushleft{\begin{hindi}
और (वह वक्त भी याद करो) जब मूसा ने अपनी क़ौम से कहा कि ऐ मेरी क़ौम तुमने बछड़े को (ख़ुदा) बना के अपने ऊपर बड़ा सख्त जुल्म किया तो अब (इसके सिवा कोई चारा नहीं कि) तुम अपने ख़ालिक की बारगाह में तौबा करो और वह ये है कि अपने को क़त्ल कर डालो तुम्हारे परवरदिगार के नज़दीक तुम्हारे हक़ में यही बेहतर है, फिर जब तुमने ऐसा किया तो खुदा ने तुम्हारी तौबा क़ुबूल कर ली बेशक वह बड़ा मेहरबान माफ़ करने वाला है
\end{hindi}}
\flushright{\begin{Arabic}
\quranayah[2][55]
\end{Arabic}}
\flushleft{\begin{hindi}
और (वह वक्त भी याद करो) जब तुमने मूसा से कहा था कि ऐ मूसा हम तुम पर उस वक्त तक ईमान न लाएँगे जब तक हम खुदा को ज़ाहिर बज़ाहिर न देख ले उस पर तुम्हें बिजली ने ले डाला, और तुम तकते ही रह गए
\end{hindi}}
\flushright{\begin{Arabic}
\quranayah[2][56]
\end{Arabic}}
\flushleft{\begin{hindi}
फिर तुम्हें तुम्हारे मरने के बाद हमने जिला उठाया ताकि तुम शुक्र करो
\end{hindi}}
\flushright{\begin{Arabic}
\quranayah[2][57]
\end{Arabic}}
\flushleft{\begin{hindi}
और हमने तुम पर अब्र का साया किया और तुम पर मन व सलवा उतारा और (ये भी तो कह दिया था कि) जो सुथरी व नफीस रोज़िया तुम्हें दी हैं उन्हें शौक़ से खाओ, और उन लोगों ने हमारा तो कुछ बिगड़ा नहीं मगर अपनी जानों पर सितम ढाते रहे
\end{hindi}}
\flushright{\begin{Arabic}
\quranayah[2][58]
\end{Arabic}}
\flushleft{\begin{hindi}
और (वह वक्त भी याद करो) जब हमने तुमसे कहा कि इस गाँव (अरीहा) में जाओ और इसमें जहाँ चाहो फराग़त से खाओ (पियो) और दरवाज़े पर सजदा करते हुए और ज़बान से हित्ता बख्शिश कहते हुए आओ तो हम तुम्हारी ख़ता ये बख्श देगे और हम नेकी करने वालों की नेकी (सवाब) बढ़ा देगें
\end{hindi}}
\flushright{\begin{Arabic}
\quranayah[2][59]
\end{Arabic}}
\flushleft{\begin{hindi}
तो जो बात उनसे कही गई थी उसे शरीरों ने बदलकर दूसरी बात कहनी शुरू कर दी तब हमने उन लोगों पर जिन्होंने शरारत की थी उनकी बदकारी की वजह से आसमानी बला नाज़िल की
\end{hindi}}
\flushright{\begin{Arabic}
\quranayah[2][60]
\end{Arabic}}
\flushleft{\begin{hindi}
और (वह वक्त भी याद करो) जब मूसा ने अपनी क़ौम के लिए पानी माँगा तो हमने कहा (ऐ मूसा) अपनी लाठी पत्थर पर मारो (लाठी मारते ही) उसमें से बारह चश्में फूट निकले और सब लोगों ने अपना-अपना घाट बखूबी जान लिया और हमने आम इजाज़त दे दी कि खुदा की दी हुईरोज़ी से खाओ पियो और मुल्क में फसाद न करते फिरो
\end{hindi}}
\flushright{\begin{Arabic}
\quranayah[2][61]
\end{Arabic}}
\flushleft{\begin{hindi}
(और वह वक्त भी याद करो) जब तुमने मूसा से कहा कि ऐ मूसा हमसे एक ही खाने पर न रहा जाएगा तो आप हमारे लिए अपने परवरदिगार से दुआ कीजिए कि जो चीज़े ज़मीन से उगती है जैसे साग पात तरकारी और ककड़ी और गेहूँ या (लहसुन) और मसूर और प्याज़ (मन व सलवा) की जगह पैदा करें (मूसा ने) कहा क्या तुम ऐसी चीज़ को जो हर तरह से बेहतर है अदना चीज़ से बदलन चाहते हो तो किसी शहर में उतर पड़ो फिर तुम्हारे लिए जो तुमने माँगा है सब मौजूद है और उन पर रूसवाई और मोहताजी की मार पड़ी और उन लोगों ने क़हरे खुदा की तरफ पलटा खाया, ये सब इस सबब से हुआ कि वह लोग खुदा की निशानियों से इन्कार करते थे और पैग़म्बरों को नाहक शहीद करते थे, और इस वजह से (भी) कि वह नाफ़रमानी और सरकशी किया करते थे
\end{hindi}}
\flushright{\begin{Arabic}
\quranayah[2][62]
\end{Arabic}}
\flushleft{\begin{hindi}
बेशक मुसलमानों और यहूदियों और नसरानियों और ला मज़हबों में से जो कोई खुदा और रोज़े आख़िरत पर ईमान लाए और अच्छे-अच्छे काम करता रहे तो उन्हीं के लिए उनका अज्र व सवाब उनके खुदा के पास है और न (क़यामत में) उन पर किसी का ख़ौफ होगा न वह रंजीदा दिल होंगे
\end{hindi}}
\flushright{\begin{Arabic}
\quranayah[2][63]
\end{Arabic}}
\flushleft{\begin{hindi}
और (वह वक्त याद करो) जब हमने (तामीले तौरेत) का तुमसे एक़रार कर लिया और हमने तुम्हारे सर पर तूर से (पहाड़ को) लाकर लटकाया और कह दिया कि तौरेत जो हमने तुमको दी है उसको मज़बूत पकड़े रहो और जो कुछ उसमें है उसको याद रखो
\end{hindi}}
\flushright{\begin{Arabic}
\quranayah[2][64]
\end{Arabic}}
\flushleft{\begin{hindi}
ताकि तुम परहेज़गार बनो फिर उसके बाद तुम (अपने एहदो पैमान से) फिर गए पस अगर तुम पर खुदा का फज़ल और उसकी मेहरबानी न होती तो तुमने सख्त घाटा उठाया होता
\end{hindi}}
\flushright{\begin{Arabic}
\quranayah[2][65]
\end{Arabic}}
\flushleft{\begin{hindi}
और अपनी क़ौम से उन लोगों की हालत तो तुम बखूबी जानते हो जो शम्बे (सनीचर) के दिन अपनी हद से गुज़र गए (कि बावजूद मुमानिअत शिकार खेलने निकले) तो हमने उन से कहा कि तुम राइन्दे गए बन्दर बन जाओ (और वह बन्दर हो गए)
\end{hindi}}
\flushright{\begin{Arabic}
\quranayah[2][66]
\end{Arabic}}
\flushleft{\begin{hindi}
पस हमने इस वाक़ये से उन लोगों के वास्ते जिन के सामने हुआ था और जो उसके बाद आनेवाले थे अज़ाब क़रार दिया, और परहेज़गारों के लिए नसीहत
\end{hindi}}
\flushright{\begin{Arabic}
\quranayah[2][67]
\end{Arabic}}
\flushleft{\begin{hindi}
और (वह वक्त याद करो) जब मूसा ने अपनी क़ौम से कहा कि खुदा तुम लोगों को ताकीदी हुक्म करता है कि तुम एक गाय ज़िबाह करो वह लोग कहने लगे क्या तुम हमसे दिल्लगी करते हो मूसा ने कहा मैं खुदा से पनाह माँगता हूँ कि मैं जाहिल बनूँ
\end{hindi}}
\flushright{\begin{Arabic}
\quranayah[2][68]
\end{Arabic}}
\flushleft{\begin{hindi}
(तब वह लोग कहने लगे कि (अच्छा) तुम अपने खुदा से दुआ करो कि हमें बता दे कि वह गाय कैसी हो मूसा ने कहा बेशक खुदा ने फरमाता है कि वह गाय न तो बहुत बूढ़ी हो और न बछिया बल्कि उनमें से औसत दरजे की हो, ग़रज़ जो तुमको हुक्म दिया गया उसको बजा लाओ
\end{hindi}}
\flushright{\begin{Arabic}
\quranayah[2][69]
\end{Arabic}}
\flushleft{\begin{hindi}
वह कहने लगे (वाह) तुम अपने खुदा से दुआ करो कि हमें ये बता दे कि उसका रंग आख़िर क्या हो मूसा ने कहा बेशक खुदा फरमाता है कि वह गाय खूब गहरे ज़र्द रंग की हो देखने वाले उसे देखकर खुश हो जाए
\end{hindi}}
\flushright{\begin{Arabic}
\quranayah[2][70]
\end{Arabic}}
\flushleft{\begin{hindi}
तब कहने लगे कि तुम अपने खुदा से दुआ करो कि हमें ज़रा ये तो बता दे कि वह (गाय) और कैसी हो (वह) गाय तो और गायों में मिल जुल गई और खुदा ने चाहा तो हम ज़रूर (उसका) पता लगा लेगे
\end{hindi}}
\flushright{\begin{Arabic}
\quranayah[2][71]
\end{Arabic}}
\flushleft{\begin{hindi}
मूसा ने कहा खुदा ज़रूर फरमाता है कि वह गाय न तो इतनी सधाई हो कि ज़मीन जोते न खेती सीचें भली चंगी एक रंग की कि उसमें कोई धब्बा तक न हो, वह बोले अब (जा के) ठीक-ठीक बयान किया, ग़रज़ उन लोगों ने वह गाय हलाल की हालाँकि उनसे उम्मीद न थी वह कि वह ऐसा करेंगे
\end{hindi}}
\flushright{\begin{Arabic}
\quranayah[2][72]
\end{Arabic}}
\flushleft{\begin{hindi}
और जब एक शख्स को मार डाला और तुममें उसकी बाबत फूट पड़ गई एक दूसरे को क़ातिल बताने लगा जो तुम छिपाते थे
\end{hindi}}
\flushright{\begin{Arabic}
\quranayah[2][73]
\end{Arabic}}
\flushleft{\begin{hindi}
खुदा को उसका ज़ाहिर करना मंजूर था पस हमने कहा कि उस गाय को कोई टुकड़ा लेकर इस (की लाश) पर मारो यूँ खुदा मुर्दे को ज़िन्दा करता है और तुम को अपनी कुदरत की निशानियाँ दिखा देता है
\end{hindi}}
\flushright{\begin{Arabic}
\quranayah[2][74]
\end{Arabic}}
\flushleft{\begin{hindi}
ताकि तुम समझो फिर उसके बाद तुम्हारे दिल सख्त हो गये पस वह मिसल पत्थर के (सख्त) थे या उससे भी ज्यादा करख्त क्योंकि पत्थरों में बाज़ तो ऐसे होते हैं कि उनसे नहरें जारी हो जाती हैं और बाज़ ऐसे होते हैं कि उनमें दरार पड़ जाती है और उनमें से पानी निकल पड़ता है और बाज़ पत्थर तो ऐसे होते हैं कि खुदा के ख़ौफ से गिर पड़ते हैं और जो कुछ तुम कर रहे हो उससे खुदा ग़ाफिल नहीं है
\end{hindi}}
\flushright{\begin{Arabic}
\quranayah[2][75]
\end{Arabic}}
\flushleft{\begin{hindi}
(मुसलमानों) क्या तुम ये लालच रखते हो कि वह तुम्हारा (सा) ईमान लाएँगें हालाँकि उनमें का एक गिरोह (साबिक़ में) ऐसा था कि खुदा का कलाम सुनाता था और अच्छी तरह समझने के बाद उलट फेर कर देता था हालाँकि वह खूब जानते थे और जब उन लोगों से मुलाक़ात करते हैं
\end{hindi}}
\flushright{\begin{Arabic}
\quranayah[2][76]
\end{Arabic}}
\flushleft{\begin{hindi}
जो ईमान लाए तो कह देते हैं कि हम तो ईमान ला चुके और जब उनसे बाज़-बाज़ के साथ तख़िलया करते हैं तो कहते हैं कि जो कुछ खुदा ने तुम पर (तौरेत) में ज़ाहिर कर दिया है क्या तुम (मुसलमानों को) बता दोगे ताकि उसके सबब से कल तुम्हारे खुदा के पास तुम पर हुज्जत लाएँ क्या तुम इतना भी नहीं समझते
\end{hindi}}
\flushright{\begin{Arabic}
\quranayah[2][77]
\end{Arabic}}
\flushleft{\begin{hindi}
लेकिन क्या वह लोग (इतना भी) नहीं जानते कि वह लोग जो कुछ छिपाते हैं या ज़ाहिर करते हैं खुदा सब कुछ जानता है
\end{hindi}}
\flushright{\begin{Arabic}
\quranayah[2][78]
\end{Arabic}}
\flushleft{\begin{hindi}
और कुछ उनमें से ऐसे अनपढ़ हैं कि वह किताबे खुदा को अपने मतलब की बातों के सिवा कुछ नहीं समझते और वह फक़त ख्याली बातें किया करते हैं,
\end{hindi}}
\flushright{\begin{Arabic}
\quranayah[2][79]
\end{Arabic}}
\flushleft{\begin{hindi}
पस वाए हो उन लोगों पर जो अपने हाथ से किताब लिखते हैं फिर (लोगों से कहते फिरते) हैं कि ये खुदा के यहाँ से (आई) है ताकि उसके ज़रिये से थोड़ी सी क़ीमत (दुनयावी फ़ायदा) हासिल करें पस अफसोस है उन पर कि उनके हाथों ने लिखा और फिर अफसोस है उनपर कि वह ऐसी कमाई करते हैं
\end{hindi}}
\flushright{\begin{Arabic}
\quranayah[2][80]
\end{Arabic}}
\flushleft{\begin{hindi}
और कहते हैं कि गिनती के चन्द दिनों के सिवा हमें आग छुएगी भी तो नहीं (ऐ रसूल) इन लोगों से कहो कि क्या तुमने खुदा से कोई इक़रार ले लिया है कि फिर वह किसी तरह अपने इक़रार के ख़िलाफ़ हरगिज़ न करेगा या बे समझे बूझे खुदा पर बोहताव जोड़ते हो
\end{hindi}}
\flushright{\begin{Arabic}
\quranayah[2][81]
\end{Arabic}}
\flushleft{\begin{hindi}
हाँ (सच तो यह है) कि जिसने बुराई हासिल की और उसके गुनाहों ने चारों तरफ से उसे घेर लिया है वही लोग तो दोज़ख़ी हैं और वही (तो) उसमें हमेशा रहेंगे
\end{hindi}}
\flushright{\begin{Arabic}
\quranayah[2][82]
\end{Arabic}}
\flushleft{\begin{hindi}
और जो लोग ईमानदार हैं और उन्होंने अच्छे काम किए हैं वही लोग जन्नती हैं कि हमेशा जन्नत में रहेंगे
\end{hindi}}
\flushright{\begin{Arabic}
\quranayah[2][83]
\end{Arabic}}
\flushleft{\begin{hindi}
और (वह वक्त याद करो) जब हमने बनी ईसराइल से (जो तुम्हारे बुर्जुग़ थे) अहद व पैमान लिया था कि खुदा के सिवा किसी की इबादत न करना और माँ बाप और क़राबतदारों और यतीमों और मोहताजों के साथ अच्छे सुलूक करना और लोगों के साथ अच्छी तरह (नरमी) से बातें करना और बराबर नमाज़ पढ़ना और ज़कात देना फिर तुममें से थोड़े आदिमियों के सिवा (सब के सब) फिर गए और तुम लोग हो ही इक़रार से मुँह फेरने वाले
\end{hindi}}
\flushright{\begin{Arabic}
\quranayah[2][84]
\end{Arabic}}
\flushleft{\begin{hindi}
और (वह वक्त याद करो) जब हमने तुम (तुम्हारे बुर्ज़ुगों) से अहद लिया था कि आपस में खूरेज़ियाँ न करना और न अपने लोगों को शहर बदर करना तो तुम (तुम्हारे बुर्जुग़ों) ने इक़रार किया था और तुम भी उसकी गवाही देते हो
\end{hindi}}
\flushright{\begin{Arabic}
\quranayah[2][85]
\end{Arabic}}
\flushleft{\begin{hindi}
(कि हाँ ऐसा हुआ था) फिर वही लोग तो तुम हो कि आपस में एक दूसरे को क़त्ल करते हो और अपनों से एक जत्थे के नाहक़ और ज़बरदस्ती हिमायती बनकर दूसरे को शहर बदर करते हो (और लुत्फ़ तो ये है कि) अगर वही लोग क़ैदी बनकर तम्हारे पास (मदद माँगने) आए तो उनको तावान देकर छुड़ा लेते हो हालाँकि उनका निकालना ही तुम पर हराम किया गया था तो फिर क्या तुम (किताबे खुदा की) बाज़ बातों पर ईमान रखते हो और बाज़ से इन्कार करते हो पस तुम में से जो लोग ऐसा करें उनकी सज़ा इसके सिवा और कुछ नहीं कि ज़िन्दगी भर की रूसवाई हो और (आख़िरकार) क़यामत के दिन सख्त अज़ाब की तरफ लौटा दिये जाए और जो कुछ तुम लोग करते हो खुदा उससे ग़ाफ़िल नहीं है
\end{hindi}}
\flushright{\begin{Arabic}
\quranayah[2][86]
\end{Arabic}}
\flushleft{\begin{hindi}
यही वह लोग हैं जिन्होंने आख़ेरत के बदले दुनिया की ज़िन्दगी ख़रीद पस न उनके अज़ाब ही में तख्फ़ीफ़ (कमी) की जाएगी और न वह लोग किसी तरह की मदद दिए जाएँगे
\end{hindi}}
\flushright{\begin{Arabic}
\quranayah[2][87]
\end{Arabic}}
\flushleft{\begin{hindi}
और ये हक़ीक़ी बात है कि हमने मूसा को किताब (तौरेत) दी और उनके बाद बहुत से पैग़म्बरों को उनके क़दम ब क़दम ले चलें और मरियम के बेटे ईसा को (भी बहुत से) वाजेए व रौशन मौजिजे दिए और पाक रूह जिबरील के ज़रिये से उनकी मदद की क्या तुम उस क़दर बददिमाग़ हो गए हो कि जब कोई पैग़म्बर तुम्हारे पास तुम्हारी ख्वाहिशे नफ़सानी के ख़िलाफ कोई हुक्म लेकर आया तो तुम अकड़ बैठे फिर तुमने बाज़ पैग़म्बरों को तो झुठलाया और बाज़ को जान से मार डाला
\end{hindi}}
\flushright{\begin{Arabic}
\quranayah[2][88]
\end{Arabic}}
\flushleft{\begin{hindi}
और कहने लगे कि हमारे दिलों पर ग़िलाफ चढ़ा हुआ है (ऐसा नहीं) बल्कि उनके कुफ्र की वजह से खुदा ने उनपर लानत की है पस कम ही लोग ईमान लाते हैं
\end{hindi}}
\flushright{\begin{Arabic}
\quranayah[2][89]
\end{Arabic}}
\flushleft{\begin{hindi}
और जब उनके पास खुदा की तरफ़ से किताब (कुरान आई और वह उस किताब तौरेत) की जो उन के पास तसदीक़ भी करती है। और उससे पहले (इसकी उम्मीद पर) काफ़िरों पर फतेहयाब होने की दुआएँ माँगते थे पस जब उनके पास वह चीज़ जिसे पहचानते थे आ गई तो लगे इन्कार करने पस काफ़िरों पर खुदा की लानत है
\end{hindi}}
\flushright{\begin{Arabic}
\quranayah[2][90]
\end{Arabic}}
\flushleft{\begin{hindi}
क्या ही बुरा है वह काम जिसके मुक़ाबले में (इतनी बात पर) वह लोग अपनी जानें बेच बैठे हैं कि खुदा अपने बन्दों से जिस पर चाहे अपनी इनायत से किताब नाज़िल किया करे इस रश्क से जो कुछ खुदा ने नाज़िल किया है सबका इन्कार कर बैठे पस उन पर ग़ज़ब पर ग़ज़ब टूट पड़ा और काफ़िरों के लिए (बड़ी) रूसवाई का अज़ाब है
\end{hindi}}
\flushright{\begin{Arabic}
\quranayah[2][91]
\end{Arabic}}
\flushleft{\begin{hindi}
और जब उनसे कहा गया कि (जो क़ुरान) खुदा ने नाज़िल किया है उस पर ईमान लाओ तो कहने लगे कि हम तो उसी किताब (तौरेत) पर ईमान लाए हैं जो हम पर नाज़िल की गई थी और उस किताब (कुरान) को जो उसके बाद आई है नहीं मानते हैं हालाँकि वह (क़ुरान) हक़ है और उस किताब (तौरेत) की जो उनके पास है तसदीक़ भी करती है मगर उस किताब कुरान का जो उसके बाद आई है इन्कार करते हैं (ऐ रसूल) उनसे ये तो पूछो कि तुम (तुम्हारे बुर्जुग़) अगर ईमानदार थे तो फिर क्यों खुदा के पैग़म्बरों का साबिक़ क़त्ल करते थे
\end{hindi}}
\flushright{\begin{Arabic}
\quranayah[2][92]
\end{Arabic}}
\flushleft{\begin{hindi}
और तुम्हारे पास मूसा तो वाज़ेए व रौशन मौजिज़े लेकर आ ही चुके थे फिर भी तुमने उनके बाद बछड़े को खुदा बना ही लिया और उससे तुम अपने ही ऊपर ज़ुल्म करने वाले थे
\end{hindi}}
\flushright{\begin{Arabic}
\quranayah[2][93]
\end{Arabic}}
\flushleft{\begin{hindi}
और (वह वक्त याद करो) जब हमने तुमसे अहद लिया और (क़ोहे) तूर को (तुम्हारी उदूले हुक्मी से) तुम्हारे सर पर लटकाया और (हमने कहा कि ये किताब तौरेत) जो हमने दी है मज़बूती से लिए रहो और (जो कुछ उसमें है) सुनो तो कहने लगे सुना तो (सही लेकिन) हम इसको मानते नहीं और उनकी बेईमानी की वजह से (गोया) बछड़े की उलफ़त घोल के उनके दिलों में पिला दी गई (ऐ रसूल) उन लोगों से कह दो कि अगर तुम ईमानदार थे तो तुमको तुम्हारा ईमान क्या ही बुरा हुक्म करता था
\end{hindi}}
\flushright{\begin{Arabic}
\quranayah[2][94]
\end{Arabic}}
\flushleft{\begin{hindi}
(ऐ रसूल) इन लोगों से कह दो कि अगर खुदा के नज़दीक आख़ेरत का घर (बेहिश्त) ख़ास तुम्हारे वास्ते है और लोगों के वासते नहीं है पस अगर तुम सच्चे हो तो मौत की आरजू क़रो
\end{hindi}}
\flushright{\begin{Arabic}
\quranayah[2][95]
\end{Arabic}}
\flushleft{\begin{hindi}
(ताकि जल्दी बेहिश्त में जाओ) लेकिन वह उन आमाले बद की वजह से जिनको उनके हाथों ने पहले से आगे भेजा है हरगिज़ मौत की आरज़ू न करेंगे और खुदा ज़ालिमों से खूब वाक़िफ है
\end{hindi}}
\flushright{\begin{Arabic}
\quranayah[2][96]
\end{Arabic}}
\flushleft{\begin{hindi}
और (ऐ रसूल) तुम उन ही को ज़िन्दगी का सबसे ज्यादा हरीस पाओगे और मुशरिक़ों में से हर एक शख्स चाहता है कि काश उसको हज़ार बरस की उम्र दी जाती हालाँकि अगर इतनी तूलानी उम्र भी दी जाए तो वह ख़ुदा के अज़ाब से छुटकारा देने वाली नहीं, और जो कुछ वह लोग करते हैं खुदा उसे देख रहा है
\end{hindi}}
\flushright{\begin{Arabic}
\quranayah[2][97]
\end{Arabic}}
\flushleft{\begin{hindi}
(ऐ रसूल उन लोगों से) कह दो कि जो जिबरील का दुशमन है (उसका खुदा दुशमन है) क्योंकि उस फ़रिश्ते ने खुदा के हुक्म से (इस कुरान को) तुम्हारे दिल पर डाला है और वह उन किताबों की भी तसदीक करता है जो (पहले नाज़िल हो चुकी हैं और सब) उसके सामने मौजूद हैं और ईमानदारों के वास्ते खुशख़बरी है
\end{hindi}}
\flushright{\begin{Arabic}
\quranayah[2][98]
\end{Arabic}}
\flushleft{\begin{hindi}
जो शख्स ख़ुदा और उसके फरिश्तों और उसके रसूलों और (ख़ासकर) जिबराईल व मीकाइल का दुशमन हो तो बेशक खुदा भी (ऐसे) काफ़िरों का दुश्मन है
\end{hindi}}
\flushright{\begin{Arabic}
\quranayah[2][99]
\end{Arabic}}
\flushleft{\begin{hindi}
और (ऐ रसूल) हमने तुम पर ऐसी निशानियाँ नाज़िल की हैं जो वाजेए और रौशन हैं और ऐसे नाफरमानों के सिवा उनका कोई इन्कार नहीं कर सकता
\end{hindi}}
\flushright{\begin{Arabic}
\quranayah[2][100]
\end{Arabic}}
\flushleft{\begin{hindi}
और उनकी ये हालत है कि जब कभी कोई अहद किया तो उनमें से एक फरीक़ ने तोड़ डाला बल्कि उनमें से अक्सर तो ईमान ही नहीं रखते
\end{hindi}}
\flushright{\begin{Arabic}
\quranayah[2][101]
\end{Arabic}}
\flushleft{\begin{hindi}
और जब उनके पास खुदा की तरफ से रसूल (मोहम्मद) आया और उस किताब (तौरेत) की जो उनके पास है तसदीक़ भी करता है तो उन अहले किताब के एक गिरोह ने किताबे खुदा को अपने पसे पुश्त फेंक दिया गोया वह लोग कुछ जानते ही नहीं और उस मंत्र के पीछे पड़ गए
\end{hindi}}
\flushright{\begin{Arabic}
\quranayah[2][102]
\end{Arabic}}
\flushleft{\begin{hindi}
जिसको सुलेमान के ज़माने की सलतनत में शयातीन जपा करते थे हालाँकि सुलेमान ने कुफ्र नहीं इख़तेयार किया लेकिन शैतानों ने कुफ्र एख़तेयार किया कि वह लोगों को जादू सिखाया करते थे और वह चीज़ें जो हारूत और मारूत दोनों फ़रिश्तों पर बाइबिल में नाज़िल की गई थी हालाँकि ये दोनों फ़रिश्ते किसी को सिखाते न थे जब तक ये न कह देते थे कि हम दोनों तो फ़क़त (ज़रियाए आज़माइश) है पस तो (इस पर अमल करके) बेईमान न हो जाना उस पर भी उनसे वह (टोटके) सीखते थे जिनकी वजह से मिया बीवी में तफ़रक़ा डालते हालाँकि बग़ैर इज्ने खुदावन्दी वह अपनी इन बातों से किसी को ज़रर नहीं पहुँचा सकते थे और ये लोग ऐसी बातें सीखते थे जो खुद उन्हें नुक़सान पहुँचाती थी और कुछ (नफा) पहुँचाती थी बावजूद कि वह यक़ीनन जान चुके थे कि जो शख्स इन (बुराईयों) का ख़रीदार हुआ वह आख़िरत में बेनसीब हैं और बेशुबह (मुआवज़ा) बहुत ही बड़ा है जिसके बदले उन्होंने अपनी जानों को बेचा काश (उसे कुछ) सोचे समझे होते
\end{hindi}}
\flushright{\begin{Arabic}
\quranayah[2][103]
\end{Arabic}}
\flushleft{\begin{hindi}
और अगर वह ईमान लाते और जादू वग़ैरह से बचकर परहेज़गार बनते तो खुदा की दरगाह से जो सवाब मिलता वह उससे कहीं बेहतर होता काश ये लोग (इतना तो) समझते
\end{hindi}}
\flushright{\begin{Arabic}
\quranayah[2][104]
\end{Arabic}}
\flushleft{\begin{hindi}
ऐ ईमानवालों तुम (रसूल को अपनी तरफ मुतावज्जे करना चाहो तो) रआना (हमारी रिआयत कर) न कहा करो बल्कि उनज़ुरना (हम पर नज़रे तवज्जो रख) कहा करो और (जी लगाकर) सुनते रहो और काफिरों के लिए दर्दनाक अज़ाब है
\end{hindi}}
\flushright{\begin{Arabic}
\quranayah[2][105]
\end{Arabic}}
\flushleft{\begin{hindi}
ऐ रसूल अहले किताब में से जिन लोगों ने कुफ्र इख़तेयार किया वह और मुशरेकीन ये नहीं चाहते हैं कि तुम पर तुम्हारे परवरदिगार की तरफ से भलाई (वही) नाज़िल की जाए और (उनका तो इसमें कुछ इजारा नहीं) खुदा जिसको चाहता है अपनी रहमत के लिए ख़ास कर लेता है और खुदा बड़ा फज़ल (करने) वाला है
\end{hindi}}
\flushright{\begin{Arabic}
\quranayah[2][106]
\end{Arabic}}
\flushleft{\begin{hindi}
(ऐ रसूल) हम जब कोई आयत मन्सूख़ करते हैं या तुम्हारे ज़ेहन से मिटा देते हैं तो उससे बेहतर या वैसी ही (और) नाज़िल भी कर देते हैं क्या तुम नहीं जानते कि बेशुबहा खुदा हर चीज़ पर क़ादिर है
\end{hindi}}
\flushright{\begin{Arabic}
\quranayah[2][107]
\end{Arabic}}
\flushleft{\begin{hindi}
क्या तुम नहीं जानते कि आसमान की सलतनत बेशुबहा ख़ास खुदा ही के लिए है और खुदा के सिवा तुम्हारा न कोई सरपरस्त है न मददगार
\end{hindi}}
\flushright{\begin{Arabic}
\quranayah[2][108]
\end{Arabic}}
\flushleft{\begin{hindi}
(मुसलमानों) क्या तुम चाहते हो कि तुम भी अपने रसूल से वैसै ही (बेढ़ंगे) सवालात करो जिस तरह साबिक़ (पहले) ज़माने में मूसा से (बेतुके) सवालात किए गए थे और जिस शख्स ने ईमान के बदले कुफ्र एख़तेयार किया वह तो यक़ीनी सीधे रास्ते से भटक गया
\end{hindi}}
\flushright{\begin{Arabic}
\quranayah[2][109]
\end{Arabic}}
\flushleft{\begin{hindi}
(मुसलमानों) अहले किताब में से अक्सर लोग अपने दिली हसद की वजह से ये ख्वाहिश रखते हैं कि तुमको ईमान लाने के बाद फिर काफ़िर बना दें (और लुत्फ तो ये है कि) उन पर हक़ ज़ाहिर हो चुका है उसके बाद भी (ये तमन्ना बाक़ी है) पस तुम माफ करो और दरगुज़र करो यहाँ तक कि खुदा अपना (कोई और) हुक्म भेजे बेशक खुदा हर चीज़ पर क़ादिर है
\end{hindi}}
\flushright{\begin{Arabic}
\quranayah[2][110]
\end{Arabic}}
\flushleft{\begin{hindi}
और नमाज़ पढ़ते रहो और ज़कात दिये जाओ और जो कुछ भलाई अपने लिए (खुदा के यहाँ) पहले से भेज दोगे उस (के सवाब) को मौजूद पाआगे जो कुछ तुम करते हो उसे खुदा ज़रूर देख रहा है
\end{hindi}}
\flushright{\begin{Arabic}
\quranayah[2][111]
\end{Arabic}}
\flushleft{\begin{hindi}
और (यहूद) कहते हैं कि यहूद (के सिवा) और (नसारा कहते हैं कि) नसारा के सिवा कोई बेहिश्त में जाने ही न पाएगा ये उनके ख्याली पुलाव है (ऐ रसूल) तुम उन से कहो कि भला अगर तुम सच्चे हो कि हम ही बेहिश्त में जाएँगे तो अपनी दलील पेश करो
\end{hindi}}
\flushright{\begin{Arabic}
\quranayah[2][112]
\end{Arabic}}
\flushleft{\begin{hindi}
हाँ अलबत्ता जिस शख्स ने खुदा के आगे अपना सर झुका दिया और अच्छे काम भी करता है तो उसके लिए उसके परवरदिगार के यहाँ उसका बदला (मौजूद) है और (आख़ेरत में) ऐसे लोगों पर न किसी तरह का ख़ौफ़ होगा और न ऐसे लोग ग़मग़ीन होगे
\end{hindi}}
\flushright{\begin{Arabic}
\quranayah[2][113]
\end{Arabic}}
\flushleft{\begin{hindi}
और यहूद कहते हैं कि नसारा का मज़हब कुछ (ठीक) नहीं और नसारा कहते हैं कि यहूद का मज़हब कुछ (ठीक) नहीं हालाँकि ये दोनों फरीक़ किताबे (खुदा) पढ़ते रहते हैं इसी तरह उन्हीं जैसी बातें वह (मुशरेकीन अरब) भी किया करते हैं जो (खुदा के एहकाम) कुछ नहीं जानते तो जिस बात में ये लोग पड़े झगड़ते हैं (दुनिया में तो तय न होगा) क़यामत के दिन खुदा उनके दरमियान ठीक फैसला कर देगा
\end{hindi}}
\flushright{\begin{Arabic}
\quranayah[2][114]
\end{Arabic}}
\flushleft{\begin{hindi}
और उससे बढ़कर ज़ालिम कौन होगा जो खुदा की मसजिदों में उसका नाम लिए जाने से (लोगों को) रोके और उनकी बरबादी के दर पे हो, ऐसों ही को उसमें जाना मुनासिब नहीं मगर सहमे हुए ऐसे ही लोगों के लिए दुनिया में रूसवाई है और ऐसे ही लोगों के लिए आख़ेरत में बड़ा भारी अज़ाब है
\end{hindi}}
\flushright{\begin{Arabic}
\quranayah[2][115]
\end{Arabic}}
\flushleft{\begin{hindi}
(तुम्हारे मसजिद में रोकने से क्या होता है क्योंकि सारी ज़मीन) खुदा ही की है (क्या) पूरब (क्या) पश्चिम बस जहाँ कहीं क़िब्ले की तरफ रूख़ करो वही खुदा का सामना है बेशक खुदा बड़ी गुन्जाइश वाला और खूब वाक़िफ है
\end{hindi}}
\flushright{\begin{Arabic}
\quranayah[2][116]
\end{Arabic}}
\flushleft{\begin{hindi}
और यहूद कहने लगे कि खुदा औलाद रखता है हालाँकि वह (इस बखेड़े से) पाक है बल्कि जो कुछ ज़मीन व आसमान में है सब उसी का है और सब उसकी के फरमाबरदार हैं
\end{hindi}}
\flushright{\begin{Arabic}
\quranayah[2][117]
\end{Arabic}}
\flushleft{\begin{hindi}
(वही) आसमान व ज़मीन का मोजिद है और जब किसी काम का करना ठान लेता है तो उसकी निसबत सिर्फ कह देता है कि ''हो जा'' पस वह (खुद ब खुद) हो जाता है
\end{hindi}}
\flushright{\begin{Arabic}
\quranayah[2][118]
\end{Arabic}}
\flushleft{\begin{hindi}
और जो (मुशरेकीन) कुछ नहीं जानते कहते हैं कि खुदा हमसे (खुद) कलाम क्यों नहीं करता, या हमारे पास (खुद) कोई निशानी क्यों नहीं आती, इसी तरह उन्हीं की सी बाते वह कर चुके हैं जो उनसे पहले थे उन सब के दिल आपस में मिलते जुलते हैं जो लोग यक़ीन रखते हैं उनको तो अपनी निशानियाँ क्यों साफतौर पर दिखा चुके
\end{hindi}}
\flushright{\begin{Arabic}
\quranayah[2][119]
\end{Arabic}}
\flushleft{\begin{hindi}
(ऐ रसूल) हमने तुमको दीने हक़ के साथ (बेहिश्त की) खुशख़बरी देने वाला और (अज़ाब से) डराने वाला बनाकर भेजा है और दोज़ख़ियों के बारे में तुमसे कुछ न पूछा जाएगा
\end{hindi}}
\flushright{\begin{Arabic}
\quranayah[2][120]
\end{Arabic}}
\flushleft{\begin{hindi}
और (ऐ रसूल) न तो यहूदी कभी तुमसे रज़ामंद होगे न नसारा यहाँ तक कि तुम उनके मज़हब की पैरवी करो (ऐ रसूल उनसे) कह दो कि बस खुदा ही की हिदायत तो हिदायत है (बाक़ी ढकोसला है) और अगर तुम इसके बाद भी कि तुम्हारे पास इल्म (क़ुरान) आ चुका है उनकी ख्वाहिशों पर चले तो (याद रहे कि फिर) तुमको खुदा (के ग़ज़ब) से बचाने वाला न कोई सरपरस्त होगा न मददगार
\end{hindi}}
\flushright{\begin{Arabic}
\quranayah[2][121]
\end{Arabic}}
\flushleft{\begin{hindi}
जिन लोगों को हमने किताब (कुरान) दी है वह लोग उसे इस तरह पढ़ते रहते हैं जो उसके पढ़ने का हक़ है यही लोग उस पर ईमान लाते हैं और जो उससे इनकार करते हैं वही लोग घाटे में हैं
\end{hindi}}
\flushright{\begin{Arabic}
\quranayah[2][122]
\end{Arabic}}
\flushleft{\begin{hindi}
बनी इसराईल मेरी उन नेअमतों को याद करो जो मैंनं तुम को दी हैं और ये कि मैंने तुमको सारे जहाँन पर फज़ीलत दी
\end{hindi}}
\flushright{\begin{Arabic}
\quranayah[2][123]
\end{Arabic}}
\flushleft{\begin{hindi}
और उस दिन से डरो जिस दिन कोई शख्स किसी की तरफ से न फिदया हो सकेगा और न उसकी तरफ से कोई मुआवेज़ा क़ुबूल किया जाएगा और न कोई सिफारिश ही फायदा पहुचाँ सकेगी, और न लोग मदद दिए जाएँगे
\end{hindi}}
\flushright{\begin{Arabic}
\quranayah[2][124]
\end{Arabic}}
\flushleft{\begin{hindi}
(ऐ रसूल) बनी इसराईल को वह वक्त भी याद दिलाओ जब इबराहीम को उनके परवरदिगार ने चन्द बातों में आज़माया और उन्होंने पूरा कर दिया तो खुदा ने फरमाया मैं तुमको (लोगों का) पेशवा बनाने वाला हूँ (हज़रत इबराहीम ने) अर्ज़ की और मेरी औलाद में से फरमाया (हाँ मगर) मेरे इस अहद पर ज़ालिमों में से कोई शख्स फ़ायज़ नहीं हो सकता
\end{hindi}}
\flushright{\begin{Arabic}
\quranayah[2][125]
\end{Arabic}}
\flushleft{\begin{hindi}
(ऐ रसूल वह वक्त भी याद दिलाओ) जब हमने ख़ानए काबा को लोगों के सवाब और पनाह की जगह क़रार दी और हुक्म दिया गया कि इबराहीम की (इस) जगह को नमाज़ की जगह बनाओ और इबराहीम व इसमाइल से अहद व पैमान लिया कि मेरे (उस) घर को तवाफ़ और एतक़ाफ़ और रूकू और सजदा करने वालों के वास्ते साफ सुथरा रखो
\end{hindi}}
\flushright{\begin{Arabic}
\quranayah[2][126]
\end{Arabic}}
\flushleft{\begin{hindi}
और (ऐ रसूल वह वक्त भी याद दिलाओ) जब इबराहीम ने दुआ माँगी कि ऐ मेरे परवरदिगार इस (शहर) को पनाह व अमन का शहर बना, और उसके रहने वालों में से जो खुदा और रोज़े आख़िरत पर ईमान लाए उसको तरह-तरह के फल खाने को दें खुदा ने फरमाया (अच्छा मगर) वो कुफ्र इख़तेयार करेगा उसकी दुनिया में चन्द रोज़ (उन चीज़ो से) फायदा उठाने दूँगा फिर (आख़ेरत में) उसको मजबूर करके दोज़ख़ की तरफ खींच ले जाऊँगा और वह बहुत बुरा ठिकाना है
\end{hindi}}
\flushright{\begin{Arabic}
\quranayah[2][127]
\end{Arabic}}
\flushleft{\begin{hindi}
और (वह वक्त याद दिलाओ) जब इबराहीम व इसमाईल ख़ानाए काबा की बुनियादें बुलन्द कर रहे थे (और दुआ) माँगते जाते थे कि ऐ हमारे परवरदिगार हमारी (ये ख़िदमत) कुबूल कर बेशक तू ही (दूआ का) सुनने वाला (और उसका) जानने वाला है
\end{hindi}}
\flushright{\begin{Arabic}
\quranayah[2][128]
\end{Arabic}}
\flushleft{\begin{hindi}
(और) ऐ हमारे पालने वाले तू हमें अपना फरमाबरदार बन्दा बना हमारी औलाद से एक गिरोह (पैदा कर) जो तेरा फरमाबरदार हो, और हमको हमारे हज की जगहों दिखा दे और हमारी तौबा क़ुबूल कर, बेशक तू ही बड़ा तौबा कुबूल करने वाला मेहरबान है
\end{hindi}}
\flushright{\begin{Arabic}
\quranayah[2][129]
\end{Arabic}}
\flushleft{\begin{hindi}
(और) ऐ हमारे पालने वाले मक्के वालों में उन्हीं में से एक रसूल को भेज जो उनको तेरी आयतें पढ़कर सुनाए और आसमानी किताब और अक्ल की बातें सिखाए और उन (के नुफ़ूस) के पाकीज़ा कर दें बेशक तू ही ग़ालिब और साहिबे तदबीर है
\end{hindi}}
\flushright{\begin{Arabic}
\quranayah[2][130]
\end{Arabic}}
\flushleft{\begin{hindi}
और कौन है जो इबराहीम के तरीक़े से नफरत करे मगर जो अपने को अहमक़ बनाए और बेशक हमने उनको दुनिया में भी मुन्तिख़ब कर लिया और वह ज़रूर आख़ेरत में भी अच्छों ही में से होगे
\end{hindi}}
\flushright{\begin{Arabic}
\quranayah[2][131]
\end{Arabic}}
\flushleft{\begin{hindi}
जब उनसे उनके परवरदिगार ने कहा इस्लाम कुबूल करो तो अर्ज़ में सारे जहाँ के परवरदिगार पर इस्लाम लाया
\end{hindi}}
\flushright{\begin{Arabic}
\quranayah[2][132]
\end{Arabic}}
\flushleft{\begin{hindi}
और इसी तरीके क़ी इबराहीम ने अपनी औलाद से वसीयत की और याकूब ने (भी) कि ऐ फरज़न्दों खुदा ने तुम्हारे वास्ते इस दीन (इस्लाम) को पसन्द फरमाया है पस तुम हरगिज़ न मरना मगर मुसलमान ही होकर
\end{hindi}}
\flushright{\begin{Arabic}
\quranayah[2][133]
\end{Arabic}}
\flushleft{\begin{hindi}
(ऐ यहूद) क्या तुम उस वक्त मौजूद थे जब याकूब के सर पर मौत आ खड़ी हुईउस वक्त उन्होंने अपने बेटों से कहा कि मेरे बाद किसी की इबादत करोगे कहने लगे हम आप के माबूद और आप के बाप दादाओं इबराहीम व इस्माइल व इसहाक़ के माबूद व यकता खुदा की इबादत करेंगे और हम उसके फरमाबरदार हैं
\end{hindi}}
\flushright{\begin{Arabic}
\quranayah[2][134]
\end{Arabic}}
\flushleft{\begin{hindi}
(ऐ यहूद) वह लोग थे जो चल बसे जो उन्होंने कमाया उनके आगे आया और जो तुम कमाओगे तुम्हारे आगे आएगा और जो कुछ भी वह करते थे उसकी पूछगछ तुमसे नहीं होगी
\end{hindi}}
\flushright{\begin{Arabic}
\quranayah[2][135]
\end{Arabic}}
\flushleft{\begin{hindi}
(यहूदी ईसाई मुसलमानों से) कहते हैं कि यहूद या नसरानी हो जाओ तो राहे रास्त पर आ जाओगे (ऐ रसूल उनसे) कह दो कि हम इबराहीम के तरीक़े पर हैं जो बातिल से कतरा कर चलते थे और मुशरेकीन से न थे
\end{hindi}}
\flushright{\begin{Arabic}
\quranayah[2][136]
\end{Arabic}}
\flushleft{\begin{hindi}
(और ऐ मुसलमानों तुम ये) कहो कि हम तो खुदा पर ईमान लाए हैं और उस पर जो हम पर नाज़िल किया गया (कुरान) और जो सहीफ़े इबराहीम व इसमाइल व इसहाक़ व याकूब और औलादे याकूब पर नाज़िल हुए थे (उन पर) और जो किताब मूसा व ईसा को दी गई (उस पर) और जो और पैग़म्बरों को उनके परवरदिगार की तरफ से उन्हें दिया गया (उस पर) हम तो उनमें से किसी (एक) में भी तफरीक़ नहीं करते और हम तो खुदा ही के फरमाबरदार हैं
\end{hindi}}
\flushright{\begin{Arabic}
\quranayah[2][137]
\end{Arabic}}
\flushleft{\begin{hindi}
पस अगर ये लोग भी उसी तरह ईमान लाए हैं जिस तरह तुम तो अलबत्ता राहे रास्त पर आ गए और अगर वह इस तरीके से मुँह फेर लें तो बस वह सिर्फ तुम्हारी ही ज़िद पर है तो (ऐ रसूल) उन (के शर) से (बचाने को) तुम्हारे लिए खुदा काफ़ी होगा और वह (सबकी हालत) खूब जानता (और) सुनता है
\end{hindi}}
\flushright{\begin{Arabic}
\quranayah[2][138]
\end{Arabic}}
\flushleft{\begin{hindi}
(मुसलमानों से कहो कि) रंग तो खुदा ही का रंग है जिसमें तुम रंगे गए और खुदाई रंग से बेहतर कौन रंग होगा और हम तो उसी की इबादत करते हैं
\end{hindi}}
\flushright{\begin{Arabic}
\quranayah[2][139]
\end{Arabic}}
\flushleft{\begin{hindi}
(ऐ रसूल) तुम उनसे पूछो कि क्या तुम हम से खुदा के बारे झगड़ते हो हालाँकि वही हमारा (भी) परवरदिगार है (वही) तुम्हारा भी (परवरदिगार है) हमारे लिए है हमारी कारगुज़ारियाँ और तुम्हारे लिए तुम्हारी कारसतानियाँ और हम तो निरेखरे उसी में हैं
\end{hindi}}
\flushright{\begin{Arabic}
\quranayah[2][140]
\end{Arabic}}
\flushleft{\begin{hindi}
क्या तुम कहते हो कि इबराहीम व इसमाइल व इसहाक़ व आलौदें याकूब सब के सब यहूदी या नसरानी थे (ऐ रसूल उनसे) पूछो तो कि तुम ज्यादा वाक़िफ़ हो या खुदा और उससे बढ़कर कौन ज़ालिम होगा जिसके पास खुदा की तरफ से गवाही (मौजूद) हो (कि वह यहूदी न थे) और फिर वह छिपाए और जो कुछ तुम करते हो खुदा उससे बेख़बर नहीं
\end{hindi}}
\flushright{\begin{Arabic}
\quranayah[2][141]
\end{Arabic}}
\flushleft{\begin{hindi}
ये वह लोग थे जो सिधार चुके जो कुछ कमा गए उनके लिए था और जो कुछ तुम कमाओगे तुम्हारे लिए होगा और जो कुछ वह कर गुज़रे उसकी पूछगछ तुमसे न होगी
\end{hindi}}
\flushright{\begin{Arabic}
\quranayah[2][142]
\end{Arabic}}
\flushleft{\begin{hindi}
बाज़ अहमक़ लोग ये कह बैठेगें कि मुसलमान जिस क़िबले बैतुल मुक़द्दस की तरफ पहले से सजदा करते थे उस से दूसरे क़िबले की तरफ मुड़ जाने का बाइस हुआ। ऐ रसूल तुम उनके जवाब में कहो कि पूरब पश्चिम सब ख़ुदा का है जिसे चाहता है सीधे रास्ते की तरफ हिदायत करता है
\end{hindi}}
\flushright{\begin{Arabic}
\quranayah[2][143]
\end{Arabic}}
\flushleft{\begin{hindi}
और जिस तरह तुम्हारे क़िबले के बारे में हिदायत की उसी तरह तुम को आदिल उम्मत बनाया ताकि और लोगों के मुक़ाबले में तुम गवाह बनो और रसूल मोहम्मद तुम्हारे मुक़ाबले में गवाह बनें और (ऐ रसूल) जिस क़िबले की तरफ़ तुम पहले सज़दा करते थे हम ने उसको को सिर्फ इस वजह से क़िबला क़रार दिया था कि जब क़िबला बदला जाए तो हम उन लोगों को जो रसूल की पैरवी करते हैं हम उन लोगों से अलग देख लें जो उलटे पाव फिरते हैं अगरचे ये उलट फेर सिवा उन लोगों के जिन की ख़ुदा ने हिदायत की है सब पर शाक़ ज़रुर है और ख़ुदा ऐसा नहीं है कि तुम्हारे ईमान नमाज़ को जो बैतुलमुक़द्दस की तरफ पढ़ चुके हो बरबाद कर दे बेशक ख़ुदा लोगों पर बड़ा ही रफ़ीक व मेहरबान है।
\end{hindi}}
\flushright{\begin{Arabic}
\quranayah[2][144]
\end{Arabic}}
\flushleft{\begin{hindi}
ऐ रसूल क़िबला बदलने के वास्ते बेशक तुम्हारा बार बार आसमान की तरफ मुँह करना हम देख रहे हैं तो हम ज़रुर तुम को ऐसे क़िबले की तरफ फेर देगें कि तुम निहाल हो जाओ अच्छा तो नमाज़ ही में तुम मस्ज़िदे मोहतरम काबे की तरफ मुँह कर लो और ऐ मुसलमानों तुम जहाँ कही भी हो उसी की तरफ़ अपना मुँह कर लिया करो और जिन लोगों को किताब तौरेत वगैरह दी गयी है वह बख़ूबी जानते हैं कि ये तबदील क़िबले बहुत बजा व दुरुस्त है और उस के परवरदिगार की तरफ़ से है और जो कुछ वह लोग करते हैं उस से ख़ुदा बेख़बर नही
\end{hindi}}
\flushright{\begin{Arabic}
\quranayah[2][145]
\end{Arabic}}
\flushleft{\begin{hindi}
और अगर अहले किताब के सामने दुनिया की सारी दलीले पेश कर दोगे तो भी वह तुम्हारे क़िबले को न मानेंगें और न तुम ही उनके क़िबले को मानने वाले हो और ख़ुद अहले किताब भी एक दूसरे के क़िबले को नहीं मानते और जो इल्म क़ुरान तुम्हारे पास आ चुका है उसके बाद भी अगर तुम उनकी ख्वाहिश पर चले तो अलबत्ता तुम नाफ़रमान हो जाओगे
\end{hindi}}
\flushright{\begin{Arabic}
\quranayah[2][146]
\end{Arabic}}
\flushleft{\begin{hindi}
जिन लोगों को हमने किताब (तौरैत वग़ैरह) दी है वह जिस तरह अपने बेटों को पहचानते है उसी तरह तरह वह उस पैग़म्बर को भी पहचानते हैं और उन में कुछ लोग तो ऐसे भी हैं जो दीदए व दानिस्ता (जान बुझकर) हक़ बात को छिपाते हैं
\end{hindi}}
\flushright{\begin{Arabic}
\quranayah[2][147]
\end{Arabic}}
\flushleft{\begin{hindi}
ऐ रसूल तबदीले क़िबला तुम्हारे परवरदिगार की तरफ से हक़ है पस तुम कहीं शक करने वालों में से न हो जाना
\end{hindi}}
\flushright{\begin{Arabic}
\quranayah[2][148]
\end{Arabic}}
\flushleft{\begin{hindi}
और हर फरीक़ के वास्ते एक सिम्त है उसी की तरफ वह नमाज़ में अपना मुँह कर लेता है पस तुम ऐ मुसलमानों झगड़े को छोड़ दो और नेकियों मे उन से लपक के आगे बढ़ जाओ तुम जहाँ कहीं होगे ख़ुदा तुम सबको अपनी तरफ ले आऐगा बेशक ख़ुदा हर चीज़ पर क़ादिर है
\end{hindi}}
\flushright{\begin{Arabic}
\quranayah[2][149]
\end{Arabic}}
\flushleft{\begin{hindi}
और (ऐ रसूल) तुम जहाँ से जाओ (यहाँ तक मक्का से) तो भी नमाज़ मे तुम अपना मुँह मस्ज़िदे मोहतरम (काबा) की तरफ़ कर लिया करो और बेशक ये नया क़िबला तुम्हारे परवरदिगार की तरफ से हक़ है
\end{hindi}}
\flushright{\begin{Arabic}
\quranayah[2][150]
\end{Arabic}}
\flushleft{\begin{hindi}
और तुम्हारे कामों से ख़ुदा ग़ाफिल नही है और (ऐ रसूल) तुम जहाँ से जाओ (यहाँ तक के मक्का से तो भी) तुम (नमाज़ में) अपना मुँह मस्ज़िदे हराम की तरफ कर लिया करो और (ऐ रसूल) तुम जहाँ कही हुआ करो तो नमाज़ में अपना मुँह उसी काबा की तरफ़ कर लिया करो (बार बार हुक्म देने का एक फायदा ये है ताकि लोगों का इल्ज़ाम तुम पर न आने पाए मगर उन में से जो लोग नाहक़ हठधर्मी करते हैं वह तो ज़रुर इल्ज़ाम देगें) तो तुम लोग उनसे डरो नहीं और सिर्फ़ मुझसे डरो और (दूसरा फ़ायदा ये है) ताकि तुम पर अपनी नेअमत पूरी कर दूँ
\end{hindi}}
\flushright{\begin{Arabic}
\quranayah[2][151]
\end{Arabic}}
\flushleft{\begin{hindi}
और तीसरा फायदा ये है ताकि तुम हिदायत पाओ मुसलमानों ये एहसान भी वैसा ही है जैसे हम ने तुम में तुम ही में का एक रसूल भेजा जो तुमको हमारी आयतें पढ़ कर सुनाए और तुम्हारे नफ्स को पाकीज़ा करे और तुम्हें किताब क़ुरान और अक्ल की बातें सिखाए और तुम को वह बातें बतांए जिन की तुम्हें पहले से खबर भी न थी
\end{hindi}}
\flushright{\begin{Arabic}
\quranayah[2][152]
\end{Arabic}}
\flushleft{\begin{hindi}
पस तुम हमारी याद रखो तो मै भी तुम्हारा ज़िक्र (खैर) किया करुगाँ और मेरा शुक्रिया अदा करते रहो और नाशुक्री न करो
\end{hindi}}
\flushright{\begin{Arabic}
\quranayah[2][153]
\end{Arabic}}
\flushleft{\begin{hindi}
ऐ ईमानदारों मुसीबत के वक्त सब्र और नमाज़ के ज़रिए से ख़ुदा की मदद माँगों बेशक ख़ुदा सब्र करने वालों ही का साथी है
\end{hindi}}
\flushright{\begin{Arabic}
\quranayah[2][154]
\end{Arabic}}
\flushleft{\begin{hindi}
और जो लोग ख़ुदा की राह में मारे गए उन्हें कभी मुर्दा न कहना बल्कि वह लोग ज़िन्दा हैं मगर तुम उनकी ज़िन्दगी की हक़ीकत का कुछ भी शऊर नहीं रखते
\end{hindi}}
\flushright{\begin{Arabic}
\quranayah[2][155]
\end{Arabic}}
\flushleft{\begin{hindi}
और हम तुम्हें कुछ खौफ़ और भूख से और मालों और जानों और फलों की कमी से ज़रुर आज़माएगें और (ऐ रसूल) ऐसे सब्र करने वालों को ख़ुशख़बरी दे दो
\end{hindi}}
\flushright{\begin{Arabic}
\quranayah[2][156]
\end{Arabic}}
\flushleft{\begin{hindi}
कि जब उन पर कोई मुसीबत आ पड़ी तो वह (बेसाख्ता) बोल उठे हम तो ख़ुदा ही के हैं और हम उसी की तरफ लौट कर जाने वाले हैं
\end{hindi}}
\flushright{\begin{Arabic}
\quranayah[2][157]
\end{Arabic}}
\flushleft{\begin{hindi}
उन्हीं लोगों पर उनके परवरदिगार की तरफ से इनायतें हैं और रहमत और यही लोग हिदायत याफ्ता है
\end{hindi}}
\flushright{\begin{Arabic}
\quranayah[2][158]
\end{Arabic}}
\flushleft{\begin{hindi}
बेशक (कोहे) सफ़ा और (कोह) मरवा ख़ुदा की निशानियों में से हैं पस जो शख्स ख़ानए काबा का हज या उमरा करे उस पर उन दोनो के (दरमियान) तवाफ़ (आमद ओ रफ्त) करने में कुछ गुनाह नहीं (बल्कि सवाब है) और जो शख्स खुश खुश नेक काम करे तो फिर ख़ुदा भी क़दरदाँ (और) वाक़िफ़कार है
\end{hindi}}
\flushright{\begin{Arabic}
\quranayah[2][159]
\end{Arabic}}
\flushleft{\begin{hindi}
बेशक जो लोग हमारी इन रौशन दलीलों और हिदायतों को जिन्हें हमने नाज़िल किया उसके बाद छिपाते हैं जबकि हम किताब तौरैत में लोगों के सामने साफ़ साफ़ बयान कर चुके हैं तो यही लोग हैं जिन पर ख़ुदा भी लानत करता है और लानत करने वाले भी लानत करते हैं
\end{hindi}}
\flushright{\begin{Arabic}
\quranayah[2][160]
\end{Arabic}}
\flushleft{\begin{hindi}
मगर जिन लोगों ने (हक़ छिपाने से) तौबा की और अपनी ख़राबी की इसलाह कर ली और जो किताबे ख़ुदा में है साफ़ साफ़ बयान कर दिया पस उन की तौबा मै क़ुबूल करता हूँ और मै तो बड़ा तौबा क़ुबूल करने वाला मेहरबान हूँ
\end{hindi}}
\flushright{\begin{Arabic}
\quranayah[2][161]
\end{Arabic}}
\flushleft{\begin{hindi}
बेशक जिन लोगों नें कुफ्र एख्तेयार किया और कुफ़्र ही की हालत में मर गए उन्ही पर ख़ुदा की और फरिश्तों की और तमाम लोगों की लानत है हमेशा उसी फटकार में रहेंगे
\end{hindi}}
\flushright{\begin{Arabic}
\quranayah[2][162]
\end{Arabic}}
\flushleft{\begin{hindi}
न तो उनके अज़ाब ही में तख्फ़ीफ़ (कमी) की जाएगी
\end{hindi}}
\flushright{\begin{Arabic}
\quranayah[2][163]
\end{Arabic}}
\flushleft{\begin{hindi}
और न उनको अज़ाब से मोहलत दी जाएगी और तुम्हारा माबूद तो वही यकता ख़ुदा है उस के सिवा कोई माबूद नहीं जो बड़ा मेहरबान रहम वाला है
\end{hindi}}
\flushright{\begin{Arabic}
\quranayah[2][164]
\end{Arabic}}
\flushleft{\begin{hindi}
बेशक आसमान व ज़मीन की पैदाइश और रात दिन के रद्दो बदल में और क़श्तियों जहाज़ों में जो लोगों के नफे क़ी चीज़े (माले तिजारत वगैरह दरिया) में ले कर चलते हैं और पानी में जो ख़ुदा ने आसमान से बरसाया फिर उस से ज़मीन को मुर्दा (बेकार) होने के बाद जिला दिया (शादाब कर दिया) और उस में हर क़िस्म के जानवर फैला दिये और हवाओं के चलाने में और अब्र में जो आसमान व ज़मीन के दरमियान ख़ुदा के हुक्म से घिरा रहता है (इन सब बातों में) अक्ल वालों के लिए बड़ी बड़ी निशानियाँ हैं
\end{hindi}}
\flushright{\begin{Arabic}
\quranayah[2][165]
\end{Arabic}}
\flushleft{\begin{hindi}
और बाज़ लोग ऐसे भी हैं जो ख़ुदा के सिवा औरों को भी ख़ुदा का मिसल व शरीक बनाते हैं (और) जैसी मोहब्बत ख़ुदा से रखनी चाहिए वैसी ही उन से रखते हैं और जो लोग ईमानदार हैं वह उन से कहीं बढ़ कर ख़ुदा की उलफ़त रखते हैं और काश ज़ालिमों को (इस वक्त) वह बात सूझती जो अज़ाब देखने के बाद सूझेगी कि यक़ीनन हर तरह की क़ूवत ख़ुदा ही को है और ये कि बेशक ख़ुदा बड़ा सख्त अज़ाब वाला है
\end{hindi}}
\flushright{\begin{Arabic}
\quranayah[2][166]
\end{Arabic}}
\flushleft{\begin{hindi}
(वह क्या सख्त वक्त होगा) जब पेशवा लोग अपने पैरवो से अपना पीछा छुड़ाएगे और (ब चश्में ख़ुद) अज़ाब को देखेगें और उनके बाहमी ताल्लुक़ात टूट जाएँगे
\end{hindi}}
\flushright{\begin{Arabic}
\quranayah[2][167]
\end{Arabic}}
\flushleft{\begin{hindi}
और पैरव कहने लगेंगे कि अगर हमें कहीं फिर (दुनिया में) पलटना मिले तो हम भी उन से इसी तरह अलग हो जायेंगे जिस तरह एैन वक्त पर ये लोग हम से अलग हो गए यूँ ही ख़ुदा उन के आमाल को दिखाएगा जो उन्हें (सर तापा पास ही) पास दिखाई देंगें और फिर भला कब वह दोज़ख़ से निकल सकतें हैं
\end{hindi}}
\flushright{\begin{Arabic}
\quranayah[2][168]
\end{Arabic}}
\flushleft{\begin{hindi}
ऐ लोगों जो कुछ ज़मीन में हैं उस में से हलाल व पाकीज़ा चीज़ (शौक़ से) खाओ और शैतान के क़दम ब क़दम न चलो वह तो तुम्हारा ज़ाहिर ब ज़ाहिर दुश्मन है
\end{hindi}}
\flushright{\begin{Arabic}
\quranayah[2][169]
\end{Arabic}}
\flushleft{\begin{hindi}
वह तो तुम्हें बुराई और बदकारी ही का हुक्म करेगा और ये चाहेगा कि तुम बे जाने बूझे ख़ुदा पर बोहतान बाँधों
\end{hindi}}
\flushright{\begin{Arabic}
\quranayah[2][170]
\end{Arabic}}
\flushleft{\begin{hindi}
और जब उन से कहा जाता है कि जो हुक्म ख़ुदा की तरफ से नाज़िल हुआ है उस को मानो तो कहते हैं कि नहीं बल्कि हम तो उसी तरीक़े पर चलेंगे जिस पर हमने अपने बाप दादाओं को पाया अगरचे उन के बाप दादा कुछ भी न समझते हों और न राहे रास्त ही पर चलते रहे हों
\end{hindi}}
\flushright{\begin{Arabic}
\quranayah[2][171]
\end{Arabic}}
\flushleft{\begin{hindi}
और जिन लोगों ने कुफ्र एख्तेयार किया उन की मिसाल तो उस शख्स की मिसाल है जो ऐसे जानवर को पुकार के अपना हलक़ फाड़े जो आवाज़ और पुकार के सिवा सुनता (समझता ख़ाक) न हो ये लोग बहरे गूँगे अन्धें हैं कि ख़ाक नहीं समझते
\end{hindi}}
\flushright{\begin{Arabic}
\quranayah[2][172]
\end{Arabic}}
\flushleft{\begin{hindi}
ऐ ईमानदारों जो कुछ हम ने तुम्हें दिया है उस में से सुथरी चीज़ें (शौक़ से) खाओं और अगर ख़ुदा ही की इबादत करते हो तो उसी का शुक्र करो
\end{hindi}}
\flushright{\begin{Arabic}
\quranayah[2][173]
\end{Arabic}}
\flushleft{\begin{hindi}
उसने तो तुम पर बस मुर्दा जानवर और खून और सूअर का गोश्त और वह जानवर जिस पर ज़बह के वक्त ख़ुदा के सिवा और किसी का नाम लिया गया हो हराम किया है पस जो शख्स मजबूर हो और सरकशी करने वाला और ज्यादती करने वाला न हो (और उनमे से कोई चीज़ खा ले) तो उसपर गुनाह नहीं है बेशक ख़ुदा बड़ा बख्शने वाला मेहरबान है
\end{hindi}}
\flushright{\begin{Arabic}
\quranayah[2][174]
\end{Arabic}}
\flushleft{\begin{hindi}
बेशक जो लोग इन बातों को जो ख़ुदा ने किताब में नाज़िल की है छिपाते हैं और उसके बदले थोड़ी सी क़ीमत (दुनयावी नफ़ा) ले लेतें है ये लोग बस अंगारों से अपने पेट भरते हैं और क़यामत के दिन ख़ुदा उन से बात तक तो करेगा नहीं और न उन्हें (गुनाहों से) पाक करेगा और उन्हीं के लिए दर्दनाक अज़ाब है
\end{hindi}}
\flushright{\begin{Arabic}
\quranayah[2][175]
\end{Arabic}}
\flushleft{\begin{hindi}
यही लोग वह हैं जिन्होंने हिदायत के बदले गुमराही मोल ली और बख्यिय (ख़ुदा) के बदले अज़ाब पस वह लोग दोज़ख़ की आग के क्योंकर बरदाश्त करेंगे
\end{hindi}}
\flushright{\begin{Arabic}
\quranayah[2][176]
\end{Arabic}}
\flushleft{\begin{hindi}
ये इसलिए कि ख़ुदा ने बरहक़ किताब नाज़िल की और बेशक जिन लोगों ने किताबे ख़ुदा में रद्दो बदल की वह लोग बड़े पल्ले दरजे की मुख़ालेफत में हैं
\end{hindi}}
\flushright{\begin{Arabic}
\quranayah[2][177]
\end{Arabic}}
\flushleft{\begin{hindi}
नेकी कुछ यही थोड़ी है कि नमाज़ में अपने मुँह पूरब या पश्चिम की तरफ़ कर लो बल्कि नेकी तो उसकी है जो ख़ुदा और रोज़े आख़िरत और फ़रिश्तों और ख़ुदा की किताबों और पैग़म्बरों पर ईमान लाए और उसकी उलफ़त में अपना माल क़राबत दारों और यतीमों और मोहताजो और परदेसियों और माँगने वालों और लौन्डी ग़ुलाम (के गुलू खलासी) में सर्फ करे और पाबन्दी से नमाज़ पढे और ज़कात देता रहे और जब कोई एहद किया तो अपने क़ौल के पूरे हो और फ़क्र व फाक़ा रन्ज और घुटन के वक्त साबित क़दम रहे यही लोग वह हैं जो दावए ईमान में सच्चे निकले और यही लोग परहेज़गार है
\end{hindi}}
\flushright{\begin{Arabic}
\quranayah[2][178]
\end{Arabic}}
\flushleft{\begin{hindi}
ऐ मोमिनों जो लोग (नाहक़) मार डाले जाएँ उनके बदले में तुम को जान के बदले जान लेने का हुक्म दिया जाता है आज़ाद के बदले आज़ाद और ग़ुलाम के बदले ग़ुलाम और औरत के बदले औरत पस जिस (क़ातिल) को उसके ईमानी भाई तालिबे केसास की तरफ से कुछ माफ़ कर दिया जाये तो उसे भी उसके क़दम ब क़दम नेकी करना और ख़ुश मआमलती से (ख़ून बहा) अदा कर देना चाहिए ये तुम्हारे परवरदिगार की तरफ आसानी और मेहरबानी है फिर उसके बाद जो ज्यादती करे तो उस के लिए दर्दनाक अज़ाब है
\end{hindi}}
\flushright{\begin{Arabic}
\quranayah[2][179]
\end{Arabic}}
\flushleft{\begin{hindi}
और ऐ अक़लमनदों क़सास (के क़वाएद मुक़र्रर कर देने) में तुम्हारी ज़िन्दगी है (और इसीलिए जारी किया गया है ताकि तुम ख़ूंरेज़ी से) परहेज़ करो
\end{hindi}}
\flushright{\begin{Arabic}
\quranayah[2][180]
\end{Arabic}}
\flushleft{\begin{hindi}
(मुसलमानों) तुम को हुक्म दिया जाता है कि जब तुम में से किसी के सामने मौत आ खड़ी हो बशर्ते कि वह कुछ माल छोड़ जाएं तो माँ बाप और क़राबतदारों के लिए अच्छी वसीयत करें जो ख़ुदा से डरते हैं उन पर ये एक हक़ है
\end{hindi}}
\flushright{\begin{Arabic}
\quranayah[2][181]
\end{Arabic}}
\flushleft{\begin{hindi}
फिर जो सुन चुका उसके बाद उसे कुछ का कुछ कर दे तो उस का गुनाह उन्हीं लोगों की गरदन पर है जो उसे बदल डालें बेशक ख़ुदा सब कुछ जानता और सुनता है
\end{hindi}}
\flushright{\begin{Arabic}
\quranayah[2][182]
\end{Arabic}}
\flushleft{\begin{hindi}
(हाँ अलबत्ता) जो शख्स वसीयत करने वाले से बेजा तरफ़दारी या बे इन्साफी का ख़ौफ रखता है और उन वारिसों में सुलह करा दे तो उस पर बदलने का कुछ गुनाह नहीं है बेशक ख़ुदा बड़ा बख्शने वाला मेहरबान है
\end{hindi}}
\flushright{\begin{Arabic}
\quranayah[2][183]
\end{Arabic}}
\flushleft{\begin{hindi}
ऐ ईमानदारों रोज़ा रखना जिस तरह तुम से पहले के लोगों पर फर्ज था उसी तरफ तुम पर भी फर्ज़ किया गया ताकि तुम उस की वजह से बहुत से गुनाहों से बचो
\end{hindi}}
\flushright{\begin{Arabic}
\quranayah[2][184]
\end{Arabic}}
\flushleft{\begin{hindi}
(वह भी हमेशा नहीं बल्कि) गिनती के चन्द रोज़ इस पर भी (रोज़े के दिनों में) जो शख्स तुम में से बीमार हो या सफर में हो तो और दिनों में जितने क़ज़ा हुए हो) गिन के रख ले और जिन्हें रोज़ा रखने की कूवत है और न रखें तो उन पर उस का बदला एक मोहताज को खाना खिला देना है और जो शख्स अपनी ख़ुशी से भलाई करे तो ये उस के लिए ज्यादा बेहतर है और अगर तुम समझदार हो तो (समझ लो कि फिदये से) रोज़ा रखना तुम्हारे हक़ में बहरहाल अच्छा है
\end{hindi}}
\flushright{\begin{Arabic}
\quranayah[2][185]
\end{Arabic}}
\flushleft{\begin{hindi}
(रोज़ों का) महीना रमज़ान है जिस में क़ुरान नाज़िल किया गया जो लोगों का रहनुमा है और उसमें रहनुमाई और (हक़ व बातिल के) तमीज़ की रौशन निशानियाँ हैं (मुसलमानों) तुम में से जो शख्स इस महीनें में अपनी जगह पर हो तो उसको चाहिए कि रोज़ा रखे और जो शख्स बीमार हो या फिर सफ़र में हो तो और दिनों में रोज़े की गिनती पूरी करे ख़ुदा तुम्हारे साथ आसानी करना चाहता है और तुम्हारे साथ सख्ती करनी नहीं चाहता और (शुमार का हुक्म इस लिए दिया है) ताकि तुम (रोज़ो की) गिनती पूरी करो और ताकि ख़ुदा ने जो तुम को राह पर लगा दिया है उस नेअमत पर उस की बड़ाई करो और ताकि तुम शुक्र गुज़ार बनो
\end{hindi}}
\flushright{\begin{Arabic}
\quranayah[2][186]
\end{Arabic}}
\flushleft{\begin{hindi}
(ऐ रसूल) जब मेरे बन्दे मेरा हाल तुमसे पूछे तो (कह दो कि) मै उन के पास ही हूँ और जब मुझसे कोई दुआ माँगता है तो मै हर दुआ करने वालों की दुआ (सुन लेता हूँ और जो मुनासिब हो तो) क़ुबूल करता हूँ पस उन्हें चाहिए कि मेरा भी कहना माने) और मुझ पर ईमान लाएँ
\end{hindi}}
\flushright{\begin{Arabic}
\quranayah[2][187]
\end{Arabic}}
\flushleft{\begin{hindi}
ताकि वह सीधी राह पर आ जाए (मुसलमानों) तुम्हारे वास्ते रोज़ों की रातों में अपनी बीवियों के पास जाना हलाल कर दिया गया औरतें (गोया) तुम्हारी चोली हैं और तुम (गोया उन के दामन हो) ख़ुदा ने देखा कि तुम (गुनाह) करके अपना नुकसान करते (कि ऑंख बचा के अपनी बीबी के पास चले जाते थे) तो उसने तुम्हारी तौबा क़ुबूल की और तुम्हारी ख़ता से दर गुज़र किया पस तुम अब उनसे हम बिस्तरी करो और (औलाद) जो कुछ ख़ुदा ने तुम्हारे लिए (तक़दीर में) लिख दिया है उसे माँगों और खाओ और पियो यहाँ तक कि सुबह की सफेद धारी (रात की) काली धारी से आसमान पर पूरब की तरफ़ तक तुम्हें साफ नज़र आने लगे फिर रात तक रोज़ा पूरा करो और हाँ जब तुम मस्ज़िदों में एतेकाफ़ करने बैठो तो उन से (रात को भी) हम बिस्तरी न करो ये ख़ुदा की (मुअय्युन की हुई) हदे हैं तो तुम उनके पास भी न जाना यूँ खुल्लम खुल्ला ख़ुदा अपने एहकाम लोगों के सामने बयान करता है ताकि वह लोग (नाफ़रमानी से) बचें
\end{hindi}}
\flushright{\begin{Arabic}
\quranayah[2][188]
\end{Arabic}}
\flushleft{\begin{hindi}
और आपस में एक दूसरे का माल नाहक़ न खाओ और न माल को (रिश्वत में) हुक्काम के यहाँ झोंक दो ताकि लोगों के माल में से (जो) कुछ हाथ लगे नाहक़ ख़ुर्द बुर्द कर जाओ हालाकि तुम जानते हो
\end{hindi}}
\flushright{\begin{Arabic}
\quranayah[2][189]
\end{Arabic}}
\flushleft{\begin{hindi}
(ऐ रसूल) तुम से लोग चाँद के बारे में पूछते हैं (कि क्यो घटता बढ़ता है) तुम कह दो कि उससे लोगों के (दुनयावी) अम्र और हज के अवक़ात मालूम होते है और ये कोई भली बात नही है कि घरो में पिछवाड़े से फाँद के) आओ बल्कि नेकी उसकी है जो परहेज़गारी करे और घरों में आना हो तो) उनके दरवाजों क़ी तरफ से आओ और ख़ुदा से डरते रहो ताकि तुम मुराद को पहुँचो
\end{hindi}}
\flushright{\begin{Arabic}
\quranayah[2][190]
\end{Arabic}}
\flushleft{\begin{hindi}
और जो लोग तुम से लड़े तुम (भी) ख़ुदा की राह में उनसे लड़ो और ज्यादती न करो (क्योंकि) ख़ुदा ज्यादती करने वालों को हरगिज़ दोस्त नहीं रखता
\end{hindi}}
\flushright{\begin{Arabic}
\quranayah[2][191]
\end{Arabic}}
\flushleft{\begin{hindi}
और तुम उन (मुशरिकों) को जहाँ पाओ मार ही डालो और उन लोगों ने जहाँ (मक्का) से तुम्हें शहर बदर किया है तुम भी उन्हें निकाल बाहर करो और फितना परदाज़ी (शिर्क) खूँरेज़ी से भी बढ़ के है और जब तक वह लोग (कुफ्फ़ार) मस्ज़िद हराम (काबा) के पास तुम से न लडे तुम भी उन से उस जगह न लड़ों पस अगर वह तुम से लड़े तो बेखटके तुम भी उन को क़त्ल करो काफ़िरों की यही सज़ा है
\end{hindi}}
\flushright{\begin{Arabic}
\quranayah[2][192]
\end{Arabic}}
\flushleft{\begin{hindi}
फिर अगर वह लोग बाज़ रहें तो बेशक ख़ुदा बड़ा बख्शने वाला मेहरबान है
\end{hindi}}
\flushright{\begin{Arabic}
\quranayah[2][193]
\end{Arabic}}
\flushleft{\begin{hindi}
और उन से लड़े जाओ यहाँ तक कि फ़साद बाक़ी न रहे और सिर्फ ख़ुदा ही का दीन रह जाए फिर अगर वह लोग बाज़ रहे तो उन पर ज्यादती न करो क्योंकि ज़ालिमों के सिवा किसी पर ज्यादती (अच्छी) नहीं
\end{hindi}}
\flushright{\begin{Arabic}
\quranayah[2][194]
\end{Arabic}}
\flushleft{\begin{hindi}
हुरमत वाला महीना हुरमत वाले महीने के बराबर है (और कुछ महीने की खुसूसियत नहीं) सब हुरमत वाली चीजे एक दूसरे के बराबर हैं पस जो शख्स तुम पर ज्यादती करे तो जैसी ज्यादती उसने तुम पर की है वैसी ही ज्यादती तुम भी उस पर करो और ख़ुदा से डरते रहो और खूब समझ लो कि ख़ुदा परहेज़गारों का साथी है
\end{hindi}}
\flushright{\begin{Arabic}
\quranayah[2][195]
\end{Arabic}}
\flushleft{\begin{hindi}
और ख़ुदा की राह में ख़र्च करो और अपने हाथ जान हलाकत मे न डालो और नेकी करो बेशक ख़ुदा नेकी करने वालों को दोस्त रखता है
\end{hindi}}
\flushright{\begin{Arabic}
\quranayah[2][196]
\end{Arabic}}
\flushleft{\begin{hindi}
और सिर्फ ख़ुदा ही के वास्ते हज और उमरा को पूरा करो अगर तुम बीमारी वगैरह की वजह से मजबूर हो जाओ तो फिर जैसी क़ुरबानी मयस्सर आये (कर दो) और जब तक कुरबानी अपनी जगह पर न पहुँच जाये अपने सर न मुँडवाओ फिर जब तुम में से कोई बीमार हो या उसके सर में कोई तकलीफ हो तो (सर मुँडवाने का बदला) रोजे या खैरात या कुरबानी है पस जब मुतमइन रहों तो जो शख्स हज तमत्तो का उमरा करे तो उसको जो कुरबानी मयस्सर आये करनी होगी और जिस से कुरबानी ना मुमकिन हो तो तीन रोजे ज़माना ए हज में (रखने होंगे) और सात रोजे ज़ब तुम वापस आओ ये पूरा दहाई है ये हुक्म उस शख्स के वास्ते है जिस के लड़के बाले मस्ज़िदुल हराम (मक्का) के बाशिन्दे न हो और ख़ुदा से डरो और समझ लो कि ख़ुदा बड़ा सख्त अज़ाब वाला है
\end{hindi}}
\flushright{\begin{Arabic}
\quranayah[2][197]
\end{Arabic}}
\flushleft{\begin{hindi}
हज के महीने तो (अब सब को) मालूम हैं (शव्वाल, ज़ीक़ादा, जिलहज) पस जो शख्स उन महीनों में अपने ऊपर हज लाज़िम करे तो (एहराम से आख़िर हज तक) न औरत के पास जाए न कोई और गुनाह करे और न झगडे और नेकी का कोई सा काम भी करों तो ख़ुदा उस को खूब जानता है और (रास्ते के लिए) ज़ाद राह मुहिय्या करो और सब मे बेहतर ज़ाद राह परहेज़गारी है और ऐ अक्लमन्दों मुझ से डरते रहो
\end{hindi}}
\flushright{\begin{Arabic}
\quranayah[2][198]
\end{Arabic}}
\flushleft{\begin{hindi}
इस में कोई इल्ज़ाम नहीं है कि (हज के साथ) तुम अपने परवरदिगार के फज़ल (नफ़ा तिजारत) की ख्वाहिश करो और फिर जब तुम अरफात से चल खड़े हो तो मशअरुल हराम के पास ख़ुदा का जिक्र करो और उस की याद भी करो तो जिस तरह तुम्हे बताया है अगरचे तुम इसके पहले तो गुमराहो से थे
\end{hindi}}
\flushright{\begin{Arabic}
\quranayah[2][199]
\end{Arabic}}
\flushleft{\begin{hindi}
फिर जहाँ से लोग चल खड़े हों वहीं से तुम भी चल खड़े हो और उससे मग़फिरत की दुआ माँगों बेशक ख़ुदा बड़ा बख्शने वाला मेहरबान है
\end{hindi}}
\flushright{\begin{Arabic}
\quranayah[2][200]
\end{Arabic}}
\flushleft{\begin{hindi}
फिर जब तुम अरक़ाने हज बजा ला चुको तो तुम इस तरह ज़िक्रे ख़ुदा करो जिस तरह तुम अपने बाप दादाओं का ज़िक्र करते हो बल्कि उससे बढ़ कर के फिर बाज़ लोग ऐसे हैं जो कहते हैं कि ऐ मेरे परवरदिगार हमको जो (देना है) दुनिया ही में दे दे हालाकि (फिर) आख़िरत में उनका कुछ हिस्सा नहीं
\end{hindi}}
\flushright{\begin{Arabic}
\quranayah[2][201]
\end{Arabic}}
\flushleft{\begin{hindi}
और बाज़ बन्दे ऐसे हैं कि जो दुआ करते हैं कि ऐ मेरे पालने वाले मुझे दुनिया में नेअमत दे और आख़िरत में सवाब दे और दोज़ख़ की बाग से बचा
\end{hindi}}
\flushright{\begin{Arabic}
\quranayah[2][202]
\end{Arabic}}
\flushleft{\begin{hindi}
यही वह लोग हैं जिनके लिए अपनी कमाई का हिस्सा चैन है
\end{hindi}}
\flushright{\begin{Arabic}
\quranayah[2][203]
\end{Arabic}}
\flushleft{\begin{hindi}
और ख़ुदा बहुत जल्द हिसाब लेने वाला है (निस्फ़) और इन गिनती के चन्द दिनों तक (तो) ख़ुदा का ज़िक्र करो फिर जो शख्स जल्दी कर बैठै और (मिना) से और दो ही दिन में चल ख़ड़ा हो तो उस पर भी गुनाह नहीं है और जो (तीसरे दिन तक) ठहरा रहे उस पर भी कुछ गुनाह नही लेकिन यह रियायत उसके वास्ते है जो परहेज़गार हो, और खुदा से डरते रहो और यक़ीन जानो कि एक दिन तुम सब के सब उसकी तरफ क़ब्रों से उठाए जाओगे
\end{hindi}}
\flushright{\begin{Arabic}
\quranayah[2][204]
\end{Arabic}}
\flushleft{\begin{hindi}
ऐ रसूल बाज़ लोग मुनाफिक़ीन से ऐसे भी हैं जिनकी चिकनी चुपड़ी बातें (इस ज़रा सी) दुनयावी ज़िन्दगी में तुम्हें बहुत भाती है और वह अपनी दिली मोहब्बत पर ख़ुदा को गवाह मुक़र्रर करते हैं हालॉकि वह तुम्हारे दुश्मनों में सबसे ज्यादा झगड़ालू हैं
\end{hindi}}
\flushright{\begin{Arabic}
\quranayah[2][205]
\end{Arabic}}
\flushleft{\begin{hindi}
और जहाँ तुम्हारी मोहब्बत से मुँह फेरा तो इधर उधर दौड़ धूप करने लगा ताकि मुल्क में फ़साद फैलाए और ज़राअत (खेती बाड़ी) और मवेशी का सत्यानास करे और ख़ुदा फसाद को अच्छा नहीं समझता
\end{hindi}}
\flushright{\begin{Arabic}
\quranayah[2][206]
\end{Arabic}}
\flushleft{\begin{hindi}
और जब कहा जाता है कि ख़ुदा से डरो तो उसे ग़ुरुर गुनाह पर उभारता है बस ऐसे कम्बख्त के लिए जहन्नुम ही काफ़ी है और बहुत ही बुरा ठिकाना है
\end{hindi}}
\flushright{\begin{Arabic}
\quranayah[2][207]
\end{Arabic}}
\flushleft{\begin{hindi}
और लोगों में से ख़ुदा के बन्दे कुछ ऐसे हैं जो ख़ुदा की (ख़ुशनूदी) हासिल करने की ग़रज़ से अपनी जान तक बेच डालते हैं और ख़ुदा ऐसे बन्दों पर बड़ा ही यफ्क्क़त वाला है
\end{hindi}}
\flushright{\begin{Arabic}
\quranayah[2][208]
\end{Arabic}}
\flushleft{\begin{hindi}
ईमान वालों तुम सबके सब एक बार इस्लाम में (पूरी तरह ) दाख़िल हो जाओ और शैतान के क़दम ब क़दम न चलो वह तुम्हारा यक़ीनी ज़ाहिर ब ज़ाहिर दुश्मन है
\end{hindi}}
\flushright{\begin{Arabic}
\quranayah[2][209]
\end{Arabic}}
\flushleft{\begin{hindi}
फिर जब तुम्हारे पास रौशन दलीले आ चुकी उसके बाद भी डगमगा गए तो अच्छी तरह समझ लो कि ख़ुदा (हर तरह) ग़ालिब और तदबीर वाला है
\end{hindi}}
\flushright{\begin{Arabic}
\quranayah[2][210]
\end{Arabic}}
\flushleft{\begin{hindi}
क्या वह लोग इसी के मुन्तज़िर हैं कि सफेद बादल के साय बानो की आड़ में अज़ाबे ख़ुदा और अज़ाब के फ़रिश्ते उन पर ही आ जाए और सब झगड़े चुक ही जाते हालॉकि आख़िर कुल उमुर ख़ुदा ही की तरफ रुजू किए जाएँगे
\end{hindi}}
\flushright{\begin{Arabic}
\quranayah[2][211]
\end{Arabic}}
\flushleft{\begin{hindi}
(ऐ रसूल) बनी इसराइल से पूछो कि हम ने उन को कैसी कैसी रौशन निशानियाँ दी और जब किसी शख्स के पास ख़ुदा की नेअमत (किताब) आ चुकी उस के बाद भी उस को बदल डाले तो बेशक़ ख़ुदा सख्त अज़ाब वाला है
\end{hindi}}
\flushright{\begin{Arabic}
\quranayah[2][212]
\end{Arabic}}
\flushleft{\begin{hindi}
जिन लोगों ने कुफ्र इख्तेयार किया उन के लिये दुनिया की ज़रा सी ज़िन्दगी ख़ूब अच्छी दिखायी गयी है और ईमानदारों से मसखरापन करते हैं हालॉकि क़यामत के दिन परहेज़गारों का दरजा उनसे (कहीं) बढ़ चढ़ के होगा और ख़ुदा जिस को चाहता है बे हिसाब रोज़ी अता फरमाता है
\end{hindi}}
\flushright{\begin{Arabic}
\quranayah[2][213]
\end{Arabic}}
\flushleft{\begin{hindi}
(पहले) सब लोग एक ही दीन रखते थे (फिर आपस में झगड़ने लगे तब) ख़ुदा ने नजात से ख़ुश ख़बरी देने वाले और अज़ाब से डराने वाले पैग़म्बरों को भेजा और इन पैग़म्बरों के साथ बरहक़ किताब भी नाज़िल की ताकि जिन बातों में लोग झगड़ते थे किताबे ख़ुदा (उसका) फ़ैसला कर दे और फिर अफ़सोस तो ये है कि इस हुक्म से इख्तेलाफ किया भी तो उन्हीं लोगों ने जिन को किताब दी गयी थी और वह भी जब उन के पास ख़ुदा के साफ एहकाम आ चुके उसके बाद और वह भी आपस की शरारत से तब ख़ुदा ने अपनी मेहरबानी से (ख़ालिस) ईमानदारों को वह राहे हक़ दिखा दी जिस में उन लोगों ने इख्तेलाफ डाल रखा था और ख़ुदा जिस को चाहे राहे रास्त की हिदायत करता है
\end{hindi}}
\flushright{\begin{Arabic}
\quranayah[2][214]
\end{Arabic}}
\flushleft{\begin{hindi}
क्या तुम ये ख्याल करते हो कि बेहश्त में पहुँच ही जाओगे हालॉकि अभी तक तुम्हे अगले ज़माने वालों की सी हालत नहीं पेश आयी कि उन्हें तरह तरह की तक़लीफों (फाक़ा कशी मोहताजी) और बीमारी ने घेर लिया था और ज़लज़ले में इस क़दर झिंझोडे ग़ए कि आख़िर (आज़िज़ हो के) पैग़म्बर और ईमान वाले जो उन के साथ थे कहने लगे देखिए ख़ुदा की मदद कब (होती) है देखो (घबराओ नहीं) ख़ुदा की मदद यक़ीनन बहुत क़रीब है
\end{hindi}}
\flushright{\begin{Arabic}
\quranayah[2][215]
\end{Arabic}}
\flushleft{\begin{hindi}
(ऐ रसूल) तुमसे लोग पूछते हैं कि हम ख़ुदा की राह में क्या खर्च करें (तो तुम उन्हें) जवाब दो कि तुम अपनी नेक कमाई से जो कुछ खर्च करो तो (वह तुम्हारे माँ बाप और क़राबतदारों और यतीमों और मोहताजो और परदेसियों का हक़ है और तुम कोई नेक सा काम करो ख़ुदा उसको ज़रुर जानता है
\end{hindi}}
\flushright{\begin{Arabic}
\quranayah[2][216]
\end{Arabic}}
\flushleft{\begin{hindi}
(मुसलमानों) तुम पर जिहाद फर्ज क़िया गया अगरचे तुम पर शाक़ ज़रुर है और अजब नहीं कि तुम किसी चीज़ (जिहाद) को नापसन्द करो हालॉकि वह तुम्हारे हक़ में बेहतर हो और अजब नहीं कि तुम किसी चीज़ को पसन्द करो हालॉकि वह तुम्हारे हक़ में बुरी हो और ख़ुदा (तो) जानता ही है मगर तुम नही जानते हो
\end{hindi}}
\flushright{\begin{Arabic}
\quranayah[2][217]
\end{Arabic}}
\flushleft{\begin{hindi}
(ऐ रसूल) तुमसे लोग हुरमत वाले महीनों की निस्बत पूछते हैं कि (आया) जिहाद उनमें जायज़ है तो तुम उन्हें जवाब दो कि इन महीनों में जेहाद बड़ा गुनाह है और ये भी याद रहे कि ख़ुदा की राह से रोकना और ख़ुदा से इन्कार और मस्जिदुल हराम (काबा) से रोकना और जो उस के अहल है उनका मस्जिद से निकाल बाहर करना (ये सब) ख़ुदा के नज़दीक इस से भी बढ़कर गुनाह है और फ़ितना परदाज़ी कुश्ती ख़़ून से भी बढ़ कर है और ये कुफ्फ़ार हमेशा तुम से लड़ते ही चले जाएँगें यहाँ तक कि अगर उन का बस चले तो तुम को तुम्हारे दीन से फिरा दे और तुम में जो शख्स अपने दीन से फिरा और कुफ़्र की हालत में मर गया तो ऐसों ही का किया कराया सब कुछ दुनिया और आखेरत (दोनों) में अकारत है और यही लोग जहन्नुमी हैं (और) वह उसी में हमेशा रहेंगें
\end{hindi}}
\flushright{\begin{Arabic}
\quranayah[2][218]
\end{Arabic}}
\flushleft{\begin{hindi}
बेशक जिन लोगों ने ईमान क़ुबूल किया और ख़ुदा की राह में हिजरत की और जिहाद किया यही लोग रहमते ख़ुदा के उम्मीदवार हैं और ख़ुदा बड़ा बख्शने वाला मेहरबान है
\end{hindi}}
\flushright{\begin{Arabic}
\quranayah[2][219]
\end{Arabic}}
\flushleft{\begin{hindi}
(ऐ रसूल) तुमसे लोग शराब और जुए के बारे में पूछते हैं तो तुम उन से कह दो कि इन दोनो में बड़ा गुनाह है और कुछ फायदे भी हैं और उन के फायदे से उन का गुनाह बढ़ के है और तुम से लोग पूछते हैं कि ख़ुदा की राह में क्या ख़र्च करे तुम उनसे कह दो कि जो तुम्हारे ज़रुरत से बचे यूँ ख़ुदा अपने एहकाम तुम से साफ़ साफ़ बयान करता है
\end{hindi}}
\flushright{\begin{Arabic}
\quranayah[2][220]
\end{Arabic}}
\flushleft{\begin{hindi}
ताकि तुम दुनिया और आख़िरत (के मामलात) में ग़ौर करो और तुम से लोग यतीमों के बारे में पूछते हैं तुम (उन से) कह दो कि उनकी (इसलाह दुरुस्ती) बेहतर है और अगर तुम उन से मिलजुल कर रहो तो (कुछ हर्ज) नहीं आख़िर वह तुम्हारें भाई ही तो हैं और ख़ुदा फ़सादी को ख़ैर ख्वाह से (अलग ख़ूब) जानता है और अगर ख़ुदा चाहता तो तुम को मुसीबत में डाल देता बेशक ख़ुदा ज़बरदस्त हिक़मत वाला है
\end{hindi}}
\flushright{\begin{Arabic}
\quranayah[2][221]
\end{Arabic}}
\flushleft{\begin{hindi}
और (मुसलमानों) तुम मुशरिक औरतों से जब तक ईमान न लाएँ निकाह न करो क्योंकि मुशरिका औरत तुम्हें अपने हुस्नो जमाल में कैसी ही अच्छी क्यों न मालूम हो मगर फिर भी ईमानदार औरत उस से ज़रुर अच्छी है और मुशरेकीन जब तक ईमान न लाएँ अपनी औरतें उन के निकाह में न दो और मुशरिक तुम्हे कैसा ही अच्छा क्यो न मालूम हो मगर फिर भी ईमानदार औरत उस से ज़रुर अच्छी है और मुशरेकीन जब तक ईमान न लाएँ अपनी औरतें उन के निकाह में न दो और मुशरिक तुम्हें क्या ही अच्छा क्यों न मालूम हो मगर फिर भी बन्दा मोमिन उनसे ज़रुर अच्छा है ये (मुशरिक मर्द या औरत) लोगों को दोज़ख़ की तरफ बुलाते हैं और ख़ुदा अपनी इनायत से बेहिश्त और बख़्शिस की तरफ बुलाता है और अपने एहकाम लोगों से साफ साफ बयान करता है ताकि ये लोग चेते
\end{hindi}}
\flushright{\begin{Arabic}
\quranayah[2][222]
\end{Arabic}}
\flushleft{\begin{hindi}
(ऐ रसूल) तुम से लोग हैज़ के बारे में पूछते हैं तुम उनसे कह दो कि ये गन्दगी और घिन की बीमारी है तो (अय्यामे हैज़) में तुम औरतों से अलग रहो और जब तक वह पाक न हो जाएँ उनके पास न जाओ पस जब वह पाक हो जाएँ तो जिधर से तुम्हें ख़ुदा ने हुक्म दिया है उन के पास जाओ बेशक ख़ुदा तौबा करने वालो और सुथरे लोगों को पसन्द करता है तुम्हारी बीवियाँ (गोया) तुम्हारी खेती हैं
\end{hindi}}
\flushright{\begin{Arabic}
\quranayah[2][223]
\end{Arabic}}
\flushleft{\begin{hindi}
तो तुम अपनी खेती में जिस तरह चाहो आओ और अपनी आइन्दा की भलाई के वास्ते (आमाल साके) पेशगी भेजो और ख़ुदा से डरते रहो और ये भी समझ रखो कि एक दिन तुमको उसके सामने जाना है और ऐ रसूल ईमानदारों को नजात की ख़ुश ख़बरी दे दो
\end{hindi}}
\flushright{\begin{Arabic}
\quranayah[2][224]
\end{Arabic}}
\flushleft{\begin{hindi}
और (मुसलमानों) तुम अपनी क़समों (के हीले) से ख़ुदा (के नाम) को लोगों के साथ सुलूक करने और ख़ुदा से डरने और लोगों के दरमियान सुलह करवा देने का मानेअ न ठहराव और ख़ुदा सबकी सुनता और सब को जानता है
\end{hindi}}
\flushright{\begin{Arabic}
\quranayah[2][225]
\end{Arabic}}
\flushleft{\begin{hindi}
तुम्हारी लग़ो (बेकार) क़समों पर जो बेइख्तेयार ज़बान से निकल जाए ख़ुदा तुम से गिरफ्तार नहीं करने का मगर उन कसमों पर ज़रुर तुम्हारी गिरफ्त करेगा जो तुमने क़सदन (जान कर) दिल से खायीं हो और ख़ुदा बख्शने वाला बुर्दबार है
\end{hindi}}
\flushright{\begin{Arabic}
\quranayah[2][226]
\end{Arabic}}
\flushleft{\begin{hindi}
जो लोग अपनी बीवियों के पास जाने से क़सम खायें उन के लिए चार महीने की मोहलत है पस अगर (वह अपनी क़सम से उस मुद्दत में बाज़ आए) और उनकी तरफ तवज्जो करें तो बेशक ख़ुदा बड़ा बख्शने वाला मेहरबान है
\end{hindi}}
\flushright{\begin{Arabic}
\quranayah[2][227]
\end{Arabic}}
\flushleft{\begin{hindi}
और अगर तलाक़ ही की ठान ले तो (भी) बेशक ख़ुदा सबकी सुनता और सब कुछ जानता है
\end{hindi}}
\flushright{\begin{Arabic}
\quranayah[2][228]
\end{Arabic}}
\flushleft{\begin{hindi}
और जिन औरतों को तलाक़ दी गयी है वह अपने आपको तलाक़ के बाद तीन हैज़ के ख़त्म हो जाने तक निकाह सानी से रोके और अगर वह औरतें ख़ुदा और रोजे आख़िरत पर ईमान लायीं हैं तो उनके लिए जाएज़ नहीं है कि जो कुछ भी ख़ुदा ने उनके रहम (पेट) में पैदा किया है उसको छिपाएँ और अगर उन के शौहर मेल जोल करना चाहें तो वह (मुद्दत मज़कूरा) में उन के वापस बुला लेने के ज्यादा हक़दार हैं और शरीयत मुवाफिक़ औरतों का (मर्दों पर) वही सब कुछ हक़ है जो मर्दों का औरतों पर है हाँ अलबत्ता मर्दों को (फ़जीलत में) औरतों पर फौक़ियत ज़रुर है और ख़ुदा ज़बरदस्त हिक़मत वाला है
\end{hindi}}
\flushright{\begin{Arabic}
\quranayah[2][229]
\end{Arabic}}
\flushleft{\begin{hindi}
तलाक़ रजअई जिसके बाद रुजू हो सकती है दो ही मरतबा है उसके बाद या तो शरीयत के मवाफिक़ रोक ही लेना चाहिए या हुस्न सुलूक से (तीसरी दफ़ा) बिल्कुल रूख़सत और तुम को ये जायज़ नहीं कि जो कुछ तुम उन्हें दे चुके हो उस में से फिर कुछ वापस लो मगर जब दोनों को इसका ख़ौफ़ हो कि ख़ुदा ने जो हदें मुक़र्रर कर दी हैं उन को दोनो मिया बीवी क़ायम न रख सकेंगे फिर अगर तुम्हे (ऐ मुसलमानो) ये ख़ौफ़ हो कि यह दोनो ख़ुदा की मुकर्रर की हुई हदो पर क़ायम न रहेंगे तो अगर औरत मर्द को कुछ देकर अपना पीछा छुड़ाए (खुला कराए) तो इसमें उन दोनों पर कुछ गुनाह नहीं है ये ख़ुदा की मुक़र्रर की हुई हदें हैं बस उन से आगे न बढ़ो और जो ख़ुदा की मुक़र्रर की हुईहदों से आगे बढ़ते हैं वह ही लोग तो ज़ालिम हैं
\end{hindi}}
\flushright{\begin{Arabic}
\quranayah[2][230]
\end{Arabic}}
\flushleft{\begin{hindi}
फिर अगर तीसरी बार भी औरत को तलाक़ (बाइन) दे तो उसके बाद जब तक दूसरे मर्द से निकाह न कर ले उस के लिए हलाल नही हाँ अगर दूसरा शौहर निकाह के बाद उसको तलाक़ दे दे तब अलबत्ता उन मिया बीबी पर बाहम मेल कर लेने में कुछ गुनाह नहीं है अगर उन दोनों को यह ग़ुमान हो कि ख़ुदा हदों को क़ायम रख सकेंगें और ये ख़ुदा की (मुक़र्रर की हुई) हदें हैं जो समझदार लोगों के वास्ते साफ साफ बयान करता है
\end{hindi}}
\flushright{\begin{Arabic}
\quranayah[2][231]
\end{Arabic}}
\flushleft{\begin{hindi}
और जब तुम अपनी बीवियों को तलाक़ दो और उनकी मुद्दत पूरी होने को आए तो अच्छे उनवान से उन को रोक लो या हुस्ने सुलूक से बिल्कुल रुख़सत ही कर दो और उन्हें तकलीफ पहुँचाने के लिए न रोको ताकि (फिर उन पर) ज्यादती करने लगो और जो ऐसा करेगा तो यक़ीनन अपने ही पर जुल्म करेगा और ख़ुदा के एहकाम को कुछ हँसी ठट्टा न समझो और ख़ुदा ने जो तुम्हें नेअमतें दी हैं उन्हें याद करो और जो किताब और अक्ल की बातें तुम पर नाज़िल की उनसे तुम्हारी नसीहत करता है और ख़ुदा से डरते रहो और समझ रखो कि ख़ुदा हर चीज़ को ज़रुर जानता है
\end{hindi}}
\flushright{\begin{Arabic}
\quranayah[2][232]
\end{Arabic}}
\flushleft{\begin{hindi}
और जब तुम औरतों को तलाक़ दो और वह अपनी मुद्दत (इद्दत) पूरी कर लें तो उन्हें अपने शौहरों के साथ निकाह करने से न रोकों जब आपस में दोनों मिया बीवी शरीयत के मुवाफिक़ अच्छी तरह मिल जुल जाएँ ये उसी शख्स को नसीहत की जाती है जो तुम में से ख़ुदा और रोजे आखेरत पर ईमान ला चुका हो यही तुम्हारे हक़ में बड़ी पाकीज़ा और सफ़ाई की बात है और उसकी ख़ूबी ख़ुदा खूब जानता है और तुम (वैसा) नहीं जानते हो
\end{hindi}}
\flushright{\begin{Arabic}
\quranayah[2][233]
\end{Arabic}}
\flushleft{\begin{hindi}
और (तलाक़ देने के बाद) जो शख्स अपनी औलाद को पूरी मुद्दत तक दूध पिलवाना चाहे तो उसकी ख़ातिर से माएँ अपनी औलाद को पूरे दो बरस दूध पिलाएँ और जिसका वह लड़का है (बाप) उस पर माओं का खाना कपड़ा दस्तूर के मुताबिक़ लाज़िम है किसी शख्स को ज़हमत नहीं दी जाती मगर उसकी गुन्जाइश भर न माँ का उस के बच्चे की वजह से नुक़सान गवारा किया जाए और न जिस का लड़का है (बाप) उसका (बल्कि दस्तूर के मुताबिक़ दिया जाए) और अगर बाप न हो तो दूध पिलाने का हक़ उसी तरह वारिस पर लाज़िम है फिर अगर दो बरस के क़ब्ल माँ बाप दोनों अपनी मरज़ी और मशवरे से दूध बढ़ाई करना चाहें तो उन दोनों पर कोई गुनाह नहीं और अगर तुम अपनी औलाद को (किसी अन्ना से) दूध पिलवाना चाहो तो उस में भी तुम पर कुछ गुनाह नहीं है बशर्ते कि जो तुमने दस्तूर के मुताबिक़ मुक़र्रर किया है उन के हवाले कर दो और ख़ुदा से डरते रहो और जान रखो कि जो कुछ तुम करते हो ख़ुदा ज़रुर देखता है
\end{hindi}}
\flushright{\begin{Arabic}
\quranayah[2][234]
\end{Arabic}}
\flushleft{\begin{hindi}
और तुममें से जो लोग बीवियाँ छोड़ के मर जाएँ तो ये औरतें चार महीने दस रोज़ (इद्दा भर) अपने को रोके (और दूसरा निकाह न करें) फिर जब (इद्दे की मुद्दत) पूरी कर ले तो शरीयत के मुताबिक़ जो कुछ अपने हक़ में करें इस बारे में तुम पर कोई इल्ज़ाम नहीं है और जो कुछ तुम करते हो ख़ुदा उस से ख़बरदार है
\end{hindi}}
\flushright{\begin{Arabic}
\quranayah[2][235]
\end{Arabic}}
\flushleft{\begin{hindi}
और अगर तुम (उस ख़ौफ से कि शायद कोई दूसरा निकाह कर ले) उन औरतों से इशारतन निकाह की (कैद इद्दा) ख़ास्तगारी (उम्मीदवारी) करो या अपने दिलो में छिपाए रखो तो उसमें भी कुछ तुम पर इल्ज़ाम नहीं हैं (क्योंकि) ख़ुदा को मालूम है कि (तुम से सब्र न हो सकेगा और) उन औरतों से निकाह करने का ख्याल आएगा लेकिन चोरी छिपे से निकाह का वायदा न करना मगर ये कि उन से अच्छी बात कह गुज़रों (तो मज़ाएक़ा नहीं) और जब तक मुक़र्रर मियाद गुज़र न जाए निकाह का क़सद (इरादा) भी न करना और समझ रखो कि जो कुछ तुम्हारी दिल में है ख़ुदा उस को ज़रुर जानता है तो उस से डरते रहो और (ये भी) जान लो कि ख़ुदा बड़ा बख्शने वाला बुर्दबार है
\end{hindi}}
\flushright{\begin{Arabic}
\quranayah[2][236]
\end{Arabic}}
\flushleft{\begin{hindi}
और अगर तुम ने अपनी बीवियों को हाथ तक न लगाया हो और न महर मुअय्युन किया हो और उसके क़ब्ल ही तुम उनको तलाक़ दे दो (तो इस में भी) तुम पर कुछ इल्ज़ाम नहीं है हाँ उन औरतों के साथ (दस्तूर के मुताबिक़) मालदार पर अपनी हैसियत के मुआफिक़ और ग़रीब पर अपनी हैसियत के मुवाफिक़ (कपड़े रुपए वग़ैरह से) कुछ सुलूक करना लाज़िम है नेकी करने वालों पर ये भी एक हक़ है
\end{hindi}}
\flushright{\begin{Arabic}
\quranayah[2][237]
\end{Arabic}}
\flushleft{\begin{hindi}
और अगर तुम उन औरतों का मेहर तो मुअय्यन कर चुके हो मगर हाथ लगाने के क़ब्ल ही तलाक़ दे दो तो उन औरतों को मेहर मुअय्यन का आधा दे दो मगर ये कि ये औरतें ख़ुद माफ कर दें या उन का वली जिसके हाथ में उनके निकाह का एख्तेयार हो माफ़ कर दे (तब कुछ नही) और अगर तुम ही सारा मेहर बख्श दो तो परहेज़गारी से बहुत ही क़रीब है और आपस की बुर्ज़ुगी तो मत भूलो और जो कुछ तुम करते हो ख़ुदा ज़रुर देख रहा है
\end{hindi}}
\flushright{\begin{Arabic}
\quranayah[2][238]
\end{Arabic}}
\flushleft{\begin{hindi}
और (मुसलमानों) तुम तमाम नमाज़ों की और ख़ुसूसन बीच वाली नमाज़ सुबह या ज़ोहर या अस्र की पाबन्दी करो और ख़ास ख़ुदा ही वास्ते नमाज़ में क़ुनूत पढ़ने वाले हो कर खड़े हो फिर अगर तुम ख़ौफ की हालत में हो
\end{hindi}}
\flushright{\begin{Arabic}
\quranayah[2][239]
\end{Arabic}}
\flushleft{\begin{hindi}
और पूरी नमाज़ न पढ़ सको तो सवार या पैदल जिस तरह बन पड़े पढ़ लो फिर जब तुम्हें इत्मेनान हो तो जिस तरह ख़ुदा ने तुम्हें (अपने रसूल की मआरफत इन बातों को सिखाया है जो तुम नहीं जानते थे
\end{hindi}}
\flushright{\begin{Arabic}
\quranayah[2][240]
\end{Arabic}}
\flushleft{\begin{hindi}
उसी तरह ख़ुदा को याद करो और तुम में से जो लोग अपनी बीवियों को छोड़ कर मर जाएँ उन पर अपनी बीबियों के हक़ में साल भर तक के नान व नुफ्के (रोटी कपड़ा) और (घर से) न निकलने की वसीयत करनी (लाज़िम) है पस अगर औरतें ख़ुद निकल खड़ी हो तो जायज़ बातों (निकाह वगैरह) से कुछ अपने हक़ में करे उसका तुम पर कुछ इल्ज़ाम नही है और ख़ुदा हर यै पर ग़ालिब और हिक़मत वाला है
\end{hindi}}
\flushright{\begin{Arabic}
\quranayah[2][241]
\end{Arabic}}
\flushleft{\begin{hindi}
और जिन औरतों को ताअय्युन मेहर और हाथ लगाए बगैर तलाक़ दे दी जाए उनके साथ जोड़े रुपए वगैरह से सुलूक करना लाज़िम है
\end{hindi}}
\flushright{\begin{Arabic}
\quranayah[2][242]
\end{Arabic}}
\flushleft{\begin{hindi}
(ये भी) परहेज़गारों पर एक हक़ है उसी तरह ख़ुदा तुम लोगों की हिदायत के वास्ते अपने एहक़ाम साफ़ साफ़ बयान फरमाता है
\end{hindi}}
\flushright{\begin{Arabic}
\quranayah[2][243]
\end{Arabic}}
\flushleft{\begin{hindi}
ताकि तुम समझो (ऐ रसूल) क्या तुम ने उन लोगों के हाल पर नज़र नही की जो मौत के डर के मारे अपने घरों से निकल भागे और वह हज़ारो आदमी थे तो ख़ुदा ने उन से फरमाया कि सब के सब मर जाओ (और वह मर गए) फिर ख़ुदा न उन्हें जिन्दा किया बेशक ख़ुदा लोगों पर बड़ा मेहरबान है मगर अक्सर लोग उसका शुक्र नहीं करते
\end{hindi}}
\flushright{\begin{Arabic}
\quranayah[2][244]
\end{Arabic}}
\flushleft{\begin{hindi}
और मुसलमानों ख़ुदा की राह मे जिहाद करो और जान रखो कि ख़ुदा ज़रुर सब कुछ सुनता (और) जानता है
\end{hindi}}
\flushright{\begin{Arabic}
\quranayah[2][245]
\end{Arabic}}
\flushleft{\begin{hindi}
है कोई जो ख़ुदा को क़र्ज़ ए हुस्ना दे ताकि ख़ुदा उसके माल को इस के लिए कई गुना बढ़ा दे और ख़ुदा ही तंगदस्त करता है और वही कशायश देता है और उसकी तरफ सब के सब लौटा दिये जाओगे
\end{hindi}}
\flushright{\begin{Arabic}
\quranayah[2][246]
\end{Arabic}}
\flushleft{\begin{hindi}
(ऐ रसूल) क्या तुमने मूसा के बाद बनी इसराइल के सरदारों की हालत पर नज़र नही की जब उन्होंने अपने नबी (शमूयेल) से कहा कि हमारे वास्ते एक बादशाह मुक़र्रर कीजिए ताकि हम राहे ख़ुदा में जिहाद करें (पैग़म्बर ने) फ़रमाया कहीं ऐसा तो न हो कि जब तुम पर जिहाद वाजिब किया जाए तो तुम न लड़ो कहने लगे जब हम अपने घरों और अपने बाल बच्चों से निकाले जा चुके तो फिर हमे कौन सा उज़्र बाक़ी है कि हम ख़ुदा की राह में जिहाद न करें फिर जब उन पर जिहाद वाजिब किया गया तो उनमें से चन्द आदमियों के सिवा सब के सब ने लड़ने से मुँह फेरा और ख़ुदा तो ज़ालिमों को खूब जानता है
\end{hindi}}
\flushright{\begin{Arabic}
\quranayah[2][247]
\end{Arabic}}
\flushleft{\begin{hindi}
और उनके नबी ने उनसे कहा कि बेशक ख़ुदा ने तुम्हारी दरख्वास्त के (मुताबिक़ तालूत को तुम्हारा बादशाह मुक़र्रर किया (तब) कहने लगे उस की हुकूमत हम पर क्यों कर हो सकती है हालाकि सल्तनत के हक़दार उससे ज्यादा तो हम हैं क्योंकि उसे तो माल के एतबार से भी फ़ारगुल बाली (ख़ुशहाली) तक नसीब नहीं (नबी ने) कहा ख़ुदा ने उसे तुम पर फज़ीलत दी है और माल में न सही मगर इल्म और जिस्म का फैलाव तो उस का ख़ुदा ने ज्यादा फरमाया हे और ख़ुदा अपना मुल्क जिसे चाहें दे और ख़ुदा बड़ी गुन्जाइश वाला और वाक़िफ़कार है
\end{hindi}}
\flushright{\begin{Arabic}
\quranayah[2][248]
\end{Arabic}}
\flushleft{\begin{hindi}
और उन के नबी ने उनसे ये भी कहा इस के (मुनाजानिब अल्लाह) बादशाह होने की ये पहचान है कि तुम्हारे पास वह सन्दूक़ आ जाएगा जिसमें तुम्हारे परवरदिगार की तरफ से तसकीन दे चीजें और उन तब्बुरक़ात से बचा खुचा होगा जो मूसा और हारुन की औलाद यादगार छोड़ गयी है और उस सन्दूक को फरिश्ते उठाए होगें अगर तुम ईमान रखते हो तो बेशक उसमें तुम्हारे वास्ते पूरी निशानी है
\end{hindi}}
\flushright{\begin{Arabic}
\quranayah[2][249]
\end{Arabic}}
\flushleft{\begin{hindi}
फिर जब तालूत लशकर समैत (शहर ऐलिया से) रवाना हुआ तो अपने साथियों से कहा देखो आगे एक नहर मिलेगी इस से यक़ीनन ख़ुदा तुम्हारे सब्र की आज़माइश करेगा पस जो शख्स उस का पानी पीयेगा मुझे (कुछ वास्ता) नही रखता और जो उस को नही चखेगा वह बेशक मुझ से होगा मगर हाँ जो अपने हाथ से एक (आधा चुल्लू भर के पी) ले तो कुछ हर्ज नही पस उन लोगों ने न माना और चन्द आदमियों के सिवा सब ने उस का पानी पिया ख़ैर जब तालूत और जो मोमिनीन उन के साथ थे नहर से पास हो गए तो (ख़ास मोमिनों के सिवा) सब के सब कहने लगे कि हम में तो आज भी जालूत और उसकी फौज से लड़ने की सकत नहीं मगर वह लोग जिनको यक़ीन है कि एक दिन ख़ुदा को मुँह दिखाना है बेधड़क बोल उठे कि ऐसा बहुत हुआ कि ख़ुदा के हुक्म से छोटी जमाअत बड़ी जमाअत पर ग़ालिब आ गयी है और ख़ुदा सब्र करने वालों का साथी है
\end{hindi}}
\flushright{\begin{Arabic}
\quranayah[2][250]
\end{Arabic}}
\flushleft{\begin{hindi}
(ग़रज़) जब ये लोग जालूत और उसकी फौज के मुक़ाबले को निकले तो दुआ की ऐ मेरे परवरदिगार हमें कामिल सब्र अता फरमा और मैदाने जंग में हमारे क़दम जमाए रख और हमें काफिरों पर फतेह इनायत कर
\end{hindi}}
\flushright{\begin{Arabic}
\quranayah[2][251]
\end{Arabic}}
\flushleft{\begin{hindi}
फिर तो उन लोगों ने ख़ुदा के हुक्म से दुशमनों को शिकस्त दी और दाऊद ने जालूत को क़त्ल किया और ख़ुदा ने उनको सल्तनत व तदबीर तम्द्दुन अता की और इल्म व हुनर जो चाहा उन्हें गोया घोल के पिला दिया और अगर ख़ुदा बाज़ लोगों के ज़रिए से बाज़ का दफाए (शर) न करता तो तमाम रुए ज़मीन पर फ़साद फैल जाता मगर ख़ुदा तो सारे जहाँन के लोगों पर फज़ल व रहम करता है
\end{hindi}}
\flushright{\begin{Arabic}
\quranayah[2][252]
\end{Arabic}}
\flushleft{\begin{hindi}
ऐ रसूल ये ख़ुदा की सच्ची आयतें हैं जो हम तुम को ठीक ठीक पढ़के सुनाते हैं और बेशक तुम ज़रुर रसूलों में से हो
\end{hindi}}
\flushright{\begin{Arabic}
\quranayah[2][253]
\end{Arabic}}
\flushleft{\begin{hindi}
यह सब रसूल (जो हमने भेजे) उनमें से बाज़ को बाज़ पर फज़ीलत दी उनमें से बाज़ तो ऐसे हैं जिनसे ख़ुद ख़ुदा ने बात की उनमें से बाज़ के (और तरह पर) दर्जे बुलन्द किये और मरियम के बेटे ईसा को (कैसे कैसे रौशन मौजिज़े अता किये) और रूहुलकुदस (जिबरईल) के ज़रिये से उनकी मदद की और अगर ख़ुदा चाहता तो लोग इन (पैग़म्बरों) के बाद हुये वह अपने पास रौशन मौजिज़े आ चुकने पर आपस में न लड़ मरते मगर उनमें फूट पड़ गई पस उनमें से बाज़ तो ईमान लाये और बाज़ काफ़िर हो गये और अगर ख़ुदा चाहता तो यह लोग आपस में लड़ते मगर ख़ुदा वही करता है जो चाहता है
\end{hindi}}
\flushright{\begin{Arabic}
\quranayah[2][254]
\end{Arabic}}
\flushleft{\begin{hindi}
ऐ ईमानदारों जो कुछ हमने तुमको दिया है उस दिन के आने से पहले (ख़ुदा की राह में) ख़र्च करो जिसमें न तो ख़रीदो फरोख्त होगी और न यारी (और न आशनाई) और न सिफ़ारिश (ही काम आयेगी) और कुफ़्र करने वाले ही तो जुल्म ढाते हैं
\end{hindi}}
\flushright{\begin{Arabic}
\quranayah[2][255]
\end{Arabic}}
\flushleft{\begin{hindi}
ख़ुदा ही वो ज़ाते पाक है कि उसके सिवा कोई माबूद नहीं (वह) ज़िन्दा है (और) सारे जहान का संभालने वाला है उसको न ऊँघ आती है न नींद जो कुछ आसमानो में है और जो कुछ ज़मीन में है (गरज़ सब कुछ) उसी का है कौन ऐसा है जो बग़ैर उसकी इजाज़त के उसके पास किसी की सिफ़ारिश करे जो कुछ उनके सामने मौजूद है (वह) और जो कुछ उनके पीछे (हो चुका) है (खुदा सबको) जानता है और लोग उसके इल्म में से किसी चीज़ पर भी अहाता नहीं कर सकते मगर वह जिसे जितना चाहे (सिखा दे) उसकी कुर्सी सब आसमानॊं और ज़मीनों को घेरे हुये है और उन दोनों (आसमान व ज़मीन) की निगेहदाश्त उसपर कुछ भी मुश्किल नहीं और वह आलीशान बुजुर्ग़ मरतबा है
\end{hindi}}
\flushright{\begin{Arabic}
\quranayah[2][256]
\end{Arabic}}
\flushleft{\begin{hindi}
दीन में किसी तरह की जबरदस्ती नहीं क्योंकि हिदायत गुमराही से (अलग) ज़ाहिर हो चुकी तो जिस शख्स ने झूठे खुदाओं बुतों से इंकार किया और खुदा ही पर ईमान लाया तो उसने वो मज़बूत रस्सी पकड़ी है जो टूट ही नहीं सकती और ख़ुदा सब कुछ सुनता और जानता है
\end{hindi}}
\flushright{\begin{Arabic}
\quranayah[2][257]
\end{Arabic}}
\flushleft{\begin{hindi}
ख़ुदा उन लोगों का सरपरस्त है जो ईमान ला चुके कि उन्हें (गुमराही की) तारीक़ियों से निकाल कर (हिदायत की) रौशनी में लाता है और जिन लोगों ने कुफ़्र इख्तेयार किया उनके सरपरस्त शैतान हैं कि उनको (ईमान की) रौशनी से निकाल कर (कुफ़्र की) तारीकियों में डाल देते हैं यही लोग तो जहन्नुमी हैं (और) यही उसमें हमेशा रहेंगे
\end{hindi}}
\flushright{\begin{Arabic}
\quranayah[2][258]
\end{Arabic}}
\flushleft{\begin{hindi}
(ऐ रसूल) क्या तुम ने उस शख्स (के हाल) पर नज़र नहीं की जो सिर्फ़ इस बिरते पर कि ख़ुदा ने उसे सल्तनत दी थी इब्राहीम से उनके परवरदिगार के बारे में उलझ पड़ा कि जब इब्राहीम ने (उससे) कहा कि मेरा परवरदिगार तो वह है जो (लोगों को) जिलाता और मारता है तो वो भी (शेख़ी में) आकर कहने लगा मैं भी जिलाता और मारता हूं (तुम्हारे ख़ुदा ही में कौन सा कमाल है) इब्राहीम ने कहा (अच्छा) खुदा तो आफ़ताब को पूरब से निकालता है भला तुम उसको पश्चिम से निकालो इस पर वह काफ़िर हक्का बक्का हो कर रह गया (मगर ईमान न लाया) और ख़ुदा ज़ालिमों को मंज़िले मक़सूद तक नहीं पहुंचाया करता
\end{hindi}}
\flushright{\begin{Arabic}
\quranayah[2][259]
\end{Arabic}}
\flushleft{\begin{hindi}
(ऐ रसूल तुमने) मसलन उस (बन्दे के हाल पर भी नज़र की जो एक गॉव पर से होकर गुज़रा और वो ऐसा उजड़ा था कि अपनी छतों पर से ढह के गिर पड़ा था ये देखकर वह बन्दा (कहने लगा) अल्लाह अब इस गॉव को ऐसी वीरानी के बाद क्योंकर आबाद करेगा इस पर ख़ुदा ने उसको (मार डाला) सौ बरस तक मुर्दा रखा फिर उसको जिला उठाया (तब) पूछा तुम कितनी देर पड़े रहे अर्ज क़ी एक दिन पड़ा रहा एक दिन से भी कम फ़रमाया नहीं तुम (इसी हालत में) सौ बरस पड़े रहे अब ज़रा अपने खाने पीने (की चीज़ों) को देखो कि बुसा तक नहीं और ज़रा अपने गधे (सवारी) को तो देखो कि उसकी हड्डियाँ ढेर पड़ी हैं और सब इस वास्ते किया है ताकि लोगों के लिये तुम्हें क़ुदरत का नमूना बनाये और अच्छा अब (इस गधे की) हड्डियों की तरफ़ नज़र करो कि हम क्योंकर जोड़ जाड़ कर ढाँचा बनाते हैं फिर उनपर गोश्त चढ़ाते हैं पस जब ये उनपर ज़ाहिर हुआ तो बेसाख्ता बोल उठे कि (अब) मैं ये यक़ीने कामिल जानता हूं कि ख़ुदा हर चीज़ पर क़ादिर है
\end{hindi}}
\flushright{\begin{Arabic}
\quranayah[2][260]
\end{Arabic}}
\flushleft{\begin{hindi}
और (ऐ रसूल) वह वाकेया भी याद करो जब इबराहीम ने (खुदा से) दरख्वास्त की कि ऐ मेरे परवरदिगार तू मुझे भी तो दिखा दे कि तू मुर्दों को क्योंकर ज़िन्दा करता है ख़ुदा ने फ़रमाया क्या तुम्हें (इसका) यक़ीन नहीं इबराहीम ने अर्ज़ की (क्यों नहीं) यक़ीन तो है मगर ऑंख से देखना इसलिए चाहता हूं कि मेरे दिल को पूरा इत्मिनान हो जाए फ़रमाया (अगर ये चाहते हो) तो चार परिन्दे लो और उनको अपने पास मॅगवा लो और टुकड़े टुकड़े कर डालो फिर हर पहाड़ पर उनका एक एक टुकड़ा रख दो उसके बाद उनको बुलाओ (फिर देखो तो क्यों कर वह सब के सब तुम्हारे पास दौड़े हुए आते हैं और समझ रखो कि ख़ुदा बेशक ग़ालिब और हिकमत वाला है
\end{hindi}}
\flushright{\begin{Arabic}
\quranayah[2][261]
\end{Arabic}}
\flushleft{\begin{hindi}
जो लोग अपने माल खुदा की राह में खर्च करते हैं उनके (खर्च) की मिसाल उस दाने की सी मिसाल है जिसकी सात बालियॉ निकलें (और) हर बाली में सौ (सौ) दाने हों और ख़ुदा जिसके लिये चाहता है दूना कर देता है और खुदा बड़ी गुन्जाइश वाला (हर चीज़ से) वाक़िफ़ है
\end{hindi}}
\flushright{\begin{Arabic}
\quranayah[2][262]
\end{Arabic}}
\flushleft{\begin{hindi}
जो लोग अपने माल ख़ुदा की राह में ख़र्च करते हैं और फिर ख़र्च करने के बाद किसी तरह का एहसान नहीं जताते हैं और न जिनपर एहसान किया है उनको सताते हैं उनका अज्र (व सवाब) उनके परवरदिगार के पास है और न आख़ेरत में उनपर कोई ख़ौफ़ होगा और न वह ग़मगीन होंगे
\end{hindi}}
\flushright{\begin{Arabic}
\quranayah[2][263]
\end{Arabic}}
\flushleft{\begin{hindi}
(सायल को) नरमी से जवाब दे देना और (उसके इसरार पर न झिड़कना बल्कि) उससे दरगुज़र करना उस खैरात से कहीं बेहतर है जिसके बाद (सायल को) ईज़ा पहुंचे और ख़ुदा हर शै से बेपरवा (और) बुर्दबार है
\end{hindi}}
\flushright{\begin{Arabic}
\quranayah[2][264]
\end{Arabic}}
\flushleft{\begin{hindi}
ऐ ईमानदारों आपनी खैरात को एहसान जताने और (सायल को) ईज़ा (तकलीफ) देने की वजह से उस शख्स की तरह अकारत मत करो जो अपना माल महज़ लोगों को दिखाने के वास्ते ख़र्च करता है और ख़ुदा और रोजे आखेरत पर ईमान नहीं रखता तो उसकी खैरात की मिसाल उस चिकनी चट्टान की सी है जिसपर कुछ ख़ाक (पड़ी हुई) हो फिर उसपर ज़ोर शोर का (बड़े बड़े क़तरों से) मेंह बरसे और उसको (मिट्टी को बहाके) चिकना चुपड़ा छोड़ जाए (इसी तरह) रियाकार अपनी उस ख़ैरात या उसके सवाब में से जो उन्होंने की है किसी चीज़ पर क़ब्ज़ा न पाएंगे (न दुनिया में न आख़ेरत में) और ख़ुदा काफ़िरों को हिदायत करके मंज़िले मक़सूद तक नहीं पहुँचाया करता
\end{hindi}}
\flushright{\begin{Arabic}
\quranayah[2][265]
\end{Arabic}}
\flushleft{\begin{hindi}
और जो लोग ख़ुदा की ख़ुशनूदी के लिए और अपने दिली एतक़ाद से अपने माल ख़र्च करते हैं उनकी मिसाल उस (हरे भरे) बाग़ की सी है जो किसी टीले या टीकरे पर लगा हो और उस पर ज़ोर शोर से पानी बरसा तो अपने दुगने फल लाया और अगर उस पर बड़े धड़ल्ले का पानी न भी बरसे तो उसके लिये हल्की फुआर (ही काफ़ी) है और जो कुछ तुम करते हो ख़ुदा उसकी देखभाल करता रहता है
\end{hindi}}
\flushright{\begin{Arabic}
\quranayah[2][266]
\end{Arabic}}
\flushleft{\begin{hindi}
भला तुम में कोई भी इसको पसन्द करेगा कि उसके लिए खजूरों और अंगूरों का एक बाग़ हो उसके नीचे नहरें जारी हों और उसके लिए उसमें तरह तरह के मेवे हों और (अब) उसको बुढ़ापे ने घेर लिया है और उसके (छोटे छोटे) नातवॉ कमज़ोर बच्चे हैं कि एकबारगी उस बाग़ पर ऐसा बगोला आ पड़ा जिसमें आग (भरी) थी कि वह बाग़ जल भुन कर रह गया ख़ुदा अपने एहकाम को तुम लोगों से साफ़ साफ़ बयान करता है ताकि तुम ग़ौर करो
\end{hindi}}
\flushright{\begin{Arabic}
\quranayah[2][267]
\end{Arabic}}
\flushleft{\begin{hindi}
ऐ ईमान वालों अपनी पाक कमाई और उन चीज़ों में से जो हमने तुम्हारे लिए ज़मीन से पैदा की हैं (ख़ुदा की राह में) ख़र्च करो और बुरे माल को (ख़ुदा की राह में) देने का क़सद भी न करो हालॉकि अगर ऐसा माल कोई तुमको देना चाहे तो तुम अपनी ख़ुशी से उसके लेने वाले नहीं हो मगर ये कि उस (के लेने) में (अमदन) आंख चुराओ और जाने रहो कि ख़ुदा बेशक बेनियाज़ (और) सज़ावारे हम्द है
\end{hindi}}
\flushright{\begin{Arabic}
\quranayah[2][268]
\end{Arabic}}
\flushleft{\begin{hindi}
शैतान तमुको तंगदस्ती से डराता है और बुरी बात (बुख्ल) का तुमको हुक्म करता है और ख़ुदा तुमसे अपनी बख्शिश और फ़ज़ल (व करम) का वायदा करता है और ख़ुदा बड़ी गुन्जाइश वाला और सब बातों का जानने वाला है
\end{hindi}}
\flushright{\begin{Arabic}
\quranayah[2][269]
\end{Arabic}}
\flushleft{\begin{hindi}
वह जिसको चाहता है हिकमत अता फ़रमाता है और जिसको (ख़ुदा की तरफ) से हिकमत अता की गई तो इसमें शक नहीं कि उसे ख़ूबियों से बड़ी दौलत हाथ लगी और अक्लमन्दों के सिवा कोई नसीहत मानता ही नहीं
\end{hindi}}
\flushright{\begin{Arabic}
\quranayah[2][270]
\end{Arabic}}
\flushleft{\begin{hindi}
और तुम जो कुछ भी ख़र्च करो या कोई मन्नत मानो ख़ुदा उसको ज़रूर जानता है और (ये भी याद रहे) कि ज़ालिमों का (जो) ख़ुदा का हक़ मार कर औरों की नज़्र करते हैं (क़यामत में) कोई मददगार न होगा
\end{hindi}}
\flushright{\begin{Arabic}
\quranayah[2][271]
\end{Arabic}}
\flushleft{\begin{hindi}
अगर ख़ैरात को ज़ाहिर में दो तो यह (ज़ाहिर करके देना) भी अच्छा है और अगर उसको छिपाओ और हाजतमन्दों को दो तो ये छिपा कर देना तुम्हारे हक़ में ज्यादा बेहतर है और ऐसे देने को ख़ुदा तुम्हारे गुनाहों का कफ्फ़ारा कर देगा और जो कुछ तुम करते हो ख़ुदा उससे ख़बरदार है
\end{hindi}}
\flushright{\begin{Arabic}
\quranayah[2][272]
\end{Arabic}}
\flushleft{\begin{hindi}
ऐ रसूल उनका मंज़िले मक़सूद तक पहुंचाना तुम्हारा काम नहीं (तुम्हारा काम सिर्फ रास्ता दिखाना है) मगर हॉ ख़ुदा जिसको चाहे मंज़िले मक़सूद तक पहुंचा दे और (लोगों) तुम जो कुछ नेक काम में ख़र्च करोगे तो अपने लिए और तुम ख़ुदा की ख़ुशनूदी के सिवा और काम में ख़र्च करते ही नहीं हो (और जो कुछ तुम नेक काम में ख़र्च करोगे) (क़यामत में) तुमको भरपूर वापस मिलेगा और तुम्हारा हक़ न मारा जाएगा
\end{hindi}}
\flushright{\begin{Arabic}
\quranayah[2][273]
\end{Arabic}}
\flushleft{\begin{hindi}
(यह खैरात) ख़ास उन हाजतमन्दों के लिए है जो ख़ुदा की राह में घिर गये हो (और) रूए ज़मीन पर (जाना चाहें तो) चल नहीं सकते नावाक़िफ़ उनको सवाल न करने की वजह से अमीर समझते हैं (लेकिन) तू (ऐ मुख़ातिब अगर उनको देखे) तो उनकी सूरत से ताड़ जाये (कि ये मोहताज हैं अगरचे) लोगों से चिमट के सवाल नहीं करते और जो कुछ भी तुम नेक काम में ख़र्च करते हो ख़ुदा उसको ज़रूर जानता है
\end{hindi}}
\flushright{\begin{Arabic}
\quranayah[2][274]
\end{Arabic}}
\flushleft{\begin{hindi}
जो लोग रात को या दिन को छिपा कर या दिखा कर (ख़ुदा की राह में) ख़र्च करते हैं तो उनके लिए उनका अज्र व सवाब उनके परवरदिगार के पास है और (क़यामत में) न उन पर किसी क़िस्म का ख़ौफ़ होगा और न वह आज़ुर्दा ख़ातिर होंगे
\end{hindi}}
\flushright{\begin{Arabic}
\quranayah[2][275]
\end{Arabic}}
\flushleft{\begin{hindi}
जो लोग सूद खाते हैं वह (क़यामत में) खड़े न हो सकेंगे मगर उस शख्स की तरह खड़े होंगे जिस को शैतान ने लिपट कर मख़बूतुल हवास (पागल) बना दिया है ये इस वजह से कि वह उसके क़ायल हो गए कि जैसा बिक्री का मामला वैसा ही सूद का मामला हालॉकि बिक्री को तो खुदा ने हलाल और सूद को हराम कर दिया बस जिस शख्स के पास उसके परवरदिगार की तरफ़ से नसीहत (मुमानियत) आये और वह बाज़ आ गया तो इस हुक्म के नाज़िल होने से पहले जो सूद ले चुका वह तो उस का हो चुका और उसका अम्र (मामला) ख़ुदा के हवाले है और जो मनाही के बाद फिर सूद ले (या बिक्री के माले को यकसा बताए जाए) तो ऐसे ही लोग जहन्नुम में रहेंगे
\end{hindi}}
\flushright{\begin{Arabic}
\quranayah[2][276]
\end{Arabic}}
\flushleft{\begin{hindi}
खुदा सूद को मिटाता है और ख़ैरात को बढ़ाता है और जितने नाशुक्रे गुनाहगार हैं खुदा उन्हें दोस्त नहीं रखता
\end{hindi}}
\flushright{\begin{Arabic}
\quranayah[2][277]
\end{Arabic}}
\flushleft{\begin{hindi}
(हॉ) जिन लोगों ने ईमान क़ुबूल किया और अच्छे-अच्छे काम किए और पाबन्दी से नमाज़ पढ़ी और ज़कात दिया किये उनके लिए अलबत्ता उनका अज्र व (सवाब) उनके परवरदिगार के पास है और (क़यामत में) न तो उन पर किसी क़िस्म का ख़ौफ़ होगा और न वह रन्जीदा दिल होंगे
\end{hindi}}
\flushright{\begin{Arabic}
\quranayah[2][278]
\end{Arabic}}
\flushleft{\begin{hindi}
ऐ ईमानदारों ख़ुदा से डरो और जो सूद लोगों के ज़िम्मे बाक़ी रह गया है अगर तुम सच्चे मोमिन हो तो छोड़ दो
\end{hindi}}
\flushright{\begin{Arabic}
\quranayah[2][279]
\end{Arabic}}
\flushleft{\begin{hindi}
और अगर तुमने ऐसा न किया तो ख़ुदा और उसके रसूल से लड़ने के लिये तैयार रहो और अगर तुमने तौबा की है तो तुम्हारे लिए तुम्हारा असल माल है न तुम किसी का ज़बरदस्ती नुकसान करो न तुम पर ज़बरदस्ती की जाएगी
\end{hindi}}
\flushright{\begin{Arabic}
\quranayah[2][280]
\end{Arabic}}
\flushleft{\begin{hindi}
और अगर कोई तंगदस्त तुम्हारा (क़र्ज़दार हो) तो उसे ख़ुशहाली तक मोहल्लत (दो) और सदक़ा करो और अगर तुम समझो तो तुम्हारे हक़ में ज्यादा बेहतर है कि असल भी बख्श दो
\end{hindi}}
\flushright{\begin{Arabic}
\quranayah[2][281]
\end{Arabic}}
\flushleft{\begin{hindi}
और उस दिन से डरो जिस दिन तुम सब के सब ख़ुदा की तरफ़ लौटाये जाओगे फिर जो कुछ जिस शख्स ने किया है उसका पूरा पूरा बदला दिया जाएगा और उनकी ज़रा भी हक़ तलफ़ी न होगी
\end{hindi}}
\flushright{\begin{Arabic}
\quranayah[2][282]
\end{Arabic}}
\flushleft{\begin{hindi}
ऐ ईमानदारों जब एक मियादे मुक़र्ररा तक के लिए आपस में क़र्ज क़ा लेन देन करो तो उसे लिखा पढ़ी कर लिया करो और लिखने वाले को चाहिये कि तुम्हारे दरमियान तुम्हारे क़ौल व क़रार को, इन्साफ़ से ठीक ठीक लिखे और लिखने वाले को लिखने से इन्कार न करना चाहिये (बल्कि) जिस तरह ख़ुदा ने उसे (लिखना पढ़ना) सिखाया है उसी तरह उसको भी वे उज़्र (बहाना) लिख देना चाहिये और जिसके ज़िम्मे क़र्ज़ आयद होता है उसी को चाहिए कि (तमस्सुक) की इबारत बताता जाये और ख़ुदा से डरे जो उसका सच्चा पालने वाला है डरता रहे और (बताने में) और क़र्ज़ देने वाले के हुक़ूक़ में कुछ कमी न करे अगर क़र्ज़ लेने वाला कम अक्ल या माज़ूर या ख़ुद (तमस्सुक) का मतलब लिखवा न सकता हो तो उसका सरपरस्त ठीक ठीक इन्साफ़ से लिखवा दे और अपने लोगों में से जिन लोगों को तुम गवाही लेने के लिये पसन्द करो (कम से कम) दो मर्दों की गवाही कर लिया करो फिर अगर दो मर्द न हो तो (कम से कम) एक मर्द और दो औरतें (क्योंकि) उन दोनों में से अगर एक भूल जाएगी तो एक दूसरी को याद दिला देगी, और जब गवाह हुक्काम के सामने (गवाही के लिए) बुलाया जाएं तो हाज़िर होने से इन्कार न करे और क़र्ज़ का मामला ख्वाह छोटा हो या उसकी मियाद मुअय्युन तक की (दस्तावेज़) लिखवाने में काहिली न करो, ख़ुदा के नज़दीक ये लिखा पढ़ी बहुत ही मुन्सिफ़ाना कारवाई है और गवाही के लिए भी बहुत मज़बूती है और बहुत क़रीन (क़यास) है कि तुम आईन्दा किसी तरह के शक व शुबहा में न पड़ो मगर जब नक़द सौदा हो जो तुम लोग आपस में उलट फेर किया करते हो तो उसकी (दस्तावेज) के न लिखने में तुम पर कुछ इल्ज़ाम नहीं है (हॉ) और जब उसी तरह की ख़रीद (फ़रोख्त) हो तो गवाह कर लिया करो और क़ातिब (दस्तावेज़) और गवाह को ज़रर न पहुँचाया जाए और अगर तुम ऐसा कर बैठे तो ये ज़रूर तुम्हारी शरारत है और ख़ुदा से डरो ख़ुदा तुमको मामले की सफ़ाई सिखाता है और वह हर चीज़ को ख़ूब जानता है
\end{hindi}}
\flushright{\begin{Arabic}
\quranayah[2][283]
\end{Arabic}}
\flushleft{\begin{hindi}
और अगर तुम सफ़र में हो और कोई लिखने वाला न मिले (और क़र्ज़ देना हो) तो रहन या कब्ज़ा रख लो और अगर तुममें एक का एक को एतबार हो तो (यूं ही क़र्ज़ दे सकता है मगर) फिर जिस शख्स पर एतबार किया गया है (क़र्ज़ लेने वाला) उसको चाहिये क़र्ज़ देने वाले की अमानत (क़र्ज़) पूरी पूरी अदा कर दे और अपने पालने वाले ख़ुदा से डरे (मुसलमानो) तुम गवाही को न छिपाओ और जो छिपाएगा तो बेशक उसका दिल गुनाहगार है और तुम लोग जो कुछ करते हो ख़ुदा उसको ख़ूब जानता है
\end{hindi}}
\flushright{\begin{Arabic}
\quranayah[2][284]
\end{Arabic}}
\flushleft{\begin{hindi}
जो कुछ आसमानों में है और जो कुछ ज़मीन में है (ग़रज़) सब कुछ खुदा ही का है और जो कुछ तुम्हारे दिलों में हे ख्वाह तुम उसको ज़ाहिर करो या उसे छिपाओ ख़ुदा तुमसे उसका हिसाब लेगा, फिर जिस को चाहे बख्श दे और जिस पर चाहे अज़ाब करे, और ख़ुदा हर चीज़ पर क़ादिर है
\end{hindi}}
\flushright{\begin{Arabic}
\quranayah[2][285]
\end{Arabic}}
\flushleft{\begin{hindi}
हमारे पैग़म्बर (मोहम्मद) जो कुछ उनपर उनके परवरदिगार की तरफ से नाज़िल किया गया है उस पर ईमान लाए और उनके (साथ) मोमिनीन भी (सबके) सब ख़ुदा और उसके फ़रिश्तों और उसकी किताबों और उसके रसूलों पर ईमान लाए (और कहते हैं कि) हम ख़ुदा के पैग़म्बरों में से किसी में तफ़रक़ा नहीं करते और कहने लगे ऐ हमारे परवरदिगार हमने (तेरा इरशाद) सुना
\end{hindi}}
\flushright{\begin{Arabic}
\quranayah[2][286]
\end{Arabic}}
\flushleft{\begin{hindi}
और मान लिया परवरदिगार हमें तेरी ही मग़फ़िरत की (ख्वाहिश है) और तेरी ही तरफ़ लौट कर जाना है ख़ुदा किसी को उसकी ताक़त से ज्यादा तकलीफ़ नहीं देता उसने अच्छा काम किया तो अपने नफ़े के लिए और बुरा काम किया तो (उसका वबाल) उसी पर पडेग़ा ऐ हमारे परवरदिगार अगर हम भूल जाऐं या ग़लती करें तो हमारी गिरफ्त न कर ऐ हमारे परवरदिगार हम पर वैसा बोझ न डाल जैसा हमसे अगले लोगों पर बोझा डाला था, और ऐ हमारे परवरदिगार इतना बोझ जिसके उठाने की हमें ताक़त न हो हमसे न उठवा और हमारे कुसूरों से दरगुज़र कर और हमारे गुनाहों को बख्श दे और हम पर रहम फ़रमा तू ही हमारा मालिक है तू ही काफ़िरों के मुक़ाबले में हमारी मदद कर
\end{hindi}}
\chapter{Al-'Imran (The Family of Amran)}
\begin{Arabic}
\Huge{\centerline{\basmalah}}\end{Arabic}
\flushright{\begin{Arabic}
\quranayah[3][1]
\end{Arabic}}
\flushleft{\begin{hindi}
अलिफ़ लाम मीम
\end{hindi}}
\flushright{\begin{Arabic}
\quranayah[3][2]
\end{Arabic}}
\flushleft{\begin{hindi}
अल्लाह ही वह (ख़ुदा) है जिसके सिवा कोई क़ाबिले परस्तिश नहीं है वही ज़िन्दा (और) सारे जहान का सॅभालने वाला है
\end{hindi}}
\flushright{\begin{Arabic}
\quranayah[3][3]
\end{Arabic}}
\flushleft{\begin{hindi}
(ऐ रसूल) उसी ने तुम पर बरहक़ किताब नाज़िल की जो (आसमानी किताबें पहले से) उसके सामने मौजूद हैं उनकी तसदीक़ करती है और उसी ने उससे पहले लोगों की हिदायत के वास्ते तौरेत व इन्जील नाज़िल की
\end{hindi}}
\flushright{\begin{Arabic}
\quranayah[3][4]
\end{Arabic}}
\flushleft{\begin{hindi}
और हक़ व बातिल में तमीज़ देने वाली किताब (कुरान) नाज़िल की बेशक जिन लोगों ने ख़ुदा की आयतों को न माना उनके लिए सख्त अज़ाब है और ख़ुदा हर चीज़ पर ग़ालिब बदला लेने वाला है
\end{hindi}}
\flushright{\begin{Arabic}
\quranayah[3][5]
\end{Arabic}}
\flushleft{\begin{hindi}
बेशक ख़ुदा पर कोई चीज़ पोशीदा नहीं है (न) ज़मीन में न आसमान में
\end{hindi}}
\flushright{\begin{Arabic}
\quranayah[3][6]
\end{Arabic}}
\flushleft{\begin{hindi}
वही तो वह ख़ुदा है जो माँ के पेट में तुम्हारी सूरत जैसी चाहता है बनाता हे उसके सिवा कोई माबूद नहीं
\end{hindi}}
\flushright{\begin{Arabic}
\quranayah[3][7]
\end{Arabic}}
\flushleft{\begin{hindi}
वही (हर चीज़ पर) ग़ालिब और दाना है (ए रसूल) वही (वह ख़ुदा) है जिसने तुमपर किताब नाज़िल की उसमें की बाज़ आयतें तो मोहकम (बहुत सरीह) हैं वही (अमल करने के लिए) असल (व बुनियाद) किताब है और कुछ (आयतें) मुतशाबेह (मिलती जुलती) (गोल गोल जिसके मायने में से पहलू निकल सकते हैं) पस जिन लोगों के दिलों में कज़ी है वह उन्हीं आयतों के पीछे पड़े रहते हैं जो मुतशाबेह हैं ताकि फ़साद बरपा करें और इस ख्याल से कि उन्हें मतलब पर ढाले लें हालाँकि ख़ुदा और उन लोगों के सिवा जो इल्म से बड़े पाए पर फ़ायज़ हैं उनका असली मतलब कोई नहीं जानता वह लोग (ये भी) कहते हैं कि हम उस पर ईमान लाए (यह) सब (मोहकम हो या मुतशाबेह) हमारे परवरदिगार की तरफ़ से है और अक्ल वाले ही समझते हैं
\end{hindi}}
\flushright{\begin{Arabic}
\quranayah[3][8]
\end{Arabic}}
\flushleft{\begin{hindi}
(और दुआ करते हैं) ऐ हमारे पालने वाले हमारे दिल को हिदायत करने के बाद डॉवाडोल न कर और अपनी बारगाह से हमें रहमत अता फ़रमा इसमें तो शक ही नहीं कि तू बड़ा देने वाला है
\end{hindi}}
\flushright{\begin{Arabic}
\quranayah[3][9]
\end{Arabic}}
\flushleft{\begin{hindi}
ऐ हमारे परवरदिगार बेशक तू एक न एक दिन जिसके आने में शुबह नहीं लोगों को इक्ट्ठा करेगा (तो हम पर नज़रे इनायत रहे) बेशक ख़ुदा अपने वायदे के ख़िलाफ़ नहीं करता
\end{hindi}}
\flushright{\begin{Arabic}
\quranayah[3][10]
\end{Arabic}}
\flushleft{\begin{hindi}
बेशक जिन लोगों ने कुफ्र किया इख्तेयार किया उनको ख़ुदा (के अज़ाब) से न उनके माल ही कुछ बचाएंगे, न उनकी औलाद (कुछ काम आएगी) और यही लोग जहन्नुम के ईधन होंगे
\end{hindi}}
\flushright{\begin{Arabic}
\quranayah[3][11]
\end{Arabic}}
\flushleft{\begin{hindi}
(उनकी भी) क़ौमे फ़िरऔन और उनसे पहले वालों की सी हालत है कि उन लोगों ने हमारी आयतों को झुठलाया तो खुदा ने उन्हें उनके गुनाहों की पादाश (सज़ा) में ले डाला और ख़ुदा सख्त सज़ा देने वाला है
\end{hindi}}
\flushright{\begin{Arabic}
\quranayah[3][12]
\end{Arabic}}
\flushleft{\begin{hindi}
(ऐ रसूल) जिन लोगों ने कुफ़्र इख्तेयार किया उनसे कह दो कि बहुत जल्द तुम (मुसलमानो के मुक़ाबले में) मग़लूब (हारे हुए) होंगे और जहन्नुम में इकट्ठे किये जाओगे और वह (क्या) बुरा ठिकाना है
\end{hindi}}
\flushright{\begin{Arabic}
\quranayah[3][13]
\end{Arabic}}
\flushleft{\begin{hindi}
बेशक तुम्हारे (समझाने के) वास्ते उन दो (मुख़ालिफ़ गिरोहों में जो (बद्र की लड़ाई में) एक दूसरे के साथ गुथ गए (रसूल की सच्चाई की) बड़ी भारी निशानी है कि एक गिरोह ख़ुदा की राह में जेहाद करता था और दूसरा काफ़िरों का जिनको मुसलमान अपनी ऑख से दुगना देख रहे थे (मगर ख़ुदा ने क़लील ही को फ़तह दी) और ख़ुदा अपनी मदद से जिस की चाहता है ताईद करता है बेशक ऑख वालों के वास्ते इस वाक़ये में बड़ी इबरत है
\end{hindi}}
\flushright{\begin{Arabic}
\quranayah[3][14]
\end{Arabic}}
\flushleft{\begin{hindi}
दुनिया में लोगों को उनकी मरग़ूब चीज़े (मसलन) बीवियों और बेटों और सोने चॉदी के बड़े बड़े लगे हुए ढेरों और उम्दा उम्दा घोड़ों और मवेशियों ओर खेती के साथ उलफ़त भली करके दिखा दी गई है ये सब दुनयावी ज़िन्दगी के (चन्द रोज़ा) फ़ायदे हैं और (हमेशा का) अच्छा ठिकाना तो ख़ुदा ही के यहॉ है
\end{hindi}}
\flushright{\begin{Arabic}
\quranayah[3][15]
\end{Arabic}}
\flushleft{\begin{hindi}
(ऐ रसूल) उन लोगों से कह दो कि क्या मैं तुमको उन सब चीज़ों से बेहतर चीज़ बता दूं (अच्छा सुनो) जिन लोगों ने परहेज़गारी इख्तेयार की उनके लिए उनके परवरदिगार के यहॉ (बेहिश्त) के वह बाग़ात हैं जिनके नीचे नहरें जारी हैं (और वह) हमेशा उसमें रहेंगे और उसके अलावा उनके लिए साफ सुथरी बीवियॉ हैं और (सबसे बढ़कर) ख़ुदा की ख़ुशनूदी है और ख़ुदा (अपने) उन बन्दों को खूब देख रहा हे जो दुआऐं मॉगा करते हैं
\end{hindi}}
\flushright{\begin{Arabic}
\quranayah[3][16]
\end{Arabic}}
\flushleft{\begin{hindi}
कि हमारे पालने वाले हम तो (बेताम्मुल) ईमान लाए हैं पस तू भी हमारे गुनाहों को बख्श दे और हमको दोज़ख़ के अज़ाब से बचा
\end{hindi}}
\flushright{\begin{Arabic}
\quranayah[3][17]
\end{Arabic}}
\flushleft{\begin{hindi}
(यही लोग हैं) सब्र करने वाले और सच बोलने वाले और (ख़ुदा के) फ़रमाबरदार और (ख़ुदा की राह में) ख़र्च करने वाले और पिछली रातों में (ख़ुदा से तौबा) इस्तग़फ़ार करने वाले
\end{hindi}}
\flushright{\begin{Arabic}
\quranayah[3][18]
\end{Arabic}}
\flushleft{\begin{hindi}
ज़रूर ख़ुदा और फ़रिश्तों और इल्म वालों ने गवाही दी है कि उसके सिवा कोई माबूद क़ाबिले परसतिश नहीं है और वह ख़ुदा अद्ल व इन्साफ़ के साथ (कारख़ानाए आलम का) सॅभालने वाला है उसके सिवा कोई माबूद नहीं (वही हर चीज़ पर) ग़ालिब और दाना है (सच्चा) दीन तो ख़ुदा के नज़दीक यक़ीनन (बस यही) इस्लाम है
\end{hindi}}
\flushright{\begin{Arabic}
\quranayah[3][19]
\end{Arabic}}
\flushleft{\begin{hindi}
और अहले किताब ने जो उस दीने हक़ से इख्तेलाफ़ किया तो महज़ आपस की शरारत और असली (अम्र) मालूम हो जाने के बाद (ही क्या है) और जिस शख्स ने ख़ुदा की निशानियों से इन्कार किया तो (वह समझ ले कि यक़ीनन ख़ुदा (उससे) बहुत जल्दी हिसाब लेने वाला है
\end{hindi}}
\flushright{\begin{Arabic}
\quranayah[3][20]
\end{Arabic}}
\flushleft{\begin{hindi}
(ऐ रसूल) पस अगर ये लोग तुमसे (ख्वाह मा ख्वाह) हुज्जत करे तो कह दो मैंने ख़ुदा के आगे अपना सरे तस्लीम ख़म कर दिया है और जो मेरे ताबे है (उन्होंने) भी) और ऐ रसूल तुम अहले किताब और जाहिलों से पूंछो कि क्या तुम भी इस्लाम लाए हो (या नही) पस अगर इस्लाम लाए हैं तो बेख़टके राहे रास्त पर आ गए और अगर मुंह फेरे तो (ऐ रसूल) तुम पर सिर्फ़ पैग़ाम (इस्लाम) पहुंचा देना फ़र्ज़ है (बस) और ख़ुदा (अपने बन्दों) को देख रहा है
\end{hindi}}
\flushright{\begin{Arabic}
\quranayah[3][21]
\end{Arabic}}
\flushleft{\begin{hindi}
बेशक जो लोग ख़ुदा की आयतों से इन्कार करते हैं और नाहक़ पैग़म्बरों को क़त्ल करते हैं और उन लोगों को (भी) क़त्ल करते हैं जो (उन्हें) इन्साफ़ करने का हुक्म करते हैं तो (ऐ रसूल) तुम उन लोगों को दर्दनाक अज़ाब की ख़ुशख़बरी दे दो
\end{hindi}}
\flushright{\begin{Arabic}
\quranayah[3][22]
\end{Arabic}}
\flushleft{\begin{hindi}
यही वह (बदनसीब) लोग हैं जिनका सारा किया कराया दुनिया और आख़ेरत (दोनों) में अकारत गया और कोई उनका मददगार नहीं
\end{hindi}}
\flushright{\begin{Arabic}
\quranayah[3][23]
\end{Arabic}}
\flushleft{\begin{hindi}
(ऐ रसूल) क्या तुमने (उलमाए यहूद) के हाल पर नज़र नहीं की जिनको किताब (तौरेत) का एक हिस्सा दिया गया था (अब) उनको किताबे ख़ुदा की तरफ़ बुलाया जाता है ताकि वही (किताब) उनके झगड़ें का फैसला कर दे इस पर भी उनमें का एक गिरोह मुंह फेर लेता है और यही लोग रूगरदानी (मुँह फेरने) करने वाले हैं
\end{hindi}}
\flushright{\begin{Arabic}
\quranayah[3][24]
\end{Arabic}}
\flushleft{\begin{hindi}
ये इस वजह से है कि वह लोग कहते हैं कि हमें गिनती के चन्द दिनों के सिवा जहन्नुम की आग हरगिज़ छुएगी भी तो नहीं जो इफ़तेरा परदाज़ी ये लोग बराबर करते आए हैं उसी ने उन्हें उनके दीन में भी धोखा दिया है
\end{hindi}}
\flushright{\begin{Arabic}
\quranayah[3][25]
\end{Arabic}}
\flushleft{\begin{hindi}
फ़िर उनकी क्या गत होगी जब हम उनको एक दिन (क़यामत) जिसके आने में कोई शुबहा नहीं इक्ट्ठा करेंगे और हर शख्स को उसके किए का पूरा पूरा बदला दिया जाएगा और उनकी किसी तरह हक़तल्फ़ी नहीं की जाएगी
\end{hindi}}
\flushright{\begin{Arabic}
\quranayah[3][26]
\end{Arabic}}
\flushleft{\begin{hindi}
(ऐ रसूल) तुम तो यह दुआ मॉगों कि ऐ ख़ुदा तमाम आलम के मालिक तू ही जिसको चाहे सल्तनत दे और जिससे चाहे सल्तनत छीन ले और तू ही जिसको चाहे इज्ज़त दे और जिसे चाहे ज़िल्लत दे हर तरह की भलाई तेरे ही हाथ में है बेशक तू ही हर चीज़ पर क़ादिर है
\end{hindi}}
\flushright{\begin{Arabic}
\quranayah[3][27]
\end{Arabic}}
\flushleft{\begin{hindi}
तू ही रात को (बढ़ा के) दिन में दाख़िल कर देता है (तो) रात बढ़ जाती है और तू ही दिन को (बढ़ा के) रात में दाख़िल करता है (तो दिन बढ़ जाता है) तू ही बेजान (अन्डा नुत्फ़ा वगैरह) से जानदार को पैदा करता है और तू ही जानदार से बेजान नुत्फ़ा (वगैरहा) निकालता है और तू ही जिसको चाहता है बेहिसाब रोज़ी देता है
\end{hindi}}
\flushright{\begin{Arabic}
\quranayah[3][28]
\end{Arabic}}
\flushleft{\begin{hindi}
मोमिनीन मोमिनीन को छोड़ के काफ़िरों को अपना सरपरस्त न बनाऐं और जो ऐसा करेगा तो उससे ख़ुदा से कुछ सरोकार नहीं मगर (इस क़िस्म की तदबीरों से) किसी तरह उन (के शर) से बचना चाहो तो (ख़ैर) और ख़ुदा तुमको अपने ही से डराता है और ख़ुदा ही की तरफ़ लौट कर जाना है
\end{hindi}}
\flushright{\begin{Arabic}
\quranayah[3][29]
\end{Arabic}}
\flushleft{\begin{hindi}
ऐ रसूल तुम उन (लोगों से) कह दो किजो कुछ तुम्हारे दिलों में है तो ख्वाह उसे छिपाओ या ज़ाहिर करो (बहरहाल) ख़ुदा तो उसे जानता है और जो कुछ आसमानों में है और जो कुछ ज़मीन में वह (सब कुछ) जानता है और ख़ुदा हर चीज़ पर क़ादिर है
\end{hindi}}
\flushright{\begin{Arabic}
\quranayah[3][30]
\end{Arabic}}
\flushleft{\begin{hindi}
(और उस दिन को याद रखो) जिस दिन हर शख्स जो कुछ उसने (दुनिया में) नेकी की है और जो कुछ बुराई की है उसको मौजूद पाएगा (और) आरज़ू करेगा कि काश उस की बदी और उसके दरमियान में ज़मानए दराज़ (हाएल) हो जाता और ख़ुदा तुमको अपने ही से डराता है और ख़ुदा अपने बन्दों पर बड़ा शफ़ीक़ और (मेहरबान भी) है
\end{hindi}}
\flushright{\begin{Arabic}
\quranayah[3][31]
\end{Arabic}}
\flushleft{\begin{hindi}
(ऐ रसूल) उन लोगों से कह दो कि अगर तुम ख़ुदा को दोस्त रखते हो तो मेरी पैरवी करो कि ख़ुदा (भी) तुमको दोस्त रखेगा और तुमको तुम्हारे गुनाह बख्श देगा और खुदा बड़ा बख्शने वाला मेहरबान है
\end{hindi}}
\flushright{\begin{Arabic}
\quranayah[3][32]
\end{Arabic}}
\flushleft{\begin{hindi}
(ऐ रसूल) कह दो कि ख़ुदा और रसूल की फ़रमाबरदारी करो फिर अगर यह लोग उससे सरताबी करें तो (समझ लें कि) ख़ुदा काफ़िरों को हरगिज़ दोस्त नहीं रखता
\end{hindi}}
\flushright{\begin{Arabic}
\quranayah[3][33]
\end{Arabic}}
\flushleft{\begin{hindi}
बेशक ख़ुदा ने आदम और नूह और ख़ानदाने इबराहीम और खानदाने इमरान को सारे जहान से बरगुज़ीदा किया है
\end{hindi}}
\flushright{\begin{Arabic}
\quranayah[3][34]
\end{Arabic}}
\flushleft{\begin{hindi}
बाज़ की औलाद को बाज़ से और ख़ुदा (सबकी) सुनता (और सब कुछ) जानता है
\end{hindi}}
\flushright{\begin{Arabic}
\quranayah[3][35]
\end{Arabic}}
\flushleft{\begin{hindi}
(ऐ रसूल वह वक्त याद करो) जब इमरान की बीवी ने (ख़ुदा से) अर्ज़ की कि ऐ मेरे पालने वाले मेरे पेट में जो बच्चा है (उसको मैं दुनिया के काम से) आज़ाद करके तेरी नज़्र करती हूं तू मेरी तरफ़ से (ये नज़्र कुबूल फ़रमा तू बेशक बड़ा सुनने वाला और जानने वाला है
\end{hindi}}
\flushright{\begin{Arabic}
\quranayah[3][36]
\end{Arabic}}
\flushleft{\begin{hindi}
फिर जब वह बेटी जन चुकी तो (हैरत से) कहने लगी ऐ मेरे परवरदिगार (मैं क्या करूं) मैं तो ये लड़की जनी हूँ और लड़का लड़की के ऐसा (गया गुज़रा) नहीं होता हालॉकि उसे कहने की ज़रूरत क्या थी जो वे जनी थी ख़ुदा उस (की शान व मरतबा) से खूब वाक़िफ़ था और मैंने उसका नाम मरियम रखा है और मैं उसको और उसकी औलाद को शैतान मरदूद (के फ़रेब) से तेरी पनाह में देती हूं
\end{hindi}}
\flushright{\begin{Arabic}
\quranayah[3][37]
\end{Arabic}}
\flushleft{\begin{hindi}
तो उसके परवरदिगार ने (उनकी नज़्र) मरियम को ख़ुशी से कुबूल फ़रमाया और उसकी नशो व नुमा (परवरिश) अच्छी तरह की और ज़करिया को उनका कफ़ील बनाया जब किसी वक्त ज़क़रिया उनके पास (उनके) इबादत के हुजरे में जाते तो मरियम के पास कुछ न कुछ खाने को मौजूद पाते तो पूंछते कि ऐ मरियम ये (खाना) तुम्हारे पास कहॉ से आया है तो मरियम ये कह देती थी कि यह खुदा के यहॉ से (आया) है बेशक ख़ुदा जिसको चाहता है बेहिसाब रोज़ी देता है
\end{hindi}}
\flushright{\begin{Arabic}
\quranayah[3][38]
\end{Arabic}}
\flushleft{\begin{hindi}
(ये माजरा देखते ही) उसी वक्त ज़करिया ने अपने परवरदिगार से दुआ कि और अर्ज क़ी ऐ मेरे पालने वाले तू मुझको (भी) अपनी बारगाह से पाकीज़ा औलाद अता फ़रमा बेशक तू ही दुआ का सुनने वाला है
\end{hindi}}
\flushright{\begin{Arabic}
\quranayah[3][39]
\end{Arabic}}
\flushleft{\begin{hindi}
अभी ज़करिया हुजरे में खड़े (ये) दुआ कर ही रहे थे कि फ़रिश्तों ने उनको आवाज़ दी कि ख़ुदा तुमको यहया (के पैदा होने) की खुशख़बरी देता है जो जो कलेमतुल्लाह (ईसा) की तस्दीक़ करेगा और (लोगों का) सरदार होगा और औरतों की तरफ़ रग़बत न करेगा और नेको कार नबी होगा
\end{hindi}}
\flushright{\begin{Arabic}
\quranayah[3][40]
\end{Arabic}}
\flushleft{\begin{hindi}
ज़करिया ने अर्ज़ की परवरदिगार मुझे लड़का क्योंकर हो सकता है हालॉकि मेरा बुढ़ापा आ पहुंचा और (उसपर) मेरी बीवी बॉझ है (ख़ुदा ने) फ़रमाया इसी तरह ख़ुदा जो चाहता है करता है
\end{hindi}}
\flushright{\begin{Arabic}
\quranayah[3][41]
\end{Arabic}}
\flushleft{\begin{hindi}
ज़करिया ने अर्ज़ की परवरदिगार मेरे इत्मेनान के लिए कोई निशानी मुक़र्रर फ़रमा इरशाद हुआ तुम्हारी निशानी ये है तुम तीन दिन तक लोगों से बात न कर सकोगे मगर इशारे से और (उसके शुक्रिये में) अपने परवरदिगार की अक्सर याद करो और रात को और सुबह तड़के (हमारी) तसबीह किया करो
\end{hindi}}
\flushright{\begin{Arabic}
\quranayah[3][42]
\end{Arabic}}
\flushleft{\begin{hindi}
और वह वाक़िया भी याद करो जब फ़रिश्तों ने मरियम से कहा, ऐ मरियम तुमको ख़ुदा ने बरगुज़ीदा किया और (तमाम) गुनाहों और बुराइयों से पाक साफ़ रखा और सारे दुनिया जहॉन की औरतों में से तुमको मुन्तख़िब किया है
\end{hindi}}
\flushright{\begin{Arabic}
\quranayah[3][43]
\end{Arabic}}
\flushleft{\begin{hindi}
(तो) ऐ मरियम इसके शुक्र से मैं अपने परवरदिगार की फ़रमाबदारी करूं सजदा और रूकूउ करने वालों के साथ रूकूउ करती रहूं
\end{hindi}}
\flushright{\begin{Arabic}
\quranayah[3][44]
\end{Arabic}}
\flushleft{\begin{hindi}
(ऐ रसूल) ये ख़बर गैब की ख़बरों में से है जो हम तुम्हारे पास 'वही' के ज़रिए से भेजते हैं (ऐ रसूल) तुम तो उन सरपरस्ताने मरियम के पास मौजूद न थे जब वह लोग अपना अपना क़लम दरिया में बतौर क़ुरआ के डाल रहे थे (देखें) कौन मरियम का कफ़ील बनता है और न तुम उस वक्त उनके पास मौजूद थे जब वह लोग आपस में झगड़ रहे थे
\end{hindi}}
\flushright{\begin{Arabic}
\quranayah[3][45]
\end{Arabic}}
\flushleft{\begin{hindi}
(वह वाक़िया भी याद करो) जब फ़रिश्तों ने (मरियम) से कहा ऐ मरियम ख़ुदा तुमको सिर्फ़ अपने हुक्म से एक लड़के के पैदा होने की खुशख़बरी देता है जिसका नाम ईसा मसीह इब्ने मरियम होगा (और) दुनिया और आखेरत (दोनों) में बाइज्ज़त (आबरू) और ख़ुदा के मुक़र्रब बन्दों में होगा
\end{hindi}}
\flushright{\begin{Arabic}
\quranayah[3][46]
\end{Arabic}}
\flushleft{\begin{hindi}
और (बचपन में) जब झूले में पड़ा होगा और बड़ी उम्र का होकर (दोनों हालतों में यकसॉ) लोगों से बाते करेगा और नेको कारों में से होगा
\end{hindi}}
\flushright{\begin{Arabic}
\quranayah[3][47]
\end{Arabic}}
\flushleft{\begin{hindi}
(ये सुनकर मरियम ताज्जुब से) कहने लगी परवरदिगार मुझे लड़का क्योंकर होगा हालॉकि मुझे किसी मर्द ने छुआ तक नहीं इरशाद हुआ इसी तरह ख़ुदा जो चाहता है करता है जब वह किसी काम का करना ठान लेता है तो बस कह देता है 'हो जा' तो वह हो जाता है
\end{hindi}}
\flushright{\begin{Arabic}
\quranayah[3][48]
\end{Arabic}}
\flushleft{\begin{hindi}
और (ऐ मरयिम) ख़ुदा उसको (तमाम) किताबे आसमानी और अक्ल की बातें और (ख़ासकर) तौरेत व इन्जील सिखा देगा
\end{hindi}}
\flushright{\begin{Arabic}
\quranayah[3][49]
\end{Arabic}}
\flushleft{\begin{hindi}
और बनी इसराइल का रसूल (क़रार देगा और वह उनसे यूं कहेगा कि) मैं तुम्हारे पास तुम्हारे परवरदिगार की तरफ़ से (अपनी नबूवत की) यह निशानी लेकर आया हूं कि मैं गुंधीं हुई मिट्टी से एक परिन्दे की सूरत बनाऊॅगा फ़िर उस पर (कुछ) दम करूंगा तो वो ख़ुदा के हुक्म से उड़ने लगेगा और मैं ख़ुदा ही के हुक्म से मादरज़ाद (पैदायशी) अंधे और कोढ़ी को अच्छा करूंगा और मुर्दो को ज़िन्दा करूंगा और जो कुछ तुम खाते हो और अपने घरों में जमा करते हो मैं (सब) तुमको बता दूंगा अगर तुम ईमानदार हो तो बेशक तुम्हारे लिये इन बातों में (मेरी नबूवत की) बड़ी निशानी है
\end{hindi}}
\flushright{\begin{Arabic}
\quranayah[3][50]
\end{Arabic}}
\flushleft{\begin{hindi}
और तौरेत जो मेरे सामने मौजूद है मैं उसकी तसदीक़ करता हूं और (मेरे आने की) एक ग़रज़ यह (भी) है कि जो चीजे तुम पर हराम है उनमें से बाज़ को (हुक्मे ख़ुदा से) हलाल कर दूं और मैं तुम्हारे परवरदिगार की तरफ़ से (अपनी नबूवत की) निशानी लेकर तुम्हारे पास आया हूं
\end{hindi}}
\flushright{\begin{Arabic}
\quranayah[3][51]
\end{Arabic}}
\flushleft{\begin{hindi}
पस तुम ख़ुदा से डरो और मेरी इताअत करो बेशक ख़ुदा ही मेरा और तुम्हारा परवरदिगार है
\end{hindi}}
\flushright{\begin{Arabic}
\quranayah[3][52]
\end{Arabic}}
\flushleft{\begin{hindi}
पस उसकी इबादत करो (क्योंकि) यही नजात का सीधा रास्ता है फिर जब ईसा ने (इतनी बातों के बाद भी) उनका कुफ़्र (पर अड़े रहना) देखा तो (आख़िर) कहने लगे कौन ऐसा है जो ख़ुदा की तरफ़ होकर मेरा मददगार बने (ये सुनकर) हवारियों ने कहा हम ख़ुदा के तरफ़दार हैं और हम ख़ुदा पर ईमान लाए
\end{hindi}}
\flushright{\begin{Arabic}
\quranayah[3][53]
\end{Arabic}}
\flushleft{\begin{hindi}
और (ईसा से कहा) आप गवाह रहिए कि हम फ़रमाबरदार हैं
\end{hindi}}
\flushright{\begin{Arabic}
\quranayah[3][54]
\end{Arabic}}
\flushleft{\begin{hindi}
और ख़ुदा की बारगाह में अर्ज़ की कि ऐ हमारे पालने वाले जो कुछ तूने नाज़िल किया हम उसपर ईमान लाए और हमने तेरे रसूल (ईसा) की पैरवी इख्तेयार की पस तू हमें (अपने रसूल के) गवाहों के दफ्तर में लिख ले
\end{hindi}}
\flushright{\begin{Arabic}
\quranayah[3][55]
\end{Arabic}}
\flushleft{\begin{hindi}
और यहूदियों (ने ईसा से) मक्कारी की और ख़ुदा ने उसके दफ़ईया की तदबीर की और ख़ुदा सब से बेहतर तदबीर करने वाला है (वह वक्त भी याद करो) जब ईसा से ख़ुदा ने फ़रमाया ऐ ईसा मैं ज़रूर तुम्हारी ज़िन्दगी की मुद्दत पूरी करके तुमको अपनी तरफ़ उठा लूंगा और काफ़िरों (की ज़िन्दगी) से तुमको पाक व पाकीज़ा रखूंगा और जिन लोगों ने तुम्हारी पैरवी की उनको क़यामत तक काफ़िरों पर ग़ालिब रखूंगा फिर तुम सबको मेरी तरफ़ लौटकर आना है
\end{hindi}}
\flushright{\begin{Arabic}
\quranayah[3][56]
\end{Arabic}}
\flushleft{\begin{hindi}
तब (उस दिन) जिन बातों में तुम (दुनिया) में झगड़े करते थे (उनका) तुम्हारे दरमियान फ़ैसला कर दूंगा पस जिन लोगों ने कुफ़्र इख्तेयार किया उनपर दुनिया और आख़िरत (दोनों में) सख्त अज़ाब करूंगा और उनका कोई मददगार न होगा
\end{hindi}}
\flushright{\begin{Arabic}
\quranayah[3][57]
\end{Arabic}}
\flushleft{\begin{hindi}
और जिन लोगों ने ईमान क़ुबूल किया और अच्छे (अच्छे) काम किए तो ख़ुदा उनको उनका पूरा अज्र व सवाब देगा और ख़ुदा ज़ालिमों को दोस्त नहीं रखता
\end{hindi}}
\flushright{\begin{Arabic}
\quranayah[3][58]
\end{Arabic}}
\flushleft{\begin{hindi}
(ऐ रसूल) ये जो हम तुम्हारे सामने बयान कर रहे हैं कुदरते ख़ुदा की निशानियॉ और हिकमत से भरे हुये तज़किरे हैं
\end{hindi}}
\flushright{\begin{Arabic}
\quranayah[3][59]
\end{Arabic}}
\flushleft{\begin{hindi}
ख़ुदा के नज़दीक तो जैसे ईसा की हालत वैसी ही आदम की हालत कि उनको को मिट्टी का पुतला बनाकर कहा कि 'हो जा' पस (फ़ौरन ही) वह (इन्सान) हो गया
\end{hindi}}
\flushright{\begin{Arabic}
\quranayah[3][60]
\end{Arabic}}
\flushleft{\begin{hindi}
(ऐ रसूल ये है) हक़ बात (जो) तुम्हारे परवरदिगार की तरफ़ से (बताई जाती है) तो तुम शक करने वालों में से न हो जाना
\end{hindi}}
\flushright{\begin{Arabic}
\quranayah[3][61]
\end{Arabic}}
\flushleft{\begin{hindi}
फिर जब तुम्हारे पास इल्म (कुरान) आ चुका उसके बाद भी अगर तुम से कोई (नसरानी) ईसा के बारे में हुज्जत करें तो कहो कि (अच्छा मैदान में) आओ हम अपने बेटों को बुलाएं तुम अपने बेटों को और हम अपनी औरतों को (बुलाएं) और तुम अपनी औरतों को और हम अपनी जानों को बुलाएं ओर तुम अपनी जानों को
\end{hindi}}
\flushright{\begin{Arabic}
\quranayah[3][62]
\end{Arabic}}
\flushleft{\begin{hindi}
उसके बाद हम सब मिलकर (खुदा की बारगाह में) गिड़गिड़ाएं और झूठों पर ख़ुदा की लानत करें (ऐ रसूल) ये सब यक़ीनी सच्चे वाक़यात हैं और ख़ुदा के सिवा कोई माबूद (क़ाबिले परसतिश) नहीं है
\end{hindi}}
\flushright{\begin{Arabic}
\quranayah[3][63]
\end{Arabic}}
\flushleft{\begin{hindi}
और बेशक ख़ुदा ही सब पर ग़ालिब और हिकमत वाला है
\end{hindi}}
\flushright{\begin{Arabic}
\quranayah[3][64]
\end{Arabic}}
\flushleft{\begin{hindi}
फिर अगर इससे भी मुंह फेरें तो (कुछ) परवाह (नहीं) ख़ुदा फ़सादी लोगों को खूब जानता है (ऐ रसूल) तुम (उनसे) कहो कि ऐ अहले किताब तुम ऐसी (ठिकाने की) बात पर तो आओ जो हमारे और तुम्हारे दरमियान यकसॉ है कि खुदा के सिवा किसी की इबादत न करें और किसी चीज़ को उसका शरीक न बनाएं और ख़ुदा के सिवा हममें से कोई किसी को अपना परवरदिगार न बनाए अगर इससे भी मुंह मोडें तो तुम गवाह रहना हम (ख़ुदा के) फ़रमाबरदार हैं
\end{hindi}}
\flushright{\begin{Arabic}
\quranayah[3][65]
\end{Arabic}}
\flushleft{\begin{hindi}
ऐ अहले किताब तुम इबराहीम के बारे में (ख्वाह मा ख्वाह) क्यों झगड़ते हो कि कोई उनको नसरानी कहता है कोई यहूदी हालॉकि तौरेत व इन्जील (जिनसे यहूद व नसारा की इब्तेदा है वह) तो उनके बाद ही नाज़िल हुई
\end{hindi}}
\flushright{\begin{Arabic}
\quranayah[3][66]
\end{Arabic}}
\flushleft{\begin{hindi}
तो क्या तुम इतना भी नहीं समझते? (ऐ लो अरे) तुम वही एहमक़ लोग हो कि जिस का तुम्हें कुछ इल्म था उसमें तो झगड़ा कर चुके (खैर) फिर तब उसमें क्या (ख्वाह मा ख्वाह) झगड़ने बैठे हो जिसकी (सिरे से) तुम्हें कुछ ख़बर नहीं और (हकॣक़ते हाल तो) खुदा जानता है और तुम नहीं जानते
\end{hindi}}
\flushright{\begin{Arabic}
\quranayah[3][67]
\end{Arabic}}
\flushleft{\begin{hindi}
इबराहीम न तो यहूदी थे और न नसरानी बल्कि निरे खरे हक़ पर थे (और) फ़रमाबरदार (बन्दे) थे और मुशरिकों से भी न थे
\end{hindi}}
\flushright{\begin{Arabic}
\quranayah[3][68]
\end{Arabic}}
\flushleft{\begin{hindi}
इबराहीम से ज्यादा ख़ुसूसियत तो उन लोगों को थी जो ख़ास उनकी पैरवी करते थे और उस पैग़म्बर और ईमानदारों को (भी) है और मोमिनीन का ख़ुदा मालिक है
\end{hindi}}
\flushright{\begin{Arabic}
\quranayah[3][69]
\end{Arabic}}
\flushleft{\begin{hindi}
(मुसलमानो) अहले किताब से एक गिरोह ने बहुत चाहा कि किसी तरह तुमको राहेरास्त से भटका दे हालॉकि वह (अपनी तदबीरों से तुमको तो नहीं मगर) अपने ही को भटकाते हैं
\end{hindi}}
\flushright{\begin{Arabic}
\quranayah[3][70]
\end{Arabic}}
\flushleft{\begin{hindi}
और उसको समझते (भी) नहीं ऐ अहले किताब तुम ख़ुदा की आयतों से क्यों इन्कार करते हो, हालॉकि तुम ख़ुद गवाह बन सकते हो
\end{hindi}}
\flushright{\begin{Arabic}
\quranayah[3][71]
\end{Arabic}}
\flushleft{\begin{hindi}
ऐ अहले किताब तुम क्यो हक़ व बातिल को गड़बड़ करते और हक़ को छुपाते हो हालॉकि तुम जानते हो
\end{hindi}}
\flushright{\begin{Arabic}
\quranayah[3][72]
\end{Arabic}}
\flushleft{\begin{hindi}
और अहले किताब से एक गिरोह ने (अपने लोगों से) कहा कि मुसलमानों पर जो किताब नाज़िल हुईहै उसपर सुबह सवेरे ईमान लाओ और आख़िर वक्त ऌन्कार कर दिया करो शायद मुसलमान (इसी तदबीर से अपने दीन से) फिर जाए
\end{hindi}}
\flushright{\begin{Arabic}
\quranayah[3][73]
\end{Arabic}}
\flushleft{\begin{hindi}
और तुम्हारे दीन की पैरवरी करे उसके सिवा किसी दूसरे की बात का ऐतबार न करो (ऐ रसूल) तुम कह दो कि बस ख़ुदा ही की हिदायत तो हिदायत है (यहूदी बाहम ये भी कहते हैं कि) उसको भी न (मानना) कि जैसा (उम्दा दीन) तुमको दिया गया है, वैसा किसी और को दिया जाय या तुमसे कोई शख्स ख़ुदा के यहॉ झगड़ा करे (ऐ रसूल तुम उनसे) कह दो कि (ये क्या ग़लत ख्याल है) फ़ज़ल (व करम) ख़ुदा के हाथ में है वह जिसको चाहे दे और ख़ुदा बड़ी गुन्जाईश वाला है (और हर शै को)े जानता है
\end{hindi}}
\flushright{\begin{Arabic}
\quranayah[3][74]
\end{Arabic}}
\flushleft{\begin{hindi}
जिसको चाहे अपनी रहमत के लिये ख़ास कर लेता है और ख़ुदा बड़ा फ़ज़लों करम वाला हे
\end{hindi}}
\flushright{\begin{Arabic}
\quranayah[3][75]
\end{Arabic}}
\flushleft{\begin{hindi}
और अहले किताब कुछ ऐसे भी हैं कि अगर उनके पास रूपए की ढेर अमानत रख दो तो भी उसे (जब चाहो) वैसे ही तुम्हारे हवाले कर देंगे और बाज़ ऐसे हें कि अगर एक अशर्फ़ी भी अमानत रखो तो जब तक तुम बराबर (उनके सर) पर खड़े न रहोगे तुम्हें वापस न देंगे ये (बदमुआम लगी) इस वजह से है कि उन का तो ये क़ौल है कि (अरब के) जाहिलो (का हक़ मार लेने) में हम पर कोई इल्ज़ाम की राह ही नहीं और जान बूझ कर खुदा पर झूठ (तूफ़ान) जोड़ते हैं
\end{hindi}}
\flushright{\begin{Arabic}
\quranayah[3][76]
\end{Arabic}}
\flushleft{\begin{hindi}
हाँ (अलबत्ता) जो शख्स अपने एहद को पूरा करे और परहेज़गारी इख्तेयार करे तो बेशक ख़ुदा परहेज़गारों को दोस्त रखता है
\end{hindi}}
\flushright{\begin{Arabic}
\quranayah[3][77]
\end{Arabic}}
\flushleft{\begin{hindi}
बेशक जो लोग अपने एहद और (क़समे) जो ख़ुदा (से किया था उसके) बदले थोड़ा (दुनयावी) मुआवेज़ा ले लेते हैं उन ही लोगों के वास्ते आख़िरत में कुछ हिस्सा नहीं और क़यामत के दिन ख़ुदा उनसे बात तक तो करेगा नहीं ओर उनकी तरफ़ नज़र (रहमत) ही करेगा और न उनको (गुनाहों की गन्दगी से) पाक करेगा और उनके लिये दर्दनाम अज़ाब है
\end{hindi}}
\flushright{\begin{Arabic}
\quranayah[3][78]
\end{Arabic}}
\flushleft{\begin{hindi}
और अहले किताब से बाज़ ऐसे ज़रूर हैं कि किताब (तौरेत) में अपनी ज़बाने मरोड़ मरोड़ के (कुछ का कुछ) पढ़ जाते हैं ताकि तुम ये समझो कि ये किताब का जुज़ है हालॉकि वह किताब का जुज़ नहीं और कहते हैं कि ये (जो हम पढ़ते हैं) ख़ुदा के यहॉ से (उतरा) है हालॉकि वह ख़ुदा के यहॉ से नहीं (उतरा) और जानबूझ कर ख़ुदा पर झूठ (तूफ़ान) जोड़ते हैं
\end{hindi}}
\flushright{\begin{Arabic}
\quranayah[3][79]
\end{Arabic}}
\flushleft{\begin{hindi}
किसी आदमी को ये ज़ेबा न था कि ख़ुदा तो उसे (अपनी) किताब और हिकमत और नबूवत अता फ़रमाए और वह लोगों से कहता फिरे कि ख़ुदा को छोड़कर मेरे बन्दे बन जाओ बल्कि (वह तो यही कहेगा कि) तुम अल्लाह वाले बन जाओ क्योंकि तुम तो (हमेशा) किताबे ख़ुदा (दूसरो) को पढ़ाते रहते हो और तुम ख़ुद भी सदा पढ़ते रहे हो
\end{hindi}}
\flushright{\begin{Arabic}
\quranayah[3][80]
\end{Arabic}}
\flushleft{\begin{hindi}
और वह तुमसे ये तो (कभी) न कहेगा कि फ़रिश्तों और पैग़म्बरों को ख़ुदा बना लो भला (कहीं ऐसा हो सकता है कि) तुम्हारे मुसलमान हो जाने के बाद तुम्हें कुफ़्र का हुक्म करेगा
\end{hindi}}
\flushright{\begin{Arabic}
\quranayah[3][81]
\end{Arabic}}
\flushleft{\begin{hindi}
(और ऐ रसूल वह वक्त भी याद दिलाओ) जब ख़ुदा ने पैग़म्बरों से इक़रार लिया कि हम तुमको जो कुछ किताब और हिकमत (वगैरह) दे उसके बाद तुम्हारे पास कोई रसूल आए और जो किताब तुम्हारे पास उसकी तसदीक़ करे तो (देखो) तुम ज़रूर उस पर ईमान लाना, और ज़रूर उसकी मदद करना (और) ख़ुदा ने फ़रमाया क्या तुमने इक़रार लिया तुमने मेरे (एहद का) बोझ उठा लिया सबने अर्ज़ की हमने इक़रार किया इरशाद हुआ (अच्छा) तो आज के क़ौल व (क़रार के) आपस में एक दूसरे के गवाह रहना
\end{hindi}}
\flushright{\begin{Arabic}
\quranayah[3][82]
\end{Arabic}}
\flushleft{\begin{hindi}
और तुम्हारे साथ मैं भी एक गवाह हूं फिर उसके बाद जो शख्स (अपने क़ौल से) मुंह फेरे तो वही लोग बदचलन हैं
\end{hindi}}
\flushright{\begin{Arabic}
\quranayah[3][83]
\end{Arabic}}
\flushleft{\begin{hindi}
तो क्या ये लोग ख़ुदा के दीन के सिवा (कोई और दीन) ढूंढते हैं हालॉकि जो (फ़रिश्ते) आसमानों में हैं और जो (लोग) ज़मीन में हैं सबने ख़ुशी ख़ुशी या ज़बरदस्ती उसके सामने अपनी गर्दन डाल दी है और (आख़िर सब) उसकी तरफ़ लौट कर जाएंगे
\end{hindi}}
\flushright{\begin{Arabic}
\quranayah[3][84]
\end{Arabic}}
\flushleft{\begin{hindi}
(ऐ रसूल उन लोगों से) कह दो कि हम तो ख़ुदा पर ईमान लाए और जो किताब हम पर नाज़िल हुई और जो (सहीफ़े) इबराहीम और इस्माईल और इसहाक़ और याकूब और औलादे याकूब पर नाज़िल हुये और मूसा और ईसा और दूसरे पैग़म्बरों को जो (जो किताब) उनके परवरदिगार की तरफ़ से इनायत हुई(सब पर ईमान लाए) हम तो उनमें से किसी एक में भी फ़र्क़ नहीं करते
\end{hindi}}
\flushright{\begin{Arabic}
\quranayah[3][85]
\end{Arabic}}
\flushleft{\begin{hindi}
और हम तो उसी (यकता ख़ुदा) के फ़रमाबरदार हैं और जो शख्स इस्लाम के सिवा किसी और दीन की ख्वाहिश करे तो उसका वह दीन हरगिज़ कुबूल ही न किया जाएगा और वह आख़िरत में सख्त घाटे में रहेगा
\end{hindi}}
\flushright{\begin{Arabic}
\quranayah[3][86]
\end{Arabic}}
\flushleft{\begin{hindi}
भला ख़ुदा ऐसे लोगों की क्योंकर हिदायत करेगा जो इमाने लाने के बाद फिर काफ़िर हो गए हालॉकि वह इक़रार कर चुके थे कि पैग़म्बर (आख़िरूज़ज़मा) बरहक़ हैं और उनके पास वाजेए व रौशन मौजिज़े भी आ चुके थे और ख़ुदा ऐसी हठधर्मी करने वाले लोगों की हिदायत नहीं करता
\end{hindi}}
\flushright{\begin{Arabic}
\quranayah[3][87]
\end{Arabic}}
\flushleft{\begin{hindi}
ऐसे लोगों की सज़ा यह है कि उनपर ख़ुदा और फ़रिश्तों और (दुनिया जहॉन के) सब लोगों की फिटकार हैं
\end{hindi}}
\flushright{\begin{Arabic}
\quranayah[3][88]
\end{Arabic}}
\flushleft{\begin{hindi}
और वह हमेशा उसी फिटकार में रहेंगे न तो उनके अज़ाब ही में तख्फ़ीफ़ की जाएगी और न उनको मोहलत दी जाएगी
\end{hindi}}
\flushright{\begin{Arabic}
\quranayah[3][89]
\end{Arabic}}
\flushleft{\begin{hindi}
मगर (हॉ) जिन लोगों ने इसके बाद तौबा कर ली और अपनी (ख़राबी की) इस्लाह कर ली तो अलबत्ता ख़ुदा बड़ा बख्शने वाला मेहरबान है
\end{hindi}}
\flushright{\begin{Arabic}
\quranayah[3][90]
\end{Arabic}}
\flushleft{\begin{hindi}
जो अपने ईमान के बाद काफ़िर हो बैठे फ़िर (रोज़ बरोज़ अपना) कुफ़्र बढ़ाते चले गये तो उनकी तौबा हरगिज़ न कुबूल की जाएगी और यही लोग (पल्ले दरजे के) गुमराह हैं
\end{hindi}}
\flushright{\begin{Arabic}
\quranayah[3][91]
\end{Arabic}}
\flushleft{\begin{hindi}
बेशक जिन लोगों ने कुफ़्र इख्तियार किया और कुफ़्र की हालत में मर गये तो अगरचे इतना सोना भी किसी की गुलू ख़लासी (छुटकारा पाने) में दिया जाए कि ज़मीन भर जाए तो भी हरगिज़ न कुबूल किया जाएगा यही लोग हैं जिनके लिए दर्दनाक अज़ाब होगा और उनका कोई मददगार भी न होगा
\end{hindi}}
\flushright{\begin{Arabic}
\quranayah[3][92]
\end{Arabic}}
\flushleft{\begin{hindi}
(लोगों) जब तक तुम अपनी पसन्दीदा चीज़ों में से कुछ राहे ख़ुदा में ख़र्च न करोगे हरगिज़ नेकी के दरजे पर फ़ायज़ नहीं हो सकते और तुम कोई
\end{hindi}}
\flushright{\begin{Arabic}
\quranayah[3][93]
\end{Arabic}}
\flushleft{\begin{hindi}
सी चीज़ भी ख़र्च करो ख़ुदा तो उसको ज़रूर जानता है तौरैत के नाज़िल होने के क़ब्ल याकूब ने जो जो चीज़े अपने ऊपर हराम कर ली थीं उनके सिवा बनी इसराइल के लिए सब खाने हलाल थे (ऐ रसूल उन यहूद से) कह दो कि अगर तुम (अपने दावे में सच्चे हो तो तौरेत ले आओ
\end{hindi}}
\flushright{\begin{Arabic}
\quranayah[3][94]
\end{Arabic}}
\flushleft{\begin{hindi}
और उसको (हमारे सामने) पढ़ो फिर उसके बाद भी जो कोई ख़ुदा पर झूठ तूफ़ान जोड़े तो (समझ लो) कि यही लोग ज़ालिम (हठधर्म) हैं
\end{hindi}}
\flushright{\begin{Arabic}
\quranayah[3][95]
\end{Arabic}}
\flushleft{\begin{hindi}
(ऐ रसूल) कह दो कि ख़ुदा ने सच फ़रमाया तो अब तुम मिल्लते इबराहीम (इस्लाम) की पैरवी करो जो बातिल से कतरा के चलते थे और मुशरेकीन से न थे
\end{hindi}}
\flushright{\begin{Arabic}
\quranayah[3][96]
\end{Arabic}}
\flushleft{\begin{hindi}
लोगों (की इबादत) के वास्ते जो घर सबसे पहले बनाया गया वह तो यक़ीनन यही (काबा) है जो मक्के में है बड़ी (खैर व बरकत) वाला और सारे जहॉन के लोगों का रहनुमा
\end{hindi}}
\flushright{\begin{Arabic}
\quranayah[3][97]
\end{Arabic}}
\flushleft{\begin{hindi}
इसमें (हुरमत की) बहुत सी वाज़े और रौशन निशानियॉ हैं (उनमें से) मुक़ाम इबराहीम है (जहॉ आपके क़दमों का पत्थर पर निशान है) और जो इस घर में दाख़िल हुआ अमन में आ गया और लोगों पर वाजिब है कि महज़ ख़ुदा के लिए ख़ानाए काबा का हज करें जिन्हे वहां तक पहुँचने की इस्तेताअत है और जिसने बावजूद कुदरत हज से इन्कार किया तो (याद रखे) कि ख़ुदा सारे जहॉन से बेपरवा है
\end{hindi}}
\flushright{\begin{Arabic}
\quranayah[3][98]
\end{Arabic}}
\flushleft{\begin{hindi}
(ऐ रसूल) तुम कह दो कि ऐ अहले किताब खुदा की आयतो से क्यो मुन्किर हुए जाते हो हालॉकि जो काम काज तुम करते हो खुदा को उसकी (पूरी) पूरी इत्तिला है
\end{hindi}}
\flushright{\begin{Arabic}
\quranayah[3][99]
\end{Arabic}}
\flushleft{\begin{hindi}
(ऐ रसूल) तुम कह दो कि ऐ अहले किताब दीदए दानिस्ता खुदा की (सीधी) राह में (नाहक़ की) कज़ी ढूंढो (ढूंढ) के ईमान लाने वालों को उससे क्यों रोकते हो ओर जो कुछ तुम करते हो खुदा उससे बेख़बर नहीं है
\end{hindi}}
\flushright{\begin{Arabic}
\quranayah[3][100]
\end{Arabic}}
\flushleft{\begin{hindi}
ऐ ईमान वालों अगर तुमने अहले किताब के किसी फ़िरके क़ा भी कहना माना तो (याद रखो कि) वह तुमको ईमान लाने के बाद (भी) फिर दुबारा काफ़िर बना छोडेंग़े
\end{hindi}}
\flushright{\begin{Arabic}
\quranayah[3][101]
\end{Arabic}}
\flushleft{\begin{hindi}
और (भला) तुम क्योंकर काफ़िर बन जाओगे हालॉकि तुम्हारे सामने ख़ुदा की आयतें (बराबर) पढ़ी जाती हैं और उसके रसूल (मोहम्मद) भी तुममें (मौजूद) हैं और जो शख्स ख़ुदा से वाबस्ता हो वह (तो) जरूर सीधी राह पर लगा दिया गया
\end{hindi}}
\flushright{\begin{Arabic}
\quranayah[3][102]
\end{Arabic}}
\flushleft{\begin{hindi}
ऐ ईमान वालों ख़ुदा से डरो जितना उससे डरने का हक़ है और तुम (दीन) इस्लाम के सिवा किसी और दीन पर हरगिज़ न मरना
\end{hindi}}
\flushright{\begin{Arabic}
\quranayah[3][103]
\end{Arabic}}
\flushleft{\begin{hindi}
और तुम सब के सब (मिलकर) ख़ुदा की रस्सी मज़बूती से थामे रहो और आपस में (एक दूसरे) के फूट न डालो और अपने हाल (ज़ार) पर ख़ुदा के एहसान को तो याद करो जब तुम आपस में (एक दूसरे के) दुश्मन थे तो ख़ुदा ने तुम्हारे दिलों में (एक दूसरे की) उलफ़त पैदा कर दी तो तुम उसके फ़ज़ल से आपस में भाई भाई हो गए और तुम गोया सुलगती हुईआग की भट्टी (दोज़ख) के लब पर (खडे) थे गिरना ही चाहते थे कि ख़ुदा ने तुमको उससे बचा लिया तो ख़ुदा अपने एहकाम यूं वाजेए करके बयान करता है ताकि तुम राहे रास्त पर आ जाओ
\end{hindi}}
\flushright{\begin{Arabic}
\quranayah[3][104]
\end{Arabic}}
\flushleft{\begin{hindi}
और तुमसे एक गिरोह ऐसे (लोगों का भी) तो होना चाहिये जो (लोगों को) नेकी की तरफ़ बुलाए अच्छे काम का हुक्म दे और बुरे कामों से रोके और ऐसे ही लोग (आख़ेरत में) अपनी दिली मुरादें पायेंगे
\end{hindi}}
\flushright{\begin{Arabic}
\quranayah[3][105]
\end{Arabic}}
\flushleft{\begin{hindi}
और तुम (कहीं) उन लोगों के ऐसे न हो जाना जो आपस में फूट डाल कर बैठ रहे और रौशन (दलील) आने के बाद भी एक मुंह एक ज़बान न रहे और ऐसे ही लोगों के वास्ते बड़ा (भारी) अज़ाब है
\end{hindi}}
\flushright{\begin{Arabic}
\quranayah[3][106]
\end{Arabic}}
\flushleft{\begin{hindi}
(उस दिन से डरो) जिस दिन कुछ लोगों के चेहरे तो सफैद नूरानी होंगे और कुछ (लोगो) के चेहरे सियाह जिन लोगों के मुहॅ में कालिक होगी (उनसे कहा जायेगा) हाए क्यों तुम तो ईमान लाने के बाद काफ़िर हो गए थे अच्छा तो (लो) (अब) अपने कुफ़्र की सज़ा में अज़ाब (के मजे) चखो
\end{hindi}}
\flushright{\begin{Arabic}
\quranayah[3][107]
\end{Arabic}}
\flushleft{\begin{hindi}
और जिनके चेहरे पर नूर बरसता होगा वह तो ख़ुदा की रहमत(बहिश्त) में होंगे (और) उसी में सदा रहेंगे
\end{hindi}}
\flushright{\begin{Arabic}
\quranayah[3][108]
\end{Arabic}}
\flushleft{\begin{hindi}
(ऐ रसूल) ये ख़ुदा की आयतें हैं जो हम तुमको ठीक (ठीक) पढ़ के सुनाते हैं और ख़ुदा सारे जहॉन के लोगों (से किसी) पर जुल्म करना नहीं चाहता
\end{hindi}}
\flushright{\begin{Arabic}
\quranayah[3][109]
\end{Arabic}}
\flushleft{\begin{hindi}
और जो कुछ आसमानों में है और जो कुछ ज़मीन में है (सब) ख़ुदा ही का है और (आख़िर) सब कामों की रूज़ु ख़ुदा ही की तरफ़ है
\end{hindi}}
\flushright{\begin{Arabic}
\quranayah[3][110]
\end{Arabic}}
\flushleft{\begin{hindi}
तुम क्या अच्छे गिरोह हो कि (लोगों की) हिदायत के वास्ते पैदा किये गए हो तुम (लोगों को) अच्छे काम का हुक्म करते हो और बुरे कामों से रोकते हो और ख़ुदा पर ईमान रखते हो और अगर अहले किताब भी (इसी तरह) ईमान लाते तो उनके हक़ में बहुत अच्छा होता उनमें से कुछ ही तो ईमानदार हैं और अक्सर बदकार
\end{hindi}}
\flushright{\begin{Arabic}
\quranayah[3][111]
\end{Arabic}}
\flushleft{\begin{hindi}
(मुसलमानों) ये लोग मामूली अज़ीयत के सिवा तुम्हें हरगिज़ ज़रर नहीं पहुंचा सकते और अगर तुमसे लड़ेंगे तो उन्हें तुम्हारी तरफ़ पीठ ही करनी होगी और फिर उनकी कहीं से मदद भी नहीं पहुंचेगी
\end{hindi}}
\flushright{\begin{Arabic}
\quranayah[3][112]
\end{Arabic}}
\flushleft{\begin{hindi}
और जहॉ कहीं हत्ते चढ़े उनपर रूसवाई की मार पड़ी मगर ख़ुदा के एहद (या) और लोगों के एहद के ज़रिये से (उनको कहीं पनाह मिल गयी) और फिर हेरफेर के खुदा के गज़ब में पड़ गए और उनपर मोहताजी की मार (अलग) पड़ी ये (क्यों) इस सबब से कि वह ख़ुदा की आयतों से इन्कार करते थे और पैग़म्बरों को नाहक़ क़त्ल करते थे ये सज़ा उसकी है कि उन्होंने नाफ़रमानी की और हद से गुज़र गए थे
\end{hindi}}
\flushright{\begin{Arabic}
\quranayah[3][113]
\end{Arabic}}
\flushleft{\begin{hindi}
और ये लोग भी सबके सब यकसॉ नहीं हैं (बल्कि) अहले किताब से कुछ लोग तो ऐसे हैं कि (ख़ुदा के दीन पर) इस तरह साबित क़दम हैं कि रातों को ख़ुदा की आयतें पढ़ा करते हैं और वह बराबर सजदे किया करते हैं
\end{hindi}}
\flushright{\begin{Arabic}
\quranayah[3][114]
\end{Arabic}}
\flushleft{\begin{hindi}
खुदा और रोज़े आख़ेरत पर ईमान रखते हैं और अच्छे काम का तो हुक्म करते हैं और बुरे कामों से रोकते हैं और नेक कामों में दौड़ पड़ते हैं और यही लोग तो नेक बन्दों से हैं
\end{hindi}}
\flushright{\begin{Arabic}
\quranayah[3][115]
\end{Arabic}}
\flushleft{\begin{hindi}
और वह जो कुछ नेकी करेंगे उसकी हरगिज़ नाक़द्री न की जाएगी और ख़ुदा परहेज़गारों से खूब वाक़िफ़ है
\end{hindi}}
\flushright{\begin{Arabic}
\quranayah[3][116]
\end{Arabic}}
\flushleft{\begin{hindi}
बेशक जिन लोगों ने कुफ़्र इख्तेयार किया ख़ुदा (के अज़ाब) से बचाने में हरगिज़ न उनके माल ही कुछ काम आएंगे न उनकी औलाद और यही लोग जहन्नुमी हैं और हमेशा उसी में रहेंगे
\end{hindi}}
\flushright{\begin{Arabic}
\quranayah[3][117]
\end{Arabic}}
\flushleft{\begin{hindi}
दुनिया की चन्द रोज़ा ज़िन्दगी में ये लोग जो कुछ (ख़िलाफ़ शरा) ख़र्च करते हैं उसकी मिसाल अन्धड़ की मिसाल है जिसमें बहुत पाला हो और वह उन लोगों के खेत पर जा पड़े जिन्होंने (कुफ़्र की वजह से) अपनी जानों पर सितम ढाया हो और फिर पाला उसे मार के (नास कर दे) और ख़ुदा ने उनपर जुल्म कुछ नहीं किया बल्कि उन्होंने आप अपने ऊपर जुल्म किया
\end{hindi}}
\flushright{\begin{Arabic}
\quranayah[3][118]
\end{Arabic}}
\flushleft{\begin{hindi}
ऐ ईमानदारों अपने (मोमिनीन) के सिवा (गैरो को) अपना राज़दार न बनाओ (क्योंकि) ये गैर लोग तुम्हारी बरबादी में कुछ उठा नहीं रखेंगे (बल्कि जितना ज्यादा तकलीफ़) में पड़ोगे उतना ही ये लोग ख़ुश होंगे दुश्मनी तो उनके मुंह से टपकती है और जो (बुग़ज़ व हसद) उनके दिलों में भरा है वह कहीं उससे बढ़कर है हमने तुमसे (अपने) एहकाम साफ़ साफ़ बयान कर दिये अगर तुम समझ रखते हो
\end{hindi}}
\flushright{\begin{Arabic}
\quranayah[3][119]
\end{Arabic}}
\flushleft{\begin{hindi}
ऐ लोगों तुम ऐसे (सीधे) हो कि तुम उनसे उलफ़त रखतो हो और वह तुम्हें (ज़रा भी) नहीं चाहते और तुम तो पूरी किताब (ख़ुदा) पर ईमान रखते हो और वह ऐसे नहीं हैं (मगर) जब तुमसे मिलते हैं तो कहने लगते हैं कि हम भी ईमान लाए और जब अकेले में होते हैं तो तुम पर गुस्से के मारे उंगलियों काटते हैं (ऐ रसूल) तुम कह दो कि (काटना क्या) तुम अपने गुस्से में जल मरो जो बातें तुम्हारे दिलों में हैं बेशक ख़ुदा ज़रूर जानता है
\end{hindi}}
\flushright{\begin{Arabic}
\quranayah[3][120]
\end{Arabic}}
\flushleft{\begin{hindi}
(ऐ ईमानदारों) अगर तुमको भलाई छू भी गयी तो उनको बुरा मालूम होता है और जब तुमपर कोई भी मुसीबत पड़ती है तो वह ख़ुश हो जाते हैं और अगर तुम सब्र करो और परहेज़गारी इख्तेयार करो तो उनका फ़रेब तुम्हें कुछ भी ज़रर नहीं पहुंचाएगा (क्योंकि) ख़ुदा तो उनकी कारस्तानियों पर हावी है
\end{hindi}}
\flushright{\begin{Arabic}
\quranayah[3][121]
\end{Arabic}}
\flushleft{\begin{hindi}
और (ऐ रसूल) एक वक्त वो भी था जब तुम अपने बाल बच्चों से तड़के ही निकल खड़े हुए और मोमिनीन को लड़ाई के मोर्चों पर बिठा रहे थे और खुदा सब कुछ जानता और सुनता है
\end{hindi}}
\flushright{\begin{Arabic}
\quranayah[3][122]
\end{Arabic}}
\flushleft{\begin{hindi}
ये उस वक्त क़ा वाक़या है जब तुममें से दो गिरोहों ने ठान लिया था कि पसपाई करें और फिर (सॅभल गए) क्योंकि ख़ुदा तो उनका सरपरस्त था और मोमिनीन को ख़ुदा ही पर भरोसा रखना चाहिये
\end{hindi}}
\flushright{\begin{Arabic}
\quranayah[3][123]
\end{Arabic}}
\flushleft{\begin{hindi}
यक़ीनन ख़ुदा ने जंगे बदर में तुम्हारी मदद की (बावजूद के) तुम (दुश्मन के मुक़ाबले में) बिल्कुल बे हक़ीक़त थे (फिर भी) ख़ुदा ने फतेह दी
\end{hindi}}
\flushright{\begin{Arabic}
\quranayah[3][124]
\end{Arabic}}
\flushleft{\begin{hindi}
पस तुम ख़ुदा से डरते रहो ताकि (उनके) शुक्रगुज़ार बनो (ऐ रसूल) उस वक्त तुम मोमिनीन से कह रहे थे कि क्या तुम्हारे लिए काफ़ी नहीं है कि तुम्हारा परवरदिगार तीन हज़ार फ़रिश्ते आसमान से भेजकर तुम्हारी मदद करे हॉ (ज़रूर काफ़ी है)
\end{hindi}}
\flushright{\begin{Arabic}
\quranayah[3][125]
\end{Arabic}}
\flushleft{\begin{hindi}
बल्कि अगर तुम साबित क़दम रहो और (रसूल की मुख़ालेफ़त से) बचो और कुफ्फ़ार अपने (जोश में) तुमपर चढ़ भी आये तो तुम्हारा परवरदिगार ऐसे पॉच हज़ार फ़रिश्तों से तुम्हारी मदद करेगा जो निशाने जंग लगाए हुए डटे होंगे और ख़ुदा ने ये मदद सिर्फ तुम्हारी ख़ुशी के लिए की है
\end{hindi}}
\flushright{\begin{Arabic}
\quranayah[3][126]
\end{Arabic}}
\flushleft{\begin{hindi}
और ताकि इससे तुम्हारे दिल की ढारस हो और (ये तो ज़ाहिर है कि) मदद जब होती है तो ख़ुदा ही की तरफ़ से जो सब पर ग़ालिब (और) हिकमत वाला है
\end{hindi}}
\flushright{\begin{Arabic}
\quranayah[3][127]
\end{Arabic}}
\flushleft{\begin{hindi}
(और यह मदद की भी तो) इसलिए कि काफ़िरों के एक गिरोह को कम कर दे या ऐसा चौपट कर दे कि (अपना सा) मुंह लेकर नामुराद अपने घर वापस चले जायें
\end{hindi}}
\flushright{\begin{Arabic}
\quranayah[3][128]
\end{Arabic}}
\flushleft{\begin{hindi}
(ऐ रसूल) तुम्हारा तो इसमें कुछ बस नहीं चाहे ख़ुदा उनकी तौबा कुबूल करे या उनको सज़ा दे क्योंकि वह ज़ालिम तो ज़रूर हैं
\end{hindi}}
\flushright{\begin{Arabic}
\quranayah[3][129]
\end{Arabic}}
\flushleft{\begin{hindi}
और जो कुछ आसमानों में है और जो कुछ ज़मीन में है सब ख़ुदा ही का है जिसको चाहे बख्शे और जिसको चाहे सज़ा करे और ख़ुदा बड़ा बख्शने वाला मेहरबार है
\end{hindi}}
\flushright{\begin{Arabic}
\quranayah[3][130]
\end{Arabic}}
\flushleft{\begin{hindi}
ऐ ईमानदारों सूद दनादन खाते न चले जाओ और ख़ुदा से डरो कि तुम छुटकारा पाओ
\end{hindi}}
\flushright{\begin{Arabic}
\quranayah[3][131]
\end{Arabic}}
\flushleft{\begin{hindi}
और जहन्नुम की उस आग से डरो जो काफ़िरों के लिए तैयार की गयी है
\end{hindi}}
\flushright{\begin{Arabic}
\quranayah[3][132]
\end{Arabic}}
\flushleft{\begin{hindi}
और ख़ुदा और रसूल की फ़रमाबरदारी करो ताकि तुम पर रहम किया जाए
\end{hindi}}
\flushright{\begin{Arabic}
\quranayah[3][133]
\end{Arabic}}
\flushleft{\begin{hindi}
और अपने परवरदिगार के (सबब) बख्शिश और जन्नत की तरफ़ दौड़ पड़ो जिसकी (वुसअत सारे) आसमान और ज़मीन के बराबर है और जो परहेज़गारों के लिये मुहय्या की गयी है
\end{hindi}}
\flushright{\begin{Arabic}
\quranayah[3][134]
\end{Arabic}}
\flushleft{\begin{hindi}
जो ख़ुशहाली और कठिन वक्त में भी (ख़ुदा की राह पर) ख़र्च करते हैं और गुस्से को रोकते हैं और लोगों (की ख़ता) से दरगुज़र करते हैं और नेकी करने वालों से अल्लाह उलफ़त रखता है
\end{hindi}}
\flushright{\begin{Arabic}
\quranayah[3][135]
\end{Arabic}}
\flushleft{\begin{hindi}
और लोग इत्तिफ़ाक़ से कोई बदकारी कर बैठते हैं या आप अपने ऊपर जुल्म करते हैं तो ख़ुदा को याद करते हैं और अपने गुनाहों की माफ़ी मॉगते हैं और ख़ुदा के सिवा गुनाहों का बख्शने वाला और कौन है और जो (क़ूसूर) वह (नागहानी) कर बैठे तो जानबूझ कर उसपर हट नहीं करते
\end{hindi}}
\flushright{\begin{Arabic}
\quranayah[3][136]
\end{Arabic}}
\flushleft{\begin{hindi}
ऐसे ही लोगों की जज़ा उनके परवरदिगार की तरफ़ से बख्शिश है और वह बाग़ात हैं जिनके नीचे नहरें जारी हैं कि वह उनमें हमेशा रहेंगे और (अच्छे) चलन वालों की (भी) ख़ूब खरी मज़दूरी है
\end{hindi}}
\flushright{\begin{Arabic}
\quranayah[3][137]
\end{Arabic}}
\flushleft{\begin{hindi}
तुमसे पहले बहुत से वाक़यात गुज़र चुके हैं पस ज़रा रूए ज़मीन पर चल फिर कर देखो तो कि (अपने अपने वक्त क़े पैग़म्बरों को) झुठलाने वालों का अन्जाम क्या हुआ
\end{hindi}}
\flushright{\begin{Arabic}
\quranayah[3][138]
\end{Arabic}}
\flushleft{\begin{hindi}
ये (जो हमने कहा) आम लोगों के लिए तो सिर्फ़ बयान (वाक़या) है मगर और परहेज़गारों के लिए हिदायत व नसीहत है
\end{hindi}}
\flushright{\begin{Arabic}
\quranayah[3][139]
\end{Arabic}}
\flushleft{\begin{hindi}
और मुसलमानों काहिली न करो और (इस) इत्तफ़ाक़ी शिकस्त (ओहद से) कुढ़ो नहीं (क्योंकि) अगर तुम सच्चे मोमिन हो तो तुम ही ग़ालिब और वर रहोगे
\end{hindi}}
\flushright{\begin{Arabic}
\quranayah[3][140]
\end{Arabic}}
\flushleft{\begin{hindi}
अगर (जंगे ओहद में) तुमको ज़ख्म लगा है तो उसी तरह (बदर में) तुम्हारे फ़रीक़ (कुफ्फ़ार को) भी ज़ख्म लग चुका है (उस पर उनकी हिम्मत तो न टूटी) ये इत्तफ़ाक़ाते ज़माना हैं जो हम लोगों के दरमियान बारी बारी उलट फेर किया करते हैं और ये (इत्तफ़ाक़ी शिकस्त इसलिए थी) ताकि ख़ुदा सच्चे ईमानदारों को (ज़ाहिरी) मुसलमानों से अलग देख लें और तुममें से बाज़ को दरजाए शहादत पर फ़ायज़ करें और ख़ुदा (हुक्मे रसूल से) सरताबी करने वालों को दोस्त नहीं रखता
\end{hindi}}
\flushright{\begin{Arabic}
\quranayah[3][141]
\end{Arabic}}
\flushleft{\begin{hindi}
और ये (भी मंजूर था) कि सच्चे ईमानदारों को (साबित क़दमी की वजह से) निरा खरा अलग कर ले और नाफ़रमानों (भागने वालों) का मटियामेट कर दे
\end{hindi}}
\flushright{\begin{Arabic}
\quranayah[3][142]
\end{Arabic}}
\flushleft{\begin{hindi}
(मुसलमानों) क्या तुम ये समझते हो कि सब के सब बहिश्त में चले ही जाओगे और क्या ख़ुदा ने अभी तक तुममें से उन लोगों को भी नहीं पहचाना जिन्होंने जेहाद किया और न साबित क़दम रहने वालों को ही पहचाना
\end{hindi}}
\flushright{\begin{Arabic}
\quranayah[3][143]
\end{Arabic}}
\flushleft{\begin{hindi}
तुम तो मौत के आने से पहले (लड़ाई में) मरने की तमन्ना करते थे बस अब तुमने उसको अपनी ऑख से देख लिया और तुम अब भी देख रहे हो
\end{hindi}}
\flushright{\begin{Arabic}
\quranayah[3][144]
\end{Arabic}}
\flushleft{\begin{hindi}
(फिर लड़ाई से जी क्यों चुराते हो) और मोहम्मद तो सिर्फ रसूल हैं (ख़ुदा नहीं) इनसे पहले बहुतेरे पैग़म्बर गुज़र चुके हैं फिर क्या अगर मोहम्मद अपनी मौत से मर जॉए या मार डाले जाएं तो तुम उलटे पॉव (अपने कुफ़्र की तरफ़) पलट जाओगे और जो उलटे पॉव फिरेगा (भी) तो (समझ लो) हरगिज़ ख़ुदा का कुछ भी नहीं बिगड़ेगा और अनक़रीब ख़ुदा का शुक्र करने वालों को अच्छा बदला देगा
\end{hindi}}
\flushright{\begin{Arabic}
\quranayah[3][145]
\end{Arabic}}
\flushleft{\begin{hindi}
और बगैर हुक्मे ख़ुदा के तो कोई शख्स मर ही नहीं सकता वक्ते मुअय्यन तक हर एक की मौत लिखी हुई है और जो शख्स (अपने किए का) बदला दुनिया में चाहे तो हम उसको इसमें से दे देते हैं और जो शख्स आख़ेरत का बदला चाहे उसे उसी में से देंगे और (नेअमत ईमान के) शुक्र करने वालों को बहुत जल्द हम जज़ाए खैर देंगे
\end{hindi}}
\flushright{\begin{Arabic}
\quranayah[3][146]
\end{Arabic}}
\flushleft{\begin{hindi}
और (मुसलमानों तुम ही नहीं) ऐसे पैग़म्बर बहुत से गुज़र चुके हैं जिनके साथ बहुतेरे अल्लाह वालों ने (राहे खुदा में) जेहाद किया और फिर उनको ख़ुदा की राह में जो मुसीबत पड़ी है न तो उन्होंने हिम्मत हारी न बोदापन किया (और न दुशमन के सामने) गिड़गिड़ाने लगे और साबित क़दम रहने वालों से ख़ुदा उलफ़त रखता है
\end{hindi}}
\flushright{\begin{Arabic}
\quranayah[3][147]
\end{Arabic}}
\flushleft{\begin{hindi}
और लुत्फ़ ये है कि उनका क़ौल इसके सिवा कुछ न था कि दुआएं मॉगने लगें कि ऐ हमारे पालने वाले हमारे गुनाह और अपने कामों में हमारी ज्यादतियॉ माफ़ कर और दुश्मनों के मुक़ाबले में हमको साबित क़दम रख और काफ़िरों के गिरोह पर हमको फ़तेह दे
\end{hindi}}
\flushright{\begin{Arabic}
\quranayah[3][148]
\end{Arabic}}
\flushleft{\begin{hindi}
तो ख़ुदा ने उनको दुनिया में बदला (दिया) और आख़िरत में अच्छा बदला ईनायत फ़रमाया और ख़ुदा नेकी करने वालों को दोस्त रखता (ही) है
\end{hindi}}
\flushright{\begin{Arabic}
\quranayah[3][149]
\end{Arabic}}
\flushleft{\begin{hindi}
ऐ ईमानदारों अगर तुम लोगों ने काफ़िरों की पैरवी कर ली तो (याद रखो) वह तुमको उलटे पॉव (कुफ़्र की तरफ़) फेर कर ले जाऐंगे फिर उलटे तुम ही घाटे में आ जाओगे
\end{hindi}}
\flushright{\begin{Arabic}
\quranayah[3][150]
\end{Arabic}}
\flushleft{\begin{hindi}
(तुम किसी की मदद के मोहताज नहीं) बल्कि ख़ुदा तुम्हारा सरपरस्त है और वह सब मददगारों से बेहतर है
\end{hindi}}
\flushright{\begin{Arabic}
\quranayah[3][151]
\end{Arabic}}
\flushleft{\begin{hindi}
(तुम घबराओ नहीं) हम अनक़रीब तुम्हारा रोब काफ़िरों के दिलों में जमा देंगे इसलिए कि उन लोगों ने ख़ुदा का शरीक बनाया (भी तो) इस चीज़ बुत को जिसे ख़ुदा ने किसी क़िस्म की हुकूमत नहीं दी और (आख़िरकार) उनका ठिकाना दौज़ख़ है और ज़ालिमों का (भी क्या) बुरा ठिकाना है
\end{hindi}}
\flushright{\begin{Arabic}
\quranayah[3][152]
\end{Arabic}}
\flushleft{\begin{hindi}
बेशक खुदा ने (जंगे औहद में भी) अपना (फतेह का) वायदा सच्चा कर दिखाया था जब तुम उसके हुक्म से (पहले ही हमले में) उन (कुफ्फ़ार) को खूब क़त्ल कर रहे थे यहॉ तक की तुम्हारे पसन्द की चीज़ (फ़तेह) तुम्हें दिखा दी उसके बाद भी तुमने (माले ग़नीमत देखकर) बुज़दिलापन किया और हुक्में रसूल (मोर्चे पर जमे रहने) झगड़ा किया और रसूल की नाफ़रमानी की तुममें से कुछ तो तालिबे दुनिया हैं (कि माले ग़नीमत की तरफ़) से झुक पड़े और कुछ तालिबे आख़िरत (कि रसूल पर अपनी जान फ़िदा कर दी) फिर (बुज़दिलेपन ने) तुम्हें उन (कुफ्फ़ार) की की तरफ से फेर दिया (और तुम भाग खड़े हुए) उससे ख़ुदा को तुम्हारा (ईमान अख़लासी) आज़माना मंज़ूर था और (उसपर भी) ख़ुदा ने तुमसे दरगुज़र की और खुदा मोमिनीन पर बड़ा फ़ज़ल करने वाला है
\end{hindi}}
\flushright{\begin{Arabic}
\quranayah[3][153]
\end{Arabic}}
\flushleft{\begin{hindi}
(मुसलमानों तुम) उस वक्त क़ो याद करके शर्माओ जब तुम (बदहवास) भागे पहाड़ पर चले जाते थे पस (चूंकि) रसूल को तुमने (आज़ारदा) किया ख़ुदा ने भी तुमको (इस) रंज की सज़ा में (शिकस्त का) रंज दिया ताकि जब कभी तुम्हारी कोई चीज़ हाथ से जाती रहे या कोई मुसीबत पड़े तो तुम रंज न करो और सब्र करना सीखो और जो कुछ तुम करते हो ख़ुदा उससे ख़बरदार है
\end{hindi}}
\flushright{\begin{Arabic}
\quranayah[3][154]
\end{Arabic}}
\flushleft{\begin{hindi}
फिर ख़ुदा ने इस रंज के बाद तुमपर इत्मिनान की हालत तारी की कि तुममें से एक गिरोह का (जो सच्चे ईमानदार थे) ख़ूब गहरी नींद आ गयी और एक गिरोह जिनको उस वक्त भी (भागने की शर्म से) जान के लाले पड़े थे ख़ुदा के साथ (ख्वाह मख्वाह) ज़मानाए जिहालत की ऐसी बदगुमानियॉ करने लगे और कहने लगे भला क्या ये अम्र (फ़तेह) कुछ भी हमारे इख्तियार में है (ऐ रसूल) कह दो कि हर अम्र का इख्तियार ख़ुदा ही को है (ज़बान से तो कहते ही है नहीं) ये अपने दिलों में ऐसी बातें छिपाए हुए हैं जो तुमसे ज़ाहिर नहीं करते (अब सुनो) कहते हैं कि इस अम्र (फ़तेह) में हमारा कुछ इख़्तियार होता तो हम यहॉ मारे न जाते (ऐ रसूल उनसे) कह दो कि तुम अपने घरों में रहते तो जिन जिन की तकदीर में लड़के मर जाना लिखा था वह अपने (घरो से) निकल निकल के अपने मरने की जगह ज़रूर आ जाते और (ये इस वास्ते किया गया) ताकि जो कुछ तुम्हारे दिल में है उसका इम्तिहान कर दे और ख़ुदा तो दिलों के राज़ खूब जानता है
\end{hindi}}
\flushright{\begin{Arabic}
\quranayah[3][155]
\end{Arabic}}
\flushleft{\begin{hindi}
बेशक जिस दिन (जंगे औहद में) दो जमाअतें आपस में गुथ गयीं उस दिन जो लोग तुम (मुसलमानों) में से भाग खड़े हुए (उसकी वजह ये थी कि) उनके बाज़ गुनाहों (मुख़ालफ़ते रसूल) की वजह से शैतान ने बहका के उनके पॉव उखाड़ दिए और (उसी वक्त तो) ख़ुदा ने ज़रूर उनसे दरगुज़र की बेशक ख़ुदा बड़ा बख्शने वाला बुर्दवार है
\end{hindi}}
\flushright{\begin{Arabic}
\quranayah[3][156]
\end{Arabic}}
\flushleft{\begin{hindi}
ऐ ईमानदारों उन लोगों के ऐसे न बनो जो काफ़िर हो गए भाई बन्द उनके परदेस में निकले हैं या जेहाद करने गए हैं (और वहॉ) मर (गए) तो उनके बारे में कहने लगे कि वह हमारे पास रहते तो न मरते ओर न मारे जाते (और ये इस वजह से कहते हैं) ताकि ख़ुदा (इस ख्याल को) उनके दिलों में (बाइसे) हसरत बना दे और (यूं तो) ख़ुदा ही जिलाता और मारता है और जो कुछ तुम करते हो ख़ुदा उसे देख रहा है
\end{hindi}}
\flushright{\begin{Arabic}
\quranayah[3][157]
\end{Arabic}}
\flushleft{\begin{hindi}
और अगर तुम ख़ुदा की राह में मारे जाओ या (अपनी मौत से) मर जाओ तो बेशक ख़ुदा की बख्शिश और रहमत इस (माल व दौलत) से जिसको तुम जमा करते हो ज़रूर बेहतर है
\end{hindi}}
\flushright{\begin{Arabic}
\quranayah[3][158]
\end{Arabic}}
\flushleft{\begin{hindi}
और अगर तुम (अपनी मौत से) मरो या मारे जाओ (आख़िरकार) ख़ुदा ही की तरफ़ (क़ब्रों से) उठाए जाओगे
\end{hindi}}
\flushright{\begin{Arabic}
\quranayah[3][159]
\end{Arabic}}
\flushleft{\begin{hindi}
(तो ऐ रसूल ये भी) ख़ुदा की एक मेहरबानी है कि तुम (सा) नरमदिल (सरदार) उनको मिला और तुम अगर बदमिज़ाज और सख्त दिल होते तब तो ये लोग (ख़ुदा जाने कब के) तुम्हारे गिर्द से तितर बितर हो गए होते पस (अब भी) तुम उनसे दरगुज़र करो और उनके लिए मग़फेरत की दुआ मॉगो और (साबिक़ दस्तूरे ज़ाहिरा) उनसे काम काज में मशवरा कर लिया करो (मगर) इस पर भी जब किसी काम को ठान लो तो ख़ुदा ही पर भरोसा रखो (क्योंकि जो लोग ख़ुदा पर भरोसा रखते हैं ख़ुदा उनको ज़रूर दोस्त रखता है
\end{hindi}}
\flushright{\begin{Arabic}
\quranayah[3][160]
\end{Arabic}}
\flushleft{\begin{hindi}
(मुसलमानों याद रखो) अगर ख़ुदा ने तुम्हारी मदद की तो फिर कोई तुमपर ग़ालिब नहीं आ सकता और अगर ख़ुदा तुमको छोड़ दे तो फिर कौन ऐसा है जो उसके बाद तुम्हारी मदद को खड़ा हो और मोमिनीन को चाहिये कि ख़ुदा ही पर भरोसा रखें
\end{hindi}}
\flushright{\begin{Arabic}
\quranayah[3][161]
\end{Arabic}}
\flushleft{\begin{hindi}
और (तुम्हारा गुमान बिल्कुल ग़लत है) किसी नबी की (हरगिज़) ये शान नहीं कि ख्यानत करे और ख्यानत करेगा तो जो चीज़ ख्यानत की है क़यामत के दिन वही चीज़ (बिलकुल वैसा ही) ख़ुदा के सामने लाना होगा फिर हर शख्स अपने किए का पूरा पूरा बदला पाएगा और उनकी किसी तरह हक़तल्फ़ी नहीं की जाएगी
\end{hindi}}
\flushright{\begin{Arabic}
\quranayah[3][162]
\end{Arabic}}
\flushleft{\begin{hindi}
भला जो शख्स ख़ुदा की ख़ुशनूदी का पाबन्द हो क्या वह उस शख्स के बराबर हो सकता है जो ख़ुदा के गज़ब में गिरफ्तार हो और जिसका ठिकाना जहन्नुम है और वह क्या बुरा ठिकाना है
\end{hindi}}
\flushright{\begin{Arabic}
\quranayah[3][163]
\end{Arabic}}
\flushleft{\begin{hindi}
वह लोग खुदा के यहॉ मुख्तलिफ़ दरजों के हैं और जो कुछ वह करते हैं ख़ुदा देख रहा है
\end{hindi}}
\flushright{\begin{Arabic}
\quranayah[3][164]
\end{Arabic}}
\flushleft{\begin{hindi}
ख़ुदा ने तो ईमानदारों पर बड़ा एहसान किया कि उनके वास्ते उन्हीं की क़ौम का एक रसूल भेजा जो उन्हें खुदा की आयतें पढ़ पढ़ के सुनाता है और उनकी तबीयत को पाकीज़ा करता है और उन्हें किताबे (ख़ुदा) और अक्ल की बातें सिखाता है अगरचे वह पहले खुली हुई गुमराही में पडे थे
\end{hindi}}
\flushright{\begin{Arabic}
\quranayah[3][165]
\end{Arabic}}
\flushleft{\begin{hindi}
मुसलमानों क्या जब तुमपर (जंगे ओहद) में वह मुसीबत पड़ी जिसकी दूनी मुसीबत तुम (कुफ्फ़ार पर) डाल चुके थे तो (घबरा के) कहने लगे ये (आफ़त) कहॉ से आ गयी (ऐ रसूल) तुम कह दो कि ये तो खुद तुम्हारी ही तरफ़ से है (न रसूल की मुख़ालेफ़त करते न सज़ा होती) बेशक ख़ुदा हर चीज़ पर क़ादिर है
\end{hindi}}
\flushright{\begin{Arabic}
\quranayah[3][166]
\end{Arabic}}
\flushleft{\begin{hindi}
और जिस दिन दो जमाअतें आपस में गुंथ गयीं उस दिन तुम पर जो मुसीबत पड़ी वह तुम्हारी शरारत की वजह से (ख़ुदा के इजाजत की वजह से आयी) और ताकि ख़ुदा सच्चे ईमान वालों को देख ले
\end{hindi}}
\flushright{\begin{Arabic}
\quranayah[3][167]
\end{Arabic}}
\flushleft{\begin{hindi}
और मुनाफ़िक़ों को देख ले (कि कौन है) और मुनाफ़िक़ों से कहा गया कि आओ ख़ुदा की राह में जेहाद करो या (ये न सही अपने दुशमन को) हटा दो तो कहने लगे (हाए क्या कहीं) अगर हम लड़ना जानते तो ज़रूर तुम्हारा साथ देते ये लोग उस दिन बनिस्बते ईमान के कुफ़्र के ज्यादा क़रीब थे अपने मुंह से वह बातें कह देते हैं जो उनके दिल में (ख़ाक) नहीं होतीं और जिसे वह छिपाते हैं ख़ुदा उसे ख़ूब जानता है
\end{hindi}}
\flushright{\begin{Arabic}
\quranayah[3][168]
\end{Arabic}}
\flushleft{\begin{hindi}
(ये वही लोग हैं) जो (आप चैन से घरों में बैठे रहते है और अपने शहीद) भाईयों के बारे में कहने लगे काश हमारी पैरवी करते तो न मारे जाते (ऐ रसूल) उनसे कहो (अच्छा) अगर तुम सच्चे हो तो ज़रा अपनी जान से मौत को टाल दो
\end{hindi}}
\flushright{\begin{Arabic}
\quranayah[3][169]
\end{Arabic}}
\flushleft{\begin{hindi}
और जो लोग ख़ुदा की राह में शहीद किए गए उन्हें हरगिज़ मुर्दा न समझना बल्कि वह लोग जीते जागते मौजूद हैं अपने परवरदिगार की तरफ़ से वह (तरह तरह की) रोज़ी पाते हैं
\end{hindi}}
\flushright{\begin{Arabic}
\quranayah[3][170]
\end{Arabic}}
\flushleft{\begin{hindi}
और ख़ुदा ने जो फ़ज़ल व करम उन पर किया है उसकी (ख़ुशी) से फूले नहीं समाते और जो लोग उनसे पीछे रह गए और उनमें आकर शामिल नहीं हुए उनकी निस्बत ये (ख्याल करके) ख़ुशियॉ मनाते हैं कि (ये भी शहीद हों तो) उनपर न किसी क़िस्म का ख़ौफ़ होगा और न आज़ुर्दा ख़ातिर होंगे
\end{hindi}}
\flushright{\begin{Arabic}
\quranayah[3][171]
\end{Arabic}}
\flushleft{\begin{hindi}
ख़ुदा नेअमत और उसके फ़ज़ल (व करम) और इस बात की ख़ुशख़बरी पाकर कि ख़ुदा मोमिनीन के सवाब को बरबाद नहीं करता
\end{hindi}}
\flushright{\begin{Arabic}
\quranayah[3][172]
\end{Arabic}}
\flushleft{\begin{hindi}
निहाल हो रहे हैं (जंगे ओहद में) जिन लोगों ने जख्म खाने के बाद भी ख़ुदा और रसूल का कहना माना उनमें से जिन लोगों ने नेकी और परहेज़गारी की (सब के लिये नहीं सिर्फ) उनके लिये बड़ा सवाब है
\end{hindi}}
\flushright{\begin{Arabic}
\quranayah[3][173]
\end{Arabic}}
\flushleft{\begin{hindi}
यह वह हैं कि जब उनसे लोगों ने आकर कहना शुरू किया कि (दुशमन) लोगों ने तुम्हारे (मुक़ाबले के) वास्ते (बड़ा लश्कर) जमा किया है पस उनसे डरते (तो बजाए ख़ौफ़ के) उनका ईमान और ज्यादा हो गया और कहने लगे (होगा भी) ख़ुदा हमारे वास्ते काफ़ी है
\end{hindi}}
\flushright{\begin{Arabic}
\quranayah[3][174]
\end{Arabic}}
\flushleft{\begin{hindi}
और वह क्या अच्छा कारसाज़ है फिर (या तो हिम्मत करके गए मगर जब लड़ाई न हुई तो) ये लोग ख़ुदा की नेअमत और फ़ज़ल के साथ (अपने घर) वापस आए और उन्हें कोई बुराई छू भी नहीं गयी और ख़ुदा की ख़ुशनूदी के पाबन्द रहे और ख़ुदा बड़ा फ़ज़ल करने वाला है
\end{hindi}}
\flushright{\begin{Arabic}
\quranayah[3][175]
\end{Arabic}}
\flushleft{\begin{hindi}
यह (मुख्बिर) बस शैतान था जो सिर्फ़ अपने दोस्तों को (रसूल का साथ देने से) डराता है पस तुम उनसे तो डरो नहीं अगर सच्चे मोमिन हो तो मुझ ही से डरते रहो
\end{hindi}}
\flushright{\begin{Arabic}
\quranayah[3][176]
\end{Arabic}}
\flushleft{\begin{hindi}
और (ऐ रसूल) जो लोग कुफ़्र की (मदद) में पेश क़दमी कर जाते हैं उनकी वजह से तुम रन्ज न करो क्योंकि ये लोग ख़ुदा को कुछ ज़रर नहीं पहुँचा सकते (बल्कि) ख़ुदा तो ये चाहता है कि आख़ेरत में उनका हिस्सा न क़रार दे और उनके लिए बड़ा (सख्त) अज़ाब है
\end{hindi}}
\flushright{\begin{Arabic}
\quranayah[3][177]
\end{Arabic}}
\flushleft{\begin{hindi}
बेशक जिन लोगों ने ईमान के एवज़ कुफ़्र ख़रीद किया वह हरगिज़ खुदा का कुछ भी नहीं बिगाड़ेंगे (बल्कि आप अपना) और उनके लिए दर्दनाक अज़ाब है
\end{hindi}}
\flushright{\begin{Arabic}
\quranayah[3][178]
\end{Arabic}}
\flushleft{\begin{hindi}
और जिन लोगों ने कुफ़्र इख्तियार किया वह हरगिज़ ये न ख्याल करें कि हमने जो उनको मोहलत व बेफिक्री दे रखी है वह उनके हक़ में बेहतर है (हालॉकि) हमने मोहल्लत व बेफिक्री सिर्फ इस वजह से दी है ताकि वह और ख़ूब गुनाह कर लें और (आख़िर तो) उनके लिए रूसवा करने वाला अज़ाब है
\end{hindi}}
\flushright{\begin{Arabic}
\quranayah[3][179]
\end{Arabic}}
\flushleft{\begin{hindi}
(मुनाफ़िक़ो) ख़ुदा ऐसा नहीं कि बुरे भले की तमीज़ किए बगैर जिस हालत पर तुम हो उसी हालत पर मोमिनों को भी छोड़ दे और ख़ुदा ऐसा भी नहीं है कि तुम्हें गैब की बातें बता दे मगर (हॉ) ख़ुदा अपने रसूलों में जिसे चाहता है (गैब बताने के वास्ते) चुन लेता है पस ख़ुदा और उसके रसूलों पर ईमान लाओ और अगर तुम ईमान लाओगे और परहेज़गारी करोगे तो तुम्हारे वास्ते बड़ी जज़ाए ख़ैर है
\end{hindi}}
\flushright{\begin{Arabic}
\quranayah[3][180]
\end{Arabic}}
\flushleft{\begin{hindi}
और जिन लोगों को ख़ुदा ने अपने फ़ज़ल (व करम) से कुछ दिया है (और फिर) बुख्ल करते हैं वह हरगिज़ इस ख्याल में न रहें कि ये उनके लिए (कुछ) बेहतर होगा बल्कि ये उनके हक़ में बदतर है क्योंकि जिस (माल) का बुख्ल करते हैं अनक़रीब ही क़यामत के दिन उसका तौक़ बनाकर उनके गले में पहनाया जाएगा और सारे आसमान व ज़मीन की मीरास ख़ुदा ही की है और जो कुछ तुम करते हो ख़ुदा उससे ख़बरदार है
\end{hindi}}
\flushright{\begin{Arabic}
\quranayah[3][181]
\end{Arabic}}
\flushleft{\begin{hindi}
जो लोग (यहूद) ये कहते हैं कि ख़ुदा तो कंगाल है और हम बड़े मालदार हैं ख़ुदा ने उनकी ये बकवास सुनी उन लोगों ने जो कुछ किया उसको और उनका पैग़म्बरों को नाहक़ क़त्ल करना हम अभी से लिख लेते हैं और (आज तो जो जी में कहें मगर क़यामत के दिन) हम कहेंगे कि अच्छा तो लो (अपनी शरारत के एवज़ में) जलाने वाले अज़ाब का मज़ा चखो
\end{hindi}}
\flushright{\begin{Arabic}
\quranayah[3][182]
\end{Arabic}}
\flushleft{\begin{hindi}
ये उन्हीं कामों का बदला है जिनको तुम्हारे हाथों ने (ज़ादे आख़ेरत बना कर) पहले से भेजा है वरना ख़ुदा तो कभी अपने बन्दों पर ज़ुल्म करने वाला नहीं
\end{hindi}}
\flushright{\begin{Arabic}
\quranayah[3][183]
\end{Arabic}}
\flushleft{\begin{hindi}
(यह वही लोग हैं) जो कहते हैं कि ख़ुदा ने तो हमसे वायदा किया है कि जब तक कोई रसूल हमें ये (मौजिज़ा) न दिखा दे कि वह कुरबानी करे और उसको (आसमानी) आग आकर चट कर जाए उस वक्त तक हम ईमान न लाएंगें (ऐ रसूल) तुम कह दो कि (भला) ये तो बताओ बहुतेरे पैग़म्बर मुझसे क़ब्ल तुम्हारे पास वाजे व रौशन मौजिज़ात और जिस चीज़ की तुमने (उस वक्त) फ़रमाइश की है (वह भी) लेकर आए फिर तुम अगर (अपने दावे में) सच्चे तो तुमने क्यों क़त्ल किया
\end{hindi}}
\flushright{\begin{Arabic}
\quranayah[3][184]
\end{Arabic}}
\flushleft{\begin{hindi}
(ऐ रसूल) अगर वह इस पर भी तुम्हें झुठलाएं तो (तुम आज़ुर्दा न हो क्योंकि) तुमसे पहले भी बहुत से पैग़म्बर रौशन मौजिज़े और सहीफे और नूरानी किताब लेकर आ चुके हैं (मगर) फिर भी लोगों ने आख़िर झुठला ही छोड़ा
\end{hindi}}
\flushright{\begin{Arabic}
\quranayah[3][185]
\end{Arabic}}
\flushleft{\begin{hindi}
हर जान एक न एक (दिन) मौत का मज़ा चखेगी और तुम लोग क़यामत के दिन (अपने किए का) पूरा पूरा बदला भर पाओगे पस जो शख्स जहन्नुम से हटा दिया गया और बहिश्त में पहुंचा दिया गया पस वही कामयाब हुआ और दुनिया की (चन्द रोज़ा) ज़िन्दगी धोखे की टट्टी के सिवा कुछ नहीं
\end{hindi}}
\flushright{\begin{Arabic}
\quranayah[3][186]
\end{Arabic}}
\flushleft{\begin{hindi}
(मुसलमानों) तुम्हारे मालों और जानों का तुमसे ज़रूर इम्तेहान लिया जाएगा और जिन लोगो को तुम से पहले किताबे ख़ुदा दी जा चुकी है (यहूद व नसारा) उनसे और मुशरेकीन से बहुत ही दुख दर्द की बातें तुम्हें ज़रूर सुननी पड़ेंगी और अगर तुम (उन मुसीबतों को) झेल जाओगे और परहेज़गारी करते रहोगे तो बेशक ये बड़ी हिम्मत का काम है
\end{hindi}}
\flushright{\begin{Arabic}
\quranayah[3][187]
\end{Arabic}}
\flushleft{\begin{hindi}
और (ऐ रसूल) इनको वह वक्त तो याद दिलाओ जब ख़ुदा ने अहले किताब से एहद व पैमान लिया था कि तुम किताबे ख़ुदा को साफ़ साफ़ बयान कर देना और (ख़बरदार) उसकी कोई बात छुपाना नहीं मगर इन लोगों ने (ज़रा भी ख्याल न किया) और उनको पसे पुश्त फेंक दिया और उसके बदले में (बस) थोड़ी सी क़ीमत हासिल कर ली पस ये क्या ही बुरा (सौदा) है जो ये लोग ख़रीद रहे हैं
\end{hindi}}
\flushright{\begin{Arabic}
\quranayah[3][188]
\end{Arabic}}
\flushleft{\begin{hindi}
(ऐ रसूल) तुम उन्हें ख्याल में भी न लाना जो अपनी कारस्तानी पर इतराए जाते हैं और किया कराया ख़ाक नहीं (मगर) तारीफ़ के ख़ास्तगार (चाहते) हैं पस तुम हरगिज़ ये ख्याल न करना कि इनको अज़ाब से छुटकारा है बल्कि उनके लिए दर्दनाक अज़ाब है
\end{hindi}}
\flushright{\begin{Arabic}
\quranayah[3][189]
\end{Arabic}}
\flushleft{\begin{hindi}
और आसमान व ज़मीन सब ख़ुदा ही का मुल्क है और ख़ुदा ही हर चीज़ पर क़ादिर है
\end{hindi}}
\flushright{\begin{Arabic}
\quranayah[3][190]
\end{Arabic}}
\flushleft{\begin{hindi}
इसमें तो शक ही नहीं कि आसमानों और ज़मीन की पैदाइश और रात दिन के फेर बदल में अक्लमन्दों के लिए (क़ुदरत ख़ुदा की) बहुत सी निशानियॉ हैं
\end{hindi}}
\flushright{\begin{Arabic}
\quranayah[3][191]
\end{Arabic}}
\flushleft{\begin{hindi}
जो लोग उठते बैठते करवट लेते (अलगरज़ हर हाल में) ख़ुदा का ज़िक्र करते हैं और आसमानों और ज़मीन की बनावट में ग़ौर व फ़िक्र करते हैं और (बेसाख्ता) कह उठते हैं कि ख़ुदावन्दा तूने इसको बेकार पैदा नहीं किया तू (फेले अबस से) पाक व पाकीज़ा है बस हमको दोज़क के अज़ाब से बचा
\end{hindi}}
\flushright{\begin{Arabic}
\quranayah[3][192]
\end{Arabic}}
\flushleft{\begin{hindi}
ऐ हमारे पालने वाले जिसको तूने दोज़ख़ में डाला तो यक़ीनन उसे रूसवा कर डाला और जुल्म करने वाले का कोई मददगार नहीं
\end{hindi}}
\flushright{\begin{Arabic}
\quranayah[3][193]
\end{Arabic}}
\flushleft{\begin{hindi}
ऐ हमारे पालने वाले (जब) हमने एक आवाज़ लगाने वाले (पैग़म्बर) को सुना कि वह (ईमान के वास्ते यूं पुकारता था) कि अपने परवरदिगार पर ईमान लाओ तो हम ईमान लाए पस ऐ हमारे पालने वाले हमें हमारे गुनाह बख्श दे और हमारी बुराईयों को हमसे दूर करे दे और हमें नेकों के साथ (दुनिया से) उठा ले
\end{hindi}}
\flushright{\begin{Arabic}
\quranayah[3][194]
\end{Arabic}}
\flushleft{\begin{hindi}
और ऐ पालने वाले अपने रसूलों की मार्फत जो कुछ हमसे वायदा किया है हमें दे और हमें क़यामत के दिन रूसवा न कर तू तो वायदा ख़िलाफ़ी करता ही नहीं
\end{hindi}}
\flushright{\begin{Arabic}
\quranayah[3][195]
\end{Arabic}}
\flushleft{\begin{hindi}
तो उनके परवरदिगार ने दुआ कुबूल कर ली और (फ़रमाया) कि हम तुममें से किसी काम करने वाले के काम को अकारत नहीं करते मर्द हो या औरत (उस में कुछ किसी की खुसूसियत नहीं क्योंकि) तुम एक दूसरे (की जिन्स) से हो जो लोग (हमारे लिए वतन आवारा हुए) और शहर बदर किए गए और उन्होंने हमारी राह में अज़ीयतें उठायीं और (कुफ्फ़र से) जंग की और शहीद हुए मैं उनकी बुराईयों से ज़रूर दरगुज़र करूंगा और उन्हें बेहिश्त के उन बाग़ों में ले जाऊॅगा जिनके नीचे नहरें जारी हैं ख़ुदा के यहॉ ये उनके किये का बदला है और ख़ुदा (ऐसा ही है कि उस) के यहॉ तो अच्छा ही बदला है
\end{hindi}}
\flushright{\begin{Arabic}
\quranayah[3][196]
\end{Arabic}}
\flushleft{\begin{hindi}
(ऐ रसूल) काफ़िरों का शहरों शहरों चैन करते फिरना तुम्हे धोखे में न डाले
\end{hindi}}
\flushright{\begin{Arabic}
\quranayah[3][197]
\end{Arabic}}
\flushleft{\begin{hindi}
ये चन्द रोज़ा फ़ायदा हैं फिर तो (आख़िरकार) उनका ठिकाना जहन्नुम ही है और क्या ही बुरा ठिकाना है
\end{hindi}}
\flushright{\begin{Arabic}
\quranayah[3][198]
\end{Arabic}}
\flushleft{\begin{hindi}
मगर जिन लोगों ने अपने परवरदिगार की परहेज़गारी (इख्तेयार की उनके लिए बेहिश्त के) वह बाग़ात हैं जिनके नीचे नहरें जारीं हैं और वह हमेशा उसी में रहेंगे ये ख़ुदा की तरफ़ से उनकी (दावत है और जो साज़ो सामान) ख़ुदा के यहॉ है वह नेको कारों के वास्ते दुनिया से कहीं बेहतर है
\end{hindi}}
\flushright{\begin{Arabic}
\quranayah[3][199]
\end{Arabic}}
\flushleft{\begin{hindi}
और अहले किताब में से कुछ लोग तो ऐसे ज़रूर हैं जो ख़ुदा पर और जो (किताब) तुम पर नाज़िल हुई और जो (किताब) उनपर नाज़िल हुई (सब पर) ईमान रखते हैं ख़ुदा के आगे सर झुकाए हुए हैं और ख़ुदा की आयतों के बदले थोड़ी सी क़ीमत (दुनियावी फ़ायदे) नहीं लेते ऐसे ही लोगों के वास्ते उनके परवरदिगार के यहॉ अच्छा बदला है बेशक ख़ुदा बहुत जल्द हिसाब करने वाला है
\end{hindi}}
\flushright{\begin{Arabic}
\quranayah[3][200]
\end{Arabic}}
\flushleft{\begin{hindi}
ऐ ईमानदारों (दीन की तकलीफ़ों को) और दूसरों को बर्दाश्त की तालीम दो और (जिहाद के लिए) कमरें कस लो और ख़ुदा ही से डरो ताकि तुम अपनी दिली मुराद पाओ
\end{hindi}}
\chapter{An-Nisa' (The Women)}
\begin{Arabic}
\Huge{\centerline{\basmalah}}\end{Arabic}
\flushright{\begin{Arabic}
\quranayah[4][1]
\end{Arabic}}
\flushleft{\begin{hindi}
ऐ लोगों अपने पालने वाले से डरो जिसने तुम सबको (सिर्फ) एक शख्स से पैदा किया और (वह इस तरह कि पहले) उनकी बाकी मिट्टी से उनकी बीवी (हव्वा) को पैदा किया और (सिर्फ़) उन्हीं दो (मियॉ बीवी) से बहुत से मर्द और औरतें दुनिया में फैला दिये और उस ख़ुदा से डरो जिसके वसीले से आपस में एक दूसरे से सवाल करते हो और क़तए रहम से भी डरो बेशक ख़ुदा तुम्हारी देखभाल करने वाला है
\end{hindi}}
\flushright{\begin{Arabic}
\quranayah[4][2]
\end{Arabic}}
\flushleft{\begin{hindi}
और यतीमों को उनके माल दे दो और बुरी चीज़ (माले हराम) को भली चीज़ (माले हलाल) के बदले में न लो और उनके माल अपने मालों में मिलाकर न चख जाओ क्योंकि ये बहुत बड़ा गुनाह है
\end{hindi}}
\flushright{\begin{Arabic}
\quranayah[4][3]
\end{Arabic}}
\flushleft{\begin{hindi}
और अगर तुमको अन्देशा हो कि (निकाह करके) तुम यतीम लड़कियों (की रखरखाव) में इन्साफ न कर सकोगे तो और औरतों में अपनी मर्ज़ी के मवाफ़िक दो दो और तीन तीन और चार चार निकाह करो (फिर अगर तुम्हें इसका) अन्देशा हो कि (मुततइद) बीवियों में (भी) इन्साफ न कर सकोगे तो एक ही पर इक्तेफ़ा करो या जो (लोंडी) तुम्हारी ज़र ख़रीद हो (उसी पर क़नाअत करो) ये तदबीर बेइन्साफ़ी न करने की बहुत क़रीने क़यास है
\end{hindi}}
\flushright{\begin{Arabic}
\quranayah[4][4]
\end{Arabic}}
\flushleft{\begin{hindi}
और औरतों को उनके महर ख़ुशी ख़ुशी दे डालो फिर अगर वह ख़ुशी ख़ुशी तुम्हें कुछ छोड़ दे तो शौक़ से नौशे जान खाओ पियो
\end{hindi}}
\flushright{\begin{Arabic}
\quranayah[4][5]
\end{Arabic}}
\flushleft{\begin{hindi}
और अपने वह माल जिनपर ख़ुदा ने तुम्हारी गुज़र न क़रार दी है बेवकूफ़ों (ना समझ यतीम) को न दे बैठो हॉ उसमें से उन्हें खिलाओ और उनको पहनाओ (तो मज़ाएक़ा नहीं) और उनसे (शौक़ से) अच्छी तरह बात करो
\end{hindi}}
\flushright{\begin{Arabic}
\quranayah[4][6]
\end{Arabic}}
\flushleft{\begin{hindi}
और यतीमों को कारोबार में लगाए रहो यहॉ तक के ब्याहने के क़ाबिल हों फिर उस वक्त तुम उन्हे (एक महीने का ख़र्च) उनके हाथ से कराके अगर होशियार पाओ तो उनका माल उनके हवाले कर दो और (ख़बरदार) ऐसा न करना कि इस ख़ौफ़ से कि कहीं ये बड़े हो जाएंगे फ़ुज़ूल ख़र्ची करके झटपट उनका माल चट कर जाओ और जो (जो वली या सरपरस्त) दौलतमन्द हो तो वह (माले यतीम अपने ख़र्च में लाने से) बचता रहे और (हॉ) जो मोहताज हो वह अलबत्ता (वाजिबी) दस्तूर के मुताबिक़ खा सकता है पस जब उनके माल उनके हवाले करने लगो तो लोगों को उनका गवाह बना लो और (यूं तो) हिसाब लेने को ख़ुदा काफ़ी ही है
\end{hindi}}
\flushright{\begin{Arabic}
\quranayah[4][7]
\end{Arabic}}
\flushleft{\begin{hindi}
मॉ बाप और क़राबतदारों के तर्के में कुछ हिस्सा ख़ास मर्दों का है और उसी तरह माँ बाब और क़राबतदारो के तरके में कुछ हिस्सा ख़ास औरतों का भी है ख्वाह तर्क कम हो या ज्यादा (हर शख्स का) हिस्सा (हमारी तरफ़ से) मुक़र्रर किया हुआ है
\end{hindi}}
\flushright{\begin{Arabic}
\quranayah[4][8]
\end{Arabic}}
\flushleft{\begin{hindi}
और जब (तर्क की) तक़सीम के वक्त (वह) क़राबतदार (जिनका कोई हिस्सा नहीं) और यतीम बच्चे और मोहताज लोग आ जाएं तो उन्हे भी कुछ उसमें से दे दो और उसे अच्छी तरह (उनवाने शाइस्ता से) बात करो
\end{hindi}}
\flushright{\begin{Arabic}
\quranayah[4][9]
\end{Arabic}}
\flushleft{\begin{hindi}
और उन लोगों को डरना (और ख्याल करना चाहिये) कि अगर वह लोग ख़ुद अपने बाद (नन्हे नन्हे) नातवॉ बच्चे छोड़ जाते तो उन पर (किस क़दर) तरस आता पस उनको (ग़रीब बच्चों पर सख्ती करने में) ख़ुदा से डरना चाहिये और उनसे सीधी तरह बात करना चाहिए
\end{hindi}}
\flushright{\begin{Arabic}
\quranayah[4][10]
\end{Arabic}}
\flushleft{\begin{hindi}
जो यतीमों के माल नाहक़ चट कर जाया करते हैं वह अपने पेट में बस अंगारे भरते हैं और अनक़रीब जहन्नुम वासिल होंगे
\end{hindi}}
\flushright{\begin{Arabic}
\quranayah[4][11]
\end{Arabic}}
\flushleft{\begin{hindi}
(मुसलमानों) ख़ुदा तुम्हारी औलाद के हक़ में तुमसे वसीयत करता है कि लड़के का हिस्सा दो लड़कियों के बराबर है और अगर (मय्यत की) औलाद में सिर्फ लड़कियॉ ही हों (दो) या (दो) से ज्यादा तो उनका (मक़र्रर हिस्सा) कुल तर्के का दो तिहाई है और अगर एक लड़की हो तो उसका आधा है और मय्यत के बाप मॉ हर एक का अगर मय्यत की कोई औलाद मौजूद न हो तो माल मुस्तरद का में से मुअय्यन (ख़ास चीज़ों में) छटा हिस्सा है और अगर मय्यत के कोई औलाद न हो और उसके सिर्फ मॉ बाप ही वारिस हों तो मॉ का मुअय्यन (ख़ास चीज़ों में) एक तिहाई हिस्सा तय है और बाक़ी बाप का लेकिन अगर मय्यत के (हक़ीक़ी और सौतेले) भाई भी मौजूद हों तो (अगरचे उन्हें कुछ न मिले) उस वक्त मॉ का हिस्सा छठा ही होगा (और वह भी) मय्यत नें जिसके बारे में वसीयत की है उसकी तालीम और (अदाए) क़र्ज़ के बाद तुम्हारे बाप हों या बेटे तुम तो यह नहीं जानते हों कि उसमें कौन तुम्हारी नाफ़रमानी में ज्यादा क़रीब है (फिर तुम क्या दख़ल दे सकते हो) हिस्सा तो सिर्फ ख़ुदा की तरफ़ से मुअय्यन होता है क्योंकि ख़ुदा तो ज़रूर हर चीज़ को जानता और तदबीर वाला है
\end{hindi}}
\flushright{\begin{Arabic}
\quranayah[4][12]
\end{Arabic}}
\flushleft{\begin{hindi}
और जो कुछ तुम्हारी बीवियां छोड़ कर (मर) जाए पस अगर उनके कोई औलाद न हो तो तुम्हारा आधा है और अगर उनके कोई औलाद हो तो जो कुछ वह तरका छोड़े उसमें से बाज़ चीज़ों में चौथाई तुम्हारा है (और वह भी) औरत ने जिसकी वसीयत की हो और (अदाए) क़र्ज़ के बाद अगर तुम्हारे कोई औलाद न हो तो तुम्हारे तरके में से तुम्हारी बीवियों का बाज़ चीज़ों में चौथाई है और अगर तुम्हारी कोई औलाद हो तो तुम्हारे तर्के में से उनका ख़ास चीज़ों में आठवॉ हिस्सा है (और वह भी) तुमने जिसके बारे में वसीयत की है उसकी तामील और (अदाए) क़र्ज़ के बाद और अगर कोई मर्द या औरत अपनी मादरजिलों (ख्याली) भाई या बहन को वारिस छोड़े तो उनमें से हर एक का ख़ास चीजों में छठा हिस्सा है और अगर उससे ज्यादा हो तो सबके सब एक ख़ास तिहाई में शरीक़ रहेंगे और (ये सब) मय्यत ने जिसके बारे में वसीयत की है उसकी तामील और (अदाए) क़र्ज क़े बाद मगर हॉ वह वसीयत (वारिसों को ख्वाह मख्वाह) नुक्सान पहुंचाने वाली न हो (तब) ये वसीयत ख़ुदा की तरफ़ से है और ख़ुदा तो हर चीज़ का जानने वाला और बुर्दबार है
\end{hindi}}
\flushright{\begin{Arabic}
\quranayah[4][13]
\end{Arabic}}
\flushleft{\begin{hindi}
यह ख़ुदा की (मुक़र्रर की हुई) हदें हैं और ख़ुदा और रसूल की इताअत करे उसको ख़ुदा आख़ेरत में ऐसे (हर भरे) बाग़ों में पहुंचा देगा जिसके नीचे नहरें जारी होंगी और वह उनमें हमेशा (चैन से) रहेंगे और यही तो बड़ी कामयाबी है
\end{hindi}}
\flushright{\begin{Arabic}
\quranayah[4][14]
\end{Arabic}}
\flushleft{\begin{hindi}
और जिस शख्स से ख़ुदा व रसूल की नाफ़रमानी की और उसकी हदों से गुज़र गया तो बस ख़ुदा उसको जहन्नुम में दाख़िल करेगा
\end{hindi}}
\flushright{\begin{Arabic}
\quranayah[4][15]
\end{Arabic}}
\flushleft{\begin{hindi}
और वह उसमें हमेशा अपना किया भुगतता रहेगा और उसके लिए बड़ी रूसवाई का अज़ाब है और तुम्हारी औरतों में से जो औरतें बदकारी करें तो उनकी बदकारी पर अपने लोगों में से चार गवाही लो और फिर अगर चारों गवाह उसकी तसदीक़ करें तो (उसकी सज़ा ये है कि) उनको घरों में बन्द रखो यहॉ तक कि मौत आ जाए या ख़ुदा उनकी कोई (दूसरी) राह निकाले
\end{hindi}}
\flushright{\begin{Arabic}
\quranayah[4][16]
\end{Arabic}}
\flushleft{\begin{hindi}
और तुम लोगों में से जिनसे बदकारी सरज़द हुई हो उनको मारो पीटो फिर अगर वह दोनों (अपनी हरकत से) तौबा करें और इस्लाह कर लें तो उनको छोड़ दो बेशक ख़ुदा बड़ा तौबा कुबूल करने वाला मेहरबान है
\end{hindi}}
\flushright{\begin{Arabic}
\quranayah[4][17]
\end{Arabic}}
\flushleft{\begin{hindi}
मगर ख़ुदा की बारगाह में तौबा तो सिर्फ उन्हीं लोगों की (ठीक) है जो नादानिस्ता बुरी हरकत कर बैठे (और) फिर जल्दी से तौबा कर ले तो ख़ुदा भी ऐसे लोगों की तौबा क़ुबूल कर लेता है और ख़ुदा तो बड़ा जानने वाला हकीम है
\end{hindi}}
\flushright{\begin{Arabic}
\quranayah[4][18]
\end{Arabic}}
\flushleft{\begin{hindi}
और तौबा उन लोगों के लिये (मुफ़ीद) नहीं है जो (उम्र भर) तो बुरे काम करते रहे यहॉ तक कि जब उनमें से किसी के सर पर मौत आ खड़ी हुई तो कहने लगे अब मैंने तौबा की और (इसी तरह) उन लोगों के लिए (भी तौबा) मुफ़ीद नहीं है जो कुफ़्र ही की हालत में मर गये ऐसे ही लोगों के वास्ते हमने दर्दनाक अज़ाब तैयार कर रखा है
\end{hindi}}
\flushright{\begin{Arabic}
\quranayah[4][19]
\end{Arabic}}
\flushleft{\begin{hindi}
ऐ ईमानदारों तुमको ये जायज़ नहीं कि (अपने मुरिस की) औरतों से (निकाह कर) के (ख्वाह मा ख्वाह) ज़बरदस्ती वारिस बन जाओ और जो कुछ तुमने उन्हें (शौहर के तर्के से) दिया है उसमें से कुछ (आपस से कुछ वापस लेने की नीयत से) उन्हें दूसरे के साथ (निकाह करने से) न रोको हॉ जब वह खुल्लम खुल्ला कोई बदकारी करें तो अलबत्ता रोकने में (मज़ाएक़ा (हर्ज)नहीं) और बीवियों के साथ अच्छा सुलूक करते रहो और अगर तुम किसी वजह से उन्हें नापसन्द करो (तो भी सब्र करो क्योंकि) अजब नहीं कि किसी चीज़ को तुम नापसन्द करते हो और ख़ुदा तुम्हारे लिए उसमें बहुत बेहतरी कर दे
\end{hindi}}
\flushright{\begin{Arabic}
\quranayah[4][20]
\end{Arabic}}
\flushleft{\begin{hindi}
और अगर तुम एक बीवी (को तलाक़ देकर उस) की जगह दूसरी बीवी (निकाह करके) तबदील करना चाहो तो अगरचे तुम उनमें से एक को (जिसे तलाक़ देना चाहते हो) बहुत सा माल दे चुके हो तो तुम उनमें से कुछ (वापस न लो) क्या तुम्हारी यही गैरत है कि (ख्वाह मा ख्वाह) बोहतान बॉधकर या सरीही जुर्म लगाकर वापस ले लो
\end{hindi}}
\flushright{\begin{Arabic}
\quranayah[4][21]
\end{Arabic}}
\flushleft{\begin{hindi}
और क्या तुम उसको (वापस लोगे हालॉकि तुममें से) एक दूसरे के साथ ख़िलवत कर चुका है और बीवियॉ तुमसे (निकाह के वक्त नुक़फ़ा वगैरह का) पक्का क़रार ले चुकी हैं
\end{hindi}}
\flushright{\begin{Arabic}
\quranayah[4][22]
\end{Arabic}}
\flushleft{\begin{hindi}
और जिन औरतों से तुम्हारे बाप दादाओं से (निकाह) जमाअ (अगरचे ज़िना) किया हो तुम उनसे निकाह न करो मगर जो हो चुका (वह तो हो चुका) वह बदकारी और ख़ुदा की नाख़ुशी की बात ज़रूर थी और बहुत बुरा तरीक़ा था
\end{hindi}}
\flushright{\begin{Arabic}
\quranayah[4][23]
\end{Arabic}}
\flushleft{\begin{hindi}
(मुसलमानों हसबे जेल) औरतें तुम पर हराम की गयी हैं तुम्हारी माएं (दादी नानी वगैरह सब) और तुम्हारी बेटियॉ (पोतियॉ) नवासियॉ (वगैरह) और तुम्हारी बहनें और तुम्हारी फुफियॉ और तुम्हारी ख़ालाएं और भतीजियॉ और भंजियॉ और तुम्हारी वह माएं जिन्होंने तुमको दूध पिलाया है और तुम्हारी रज़ाई (दूध शरीक) बहनें और तुम्हारी बीवीयों की माँए और वह (मादर ज़िलो) लड़कियां जो तुम्हारी गोद में परवरिश पा चुकी हो और उन औरतों (के पेट) से (पैदा हुई) हैं जिनसे तुम हमबिस्तरी कर चुके हो हाँ अगर तुमने उन बीवियों से (सिर्फ निकाह किया हो) हमबिस्तरी न की तो अलबत्ता उन मादरज़िलों (लड़कियों से) निकाह (करने में) तुम पर कुछ गुनाह नहीं है और तुम्हारे सुलबी लड़को (पोतों नवासों वगैरह) की बीवियॉ (बहुएं) और दो बहनों से एक साथ निकाह करना मगर जो हो चुका (वह माफ़ है) बेशक ख़ुदा बड़ा बख्शने वाला मेहरबान है
\end{hindi}}
\flushright{\begin{Arabic}
\quranayah[4][24]
\end{Arabic}}
\flushleft{\begin{hindi}
और शौहरदार औरतें मगर वह औरतें जो (जिहाद में कुफ्फ़ार से) तुम्हारे कब्ज़े में आ जाएं हराम नहीं (ये) ख़ुदा का तहरीरी हुक्म (है जो) तुमपर (फ़र्ज़ किया गया) है और उन औरतों के सिवा (और औरतें) तुम्हारे लिए जायज़ हैं बशर्ते कि बदकारी व ज़िना नहीं बल्कि तुम इफ्फ़त या पाकदामिनी की ग़रज़ से अपने माल (व मेहर) के बदले (निकाह करना) चाहो हॉ जिन औरतों से तुमने मुताअ किया हो तो उन्हें जो मेहर मुअय्यन किया है दे दो और मेहर के मुक़र्रर होने के बाद अगर आपस में (कम व बेश पर) राज़ी हो जाओ तो उसमें तुमपर कुछ गुनाह नहीं है बेशक ख़ुदा (हर चीज़ से) वाक़िफ़ और मसलेहतों का पहचानने वाला है
\end{hindi}}
\flushright{\begin{Arabic}
\quranayah[4][25]
\end{Arabic}}
\flushleft{\begin{hindi}
और तुममें से जो शख्स आज़ाद इफ्फ़तदार औरतों से निकाह करने की माली हैसियत से क़ुदरत न रखता हो तो वह तुम्हारी उन मोमिना लौन्डियों से जो तुम्हारे कब्ज़े में हैं निकाह कर सकता है और ख़ुदा तुम्हारे ईमान से ख़ूब वाक़िफ़ है (ईमान की हैसियत से तो) तुममें एक दूसरे का हमजिन्स है पस (बे ताम्मुल) उनके मालिकों की इजाज़त से लौन्डियों से निकाह करो और उनका मेहर हुस्ने सुलूक से दे दो मगर उन्हीं (लौन्डियो) से निकाह करो जो इफ्फ़त के साथ तुम्हारी पाबन्द रहें न तो खुले आम ज़िना करना चाहें और न चोरी छिपे से आशनाई फिर जब तुम्हारी पाबन्द हो चुकी उसके बाद कोई बदकारी करे तो जो सज़ा आज़ाद बीवियों को दी जाती है उसकी आधी (सज़ा) लौन्डियों को दी जाएगी (और लौन्डियों) से निकाह कर भी सकता है तो वह शख्स जिसको ज़िना में मुब्तिला हो जाने का ख़ौफ़ हो और सब्र करे तो तुम्हारे हक़ में ज्यादा बेहतर है और ख़ुदा बख्शने वाला मेहरबान है
\end{hindi}}
\flushright{\begin{Arabic}
\quranayah[4][26]
\end{Arabic}}
\flushleft{\begin{hindi}
ख़ुदा तो ये चाहता है कि (अपने) एहकाम तुम लोगों से साफ़ साफ़ बयान कर दे और जो (अच्छे) लोग तुमसे पहले गुज़र चुके हैं उनके तरीक़े पर चला दे और तुम्हारी तौबा कुबूल करे और ख़ुदा तो (हर चीज़ से) वाक़िफ़ और हिकमत वाला है
\end{hindi}}
\flushright{\begin{Arabic}
\quranayah[4][27]
\end{Arabic}}
\flushleft{\begin{hindi}
और ख़ुदा तो चाहता है कि तुम्हारी तौबा क़ुबूल
\end{hindi}}
\flushright{\begin{Arabic}
\quranayah[4][28]
\end{Arabic}}
\flushleft{\begin{hindi}
करे और जो लोग नफ़सियानी ख्वाहिश के पीछे पडे हैं वह ये चाहते हैं कि तुम लोग (राहे हक़ से) बहुत दूर हट जाओ और ख़ुदा चाहता है कि तुमसे बार में तख़फ़ीफ़ कर दें क्योंकि आदमी तो बहुत कमज़ोर पैदा किया गया है
\end{hindi}}
\flushright{\begin{Arabic}
\quranayah[4][29]
\end{Arabic}}
\flushleft{\begin{hindi}
ए ईमानवालों आपस में एक दूसरे का माल नाहक़ न खा जाया करो लेकिन (हॉ) तुम लोगों की बाहमी रज़ामन्दी से तिजारत हो (और उसमें एक दूसरे का माल हो तो मुज़ाएक़ा नहीं) और अपना गला आप घूंट के अपनी जान न दो (क्योंकि) ख़ुदा तो ज़रूर तुम्हारे हाल पर मेहरबान है
\end{hindi}}
\flushright{\begin{Arabic}
\quranayah[4][30]
\end{Arabic}}
\flushleft{\begin{hindi}
और जो शख्स जोरो ज़ुल्म से नाहक़ ऐसा करेगा (ख़ुदकुशी करेगा) तो (याद रहे कि) हम बहुत जल्द उसको जहन्नुम की आग में झोंक देंगे यह ख़ुदा के लिये आसान है
\end{hindi}}
\flushright{\begin{Arabic}
\quranayah[4][31]
\end{Arabic}}
\flushleft{\begin{hindi}
जिन कामों की तुम्हें मनाही की जाती है अगर उनमें से तुम गुनाहे कबीरा से बचते रहे तो हम तुम्हारे (सग़ीरा) गुनाहों से भी दरगुज़र करेंगे और तुमको बहुत अच्छी इज्ज़त की जगह पहुंचा देंगे
\end{hindi}}
\flushright{\begin{Arabic}
\quranayah[4][32]
\end{Arabic}}
\flushleft{\begin{hindi}
और ख़ुदा ने जो तुममें से एक दूसरे पर तरजीह दी है उसकी हवस न करो (क्योंकि फ़ज़ीलत तो आमाल से है) मर्दो को अपने किए का हिस्सा है और औरतों को अपने किए का हिस्सा और ये और बात है कि तुम ख़ुदा से उसके फज़ल व करम की ख्वाहिश करो ख़ुदा तो हर चीज़े से वाक़िफ़ है
\end{hindi}}
\flushright{\begin{Arabic}
\quranayah[4][33]
\end{Arabic}}
\flushleft{\begin{hindi}
और मां बाप (या) और क़राबतदार (ग़रज़) तो शख्स जो तरका छोड़ जाए हमने हर एक का (वाली) वारिस मुक़र्रर कर दिया है और जिन लोगों से तुमने मुस्तहकम (पक्का) एहद किया है उनका मुक़र्रर हिस्सा भी तुम दे दो बेशक ख़ुदा तो हर चीज़ पर गवाह है
\end{hindi}}
\flushright{\begin{Arabic}
\quranayah[4][34]
\end{Arabic}}
\flushleft{\begin{hindi}
मर्दो का औरतों पर क़ाबू है क्योंकि (एक तो) ख़ुदा ने बाज़ आदमियों (मर्द) को बाज़ अदमियों (औरत) पर फ़ज़ीलत दी है और (इसके अलावा) चूंकि मर्दो ने औरतों पर अपना माल ख़र्च किया है पस नेक बख्त बीवियॉ तो शौहरों की ताबेदारी करती हैं (और) उनके पीठ पीछे जिस तरह ख़ुदा ने हिफ़ाज़त की वह भी (हर चीज़ की) हिफ़ाज़त करती है और वह औरतें जिनके नाफरमान सरकश होने का तुम्हें अन्देशा हो तो पहले उन्हें समझाओ और (उसपर न माने तो) तुम उनके साथ सोना छोड़ दो और (इससे भी न माने तो) मारो मगर इतना कि खून न निकले और कोई अज़ो न (टूटे) पस अगर वह तुम्हारी मुतीइ हो जाएं तो तुम भी उनके नुक़सान की राह न ढूंढो ख़ुदा तो ज़रूर सबसे बरतर बुजुर्ग़ है
\end{hindi}}
\flushright{\begin{Arabic}
\quranayah[4][35]
\end{Arabic}}
\flushleft{\begin{hindi}
और ऐ हुक्काम (वक्त) अगर तुम्हें मियॉ बीवी की पूरी नाइत्तेफ़ाक़ी का तरफैन से अन्देशा हो तो एक सालिस (पन्च) मर्द के कुनबे में से एक और सालिस औरत के कुनबे में मुक़र्रर करो अगर ये दोनों सालिस दोनों में मेल करा देना चाहें तो ख़ुदा उन दोनों के दरमियान उसका अच्छा बन्दोबस्त कर देगा ख़ुदा तो बेशक वाक़िफ व ख़बरदार है
\end{hindi}}
\flushright{\begin{Arabic}
\quranayah[4][36]
\end{Arabic}}
\flushleft{\begin{hindi}
और ख़ुदा ही की इबादत करो और किसी को उसका शरीक न बनाओ और मॉ बाप और क़राबतदारों और यतीमों और मोहताजों और रिश्तेदार पड़ोसियों और अजनबी पड़ोसियों और पहलू में बैठने वाले मुसाहिबों और पड़ोसियों और ज़र ख़रीद लौन्डी और गुलाम के साथ एहसान करो बेशक ख़ुदा अकड़ के चलने वालों और शेख़ीबाज़ों को दोस्त नहीं रखता
\end{hindi}}
\flushright{\begin{Arabic}
\quranayah[4][37]
\end{Arabic}}
\flushleft{\begin{hindi}
ये वह लोग हैं जो ख़ुद तो बुख्ल करते ही हैं और लोगों को भी बुख्ल का हुक्म देते हैं और जो माल ख़ुदा ने अपने फ़ज़ल व (करम) से उन्हें दिया है उसे छिपाते हैं और हमने तो कुफ़राने नेअमत करने वालों के वास्ते सख्त ज़िल्लत का अज़ाब तैयार कर रखा है
\end{hindi}}
\flushright{\begin{Arabic}
\quranayah[4][38]
\end{Arabic}}
\flushleft{\begin{hindi}
और जो लोग महज़ लोगों को दिखाने के वास्ते अपने माल ख़र्च करते हैं और न खुदा ही पर ईमान रखते हैं और न रोजे आख़ेरत पर ख़ुदा भी उनके साथ नहीं क्योंकि उनका साथी तो शैतान है और जिसका साथी शैतान हो तो क्या ही बुरा साथी है
\end{hindi}}
\flushright{\begin{Arabic}
\quranayah[4][39]
\end{Arabic}}
\flushleft{\begin{hindi}
अगर ये लोग ख़ुदा और रोज़े आख़िरत पर ईमान लाते और जो कुछ ख़ुदा ने उन्हें दिया है उसमें से राहे ख़ुदा में ख़र्च करते तो उन पर क्या आफ़त आ जाती और ख़ुदा तो उनसे ख़ूब वाक़िफ़ है
\end{hindi}}
\flushright{\begin{Arabic}
\quranayah[4][40]
\end{Arabic}}
\flushleft{\begin{hindi}
ख़ुदा तो हरगिज़ ज़र्रा बराबर भी ज़ुल्म नहीं करता बल्कि अगर ज़र्रा बराबर भी किसी की कोई नेकी हो तो उसको दूना करता है और अपनी तरफ़ से बड़ा सवाब अता फ़रमाता है
\end{hindi}}
\flushright{\begin{Arabic}
\quranayah[4][41]
\end{Arabic}}
\flushleft{\begin{hindi}
(ख़ैर दुनिया में तो जो चाहे करें) भला उस वक्त क्या हाल होगा जब हम हर गिरोह के गवाह तलब करेंगे और (मोहम्मद) तुमको उन सब पर गवाह की हैसियत में तलब करेंगे
\end{hindi}}
\flushright{\begin{Arabic}
\quranayah[4][42]
\end{Arabic}}
\flushleft{\begin{hindi}
उस दिन जिन लोगों ने कुफ़्र इख्तेयार किया और रसूल की नाफ़रमानी की ये आरज़ू करेंगे कि काश (वह पेवन्दे ख़ाक हो जाते) और उनके ऊपर से ज़मीन बराबर कर दी जाती और अफ़सोस ये लोग ख़ुदा से कोई बात उस दिन छुपा भी न सकेंगे
\end{hindi}}
\flushright{\begin{Arabic}
\quranayah[4][43]
\end{Arabic}}
\flushleft{\begin{hindi}
ऐ ईमानदारों तुम नशे की हालत में नमाज़ के क़रीब न जाओ ताकि तुम जो कुछ मुंह से कहो समझो भी तो और न जिनाबत की हालत में यहॉ तक कि ग़ुस्ल कर लो मगर राह गुज़र में हो (और गुस्ल मुमकिन नहीं है तो अलबत्ता ज़रूरत नहीं) बल्कि अगर तुम मरीज़ हो और पानी नुक़सान करे या सफ़र में हो तुममें से किसी का पैख़ाना निकल आए या औरतों से सोहबत की हो और तुमको पानी न मयस्सर हो (कि तहारत करो) तो पाक मिट्टी पर तैमूम कर लो और (उस का तरीक़ा ये है कि) अपने मुंह और हाथों पर मिट्टी भरा हाथ फेरो तो बेशक ख़ुदा माफ़ करने वाला है (और) बख्श ने वाला है
\end{hindi}}
\flushright{\begin{Arabic}
\quranayah[4][44]
\end{Arabic}}
\flushleft{\begin{hindi}
(ऐ रसूल) क्या तूमने उन लोगों के हाल पर नज़र नहीं की जिन्हें किताबे ख़ुदा का कुछ हिस्सा दिया गया था (मगर) वह लोग (हिदायत के बदले) गुमराही ख़रीदने लगे उनकी ऐन मुराद यह है कि तुम भी राहे रास्त से बहक जाओ
\end{hindi}}
\flushright{\begin{Arabic}
\quranayah[4][45]
\end{Arabic}}
\flushleft{\begin{hindi}
और ख़ुदा तुम्हारे दुशमनों से ख़ूब वाक़िफ़ है और दोस्ती के लिए बस ख़ुदा काफ़ी है और हिमायत के वास्ते भी ख़ुदा ही काफ़ी है
\end{hindi}}
\flushright{\begin{Arabic}
\quranayah[4][46]
\end{Arabic}}
\flushleft{\begin{hindi}
(ऐ रसूल) यहूद से कुछ लोग ऐसे भी हैं जो बातों में उनके महल व मौक़े से हेर फेर डाल देते हैं और अपनी ज़बानों को मरोड़कर और दीन पर तानाज़नी की राह से तुमसे समेअना व असैना (हमने सुना और नाफ़रमानी की) और वसमअ गैरा मुसमइन (तुम मेरी सुनो ख़ुदा तुमको न सुनवाए) राअना मेरा ख्याल करो मेरे चरवाहे कहा करते हैं और अगर वह इसके बदले समेअना व अताअना (हमने सुना और माना) और इसमाआ (मेरी सुनो) और (राअना) के एवज़ उनजुरना (हमपर निगाह रख) कहते तो उनके हक़ में कहीं बेहतर होता और बिल्कुल सीधी बात थी मगर उनपर तो उनके कुफ़्र की वजह से ख़ुदा की फ़िटकार है
\end{hindi}}
\flushright{\begin{Arabic}
\quranayah[4][47]
\end{Arabic}}
\flushleft{\begin{hindi}
पस उनमें से चन्द लोगों के सिवा और लोग ईमान ही न लाएंगे ऐ अहले किताब जो (किताब) हमने नाज़िल की है और उस (किताब) की भी तस्दीक़ करती है जो तुम्हारे पास है उस पर ईमान लाओ मगर क़ब्ल इसके कि हम कुछ लोगों के चेहरे बिगाड़कर उनके पुश्त की तरफ़ फेर दें या जिस तरह हमने असहाबे सबत (हफ्ते वालों) पर फिटकार बरसायी वैसी ही फिटकार उनपर भी करें
\end{hindi}}
\flushright{\begin{Arabic}
\quranayah[4][48]
\end{Arabic}}
\flushleft{\begin{hindi}
और ख़ुदा का हुक्म किया कराया हुआ काम समझो ख़ुदा उस जुर्म को तो अलबत्ता नहीं माफ़ करता कि उसके साथ शिर्क किया जाए हॉ उसके सिवा जो गुनाह हो जिसको चाहे माफ़ कर दे और जिसने (किसी को) ख़ुदा का शरीक बनाया तो उसने बड़े गुनाह का तूफान बॉधा
\end{hindi}}
\flushright{\begin{Arabic}
\quranayah[4][49]
\end{Arabic}}
\flushleft{\begin{hindi}
(ऐ रसूल) क्या तुमने उन लोगों के हाल पर नज़र नहीं की जो आप बड़े मुक़द्दस बनते हैं (मगर उससे क्या होता है) बल्कि ख़ुदा जिसे चाहता है मुक़द्दस बनाता है और ज़ुल्म तो किसी पर धागे के बराबर हो ही गा नहीं
\end{hindi}}
\flushright{\begin{Arabic}
\quranayah[4][50]
\end{Arabic}}
\flushleft{\begin{hindi}
(ऐ रसूल) ज़रा देखो तो ये लोग ख़ुदा पर कैसे कैसे झूठ तूफ़ान जोड़ते हैं और खुल्लम खुल्ला गुनाह के वास्ते तो यही काफ़ी है
\end{hindi}}
\flushright{\begin{Arabic}
\quranayah[4][51]
\end{Arabic}}
\flushleft{\begin{hindi}
(ऐ रसूल) क्या तुमने उन लोगों के (हाल पर) नज़र नहीं की जिन्हें किताबे ख़ुदा का कुछ हिस्सा दिया गया था और (फिर) शैतान और बुतों का कलमा पढ़ने लगे और जिन लोगों ने कुफ़्र इख्तेयार किया है उनकी निस्बत कहने लगे कि ये तो ईमान लाने वालों से ज्यादा राहे रास्त पर हैं
\end{hindi}}
\flushright{\begin{Arabic}
\quranayah[4][52]
\end{Arabic}}
\flushleft{\begin{hindi}
(ऐ रसूल) यही वह लोग हैं जिनपर ख़ुदा ने लानत की है और जिस पर ख़ुदा ने लानत की है तुम उनका मददगार हरगिज़ किसी को न पाओगे
\end{hindi}}
\flushright{\begin{Arabic}
\quranayah[4][53]
\end{Arabic}}
\flushleft{\begin{hindi}
क्या (दुनिया) की सल्तनत में कुछ उनका भी हिस्सा है कि इस वजह से लोगों को भूसी भर भी न देंगे
\end{hindi}}
\flushright{\begin{Arabic}
\quranayah[4][54]
\end{Arabic}}
\flushleft{\begin{hindi}
या ख़ुदा ने जो अपने फ़ज़ल से (तुम) लोगों को (कुरान) अता फ़रमाया है इसके रश्क पर चले जाते हैं (तो उसका क्या इलाज है) हमने तो इबराहीम की औलाद को किताब और अक्ल की बातें अता फ़रमायी हैं और उनको बहुत बड़ी सल्तनत भी दी
\end{hindi}}
\flushright{\begin{Arabic}
\quranayah[4][55]
\end{Arabic}}
\flushleft{\begin{hindi}
फिर कुछ लोग तो इस (किताब) पर ईमान लाए और कुछ लोगों ने उससे इन्कार किया और इसकी सज़ा के लिए जहन्नुम की दहकती हुई आग काफ़ी है
\end{hindi}}
\flushright{\begin{Arabic}
\quranayah[4][56]
\end{Arabic}}
\flushleft{\begin{hindi}
(याद रहे) कि जिन लोगों ने हमारी आयतों से इन्कार किया उन्हें ज़रूर अनक़रीब जहन्नुम की आग में झोंक देंगे (और जब उनकी खालें जल कर) जल जाएंगी तो हम उनके लिए दूसरी खालें बदल कर पैदा करे देंगे ताकि वह अच्छी तरह अज़ाब का मज़ा चखें बेशक ख़ुदा हरचीज़ पर ग़ालिब और हिकमत वाला है
\end{hindi}}
\flushright{\begin{Arabic}
\quranayah[4][57]
\end{Arabic}}
\flushleft{\begin{hindi}
और जो लोग ईमान लाए और अच्छे अच्छे काम किए हम उनको अनक़रीब ही (बेहिश्त के) ऐसे ऐसे (हरे भरे) बाग़ों में जा पहुंचाएंगे जिन के नीचे नहरें जारी होंगी और उनमें हमेशा रहेंगे वहां उनकी साफ़ सुथरी बीवियॉ होंगी और उन्हे घनी छॉव में ले जाकर रखेंगे
\end{hindi}}
\flushright{\begin{Arabic}
\quranayah[4][58]
\end{Arabic}}
\flushleft{\begin{hindi}
ऐ ईमानदारों ख़ुदा तुम्हें हुक्म देता है कि लोगों की अमानतें अमानत रखने वालों के हवाले कर दो और जब लोगों के बाहमी झगड़ों का फैसला करने लगो तो इन्साफ़ से फैसला करो (ख़ुदा तुमको) इसकी क्या ही अच्छी नसीहत करता है इसमें तो शक नहीं कि ख़ुदा सबकी सुनता है (और सब कुछ) देखता है
\end{hindi}}
\flushright{\begin{Arabic}
\quranayah[4][59]
\end{Arabic}}
\flushleft{\begin{hindi}
ऐ ईमानदारों ख़ुदा की इताअत करो और रसूल की और जो तुममें से साहेबाने हुकूमत हों उनकी इताअत करो और अगर तुम किसी बात में झगड़ा करो पस अगर तुम ख़ुदा और रोज़े आख़िरत पर ईमान रखते हो तो इस अम्र में ख़ुदा और रसूल की तरफ़ रूजू करो यही तुम्हारे हक़ में बेहतर है और अन्जाम की राह से बहुत अच्छा है
\end{hindi}}
\flushright{\begin{Arabic}
\quranayah[4][60]
\end{Arabic}}
\flushleft{\begin{hindi}
(ऐ रसूल) क्या तुमने उन लोगों की (हालत) पर नज़र नहीं की जो ये ख्याली पुलाओ पकाते हैं कि जो किताब तुझ पर नाज़िल की गयी और जो किताबें तुम से पहले नाज़िल की गयी (सब पर ईमान है) लाए और दिली तमन्ना ये है कि सरकशों को अपना हाकिम बनाएं हालॉकि उनको हुक्म दिया गया कि उसकी बात न मानें और शैतान तो यह चाहता है कि उन्हें बहका के बहुत दूर ले जाए
\end{hindi}}
\flushright{\begin{Arabic}
\quranayah[4][61]
\end{Arabic}}
\flushleft{\begin{hindi}
और जब उनसे कहा जाता है कि ख़ुदा ने जो किताब नाज़िल की है उसकी तरफ़ और रसूल की तरफ़ रूजू करो तो तुम मुनाफ़िक़ीन को देखते हो कि तुमसे किस तरह मुंह फेर लेते हैं
\end{hindi}}
\flushright{\begin{Arabic}
\quranayah[4][62]
\end{Arabic}}
\flushleft{\begin{hindi}
कि जब उनपर उनके करतूत की वजह से कोई मुसीबत पड़ती है तो क्योंकर तुम्हारे पास ख़ुदा की क़समें खाते हैं कि हमारा मतलब नेकी और मेल मिलाप के सिवा कुछ न था ये वह लोग हैं कि कुछ ख़ुदा ही उनके दिल की हालत ख़ूब जानता है
\end{hindi}}
\flushright{\begin{Arabic}
\quranayah[4][63]
\end{Arabic}}
\flushleft{\begin{hindi}
पस तुम उनसे दरगुज़र करो और उनको नसीहत करो और उनसे उनके दिल में असर करने वाली बात कहो और हमने कोई रसूल नहीं भेजा मगर इस वास्ते कि ख़ुदा के हुक्म से लोग उसकी इताअत करें
\end{hindi}}
\flushright{\begin{Arabic}
\quranayah[4][64]
\end{Arabic}}
\flushleft{\begin{hindi}
और (रसूल) जब उन लोगों ने (नाफ़रमानी करके) अपनी जानों पर जुल्म किया था अगर तुम्हारे पास चले आते और ख़ुदा से माफ़ी मॉगते और रसूल (तुम) भी उनकी मग़फ़िरत चाहते तो बेशक वह लोग ख़ुदा को बड़ा तौबा क़ुबूल करने वाला मेहरबान पाते
\end{hindi}}
\flushright{\begin{Arabic}
\quranayah[4][65]
\end{Arabic}}
\flushleft{\begin{hindi}
पस (ऐ रसूल) तुम्हारे परवरदिगार की क़सम ये लोग सच्चे मोमिन न होंगे तावक्ते क़ि अपने बाहमी झगड़ों में तुमको अपना हाकिम (न) बनाएं फिर (यही नहीं बल्कि) जो कुछ तुम फैसला करो उससे किसी तरह दिलतंग भी न हों बल्कि ख़ुशी ख़ुशी उसको मान लें
\end{hindi}}
\flushright{\begin{Arabic}
\quranayah[4][66]
\end{Arabic}}
\flushleft{\begin{hindi}
(इस्लामी शरीयत में तो उनका ये हाल है) और अगर हम बनी इसराइल की तरह उनपर ये हुक्म जारी कर देते कि तुम अपने आपको क़त्ल कर डालो या शहर बदर हो जाओ तो उनमें से चन्द आदमियों के सिवा ये लोग तो उसको न करते और अगर ये लोग इस बात पर अमल करते जिसकी उन्हें नसीहत की जाती है तो उनके हक़ में बहुत बेहतर होता
\end{hindi}}
\flushright{\begin{Arabic}
\quranayah[4][67]
\end{Arabic}}
\flushleft{\begin{hindi}
और (दीन में भी) बहुत साबित क़दमी से जमे रहते और इस सूरत में हम भी अपनी तरफ़ से ज़रूर बड़ा अच्छा बदला देते
\end{hindi}}
\flushright{\begin{Arabic}
\quranayah[4][68]
\end{Arabic}}
\flushleft{\begin{hindi}
और उनको राहे रास्त की भी ज़रूर हिदायत करते
\end{hindi}}
\flushright{\begin{Arabic}
\quranayah[4][69]
\end{Arabic}}
\flushleft{\begin{hindi}
और जिस शख्स ने ख़ुदा और रसूल की इताअत की तो ऐसे लोग उन (मक़बूल) बन्दों के साथ होंगे जिन्हें ख़ुदा ने अपनी नेअमतें दी हैं यानि अम्बिया और सिद्दीक़ीन और शोहदा और सालेहीन और ये लोग क्या ही अच्छे रफ़ीक़ हैं
\end{hindi}}
\flushright{\begin{Arabic}
\quranayah[4][70]
\end{Arabic}}
\flushleft{\begin{hindi}
ये ख़ुदा का फ़ज़ल (व करम) है और ख़ुदा तो वाक़िफ़कारी में बस है
\end{hindi}}
\flushright{\begin{Arabic}
\quranayah[4][71]
\end{Arabic}}
\flushleft{\begin{hindi}
ऐ ईमानवालों (जिहाद के वक्त) अपनी हिफ़ाज़त के (ज़राए) अच्छी तरह देखभाल लो फिर तुम्हें इख्तेयार है ख्वाह दस्ता दस्ता निकलो या सबके सब इकट्ठे होकर निकल खड़े हो
\end{hindi}}
\flushright{\begin{Arabic}
\quranayah[4][72]
\end{Arabic}}
\flushleft{\begin{hindi}
और तुममें से बाज़ ऐसे भी हैं जो (जेहाद से) ज़रूर पीछे रहेंगे फिर अगर इत्तेफ़ाक़न तुमपर कोई मुसीबत आ पड़ी तो कहने लगे ख़ुदा ने हमपर बड़ा फ़ज़ल किया कि उनमें (मुसलमानों) के साथ मौजूद न हुआ
\end{hindi}}
\flushright{\begin{Arabic}
\quranayah[4][73]
\end{Arabic}}
\flushleft{\begin{hindi}
और अगर तुमपर ख़ुदा ने फ़ज़ल किया (और दुश्मन पर ग़ालिब आए) तो इस तरह अजनबी बनके कि गोया तुममें उसमें कभी मोहब्बत ही न थी यूं कहने लगा कि ऐ काश उनके साथ होता तो मैं भी बड़ी कामयाबी हासिल करता
\end{hindi}}
\flushright{\begin{Arabic}
\quranayah[4][74]
\end{Arabic}}
\flushleft{\begin{hindi}
पस जो लोग दुनिया की ज़िन्दगी (जान तक) आख़ेरत के वास्ते दे डालने को मौजूद हैं उनको ख़ुदा की राह में जेहाद करना चाहिए और जिसने ख़ुदा की राह में जेहाद किया फिर शहीद हुआ तो गोया ग़ालिब आया तो (बहरहाल) हम तो अनक़रीब ही उसको बड़ा अज्र अता फ़रमायेंगे
\end{hindi}}
\flushright{\begin{Arabic}
\quranayah[4][75]
\end{Arabic}}
\flushleft{\begin{hindi}
(और मुसलमानों) तुमको क्या हो गया है कि ख़ुदा की राह में उन कमज़ोर और बेबस मर्दो और औरतों और बच्चों (को कुफ्फ़ार के पंजे से छुड़ाने) के वास्ते जेहाद नहीं करते जो (हालते मजबूरी में) ख़ुदा से दुआएं मॉग रहे हैं कि ऐ हमारे पालने वाले किसी तरह इस बस्ती (मक्का) से जिसके बाशिन्दे बड़े ज़ालिम हैं हमें निकाल और अपनी तरफ़ से किसी को हमारा सरपरस्त बना और तू ख़ुद ही किसी को अपनी तरफ़ से हमारा मददगार बना
\end{hindi}}
\flushright{\begin{Arabic}
\quranayah[4][76]
\end{Arabic}}
\flushleft{\begin{hindi}
(पस देखो) ईमानवाले तो ख़ुदा की राह में लड़ते हैं और कुफ्फ़ार शैतान की राह में लड़ते मरते हैं पस (मुसलमानों) तुम शैतान के हवा ख़ाहों से लड़ो और (कुछ परवाह न करो) क्योंकि शैतान का दाओ तो बहुत ही बोदा है
\end{hindi}}
\flushright{\begin{Arabic}
\quranayah[4][77]
\end{Arabic}}
\flushleft{\begin{hindi}
(ऐ रसूल) क्या तुमने उन लोगों (के हाल) पर नज़र नहीं की जिनको (जेहाद की आरज़ू थी) और उनको हुक्म दिया गया था कि (अभी) अपने हाथ रोके रहो और पाबन्दी से नमाज़ पढ़ो और ज़कात दिए जाओ मगर जब जिहाद (उनपर वाजिब किया गया तो) उनमें से कुछ लोग (बोदेपन में) लोगों से इस तरह डरने लगे जैसे कोई ख़ुदा से डरे बल्कि उससे कहीं ज्यादा और (घबराकर) कहने लगे ख़ुदाया तूने हमपर जेहाद क्यों वाजिब कर दिया हमको कुछ दिनों की और मोहलत क्यों न दी (ऐ रसूल) उनसे कह दो कि दुनिया की आसाइश बहुत थोड़ा सा है और जो (ख़ुदा से) डरता है उसकी आख़ेरत उससे कहीं बेहतर है
\end{hindi}}
\flushright{\begin{Arabic}
\quranayah[4][78]
\end{Arabic}}
\flushleft{\begin{hindi}
और वहां रेशा (बाल) बराबर भी तुम लोगों पर जुल्म नहीं किया जाएगा तुम चाहे जहॉ हो मौत तो तुमको ले डालेगी अगरचे तुम कैसे ही मज़बूत पक्के गुम्बदों में जा छुपो और उनको अगर कोई भलाई पहुंचती है तो कहने लगते हैं कि ये ख़ुदा की तरफ़ से है और अगर उनको कोई तकलीफ़ पहुंचती है तो (शरारत से) कहने लगते हैं कि (ऐ रसूल) ये तुम्हारी बदौलत है (ऐ रसूल) तुम कह दो कि सब ख़ुदा की तरफ़ से है पस उन लोगों को क्या हो गया है कि र्कोई बात ही नहीं समझते
\end{hindi}}
\flushright{\begin{Arabic}
\quranayah[4][79]
\end{Arabic}}
\flushleft{\begin{hindi}
हालॉकि (सच तो यूं है कि) जब तुमको कोई फ़ायदा पहुंचे तो (समझो कि) ख़ुदा की तरफ़ से है और जब तुमको कोई फ़ायदा पहुंचे तो (समझो कि) ख़ुद तुम्हारी बदौलत है और (ऐ रसूल) हमने तुमको लोगों के पास पैग़म्बर बनाकर भेजा है और (इसके लिए) ख़ुदा की गवाही काफ़ी है
\end{hindi}}
\flushright{\begin{Arabic}
\quranayah[4][80]
\end{Arabic}}
\flushleft{\begin{hindi}
जिसने रसूल की इताअत की तो उसने ख़ुदा की इताअत की और जिसने रूगरदानी की तो तुम कुछ ख्याल न करो (क्योंकि) हमने तुम को पासबान (मुक़र्रर) करके तो भेजा नहीं है
\end{hindi}}
\flushright{\begin{Arabic}
\quranayah[4][81]
\end{Arabic}}
\flushleft{\begin{hindi}
(ये लोग तुम्हारे सामने) तो कह देते हैं कि हम (आपके) फ़रमाबरदार हैं लेकिन जब तुम्हारे पास से बाहर निकले तो उनमें से कुछ लोग जो कुछ तुमसे कह चुके थे उसके ख़िलाफ़ रातों को मशवरा करते हैं हालॉकि (ये नहीं समझते) ये लोग रातों को जो कुछ भी मशवरा करते हैं उसे ख़ुदा लिखता जाता है पास तुम उन लोगों की कुछ परवाह न करो और ख़ुदा पर भरोसा रखो और ख़ुदा कारसाज़ी के लिए काफ़ी है
\end{hindi}}
\flushright{\begin{Arabic}
\quranayah[4][82]
\end{Arabic}}
\flushleft{\begin{hindi}
तो क्या ये लोग क़ुरान में भी ग़ौर नहीं करते और (ये नहीं ख्याल करते कि) अगर ख़ुदा के सिवा किसी और की तरफ़ से (आया) होता तो ज़रूर उसमें बड़ा इख्तेलाफ़ पाते
\end{hindi}}
\flushright{\begin{Arabic}
\quranayah[4][83]
\end{Arabic}}
\flushleft{\begin{hindi}
और जब उनके (मुसलमानों के) पास अमन या ख़ौफ़ की ख़बर आयी तो उसे फ़ौरन मशहूर कर देते हैं हालॉकि अगर वह उसकी ख़बर को रसूल (या) और ईमानदारो में से जो साहबाने हुकूमत तक पहुंचाते तो बेशक जो लोग उनमें से उसकी तहक़ीक़ करने वाले हैं (पैग़म्बर या वली) उसको समझ लेते कि (मशहूर करने की ज़रूरत है या नहीं) और (मुसलमानों) अगर तुमपर ख़ुदा का फ़ज़ल (व करम) और उसकी मेहरबानी न होती तो चन्द आदमियों के सिवा तुम सबके सब शैतान की पैरवी करने लगते
\end{hindi}}
\flushright{\begin{Arabic}
\quranayah[4][84]
\end{Arabic}}
\flushleft{\begin{hindi}
पस (ऐ रसूल) तुम ख़ुदा की राह में जिहाद करो और तुम अपनी ज़ात के सिवा किसी और के ज़िम्मेदार नहीं हो और ईमानदारों को (जेहाद की) तरग़ीब दो और अनक़रीब ख़ुदा काफ़िरों की हैबत रोक देगा और ख़ुदा की हैबत सबसे ज्यादा है और उसकी सज़ा बहुत सख्त है
\end{hindi}}
\flushright{\begin{Arabic}
\quranayah[4][85]
\end{Arabic}}
\flushleft{\begin{hindi}
जो शख्स अच्छे काम की सिफ़ारिश करे तो उसको भी उस काम के सवाब से कुछ हिस्सा मिलेगा और जो बुरे काम की सिफ़ारिश करे तो उसको भी उसी काम की सज़ा का कुछ हिस्सा मिलेगा और ख़ुदा तो हर चीज़ पर निगेहबान है
\end{hindi}}
\flushright{\begin{Arabic}
\quranayah[4][86]
\end{Arabic}}
\flushleft{\begin{hindi}
और जब कोई शख्स सलाम करे तो तुम भी उसके जवाब में उससे बेहतर तरीक़े से सलाम करो या वही लफ्ज़ जवाब में कह दो बेशक ख़ुदा हर चीज़ का हिसाब करने वाला है
\end{hindi}}
\flushright{\begin{Arabic}
\quranayah[4][87]
\end{Arabic}}
\flushleft{\begin{hindi}
अल्लाह तो वही परवरदिगार है जिसके सिवा कोई क़ाबिले परस्तिश नहीं वह तुमको क़यामत के दिन जिसमें ज़रा भी शक नहीं ज़रूर इकट्ठा करेगा और ख़ुदा से बढ़कर बात में सच्चा कौन होगा
\end{hindi}}
\flushright{\begin{Arabic}
\quranayah[4][88]
\end{Arabic}}
\flushleft{\begin{hindi}
(मुसलमानों) फिर तुमको क्या हो गया है कि तुम मुनाफ़िक़ों के बारे में दो फ़रीक़ हो गए हो (एक मुवाफ़िक़ एक मुख़ालिफ़) हालॉकि ख़ुद ख़ुदा ने उनके करतूतों की बदौलत उनकी अक्लों को उलट पुलट दिया है क्या तुम ये चाहते हो कि जिसको ख़ुदा ने गुमराही में छोड़ दिया है तुम उसे राहे रास्त पर ले आओ हालॉकि ख़ुदा ने जिसको गुमराही में छोड़ दिया है उसके लिए तुममें से कोई शख्स रास्ता निकाल ही नहीं सकता
\end{hindi}}
\flushright{\begin{Arabic}
\quranayah[4][89]
\end{Arabic}}
\flushleft{\begin{hindi}
उन लोगों की ख्वाहिश तो ये है कि जिस तरह वह काफ़िर हो गए तुम भी काफ़िर हो जाओ ताकि तुम उनके बराबर हो जाओ पस जब तक वह ख़ुदा की राह में हिजरत न करें तो उनमें से किसी को दोस्त न बनाओ फिर अगर वह उससे भी मुंह मोड़ें तो उन्हें गिरफ्तार करो और जहॉ पाओ उनको क़त्ल करो और उनमें से किसी को न अपना दोस्त बनाओ न मददगार
\end{hindi}}
\flushright{\begin{Arabic}
\quranayah[4][90]
\end{Arabic}}
\flushleft{\begin{hindi}
मगर जो लोग किसी ऐसी क़ौम से जा मिलें कि तुममें और उनमें (सुलह का) एहद व पैमान हो चुका है या तुमसे जंग करने या अपनी क़ौम के साथ लड़ने से दिलतंग होकर तुम्हारे पास आए हों (तो उन्हें आज़ार न पहुंचाओ) और अगर ख़ुदा चाहता तो उनको तुमपर ग़लबा देता तो वह तुमसे ज़रूर लड़ पड़ते पस अगर वह तुमसे किनारा कशी करे और तुमसे न लड़े और तुम्हारे पास सुलाह का पैग़ाम दे तो तुम्हारे लिए उन लोगों पर आज़ार पहुंचाने की ख़ुदा ने कोई सबील नहीं निकाली
\end{hindi}}
\flushright{\begin{Arabic}
\quranayah[4][91]
\end{Arabic}}
\flushleft{\begin{hindi}
अनक़रीब तुम कुछ ऐसे और लोगों को भी पाओगे जो चाहते हैं कि तुमसे भी अमन में रहें और अपनी क़ौम से भी अमन में रहें (मगर) जब कभी झगड़े की तरफ़ बुलाए गए तो उसमें औंधे मुंह के बल गिर पड़े पस अगर वह तुमसे न किनारा कशी करें और न तुम्हें सुलह का पैग़ाम दें और न लड़ाई से अपने हाथ रोकें पस उनको पकड़ों और जहॉ पाओ उनको क़त्ल करो और यही वह लोग हैं जिनपर हमने तुम्हें सरीही ग़लबा अता फ़रमाया
\end{hindi}}
\flushright{\begin{Arabic}
\quranayah[4][92]
\end{Arabic}}
\flushleft{\begin{hindi}
और किसी ईमानदार को ये जायज़ नहीं कि किसी मोमिन को जान से मार डाले मगर धोखे से (क़त्ल किया हो तो दूसरी बात है) और जो शख्स किसी मोमिन को धोखे से (भी) मार डाले तो (उसपर) एक ईमानदार गुलाम का आज़ाद करना और मक़तूल के क़राबतदारों को खूंन बहा देना (लाज़िम) है मगर जब वह लोग माफ़ करें फिर अगर मक़तूल उन लोगों में से हो वह जो तुम्हारे दुशमन (काफ़िर हरबी) हैं और ख़ुद क़ातिल मोमिन है तो (सिर्फ) एक मुसलमान ग़ुलाम का आज़ाद करना और अगर मक़तूल उन (काफ़िर) लोगों में का हो जिनसे तुम से एहद व पैमान हो चुका है तो (क़ातिल पर) वारिसे मक़तूल को ख़ून बहा देना और एक बन्दए मोमिन का आज़ाद करना (वाजिब) है फ़िर जो शख्स (ग़ुलाम आज़ाद करने को) न पाये तो उसका कुफ्फ़ारा ख़ुदा की तरफ़ से लगातार दो महीने के रोज़े हैं और ख़ुदा ख़ूब वाकिफ़कार (और) हिकमत वाला है
\end{hindi}}
\flushright{\begin{Arabic}
\quranayah[4][93]
\end{Arabic}}
\flushleft{\begin{hindi}
और जो शख्स किसी मोमिन को जानबूझ के मार डाले (ग़ुलाम की आज़ादी वगैरह उसका कुफ्फ़ारा नहीं बल्कि) उसकी सज़ा दोज़क है और वह उसमें हमेशा रहेगा उसपर ख़ुदा ने (अपना) ग़ज़ब ढाया है और उसपर लानत की है और उसके लिए बड़ा सख्त अज़ाब तैयार कर रखा है
\end{hindi}}
\flushright{\begin{Arabic}
\quranayah[4][94]
\end{Arabic}}
\flushleft{\begin{hindi}
ऐ ईमानदारों जब तुम ख़ुदा की राह में (जेहाद करने को) सफ़र करो तो (किसी के क़त्ल करने में जल्दी न करो बल्कि) अच्छी तरह जॉच कर लिया करो और जो शख्स (इज़हारे इस्लाम की ग़रज़ से) तुम्हे सलाम करे तो तुम बे सोचे समझे न कह दिया करो कि तू ईमानदार नहीं है (इससे ज़ाहिर होता है) कि तुम (फ़क्त) दुनियावी आसाइश की तमन्ना रखते हो मगर इसी बहाने क़त्ल करके लूट लो और ये नहीं समझते कि (अगर यही है) तो ख़ुदा के यहॉ बहुत से ग़नीमतें हैं (मुसलमानों) पहले तुम ख़ुद भी तो ऐसे ही थे फिर ख़ुदा ने तुमपर एहसान किया (कि बेखटके मुसलमान हो गए) ग़रज़ ख़ूब छानबीन कर लिया करो बेशक ख़ुदा तुम्हारे हर काम से ख़बरदार है
\end{hindi}}
\flushright{\begin{Arabic}
\quranayah[4][95]
\end{Arabic}}
\flushleft{\begin{hindi}
माज़ूर लोगों के सिवा जेहाद से मुंह छिपा के घर में बैठने वाले और ख़ुदा की राह में अपने जान व माल से जिहाद करने वाले हरगिज़ बराबर नहीं हो सकते (बल्कि) अपने जान व माल से जिहाद करने वालों को घर बैठे रहने वालें पर ख़ुदा ने दरजे के एतबार से बड़ी फ़ज़ीलत दी है (अगरचे) ख़ुदा ने सब ईमानदारों से (ख्वाह जिहाद करें या न करें) भलाई का वायदा कर लिया है मगर ग़ाज़ियों को खाना नशीनों पर अज़ीम सवाब के एतबार से ख़ुदा ने बड़ी फ़ज़ीलत दी है
\end{hindi}}
\flushright{\begin{Arabic}
\quranayah[4][96]
\end{Arabic}}
\flushleft{\begin{hindi}
(यानी उन्हें) अपनी तरफ़ से बड़े बड़े दरजे और बख्शिश और रहमत (अता फ़रमाएगा) और ख़ुदा तो बड़ा बख्शने वाला मेहरबान है
\end{hindi}}
\flushright{\begin{Arabic}
\quranayah[4][97]
\end{Arabic}}
\flushleft{\begin{hindi}
बेशक जिन लोगों की क़ब्जे रूह फ़रिश्ते ने उस वक़त की है कि (दारूल हरब में पड़े) अपनी जानों पर ज़ुल्म कर रहे थे और फ़रिश्ते कब्जे रूह के बाद हैरत से कहते हैं तुम किस (हालत) ग़फ़लत में थे तो वह (माज़ेरत के लहजे में) कहते है कि हम तो रूए ज़मीन में बेकस थे तो फ़रिश्ते कहते हैं कि ख़ुदा की (ऐसी लम्बी चौड़ी) ज़मीन में इतनी सी गुन्जाइश न थी कि तुम (कहीं) हिजरत करके चले जाते पस ऐसे लोगों का ठिकाना जहन्नुम है और वह बुरा ठिकाना है
\end{hindi}}
\flushright{\begin{Arabic}
\quranayah[4][98]
\end{Arabic}}
\flushleft{\begin{hindi}
मगर जो मर्द और औरतें और बच्चे इस क़दर बेबस हैं कि न तो (दारूल हरब से निकलने की) काई तदबीर कर सकते हैं और उनकी रिहाई की कोई राह दिखाई देती है
\end{hindi}}
\flushright{\begin{Arabic}
\quranayah[4][99]
\end{Arabic}}
\flushleft{\begin{hindi}
तो उम्मीद है कि ख़ुदा ऐसे लोगों से दरगुज़रे और ख़ुदा तो बड़ा माफ़ करने वाला और बख्शने वाला है
\end{hindi}}
\flushright{\begin{Arabic}
\quranayah[4][100]
\end{Arabic}}
\flushleft{\begin{hindi}
और जो शख्स ख़ुदा की राह में हिजरत करेगा तो वह रूए ज़मीन में बा फ़राग़त (चैन से रहने सहने के) बहुत से कुशादा मक़ाम पाएगा और जो शख्स अपने घर से जिलावतन होके ख़ुदा और उसके रसूल की तरफ़ निकल ख़ड़ा हुआ फिर उसे (मंज़िले मक़सूद) तक पहुंचने से पहले मौत आ जाए तो ख़ुदा पर उसका सवाब लाज़िम हो गया और ख़ुदा तो बड़ा बख्श ने वाला मेहरबान है ही
\end{hindi}}
\flushright{\begin{Arabic}
\quranayah[4][101]
\end{Arabic}}
\flushleft{\begin{hindi}
(मुसलमानों जब तुम रूए ज़मीन पर सफ़र करो) और तुमको इस अम्र का ख़ौफ़ हो कि कुफ्फ़ार (असनाए नमाज़ में) तुमसे फ़साद करेंगे तो उसमें तुम्हारे वास्ते कुछ मुज़ाएक़ा नहीं कि नमाज़ में कुछ कम कर दिया करो बेशक कुफ्फ़ार तो तुम्हारे ख़ुल्लम ख़ुल्ला दुश्मन हैं
\end{hindi}}
\flushright{\begin{Arabic}
\quranayah[4][102]
\end{Arabic}}
\flushleft{\begin{hindi}
और (ऐ रसूल) तुम मुसलमानों में मौजूद हो और (लड़ाई हो रही हो) कि तुम उनको नमाज़ पढ़ाने लगो तो (दो गिरोह करके) एक को लड़ाई के वास्ते छोड़ दो (और) उनमें से एक जमाअत तुम्हारे साथ नमाज़ पढ़े और अपने हरबे तैयार अपने साथ लिए रहे फिर जब (पहली रकअत के) सजदे कर (दूसरी रकअत फुरादा पढ़) ले तो तुम्हारे पीछे पुश्त पनाह बनें और दूसरी जमाअत जो (लड़ रही थी और) जब तक नमाज़ नहीं पढ़ने पायी है और (तुम्हारी दूसरी रकअत में) तुम्हारे साथ नमाज़ पढ़े और अपनी हिफ़ाज़त की चीजे अौर अपने हथियार (नमाज़ में साथ) लिए रहे कुफ्फ़ार तो ये चाहते ही हैं कि काश अपने हथियारों और अपने साज़ व सामान से ज़रा भी ग़फ़लत करो तो एक बारगी सबके सब तुम पर टूट पड़ें हॉ अलबत्ता उसमें कुछ मुज़ाएक़ा नहीं कि (इत्तेफ़ाक़न) तुमको बारिश के सबब से कुछ तकलीफ़ पहुंचे या तुम बीमार हो तो अपने हथियार (नमाज़ में) उतार के रख दो और अपनी हिफ़ाज़त करते रहो और ख़ुदा ने तो काफ़िरों के लिए ज़िल्लत का अज़ाब तैयार कर रखा है
\end{hindi}}
\flushright{\begin{Arabic}
\quranayah[4][103]
\end{Arabic}}
\flushleft{\begin{hindi}
फिर जब तुम नमाज़ अदा कर चुको तो उठते बैठते लेटते (ग़रज़ हर हाल में) ख़ुदा को याद करो फिर जब तुम (दुश्मनों से) मुतमईन हो जाओ तो (अपने मअमूल) के मुताबिक़ नमाज़ पढ़ा करो क्योंकि नमाज़ तो ईमानदारों पर वक्त मुतय्यन करके फ़र्ज़ की गयी है
\end{hindi}}
\flushright{\begin{Arabic}
\quranayah[4][104]
\end{Arabic}}
\flushleft{\begin{hindi}
और (मुसलमानों) दुशमनों के पीछा करने में सुस्ती न करो अगर लड़ाई में तुमको तकलीफ़ पहुंचती है तो जैसी तुमको तकलीफ़ पहुंचती है उनको भी वैसी ही अज़ीयत होती है और (तुमको) ये भी (उम्मीद है कि) तुम ख़ुदा से वह वह उम्मीदें रखते हो जो (उनको) नसीब नहीं और ख़ुदा तो सबसे वाक़िफ़ (और) हिकमत वाला है
\end{hindi}}
\flushright{\begin{Arabic}
\quranayah[4][105]
\end{Arabic}}
\flushleft{\begin{hindi}
(ऐ रसूल) हमने तुमपर बरहक़ किताब इसलिए नाज़िल की है कि ख़ुदा ने तुम्हारी हिदायत की है उसी तरह लोगों के दरमियान फ़ैसला करो और ख्यानत करने वालों के तरफ़दार न बनो
\end{hindi}}
\flushright{\begin{Arabic}
\quranayah[4][106]
\end{Arabic}}
\flushleft{\begin{hindi}
और (अपनी उम्मत के लिये) ख़ुदा से मग़फ़िरत की दुआ मॉगों बेशक ख़ुदा बड़ा बख्शने वाला मेहरबान है
\end{hindi}}
\flushright{\begin{Arabic}
\quranayah[4][107]
\end{Arabic}}
\flushleft{\begin{hindi}
और (ऐ रसूल) तुम (उन बदमाशों) की तरफ़ होकर (लोगों से) न लड़ो जो अपने ही (लोगों) से दग़ाबाज़ी करते हैं बेशक ख़ुदा ऐसे शख्स को दोस्त नहीं रखता जो दग़ाबाज़ गुनाहगार हो
\end{hindi}}
\flushright{\begin{Arabic}
\quranayah[4][108]
\end{Arabic}}
\flushleft{\begin{hindi}
लोगों से तो अपनी शरारत छुपाते हैं और (ख़ुदा से नहीं छुपा सकते) हालॉकि वह तो उस वक्त भी उनके साथ साथ है जब वह लोग रातों को (बैठकर) उन बातों के मशवरे करते हैं जिनसे ख़ुदा राज़ी नहीं और ख़ुदा तो उनकी सब करतूतों को (इल्म के अहाते में) घेरे हुए है
\end{hindi}}
\flushright{\begin{Arabic}
\quranayah[4][109]
\end{Arabic}}
\flushleft{\begin{hindi}
(मुसलमानों) ख़बरदार हो जाओ भला दुनिया की (ज़रा सी) ज़िन्दगी में तो तुम उनकी तरफ़ होकर लड़ने खडे हो गए (मगर ये तो बताओ) फिर क़यामत के दिन उनका तरफ़दार बनकर ख़ुदा से कौन लड़ेगा या कौन उनका वकील होगा
\end{hindi}}
\flushright{\begin{Arabic}
\quranayah[4][110]
\end{Arabic}}
\flushleft{\begin{hindi}
और जो शख्स कोई बुरा काम करे या (किसी तरह) अपने नफ्स पर ज़ुल्म करे उसके बाद ख़ुदा से अपनी मग़फ़िरत की दुआ मॉगे तो ख़ुदा को बड़ा बख्शने वाला मेहरबान पाएगा
\end{hindi}}
\flushright{\begin{Arabic}
\quranayah[4][111]
\end{Arabic}}
\flushleft{\begin{hindi}
और जो शख्स कोई गुनाह करता है तो उससे कुछ अपना ही नुक़सान करता है और ख़ुदा तो (हर चीज़ से) वाक़िफ़ (और) बड़ी तदबीर वाला है
\end{hindi}}
\flushright{\begin{Arabic}
\quranayah[4][112]
\end{Arabic}}
\flushleft{\begin{hindi}
और जो शख्स कोई ख़ता या गुनाह करे फिर उसे किसी बेक़सूर के सर थोपे तो उसने एक बड़े (इफ़तेरा) और सरीही गुनाह को अपने ऊपर लाद लिया
\end{hindi}}
\flushright{\begin{Arabic}
\quranayah[4][113]
\end{Arabic}}
\flushleft{\begin{hindi}
और (ऐ रसूल) अगर तुमपर ख़ुदा का फ़ज़ल (व करम) और उसकी मेहरबानी न होती तो उन (बदमाशों) में से एक गिरोह तुमको गुमराह करने का ज़रूर क़सद करता हालॉकि वह लोग बस अपने आप को गुमराह कर रहे हैं और यह लोग तुम्हें कुछ भी ज़रर नहीं पहुंचा सकते और ख़ुदा ही ने तो (मेहरबानी की कि) तुमपर अपनी किताब और हिकमत नाज़िल की और जो बातें तुम नहीं जानते थे तुम्हें सिखा दी और तुम पर तो ख़ुदा का बड़ा फ़ज़ल है
\end{hindi}}
\flushright{\begin{Arabic}
\quranayah[4][114]
\end{Arabic}}
\flushleft{\begin{hindi}
(ऐ रसूल) उनके राज़ की बातों में अक्सर में भलाई (का तो नाम तक) नहीं मगर (हॉ) जो शख्स किसी को सदक़ा देने या अच्छे काम करे या लोगों के दरमियान मेल मिलाप कराने का हुक्म दे (तो अलबत्ता एक बात है) और जो शख्स (महज़) ख़ुदा की ख़ुशनूदी की ख्वाहिश में ऐसे काम करेगा तो हम अनक़रीब ही उसे बड़ा अच्छा बदला अता फरमाएंगे
\end{hindi}}
\flushright{\begin{Arabic}
\quranayah[4][115]
\end{Arabic}}
\flushleft{\begin{hindi}
और जो शख्स राहे रास्त के ज़ाहिर होने के बाद रसूल से सरकशी करे और मोमिनीन के तरीक़े के सिवा किसी और राह पर चले तो जिधर वह फिर गया है हम भी उधर ही फेर देंगे और (आख़िर) उसे जहन्नुम में झोंक देंगे और वह तो बहुत ही बुरा ठिकाना
\end{hindi}}
\flushright{\begin{Arabic}
\quranayah[4][116]
\end{Arabic}}
\flushleft{\begin{hindi}
ख़ुदा बेशक उसको तो नहीं बख्शता कि उसका कोई और शरीक बनाया जाए हॉ उसके सिवा जो गुनाह हो जिसको चाहे बख्श दे और (माज़ अल्लाह) जिसने किसी को ख़ुदा का शरीक बनाया तो वह बस भटक के बहुत दूर जा पड़ा
\end{hindi}}
\flushright{\begin{Arabic}
\quranayah[4][117]
\end{Arabic}}
\flushleft{\begin{hindi}
मुशरेकीन ख़ुदा को छोड़कर बस औरतों ही की परसतिश करते हैं (यानी बुतों की जो उनके) ख्याल में औरतें हैं (दर हक़ीक़त) ये लोग सरकश शैतान की परसतिश करते हैं
\end{hindi}}
\flushright{\begin{Arabic}
\quranayah[4][118]
\end{Arabic}}
\flushleft{\begin{hindi}
जिसपर ख़ुदा ने लानत की है और जिसने (इब्तिदा ही में) कहा था कि (ख़ुदावन्दा) मैं तेरे बन्दों में से कुछ ख़ास लोगों को (अपनी तरफ) ज़रूर ले लूंगा
\end{hindi}}
\flushright{\begin{Arabic}
\quranayah[4][119]
\end{Arabic}}
\flushleft{\begin{hindi}
और फिर उन्हें ज़रूर गुमराह करूंगा और उन्हें बड़ी बड़ी उम्मीदें भी ज़रूर दिलाऊंगा और यक़ीनन उन्हें सिखा दूंगा फिर वो (बुतों के वास्ते) जानवरों के काम ज़रूर चीर फाड़ करेंगे और अलबत्ता उनसे कह दूंगा बस फिर वो (मेरी तालीम के मुवाफ़िक़) ख़ुदा की बनाई हुई सूरत को ज़रूर बदल डालेंगे और (ये याद रहे कि) जिसने ख़ुदा को छोड़कर शैतान को अपना सरपरस्त बनाया तो उसने खुल्लम खुल्ला सख्त घाटा उठाया
\end{hindi}}
\flushright{\begin{Arabic}
\quranayah[4][120]
\end{Arabic}}
\flushleft{\begin{hindi}
शैतान उनसे अच्छे अच्छे वायदे भी करता है (और बड़ी बड़ी) उम्मीदें भी दिलाता है और शैतान उनसे जो कुछ वायदे भी करता है वह बस निरा धोखा (ही धोखा) है
\end{hindi}}
\flushright{\begin{Arabic}
\quranayah[4][121]
\end{Arabic}}
\flushleft{\begin{hindi}
यही तो वह लोग हैं जिनका ठिकाना बस जहन्नुम है और वहॉ से भागने की जगह भी न पाएंगे
\end{hindi}}
\flushright{\begin{Arabic}
\quranayah[4][122]
\end{Arabic}}
\flushleft{\begin{hindi}
और जिन लोगों ने ईमान क़ुबूल किया और अच्छे अच्छे काम किए उन्हें हम अनक़रीब ही (बेहिश्त के) उन (हरे भरे) बाग़ों में जा पहुंचाएगें जिनके (दरख्तों के) नीचे नहरें जारी होंगी और ये लोग उसमें हमेशा आबादुल आबाद तक रहेंगे (ये उनसे) ख़ुदा का पक्का वायदा है और ख़ुदा से ज्यादा (अपनी) बात में सच्चा कौन होगा
\end{hindi}}
\flushright{\begin{Arabic}
\quranayah[4][123]
\end{Arabic}}
\flushleft{\begin{hindi}
न तुम लोगों की आरज़ू से (कुछ काम चल सकता है) न अहले किताब की तमन्ना से कुछ हासिल हो सकता है बल्कि (जैसा काम वैसा दाम) जो बुरा काम करेगा उसे उसका बदला दिया जाएगा और फिर ख़ुदा के सिवा किसी को न तो अपना सरपरस्त पाएगा और न मददगार
\end{hindi}}
\flushright{\begin{Arabic}
\quranayah[4][124]
\end{Arabic}}
\flushleft{\begin{hindi}
और जो शख्स अच्छे अच्छे काम करेगा (ख्वाह) मर्द हो या औरत और ईमानदार (भी) हो तो ऐसे लोग बेहिश्त में (बेखटके) जा पहुंचेंगे और उनपर तिल भी ज़ुल्म न किया जाएगा
\end{hindi}}
\flushright{\begin{Arabic}
\quranayah[4][125]
\end{Arabic}}
\flushleft{\begin{hindi}
और उस शख्स से दीन में बेहतर कौन होगा जिसने ख़ुदा के सामने अपना सरे तसलीम झुका दिया और नेको कार भी है और इबराहीम के तरीके पर चलता है जो बातिल से कतरा कर चलते थे और ख़ुदा ने इब्राहिम को तो अपना ख़लिस दोस्त बना लिया
\end{hindi}}
\flushright{\begin{Arabic}
\quranayah[4][126]
\end{Arabic}}
\flushleft{\begin{hindi}
और जो कुछ आसमानों में है और जो कुछ ज़मीन में है (ग़रज़ सब कुछ) ख़ुदा ही का है और ख़ुदा ही सब चीज़ को (अपनी) कुदरत से घेरे हुए है
\end{hindi}}
\flushright{\begin{Arabic}
\quranayah[4][127]
\end{Arabic}}
\flushleft{\begin{hindi}
(ऐ रसूल) ये लोग तुमसे (यतीम लड़कियों) से निकाह के बारे में फ़तवा तलब करते हैं तुम उनसे कह दो कि ख़ुदा तुम्हें उनसे (निकाह करने) की इजाज़त देता है और जो हुक्म मनाही का कुरान में तुम्हें (पहले) सुनाया जा चुका है वह हक़ीक़तन उन यतीम लड़कियों के वास्ते था जिन्हें तुम उनका मुअय्यन किया हुआ हक़ नहीं देते और चाहते हो (कि यूं ही) उनसे निकाह कर लो और उन कमज़ोर नातवॉ (कमजोर) बच्चों के बारे में हुक्म फ़रमाता है और (वो) ये है कि तुम यतीमों के हुक़ूक़ के बारे में इन्साफ पर क़ायम रहो और (यक़ीन रखो कि) जो कुछ तुम नेकी करोगे तो ख़ुदा ज़रूर वाक़िफ़कार है
\end{hindi}}
\flushright{\begin{Arabic}
\quranayah[4][128]
\end{Arabic}}
\flushleft{\begin{hindi}
और अगर कोई औरत अपने शौहर की ज्यादती व बेतवज्जोही से (तलाक़ का) ख़ौफ़ रखती हो तो मियॉ बीवी के बाहम किसी तरह मिलाप कर लेने में दोनों (में से किसी पर) कुछ गुनाह नहीं है और सुलह तो (बहरहाल) बेहतर है और बुख्ल से तो क़रीब क़रीब हर तबियत के हम पहलू है और अगर तुम नेकी करो और परहेजदारी करो तो ख़ुदा तुम्हारे हर काम से ख़बरदार है (वही तुमको अज्र देगा)
\end{hindi}}
\flushright{\begin{Arabic}
\quranayah[4][129]
\end{Arabic}}
\flushleft{\begin{hindi}
और अगरचे तुम बहुतेरा चाहो (लेकिन) तुममें इतनी सकत तो हरगिज़ नहीं है कि अपनी कई बीवियों में (पूरा पूरा) इन्साफ़ कर सको (मगर) ऐसा भी तो न करो कि (एक ही की तरफ़) हमातन माएल हो जाओ कि (दूसरी को अधड़ में) लटकी हुई छोड़ दो और अगर बाहम मेल कर लो और (ज्यादती से) बचे रहो तो ख़ुदा यक़ीनन बड़ा बख्शने वाला मेहरबान है
\end{hindi}}
\flushright{\begin{Arabic}
\quranayah[4][130]
\end{Arabic}}
\flushleft{\begin{hindi}
और अगर दोनों मियॉ बीवी एक दूसरे से बाज़रिए तलाक़ जुदा हो जाएं तो ख़ुदा अपने वसी ख़ज़ाने से (फ़रागुल बाली अता फ़रमाकर) दोनों को (एक दूसरे से) बेनियाज़ कर देगा और ख़ुदा तो बड़ी गुन्जाइश और तदबीर वाला है और जो कुछ आसमानों में है और जो कुछ ज़मीन में है (ग़रज सब कुछ) ख़ुदा ही का है
\end{hindi}}
\flushright{\begin{Arabic}
\quranayah[4][131]
\end{Arabic}}
\flushleft{\begin{hindi}
और जिन लोगों को तुमसे पहले किताबे ख़ुदा अता की गयी है उनको और तुमको भी उसकी हमने वसीयत की थी कि (ख़ुदा) (की नाफ़रमानी) से डरते रहो और अगर (कहीं) तुमने कुफ़्र इख्तेयार किया तो (याद रहे कि) जो कुछ आसमानों में है और जो कुछ ज़मीन में है (ग़रज सब कुछ) ख़ुदा ही का है (जो चाहे कर सकता है) और ख़ुदा तो सबसे बेपरवा और (हमा सिफ़त) मौसूफ़ हर हम्द वाला है
\end{hindi}}
\flushright{\begin{Arabic}
\quranayah[4][132]
\end{Arabic}}
\flushleft{\begin{hindi}
जो कुछ आसमानों में है और जो कुछ ज़मीन में है (ग़रज सब कुछ) ख़ास ख़ुदा ही का है और ख़ुदा तो कारसाज़ी के लिये काफ़ी है
\end{hindi}}
\flushright{\begin{Arabic}
\quranayah[4][133]
\end{Arabic}}
\flushleft{\begin{hindi}
ऐ लोगों अगर ख़ुदा चाहे तो तुमको (दुनिया के परदे से) बिल्कुल उठा ले और (तुम्हारे बदले) दूसरों को ला (बसाए) और ख़ुदा तो इसपर क़ादिर व तवाना है
\end{hindi}}
\flushright{\begin{Arabic}
\quranayah[4][134]
\end{Arabic}}
\flushleft{\begin{hindi}
और जो शख्स (अपने आमाल का) बदला दुनिया ही में चाहता है तो ख़ुदा के पास दुनिया व आख़िरत दोनों का अज्र मौजूद है और ख़ुदा तो हर शख्स की सुनता और सबको देखता है
\end{hindi}}
\flushright{\begin{Arabic}
\quranayah[4][135]
\end{Arabic}}
\flushleft{\begin{hindi}
ऐ ईमानवालों मज़बूती के साथ इन्साफ़ पर क़ायम रहो और ख़ुदा के लये गवाही दो अगरचे (ये गवाही) ख़ुद तुम्हारे या तुम्हारे मॉ बाप या क़राबतदारों के लिए खिलाफ़ (ही क्यो) न हो ख्वाह मालदार हो या मोहताज (क्योंकि) ख़ुदा तो (तुम्हारी बनिस्बत) उनपर ज्यादा मेहरबान है तो तुम (हक़ से) कतराने में ख्वाहिशे नफ़सियानी की पैरवी न करो और अगर घुमा फिरा के गवाही दोगे या बिल्कुल इन्कार करोगे तो (याद रहे जैसी करनी वैसी भरनी क्योंकि) जो कुछ तुम करते हो ख़ुद उससे ख़ूब वाक़िफ़ है
\end{hindi}}
\flushright{\begin{Arabic}
\quranayah[4][136]
\end{Arabic}}
\flushleft{\begin{hindi}
ऐ ईमानवालों ख़ुदा और उसके रसूल (मोहम्मद) पर और उसकी किताब पर जो उसने अपने रसूल (मोहम्मद) पर नाज़िल की है और उस किताब पर जो उसने पहले नाज़िल की ईमान लाओ और (ये भी याद रहे कि) जो शख्स ख़ुदा और उसके फ़रिश्तों और उसकी किताबों और उसके रसूलों और रोज़े आख़िरत का मुन्किर हुआ तो वह राहे रास्त से भटक के जूर जा पड़ा
\end{hindi}}
\flushright{\begin{Arabic}
\quranayah[4][137]
\end{Arabic}}
\flushleft{\begin{hindi}
बेशक जो लोग ईमान लाए उसके बाद फ़िर काफ़िर हो गए फिर ईमान लाए और फिर उसके बाद काफ़िर हो गये और कुफ़्र में बढ़ते चले गए तो ख़ुदा उनकी मग़फ़िरत करेगा और न उन्हें राहे रास्त की हिदायत ही करेगा
\end{hindi}}
\flushright{\begin{Arabic}
\quranayah[4][138]
\end{Arabic}}
\flushleft{\begin{hindi}
(ऐ रसूल) मुनाफ़िक़ों को ख़ुशख़बरी दे दो कि उनके लिए ज़रूर दर्दनाक अज़ाब है
\end{hindi}}
\flushright{\begin{Arabic}
\quranayah[4][139]
\end{Arabic}}
\flushleft{\begin{hindi}
जो लोग मोमिनों को छोड़कर काफ़िरों को अपना सरपरस्त बनाते हैं क्या उनके पास इज्ज़त (व आबरू) की तलाश करते हैं इज्ज़त सारी बस ख़ुदा ही के लिए ख़ास है
\end{hindi}}
\flushright{\begin{Arabic}
\quranayah[4][140]
\end{Arabic}}
\flushleft{\begin{hindi}
(मुसलमानों) हालॉकि ख़ुदा तुम पर अपनी किताब कुरान में ये हुक्म नाज़िल कर चुका है कि जब तुम सुन लो कि ख़ुदा की आयतों से ईन्कार किया जाता है और उससे मसख़रापन किया जाता है तो तुम उन (कुफ्फ़ार) के साथ मत बैठो यहॉ तक कि वह किसी दूसरी बात में ग़ौर करने लगें वरना तुम भी उस वक्त उनके बराबर हो जाओगे उसमें तो शक ही नहीं कि ख़ुदा तमाम मुनाफ़िक़ों और काफ़िरों को (एक न एक दिन) जहन्नुम में जमा ही करेगा
\end{hindi}}
\flushright{\begin{Arabic}
\quranayah[4][141]
\end{Arabic}}
\flushleft{\begin{hindi}
(वो मुनाफ़ेकीन) जो तुम्हारे मुन्तज़िर है (कि देखिए फ़तेह होती है या शिकस्त) तो अगर ख़ुदा की तरफ़ से तुम्हें फ़तेह हुई तो कहने लगे कि क्या हम तुम्हारे साथ न थे और अगर (फ़तेह का) हिस्सा काफ़िरों को मिला तो (काफ़िरों के तरफ़दार बनकर) कहते हैं क्या हम तुमपर ग़ालिब न आ गए थे (मगर क़सदन तुमको छोड़ दिया) और तुमको मोमिनीन (के हाथों) से हमने बचाया नहीं था (मुनाफ़िक़ों) क़यामत के दिन तो ख़ुदा तुम्हारे दरमियान फैसला करेगा और ख़ुदा ने काफ़िरों को मोमिनीन पर वर (ऊँचा) रहने की हरगिज़ कोई राह नहीं क़रार दी है
\end{hindi}}
\flushright{\begin{Arabic}
\quranayah[4][142]
\end{Arabic}}
\flushleft{\begin{hindi}
बेशक मुनाफ़िक़ीन (अपने ख्याल में) ख़ुदा को फरेब देते हैं हालॉकि ख़ुदा ख़ुद उन्हें धोखा देता है और ये लोग जब नमाज़ पढ़ने खड़े होते हैं तो (बे दिल से) अलकसाए हुए खड़े होते हैं और सिर्फ लोगों को दिखाते हैं और दिल से तो ख़ुदा को कुछ यूं ही सा याद करते हैं
\end{hindi}}
\flushright{\begin{Arabic}
\quranayah[4][143]
\end{Arabic}}
\flushleft{\begin{hindi}
इस कुफ़्र व ईमान के बीच अधड़ में पड़े झूल रहे हैं न उन (मुसलमानों) की तरफ़ न उन काफ़िरों की तरफ़ और (ऐ रसूल) जिसे ख़ुदा गुमराही में छोड़ दे उसकी (हिदायत की) तुम हरगिज़ सबील नहीं कर सकते
\end{hindi}}
\flushright{\begin{Arabic}
\quranayah[4][144]
\end{Arabic}}
\flushleft{\begin{hindi}
ऐ ईमान वालों मोमिनीन को छोड़कर काफ़िरों को (अपना) सरपरस्त न बनाओ क्या ये तुम चाहते हो कि ख़ुदा का सरीही इल्ज़ाम अपने सर क़ायम कर लो
\end{hindi}}
\flushright{\begin{Arabic}
\quranayah[4][145]
\end{Arabic}}
\flushleft{\begin{hindi}
इसमें तो शक ही नहीं कि मुनाफ़िक जहन्नुम के सबसे नीचे तबके में होंगे और (ऐ रसूल) तुम वहॉ किसी को उनका हिमायती भी न पाओगे
\end{hindi}}
\flushright{\begin{Arabic}
\quranayah[4][146]
\end{Arabic}}
\flushleft{\begin{hindi}
मगर (हॉ) जिन लोगों ने (निफ़ाक़ से) तौबा कर ली और अपनी हालत दुरूस्त कर ली और ख़ुदा से लगे लिपटे रहे और अपने दीन को महज़ ख़ुदा के वास्ते निरा खरा कर लिया तो ये लोग मोमिनीन के साथ (बेहिश्त में) होंगे और मोमिनीन को ख़ुदा अनक़रीब ही बड़ा (अच्छा) बदला अता फ़रमाएगा
\end{hindi}}
\flushright{\begin{Arabic}
\quranayah[4][147]
\end{Arabic}}
\flushleft{\begin{hindi}
अगर तुमने ख़ुदा का शुक्र किया और उसपर ईमान लाए तो ख़ुदा तुम पर अज़ाब करके क्या करेगा बल्कि ख़ुदा तो (ख़ुद शुक्र करने वालों का) क़दरदॉ और वाक़िफ़कार है
\end{hindi}}
\flushright{\begin{Arabic}
\quranayah[4][148]
\end{Arabic}}
\flushleft{\begin{hindi}
ख़ुदा (किसी को) हॉक पुकार कर बुरा कहने को पसन्द नहीं करता मगर मज़लूम (ज़ालिम की बुराई बयान कर सकता है) और ख़ुदा तो (सबकी) सुनता है (और हर एक को) जानता है
\end{hindi}}
\flushright{\begin{Arabic}
\quranayah[4][149]
\end{Arabic}}
\flushleft{\begin{hindi}
अगर खुल्लम खुल्ला नेकी करते हो या छिपा कर या किसी की बुराई से दरगुज़र करते हो तो तो ख़ुदा भी बड़ा दरगुज़र करने वाला (और) क़ादिर है
\end{hindi}}
\flushright{\begin{Arabic}
\quranayah[4][150]
\end{Arabic}}
\flushleft{\begin{hindi}
बेशक जो लोग ख़ुदा और उसके रसूलों से इन्कार करते हैं और ख़ुदा और उसके रसूलों में तफ़रक़ा डालना चाहते हैं और कहते हैं कि हम बाज़ (पैग़म्बरों) पर ईमान लाए हैं और बाज़ का इन्कार करते हैं और चाहते हैं कि इस (कुफ़्र व ईमान) के दरमियान एक दूसरी राह निकलें
\end{hindi}}
\flushright{\begin{Arabic}
\quranayah[4][151]
\end{Arabic}}
\flushleft{\begin{hindi}
यही लोग हक़ीक़तन काफ़िर हैं और हमने काफ़िरों के वास्ते ज़िल्लत देने वाला अज़ाब तैयार कर रखा है
\end{hindi}}
\flushright{\begin{Arabic}
\quranayah[4][152]
\end{Arabic}}
\flushleft{\begin{hindi}
और जो लोग ख़ुदा और उसके रसूलों पर ईमान लाए और उनमें से किसी में तफ़रक़ा नहीं करते तो ऐसे ही लोगों को ख़ुदा बहुत जल्द उनका अज्र अता फ़रमाएगा और ख़ुदा तो बड़ा बख्शने वाला मेहरबान है
\end{hindi}}
\flushright{\begin{Arabic}
\quranayah[4][153]
\end{Arabic}}
\flushleft{\begin{hindi}
(ऐ रसूल) अहले किताब (यहूदी) जो तुमसे (ये) दरख्वास्त करते हैं कि तुम उनपर एक किताब आसमान से उतरवा दो (तुम उसका ख्याल न करो क्योंकि) ये लोग मूसा से तो इससे कहीं बढ़ (बढ़) के दरख्वास्त कर चुके हैं चुनान्चे कहने लगे कि हमें ख़ुदा को खुल्लम खुल्ला दिखा दो तब उनकी शरारत की वजह से बिजली ने ले डाला फिर (बावजूद के) उन लोगों के पास तौहीद की वाजैए और रौशन (दलीलें) आ चुकी थी उसके बाद भी उन लोगों ने बछड़े को (ख़ुदा) बना लिया फिर हमने उससे भी दरगुज़र किया और मूसा को हमने सरीही ग़लबा अता किया
\end{hindi}}
\flushright{\begin{Arabic}
\quranayah[4][154]
\end{Arabic}}
\flushleft{\begin{hindi}
और हमने उनके एहद व पैमान की वजह से उनके (सर) पर (कोहे) तूर को लटका दिया और हमने उनसे कहा कि (शहर के) दरवाज़े में सजदा करते हुए दाख़िल हो और हमने (ये भी) कहा कि तुम हफ्ते के दिन (हमारे हुक्म से) तजावुज़ न करना और हमने उनसे बहुत मज़बूत एहदो पैमान ले लिया
\end{hindi}}
\flushright{\begin{Arabic}
\quranayah[4][155]
\end{Arabic}}
\flushleft{\begin{hindi}
फिर उनके अपने एहद तोड़ डालने और एहकामे ख़ुदा से इन्कार करने और नाहक़ अम्बिया को क़त्ल करने और इतरा कर ये कहने की वजह से कि हमारे दिलों पर ग़िलाफ़ चढे हुए हैं (ये तो नहीं) बल्कि ख़ुदा ने उनके कुफ़्र की वजह से उनके दिलों पर मोहर कर दी है तो चन्द आदमियों के सिवा ये लोग ईमान नहीं लाते
\end{hindi}}
\flushright{\begin{Arabic}
\quranayah[4][156]
\end{Arabic}}
\flushleft{\begin{hindi}
और उनके काफ़िर होने और मरियम पर बहुत बड़ा बोहतान बॉधने कि वजह से
\end{hindi}}
\flushright{\begin{Arabic}
\quranayah[4][157]
\end{Arabic}}
\flushleft{\begin{hindi}
और उनके यह कहने की वजह से कि हमने मरियम के बेटे ईसा (स.) ख़ुदा के रसूल को क़त्ल कर डाला हालॉकि न तो उन लोगों ने उसे क़त्ल ही किया न सूली ही दी उनके लिए (एक दूसरा शख्स ईसा) से मुशाबेह कर दिया गया और जो लोग इस बारे में इख्तेलाफ़ करते हैं यक़ीनन वह लोग (उसके हालत) की तरफ़ से धोखे में (आ पड़े) हैं उनको उस (वाक़िये) की ख़बर ही नहीं मगर फ़क्त अटकल के पीछे (पड़े) हैं और ईसा को उन लोगों ने यक़ीनन क़त्ल नहीं किया
\end{hindi}}
\flushright{\begin{Arabic}
\quranayah[4][158]
\end{Arabic}}
\flushleft{\begin{hindi}
बल्कि ख़ुदा ने उन्हें अपनी तरफ़ उठा लिया और ख़ुदा तो बड़ा ज़बरदस्त तदबीर वाला है
\end{hindi}}
\flushright{\begin{Arabic}
\quranayah[4][159]
\end{Arabic}}
\flushleft{\begin{hindi}
और (जब ईसा मेहदी मौऊद के ज़हूर के वक्त आसमान से उतरेंगे तो) अहले किताब में से कोई शख्स ऐसा न होगा जो उनपर उनके मरने के क़ब्ल ईमान न लाए और ख़ुद ईसा क़यामत के दिन उनके ख़िलाफ़ गवाही देंगे
\end{hindi}}
\flushright{\begin{Arabic}
\quranayah[4][160]
\end{Arabic}}
\flushleft{\begin{hindi}
ग़रज़ यहूदियों की (उन सब) शरारतों और गुनाह की वजह से हमने उनपर वह साफ़ सुथरी चीजें ज़ो उनके लिए हलाल की गयी थीं हराम कर दी और उनके ख़ुदा की राह से बहुत से लोगों को रोकने कि वजह से भी
\end{hindi}}
\flushright{\begin{Arabic}
\quranayah[4][161]
\end{Arabic}}
\flushleft{\begin{hindi}
और बावजूद मुमानिअत सूद खा लेने और नाहक़ ज़बरदस्ती लोगों के माल खाने की वजह से उनमें से जिन लोगों ने कुफ़्र इख्तेयार किया उनके वास्ते हमने दर्दनाक अज़ाब तैयार कर रखा है
\end{hindi}}
\flushright{\begin{Arabic}
\quranayah[4][162]
\end{Arabic}}
\flushleft{\begin{hindi}
लेकिन (ऐ रसूल) उनमें से जो लोग इल्म (दीन) में बड़े मज़बूत पाए पर फ़ायज़ हैं वह और ईमान वाले तो जो (किताब) तुमपर नाज़िल हुई है (सब पर ईमान रखते हैं) और से नमाज़ पढ़ते हैं और ज़कात अदा करते हैं और ख़ुदा और रोज़े आख़ेरत का यक़ीन रखते हैं ऐसे ही लोगों को हम अनक़रीब बहुत बड़ा अज्र अता फ़रमाएंगे
\end{hindi}}
\flushright{\begin{Arabic}
\quranayah[4][163]
\end{Arabic}}
\flushleft{\begin{hindi}
(ऐ रसूल) हमने तुम्हारे पास (भी) तो इसी तरह 'वही' भेजी जिस तरह नूह और उसके बाद वाले पैग़म्बरों पर भेजी थी और जिस तरह इबराहीम और इस्माइल और इसहाक़ और याक़ूब और औलादे याक़ूब व ईसा व अय्यूब व युनुस व हारून व सुलेमान के पास 'वही' भेजी थी और हमने दाऊद को ज़ुबूर अता की
\end{hindi}}
\flushright{\begin{Arabic}
\quranayah[4][164]
\end{Arabic}}
\flushleft{\begin{hindi}
जिनका हाल हमने तुमसे पहले ही बयान कर दिया और बहुत से ऐसे रसूल (भेजे) जिनका हाल तुमसे बयान नहीं किया और ख़ुदा ने मूसा से (बहुत सी) बातें भी कीं
\end{hindi}}
\flushright{\begin{Arabic}
\quranayah[4][165]
\end{Arabic}}
\flushleft{\begin{hindi}
और हमने नेक लोगों को बेहिश्त की ख़ुशख़बरी देने वाले और बुरे लोगों को अज़ाब से डराने वाले पैग़म्बर (भेजे) ताकि पैग़म्बरों के आने के बाद लोगों की ख़ुदा पर कोई हुज्जत बाक़ी न रह जाए और ख़ुदा तो बड़ा ज़बरदस्त हकीम है (ये कुफ्फ़ार नहीं मानते न मानें)
\end{hindi}}
\flushright{\begin{Arabic}
\quranayah[4][166]
\end{Arabic}}
\flushleft{\begin{hindi}
मगर ख़ुदा तो इस पर गवाही देता है जो कुछ तुम पर नाज़िल किया है ख़ूब समझ बूझ कर नाज़िल किया है (बल्कि) उसकी गवाही तो फ़रिश्ते तक देते हैं हालॉकि ख़ुदा गवाही के लिए काफ़ी है
\end{hindi}}
\flushright{\begin{Arabic}
\quranayah[4][167]
\end{Arabic}}
\flushleft{\begin{hindi}
बेशक जिन लोगों ने कुफ़्र इख्तेयार किया और ख़ुदा की राह से (लोगों) को रोका वह राहे रास्त से भटक के बहुत दूर जा पडे
\end{hindi}}
\flushright{\begin{Arabic}
\quranayah[4][168]
\end{Arabic}}
\flushleft{\begin{hindi}
बेशक जिन लोगों ने कुफ़्र इख्तेयार किया और (उस पर) ज़ुल्म (भी) करते रहे न तो ख़ुदा उनको बख्शेगा ही और न ही उन्हें किसी तरीक़े की हिदायत करेगा
\end{hindi}}
\flushright{\begin{Arabic}
\quranayah[4][169]
\end{Arabic}}
\flushleft{\begin{hindi}
मगर (हॉ) जहन्नुम का रास्ता (दिखा देगा) जिसमें ये लोग हमेशा (पडे) रहेंगे और ये तो ख़ुदा के वास्ते बहुत ही आसान बात है
\end{hindi}}
\flushright{\begin{Arabic}
\quranayah[4][170]
\end{Arabic}}
\flushleft{\begin{hindi}
ऐ लोगों तुम्हारे पास तुम्हारे परवरदिगार की तरफ़ से रसूल (मोहम्मद) दीने हक़ के साथ आ चुके हैं ईमान लाओ (यही) तुम्हारे हक़ में बेहतर है और अगर इन्कार करोगे तो (समझ रखो कि) जो कुछ ज़मीन और आसमानों में है सब ख़ुदा ही का है और ख़ुदा बड़ा वाक़िफ़कार हकीम है
\end{hindi}}
\flushright{\begin{Arabic}
\quranayah[4][171]
\end{Arabic}}
\flushleft{\begin{hindi}
ऐ अहले किताब अपने दीन में हद (एतदाल) से तजावुज़ न करो और ख़ुदा की शान में सच के सिवा (कोई दूसरी बात) न कहो मरियम के बेटे ईसा मसीह (न ख़ुदा थे न ख़ुदा के बेटे) पस ख़ुदा के एक रसूल और उसके कलमे (हुक्म) थे जिसे ख़ुदा ने मरियम के पास भेज दिया था (कि हामला हो जा) और ख़ुदा की तरफ़ से एक जान थे पस ख़ुदा और उसके रसूलों पर ईमान लाओ और तीन (ख़ुदा) के क़ायल न बनो (तसलीस से) बाज़ रहो (और) अपनी भलाई (तौहीद) का क़सद करो अल्लाह तो बस यकता माबूद है वह उस (नुक्स) से पाक व पाकीज़ा है उसका कोई लड़का हो (उसे लड़के की हाजत ही क्या है) जो कुछ आसमानों में है और जो कुछ ज़मीन में है सब तो उसी का है और ख़ुदा तो कारसाज़ी में काफ़ी है
\end{hindi}}
\flushright{\begin{Arabic}
\quranayah[4][172]
\end{Arabic}}
\flushleft{\begin{hindi}
न तो मसीह ही ख़ुदा का बन्दा होने से हरगिज़ इन्कार कर सकते हैं और न (ख़ुदा के) मुक़र्रर फ़रिश्ते और (याद रहे) जो शख्स उसके बन्दा होने से इन्कार करेगा और शेख़ी करेगा तो अनक़रीब ही ख़ुदा उन सबको अपनी तरफ़ उठा लेगा (और हर एक को उसके काम की सज़ा देगा)
\end{hindi}}
\flushright{\begin{Arabic}
\quranayah[4][173]
\end{Arabic}}
\flushleft{\begin{hindi}
पस जिन लोगों ने ईमान कुबूल किया है और अच्छे (अच्छे) काम किए हैं उनका उन्हें सवाब पूरा पूरा भर देगा बल्कि अपने फ़ज़ल (व करम) से कुछ और ज्यादा ही देगा और लोग उसका बन्दा होने से इन्कार करते थे और शेख़ी करते थे उन्हें तो दर्दनाक अज़ाब में मुब्तिला करेगा और लुत्फ़ ये है कि वह लोग ख़ुदा के सिवा न अपना सरपरस्त ही पाएंगे और न मददगार
\end{hindi}}
\flushright{\begin{Arabic}
\quranayah[4][174]
\end{Arabic}}
\flushleft{\begin{hindi}
ऐ लोगों इसमें तो शक ही नहीं कि तुम्हारे परवरदिगार की तरफ़ से (दीने हक़ की) दलील आ चुकी और हम तुम्हारे पास एक चमकता हुआ नूर नाज़िल कर चुके हैं
\end{hindi}}
\flushright{\begin{Arabic}
\quranayah[4][175]
\end{Arabic}}
\flushleft{\begin{hindi}
पस जो लोग ख़ुदा पर ईमान लाए और उसी से लगे लिपटे रहे तो ख़ुदा भी उन्हें अनक़रीब ही अपनी रहमत व फ़ज़ल के सादाब बाग़ो में पहुंचा देगा और उन्हे अपने हुज़ूरी का सीधा रास्ता दिखा देगा 175
\end{hindi}}
\flushright{\begin{Arabic}
\quranayah[4][176]
\end{Arabic}}
\flushleft{\begin{hindi}
(ऐ रसूल) तुमसे लोग फ़तवा तलब करते हैं तुम कह दो कि कलाला (भाई बहन) के बारे में ख़ुदा तो ख़ुद तुम्हे फ़तवा देता है कि अगर कोई ऐसा शख्स मर जाए कि उसके न कोई लड़का बाला हो (न मॉ बाप) और उसके (सिर्फ) एक बहन हो तो उसका तर्के से आधा होगा (और अगर ये बहन मर जाए) और उसके कोई औलाद न हो (न मॉ बाप) तो उसका वारिस बस यही भाई होगा और अगर दो बहनें (ज्यादा) हों तो उनको (भाई के) तर्के से दो तिहाई मिलेगा और अगर किसी के वारिस भाई बहन दोनों (मिले जुले) हों तो मर्द को औरत के हिस्से का दुगना मिलेगा तुम लोगों के भटकने के ख्याल से ख़ुदा अपने एहकाम वाजेए करके बयान फ़रमाता है और ख़ुदा तो हर चीज़ से वाक़िफ़ है
\end{hindi}}
\chapter{Al-Ma'idah (The Food)}
\begin{Arabic}
\Huge{\centerline{\basmalah}}\end{Arabic}
\flushright{\begin{Arabic}
\quranayah[5][1]
\end{Arabic}}
\flushleft{\begin{hindi}
ऐ ईमानदारों (अपने) इक़रारों को पूरा करो (देखो) तुम्हारे वास्ते चौपाए जानवर हलाल कर दिये गये उन के सिवा जो तुमको पढ़ कर सुनाए जाएंगे हलाल कर दिए गए मगर जब तुम हालते एहराम में हो तो शिकार को हलाल न समझना बेशक ख़ुदा जो चाहता है हुक्म देता है
\end{hindi}}
\flushright{\begin{Arabic}
\quranayah[5][2]
\end{Arabic}}
\flushleft{\begin{hindi}
ऐ ईमानदारों (देखो) न ख़ुदा की निशानियों की बेतौक़ीरी करो और न हुरमत वाले महिने की और न क़ुरबानी की और न पट्टे वाले जानवरों की (जो नज़रे ख़ुदा के लिए निशान देकर मिना में ले जाते हैं) और न ख़ानाए काबा की तवाफ़ (व ज़ियारत) का क़स्द करने वालों की जो अपने परवरदिगार की ख़ुशनूदी और फ़ज़ल (व करम) के जोयाँ हैं और जब तुम (एहराम) खोल दो तो शिकार कर सकते हो और किसी क़बीले की यह अदावत कि तुम्हें उन लोगों ने ख़ानाए काबा (में जाने) से रोका था इस जुर्म में न फॅसवा दे कि तुम उनपर ज्यादती करने लगो और (तुम्हारा तो फ़र्ज यह है कि ) नेकी और परहेज़गारी में एक दूसरे की मदद किया करो और गुनाह और ज्यादती में बाहम किसी की मदद न करो और ख़ुदा से डरते रहो (क्योंकि) ख़ुदा तो यक़ीनन बड़ा सख्त अज़ाब वाला है
\end{hindi}}
\flushright{\begin{Arabic}
\quranayah[5][3]
\end{Arabic}}
\flushleft{\begin{hindi}
(लोगों) मरा हुआ जानवर और ख़ून और सुअर का गोश्त और जिस (जानवर) पर (ज़िबाह) के वक्त ख़ुदा के सिवा किसी दूसरे का नाम लिया जाए और गर्दन मरोड़ा हुआ और चोट खाकर मरा हुआ और जो कुएं (वगैरह) में गिरकर मर जाए और जो सींग से मार डाला गया हो और जिसको दरिन्दे ने फाड़ खाया हो मगर जिसे तुमने मरने के क़ब्ल ज़िबाह कर लो और (जो जानवर) बुतों (के थान) पर चढ़ा कर ज़िबाह किया जाए और जिसे तुम (पाँसे) के तीरों से बाहम हिस्सा बॉटो (ग़रज़ यह सब चीज़ें) तुम पर हराम की गयी हैं ये गुनाह की बात है (मुसलमानों) अब तो कुफ्फ़ार तुम्हारे दीन से (फिर जाने से) मायूस हो गए तो तुम उनसे तो डरो ही नहीं बल्कि सिर्फ मुझी से डरो आज मैंने तुम्हारे दीन को कामिल कर दिया और तुमपर अपनी नेअमत पूरी कर दी और तुम्हारे (इस) दीने इस्लाम को पसन्द किया पस जो शख्स भूख़ में मजबूर हो जाए और गुनाह की तरफ़ माएल भी न हो (और कोई चीज़ खा ले) तो ख़ुदा बेशक बड़ा बख्शने वाला मेहरबान है
\end{hindi}}
\flushright{\begin{Arabic}
\quranayah[5][4]
\end{Arabic}}
\flushleft{\begin{hindi}
(ऐ रसूल) तुमसे लोग पूंछते हैं कि कौन (कौन) चीज़ उनके लिए हलाल की गयी है तुम (उनसे) कह दो कि तुम्हारे लिए पाकीज़ा चीजें हलाल की गयीं और शिकारी जानवर जो तुमने शिकार के लिए सधा रखें है और जो (तरीके) ख़ुदा ने तुम्हें बताये हैं उनमें के कुछ तुमने उन जानवरों को भी सिखाया हो तो ये शिकारी जानवर जिस शिकार को तुम्हारे लिए पकड़ रखें उसको (बेताम्मुल) खाओ और (जानवर को छोंड़ते वक्त) ख़ुदा का नाम ले लिया करो और ख़ुदा से डरते रहो (क्योंकि) इसमें तो शक ही नहीं कि ख़ुदा बहुत जल्द हिसाब लेने वाला है
\end{hindi}}
\flushright{\begin{Arabic}
\quranayah[5][5]
\end{Arabic}}
\flushleft{\begin{hindi}
आज तमाम पाकीज़ा चीजें तुम्हारे लिए हलाल कर दी गयी हैं और अहले किताब की ख़ुश्क चीजें ग़ेहूं (वगैरह) तुम्हारे लिए हलाल हैं और तुम्हारी ख़ुश्क चीजें ग़ेहूं (वगैरह) उनके लिए हलाल हैं और आज़ाद पाक दामन औरतें और उन लोगों में की आज़ाद पाक दामन औरतें जिनको तुमसे पहले किताब दी जा चुकी है जब तुम उनको उनके मेहर दे दो (और) पाक दामिनी का इरादा करो न तो खुल्लम खुल्ला ज़िनाकारी का और न चोरी छिपे से आशनाई का और जिस शख्स ने ईमान से इन्कार किया तो उसका सब किया (धरा) अकारत हो गया और (तुल्फ़ तो ये है कि) आख़ेरत में भी वही घाटे में रहेगा
\end{hindi}}
\flushright{\begin{Arabic}
\quranayah[5][6]
\end{Arabic}}
\flushleft{\begin{hindi}
ऐ ईमानदारों जब तुम नमाज़ के लिये आमादा हो तो अपने मुंह और कोहनियों तक हाथ धो लिया करो और अपने सरों का और टखनों तक पॉवों का मसाह कर लिया करो और अगर तुम हालते जनाबत में हो तो तुम तहारत (ग़ुस्ल) कर लो (हॉ) और अगर तुम बीमार हो या सफ़र में हो या तुममें से किसी को पैख़ाना निकल आए या औरतों से हमबिस्तरी की हो और तुमको पानी न मिल सके तो पाक ख़ाक से तैमूम कर लो यानि (दोनों हाथ मारकर) उससे अपने मुंह और अपने हाथों का मसा कर लो (देखो तो ख़ुदा ने कैसी आसानी कर दी) ख़ुदा तो ये चाहता ही नहीं कि तुम पर किसी तरह की तंगी करे बल्कि वो ये चाहता है कि पाक व पाकीज़ा कर दे और तुमपर अपनी नेअमते पूरी कर दे ताकि तुम शुक्रगुज़ार बन जाओ
\end{hindi}}
\flushright{\begin{Arabic}
\quranayah[5][7]
\end{Arabic}}
\flushleft{\begin{hindi}
और जो एहसानात ख़ुदा ने तुमपर किए हैं उनको और उस (एहद व पैमान) को याद करो जिसका तुमसे पक्का इक़रार ले चुका है जब तुमने कहा था कि हमने (एहकामे ख़ुदा को) सुना और दिल से मान लिया और ख़ुदा से डरते रहो क्योंकि इसमें ज़रा भी शक नहीं कि ख़ुदा दिलों के राज़ से भी बाख़बर है
\end{hindi}}
\flushright{\begin{Arabic}
\quranayah[5][8]
\end{Arabic}}
\flushleft{\begin{hindi}
ऐ ईमानदारों ख़ुदा (की ख़ुशनूदी) के लिए इन्साफ़ के साथ गवाही देने के लिए तैयार रहो और तुम्हें किसी क़बीले की अदावत इस जुर्म में न फॅसवा दे कि तुम नाइन्साफी करने लगो (ख़बरदार बल्कि) तुम (हर हाल में) इन्साफ़ करो यही परहेज़गारी से बहुत क़रीब है और ख़ुदा से डरो क्योंकि जो कुछ तुम करते हो (अच्छा या बुरा) ख़ुदा उसे ज़रूर जानता है
\end{hindi}}
\flushright{\begin{Arabic}
\quranayah[5][9]
\end{Arabic}}
\flushleft{\begin{hindi}
और जिन लोगों ने ईमान क़ुबूल किया और अच्छे अच्छे काम किए ख़ुदा ने वायदा किया है कि उनके लिए (आख़िरत में) मग़फेरत और बड़ा सवाब है
\end{hindi}}
\flushright{\begin{Arabic}
\quranayah[5][10]
\end{Arabic}}
\flushleft{\begin{hindi}
और जिन लोगों ने कुफ़्र इख्तेयार किया और हमारी आयतों को झुठलाया वह जहन्नुमी हैं (
\end{hindi}}
\flushright{\begin{Arabic}
\quranayah[5][11]
\end{Arabic}}
\flushleft{\begin{hindi}
ऐ ईमानदारों ख़ुदा ने जो एहसानात तुमपर किए हैं उनको याद करो और ख़ूसूसन जब एक क़बीले ने तुम पर दस्त दराज़ी का इरादा किया था तो ख़ुदा ने उनके हाथों को तुम तक पहुंचने से रोक दिया और ख़ुदा से डरते रहो और मोमिनीन को ख़ुदा ही पर भरोसा रखना चाहिए
\end{hindi}}
\flushright{\begin{Arabic}
\quranayah[5][12]
\end{Arabic}}
\flushleft{\begin{hindi}
और इसमें भी शक नहीं कि ख़ुदा ने बनी इसराईल से (भी ईमान का) एहद व पैमान ले लिया था और हम (ख़ुदा) ने इनमें के बारह सरदार उनपर मुक़र्रर किए और ख़ुदा ने बनी इसराईल से फ़रमाया था कि मैं तो यक़ीनन तुम्हारे साथ हूं अगर तुम भी पाबन्दी से नमाज़ पढ़ते और ज़कात देते रहो और हमारे पैग़म्बरों पर ईमान लाओ और उनकी मदद करते रहो और ख़ुदा (की ख़ुशनूदी के वास्ते लोगों को) क़र्जे हसना देते रहो तो मैं भी तुम्हारे गुनाह तुमसे ज़रूर दूर करूंगा और तुमको बेहिश्त के उन (हरे भरे ) बाग़ों में जा पहुंचाऊॅगा जिनके (दरख्तों के) नीचे नहरें जारी हैं फिर तुममें से जो शख्स इसके बाद भी इन्कार करे तो यक़ीनन वह राहे रास्त से भटक गया
\end{hindi}}
\flushright{\begin{Arabic}
\quranayah[5][13]
\end{Arabic}}
\flushleft{\begin{hindi}
पस हमने उनकीे एहद शिकनी की वजह से उनपर लानत की और उनके दिलों को (गोया) हमने ख़ुद सख्त बना दिया कि (हमारे) कलमात को उनके असली मायनों से बदल कर दूसरे मायनो में इस्तेमाल करते हैं और जिन जिन बातों की उन्हें नसीहत की गयी थी उनमें से एक बड़ा हिस्सा भुला बैठे और (ऐ रसूल) अब तो उनमें से चन्द आदमियों के सिवा एक न एक की ख्यानत पर बराबर मुत्तेला होते रहते हो तो तुम उन (के क़सूर) को माफ़ कर दो और (उनसे) दरगुज़र करो (क्योंकि) ख़ुदा एहसान करने वालों को ज़रूर दोस्त रखता है
\end{hindi}}
\flushright{\begin{Arabic}
\quranayah[5][14]
\end{Arabic}}
\flushleft{\begin{hindi}
और जो लोग कहते हैं कि हम नसरानी हैं उनसे (भी) हमने ईमान का एहद (व पैमान) लिया था मगर जब जिन जिन बातों की उन्हें नसीहत की गयी थी उनमें से एक बड़ा हिस्सा (रिसालत) भुला बैठे तो हमने भी (उसकी सज़ा में) क़यामत तक उनमें बाहम अदावत व दुशमनी की बुनियाद डाल दी और ख़ुदा उन्हें बहुत जल्द (क़यामत के दिन) बता देगा कि वह क्या क्या करते थे
\end{hindi}}
\flushright{\begin{Arabic}
\quranayah[5][15]
\end{Arabic}}
\flushleft{\begin{hindi}
ऐ अहले किताब तुम्हारे पास हमारा पैगम्बर (मोहम्मद सल्ल) आ चुका जो किताबे ख़ुदा की उन बातों में से जिन्हें तुम छुपाया करते थे बहुतेरी तो साफ़ साफ़ बयान कर देगा और बहुतेरी से (अमदन) दरगुज़र करेगा तुम्हरे पास तो ख़ुदा की तरफ़ से एक (चमकता हुआ) नूर और साफ़ साफ़ बयान करने वाली किताब (कुरान) आ चुकी है
\end{hindi}}
\flushright{\begin{Arabic}
\quranayah[5][16]
\end{Arabic}}
\flushleft{\begin{hindi}
जो लोग ख़ुदा की ख़ुशनूदी के पाबन्द हैं उनको तो उसके ज़रिए से राहे निजात की हिदायत करता है और अपने हुक्म से (कुफ़्र की) तारीकी से निकालकर (ईमान की) रौशनी में लाता है और राहे रास्त पर पहुंचा देता है
\end{hindi}}
\flushright{\begin{Arabic}
\quranayah[5][17]
\end{Arabic}}
\flushleft{\begin{hindi}
जो लोग उसके क़ायल हैं कि मरियम के बेटे मसीह बस ख़ुदा हैं वह ज़रूर काफ़िर हो गए (ऐ रसूल) उनसे पूंछो तो कि भला अगर ख़ुदा मरियम के बेटे मसीह और उनकी मॉ को और जितने लोग ज़मीन में हैं सबको मार डालना चाहे तो कौन ऐसा है जिसका ख़ुदा से भी ज़ोर चले (और रोक दे) और सारे आसमान और ज़मीन में और जो कुछ भी उनके दरमियान में है सब ख़ुदा ही की सल्तनत है जो चाहता है पैदा करता है और ख़ुदा तो हर चीज़ पर क़ादिर है
\end{hindi}}
\flushright{\begin{Arabic}
\quranayah[5][18]
\end{Arabic}}
\flushleft{\begin{hindi}
और नसरानी और यहूदी तो कहते हैं कि हम ही ख़ुदा के बेटे और उसके चहेते हैं (ऐ रसूल) उनसे तुम कह दो (कि अगर ऐसा है) तो फिर तुम्हें तुम्हारे गुनाहों की सज़ा क्यों देता है (तुम्हारा ख्याल लग़ो है) बल्कि तुम भी उसकी मख़लूक़ात से एक बशर हो ख़ुदा जिसे चाहेगा बख़ देगा और जिसको चाहेगा सज़ा देगा आसमान और ज़मीन और जो कुछ उन दोनों के दरमियान में है सब ख़ुदा ही का मुल्क है और सबको उसी की तरफ़ लौट कर जाना है
\end{hindi}}
\flushright{\begin{Arabic}
\quranayah[5][19]
\end{Arabic}}
\flushleft{\begin{hindi}
ऐ अहले किताब जब पैग़म्बरों की आमद में बहुत रूकावट हुई तो हमारा रसूल तुम्हारे पास आया जो एहकामे ख़ुदा को साफ़ साफ़ बयान करता है ताकि तुम कहीं ये न कह बैठो कि हमारे पास तो न कोई ख़ुशख़बरी देने वाला (पैग़म्बर) आया न (अज़ाब से) डराने वाला अब तो (ये नहीं कह सकते क्योंकि) यक़ीनन तुम्हारे पास ख़ुशख़बरी देने वाला और डराने वाला पैग़म्बर आ गया और ख़ुदा हर चीज़ पर क़ादिर है
\end{hindi}}
\flushright{\begin{Arabic}
\quranayah[5][20]
\end{Arabic}}
\flushleft{\begin{hindi}
ऐ रसूल उनको वह वक्त याद (दिलाओ) जब मूसा ने अपनी क़ौम से कहा था कि ऐ मेरी क़ौम जो नेअमते ख़ुदा ने तुमको दी है उसको याद करो इसलिए कि उसने तुम्हीं लोगों से बहुतेरे पैग़म्बर बनाए और तुम ही लोगों को बादशाह (भी) बनाया और तुम्हें वह नेअमतें दी हैं जो सारी ख़ुदायी में किसी एक को न दीं
\end{hindi}}
\flushright{\begin{Arabic}
\quranayah[5][21]
\end{Arabic}}
\flushleft{\begin{hindi}
ऐ मेरी क़ौम (शाम) की उस मुक़द्दस ज़मीन में जाओ जहॉ ख़ुदा ने तुम्हारी तक़दीर में (हुकूमत) लिख दी है और दुशमन के मुक़ाबले पीठ न फेरो (क्योंकि) इसमें तो तुम ख़ुद उलटा घाटा उठाओगे
\end{hindi}}
\flushright{\begin{Arabic}
\quranayah[5][22]
\end{Arabic}}
\flushleft{\begin{hindi}
वह लोग कहने लगे कि ऐ मूसा इस मुल्क में तो बड़े ज़बरदस्त (सरकश) लोग रहते हैं और जब तक वह लोग इसमें से निकल न जाएं हम तो उसमें कभी पॉव भी न रखेंगे हॉ अगर वह लोग ख़ुद इसमें से निकल जाएं तो अलबत्ता हम ज़रूर जाएंगे
\end{hindi}}
\flushright{\begin{Arabic}
\quranayah[5][23]
\end{Arabic}}
\flushleft{\begin{hindi}
(मगर) वह आदमी (यूशा कालिब) जो ख़ुदा का ख़ौफ़ रखते थे और जिनपर ख़ुदा ने ख़ास अपना फ़ज़ल (करम) किया था बेधड़क बोल उठे कि (अरे) उनपर हमला करके (बैतुल मुक़दस के फाटक में तो घुस पड़ो फिर देखो तो यह ऐसे बोदे हैं कि) इधर तुम फाटक में घुसे और (ये सब भाग खड़े हुए और) तुम्हारी जीत हो गयी और अगर सच्चे ईमानदार हो तो ख़ुदा ही पर भरोसा रखो
\end{hindi}}
\flushright{\begin{Arabic}
\quranayah[5][24]
\end{Arabic}}
\flushleft{\begin{hindi}
वह कहने लगे एक मूसा (चाहे जो कुछ हो) जब तक वह लोग इसमें हैं हम तो उसमें हरगिज़ (लाख बरस) पॉव न रखेंगे हॉ तुम जाओ और तुम्हारा ख़ुदा जाए ओर दोनों (जाकर) लड़ो हम तो यहीं जमे बैठे हैं
\end{hindi}}
\flushright{\begin{Arabic}
\quranayah[5][25]
\end{Arabic}}
\flushleft{\begin{hindi}
तब मूसा ने अर्ज़ की ख़ुदावन्दा तू ख़ूब वाक़िफ़ है कि अपनी ज़ाते ख़ास और अपने भाई के सिवा किसी पर मेरा क़ाबू नहीं बस अब हमारे और उन नाफ़रमान लोगों के दरमियान जुदाई डाल दे
\end{hindi}}
\flushright{\begin{Arabic}
\quranayah[5][26]
\end{Arabic}}
\flushleft{\begin{hindi}
हमारा उनका साथ नहीं हो सकता (ख़ुदा ने फ़रमाया) (अच्छा) तो उनकी सज़ा यह है कि उनको चालीस बरस तक की हुकूमत नसीब न होगा (और उस मुद्दते दराज़ तक) यह लोग (मिस्र के) जंगल में सरगरदॉ रहेंगे तो फिर तुम इन बदचलन बन्दों पर अफ़सोस न करना
\end{hindi}}
\flushright{\begin{Arabic}
\quranayah[5][27]
\end{Arabic}}
\flushleft{\begin{hindi}
(ऐ रसूल) तुम इन लोगों से आदम के दो बेटों (हाबील, क़ाबील) का सच्चा क़स्द बयान कर दो कि जब उन दोनों ने ख़ुदा की दरगाह में नियाज़ें चढ़ाई तो (उनमें से) एक (हाबील) की (नज़र तो) क़ुबूल हुई और दूसरे (क़ाबील) की नज़र न क़ुबूल हुई तो (मारे हसद के) हाबील से कहने लगा मैं तो तुझे ज़रूर मार डालूंगा उसने जवाब दिया कि (भाई इसमें अपना क्या बस है) ख़ुदा तो सिर्फ परहेज़गारों की नज़र कुबूल करता है
\end{hindi}}
\flushright{\begin{Arabic}
\quranayah[5][28]
\end{Arabic}}
\flushleft{\begin{hindi}
अगर तुम मेरे क़त्ल के इरादे से मेरी तरफ़ अपना हाथ बढ़ाओगे (तो ख़ैर बढ़ाओ) (मगर) मैं तो तुम्हारे क़त्ल के ख्याल से अपना हाथ बढ़ाने वाला नहीं (क्योंकि) मैं तो उस ख़ुदा से जो सारे जहॉन का पालने वाला है ज़रूर डरता हूं
\end{hindi}}
\flushright{\begin{Arabic}
\quranayah[5][29]
\end{Arabic}}
\flushleft{\begin{hindi}
मैं तो ज़रूर ये चाहता हूं कि मेरे गुनाह और तेरे गुनाह दोनों तेरे सर हो जॉए तो तू (अच्छा ख़ासा) जहन्नुमी बन जाए और ज़ालिमों की तो यही सज़ा है
\end{hindi}}
\flushright{\begin{Arabic}
\quranayah[5][30]
\end{Arabic}}
\flushleft{\begin{hindi}
फिर तो उसके नफ्स ने अपने भाई के क़त्ल पर उसे भड़का ही दिया आख़िर उस (कम्बख्त ने) उसको मार ही डाला तो घाटा उठाने वालों में से हो गया
\end{hindi}}
\flushright{\begin{Arabic}
\quranayah[5][31]
\end{Arabic}}
\flushleft{\begin{hindi}
(तब उसे फ़िक्र हुई कि लाश को क्या करे) तो ख़ुदा ने एक कौवे को भेजा कि वह ज़मीन को कुरेदने लगा ताकि उसे (क़ाबील) को दिखा दे कि उसे अपने भाई की लाश क्योंकर छुपानी चाहिए (ये देखकर) वह कहने लगा हाए अफ़सोस क्या मैं उस से भी आजिज़ हूं कि उस कौवे की बराबरी कर सकॅू कि (बला से यह भी होता) तो अपने भाई की लाश छुपा देता अलगरज़ वह (अपनी हरकत से) बहुत पछताया
\end{hindi}}
\flushright{\begin{Arabic}
\quranayah[5][32]
\end{Arabic}}
\flushleft{\begin{hindi}
इसी सबब से तो हमने बनी इसराईल पर वाजिब कर दिया था कि जो शख्स किसी को न जान के बदले में और न मुल्क में फ़साद फैलाने की सज़ा में (बल्कि नाहक़) क़त्ल कर डालेगा तो गोया उसने सब लोगों को क़त्ल कर डाला और जिसने एक आदमी को जिला दिया तो गोया उसने सब लोगों को जिला लिया और उन (बनी इसराईल) के पास तो हमारे पैग़म्बर (कैसे कैसे) रौशन मौजिज़े लेकर आ चुके हैं (मगर) फिर उसके बाद भी यक़ीनन उसमें से बहुतेरे ज़मीन पर ज्यादतियॉ करते रहे
\end{hindi}}
\flushright{\begin{Arabic}
\quranayah[5][33]
\end{Arabic}}
\flushleft{\begin{hindi}
जो लोग ख़ुदा और उसके रसूल से लड़ते भिड़ते हैं (और एहकाम को नहीं मानते) और फ़साद फैलाने की ग़रज़ से मुल्को (मुल्को) दौड़ते फिरते हैं उनकी सज़ा बस यही है कि (चुन चुनकर) या तो मार डाले जाएं या उन्हें सूली दे दी जाए या उनके हाथ पॉव हेर फेर कर एक तरफ़ का हाथ दूसरी तरफ़ का पॉव काट डाले जाएं या उन्हें (अपने वतन की) सरज़मीन से शहर बदर कर दिया जाए यह रूसवाई तो उनकी दुनिया में हुई और फिर आख़ेरत में तो उनके लिए बहुत बड़ा अज़ाब ही है
\end{hindi}}
\flushright{\begin{Arabic}
\quranayah[5][34]
\end{Arabic}}
\flushleft{\begin{hindi}
मगर (हॉ) जिन लोगों ने इससे पहले कि तुम इनपर क़ाबू पाओ तौबा कर लो तो उनका गुनाह बख्श दिया जाएगा क्योंकि समझ लो कि ख़ुदा बेशक बड़ा बख्शने वाला मेहरबान है
\end{hindi}}
\flushright{\begin{Arabic}
\quranayah[5][35]
\end{Arabic}}
\flushleft{\begin{hindi}
ऐ ईमानदारों ख़ुदा से डरते रहो और उसके (तक़र्रब (क़रीब होने) के) ज़रिये की जुस्तजू में रहो और उसकी राह में जेहाद करो ताकि तुम कामयाब हो जाओ
\end{hindi}}
\flushright{\begin{Arabic}
\quranayah[5][36]
\end{Arabic}}
\flushleft{\begin{hindi}
इसमें शक नहीं कि जिन लोगों ने कुफ़्र इख्तेयार किया अगर उनके पास ज़मीन में जो कुछ (माल ख़ज़ाना) है (वह) सब बल्कि उतना और भी उसके साथ हो कि रोज़े क़यामत के अज़ाब का मुआवेज़ा दे दे (और ख़ुद बच जाए) तब भी (उसका ये मुआवेज़ा) कुबूल न किया जाएगा और उनके लिए दर्दनाक अज़ाब है
\end{hindi}}
\flushright{\begin{Arabic}
\quranayah[5][37]
\end{Arabic}}
\flushleft{\begin{hindi}
वह लोग तो चाहेंगे कि किसी तरह जहन्नुम की आग से निकल भागे मगर वहॉ से तो वह निकल ही नहीं सकते और उनके लिए तो दाएमी अज़ाब है
\end{hindi}}
\flushright{\begin{Arabic}
\quranayah[5][38]
\end{Arabic}}
\flushleft{\begin{hindi}
और चोर ख्वाह मर्द हो या औरत तुम उनके करतूत की सज़ा में उनका (दाहिना) हाथ काट डालो ये (उनकी सज़ा) ख़ुदा की तरफ़ से है और ख़ुदा (तो) बड़ा ज़बरदस्त हिकमत वाला है
\end{hindi}}
\flushright{\begin{Arabic}
\quranayah[5][39]
\end{Arabic}}
\flushleft{\begin{hindi}
हॉ जो अपने गुनाह के बाद तौबा कर ले और अपने चाल चलन दुरूस्त कर लें तो बेशक ख़ुदा भी तौबा कुबूल कर लेता है क्योंकि ख़ुदा तो बड़ा बख्शने वाला मेहरबान है
\end{hindi}}
\flushright{\begin{Arabic}
\quranayah[5][40]
\end{Arabic}}
\flushleft{\begin{hindi}
ऐ शख्स क्या तू नहीं जानता कि सारे आसमान व ज़मीन (ग़रज़ दुनिया जहान) में ख़ास ख़ुदा की हुकूमत है जिसे चाहे अज़ाब करे और जिसे चाहे माफ़ कर दे और ख़ुदा तो हर चीज़ पर क़ादिर है
\end{hindi}}
\flushright{\begin{Arabic}
\quranayah[5][41]
\end{Arabic}}
\flushleft{\begin{hindi}
ऐ रसूल जो लोग कुफ़्र की तरफ़ लपक के चले जाते हैं तुम उनका ग़म न खाओ उनमें बाज़ तो ऐसे हैं कि अपने मुंह से बे तकल्लुफ़ कह देते हैं कि हम ईमान लाए हालॉकि उनके दिल बेईमान हैं और बाज़ यहूदी ऐसे हैं कि (जासूसी की ग़रज़ से) झूठी बातें बहुत (शौक से) सुनते हैं ताकि कुफ्फ़ार के दूसरे गिरोह को जो (अभी तक) तुम्हारे पास नहीं आए हैं सुनाएं ये लोग (तौरैत के) अल्फ़ाज़ की उनके असली मायने (मालूम होने) के बाद भी तहरीफ़ करते हैं (और लोगों से) कहते हैं कि (ये तौरैत का हुक्म है) अगर मोहम्मद की तरफ़ से (भी) तुम्हें यही हुक्म दिया जाय तो उसे मान लेना और अगर यह हुक्म तुमको न दिया जाए तो उससे अलग ही रहना और (ऐ रसूल) जिसको ख़ुदा ख़राब करना चाहता है तो उसके वास्ते ख़ुदा से तुम्हारा कुछ ज़ोर नहीं चल सकता यह लोग तो वही हैं जिनके दिलों को ख़ुदा ने (गुनाहों से) पाक करने का इरादा ही नहीं किया (बल्कि) उनके लिए तो दुनिया में भी रूसवाई है और आख़ेरत में भी (उनके लिए) बड़ा (भारी) अज़ाब होगा
\end{hindi}}
\flushright{\begin{Arabic}
\quranayah[5][42]
\end{Arabic}}
\flushleft{\begin{hindi}
ये (कम्बख्त) झूठी बातों को बड़े शौक़ से सुनने वाले और बड़े ही हरामख़ोर हैं तो (ऐ रसूल) अगर ये लोग तुम्हारे पास (कोई मामला लेकर) आए तो तुमको इख्तेयार है ख्वाह उनके दरमियान फैसला कर दो या उनसे किनाराकशी करो और अगर तुम किनाराकश रहोगे तो (कुछ ख्याल न करो) ये लोग तुम्हारा हरगिज़ कुछ बिगाड़ नहीं सकते और अगर उनमें फैसला करो तो इन्साफ़ से फैसला करो क्योंकि ख़ुदा इन्साफ़ करने वालों को दोस्त रखता है
\end{hindi}}
\flushright{\begin{Arabic}
\quranayah[5][43]
\end{Arabic}}
\flushleft{\begin{hindi}
और जब ख़ुद उनके पास तौरेत है और उसमें ख़ुदा का हुक्म (मौजूद) है तो फिर तुम्हारे पास फैसला कराने को क्यों आते हैं और (लुत्फ़ तो ये है कि) इसके बाद फिर (तुम्हारे हुक्म से) फिर जाते हैं ओर सच तो यह है कि यह लोग ईमानदार ही नहीं हैं
\end{hindi}}
\flushright{\begin{Arabic}
\quranayah[5][44]
\end{Arabic}}
\flushleft{\begin{hindi}
बेशक हम ने तौरेत नाज़िल की जिसमें (लोगों की) हिदायत और नूर (ईमान) है उसी के मुताबिक़ ख़ुदा के फ़रमाबरदार बन्दे (अम्बियाए बनी इसराईल) यहूदियों को हुक्म देते रहे और अल्लाह वाले और उलेमाए (यहूद) भी किताबे ख़ुदा से (हुक्म देते थे) जिसके वह मुहाफ़िज़ बनाए गए थे और वह उसके गवाह भी थे पस (ऐ मुसलमानों) तुम लोगों से (ज़रा भी) न डरो (बल्कि) मुझ ही से डरो और मेरी आयतों के बदले में (दुनिया की दौलत जो दर हक़ीक़त बहुत थोड़ी क़ीमत है) न लो और (समझ लो कि) जो शख्स ख़ुदा की नाज़िल की हुई (किताब) के मुताबिक़ हुक्म न दे तो ऐसे ही लोग काफ़िर हैं
\end{hindi}}
\flushright{\begin{Arabic}
\quranayah[5][45]
\end{Arabic}}
\flushleft{\begin{hindi}
और हम ने तौरेत में यहूदियों पर यह हुक्म फर्ज क़र दिया था कि जान के बदले जान और ऑख के बदले ऑख और नाक के बदले नाक और कान के बदले कान और दॉत के बदले दॉत और जख्म के बदले (वैसा ही) बराबर का बदला (जख्म) है फिर जो (मज़लूम ज़ालिम की) ख़ता माफ़ कर दे तो ये उसके गुनाहों का कफ्फ़ारा हो जाएगा और जो शख्स ख़ुदा की नाज़िल की हुई (किताब) के मुवाफ़िक़ हुक्म न दे तो ऐसे ही लोग ज़ालिम हैं
\end{hindi}}
\flushright{\begin{Arabic}
\quranayah[5][46]
\end{Arabic}}
\flushleft{\begin{hindi}
और हम ने उन्हीं पैग़म्बरों के क़दम ब क़दम मरियम के बेटे ईसा को चलाया और वह इस किताब तौरैत की भी तस्दीक़ करते थे जो उनके सामने (पहले से) मौजूद थी और हमने उनको इन्जील (भी) अता की जिसमें (लोगों के लिए हर तरह की) हिदायत थी और नूर (ईमान) और वह इस किताब तौरेत की जो वक्ते नुज़ूले इन्जील (पहले से) मौजूद थी तसदीक़ करने वाली और परहेज़गारों की हिदायत व नसीहत थी
\end{hindi}}
\flushright{\begin{Arabic}
\quranayah[5][47]
\end{Arabic}}
\flushleft{\begin{hindi}
और इन्जील वालों (नसारा) को जो कुछ ख़ुदा ने (उसमें) नाज़िल किया है उसके मुताबिक़ हुक्म करना चाहिए और जो शख्स ख़ुदा की नाज़िल की हुई (किताब के मुआफ़िक) हुक्म न दे तो ऐसे ही लोग बदकार हैं
\end{hindi}}
\flushright{\begin{Arabic}
\quranayah[5][48]
\end{Arabic}}
\flushleft{\begin{hindi}
और (ऐ रसूल) हमने तुम पर भी बरहक़ किताब नाज़िल की जो किताब (उसके पहले से) उसके वक्त में मौजूद है उसकी तसदीक़ करती है और उसकी निगेहबान (भी) है जो कुछ तुम पर ख़ुदा ने नाज़िल किया है उसी के मुताबिक़ तुम भी हुक्म दो और जो हक़ बात ख़ुदा की तरफ़ से आ चुकी है उससे कतरा के उन लोगों की ख्वाहिशे नफ़सियानी की पैरवी न करो और हमने तुम में हर एक के वास्ते (हस्बे मसलेहते वक्त) एक एक शरीयत और ख़ास तरीक़े पर मुक़र्रर कर दिया और अगर ख़ुदा चाहता तो तुम सब के सब को एक ही (शरीयत की) उम्मत बना देता मगर (मुख़तलिफ़ शरीयतों से) ख़ुदा का मतलब यह था कि जो कुछ तुम्हें दिया है उसमें तुम्हारा इमतेहान करे बस तुम नेकी में लपक कर आगे बढ़ जाओ और (यक़ीन जानो कि) तुम सब को ख़ुदा ही की तरफ़ लौट कर जाना है
\end{hindi}}
\flushright{\begin{Arabic}
\quranayah[5][49]
\end{Arabic}}
\flushleft{\begin{hindi}
तब (उस वक्त) ज़िन बातों में तुम इख्तेलाफ़ करते वह तुम्हें बता देगा और (ऐ रसूल) हम फिर कहते हैं कि जो एहकाम ख़ुदा नाज़िल किए हैं तुम उसके मुताबिक़ फैसला करो और उनकी (बेजा) ख्वाहिशे नफ़सियानी की पैरवी न करो (बल्कि) तुम उनसे बचे रहो (ऐसा न हो) कि किसी हुक्म से जो ख़ुदा ने तुम पर नाज़िल किया है तुमको ये लोग भटका दें फिर अगर ये लोग तुम्हारे हुक्म से मुंह मोड़ें तो समझ लो कि (गोया) ख़ुदा ही की मरज़ी है कि उनके बाज़ गुनाहों की वजह से उन्हें मुसीबत में फॅसा दे और इसमें तो शक ही नहीं कि बहुतेरे लोग बदचलन हैं
\end{hindi}}
\flushright{\begin{Arabic}
\quranayah[5][50]
\end{Arabic}}
\flushleft{\begin{hindi}
क्या ये लोग (ज़मानाए) जाहिलीयत के हुक्म की (तुमसे भी) तमन्ना रखते हैं हालॉकि यक़ीन करने वाले लोगों के वास्ते हुक्मे ख़ुदा से बेहतर कौन होगा
\end{hindi}}
\flushright{\begin{Arabic}
\quranayah[5][51]
\end{Arabic}}
\flushleft{\begin{hindi}
ऐ ईमानदारों यहूदियों और नसरानियों को अपना सरपरस्त न बनाओ (क्योंकि) ये लोग (तुम्हारे मुख़ालिफ़ हैं मगर) बाहम एक दूसरे के दोस्त हैं और (याद रहे कि) तुममें से जिसने उनको अपना सरपरस्त बनाया पस फिर वह भी उन्हीं लोगों में से हो गया बेशक ख़ुदा ज़ालिम लोगों को राहे रास्त पर नहीं लाता
\end{hindi}}
\flushright{\begin{Arabic}
\quranayah[5][52]
\end{Arabic}}
\flushleft{\begin{hindi}
तो (ऐ रसूल) जिन लोगों के दिलों में (नेफ़ाक़ की) बीमारी है तुम उन्हें देखोगे कि उनमें दौड़ दौड़ के मिले जाते हैं और तुमसे उसकी वजह यह बयान करते हैं कि हम तो इससे डरते हैं कि कहीं ऐसा न हो उनके न (मिलने से) ज़माने की गर्दिश में न मुब्तिला हो जाएं तो अनक़रीब ही ख़ुदा (मुसलमानों की) फ़तेह या कोई और बात अपनी तरफ़ से ज़ाहिर कर देगा तब यह लोग इस बदगुमानी पर जो अपने जी में छिपाते थे शर्माएंगे (
\end{hindi}}
\flushright{\begin{Arabic}
\quranayah[5][53]
\end{Arabic}}
\flushleft{\begin{hindi}
और मोमिनीन (जब उन पर नेफ़ाक़ ज़ाहिर हो जाएगा तो) कहेंगे क्या ये वही लोग हैं जो सख्त से सख्त क़समें खाकर (हमसे) कहते थे कि हम ज़रूर तुम्हारे साथ हैं उनका सारा किया धरा अकारत हुआ और सख्त घाटे में आ गए
\end{hindi}}
\flushright{\begin{Arabic}
\quranayah[5][54]
\end{Arabic}}
\flushleft{\begin{hindi}
ऐ ईमानदारों तुममें से जो कोई अपने दीन से फिर जाएगा तो (कुछ परवाह नहीं फिर जाए) अनक़रीब ही ख़ुदा ऐसे लोगों को ज़ाहिर कर देगा जिन्हें ख़ुदा दोस्त रखता होगा और वह उसको दोस्त रखते होंगे ईमानदारों के साथ नर्म और मुन्किर (और) काफ़िरों के साथ सख्त ख़ुदा की राह में जेहाद करेंगे और किसी मलामत करने वाले की मलामत की कुछ परवाह न करेंगे ये ख़ुदा का फ़ज़ल (व करम) है वह जिसे चाहता हे देता है और ख़ुदा तो बड़ी गुन्जाइश वाला वाक़िफ़कार है
\end{hindi}}
\flushright{\begin{Arabic}
\quranayah[5][55]
\end{Arabic}}
\flushleft{\begin{hindi}
(ऐ ईमानदारों) तुम्हारे मालिक सरपरस्त तो बस यही हैं ख़ुदा और उसका रसूल और वह मोमिनीन जो पाबन्दी से नमाज़ अदा करते हैं और हालत रूकूउ में ज़कात देते हैं
\end{hindi}}
\flushright{\begin{Arabic}
\quranayah[5][56]
\end{Arabic}}
\flushleft{\begin{hindi}
और जिस शख्स ने ख़ुदा और रसूल और (उन्हीं) ईमानदारों को अपना सरपरस्त बनाया तो (ख़ुदा के लशकर में आ गया और) इसमें तो शक नहीं कि ख़ुदा ही का लशकर वर (ग़ालिब) रहता है
\end{hindi}}
\flushright{\begin{Arabic}
\quranayah[5][57]
\end{Arabic}}
\flushleft{\begin{hindi}
ऐ ईमानदारों जिन लोगों (यहूद व नसारा) को तुम से पहले किताबे (ख़ुदा तौरेत, इन्जील) दी जा चुकी है उनमें से जिन लोगों ने तुम्हारे दीन को हॅसी खेल बना रखा है उनको और कुफ्फ़ार को अपना सरपरस्त न बनाओ और अगर तुम सच्चे ईमानदार हो तो ख़ुदा ही से डरते रहो
\end{hindi}}
\flushright{\begin{Arabic}
\quranayah[5][58]
\end{Arabic}}
\flushleft{\begin{hindi}
और (उनकी शरारत यहॉ तक पहुंची) कि जब तुम (अज़ान देकर) नमाज़ के वास्ते (लोगों को) बुलाते हो ये लोग नमाज़ को हॅसी खेल बनाते हैं ये इस वजह से कि (लोग बिल्कुल बे अक्ल हैं) और कुछ नहीं समझते
\end{hindi}}
\flushright{\begin{Arabic}
\quranayah[5][59]
\end{Arabic}}
\flushleft{\begin{hindi}
(ऐ रसूल अहले किताब से कहो कि) आख़िर तुम हमसे इसके सिवा और क्या ऐब लगा सकते हो कि हम ख़ुदा पर और जो (किताब) हमारे पास भेजी गयी है और जो हमसे पहले भेजी गयी ईमान लाए हैं और ये तुममें के अक्सर बदकार हैं
\end{hindi}}
\flushright{\begin{Arabic}
\quranayah[5][60]
\end{Arabic}}
\flushleft{\begin{hindi}
(ऐ रसूल) तुम कह दो कि मैं तुम्हें ख़ुदा के नज़दीक सज़ा में इससे कहीं बदतर ऐब बता दूं (अच्छा लो सुनो) जिसपर ख़ुदा ने लानत की हो और उस पर ग़ज़ब ढाया हो और उनमें से किसी को (मसख़ करके) बन्दर और (किसी को) सूअर बना दिया हो और (ख़ुदा को छोड़कर) शैतान की परस्तिश की हो पस ये लोग दरजे में कहीं बदतर और राहे रास्त से भटक के सबसे ज्यादा दूर जा पहुंचे हैं
\end{hindi}}
\flushright{\begin{Arabic}
\quranayah[5][61]
\end{Arabic}}
\flushleft{\begin{hindi}
और (मुसलमानों) जब ये लोग तुम्हारे पास आ जाते हैं तो कहते हैं कि हम तो ईमान लाए हैं हालॉकि वह कुफ़्र ही को साथ लेकर आए और फिर निकले भी तो साथ लिए हुए और जो नेफ़ाक़ वह छुपाए हुए थे ख़ुदा उसे ख़ूब जानता है
\end{hindi}}
\flushright{\begin{Arabic}
\quranayah[5][62]
\end{Arabic}}
\flushleft{\begin{hindi}
(ऐ रसूल) तुम उनमें से बहुतेरों को देखोगे कि गुनाह और सरकशी और हरामख़ोरी की तरफ़ दौड़ पड़ते हैं जो काम ये लोग करते थे वह यक़ीनन बहुत बुरा है
\end{hindi}}
\flushright{\begin{Arabic}
\quranayah[5][63]
\end{Arabic}}
\flushleft{\begin{hindi}
उनको अल्लाह वाले और उलेमा झूठ बोलने और हरामख़ोरी से क्यों नहीं रोकते जो (दरगुज़र) ये लोग करते हैं यक़ीनन बहुत ही बुरी है
\end{hindi}}
\flushright{\begin{Arabic}
\quranayah[5][64]
\end{Arabic}}
\flushleft{\begin{hindi}
और यहूदी कहने लगे कि ख़ुदा का हाथ बॅधा हुआ है (बख़ील हो गया) उन्हीं के हाथ बॉध दिए जाएं और उनके (इस) कहने पर (ख़ुदा की) फिटकार बरसे (ख़ुदा का हाथ बॅधने क्यों लगा) बल्कि उसके दोनों हाथ कुशादा हैं जिस तरह चाहता है ख़र्च करता है और जो (किताब) तुम्हारे पास नाज़िल की गयी है (उनका शक व हसद) उनमें से बहुतेरों को कुफ़्र व सरकशी को और बढ़ा देगा और (गोया) हमने ख़ुद उनके आपस में रोज़े क़यामत तक अदावत और कीने की बुनियाद डाल दी जब ये लोग लड़ाई की आग भड़काते हैं तो ख़ुदा उसको बुझा देता है और रूए ज़मीन में फ़साद फेलाने के लिए दौड़ते फिरते हैं और ख़ुदा फ़सादियों को दोस्त नहीं रखता
\end{hindi}}
\flushright{\begin{Arabic}
\quranayah[5][65]
\end{Arabic}}
\flushleft{\begin{hindi}
और अगर अहले किताब ईमान लाते और (हमसे) डरते तो हम ज़रूर उनके गुनाहों से दरगुज़र करते और उनको नेअमत व आराम (बेहिशत के बाग़ों में) पहुंचा देते
\end{hindi}}
\flushright{\begin{Arabic}
\quranayah[5][66]
\end{Arabic}}
\flushleft{\begin{hindi}
और अगर यह लोग तौरैत और इन्जील और (सहीफ़े) उनके पास उनके परवरदिगार की तरफ़ से नाज़िल किये गए थे (उनके एहकाम को) क़ायम रखते तो ज़रूर (उनके) ऊपर से भी (रिज़क़ बरस पड़ता) और पॉवों के नीचे से भी उबल आता और (ये ख़ूब चैन से) खाते उनमें से कुछ लोग तो एतदाल पर हैं (मगर) उनमें से बहुतेरे जो कुछ करते हैं बुरा ही करते हैं
\end{hindi}}
\flushright{\begin{Arabic}
\quranayah[5][67]
\end{Arabic}}
\flushleft{\begin{hindi}
ऐ रसूल जो हुक्म तुम्हारे परवरदिगार की तरफ़ से तुम पर नाज़िल किया गया है पहुंचा दो और अगर तुमने ऐसा न किया तो (समझ लो कि) तुमने उसका कोई पैग़ाम ही नहीं पहुंचाया और (तुम डरो नहीं) ख़ुदा तुमको लोगों के श्शर से महफ़ूज़ रखेगा ख़ुदा हरगिज़ काफ़िरों की क़ौम को मंज़िले मक़सूद तक नहीं पहुंचाता
\end{hindi}}
\flushright{\begin{Arabic}
\quranayah[5][68]
\end{Arabic}}
\flushleft{\begin{hindi}
(ऐ रसूल) तुम कह दो कि ऐ अहले किताब जब तक तुम तौरेत और इन्जील और जो (सहीफ़े) तुम्हारे परवरदिगार की तरफ़ से तुम पर नाज़िल हुए हैं उनके (एहकाम) को क़ायम न रखोगे उस वक्त तक तुम्हारा मज़बह कुछ भी नहीं और (ऐ रसूल) जो (किताब) तुम्हारे पास तुम्हारे परवरदिगार की तरफ़ से भेजी गयी है (उसका) रश्क (हसद) उनमें से बहुतेरों की सरकशी व कुफ़्र को और बढ़ा देगा तुम काफ़िरों के गिरोह पर अफ़सोस न करना
\end{hindi}}
\flushright{\begin{Arabic}
\quranayah[5][69]
\end{Arabic}}
\flushleft{\begin{hindi}
इसमें तो शक ही नहीं कि मुसलमान हो या यहूदी हकीमाना ख्याल के पाबन्द हों ख्वाह नसरानी (गरज़ कुछ भी हो) जो ख़ुदा और रोज़े क़यामत पर ईमान लाएगा और अच्छे (अच्छे) काम करेगा उन पर अलबत्ता न तो कोई ख़ौफ़ होगा न वह लोग आज़ुर्दा ख़ातिर होंगे
\end{hindi}}
\flushright{\begin{Arabic}
\quranayah[5][70]
\end{Arabic}}
\flushleft{\begin{hindi}
हमने बनी इसराईल से एहद व पैमान ले लिया था और उनके पास बहुत रसूल भी भेजे थे (इस पर भी) जब उनके पास कोई रसूल उनकी मर्ज़ी के ख़िलाफ़ हुक्म लेकर आया तो इन (कम्बख्त) लोगों ने किसी को झुठला दिया और किसी को क़त्ल ही कर डाला
\end{hindi}}
\flushright{\begin{Arabic}
\quranayah[5][71]
\end{Arabic}}
\flushleft{\begin{hindi}
और समझ लिया कि (इसमें हमारे लिए) कोई ख़राबी न होगी पस (गोया) वह लोग (अम्र हक़ से) अंधे और बहरे बन गए (मगर बावजूद इसके) जब इन लोगों ने तौबा की तो फिर ख़ुदा ने उनकी तौबा क़ुबूल कर ली (मगर) फिर (इस पर भी) उनमें से बहुतेरे अंधे और बहरे बन गए और जो कुछ ये लोग कर रहे हैं अल्लाह तो देखता है
\end{hindi}}
\flushright{\begin{Arabic}
\quranayah[5][72]
\end{Arabic}}
\flushleft{\begin{hindi}
जो लोग उसके क़ायल हैं कि मरियम के बेटे ईसा मसीह ख़ुदा हैं वह सब काफ़िर हैं हालॉकि मसीह ने ख़ुद यूं कह दिया था कि ऐ बनी इसराईल सिर्फ उसी ख़ुदा की इबादत करो जो हमारा और तुम्हारा पालने वाला है क्योंकि (याद रखो) जिसने ख़ुदा का शरीक बनाया उस पर ख़ुदा ने बेहिश्त को हराम कर दिया है और उसका ठिकाना जहन्नुम है और ज़ालिमों का कोई मददगार नहीं
\end{hindi}}
\flushright{\begin{Arabic}
\quranayah[5][73]
\end{Arabic}}
\flushleft{\begin{hindi}
जो लोग इसके क़ायल हैं कि ख़ुदा तीन में का (तीसरा) है वह यक़ीनन काफ़िर हो गए (याद रखो कि) ख़ुदाए यकता के सिवा कोई माबूद नहीं और (ख़ुदा के बारे में) ये लोग जो कुछ बका करते हैं अगर उससे बाज़ न आए तो (समझ रखो कि) जो लोग उसमें से (काफ़िर के) काफ़िर रह गए उन पर ज़रूर दर्दनाक अज़ाब नाज़िल होगा
\end{hindi}}
\flushright{\begin{Arabic}
\quranayah[5][74]
\end{Arabic}}
\flushleft{\begin{hindi}
तो ये लोग ख़ुदा की बारगाह में तौबा क्यों नहीं करते और अपने (क़सूरों की) माफ़ी क्यों नहीं मॉगते हालॉकि ख़ुदा तो बड़ा बख्शने वाला मेहरबान है
\end{hindi}}
\flushright{\begin{Arabic}
\quranayah[5][75]
\end{Arabic}}
\flushleft{\begin{hindi}
मरियम के बेटे मसीह तो बस एक रसूल हैं और उनके क़ब्ल (और भी) बहुतेरे रसूल गुज़र चुके हैं और उनकी मॉ भी (ख़ुदा की) एक सच्ची बन्दी थी (और आदमियों की तरह) ये दोनों (के दोनों भी) खाना खाते थे (ऐ रसूल) ग़ौर तो करो हम अपने एहकाम इनसे कैसा साफ़ साफ़ बयान करते हैं
\end{hindi}}
\flushright{\begin{Arabic}
\quranayah[5][76]
\end{Arabic}}
\flushleft{\begin{hindi}
फिर देखो तो कि (उसपर भी उलटे) ये लोग कहॉ भटके जा रहे हैं (ऐ रसूल) तुम कह दो कि क्या तुम ख़ुदा (जैसे क़ादिर व तवाना) को छोड़कर (ऐसी ज़लील) चीज़ की इबादत करते हो जिसको न तो नुक़सान ही इख्तेयार है और न नफ़े का और ख़ुदा तो (सबकी) सुनता (और सब कुछ) जानता है
\end{hindi}}
\flushright{\begin{Arabic}
\quranayah[5][77]
\end{Arabic}}
\flushleft{\begin{hindi}
ऐ रसूल तुम कह दो कि ऐ अहले किताब तुम अपने दीन में नाहक़ ज्यादती न करो और न उन लोगों (अपने बुज़ुगों) की नफ़सियानी ख्वाहिशों पर चलो जो पहले ख़ुद ही गुमराह हो चुके और (अपने साथ और भी) बहुतेरों को गुमराह कर छोड़ा और राहे रास्त से (दूर) भटक गए
\end{hindi}}
\flushright{\begin{Arabic}
\quranayah[5][78]
\end{Arabic}}
\flushleft{\begin{hindi}
बनी इसराईल में से जो लोग काफ़िर थे उन पर दाऊद और मरियम के बेटे ईसा की ज़बानी लानत की गयी ये (लानत उन पर पड़ी तो सिर्फ) इस वजह से कि (एक तो) उन लोगों ने नाफ़रमानी की और (फिर हर मामले में) हद से बढ़ जाते थे
\end{hindi}}
\flushright{\begin{Arabic}
\quranayah[5][79]
\end{Arabic}}
\flushleft{\begin{hindi}
और किसी बुरे काम से जिसको उन लोगों ने किया बाज़ न आते थे (बल्कि उस पर बावजूद नसीहत अड़े रहते) जो काम ये लोग करते थे क्या ही बुरा था
\end{hindi}}
\flushright{\begin{Arabic}
\quranayah[5][80]
\end{Arabic}}
\flushleft{\begin{hindi}
(ऐ रसूल) तुम उन (यहूदियों) में से बहुतेरों को देखोगे कि कुफ्फ़ार से दोस्ती रखते हैं जो सामान पहले से उन लोगों ने ख़ुद अपने वास्ते दुरूस्त किया है किस क़दर बुरा है (जिसका नतीजा ये है) कि (दुनिया में भी) ख़ुदा उन पर गज़बनाक हुआ और (आख़ेरत में भी) हमेशा अज़ाब ही में रहेंगे
\end{hindi}}
\flushright{\begin{Arabic}
\quranayah[5][81]
\end{Arabic}}
\flushleft{\begin{hindi}
और अगर ये लोग ख़ुदा और रसूल पर और जो कुछ उनपर नाज़िल किया गया है ईमान रखते हैं तो हरगिज़ (उनको अपना) दोस्त न बनाते मगर उनमें के बहुतेरे तो बदचलन हैं
\end{hindi}}
\flushright{\begin{Arabic}
\quranayah[5][82]
\end{Arabic}}
\flushleft{\begin{hindi}
(ऐ रसूल) ईमान लाने वालों का दुशमन सबसे बढ़के यहूदियों और मुशरिकों को पाओगे और ईमानदारों का दोस्ती में सबसे बढ़के क़रीब उन लोगों को पाओगे जो अपने को नसारा कहते हैं क्योंकि इन (नसारा) में से यक़ीनी बहुत से आमिल और आबिद हैं और इस सबब से (भी) कि ये लोग हरगिज़ शेख़ी नहीं करते
\end{hindi}}
\flushright{\begin{Arabic}
\quranayah[5][83]
\end{Arabic}}
\flushleft{\begin{hindi}
और तू देखता है कि जब यह लोग (इस कुरान) को सुनते हैं जो हमारे रसूल पर नाज़िल किया गया है तो उनकी ऑंखों से बेसाख्ता (छलक कर) ऑंसू जारी हो जातें है क्योंकि उन्होंने (अम्र) हक़ को पहचान लिया है (और) अर्ज़ करते हैं कि ऐ मेरे पालने वाले हम तो ईमान ला चुके तो (रसूल की) तसदीक़ करने वालों के साथ हमें भी लिख रख
\end{hindi}}
\flushright{\begin{Arabic}
\quranayah[5][84]
\end{Arabic}}
\flushleft{\begin{hindi}
और हमको क्या हो गया है कि हम ख़ुदा और जो हक़ बात हमारे पास आ चुकी है उस पर तो ईमान न लाएँ और (फिर) ख़ुदा से उम्मीद रखें कि वह अपने नेक बन्दों के साथ हमें (बेहिश्त में) पहुँचा ही देगा
\end{hindi}}
\flushright{\begin{Arabic}
\quranayah[5][85]
\end{Arabic}}
\flushleft{\begin{hindi}
तो ख़ुदा ने उन्हें उनके (सदक़ दिल से) अर्ज़ करने के सिले में उन्हें वह (हरे भरे) बाग़ात अता फरमाए जिनके (दरख्तों के) नीचे नहरें जारी हैं (और) वह उसमें हमेशा रहेंगे और (सदक़ दिल से) नेकी करने वालों का यही ऐवज़ है
\end{hindi}}
\flushright{\begin{Arabic}
\quranayah[5][86]
\end{Arabic}}
\flushleft{\begin{hindi}
और जिन लोगों ने कुफ्र एख्तेयार किया और हमारी आयतों को झुठलाया यही लोग जहन्नुमी हैं
\end{hindi}}
\flushright{\begin{Arabic}
\quranayah[5][87]
\end{Arabic}}
\flushleft{\begin{hindi}
ऐ ईमानदार जो पाक चीज़े ख़ुदा ने तुम्हारे वास्ते हलाल कर दी हैं उनको अपने ऊपर हराम न करो और हद से न बढ़ो क्यों कि ख़ुदा हद से बढ़ जाने वालों को हरगिज़ दोस्त नहीं रखता
\end{hindi}}
\flushright{\begin{Arabic}
\quranayah[5][88]
\end{Arabic}}
\flushleft{\begin{hindi}
और जो हलाल साफ सुथरी चीज़ें ख़ुदा ने तुम्हें दी हैं उनको (शौक़ से) खाओ और जिस ख़ुदा पर तुम ईमान लाए हो उससे डरते रहो
\end{hindi}}
\flushright{\begin{Arabic}
\quranayah[5][89]
\end{Arabic}}
\flushleft{\begin{hindi}
ख़ुदा तुम्हारे बेकार (बेकार) क़समों (के खाने) पर तो ख़ैर गिरफ्तार न करेगा मगर बाक़सद (सच्ची) पक्की क़सम खाने और उसके ख़िलाफ करने पर तो ज़रुर तुम्हारी ले दे करेगा (लो सुनो) उसका जुर्माना जैसा तुम ख़ुद अपने एहलोअयाल को खिलाते हो उसी क़िस्म का औसत दर्जे का दस मोहताजों को खाना खिलाना या उनको कपड़े पहनाना या एक गुलाम आज़ाद करना है फिर जिससे यह सब न हो सके तो मैं तीन दिन के रोज़े (रखना) ये (तो) तुम्हारी क़समों का जुर्माना है जब तुम क़सम खाओ (और पूरी न करो) और अपनी क़समों (के पूरा न करने) का ख्याल रखो ख़ुदा अपने एहकाम को तुम्हारे वास्ते यूँ साफ़ साफ़ बयान करता है ताकि तुम शुक्र करो
\end{hindi}}
\flushright{\begin{Arabic}
\quranayah[5][90]
\end{Arabic}}
\flushleft{\begin{hindi}
ऐ ईमानदारों शराब, जुआ और बुत और पाँसे तो बस नापाक (बुरे) शैतानी काम हैं तो तुम लोग इससे बचे रहो ताकि तुम फलाह पाओ
\end{hindi}}
\flushright{\begin{Arabic}
\quranayah[5][91]
\end{Arabic}}
\flushleft{\begin{hindi}
शैतान की तो बस यही तमन्ना है कि शराब और जुए की बदौलत तुममें बाहम अदावत व दुशमनी डलवा दे और ख़ुदा की याद और नमाज़ से बाज़ रखे तो क्या तुम उससे बाज़ आने वाले हो
\end{hindi}}
\flushright{\begin{Arabic}
\quranayah[5][92]
\end{Arabic}}
\flushleft{\begin{hindi}
और ख़ुदा का हुक्म मानों और रसूल का हुक्म मानों और (नाफ़रमानी) से बचे रहो इस पर भी अगर तुमने (हुक्म ख़ुदा से) मुँह फेरा तो समझ रखो कि हमारे रसूल पर बस साफ़ साफ़ पैग़ाम पहुँचा देना फर्ज है
\end{hindi}}
\flushright{\begin{Arabic}
\quranayah[5][93]
\end{Arabic}}
\flushleft{\begin{hindi}
(फिर करो चाहे न करो तुम मुख़तार हो) जिन लोगों ने ईमान कुबूल किया और अच्छे (अच्छे) काम किए हैं उन पर जो कुछ खा (पी) चुके उसमें कुछ गुनाह नहीं जब उन्होंने परहेज़गारी की और ईमान ले आए और अच्छे (अच्छे) काम किए फिर परहेज़गारी की और नेकियाँ कीं और ख़ुदा नेकी करने वालों को दोस्त रखता है
\end{hindi}}
\flushright{\begin{Arabic}
\quranayah[5][94]
\end{Arabic}}
\flushleft{\begin{hindi}
ऐ ईमानदारों कुछ शिकार से जिन तक तुम्हारे हाथ और नैज़ें पहुँच सकते हैं ख़ुदा ज़रुर इम्तेहान करेगा ताकि ख़ुदा देख ले कि उससे बे देखे भाले कौन डरता है फिर उसके बाद भी जो ज्यादती करेगा तो उसके लिए दर्दनाक अज़ाब है
\end{hindi}}
\flushright{\begin{Arabic}
\quranayah[5][95]
\end{Arabic}}
\flushleft{\begin{hindi}
(ऐ ईमानदारों जब तुम हालते एहराम में हो तो शिकार न मारो और तुममें से जो कोई जान बूझ कर शिकार मारेगा तो जिस (जानवर) को मारा है चौपायों में से उसका मसल तुममें से जो दो मुन्सिफ आदमी तजवीज़ कर दें उसका बदला (देना) होगा (और) काबा तक पहुँचा कर कुर्बानी की जाए या (उसका) जुर्माना (उसकी क़ीमत से) मोहताजों को खाना खिलाना या उसके बराबर रोज़े रखना (यह जुर्माना इसलिए है) ताकि अपने किए की सज़ा का मज़ा चखो जो हो चुका उससे तो ख़ुदा ने दरग़ुज़र की और जो फिर ऐसी हरकत करेगा तो ख़ुदा उसकी सज़ा देगा और ख़ुदा ज़बरदस्त बदला लेने वाला है
\end{hindi}}
\flushright{\begin{Arabic}
\quranayah[5][96]
\end{Arabic}}
\flushleft{\begin{hindi}
तुम्हारे और काफ़िले के वास्ते दरियाई शिकार और उसका खाना तो (हर हालत में) तुम्हारे वास्ते जायज़ कर दिया है मगर खुश्की का शिकार जब तक तुम हालते एहराम में रहो तुम पर हराम है और उस ख़ुदा से डरते रहो जिसकी तरफ (मरने के बाद) उठाए जाओगे
\end{hindi}}
\flushright{\begin{Arabic}
\quranayah[5][97]
\end{Arabic}}
\flushleft{\begin{hindi}
ख़ुदा ने काबा को जो (उसका) मोहतरम घर है और हुरमत दार महीनों को और कुरबानी को और उस जानवर को जिसके गले में (क़ुरबानी के वास्ते) पट्टे डाल दिए गए हों लोगों के अमन क़ायम रखने का सबब क़रार दिया यह इसलिए कि तुम जान लो कि ख़ुदा जो कुछ आसमानों में है और जो कुछ ज़मीन में है यक़ीनन (सब) जानता है और ये भी (समझ लो) कि बेशक ख़ुदा हर चीज़ से वाक़िफ है
\end{hindi}}
\flushright{\begin{Arabic}
\quranayah[5][98]
\end{Arabic}}
\flushleft{\begin{hindi}
जान लो कि यक़ीनन ख़ुदा बड़ा अज़ाब वाला है और ये (भी) कि बड़ा बख्शने वाला मेहरबान है
\end{hindi}}
\flushright{\begin{Arabic}
\quranayah[5][99]
\end{Arabic}}
\flushleft{\begin{hindi}
(हमारे) रसूल पर पैग़ाम पहुँचा देने के सिवा (और) कुछ (फर्ज़) नहीं और जो कुछ तुम ज़ाहिर बा ज़ाहिर करते हो और जो कुछ तुम छुपा कर करते हो ख़ुदा सब जानता है
\end{hindi}}
\flushright{\begin{Arabic}
\quranayah[5][100]
\end{Arabic}}
\flushleft{\begin{hindi}
(ऐ रसूल) कह दो कि नापाक (हराम) और पाक (हलाल) बराबर नहीं हो सकता अगरचे नापाक की कसरत तुम्हें भला क्यों न मालूम हो तो ऐसे अक्लमन्दों अल्लाह से डरते रहो ताकि तुम कामयाब रहो
\end{hindi}}
\flushright{\begin{Arabic}
\quranayah[5][101]
\end{Arabic}}
\flushleft{\begin{hindi}
ऐ ईमान वालों ऐसी चीज़ों के बारे में (रसूल से) न पूछा करो कि अगर तुमको मालूम हो जाए तो तुम्हें बुरी मालूम हो और अगर उनके बारे में कुरान नाज़िल होने के वक्त पूछ बैठोगे तो तुम पर ज़ाहिर कर दी जाएगी (मगर तुमको बुरा लगेगा जो सवालात तुम कर चुके) ख़ुदा ने उनसे दरगुज़र की और ख़ुदा बड़ा बख्शने वाला बुर्दबार है
\end{hindi}}
\flushright{\begin{Arabic}
\quranayah[5][102]
\end{Arabic}}
\flushleft{\begin{hindi}
तुमसे पहले भी लोगों ने इस किस्म की बातें (अपने वक्त क़े पैग़म्बरों से) पूछी थीं
\end{hindi}}
\flushright{\begin{Arabic}
\quranayah[5][103]
\end{Arabic}}
\flushleft{\begin{hindi}
फिर (जब बरदाश्त न हो सका तो) उसके मुन्किर हो गए ख़ुदा ने न तो कोई बहीरा (कान फटी ऊँटनी) मुक़र्रर किया है न सायवा (साँढ़) न वसीला (जुडवा बच्चे) न हाम (बुढ़ा साँढ़) मुक़र्रर किया है मगर कुफ्फ़ार ख़ुदा पर ख्वाह मा ख्वाह झूठ (मूठ) बोहतान बाँधते हैं और उनमें के अक्सर नहीं समझते
\end{hindi}}
\flushright{\begin{Arabic}
\quranayah[5][104]
\end{Arabic}}
\flushleft{\begin{hindi}
और जब उनसे कहा जाता है कि जो (क़ुरान) ख़ुदा ने नाज़िल फरमाया है उसकी तरफ और रसूल की आओ (और जो कुछ कहे उसे मानों तो कहते हैं कि हमने जिस (रंग) में अपने बाप दादा को पाया वही हमारे लिए काफी है क्या (ये लोग लकीर के फकीर ही रहेंगे) अगरचे उनके बाप दादा न कुछ जानते ही हों न हिदायत ही पायी हो
\end{hindi}}
\flushright{\begin{Arabic}
\quranayah[5][105]
\end{Arabic}}
\flushleft{\begin{hindi}
ऐ ईमान वालों तुम अपनी ख़बर लो जब तुम राहे रास्त पर हो तो कोई गुमराह हुआ करे तुम्हें नुक़सान नहीं पहुँचा सकता तुम सबके सबको ख़ुदा ही की तरफ लौट कर जाना है तब (उस वक्त नेक व बद) जो कुछ (दुनिया में) करते थे तुम्हें बता देगा
\end{hindi}}
\flushright{\begin{Arabic}
\quranayah[5][106]
\end{Arabic}}
\flushleft{\begin{hindi}
ऐ ईमान वालों जब तुममें से किसी (के सर) पर मौत खड़ी हो तो वसीयत के वक्त तुम (मोमिन) में से दो आदिलों की गवाही होनी ज़रुरी है और जब तुम इत्तेफाक़न कहीं का सफर करो और (सफर ही में) तुमको मौत की मुसीबत का सामना हो तो (भी) दो गवाह ग़ैर (मोमिन) सही (और) अगर तुम्हें शक़ हो तो उन दोनों को नमाज़ के बाद रोक लो फिर वह दोनों ख़ुदा की क़सम खाएँ कि हम इस (गवाही) के (ऐवज़ कुछ दाम नहीं लेंगे अगरचे हम जिसकी गवाही देते हैं हमारा अज़ीज़ ही क्यों न) हो और हम खुदा लगती गवाही न छुपाएँगे अगर ऐसा करें तो हम बेशक गुनाहगार हैं
\end{hindi}}
\flushright{\begin{Arabic}
\quranayah[5][107]
\end{Arabic}}
\flushleft{\begin{hindi}
अगर इस पर मालूम हो जाए कि वह दोनों (दरोग़ हलफ़ी (झूठी कसम) से) गुनाह के मुस्तहक़ हो गए तो दूसरे दो आदमी उन लोगों में से जिनका हक़ दबाया गया है और (मय्यत) के ज्यादा क़राबतदार हैं (उनकी तरवीद में) उनकी जगह खड़े हो जाएँ फिर दो नए गवाह ख़ुदा की क़सम खाएँ कि पहले दो गवाहों की निस्बत हमारी गवाही ज्यादा सच्ची है और हमने (हक़) नहीं छुपाया और अगर ऐसा किया हो तो उस वक्त बेशक हम ज़ालिम हैं
\end{hindi}}
\flushright{\begin{Arabic}
\quranayah[5][108]
\end{Arabic}}
\flushleft{\begin{hindi}
ये ज्यादा क़रीन क़यास है कि इस तरह पर (आख़ेरत के डर से) ठीक ठीक गवाही दें या (दुनिया की रूसवाई का) अन्देशा हो कि कहीं हमारी क़समें दूसरे फरीक़ की क़समों के बाद रद न कर दी जाएँ मुसलमानों ख़ुदा से डरो और (जी लगा कर) सुन लो और ख़ुदा बदचलन लोगों को मंज़िले मक़सूद तक नहीं पहुँचाता
\end{hindi}}
\flushright{\begin{Arabic}
\quranayah[5][109]
\end{Arabic}}
\flushleft{\begin{hindi}
(उस वक्त क़ो याद करो) जिस दिन ख़ुदा अपने पैग़म्बरों को जमा करके पूछेगा कि (तुम्हारी उममत की तरफ से तबलीग़े एहकाम का) क्या जवाब दिया गया तो अर्ज क़रेगें कि हम तो (चन्द ज़ाहिरी बातों के सिवा) कुछ नहीं जानते तू तो खुद बड़ा ग़ैब वॉ है
\end{hindi}}
\flushright{\begin{Arabic}
\quranayah[5][110]
\end{Arabic}}
\flushleft{\begin{hindi}
(वह वक्त याद करो) जब ख़ुदा फरमाएगा कि ये मरियम के बेटे ईसा हमने जो एहसानात तुम पर और तुम्हारी माँ पर किये उन्हे याद करो जब हमने रूहुलक़ुदूस (जिबरील) से तुम्हारी ताईद की कि तुम झूले में (पड़े पड़े) और अधेड़ होकर (शक़ सा बातें) करने लगे और जब हमने तुम्हें लिखना और अक़ल व दानाई की बातें और (तौरेत व इन्जील (ये सब चीजे) सिखायी और जब तुम मेरे हुक्म से मिट्टी से चिड़िया की मूरत बनाते फिर उस पर कुछ दम कर देते तो वह मेरे हुक्म से (सचमुच) चिड़िया बन जाती थी और मेरे हुक्म से मादरज़ाद (पैदायशी) अंधे और कोढ़ी को अच्छा कर देते थे और जब तुम मेरे हुक्म से मुर्दों को ज़िन्दा (करके क़ब्रों से) निकाल खड़ा करते थे और जिस वक्त तुम बनी इसराईल के पास मौजिज़े लेकर आए और उस वक्त मैने उनको तुम (पर दस्त दराज़ी करने) से रोका तो उनमें से बाज़ कुफ्फ़ार कहने लगे ये तो बस खुला हुआ जादू है
\end{hindi}}
\flushright{\begin{Arabic}
\quranayah[5][111]
\end{Arabic}}
\flushleft{\begin{hindi}
और जब मैने हवारियों से इलहाम किया कि मुझ पर और मेरे रसूल पर ईमान लाओ तो अर्ज़ करने लगे हम ईमान लाए और तू गवाह रहना कि हम तेरे फरमाबरदार बन्दे हैं
\end{hindi}}
\flushright{\begin{Arabic}
\quranayah[5][112]
\end{Arabic}}
\flushleft{\begin{hindi}
(वह वक्त याद करो) जब हवारियों ने ईसा से अर्ज़ की कि ऐ मरियम के बेटे ईसा क्या आप का ख़ुदा उस पर क़ादिर है कि हम पर आसमान से (नेअमत की) एक ख्वान नाज़िल फरमाए ईसा ने कहा अगर तुम सच्चे ईमानदार हो तो ख़ुदा से डरो (ऐसी फरमाइश जिसमें इम्तेहान मालूम हो न करो)
\end{hindi}}
\flushright{\begin{Arabic}
\quranayah[5][113]
\end{Arabic}}
\flushleft{\begin{hindi}
वह अर्ज़ करने लगे हम तो फक़त ये चाहते है कि इसमें से (बरतकन) कुछ खाएँ और हमारे दिल को (आपकी रिसालत का पूरा पूरा) इत्मेनान हो जाए और यक़ीन कर लें कि आपने हमसे (जो कुछ कहा था) सच फरमाया था और हम लोग इस पर गवाह रहें
\end{hindi}}
\flushright{\begin{Arabic}
\quranayah[5][114]
\end{Arabic}}
\flushleft{\begin{hindi}
(तब) मरियम के बेटे ईसा ने (बारगाहे ख़ुदा में) अर्ज़ की ख़ुदा वन्दा ऐ हमारे पालने वाले हम पर आसमान से एक ख्वान (नेअमत) नाज़िल फरमा कि वह दिन हम लोगों के लिए हमारे अगलों के लिए और हमारे पिछलों के लिए ईद का करार पाए (और हमारे हक़ में) तेरी तरफ से एक बड़ी निशानी हो और तू हमें रोज़ी दे और तू सब रोज़ी देने वालो से बेहतर है
\end{hindi}}
\flushright{\begin{Arabic}
\quranayah[5][115]
\end{Arabic}}
\flushleft{\begin{hindi}
खुदा ने फरमाया मै ख्वान तो तुम पर ज़रुर नाज़िल करुगाँ (मगर याद रहे कि) फिर तुममें से जो भी शख़्श उसके बाद काफ़िर हुआ तो मै उसको यक़ीन ऐसे सख्त अज़ाब की सज़ा दूँगा कि सारी ख़ुदायी में किसी एक पर भी वैसा (सख्त) अज़ाब न करुगाँ
\end{hindi}}
\flushright{\begin{Arabic}
\quranayah[5][116]
\end{Arabic}}
\flushleft{\begin{hindi}
और (वह वक्त भी याद करो) जब क़यामत में ईसा से ख़ुदा फरमाएग कि (क्यों) ऐ मरियम के बेटे ईसा क्या तुमने लोगों से ये कह दिया था कि ख़ुदा को छोड़कर मुझ को और मेरी माँ को ख़ुदा बना लो ईसा अर्ज़ करेगें सुबहान अल्लाह मेरी तो ये मजाल न थी कि मै ऐसी बात मुँह से निकालूं जिसका मुझे कोई हक़ न हो (अच्छा) अगर मैने कहा होगा तो तुझको ज़रुर मालूम ही होगा क्योंकि तू मेरे दिल की (सब बात) जानता है हाँ अलबत्ता मै तेरे जी की बात नहीं जानता (क्योंकि) इसमें तो शक़ ही नहीं कि तू ही ग़ैब की बातें ख़ूब जानता है
\end{hindi}}
\flushright{\begin{Arabic}
\quranayah[5][117]
\end{Arabic}}
\flushleft{\begin{hindi}
तूने मुझे जो कुछ हुक्म दिया उसके सिवा तो मैने उनसे कुछ भी नहीं कहा यही कि ख़ुदा ही की इबादत करो जो मेरा और तुम्हारा सबका पालने वाला है और जब तक मैं उनमें रहा उन की देखभाल करता रहा फिर जब तूने मुझे (दुनिया से) उठा लिया तो तू ही उनका निगेहबान था और तू तो ख़ुद हर चीज़ का गवाह (मौजूद) है
\end{hindi}}
\flushright{\begin{Arabic}
\quranayah[5][118]
\end{Arabic}}
\flushleft{\begin{hindi}
तू अगर उन पर अज़ाब करेगा तो (तू मालिक है) ये तेरे बन्दे हैं और अगर उन्हें बख्श देगा तो (कोई तेरा हाथ नहीं पकड़ सकता क्योंकि) बेशक तू ज़बरदस्त हिकमत वाला है
\end{hindi}}
\flushright{\begin{Arabic}
\quranayah[5][119]
\end{Arabic}}
\flushleft{\begin{hindi}
ख़ुदा फरमाएगा कि ये वह दिन है कि सच्चे बन्दों को उनकी सच्चाई (आज) काम आएगी उनके लिए (हरे भरे बेहिश्त के) वह बाग़ात है जिनके (दरख्तो के) नीचे नहरे जारी हैं (और) वह उसमें अबादुल आबाद तक रहेंगे ख़ुदा उनसे राज़ी और वह ख़ुदा से खुश यही बहुत बड़ी कामयाबी है
\end{hindi}}
\flushright{\begin{Arabic}
\quranayah[5][120]
\end{Arabic}}
\flushleft{\begin{hindi}
सारे आसमान व ज़मीन और जो कुछ उनमें है सब ख़ुदा ही की सल्तनत है और वह हर चीज़ पर क़ादिर (व तवाना) है
\end{hindi}}
\chapter{Al-An'am (The Cattle)}
\begin{Arabic}
\Huge{\centerline{\basmalah}}\end{Arabic}
\flushright{\begin{Arabic}
\quranayah[6][1]
\end{Arabic}}
\flushleft{\begin{hindi}
सब तारीफ ख़ुदा ही को (सज़ावार) है जिसने बहुतेरे आसमान और ज़मीन को पैदा किया और उसमें मुख्तलिफ क़िस्मों की तारीकी रोशनी बनाई फिर (बावजूद उसके) कुफ्फार (औरों को) अपने परवरदिगार के बराबर करते हैं
\end{hindi}}
\flushright{\begin{Arabic}
\quranayah[6][2]
\end{Arabic}}
\flushleft{\begin{hindi}
वह तो वही ख़ुदा है जिसने तुमको मिट्टी से पैदा किया फिर (फिर तुम्हारे मरने का) एक वक्त मुक़र्रर कर दिया और (अगरचे तुमको मालूम नहीं मगर) उसके नज़दीक (क़यामत) का वक्त मुक़र्रर है
\end{hindi}}
\flushright{\begin{Arabic}
\quranayah[6][3]
\end{Arabic}}
\flushleft{\begin{hindi}
फिर (यही) तुम शक़ करते हो और वही तो आसमानों में (भी) और ज़मीन में (भी) ख़ुदा है वही तुम्हारे ज़ाहिर व बातिन से (भी) ख़बरदार है और वही जो कुछ भी तुम करते हो जानता है
\end{hindi}}
\flushright{\begin{Arabic}
\quranayah[6][4]
\end{Arabic}}
\flushleft{\begin{hindi}
और उन (लोगों का) अजब हाल है कि उनके पास ख़ुदा की आयत में से जब कोई आयत आती तो बस ये लोग ज़रुर उससे मुंह फेर लेते थे
\end{hindi}}
\flushright{\begin{Arabic}
\quranayah[6][5]
\end{Arabic}}
\flushleft{\begin{hindi}
चुनान्चे जब उनके पास (क़ुरान बरहक़) आया तो उसको भी झुठलाया तो ये लोग जिसके साथ मसख़रापन कर रहे है उनकी हक़ीक़त उन्हें अनक़रीब ही मालूम हो जाएगी
\end{hindi}}
\flushright{\begin{Arabic}
\quranayah[6][6]
\end{Arabic}}
\flushleft{\begin{hindi}
क्या उन्हें सूझता नहीं कि हमने उनसे पहले कितने गिरोह (के गिरोह) हलाक कर डाले जिनको हमने रुए ज़मीन मे वह (कूवत) क़ुदरत अता की थी जो अभी तक तुमको नहीं दी और हमने आसमान तो उन पर मूसलाधार पानी बरसता छोड़ दिया था और उनके (मकानात के) नीचे बहती हुई नहरें बना दी थी (मगर) फिर भी उनके गुनाहों की वजह से उनको मार डाला और उनके बाद एक दूसरे गिरोह को पैदा कर दिया
\end{hindi}}
\flushright{\begin{Arabic}
\quranayah[6][7]
\end{Arabic}}
\flushleft{\begin{hindi}
और (ऐ रसूल) अगर हम कागज़ पर (लिखी लिखाई) किताब (भी) तुम पर नाज़िल करते और ये लोग उसे अपने हाथों से छू भी लेते फिर भी कुफ्फार (न मानते और) कहते कि ये तो बस खुला हुआ जादू है
\end{hindi}}
\flushright{\begin{Arabic}
\quranayah[6][8]
\end{Arabic}}
\flushleft{\begin{hindi}
और (ये भी) कहते कि उस (नबी) पर कोई फरिश्ता क्यों नहीं नाज़िल किया गया (जो साथ साथ रहता) हालॉकि अगर हम फरिश्ता भेज देते तो (उनका) काम ही तमाम हो जाता (और) फिर उन्हें मोहलत भी न दी जाती
\end{hindi}}
\flushright{\begin{Arabic}
\quranayah[6][9]
\end{Arabic}}
\flushleft{\begin{hindi}
और अगर हम फरिश्ते को नबी बनाते तो (आख़िर) उसको भी मर्द सूरत बनाते और जो शुबहे ये लोग कर रहे हैं वही शुबहे (गोया) हम ख़ुद उन पर (उस वक्त भी) उठा देते
\end{hindi}}
\flushright{\begin{Arabic}
\quranayah[6][10]
\end{Arabic}}
\flushleft{\begin{hindi}
(ऐ रसूल तुम दिल तंग न हो) तुम से पहले (भी) पैग़म्बरों के साथ मसख़रापन किया गया है पस जो लोग मसख़रापन करते थे उनको उस अज़ाब ने जिसके ये लोग हॅसी उड़ाते थे घेर लिया
\end{hindi}}
\flushright{\begin{Arabic}
\quranayah[6][11]
\end{Arabic}}
\flushleft{\begin{hindi}
(ऐ रसूल उनसे) कहो कि ज़रुर रुए ज़मीन पर चल फिर कर देखो तो कि (अम्बिया के) झुठलाने वालो का क्या (बुरा) अन्जाम हुआ
\end{hindi}}
\flushright{\begin{Arabic}
\quranayah[6][12]
\end{Arabic}}
\flushleft{\begin{hindi}
(ऐ रसूल उनसे) पूछो तो कि (भला) जो कुछ आसमान और ज़मीन में है किसका है (वह जवाब देगें) तुम ख़ुद कह दो कि ख़ास ख़ुदा का है उसने अपनी ज़ात पर मेहरबानी लाज़िम कर ली है वह तुम सब के सब को क़यामत के दिन जिसके आने मे कुछ शक़ नहीं ज़रुर जमा करेगा (मगर) जिन लोगों ने अपना आप नुक़सान किया वह तो (क़यामत पर) ईमान न लाएंगें
\end{hindi}}
\flushright{\begin{Arabic}
\quranayah[6][13]
\end{Arabic}}
\flushleft{\begin{hindi}
हालॉकि (ये नहीं समझते कि) जो कुछ रात को और दिन को (रुए ज़मीन पर) रहता (सहता) है (सब) ख़ास उसी का है और वही (सब की) सुनता (और सब कुछ) जानता है
\end{hindi}}
\flushright{\begin{Arabic}
\quranayah[6][14]
\end{Arabic}}
\flushleft{\begin{hindi}
ऐ रसूल) तुम कह दो कि क्या ख़ुदा को जो सारे आसमान व ज़मीन का पैदा करने वाला है छोड़ कर दूसरे को (अपना) सरपरस्त बनाओ और वही (सब को) रोज़ी देता है और उसको कोई रोज़ी नहीं देता (ऐ रसूल) तुम कह दो कि मुझे हुक्म दिया गया है कि सब से पहले इस्लाम लाने वाला मैं हूँ और (ये भी कि ख़बरदार) मुशरेकीन से न होना
\end{hindi}}
\flushright{\begin{Arabic}
\quranayah[6][15]
\end{Arabic}}
\flushleft{\begin{hindi}
(ऐ रसूल) तुम कहो कि अगर मैं नाफरमानी करुँ तो बेशक एक बड़े (सख्त) दिल के अज़ाब से डरता हूँ
\end{hindi}}
\flushright{\begin{Arabic}
\quranayah[6][16]
\end{Arabic}}
\flushleft{\begin{hindi}
उस दिन जिस (के सर) से अज़ाब टल गया तो (समझो कि) ख़ुदा ने उस पर (बड़ा) रहम किया और यही तो सरीही कामयाबी है
\end{hindi}}
\flushright{\begin{Arabic}
\quranayah[6][17]
\end{Arabic}}
\flushleft{\begin{hindi}
और अगर ख़ुदा तुम को किसी क़िस्म की तकलीफ पहुचाए तो उसके सिवा कोई उसका दफा करने वाला नहीं है और अगर तुम्हें कुछ फायदा) पहुंचाए तो भी (कोई रोक नहीं सकता क्योंकि) वह हर चीज़ पर क़ादिर है
\end{hindi}}
\flushright{\begin{Arabic}
\quranayah[6][18]
\end{Arabic}}
\flushleft{\begin{hindi}
वही अपने तमाम बन्दों पर ग़ालिब है और वह वाक़िफकार हकीम है
\end{hindi}}
\flushright{\begin{Arabic}
\quranayah[6][19]
\end{Arabic}}
\flushleft{\begin{hindi}
(ऐ रसूल) तुम पूछो कि गवाही में सबसे बढ़के कौन चीज़ है तुम खुद ही कह दो कि मेरे और तुम्हारे दरमियान ख़ुदा गवाह है और मेरे पास ये क़ुरान वही के तौर पर इसलिए नाज़िल किया गया ताकि मैं तुम्हें और जिसे (उसकी) ख़बर पहुँचे उसके ज़रिए से डराओ क्या तुम यक़ीनन यह गवाही दे सकते हो कि अल्लाह के साथ और दूसरे माबूद भी हैं (ऐ रसूल) तुम कह दो कि मै तो उसकी गवाही नहीं देता (तुम दिया करो) तुम (उन लोगों से) कहो कि वह तो बस एक ही ख़ुदा है और जिन चीज़ों को तुम (ख़ुदा का) शरीक बनाते हो
\end{hindi}}
\flushright{\begin{Arabic}
\quranayah[6][20]
\end{Arabic}}
\flushleft{\begin{hindi}
मै तो उनसे बेज़र हूँ जिन लोगों को हमने किताब अता फरमाई है (यहूद व नसारा) वह तो जिस तरह अपने बाल बच्चों को पहचानते है उसी तरह उस नबी (मोहम्मद) को भी पहचानते हैं (मगर) जिन लोगों ने अपना आप नुक़सान किया वह तो (किसी तरह) ईमान न लाएंगें
\end{hindi}}
\flushright{\begin{Arabic}
\quranayah[6][21]
\end{Arabic}}
\flushleft{\begin{hindi}
और जो शख़्श ख़ुदा पर झूठ बोहतान बॉधे या उसकी आयतों को झुठलाए उससे बढ़के ज़ालिम कौन होगा और ज़ालिमों को हरगिज़ नजात न होगी
\end{hindi}}
\flushright{\begin{Arabic}
\quranayah[6][22]
\end{Arabic}}
\flushleft{\begin{hindi}
और (उस दिन को याद करो) जिस दिन हम उन सबको जमा करेंगे फिर जिन लोगों ने शिर्क किया उनसे पूछेगें कि जिनको तुम (ख़ुदा का) शरीक ख्याल करते थे कहाँ हैं
\end{hindi}}
\flushright{\begin{Arabic}
\quranayah[6][23]
\end{Arabic}}
\flushleft{\begin{hindi}
फिर उनकी कोई शरारत (बाक़ी) न रहेगी बल्कि वह तो ये कहेगें क़सम है उस ख़ुदा की जो हमारा पालने वाला है हम किसी को उसका शरीक नहीं बनाते थे
\end{hindi}}
\flushright{\begin{Arabic}
\quranayah[6][24]
\end{Arabic}}
\flushleft{\begin{hindi}
(ऐ रसूल भला) देखो तो ये लोग अपने ही ऊपर आप किस तरह झूठ बोलने लगे और ये लोग (दुनिया में) जो कुछ इफ़तेरा परदाज़ी (झूठी बातें) करते थे
\end{hindi}}
\flushright{\begin{Arabic}
\quranayah[6][25]
\end{Arabic}}
\flushleft{\begin{hindi}
वह सब ग़ायब हो गयी और बाज़ उनमें के ऐसे भी हैं जो तुम्हारी (बातों की) तरफ कान लगाए रहते हैं और (उनकी हठ धर्मी इस हद को पहुँची है कि गोया हमने ख़ुद उनके दिलों पर परदे डाल दिए हैं और उनके कानों में बहरापन पैदा कर दिया है कि उसे समझ न सकें और अगर वह सारी (ख़ुदाई के) मौजिज़े भी देखे लें तब भी ईमान न लाएंगें यहाँ तक (हठ धर्मी पहुची) कि जब तुम्हारे पास तुम से उलझे हुए आ निकलते हैं तो कुफ्फ़ार (क़ुरान लेकर) कहा बैठे है (कि भला इसमें रखा ही क्या है) ये तो अगलों की कहानियों के सिवा कुछ भी नहीं
\end{hindi}}
\flushright{\begin{Arabic}
\quranayah[6][26]
\end{Arabic}}
\flushleft{\begin{hindi}
और ये लोग (दूसरों को भी) उस के (सुनने से) से रोकते हैं और ख़ुद तो अलग थलग रहते ही हैं और (इन बातों से) बस आप ही अपने को हलाक करते हैं और (अफसोस) समझते नहीं
\end{hindi}}
\flushright{\begin{Arabic}
\quranayah[6][27]
\end{Arabic}}
\flushleft{\begin{hindi}
(ऐ रसूल) अगर तुम उन लोगों को उस वक्त देखते (तो ताज्जुब करते) जब जहन्नुम (के किनारे) पर लाकर खड़े किए जाओगे तो (उसे देखकर) कहेगें ऐ काश हम (दुनिया में) फिर (दुबारा) लौटा भी दिए जाते और अपने परवरदिगार की आयतों को न झुठलाते और हम मोमिनीन से होते (मगर उनकी आरज़ू पूरी न होगी)
\end{hindi}}
\flushright{\begin{Arabic}
\quranayah[6][28]
\end{Arabic}}
\flushleft{\begin{hindi}
बल्कि जो (बेईमानी) पहले से छिपाते थे आज (उसकी हक़ीक़त) उन पर खुल गयी और (हम जानते हैं कि) अगर ये लोग (दुनिया में) लौटा भी दिए जाएं तो भी जिस चीज़ की मनाही की गयी है उसे करें और ज़रुर करें और इसमें शक़ नहीं कि ये लोग ज़रुर झूठे हैं
\end{hindi}}
\flushright{\begin{Arabic}
\quranayah[6][29]
\end{Arabic}}
\flushleft{\begin{hindi}
और कुफ्फार ये भी तो कहते हैं कि हमारी इस दुनिया ज़िन्दगी के सिवा कुछ भी नहीं और (क़यामत वग़ैरह सब ढकोसला है) हम (मरने के बाद) भी उठाए ही न जायेंगे
\end{hindi}}
\flushright{\begin{Arabic}
\quranayah[6][30]
\end{Arabic}}
\flushleft{\begin{hindi}
और (ऐ रसूल) अगर तुम उनको उस वक्त देखते (तो ताज्जुब करते) जब वे लोग ख़ुदा के सामने खड़े किए जाएंगें और ख़ुदा उनसे पूछेगा कि क्या ये (क़यामत का दिन) अब भी सही नहीं है वह (जवाब में) कहेगें कि (दुनिया में) इससे इन्कार करते थे
\end{hindi}}
\flushright{\begin{Arabic}
\quranayah[6][31]
\end{Arabic}}
\flushleft{\begin{hindi}
उसकी सज़ा में अज़ाब (के मजे) चखो बेशक जिन लोगों ने क़यामत के दिन ख़ुदा की हुज़ूरी को झुठलाया वह बड़े घाटे में हैं यहाँ तक कि जब उनके सर पर क़यामत नागहा (एक दम आ) पहँचेगी तो कहने लगेगें ऐ है अफसोस हम ने तो इसमें बड़ी कोताही की (ये कहते जाएगे) और अपने गुनाहों का पुश्तारा अपनी अपनी पीठ पर लादते जाएंगे देखो तो (ये) क्या बुरा बोझ है जिसको ये लादे (लादे फिर रहे) हैं
\end{hindi}}
\flushright{\begin{Arabic}
\quranayah[6][32]
\end{Arabic}}
\flushleft{\begin{hindi}
और (ये) दुनियावी ज़िन्दगी तो खेल तमाशे के सिवा कुछ भी नहीं और ये तो ज़ाहिर है कि आख़िरत का घर (बेहिश्त) परहेज़गारो के लिए उसके बदर वहॉ (कई गुना) बेहतर है तो क्या तुम (इतना भी) नहीं समझते
\end{hindi}}
\flushright{\begin{Arabic}
\quranayah[6][33]
\end{Arabic}}
\flushleft{\begin{hindi}
हम खूब जानते हैं कि उन लोगों की बकबक तुम को सदमा पहुँचाती है तो (तुम को समझना चाहिए कि) ये लोग तुम को नहीं झुठलाते बल्कि (ये) ज़ालिम (हक़ीक़तन) ख़ुदा की आयतों से इन्कार करते हैं
\end{hindi}}
\flushright{\begin{Arabic}
\quranayah[6][34]
\end{Arabic}}
\flushleft{\begin{hindi}
और (कुछ तुम ही पर नहीं) तुमसे पहले भी बहुतेरे रसूल झुठलाए जा चुके हैं तो उन्होनें अपने झुठलाए जाने और अज़ीयत (व तकलीफ) पर सब्र किया यहाँ तक कि हमारी मदद उनके पास आयी और (क्यों न आती) ख़ुदा की बातों का कोई बदलने वाला नहीं है और पैग़म्बर के हालात तो तुम्हारे पास पहुँच ही चुके हैं
\end{hindi}}
\flushright{\begin{Arabic}
\quranayah[6][35]
\end{Arabic}}
\flushleft{\begin{hindi}
अगरचे उन लोगों की रदगिरदानी (मुँह फेरना) तुम पर शाक़ ज़रुर है (लेकिन) अगर तुम्हारा बस चले तो ज़मीन के अन्दर कोई सुरगं ढँढ निकालो या आसमान में सीढ़ी लगाओ और उन्हें कोई मौजिज़ा ला दिखाओ (तो ये भी कर देखो) अगर ख़ुदा चाहता तो उन सब को राहे रास्त पर इकट्ठा कर देता (मगर वह तो इम्तिहान करता है) बस (देखो) तुम हरगिज़ ज़ालिमों में (शामिल) न होना
\end{hindi}}
\flushright{\begin{Arabic}
\quranayah[6][36]
\end{Arabic}}
\flushleft{\begin{hindi}
(तुम्हारा कहना तो) सिर्फ वही लोग मानते हैं जो (दिल से) सुनते हैं और मुर्दो को तो ख़ुदा क़यामत ही में उठाएगा फिर उसी की तरफ लौटाए जाएंगें
\end{hindi}}
\flushright{\begin{Arabic}
\quranayah[6][37]
\end{Arabic}}
\flushleft{\begin{hindi}
और कुफ्फ़ार कहते हैं कि (आख़िर) उस नबी पर उसके परवरदिगार की तरफ से कोई मौजिज़ा क्यों नहीं नाज़िल होता तो तुम (उनसे) कह दो कि ख़ुदा मौजिज़े के नाज़िल करने पर ज़रुर क़ादिर है मगर उनमें के अक्सर लोग (ख़ुदा की मसलहतों को) नहीं जानते
\end{hindi}}
\flushright{\begin{Arabic}
\quranayah[6][38]
\end{Arabic}}
\flushleft{\begin{hindi}
ज़मीन में जो चलने फिरने वाला (हैवान) या अपने दोनों परों से उड़ने वाला परिन्दा है उनकी भी तुम्हारी तरह जमाअतें हैं और सब के सब लौह महफूज़ में मौजूद (हैं) हमने किताब (क़ुरान) में कोई बात नहीं छोड़ी है फिर सब के सब (चरिन्द हों या परिन्द) अपने परवरदिगार के हुज़ूर में लाए जायेंगे।
\end{hindi}}
\flushright{\begin{Arabic}
\quranayah[6][39]
\end{Arabic}}
\flushleft{\begin{hindi}
और जिन लोगों ने हमारी आयतों को झुठला दिया गोया वह (कुफ्र के घटाटोप) ऍधेरों में गूंगें बहरे (पड़े हैं) ख़ुदा जिसे चाहे उसे गुमराही में छोड़ दे और जिसे चाहे उसे सीधे ढर्रे पर लगा दे
\end{hindi}}
\flushright{\begin{Arabic}
\quranayah[6][40]
\end{Arabic}}
\flushleft{\begin{hindi}
(ऐ रसूल उनसे) पूछो तो कि क्या तुम यह समझते हो कि अगर तुम्हारे सामने ख़ुदा का अज़ाब आ जाए या तुम्हारे सामने क़यामत ही आ खड़ी मौजूद हो तो तुम अगर (अपने दावे में) सच्चे हो तो (बताओ कि मदद के वास्ते) क्या ख़ुदा को छोड़कर दूसरे को पुकारोगे
\end{hindi}}
\flushright{\begin{Arabic}
\quranayah[6][41]
\end{Arabic}}
\flushleft{\begin{hindi}
(दूसरों को तो क्या) बल्कि उसी को पुकारोगे फिर अगर वह चाहेगा तो जिस के वास्ते तुमने उसको पुकारा है उसे दफा कर देगा और (उस वक्त) तुम दूसरे माबूदों को जिन्हे तुम (ख़ुदा का) शरीक समझते थे भूल जाओगे
\end{hindi}}
\flushright{\begin{Arabic}
\quranayah[6][42]
\end{Arabic}}
\flushleft{\begin{hindi}
और (ऐ रसूल) जो उम्मतें तुमसे पहले गुज़र चुकी हैं हम उनके पास भी बहुतेरे रसूल भेज चुके हैं फिर (जब नाफ़रमानी की) तो हमने उनको सख्ती
\end{hindi}}
\flushright{\begin{Arabic}
\quranayah[6][43]
\end{Arabic}}
\flushleft{\begin{hindi}
और तकलीफ़ में गिरफ्तार किया ताकि वह लोग (हमारी बारगाह में) गिड़गिड़ाए तो जब उन (के सर) पर हमारा अज़ाब आ खड़ा हुआ तो वह लोग क्यों नहीं गिड़गिड़ाए (कि हम अज़ाब दफा कर देते) मगर उनके दिल तो सख्त हो गए थे ओर उनकी कारस्तानियों को शैतान ने आरास्ता कर दिखाया था (फिर क्योंकर गिड़गिड़ाते)
\end{hindi}}
\flushright{\begin{Arabic}
\quranayah[6][44]
\end{Arabic}}
\flushleft{\begin{hindi}
फिर जिसकी उन्हें नसीहत की गयी थी जब उसको भूल गए तो हमने उन पर (ढील देने के लिए) हर तरह की (दुनियावी) नेअमतों के दरवाज़े खोल दिए यहाँ तक कि जो नेअमतें उनको दी गयी थी जब उनको पाकर ख़ुश हुए तो हमने उन्हें नागाहाँ (एक दम) ले डाला तो उस वक्त वह नाउम्मीद होकर रह गए
\end{hindi}}
\flushright{\begin{Arabic}
\quranayah[6][45]
\end{Arabic}}
\flushleft{\begin{hindi}
फिर ज़ालिम लोगों की जड़ काट दी गयी और सारे जहाँन के मालिक ख़ुदा का शुक्र है
\end{hindi}}
\flushright{\begin{Arabic}
\quranayah[6][46]
\end{Arabic}}
\flushleft{\begin{hindi}
(कि क़िस्सा पाक हुआ) (ऐ रसूल) उनसे पूछो तो कि क्या तुम ये समझते हो कि अगर ख़ुदा तुम्हारे कान और तुम्हारी ऑंखे लें ले और तुम्हारे दिलों पर मोहर कर दे तो ख़ुदा के सिवा और कौन मौजूद है जो (फिर) तुम्हें ये नेअमतें (वापस) दे (ऐ रसूल) देखो तो हम किस किस तरह अपनी दलीले बयान करते हैं इस पर भी वह लोग मुँह मोडे ज़ाते हैं
\end{hindi}}
\flushright{\begin{Arabic}
\quranayah[6][47]
\end{Arabic}}
\flushleft{\begin{hindi}
(ऐ रसूल) उनसे पूछो कि क्या तुम ये समझते हो कि अगर तुम्हारे सर पर ख़ुदा का अज़ाब बेख़बरी में या जानकारी में आ जाए तो क्या गुनाहगारों के सिवा और लोग भी हलाक़ किए जाएंगें (हरगिज़ नहीं)
\end{hindi}}
\flushright{\begin{Arabic}
\quranayah[6][48]
\end{Arabic}}
\flushleft{\begin{hindi}
और हम तो रसूलों को सिर्फ इस ग़रज़ से भेजते हैं कि (नेको को जन्नत की) खुशख़बरी दें और (बदो को अज़ाब जहन्नुम से) डराएं फिर जिसने ईमान कुबूल किया और अच्छे अच्छे काम किए तो ऐसे लोगों पर (क़यामत में) न कोई ख़ौफ होगा और न वह ग़मग़ीन होगें
\end{hindi}}
\flushright{\begin{Arabic}
\quranayah[6][49]
\end{Arabic}}
\flushleft{\begin{hindi}
और जिन लोगों ने हमारी आयतों को झुठलाया तो चूंकि बदकारी करते थे (हमारा) अज़ाब उनको पलट जाएगा
\end{hindi}}
\flushright{\begin{Arabic}
\quranayah[6][50]
\end{Arabic}}
\flushleft{\begin{hindi}
(ऐ रसूल) उनसे कह दो कि मै तो ये नहीं कहता कि मेरे पास ख़ुदा के ख़ज़ाने हैं (कि ईमान लाने पर दे दूंगा) और न मै गैब के (कुल हालात) जानता हूँ और न मै तुमसे ये कहता हूँ कि मै फरिश्ता हूँ मै तो बस जो (ख़ुदा की तरफ से) मेरे पास वही की जाती है उसी का पाबन्द हूँ (उनसे पूछो तो) कि अन्धा और ऑंख वाला बराबर हो सकता है तो क्या तुम (इतना भी) नहीं सोचते
\end{hindi}}
\flushright{\begin{Arabic}
\quranayah[6][51]
\end{Arabic}}
\flushleft{\begin{hindi}
और इस क़ुरान के ज़रिए से तुम उन लोगों को डराओ जो इस बात का ख़ौफ रखते हैं कि वह (मरने के बाद) अपने ख़ुदा के सामने जमा किये जायेंगे (और यह समझते है कि) उनका ख़ुदा के सिवा न कोई सरपरस्त हे और न कोई सिफारिश करने वाला ताकि ये लोग परहेज़गार बन जाएं
\end{hindi}}
\flushright{\begin{Arabic}
\quranayah[6][52]
\end{Arabic}}
\flushleft{\begin{hindi}
और (ऐ रसूल) जो लोग सुबह व शाम अपने परवरदिगार से उसकी ख़ुशनूदी की तमन्ना में दुऑए मॉगा करते हैं- उनको अपने पास से न धुत्कारो-न उनके (हिसाब किताब की) जवाब देही कुछ उनके ज़िम्मे है ताकि तुम उन्हें (इस ख्याल से) धुत्कार बताओ तो तुम ज़ालिम (के शुमार) में हो जाओगे
\end{hindi}}
\flushright{\begin{Arabic}
\quranayah[6][53]
\end{Arabic}}
\flushleft{\begin{hindi}
और इसी तरह हमने बाज़ आदमियों को बाज़ से आज़माया ताकि वह लोग कहें कि हाएं क्या ये लोग हममें से हैं जिन पर ख़ुदा ने अपना फ़जल व करम किया है (यह तो समझते की) क्या ख़ुदा शुक्र गुज़ारों को भी नही जानता
\end{hindi}}
\flushright{\begin{Arabic}
\quranayah[6][54]
\end{Arabic}}
\flushleft{\begin{hindi}
और जो लोग हमारी आयतों पर ईमान लाए हैं तुम्हारे पास ऑंए तो तुम सलामुन अलैकुम (तुम पर ख़ुदा की सलामती हो) कहो तुम्हारे परवरदिगार ने अपने ऊपर रहमत लाज़िम कर ली है बेशक तुम में से जो शख़्श नादानी से कोई गुनाह कर बैठे उसके बाद फिर तौबा करे और अपनी हालत की (असलाह करे ख़ुदा उसका गुनाह बख्श देगा क्योंकि) वह यक़ीनी बड़ा बख्शने वाला मेहरबान है
\end{hindi}}
\flushright{\begin{Arabic}
\quranayah[6][55]
\end{Arabic}}
\flushleft{\begin{hindi}
और हम (अपनी) आयतों को यूं तफ़सील से बयान करते हैं ताकि गुनाहगारों की राह (सब पर) खुल जाए और वह इस पर न चले
\end{hindi}}
\flushright{\begin{Arabic}
\quranayah[6][56]
\end{Arabic}}
\flushleft{\begin{hindi}
(ऐ रसूल) तुम कह दो कि मुझसे उसकी मनाही की गई है कि मैं ख़ुदा को छोड़कर उन माबूदों की इबादत करुं जिन को तुम पूजा करते हो (ये भी) कह दो कि मै तो तुम्हारी (नाफसानी) ख्वाहिश पर चलने का नहीं (वरना) फिर तो मै गुमराह हो जाऊॅगा और हिदायत याफता लोगों में न रहूँगा
\end{hindi}}
\flushright{\begin{Arabic}
\quranayah[6][57]
\end{Arabic}}
\flushleft{\begin{hindi}
तुम कह दो कि मै तो अपने परवरदिगार की तरफ से एक रौशन दलील पर हूं और तुमने उसे झुठला दिया (तो) तुम जिस की जल्दी करते हो (अज़ाब) वह कुछ मेरे पास (एख्तियार में) तो है नहीं हुकूमत तो बस ज़रुर ख़ुदा ही के लिए है वह तो (हक़) बयान करता है और वह तमाम फैसला करने वालों से बेहतर है
\end{hindi}}
\flushright{\begin{Arabic}
\quranayah[6][58]
\end{Arabic}}
\flushleft{\begin{hindi}
(उन लोगों से) कह दो कि जिस (अज़ाब) की तुम जल्दी करते हो अगर वह मेरे पास (एख्तियार में) होता तो मेरे और तुम्हारे दरमियान का फैसला कब का चुक गया होता और ख़ुदा तो ज़ालिमों से खूब वाक़िफ है
\end{hindi}}
\flushright{\begin{Arabic}
\quranayah[6][59]
\end{Arabic}}
\flushleft{\begin{hindi}
और उसके पास ग़ैब की कुन्जियॉ हैं जिनको उसके सिवा कोई नही जानता और जो कुछ ख़ुशकी और तरी में है उसको (भी) वही जानता है और कोई पत्ता भी नहीं खटकता मगर वह उसे ज़रुर जानता है और ज़मीन की तारिक़ियों में कोई दाना और न कोई ख़ुश्क चीज़ है मगर वह नूरानी किताब (लौहे महफूज़) में मौजूद है
\end{hindi}}
\flushright{\begin{Arabic}
\quranayah[6][60]
\end{Arabic}}
\flushleft{\begin{hindi}
वह वही (ख़ुदा) है जो तुम्हें रात को (नींद में एक तरह पर दुनिया से) उठा लेता हे और जो कुछ तूने दिन को किया है जानता है फिर तुम्हें दिन को उठा कर खड़ा करता है ताकि (ज़िन्दगी की) (वह) मियाद जो (उसके इल्म में) मुअय्युन है पूरी की जाए फिर (तो आख़िर) तुम सबको उसी की तरफ लौटना है फिर जो कुछ तुम (दुनिया में भला बुरा) करते हो तुम्हें बता देगा
\end{hindi}}
\flushright{\begin{Arabic}
\quranayah[6][61]
\end{Arabic}}
\flushleft{\begin{hindi}
वह अपने बन्दों पर ग़ालिब है वह तुम लोगों पर निगेहबान (फ़रिश्ततें तैनात करके) भेजता है-यहाँ तक कि जब तुम में से किसी की मौत आए तो हमारे भेजे हुये फ़रिश्ते उसको (दुनिया से) उठा लेते हैं और वह (हमारे तामीले हुक्म में ज़रा भी) कोताही नहीं करते
\end{hindi}}
\flushright{\begin{Arabic}
\quranayah[6][62]
\end{Arabic}}
\flushleft{\begin{hindi}
फिर ये लोग अपने सच्चे मालिक ख़ुदा के पास वापस बुलाए गए-आगाह रहो कि हुक़ूमत ख़ास उसी के लिए है और वह सबसे ज्यादा हिसाब लेने वाला है
\end{hindi}}
\flushright{\begin{Arabic}
\quranayah[6][63]
\end{Arabic}}
\flushleft{\begin{hindi}
(ऐ रसूल) उनसे पूछो कि तुम ख़ुश्की और तरी के (घटाटोप) ऍधेरों से कौन छुटकारा देता है जिससे तुम गिड़ गिड़ाकर और (चुपके) दुआएं मॉगते हो कि अगर वह हमें (अब की दफ़ा) उस (बला) से छुटकारा दे तो हम ज़रुर उसके शुक्र गुज़ार (बन्दे होकर) रहेगें
\end{hindi}}
\flushright{\begin{Arabic}
\quranayah[6][64]
\end{Arabic}}
\flushleft{\begin{hindi}
तुम कहो उन (मुसीबतों) से और हर बला में ख़ुदा तुम्हें नजात देता है (मगर अफसोस) उस पर भी तुम शिर्क करते ही जाते हो
\end{hindi}}
\flushright{\begin{Arabic}
\quranayah[6][65]
\end{Arabic}}
\flushleft{\begin{hindi}
(ऐ रसूल) तुम कह दो कि वही उस पर अच्छी तरह क़ाबू रखता है कि अगर (चाहे तो) तुम पर अज़ाब तुम्हारे (सर के) ऊपर से नाज़िल करे या तुम्हारे पॉव के नीचे से (उठाकर खड़ा कर दे) या एक गिरोह को दूसरे से भिड़ा दे और तुम में से कुछ लोगों को बाज़ आदमियों की लड़ाई का मज़ा चखा दे ज़रा ग़ौर तो करो हम किस किस तरह अपनी आयतों को उलट पुलट के बयान करते हैं ताकि लोग समझे
\end{hindi}}
\flushright{\begin{Arabic}
\quranayah[6][66]
\end{Arabic}}
\flushleft{\begin{hindi}
और उसी (क़ुरान) को तुम्हारी क़ौम ने झुठला दिया हालॉकि वह बरहक़ है (ऐ रसूल) तुम उनसे कहो कि मैं तुम पर कुछ निगेहबान तो हूं नहीं हर ख़बर (के पूरा होने) का एक ख़ास वक्त मुक़र्रर है और अनक़रीब (जल्दी) ही तुम जान लोगे
\end{hindi}}
\flushright{\begin{Arabic}
\quranayah[6][67]
\end{Arabic}}
\flushleft{\begin{hindi}
और जब तुम उन लोगों को देखो जो हमारी आयतों में बेहूदा बहस कर रहे हैं तो उन (के पास) से टल जाओ यहाँ तक कि वह लोग उसके सिवा किसी और बात में बहस करने लगें और अगर (हमारा ये हुक्म) तुम्हें शैतान भुला दे तो याद आने के बाद ज़ालिम लोगों के साथ हरगिज़ न बैठना
\end{hindi}}
\flushright{\begin{Arabic}
\quranayah[6][68]
\end{Arabic}}
\flushleft{\begin{hindi}
और ऐसे लोगों (के हिसाब किताब) का जवाब देही कुछ परहेज़गारो पर तो है नहीं मगर (सिर्फ नसीहतन) याद दिलाना (चाहिए) ताकि ये लोग भी परहेज़गार बनें
\end{hindi}}
\flushright{\begin{Arabic}
\quranayah[6][69]
\end{Arabic}}
\flushleft{\begin{hindi}
और जिन लोगों ने अपने दीन को खेल और तमाशा बना रखा है और दुनिया की ज़िन्दगी ने उन को धोके में डाल रखा है ऐसे लोगों को छोड़ो और क़ुरान के ज़रिए से उनको नसीहत करते रहो (ऐसा न हो कि कोई) शख़्श अपने करतूत की बदौलत मुब्तिलाए बला हो जाए (क्योंकि उस वक्त) तो ख़ुदा के सिवा उसका न कोई सरपरस्त होगा न सिफारिशी और अगर वह अपने गुनाह के ऐवज़ सारे (जहाँन का) बदला भी दे तो भी उनमें से एक न लिया जाएगा जो लोग अपनी करनी की बदौलत मुब्तिलाए बला हुए है उनको पीने के लिए खौलता हुआ गर्म पानी (मिलेगा) और (उन पर) दर्दनाक अज़ाब होगा क्योंकर वह कुफ़्र किया करते थे
\end{hindi}}
\flushright{\begin{Arabic}
\quranayah[6][70]
\end{Arabic}}
\flushleft{\begin{hindi}
(ऐ रसूल) उनसे पूछो तो कि क्या हम लोग ख़ुदा को छोड़कर उन (माबूदों) से मुनाज़ात (दुआ) करे जो न तो हमें नफ़ा पहुंचा सकते हैं न हमारा कुछ बिगाड़ ही सकते हैं- और जब ख़ुदा हमारी हिदायत कर चुका) उसके बाद उल्टे पावँ कुफ्र की तरफ उस शख़्श की तरह फिर जाएं जिसे शैतानों ने जंगल में भटका दिया हो और वह हैरान (परेशान) हो (कि कहा जाए क्या करें) और उसके कुछ रफीक़ हो कि उसे राहे रास्त (सीधे रास्ते) की तरफ पुकारते रह जाएं कि (उधर) हमारे पास आओ और वह एक न सुने (ऐ रसूल) तुम कह दो कि हिदायत तो बस ख़ुदा की हिदायत है और हमें तो हुक्म ही दिया गया है कि हम सारे जहॉन के परवरदिगार ख़ुदा के फरमाबरदार हैं
\end{hindi}}
\flushright{\begin{Arabic}
\quranayah[6][71]
\end{Arabic}}
\flushleft{\begin{hindi}
और ये (भी हुक्म हुआ है) कि पाबन्दी से नमाज़ पढ़ा करो और उसी से डरते रहो और वही तो वह (ख़ुदा) है जिसके हुज़ूर में तुम सब के सब हाज़िर किए जाओगे
\end{hindi}}
\flushright{\begin{Arabic}
\quranayah[6][72]
\end{Arabic}}
\flushleft{\begin{hindi}
वह तो वह (ख़ुदा है) जिसने ठीक ठीक बहुतेरे आसमान व ज़मीन पैदा किए और जिस दिन (किसी चीज़ को) कहता है कि हो जा तो (फौरन) हो जाती है
\end{hindi}}
\flushright{\begin{Arabic}
\quranayah[6][73]
\end{Arabic}}
\flushleft{\begin{hindi}
उसका क़ौल सच्चा है और जिस दिन सूर फूंका जाएगा (उस दिन) ख़ास उसी की बादशाहत होगी (वही) ग़ायब हाज़िर (सब) का जानने वाला है और वही दाना वाक़िफकार है
\end{hindi}}
\flushright{\begin{Arabic}
\quranayah[6][74]
\end{Arabic}}
\flushleft{\begin{hindi}
(ऐ रसूल) उस वक्त क़ा याद करो) जब इबराहीम ने अपने (मुंह बोले) बाप आज़र से कहा क्या तुम बुतों को ख़ुदा मानते हो-मै तो तुमको और तुम्हारी क़ौम को खुली गुमराही में देखता हूँ
\end{hindi}}
\flushright{\begin{Arabic}
\quranayah[6][75]
\end{Arabic}}
\flushleft{\begin{hindi}
और (जिस तरह हमने इबराहीम को दिखाया था कि बुत क़ाबिले परसतिश (पूजने के क़ाबिल) नहीं) उसी तरह हम इबराहीम को सारे आसमान और ज़मीन की सल्तनत का (इन्तज़ाम) दिखाते रहे ताकि वह (हमारी वहदानियत का) यक़ीन करने वालों से हो जाएं
\end{hindi}}
\flushright{\begin{Arabic}
\quranayah[6][76]
\end{Arabic}}
\flushleft{\begin{hindi}
तो जब उन पर रात की तारीक़ी (अंधेरा) छा गयी तो एक सितारे को देखा तो दफअतन बोल उठे (हाए क्या) यही मेरा ख़ुदा है फिर जब वह डूब गया तो कहने लगे ग़ुरुब (डूब) हो जाने वाली चीज़ को तो मै (ख़ुदा बनाना) पसन्द नहीं करता
\end{hindi}}
\flushright{\begin{Arabic}
\quranayah[6][77]
\end{Arabic}}
\flushleft{\begin{hindi}
फिर जब चाँद को जगमगाता हुआ देखा तो बोल उठे (क्या) यही मेरा ख़ुदा है फिर जब वह भी ग़ुरुब हो गया तो कहने लगे कि अगर (कहीं) मेरा (असली) परवरदिगार मेरी हिदायत न करता तो मैं ज़रुर गुमराह लोगों में हो जाता
\end{hindi}}
\flushright{\begin{Arabic}
\quranayah[6][78]
\end{Arabic}}
\flushleft{\begin{hindi}
फिर जब आफताब को दमकता हुआ देखा तो कहने लगे (क्या) यही मेरा ख़ुदा है ये तो सबसे बड़ा (भी) है फिर जब ये भी ग़ुरुब हो गया तो कहने लगे ऐ मेरी क़ौम जिन जिन चीज़ों को तुम लोग (ख़ुदा का) शरीक बनाते हो उनसे मैं बेज़ार हूँ
\end{hindi}}
\flushright{\begin{Arabic}
\quranayah[6][79]
\end{Arabic}}
\flushleft{\begin{hindi}
(ये हरगिज़ नहीं हो सकते) मैने तो बातिल से कतराकर उसकी तरफ से मुँह कर लिया है जिसने बहुतेरे आसमान और ज़मीन पैदा किए और मैं मुशरेकीन से नहीं हूँ
\end{hindi}}
\flushright{\begin{Arabic}
\quranayah[6][80]
\end{Arabic}}
\flushleft{\begin{hindi}
और उनकी क़ौम के लोग उनसे हुज्जत करने लगे तो इबराहीम ने कहा था क्या तुम मुझसे ख़ुदा के बारे में हुज्जत करते हो हालॉकि वह यक़ीनी मेरी हिदायत कर चुका और तुम मे जिन बुतों को उसका शरीक मानते हो मै उनसे डरता (वरता) नहीं (वह मेरा कुछ नहीं कर सकते) मगर हॉ मेरा ख़ुदा खुद (करना) चाहे तो अलबत्ता कर सकता है मेरा परवरदिगार तो बाएतबार इल्म के सब पर हावी है तो क्या उस पर भी तुम नसीहत नहीं मानते
\end{hindi}}
\flushright{\begin{Arabic}
\quranayah[6][81]
\end{Arabic}}
\flushleft{\begin{hindi}
और जिन्हें तुम ख़ुदा का शरीक बताते हो मै उन से क्यों डरुँ जब तुम इस बात से नहीं डरते कि तुमने ख़ुदा का शरीक ऐसी चीज़ों को बनाया है जिनकी ख़ुदा ने कोई सनद तुम पर नहीं नाज़िल की फिर अगर तुम जानते हो तो (भला बताओ तो सही कि) हम दोनों फरीक़ (गिरोह) में अमन क़ायम रखने का ज्यादा हक़दार कौन है
\end{hindi}}
\flushright{\begin{Arabic}
\quranayah[6][82]
\end{Arabic}}
\flushleft{\begin{hindi}
जिन लोगों ने ईमान क़ुबूल किया और अपने ईमान को ज़ुल्म (शिर्क) से आलूदा नहीं किया उन्हीं लोगों के लिए अमन (व इतमिनान) है और यही लोग हिदायत याफ़ता हैं
\end{hindi}}
\flushright{\begin{Arabic}
\quranayah[6][83]
\end{Arabic}}
\flushleft{\begin{hindi}
और ये हमारी (समझाई बुझाई) दलीलें हैं जो हमने इबराहीम को अपनी क़ौम पर (ग़ालिब आने के लिए) अता की थी हम जिसके मरतबे चाहते हैं बुलन्द करते हैं बेशक तुम्हारा परवरदिगार हिक़मत वाला बाख़बर है
\end{hindi}}
\flushright{\begin{Arabic}
\quranayah[6][84]
\end{Arabic}}
\flushleft{\begin{hindi}
और हमने इबराहीम को इसहाक़ वा याक़ूब (सा बेटा पोता) अता किया हमने सबकी हिदायत की और उनसे पहले नूह को (भी) हम ही ने हिदायत की और उन्हीं (इबराहीम) को औलाद से दाऊद व सुलेमान व अय्यूब व यूसुफ व मूसा व हारुन (सब की हमने हिदायत की) और नेकों कारों को हम ऐसा ही इल्म अता फरमाते हैं
\end{hindi}}
\flushright{\begin{Arabic}
\quranayah[6][85]
\end{Arabic}}
\flushleft{\begin{hindi}
और ज़करिया व यहया व ईसा व इलियास (सब की हिदायत की (और ये) सब (ख़ुदा के) नेक बन्दों से हैं
\end{hindi}}
\flushright{\begin{Arabic}
\quranayah[6][86]
\end{Arabic}}
\flushleft{\begin{hindi}
और इसमाइल व इलियास व युनूस व लूत (की भी हिदायत की) और सब को सारे जहाँन पर फज़ीलत अता की
\end{hindi}}
\flushright{\begin{Arabic}
\quranayah[6][87]
\end{Arabic}}
\flushleft{\begin{hindi}
और (सिर्फ उन्हीं को नहीं बल्कि) उनके बाप दादाओं और उनकी औलाद और उनके भाई बन्दों में से (बहुतेरों को) और उनके मुन्तख़िब किया और उन्हें सीधी राह की हिदायत की
\end{hindi}}
\flushright{\begin{Arabic}
\quranayah[6][88]
\end{Arabic}}
\flushleft{\begin{hindi}
(देखो) ये ख़ुदा की हिदायत है अपने बन्दों से जिसको चाहे उसीकी वजह से राह पर लाए और अगर उन लोगों ने शिर्क किया होता तो उनका किया (धरा) सब अकारत हो जाता
\end{hindi}}
\flushright{\begin{Arabic}
\quranayah[6][89]
\end{Arabic}}
\flushleft{\begin{hindi}
(पैग़म्बर) वह लोग थे जिनको हमने (आसमानी) किताब और हुकूमत और नुबूवत अता फरमाई पस अगर ये लोग उसे भी न माने तो (कुछ परवाह नहीं) हमने तो उस पर ऐसे लोगों को मुक़र्रर कर दिया हे जो (उनकी तरह) इन्कार करने वाले नहीं
\end{hindi}}
\flushright{\begin{Arabic}
\quranayah[6][90]
\end{Arabic}}
\flushleft{\begin{hindi}
(ये अगले पैग़म्बर) वह लोग थे जिनकी ख़ुदा ने हिदायत की पस तुम भी उनकी हिदायत की पैरवी करो (ऐ रसूल उन से) कहो कि मै तुम से इस (रिसालत) की मज़दूरी कुछ नहीं चाहता सारे जहाँन के लिए सिर्फ नसीहत है
\end{hindi}}
\flushright{\begin{Arabic}
\quranayah[6][91]
\end{Arabic}}
\flushleft{\begin{hindi}
और बस और उन लोगों (यहूद) ने ख़ुदा की जैसी क़दर करनी चाहिए न की इसलिए कि उन लोगों ने (बेहूदे पन से) ये कह दिया कि ख़ुदा ने किसी बशर (इनसान) पर कुछ नाज़िल नहीं किया (ऐ रसूल) तुम पूछो तो कि फिर वह किताब जिसे मूसा लेकर आए थे किसने नाज़िल की जो लोगों के लिए रौशनी और (अज़सरतापा(सर से पैर तक)) हिदायत (थी जिसे तुम लोगों ने अलग-अलग करके काग़जी औराक़ (कागज़ के पन्ने) बना डाला और इसमें को कुछ हिस्सा (जो तुम्हारे मतलब का है वह) तो ज़ाहिर करते हो और बहुतेरे को (जो ख़िलाफ मदआ है) छिपाते हो हालॉकि उसी किताब के ज़रिए से तुम्हें वो बातें सिखायी गयी जिन्हें न तुम जानते थे और न तुम्हारे बाप दादा (ऐ रसूल वह तो जवाब देगें नहीं) तुम ही कह दो कि ख़ुदा ने (नाज़िल फरमाई)
\end{hindi}}
\flushright{\begin{Arabic}
\quranayah[6][92]
\end{Arabic}}
\flushleft{\begin{hindi}
उसके बाद उन्हें छोड़ के (पडे झक मारा करें (और) अपनी तू तू मै मै में खेलते फिरें और (क़ुरान) भी वह किताब है जिसे हमने बाबरकत नाज़िल किया और उस किताब की तसदीक़ करती है जो उसके सामने (पहले से) मौजूद है और (इस वास्ते नाज़िल किया है) ताकि तुम उसके ज़रिए से अहले मक्का और उसके एतराफ़ के रहने वालों को (ख़ौफ ख़ुदा से) डराओ और जो लोग आख़िरत पर ईमान रखते हैं वह तो उस पर (बे ताम्मुल) ईमान लाते है और वही अपनी अपनी नमाज़ में भी पाबन्दी करते हैं
\end{hindi}}
\flushright{\begin{Arabic}
\quranayah[6][93]
\end{Arabic}}
\flushleft{\begin{hindi}
और उससे बढ़ कर ज़ालिम कौन होगा जो ख़ुदा पर झूठ (मूठ) इफ़तेरा करके कहे कि हमारे पास वही आयी है हालॉकि उसके पास वही वगैरह कुछ भी नही आयी या वह शख़्श दावा करे कि जैसा क़ुरान ख़ुदा ने नाज़िल किया है वैसा मै भी (अभी) अनक़रीब (जल्दी) नाज़िल किए देता हूँ और (ऐ रसूल) काश तुम देखते कि ये ज़ालिम मौत की सख्तियों में पड़ें हैं और फरिश्ते उनकी तरफ (जान निकाल लेने के वास्ते) हाथ लपका रहे हैं और कहते जाते हैं कि अपनी जानें निकालो आज ही तो तुम को रुसवाई के अज़ाब की सज़ा दी जाएगी क्योंकि तुम ख़ुदा पर नाहक़ (नाहक़) झूठ छोड़ा करते थे और उसकी आयतों को (सुनकर उन) से अकड़ा करते थे
\end{hindi}}
\flushright{\begin{Arabic}
\quranayah[6][94]
\end{Arabic}}
\flushleft{\begin{hindi}
और आख़िर तुम हमारे पास इसी तरह तन्हा आए (ना) जिस तरह हमने तुम को पहली बार पैदा किया था और जो (माल व औलाद) हमने तुमको दिया था वह सब अपने पस्त पुश्त (पीछे) छोड़ आए और तुम्हारे साथ तुम्हारे उन सिफारिश करने वालों को भी नहीं देखते जिन को तुम ख्याल करते थे कि वह तुम्हारी (परवरिश वगैरह) मै (हमारे) साझेदार है अब तो तुम्हारे बाहरी ताल्लुक़ात मनक़तआ (ख़त्म) हो गए और जो कुछ ख्याल करते थे वह सब तुम से ग़ायब हो गए
\end{hindi}}
\flushright{\begin{Arabic}
\quranayah[6][95]
\end{Arabic}}
\flushleft{\begin{hindi}
ख़ुदा ही तो गुठली और दाने को चीर (करके दरख्त ऊगाता) है वही मुर्दे में से ज़िन्दे को निकालता है और वही ज़िन्दा से मुर्दे को निकालने वाला है (लोगों) वही तुम्हारा ख़ुदा है फिर तुम किधर बहके जा रहे हो
\end{hindi}}
\flushright{\begin{Arabic}
\quranayah[6][96]
\end{Arabic}}
\flushleft{\begin{hindi}
उसी के लिए सुबह की पौ फटी और उसी ने आराम के लिए रात और हिसाब के लिए सूरज और चाँद बनाए ये ख़ुदाए ग़ालिब व दाना के मुक़र्रर किए हुए किरदा (उसूल) हैं
\end{hindi}}
\flushright{\begin{Arabic}
\quranayah[6][97]
\end{Arabic}}
\flushleft{\begin{hindi}
और वह वही (ख़ुदा) है जिसने तुम्हारे (नफे के) वास्ते सितारे पैदा किए ताकि तुम जॅगलों और दरियाओं की तारिक़ियों (अंधेरों) में उनसे राह मालूम करो जो लोग वाक़िफकार हैं उनके लिए हमने (अपनी क़ुदरत की) निशानियाँ ख़ूब तफ़सील से बयान कर दी हैं
\end{hindi}}
\flushright{\begin{Arabic}
\quranayah[6][98]
\end{Arabic}}
\flushleft{\begin{hindi}
और वह वही ख़ुदा है जिसने तुम लोगों को एक शख़्श से पैदा किया फिर (हर शख़्श के) क़रार की जगह (बाप की पुश्त (पीठ)) और सौंपने की जगह (माँ का पेट) मुक़र्रर है हमने समझदार लोगों के वास्ते (अपनी कुदरत की) निशानियाँ ख़ूब तफसील से बयान कर दी हैं
\end{hindi}}
\flushright{\begin{Arabic}
\quranayah[6][99]
\end{Arabic}}
\flushleft{\begin{hindi}
और वह वही (क़ादिर तवाना है) जिसने आसमान से पानी बरसाया फिर हम ही ने उसके ज़रिए से हर चीज़ के कोए निकालें फिर हम ही ने उससे हरी भरी टहनियाँ निकालीं कि उससे हम बाहम गुत्थे दाने निकालते हैं और छुहारे के बोर (मुन्जिर) से लटके हुए गुच्छे पैदा किए और अंगूर और ज़ैतून और अनार के बाग़ात जो बाहम सूरत में एक दूसरे से मिलते जुलते और (मजे में) जुदा जुदा जब ये पिघले और पक्के तो उसके फल की तरफ ग़ौर तो करो बेशक अमन में ईमानदार लोगों के लिए बहुत सी (ख़ुदा की) निशानियाँ हैं
\end{hindi}}
\flushright{\begin{Arabic}
\quranayah[6][100]
\end{Arabic}}
\flushleft{\begin{hindi}
और उन (कम्बख्तों) ने जिन्नात को ख़ुदा का शरीक बनाया हालॉकि जिन्नात को भी ख़ुदा ही ने पैदा किया उस पर भी उन लोगों ने बे समझे बूझे ख़ुदा के लिए बेटे बेटियाँ गढ़ डालीं जो बातों में लोग (उसकी शान में) बयान करते हैं उससे वह पाक व पाकीज़ा और बरतर है
\end{hindi}}
\flushright{\begin{Arabic}
\quranayah[6][101]
\end{Arabic}}
\flushleft{\begin{hindi}
सारे आसमान और ज़मीन का मवविद (बनाने वाला) है उसके कोई लड़का क्योंकर हो सकता है जब उसकी कोई बीबी ही नहीं है और उसी ने हर चीज़ को पैदा किया और वही हर चीज़ से खूब वाक़िफ है
\end{hindi}}
\flushright{\begin{Arabic}
\quranayah[6][102]
\end{Arabic}}
\flushleft{\begin{hindi}
(लोगों) वही अल्लाह तुम्हारा परवरदिगार है उसके सिवा कोई माबूद नहीं वही हर चीज़ का पैदा करने वाला है तो उसी की इबादत करो और वही हर चीज़ का निगेह बान है
\end{hindi}}
\flushright{\begin{Arabic}
\quranayah[6][103]
\end{Arabic}}
\flushleft{\begin{hindi}
उसको ऑंखें देख नहीं सकती (न दुनिया में न आख़िरत में) और वह (लोगों की) नज़रों को खूब देखता है और वह बड़ा बारीक बीन (देख़ने वाला) ख़बरदार है
\end{hindi}}
\flushright{\begin{Arabic}
\quranayah[6][104]
\end{Arabic}}
\flushleft{\begin{hindi}
तुम्हारे पास तो सुझाने वाली चीज़े आ ही चुकीं फिर जो देखे (समझे) तो अपने दम के लिए और जो अन्धा बने तो (उसका ज़रर (नुकसान) भी) ख़ुद उस पर है और (ऐ रसूल उन से कह दो) कि मै तुम लोगों का कुछ निगेहबान तो हूँ नहीं
\end{hindi}}
\flushright{\begin{Arabic}
\quranayah[6][105]
\end{Arabic}}
\flushleft{\begin{hindi}
और हम (अपनी) आयतें यूं उलट फेरकर बयान करते है (ताकि हुज्जत तमाम हो) और ताकि वह लोग ज़बानी भी इक़रार कर लें कि तुमने (क़ुरान उनके सामने) पढ़ दिया और ताकि जो लोग जानते है उनके लिए (क़ुरान का) खूब वाजेए करके बयान कर दें
\end{hindi}}
\flushright{\begin{Arabic}
\quranayah[6][106]
\end{Arabic}}
\flushleft{\begin{hindi}
जो कुछ तुम्हारे पास तुम्हारे परवरदिगार की तरफ से 'वही' की जाए बस उसी पर चलो अल्लाह के सिवा कोई माबूद नहीं और मुशरिको से किनारा कश रहो
\end{hindi}}
\flushright{\begin{Arabic}
\quranayah[6][107]
\end{Arabic}}
\flushleft{\begin{hindi}
और अगर ख़ुदा चाहता तो ये लोग शिर्क ही न करते और हमने तुमको उन लोगों का निगेहबान तो बनाया नहीं है और न तुम उनके ज़िम्मेदार हो
\end{hindi}}
\flushright{\begin{Arabic}
\quranayah[6][108]
\end{Arabic}}
\flushleft{\begin{hindi}
और ये (मुशरेकीन) जिन की अल्लाह के सिवा (ख़ुदा समझ कर) इबादत करते हैं उन्हें तुम बुरा न कहा करो वरना ये लोग भी ख़ुदा को बिना समझें अदावत से बुरा (भला) कह बैठें (और लोग उनकी ख्वाहिश नफसानी के) इस तरह पाबन्द हुए कि गोया हमने ख़ुद हर गिरोह के आमाल उनको सॅवाकर अच्छे कर दिखाए फिर उन्हें तो (आख़िरकार) अपने परवरदिगार की तरफ लौट कर जाना है तब जो कुछ दुनिया में कर रहे थे ख़ुदा उन्हें बता देगा
\end{hindi}}
\flushright{\begin{Arabic}
\quranayah[6][109]
\end{Arabic}}
\flushleft{\begin{hindi}
और उन लोगों ने ख़ुदा की सख्त सख्त क़समें खायीं कि अगर उनके पास कोई मौजिजा आए तो वह ज़रूर उस पर ईमान लाएँगे (ऐ रसूल) तुम कहो कि मौजिज़े तो बस ख़ुदा ही के पास हैं और तुम्हें क्या मालूम ये यक़ीनी बात है कि जब मौजिज़ा भी आएगा तो भी ये ईमान न लाएँगे
\end{hindi}}
\flushright{\begin{Arabic}
\quranayah[6][110]
\end{Arabic}}
\flushleft{\begin{hindi}
और हम उनके दिल और उनकी ऑंखें उलट पलट कर देंगे जिस तरह ये लोग कुरान पर पहली मरतबा ईमान न लाए और हम उन्हें उनकी सरकशी की हालत में छोड़ देंगे कि सरगिरदाँ (परेशान) रहें
\end{hindi}}
\flushright{\begin{Arabic}
\quranayah[6][111]
\end{Arabic}}
\flushleft{\begin{hindi}
और (ऐ रसूल सच तो ये है कि) हम अगर उनके पास फरिश्ते भी नाज़िल करते और उनसे मुर्दे भी बातें करने लगते और तमाम (मख़फ़ी(छुपी)) चीज़ें (जैसे जन्नत व नार वग़ैरह) अगर वह गिरोह उनके सामने ला खड़े करते तो भी ये ईमान लाने वाले न थे मगर जब अल्लाह चाहे लेकिन उनमें के अक्सर नहीं जानते
\end{hindi}}
\flushright{\begin{Arabic}
\quranayah[6][112]
\end{Arabic}}
\flushleft{\begin{hindi}
कि और (ऐ रसूल जिस तरह ये कुफ्फ़ार तुम्हारे दुश्मन हैं) उसी तरह (गोया हमने ख़ुद आज़माइश के लिए शरीर आदमियों और जिनों को हर नबी का दुश्मन बनाया वह लोग एक दूसरे को फरेब देने की ग़रज़ से चिकनी चुपड़ी बातों की सरग़ोशी करते हैं और अगर तुम्हारा परवरदिगार चाहता तो ये लोग) ऐसी हरकत करने न पाते
\end{hindi}}
\flushright{\begin{Arabic}
\quranayah[6][113]
\end{Arabic}}
\flushleft{\begin{hindi}
तो उनको और उनकी इफ़तेरा परदाज़ियों को छोड़ दो और ये (ये सरग़ोशियाँ इसलिए थीं) ताकि जो लोग आख़िरत पर ईमान नहीं लाए उनके दिल उन (की शरारत) की तरफ मायल (ख्ािंच) हो जाएँ और उन्हें पसन्द करें
\end{hindi}}
\flushright{\begin{Arabic}
\quranayah[6][114]
\end{Arabic}}
\flushleft{\begin{hindi}
और ताकि जो लोग इफ़तेरा परदाज़ियाँ ये लोग ख़ुद करते हैं वह भी करने लगें (क्या तुम ये चाहते हो कि) मैं ख़ुदा को छोड़ कर किसी और को सालिस तलाश करुँ हालॉकि वह वही ख़ुदा है जिसने तुम्हारे पास वाज़ेए किताब नाज़िल की और जिन लोगों को हमने किताब अता फरमाई है वह यक़ीनी तौर पर जानते हैं कि ये (कुरान भी) तुम्हारे परवरदिगार की तरफ से बरहक़ नाज़िल किया गया है
\end{hindi}}
\flushright{\begin{Arabic}
\quranayah[6][115]
\end{Arabic}}
\flushleft{\begin{hindi}
तो तुम (कहीं) शक़ करने वालों से न हो जाना और सच्चाई और इन्साफ में तो तुम्हारे परवरदिगार की बात पूरी हो गई कोई उसकी बातों का बदलने वाला नहीं और वही बड़ा सुनने वाला वाक़िफकार है
\end{hindi}}
\flushright{\begin{Arabic}
\quranayah[6][116]
\end{Arabic}}
\flushleft{\begin{hindi}
और (ऐ रसूल) दुनिया में तो बहुतेरे लोग ऐसे हैं कि तुम उनके कहने पर चलो तो तुमको ख़ुदा की राह से बहका दें ये लोग तो सिर्फ अपने ख्यालात की पैरवी करते हैं और ये लोग तो बस अटकल पच्चू बातें किया करते हैं
\end{hindi}}
\flushright{\begin{Arabic}
\quranayah[6][117]
\end{Arabic}}
\flushleft{\begin{hindi}
(तो तुम क्या जानों) जो लोग उसकी राह से बहके हुए हैं उनको (कुछ) ख़ुदा ही ख़ूब जानता है और वह तो हिदायत याफ्ता लोगों से भी ख़ूब वाक़िफ है
\end{hindi}}
\flushright{\begin{Arabic}
\quranayah[6][118]
\end{Arabic}}
\flushleft{\begin{hindi}
तो अगर तुम उसकी आयतों पर ईमान रखते हो तो जिस ज़ीबह पर (वक्ते ़ज़िबाह) ख़ुदा का नाम लिया गया हो उसी को खाओ
\end{hindi}}
\flushright{\begin{Arabic}
\quranayah[6][119]
\end{Arabic}}
\flushleft{\begin{hindi}
और तुम्हें क्या हो गया है कि जिस पर ख़ुदा का नाम लिया गया हो उसमें नहीं खाते हो हालॉकि जो चीज़ें उसने तुम पर हराम कर दीं हैं वह तुमसे तफसीलन बयान कर दीं हैं मगर (हाँ) जब तुम मजबूर हो तो अलबत्ता (हराम भी खा सकते हो) और बहुतेरे तो (ख्वाहमख्वाह) अपनी नफसानी ख्वाहिशों से बे समझे बूझे (लोगों को) बहका देते हैं और तुम्हारा परवरदिगार तो हक़ से तजाविज़ करने वालों से ख़ूब वाक़िफ है
\end{hindi}}
\flushright{\begin{Arabic}
\quranayah[6][120]
\end{Arabic}}
\flushleft{\begin{hindi}
(ऐ लोगों) ज़ाहिरी और बातिनी गुनाह (दोनों) को (बिल्कुल) छोड़ दो जो लोग गुनाह करते हैं उन्हें अपने आमाल का अनक़रीब ही बदला दिया जाएगा
\end{hindi}}
\flushright{\begin{Arabic}
\quranayah[6][121]
\end{Arabic}}
\flushleft{\begin{hindi}
और जिस (ज़बीहे) पर ख़ुदा का नाम न लिया गया उसमें से मत खाओ (क्योंकि) ये बेशक बदचलनी है और शयातीन तो अपने हवा ख़वाहों के दिल में वसवसा डाला ही करते हैं ताकि वह तुमसे (बेकार) झगड़े किया करें और अगर (कहीं) तुमने उनका कहना मान लिया तो (समझ रखो कि) बेशुबहा तुम भी मुशरिक हो
\end{hindi}}
\flushright{\begin{Arabic}
\quranayah[6][122]
\end{Arabic}}
\flushleft{\begin{hindi}
क्या जो शख़्श (पहले) मुर्दा था फिर हमने उसको ज़िन्दा किया और उसके लिए एक नूर बनाया जिसके ज़रिए वह लोगों में (बेतकल्लुफ़) चलता फिरता है उस शख़्श का सामना हो सकता है जिसकी ये हालत है कि (हर तरफ से) अंधेरे में (फँसा हुआ है) कि वहाँ से किसी तरह निकल नहीं सकता (जिस तरह मोमिनों के वास्ते ईमान आरास्ता किया गया) उसी तरह काफिरों के वास्ते उनके आमाल (बद) आरास्ता कर दिए गए हैं
\end{hindi}}
\flushright{\begin{Arabic}
\quranayah[6][123]
\end{Arabic}}
\flushleft{\begin{hindi}
(कि भला ही भला नज़र आता है) और जिस तरह मक्के में है उसी तरह हमने हर बस्ती में उनके कुसूरवारों को सरदार बनाया ताकि उनमें मक्कारी किया करें और वह लोग जो कुछ करते हैं अपने ही हक़ में (बुरा) करते हैं और समझते (तक) नहीं
\end{hindi}}
\flushright{\begin{Arabic}
\quranayah[6][124]
\end{Arabic}}
\flushleft{\begin{hindi}
और जब उनके पास कोई निशानी (नबी की तसदीक़ के लिए) आई है तो कहते हैं जब तक हमको ख़ुद वैसी चीज़ (वही वग़ैरह) न दी जाएगी जो पैग़म्बराने ख़ुदा को दी गई है उस वक्त तक तो हम ईमान न लाएँगे और ख़ुदा जहाँ (जिस दिल में) अपनी पैग़म्बरी क़रार देता है उसकी (काबलियत व सलाहियत) को ख़ूब जानता है जो लोग (उस जुर्म के) मुजरिम हैं उनको अनक़रीब उनकी मक्कारी की सज़ा में ख़ुदा के यहाँ बड़ी ज़िल्लत और सख्त अज़ाब होगा
\end{hindi}}
\flushright{\begin{Arabic}
\quranayah[6][125]
\end{Arabic}}
\flushleft{\begin{hindi}
तो ख़ुदा जिस शख़्श को राह रास्त दिखाना चाहता है उसके सीने को इस्लाम (की दौलियत) के वास्ते (साफ़ और) कुशादा (चौड़ा) कर देता है और जिसको गुमराही की हालत में छोड़ना चाहता है उनके सीने को तंग दुश्वार ग़ुबार कर देता है गोया (कुबूल ईमान) उसके लिए आसमान पर चढ़ना है जो लोग ईमान नहीं लाते ख़ुदा उन पर बुराई को उसी तरह मुसल्लत कर देता है
\end{hindi}}
\flushright{\begin{Arabic}
\quranayah[6][126]
\end{Arabic}}
\flushleft{\begin{hindi}
और (ऐ रसूल) ये (इस्लाम) तुम्हारे परवरदिगार का (बनाया हुआ) सीधा रास्ता है इबरत हासिल करने वालों के वास्ते हमने अपने आयात तफसीलन बयान कर दिए हैं
\end{hindi}}
\flushright{\begin{Arabic}
\quranayah[6][127]
\end{Arabic}}
\flushleft{\begin{hindi}
उनके वास्ते उनके परवरदिगार के यहाँ अमन व चैन का घर (बेहश्त) है और दुनिया में जो कारगुज़ारियाँ उन्होने की थीं उसके ऐवज़ ख़ुदा उन का सरपरस्त होगा
\end{hindi}}
\flushright{\begin{Arabic}
\quranayah[6][128]
\end{Arabic}}
\flushleft{\begin{hindi}
और (ऐ रसूल वह दिन याद दिलाओ) जिस दिन ख़ुदा सब लोगों को जमा करेगा और शयातीन से फरमाएगा, ऐ गिरोह जिन्नात तुमने तो बहुतेरे आदमियों को (बहका बहका कर) अपनी जमाअत बड़ी कर ली (और) आदमियों से जो लोग (उन शयातीन के दुनिया में) दोस्त थे कहेंगे ऐ हमारे पालने वाले (दुनिया में) हमने एक दूसरे से फायदा हासिल किया और अपने किए की सज़ा पाने को, जो वक्त तू ने हमारे लिए मुअय्युन किया था अब हम अपने उस वक्त (क़यामत) में पहुँच गए ख़ुदा उसके जवाब में, फरमाएगा तुम सब का ठिकाना जहन्नुम है और उसमें हमेशा रहोगे मगर जिसे ख़ुदा चाहे (नजात दे) बेशक तेरा परवरदिगार हिकमत वाला वाक़िफकार है
\end{hindi}}
\flushright{\begin{Arabic}
\quranayah[6][129]
\end{Arabic}}
\flushleft{\begin{hindi}
और इसी तरह हम बाज़ ज़ालिमों को बाज़ का उनके करतूतों की बदौलत सरपरस्त बनाएँगे
\end{hindi}}
\flushright{\begin{Arabic}
\quranayah[6][130]
\end{Arabic}}
\flushleft{\begin{hindi}
(फिर हम पूछेंगे कि क्यों) ऐ गिरोह जिन व इन्स क्या तुम्हारे पास तुम ही में के पैग़म्बर नहीं आए जो तुम तुमसे हमारी आयतें बयान करें और तुम्हें तुम्हारे उस रोज़ (क़यामत) के पेश आने से डराएँ वह सब अर्ज करेंगे (बेशक आए थे) हम ख़ुद अपने ऊपर आप अपने (ख़िलाफ) गवाही देते हैं (वाकई) उनको दुनिया की (चन्द रोज़) ज़िन्दगी ने उन्हें अंधेरे में डाल रखा और उन लोगों ने अपने ख़िलाफ आप गवाही दीं
\end{hindi}}
\flushright{\begin{Arabic}
\quranayah[6][131]
\end{Arabic}}
\flushleft{\begin{hindi}
बेशक ये सब के सब काफिर थे और ये (पैग़म्बरों का भेजना सिर्फ) उस वजह से है कि तुम्हारा परवरदिगार कभी बस्तियों को ज़ुल्म ज़बरदस्ती से वहाँ के बाशिन्दों के ग़फलत की हालत में हलाक नहीं किया करता
\end{hindi}}
\flushright{\begin{Arabic}
\quranayah[6][132]
\end{Arabic}}
\flushleft{\begin{hindi}
और जिसने जैसा (भला या बुरा) किया है उसी के मुवाफ़िक हर एक के दरजात हैं
\end{hindi}}
\flushright{\begin{Arabic}
\quranayah[6][133]
\end{Arabic}}
\flushleft{\begin{hindi}
और जो कुछ वह लोग करते हैं तुम्हारा परवरदिगार उससे बेख़बर नहीं और तुम्हारा परवरदिगार बे परवाह रहम वाला है - अगर चाहे तो तुम सबके सबको (दुनिया से उड़ा) ले लाए और तुम्हारे बाद जिसको चाहे तुम्हार जानशीन बनाए जिस तरह आख़िर तुम्हें दूसरे लोगों की औलाद से पैदा किया है
\end{hindi}}
\flushright{\begin{Arabic}
\quranayah[6][134]
\end{Arabic}}
\flushleft{\begin{hindi}
बेशक जिस चीज़ का तुमसे वायदा किया जाता है वह ज़रूर (एक न एक दिन) आने वाली है
\end{hindi}}
\flushright{\begin{Arabic}
\quranayah[6][135]
\end{Arabic}}
\flushleft{\begin{hindi}
और तुम उसके लाने में (ख़ुदा को) आजिज़ नहीं कर सकते (ऐ रसूल तुम उनसे) कहो कि ऐ मेरी क़ौम तुम बजाए ख़ुद जो चाहो करो मैं (बजाए ख़ुद) अमल कर रहा हूँ फिर अनक़रीब तुम्हें मालूम हो जाएगा कि आख़ेरत (बेहश्त) किसके लिए है (तुम्हारे लिए या हमारे लिए) ज़ालिम लोग तो हरगिज़ कामयाब न होंगे
\end{hindi}}
\flushright{\begin{Arabic}
\quranayah[6][136]
\end{Arabic}}
\flushleft{\begin{hindi}
और ये लोग ख़ुदा की पैदा की हुई खेती और चौपायों में से हिस्सा क़रार देते हैं और अपने ख्याल के मुवाफिक कहते हैं कि ये तो ख़ुदा का (हिस्सा) है और ये हमारे शरीकों का (यानि जिनको हमने ख़ुदा का शरीक बनाया) फिर जो ख़ास उनके शरीकों का है वह तो ख़ुदा तक नहीं पहुँचने का और जो हिस्सा ख़ुदा का है वो उसके शरीकों तक पहुँच जाएगा ये क्या ही बुरा हुक्म लगाते हैं और उसी तरह बहुतेरे मुशरकीन को उनके शरीकों ने अपने बच्चों को मार डालने को अच्छा कर दिखाया है
\end{hindi}}
\flushright{\begin{Arabic}
\quranayah[6][137]
\end{Arabic}}
\flushleft{\begin{hindi}
ताकि उन्हें (बदी) हलाकत में डाल दें और उनके सच्चे दीन को उन पर मिला जुला दें और अगर ख़ुदा चाहता तो लोग ऐसा काम न करते तो तुम (ऐ रसूल) और उनकी इफ़तेरा परदाज़ियों को (ख़ुदा पर) छोड़ दो और ये लोग अपने ख्याल के मुवाफिक कहने लगे कि ये चौपाए और ये खेती अछूती है
\end{hindi}}
\flushright{\begin{Arabic}
\quranayah[6][138]
\end{Arabic}}
\flushleft{\begin{hindi}
उनको सिवा उसके जिसे हम चाहें कोई नहीं खा सकता और (उनका ये भी ख्याल है) कि कुछ चारपाए ऐसे हैं जिनकी पीठ पर सवारी लादना हराम किया गया और कुछ चारपाए ऐसे है जिन पर (ज़िबह के वक्त) ख़ुदा का नाम तक नहीं लेते और फिर यह ढकोसले (ख़ुदा की तरफ मनसूब करते) हैं ये सब ख़ुदा पर इफ़तेरा व बोहतान है ख़ुदा उनके इफ़तेरा परदाज़ियों को बहुत जल्द सज़ा देगा
\end{hindi}}
\flushright{\begin{Arabic}
\quranayah[6][139]
\end{Arabic}}
\flushleft{\begin{hindi}
और कुफ्फ़ार ये भी कहते हैं कि जो बच्चा (वक्त ़ज़बाह) उन जानवरों के पेट में है (जिन्हें हमने बुतों के नाम कर छोड़ा और ज़िन्दा पैदा होता तो) सिर्फ हमारे मर्दों के लिए हलाल है और हमारी औरतों पर हराम है और अगर वह मरा हुआ हो तो सब के सब उसमें शरीक हैं ख़ुदा अनक़रीब उनको बातें बनाने की सज़ा देगा बेशक वह हिकमत वाला बड़ा वाक़िफकार है
\end{hindi}}
\flushright{\begin{Arabic}
\quranayah[6][140]
\end{Arabic}}
\flushleft{\begin{hindi}
बेशक जिन लोगों ने अपनी औलाद को बे समझे बूझे बेवकूफी से मार डाला और जो रोज़ी ख़ुदा ने उन्हें दी थी उसे ख़ुदा पर इफ़तेरा (बोहतान) बाँध कर अपने ऊपर हराम कर डाला और वह सख्त घाटे में है ये यक़ीनन राहे हक़ से भटक गऐ और ये हिदायत पाने वाले थे भी नहीं
\end{hindi}}
\flushright{\begin{Arabic}
\quranayah[6][141]
\end{Arabic}}
\flushleft{\begin{hindi}
और वह तो वही ख़ुदा है जिसने बहुतेरे बाग़ पैदा किए (जिनमें मुख्तलिफ दरख्त हैं - कुछ तो अंगूर की तरह टट्टियों पर) चढ़ाए हुए और (कुछ) बे चढ़ाए हुए और खजूर के दरख्त और खेती जिसमें फल मुख्तलिफ़ किस्म के हैं और ज़ैतून और अनार बाज़ तो सूरत रंग मज़े में, मिलते जुलते और (बाज़) बेमेल (लोगों) जब ये चीज़े फलें तो उनका फल खाओ और उन चीज़ों के काटने के दिन ख़ुदा का हक़ (ज़कात) दे दो और ख़बरदार फज़ूल ख़र्ची न करो - क्यों कि वह (ख़ुदा) फुज़ूल ख़र्चे से हरगिज़ उलफत नहीं रखता
\end{hindi}}
\flushright{\begin{Arabic}
\quranayah[6][142]
\end{Arabic}}
\flushleft{\begin{hindi}
और चारपायों में से कुछ तो बोझ उठाने वाले (बड़े बड़े) और कुछ ज़मीन से लगे हुए (छोटे छोटे) पैदा किए ख़ुदा ने जो तुम्हें रोज़ी दी है उस में से खाओ और शैतान के क़दम ब क़दम न चलो
\end{hindi}}
\flushright{\begin{Arabic}
\quranayah[6][143]
\end{Arabic}}
\flushleft{\begin{hindi}
(क्यों कि) वह तो यक़ीनन तुम्हारा खुला हुआ दुश्मन है (ख़ुदा ने नर मादा मिलाकर) आठ (क़िस्म के) जोड़े पैदा किए हैं - भेड़ से (नर मादा) दो और बकरी से (नर मादा) दो (ऐ रसूल उन काफिरों से) पूछो तो कि ख़ुदा ने (उन दोनों भेड़ बकरी के) दोनों नरों को हराम कर दिया है या उन दोनों मादनियों को या उस बच्चे को जो उन दोनों मादनियों के पेट से अन्दर लिए हुए हैं
\end{hindi}}
\flushright{\begin{Arabic}
\quranayah[6][144]
\end{Arabic}}
\flushleft{\begin{hindi}
अगर तुम सच्चे हो तो ज़रा समझ के मुझे बताओ और ऊँट के (नर मादा) दो और गाय के (नर मादा) दो (ऐ रसूल तुम उनसे) पूछो कि ख़ुदा ने उन दोनों (ऊँट गाय के) नरों को हराम किया या दोनों मादनियोंको या उस बच्चे को जो दोनों मादनियों के पेट अपने अन्दर लिये हुए है क्या जिस वक्त ख़ुदा ने तुमको उसका हुक्म दिया था तुम उस वक्त मौजूद थे फिर जो ख़ुदा पर झूठ बोताहन बॉधे उससे ज्यादा ज़ालिम कौन होगा ताकि लोगों के वे समझे बूझे गुमराह करें ख़ुदा हरगिज़ ज़ालिम क़ौम में मंज़िले मक़सूद तक नहीं पहुचाता
\end{hindi}}
\flushright{\begin{Arabic}
\quranayah[6][145]
\end{Arabic}}
\flushleft{\begin{hindi}
(ऐ रसूल) तुम कहो कि मै तो जो (क़ुरान) मेरे पास वही के तौर पर आया है उसमें कोई चीज़ किसी खाने वाले पर जो उसको खाए हराम नहीं पाता मगर जबकि वह मुर्दा या बहता हुआ ख़़ून या सूअर का गोश्त हो तो बेशक ये (चीजे) नापाक और हराम हैं या (वह जानवर) नाफरमानी का बाएस हो कि (वक्ते ़ज़िबहा) ख़ुदा के सिवा किसी और का नाम लिया गया हो फिर जो शख्स (हर तरह) बेबस हो जाए (और) नाफरमान व सरकश न हो और इस हालत में खाए तो अलबत्ता तुम्हारा परवरदिगार बड़ा बख्शने वाला मेहरबान है
\end{hindi}}
\flushright{\begin{Arabic}
\quranayah[6][146]
\end{Arabic}}
\flushleft{\begin{hindi}
और हमने यहूदियों पर तमाम नाख़ूनदार जानवर हराम कर दिये थे और गाय और बकरी दोनों की चरबियां भी उन पर हराम कर दी थी मगर जो चरबी उनकी दोनों पीठ या आतों पर लगी हो या हडड्ी से मिली हुई हो (वह हलाल थी) ये हमने उन्हें उनकी सरक़शी की सज़ा दी थी और उसमें तो शक ही नहीं कि हम ज़रूर सच्चे हैं
\end{hindi}}
\flushright{\begin{Arabic}
\quranayah[6][147]
\end{Arabic}}
\flushleft{\begin{hindi}
(ऐ रसूल) पर अगर वह तुम्हें झुठलाएं तो तुम (जवाब) में कहो कि (अगरचे) तुम्हारा परवरदिगार बड़ी वसीह रहमत वाला है मगर उसका अज़ाब गुनाहगार लोगों से टलता भी नहीं
\end{hindi}}
\flushright{\begin{Arabic}
\quranayah[6][148]
\end{Arabic}}
\flushleft{\begin{hindi}
अनक़रीब मुशरेकीन कहेंगें कि अगर ख़ुदा चाहता तो न हम लोग शिर्क करते और न हमारे बाप दादा और न हम कोई चीज़ अपने ऊपर हराम करते उसी तरह (बातें बना बना के) जो लोग उनसे पहले हो गुज़रे हैं (पैग़म्बरों को) झुठलाते रहे यहाँ तक कि उन लोगों ने हमारे अज़ाब (के मज़े)े को चख़ा (ऐ रसूल) तुम कहो कि तुम्हारे पास कोई दलील है (अगर है) तो हमारे (दिखाने के) वास्ते उसको निकालो (दलील तो क्या) पेश करोगे तुम लोग तो सिर्फ अपने ख्याल ख़ाम की पैरवी करते हो और सिर्फ अटकल पच्चू बातें करते हो
\end{hindi}}
\flushright{\begin{Arabic}
\quranayah[6][149]
\end{Arabic}}
\flushleft{\begin{hindi}
(ऐ रसूल) तुम कहो कि (अब तुम्हारे पास कोई दलील नहीं है) ख़ुदा तक पहुंचाने वाली दलील ख़ुदा ही के लिए ख़ास है
\end{hindi}}
\flushright{\begin{Arabic}
\quranayah[6][150]
\end{Arabic}}
\flushleft{\begin{hindi}
फिर अगर वही चाहता तो तुम सबकी हिदायत करता (ऐ रसूल) तुम कह दो कि ( अच्छा) अपने गवाहों को लाकर हाज़िर करो जो ये गवाही दें कि ये चीज़े (जिन्हें तुम हराम मानते हो) खुदा ही ने हराम कर दी हैं फिर अगर (बिलग़रज़) वह गवाही दे भी दे तो (ऐ रसूल) कहीं तुम उनके साथ गवाही न देना और जिन लोगों ने हमारी आयतों को झुठलाया और आख़िरत पर ईमान नहीं लाते और दूसरों को अपने परवरदिगार का हम सर बनाते है उनकी नफ़सियानी ख्वाहिशों पर न चलना
\end{hindi}}
\flushright{\begin{Arabic}
\quranayah[6][151]
\end{Arabic}}
\flushleft{\begin{hindi}
(ऐ रसूल) तुम उनसे कहो कि (बेबस) आओ जो चीज़ें ख़ुदा ने तुम पर हराम की हैं वह मैं तुम्हें पढ़ कर सुनाऊँ (वह) यह कि किसी चीज़ को ख़ुदा का शरीक़ न बनाओ और माँ बाप के साथ नेक सुलूक़ करो और मुफ़लिसी के ख़ौफ से अपनी औलाद को मार न डालना (क्योंकि) उनको और तुमको रिज़क देने वाले तो हम हैं और बदकारियों के क़रीब भी न जाओ ख्वाह (चाहे) वह ज़ाहिर हो या पोशीदा और किसी जान वाले को जिस के क़त्ल को ख़ुदा ने हराम किया है न मार डालना मगर (किसी) हक़ के ऐवज़ में वह बातें हैं जिनका ख़ुदा ने तुम्हें हुक्म दिया है ताकि तुम लोग समझो और यतीम के माल के करीब भी न जाओ
\end{hindi}}
\flushright{\begin{Arabic}
\quranayah[6][152]
\end{Arabic}}
\flushleft{\begin{hindi}
लेकिन इस तरीके पर कि (उसके हक़ में) बेहतर हो यहाँ तक कि वह अपनी जवानी की हद को पहुंच जाए और इन्साफ के साथ नाप और तौल पूरी किया करो हम किसी शख्स को उसकी ताक़त से बढ़कर तकलीफ नहीं देते और (चाहे कुछ हो मगर) जब बात कहो तो इन्साफ़ से अगरचे वह (जिसके तुम ख़िलाफ न हो) तुम्हारा अज़ीज़ ही (क्यों न) हो और ख़ुदा के एहद व पैग़ाम को पूरा करो यह वह बातें हैं जिनका ख़ुदा ने तुम्हे हुक्म दिया है कि तुम इबरत हासिल करो और ये भी (समझ लो) कि यही मेरा सीधा रास्ता है
\end{hindi}}
\flushright{\begin{Arabic}
\quranayah[6][153]
\end{Arabic}}
\flushleft{\begin{hindi}
तो उसी पर चले जाओ और दूसरे रास्ते पर न चलो कि वह तुमको ख़ुदा के रास्ते से (भटकाकर) तितिर बितिर कर देगें यह वह बातें हैं जिनका ख़ुदा ने तुमको हक्म दिया है ताकि तुम परहेज़गार बनो
\end{hindi}}
\flushright{\begin{Arabic}
\quranayah[6][154]
\end{Arabic}}
\flushleft{\begin{hindi}
फिर हमनें जो नेक़ी करें उस पर अपनी नेअमत पूरी करने के वास्ते मूसा को क़िताब (तौरौत) अता फरमाई और उसमें हर चीज़ की तफ़सील (बयान कर दी ) थी और (लोगों के लिए अज़सरतापा(सर से पैर तक)) हिदायत व रहमत है ताकि वह लोग अपनें परवरदिगार के सामने हाज़िर होने का यक़ीन करें
\end{hindi}}
\flushright{\begin{Arabic}
\quranayah[6][155]
\end{Arabic}}
\flushleft{\begin{hindi}
और ये किताब (क़ुरान) जिसको हमने (अब नाज़िल किया है क्या है-बरक़त वाली किताब) है तो तुम लोग उसी की पैरवी करो (और ख़ुदा से) डरते रहो ताकि तुम पर रहम किया जाए
\end{hindi}}
\flushright{\begin{Arabic}
\quranayah[6][156]
\end{Arabic}}
\flushleft{\begin{hindi}
(और ऐ मुशरेकीन ये किताब हमने इसलिए नाज़िल की कि तुम कहीं) यह कह बैठो कि हमसे पहले किताब ख़ुदा तो बस सिर्फ दो ही गिरोहों (यहूद व नसारा) पर नाज़िल हुई थी अगरचे हम तो उनके पढ़ने (पढ़ाने) से बेखबर थे
\end{hindi}}
\flushright{\begin{Arabic}
\quranayah[6][157]
\end{Arabic}}
\flushleft{\begin{hindi}
या ये कहने लगो कि अगर हम पर किताबे (ख़ुदा नाज़िल होती तो हम उन लोगों से कहीं बढ़कर राहे रास्त पर होते तो (देखो) अब तो यक़ीनन तुम्हारे परवरदिगार की तरफ से तुम्हारे पास एक रौशन दलील है (किताबे ख़ुदा) और हिदायत और रहमत आ चुकी तो जो शख्स ख़ुदा के आयात को झुठलाए और उससे मुँह फेरे उनसे बढ़ कर ज़ालिम कौन है जो लोग हमारी आयतों से मुँह फेरते हैं हम उनके मुँह फेरने के बदले में अनक़रीब ही बुरे अज़ाब की सज़ा देगें (ऐ रसूल) क्या ये लोग सिर्फ उसके मुन्तिज़र है कि उनके पास फरिश्ते आएं
\end{hindi}}
\flushright{\begin{Arabic}
\quranayah[6][158]
\end{Arabic}}
\flushleft{\begin{hindi}
या तुम्हारा परवरदिगार खुद (तुम्हारे पास) आये या तुम्हारे परवरदिगार की कुछ निशानियाँ आ जाएं (आख़िरकार क्योकर समझाया जाए) हालांकि जिस दिन तुम्हारे परवरदिगार की बाज़ निशानियाँ आ जाएंगी तो जो शख्स पहले से ईमान नहीं लाया होगा या अपने मोमिन होने की हालत में कोई नेक काम नहीं किया होगा तो अब उसका ईमान लाना उसको कुछ भी मुफ़ीद न होगा - (ऐ रसूल) तुम (उनसे) कह दो कि (अच्छा यही सही) तुम (भी) इन्तिज़ार करो हम भी इन्तिज़ार करते हैं
\end{hindi}}
\flushright{\begin{Arabic}
\quranayah[6][159]
\end{Arabic}}
\flushleft{\begin{hindi}
बेशक जिन लोगों ने आपने दीन में तफरक़ा डाला और कई फरीक़ बन गए थे उनसे कुछ सरोकार नहीं उनका मामला तो सिर्फ ख़ुदा के हवाले है फिर जो कुछ वह दुनिया में नेक या बद किया करते थे वह उन्हें बता देगा (उसकी रहमत तो देखो)
\end{hindi}}
\flushright{\begin{Arabic}
\quranayah[6][160]
\end{Arabic}}
\flushleft{\begin{hindi}
जो शख्स नेकी करेगा तो उसको दस गुना सवाब अता होगा और जो शख्स बदी करेगा तो उसकी सज़ा उसको बस उतनी ही दी जाएगी और वह लोग (किसी तरह) सताए न जाएगें
\end{hindi}}
\flushright{\begin{Arabic}
\quranayah[6][161]
\end{Arabic}}
\flushleft{\begin{hindi}
(ऐ रसूल) तुम उनसे कहो कि मुझे तो मेरे परवरदिगार ने सीधी राह यानि एक मज़बूत दीन इबराहीम के मज़हब की हिदायत फरमाई है बातिल से कतरा के चलते थे और मुशरेकीन से न थे
\end{hindi}}
\flushright{\begin{Arabic}
\quranayah[6][162]
\end{Arabic}}
\flushleft{\begin{hindi}
(ऐ रसूल) तुम उन लोगों से कह दो कि मेरी नमाज़ मेरी इबादत मेरा जीना मेरा मरना सब ख़ुदा ही के वास्ते है जो सारे जहाँ का परवरदिगार है
\end{hindi}}
\flushright{\begin{Arabic}
\quranayah[6][163]
\end{Arabic}}
\flushleft{\begin{hindi}
और उसका कोई शरीक़ नहीं और मुझे इसी का हुक्म दिया गया है और मैं सबसे पहले इस्लाम लाने वाला हूँ
\end{hindi}}
\flushright{\begin{Arabic}
\quranayah[6][164]
\end{Arabic}}
\flushleft{\begin{hindi}
(ऐ रसूल) तुम पूछो तो कि क्या मैं ख़ुदा के सिवा किसी और को परवरदिगार तलाश करुँ हालॉकि वह तमाम चीज़ो का मालिक है और जो शख्स कोई बुरा काम करता है उसका (वबाल) उसी पर है और कोई शख्स किसी दूसरे के गुनाह का बोझ नहीं उठाने का फिर तुम सबको अपने परवरदिगार के हुज़ूर में लौट कर जाना है तब तुम लोग जिन बातों में बाहम झगड़ते थे वह सब तुम्हें बता देगा
\end{hindi}}
\flushright{\begin{Arabic}
\quranayah[6][165]
\end{Arabic}}
\flushleft{\begin{hindi}
और वही तो वह (ख़ुदा) है जिसने तुम्हें ज़मीन में (अपना) नायब बनाया और तुममें से बाज़ के बाज़ पर दर्जे बुलन्द किये ताकि वो (नेअमत) तुम्हें दी है उसी पर तुम्हारा इमतेहान करे उसमें तो शक ही नहीं कि तुम्हारा परवरदिगार बहुत जल्द अज़ाब करने वाला है और इसमें भी शक नहीं कि वह बड़ा बख्शने वाला मेहरबान है
\end{hindi}}
\chapter{Al-A'raf (The Elevated Places)}
\begin{Arabic}
\Huge{\centerline{\basmalah}}\end{Arabic}
\flushright{\begin{Arabic}
\quranayah[7][1]
\end{Arabic}}
\flushleft{\begin{hindi}
अलिफ़ लाम मीम स्वाद
\end{hindi}}
\flushright{\begin{Arabic}
\quranayah[7][2]
\end{Arabic}}
\flushleft{\begin{hindi}
(ऐ रसूल) ये किताब ख़ुदा (क़ुरान) तुम पर इस ग़रज़ से नाज़िल की गई है ताकि तुम उसके ज़रिये से लोगों को अज़ाबे ख़ुदा से डराओ और ईमानदारों के लिए नसीहत का बायस हो
\end{hindi}}
\flushright{\begin{Arabic}
\quranayah[7][3]
\end{Arabic}}
\flushleft{\begin{hindi}
तुम्हारे दिल में उसकी वजह से कोई न तंगी पैदा हो (लोगों) जो तुम्हारे परवरदिगार की तरफ से तुम पर नाज़िल किया गया है उसकी पैरवी करो और उसके सिवा दूसरे (फर्ज़ी) बुतों (माबुदों) की पैरवी न करो
\end{hindi}}
\flushright{\begin{Arabic}
\quranayah[7][4]
\end{Arabic}}
\flushleft{\begin{hindi}
तुम लोग बहुत ही कम नसीहत क़ुबूल करते हो और क्या (तुम्हें) ख़बर नहीं कि ऐसी बहुत सी बस्तियाँ हैं जिन्हें हमने हलाक कर डाला तो हमारा अज़ाब (ऐसे वक्त) आ पहुचा
\end{hindi}}
\flushright{\begin{Arabic}
\quranayah[7][5]
\end{Arabic}}
\flushleft{\begin{hindi}
कि वह लोग या तो रात की नींद सो रहे थे या दिन को क़लीला (खाने के बाद का लेटना) कर रहे थे तब हमारा अज़ाब उन पर आ पड़ा तो उनसे सिवाए इसके और कुछ न कहते बन पड़ा कि हम बेशक ज़ालिम थे
\end{hindi}}
\flushright{\begin{Arabic}
\quranayah[7][6]
\end{Arabic}}
\flushleft{\begin{hindi}
फिर हमने तो ज़रूर उन लोगों से जिनकी तरफ पैग़म्बर भेजे गये थे (हर चीज़ का) सवाल करेगें और ख़ुद पैग़म्बरों से भी ज़रूर पूछेगें
\end{hindi}}
\flushright{\begin{Arabic}
\quranayah[7][7]
\end{Arabic}}
\flushleft{\begin{hindi}
फिर हम उनसे हक़ीक़त हाल ख़ूब समझ बूझ के (ज़रा ज़रा) दोहराएगें
\end{hindi}}
\flushright{\begin{Arabic}
\quranayah[7][8]
\end{Arabic}}
\flushleft{\begin{hindi}
और हम कुछ ग़ायब तो थे नहीं और उस दिन (आमाल का) तौला जाना बिल्कुल ठीक है फिर तो जिनके (नेक अमाल के) पल्ले भारी होगें तो वही लोग फायज़ुलहराम (नजात पाये हुए) होगें
\end{hindi}}
\flushright{\begin{Arabic}
\quranayah[7][9]
\end{Arabic}}
\flushleft{\begin{hindi}
(और जिनके नेक अमाल के) पल्ले हलके होगें तो उन्हीं लोगों ने हमारी आयत से नाफरमानी करने की वजह से यक़ीनन अपना आप नुक़सान किया
\end{hindi}}
\flushright{\begin{Arabic}
\quranayah[7][10]
\end{Arabic}}
\flushleft{\begin{hindi}
और (ऐ बनीआदम) हमने तो यक़ीनन तुमको ज़मीन में क़ुदरत व इख़तेदार दिया और उसमें तुम्हारे लिए असबाब ज़िन्दगी मुहय्या किए (मगर) तुम बहुत ही कम शुक्र करते हो
\end{hindi}}
\flushright{\begin{Arabic}
\quranayah[7][11]
\end{Arabic}}
\flushleft{\begin{hindi}
हालाकि इसमें तो शक ही नहीं कि हमने तुम्हारे बाप आदम को पैदा किया फिर तुम्हारी सूरते बनायीं फिर हमनें फ़रिश्तों से कहा कि तुम सब के सब आदम को सजदा करो तो सब के सब झुक पड़े मगर शैतान कि वह सजदा करने वालों में शामिल न हुआ।
\end{hindi}}
\flushright{\begin{Arabic}
\quranayah[7][12]
\end{Arabic}}
\flushleft{\begin{hindi}
ख़ुदा ने (शैतान से) फरमाया जब मैनें तुझे हुक्म दिया कि तू फिर तुझे सजदा करने से किसी ने रोका कहने लगा मैं उससे अफ़ज़ल हूँ (क्योंकि) तूने मुझे आग (ऐसे लतीफ अनसर) से पैदा किया
\end{hindi}}
\flushright{\begin{Arabic}
\quranayah[7][13]
\end{Arabic}}
\flushleft{\begin{hindi}
और उसको मिट्टी (ऐसी कशीफ़ अनसर) से पैदा किया ख़ुदा ने फरमाया (तुझको ये ग़ुरूर है) तो बेहश्त से नीचे उतर जाओ क्योंकि तेरी ये मजाल नहीं कि तू यहाँ रहकर ग़ुरूर करे तो यहाँ से (बाहर) निकल बेशक तू ज़लील लोगों से है
\end{hindi}}
\flushright{\begin{Arabic}
\quranayah[7][14]
\end{Arabic}}
\flushleft{\begin{hindi}
कहने लगा तो (ख़ैर) हमें उस दिन तक की (मौत से) मोहलत दे
\end{hindi}}
\flushright{\begin{Arabic}
\quranayah[7][15]
\end{Arabic}}
\flushleft{\begin{hindi}
जिस दिन सारी ख़ुदाई के लोग दुबारा जिलाकर उठा खड़े किये जाएगें
\end{hindi}}
\flushright{\begin{Arabic}
\quranayah[7][16]
\end{Arabic}}
\flushleft{\begin{hindi}
फ़रमाया (अच्छा मंजूर) तुझे ज़रूर मोहलत दी गयी कहने लगा चूँकि तूने मेरी राह मारी तो मैं भी तेरी सीधी राह पर बनी आदम को (गुमराह करने के लिए) ताक में बैठूं तो सही
\end{hindi}}
\flushright{\begin{Arabic}
\quranayah[7][17]
\end{Arabic}}
\flushleft{\begin{hindi}
फिर उन लोगों से और उनके पीछे से और उनके दाहिने से और उनके बाएं से (गरज़ हर तरफ से) उन पर आ पडूंगा और (उनको बहकाउंगा) और तू उन में से बहुतरों की शुक्रग़ुज़ार नहीं पायेगा
\end{hindi}}
\flushright{\begin{Arabic}
\quranayah[7][18]
\end{Arabic}}
\flushleft{\begin{hindi}
ख़ुदा ने फरमाया यहाँ से बुरे हाल में (राइन्दा होकर निकल) (दूर) जा उन लोगों से जो तेरा कहा मानेगा तो मैं यक़ीनन तुम (और उन) सबको जहन्नुम में भर दूंगा
\end{hindi}}
\flushright{\begin{Arabic}
\quranayah[7][19]
\end{Arabic}}
\flushleft{\begin{hindi}
और (आदम से कहा) ऐ आदम तुम और तुम्हारी बीबी (दोनों) बेहश्त में रहा सहा करो और जहाँ से चाहो खाओ (पियो) मगर (ख़बरदार) उस दरख्त के करीब न जाना वरना तुम अपना आप नुक़सान करोगे
\end{hindi}}
\flushright{\begin{Arabic}
\quranayah[7][20]
\end{Arabic}}
\flushleft{\begin{hindi}
फिर शैतान ने उन दोनों को वसवसा (शक) दिलाया ताकि (नाफरमानी की वजह से) उनके अस्तर की चीज़े जो उनकी नज़र से बेहश्ती लिबास की वजह से पोशीदा थी खोल डाले कहने लगा कि तुम्हारे परवरदिगार ने दोनों को दरख्त (के फल खाने) से सिर्फ इसलिए मना किया है (कि मुबादा) तुम दोनों फरिश्ते बन जाओ या हमेशा (ज़िन्दा) रह जाओ
\end{hindi}}
\flushright{\begin{Arabic}
\quranayah[7][21]
\end{Arabic}}
\flushleft{\begin{hindi}
और उन दोनों के सामने क़समें खायीं कि मैं यक़ीनन तुम्हारा ख़ैर ख्वाह हूँ
\end{hindi}}
\flushright{\begin{Arabic}
\quranayah[7][22]
\end{Arabic}}
\flushleft{\begin{hindi}
ग़रज़ धोखे से उन दोनों को उस (के खाने) की तरफ ले गया ग़रज़ जो ही उन दोनों ने इस दरख्त (के फल) को चखा कि (बेहश्ती लिबास गिर गया और समझ पैदा हुई) उन पर उनकी शर्मगाहें ज़ाहिर हो गयीं और बेहश्त के पत्ते (तोड़ जोड़ कर) अपने ऊपर ढापने लगे तब उनको परवरदिगार ने उनको आवाज़ दी कि क्यों मैंने तुम दोनों को इस दरख्त के पास (जाने) से मना नहीं किया था और (क्या) ये न जता दिया था कि शैतान तुम्हारा यक़ीनन खुला हुआ दुश्मन है
\end{hindi}}
\flushright{\begin{Arabic}
\quranayah[7][23]
\end{Arabic}}
\flushleft{\begin{hindi}
ये दोनों अर्ज क़रने लगे ऐ हमारे पालने वाले हमने अपना आप नुकसान किया और अगर तू हमें माफ न फरमाएगा और हम पर रहम न करेगा तो हम बिल्कुल घाटे में ही रहेगें
\end{hindi}}
\flushright{\begin{Arabic}
\quranayah[7][24]
\end{Arabic}}
\flushleft{\begin{hindi}
हुक्म हुआ तुम (मियां बीबी शैतान) सब के सब बेहशत से नीचे उतरो तुममें से एक का एक दुश्मन है और (एक ख़ास) वक्त तक तुम्हारा ज़मीन में ठहराव (ठिकाना) और ज़िन्दगी का सामना है
\end{hindi}}
\flushright{\begin{Arabic}
\quranayah[7][25]
\end{Arabic}}
\flushleft{\begin{hindi}
ख़ुदा ने (ये भी) फरमाया कि तुम ज़मीन ही में जिन्दगी बसर करोगे और इसी में मरोगे
\end{hindi}}
\flushright{\begin{Arabic}
\quranayah[7][26]
\end{Arabic}}
\flushleft{\begin{hindi}
और उसी में से (और) उसी में से फिर दोबारा तुम ज़िन्दा करके निकाले जाओगे ऐ आदम की औलाद हमने तुम्हारे लिए पोशाक नाज़िल की जो तुम्हारे शर्मगाहों को छिपाती है और ज़ीनत के लिए कपड़े और इसके अलावा परहेज़गारी का लिबास है और ये सब (लिबासों) से बेहतर है ये (लिबास) भी ख़ुदा (की कुदरत) की निशानियों से है
\end{hindi}}
\flushright{\begin{Arabic}
\quranayah[7][27]
\end{Arabic}}
\flushleft{\begin{hindi}
ताकि लोग नसीहत व इबरत हासिल करें ऐ औलादे आदम (होशियार रहो) कहीं तुम्हें शैतान बहका न दे जिस तरह उसने तुम्हारे बाप माँ आदम व हव्वा को बेहश्त से निकलवा छोड़ा उसी ने उन दोनों से (बेहश्ती) पोशाक उतरवाई ताकि उन दोनों को उनकी शर्मगाहें दिखा दे वह और उसका क़ुनबा ज़रूर तुम्हें इस तरह देखता रहता है कि तुम उन्हे नहीं देखने पाते हमने शैतानों को उन्हीं लोगों का रफीक़ क़रार दिया है
\end{hindi}}
\flushright{\begin{Arabic}
\quranayah[7][28]
\end{Arabic}}
\flushleft{\begin{hindi}
जो ईमान नही रखते और वह लोग जब कोई बुरा काम करते हैं कि हमने उस तरीके पर अपने बाप दादाओं को पाया और ख़ुदा ने (भी) यही हुक्म दिया है (ऐ रसूल) तुम साफ कह दो कि ख़ुदा ने (भी) यही हुक्म दिया है (ऐ रसूल) तुम (साफ) कह दो कि ख़ुदा हरगिज़ बुरे काम का हुक्म नहीं देता क्या तुम लोग ख़ुदा पर (इफ्तिरा करके) वह बातें कहते हो जो तुम नहीं जानते
\end{hindi}}
\flushright{\begin{Arabic}
\quranayah[7][29]
\end{Arabic}}
\flushleft{\begin{hindi}
(ऐ रसूल) तुम कह दो कि मेरे परवरदिगार ने तो इन्साफ का हुक्म दिया है और (ये भी क़रार दिया है कि) हर नमाज़ के वक्त अपने अपने मुँह (क़िबले की तरफ़) सीधे कर लिया करो और इसके लिए निरी खरी इबादत करके उससे दुआ मांगो जिस तरह उसने तुम्हें शुरू शुरू पैदा किया था
\end{hindi}}
\flushright{\begin{Arabic}
\quranayah[7][30]
\end{Arabic}}
\flushleft{\begin{hindi}
उसी तरह फिर (दोबारा) ज़िन्दा किये जाओगे उसी ने एक फरीक़ की हिदायत की और एक गिरोह (के सर) पर गुमराही सवार हो गई उन लोगों ने ख़ुदा को छोड़कर शैतानों को अपना सरपरस्त बना लिया और बावजूद उसके गुमराह करते हैं कि वह राह रास्ते पर है
\end{hindi}}
\flushright{\begin{Arabic}
\quranayah[7][31]
\end{Arabic}}
\flushleft{\begin{hindi}
ऐ औलाद आदम हर नमाज़ के वक्त बन सवर के निखर जाया करो और खाओ और पियो और फिज़ूल ख़र्ची मत करो (क्योंकि) ख़ुदा फिज़ूल ख़र्च करने वालों को दोस्त नहीं रखता
\end{hindi}}
\flushright{\begin{Arabic}
\quranayah[7][32]
\end{Arabic}}
\flushleft{\begin{hindi}
(ऐ रसूल से) पूछो तो कि जो ज़ीनत (के साज़ों सामान) और खाने की साफ सुथरी चीज़ें ख़ुदा ने अपने बन्दो के वास्ते पैदा की हैं किसने हराम कर दी तुम ख़ुद कह दो कि सब पाक़ीज़ा चीज़े क़यामत के दिन उन लोगों के लिए ख़ास हैं जो दुनिया की (ज़रा सी) ज़िन्दगी में ईमान लाते थे हम यूँ अपनी आयतें समझदार लोगों के वास्ते तफसीलदार बयान करतें हैं
\end{hindi}}
\flushright{\begin{Arabic}
\quranayah[7][33]
\end{Arabic}}
\flushleft{\begin{hindi}
(ऐ रसूल) तुम साफ कह दो कि हमारे परवरदिगार ने तो तमाम बदकारियों को ख्वाह (चाहे) ज़ाहिरी हो या बातिनी और गुनाह और नाहक़ ज्यादती करने को हराम किया है और इस बात को कि तुम किसी को ख़ुदा का शरीक बनाओ जिनकी उनसे कोई दलील न ही नाज़िल फरमाई और ये भी कि बे समझे बूझे ख़ुदा पर बोहतान बॉधों
\end{hindi}}
\flushright{\begin{Arabic}
\quranayah[7][34]
\end{Arabic}}
\flushleft{\begin{hindi}
और हर गिरोह (के न पैदा होने) का एक ख़ास वक्त है फिर जब उनका वक्त आ पहुंचता है तो न एक घड़ी पीछे रह सकते हैं और न आगे बढ़ सकते हैं
\end{hindi}}
\flushright{\begin{Arabic}
\quranayah[7][35]
\end{Arabic}}
\flushleft{\begin{hindi}
ऐ औलादे आदम जब तुम में के (हमारे) पैग़म्बर तुम्हारे पास आए और तुमसे हमारे एहकाम बयान करे तो (उनकी इताअत करना क्योंकि जो शख्स परहेज़गारी और नेक काम करेगा तो ऐसे लोगों पर न तो (क़यामत में) कोई ख़ौफ़ होगा और न वह आर्ज़दा ख़ातिर (परेशान) होंगे
\end{hindi}}
\flushright{\begin{Arabic}
\quranayah[7][36]
\end{Arabic}}
\flushleft{\begin{hindi}
और जिन लोगों ने हमारी आयतों को झुठलाया और उनसे सरताबी कर बैठे वह लोग जहन्नुमी हैं कि वह उसमें हमेशा रहेगें
\end{hindi}}
\flushright{\begin{Arabic}
\quranayah[7][37]
\end{Arabic}}
\flushleft{\begin{hindi}
तो जो शख्स ख़ुदा पर झूठ बोहतान बॉधे या उसकी आयतों को झुठलाए उससे बढ़कर ज़ालिम और कौन होगा फिर तो वह लोग हैं जिन्हें उनकी (तक़दीर) का लिखा हिस्सा (रिज़क) वग़ैरह मिलता रहेगा यहाँ तक कि जब हमारे भेजे हुए (फरिश्ते) उनके पास आकर उनकी रूह कब्ज़ करेगें तो (उनसे) पूछेगें कि जिन्हें तुम ख़ुदा को छोड़कर पुकारा करते थे अब वह (कहाँ हैं तो वह कुफ्फार) जवाब देगें कि वह सब तो हमें छोड़ कर चल चंपत हुए और अपने खिलाफ आप गवाही देगें कि वह बेशक काफ़िर थे
\end{hindi}}
\flushright{\begin{Arabic}
\quranayah[7][38]
\end{Arabic}}
\flushleft{\begin{hindi}
(तब ख़ुदा उनसे) फरमाएगा कि जो लोग जिन व इन्स के तुम से पहले बसे हैं उन्हीं में मिलजुल कर तुम भी जहन्नुम वासिल हो जाओ (और ) अहले जहन्नुम का ये हाल होगा कि जब उसमें एक गिरोह दाख़िल होगा तो अपने साथी दूसरे गिरोह पर लानत करेगा यहाँ तक कि जब सब के सब पहुंच जाएगें तो उनमें की पिछली जमात अपने से पहली जमाअत के वास्ते बदद्आ करेगी कि परवरदिगार उन्हीं लोगों ने हमें गुमराह किया था तो उन पर जहन्नुम का दोगुना अज़ाब फरमा (इस पर) ख़ुदा फरमाएगा कि हर एक के वास्ते दो गुना अज़ाब है लेकिन (तुम पर) तुफ़ है तुम जानते नहीं
\end{hindi}}
\flushright{\begin{Arabic}
\quranayah[7][39]
\end{Arabic}}
\flushleft{\begin{hindi}
और उनमें से पहली जमाअत पिछली जमाअत की तरफ मुख़ातिब होकर कहेगी कि अब तो तुमको हमपर कोई फज़ीलत न रही पस (हमारी तरह) तुम भी अपने करतूत की बदौलत अज़ाब (के मज़े) चखो बेशक जिन लोगों ने हमारे आयात को झुठलाया
\end{hindi}}
\flushright{\begin{Arabic}
\quranayah[7][40]
\end{Arabic}}
\flushleft{\begin{hindi}
और उनसे सरताबी की न उनके लिए आसमान के दरवाज़े खोले जाएंगें और वह बेहश्त ही में दाखिल होने पाएगें यहाँ तक कि ऊँट सूई के नाके में होकर निकल जाए (यानि जिस तरह ये मुहाल है) उसी तरह उनका बेहश्त में दाखिल होना मुहाल है और हम मुजरिमों को ऐसी ही सज़ा दिया करते हैं उनके लिए जहन्नुम (की आग) का बिछौना होगा
\end{hindi}}
\flushright{\begin{Arabic}
\quranayah[7][41]
\end{Arabic}}
\flushleft{\begin{hindi}
और उनके ऊपर से (आग ही का) ओढ़ना भी और हम ज़ालिमों को ऐसी ही सज़ा देते हैं और जिन लोगों ने ईमान कुबुल किया
\end{hindi}}
\flushright{\begin{Arabic}
\quranayah[7][42]
\end{Arabic}}
\flushleft{\begin{hindi}
और अच्छे अच्छे काम किये और हम तो किसी शख्स को उसकी ताकत से ज्यादा तकलीफ देते ही नहीं यहीं लोग जन्नती हैं कि वह हमेशा जन्नत ही में रहा (सहा) करेगें
\end{hindi}}
\flushright{\begin{Arabic}
\quranayah[7][43]
\end{Arabic}}
\flushleft{\begin{hindi}
और उन लोगों के दिल में जो कुछ (बुग़ज़ व कीना) होगा वह सब हम निकाल (बाहर कर) देगें उनके महलों के नीचे नहरें जारी होगीं और कहते होगें शुक्र है उस ख़ुदा का जिसने हमें इस (मंज़िले मक़सूद) तक पहुंचाया और अगर ख़ुदा हमें यहाँ न पहुंचाता तो हम किसी तरह यहाँ न पहुंच सकते बेशक हमारे परवरदिगार के पैग़म्बर दीने हक़ लेकर आये थे और उन लोगों से पुकार कर कह दिया जाएगा कि वह बेहिश्त हैं जिसके तुम अपनी कारग़ुज़ारियों की जज़ा में वारिस व मालिक बनाए गये हों
\end{hindi}}
\flushright{\begin{Arabic}
\quranayah[7][44]
\end{Arabic}}
\flushleft{\begin{hindi}
और जन्नती लोग जहन्नुमी वालों से पुकार कर कहेगें हमने तो बेशक जो हमारे परवरदिगार ने हमसे वायदा किया था ठीक ठीक पा लिया तो क्या तुमने भी जो तुमसे तम्हारे परवरदिगार ने वायदा किया था ठीक पाया (या नहीं) अहले जहन्नुम कहेगें हाँ (पाया) एक मुनादी उनके दरमियान निदा करेगा कि ज़ालिमों पर ख़ुदा की लानत है
\end{hindi}}
\flushright{\begin{Arabic}
\quranayah[7][45]
\end{Arabic}}
\flushleft{\begin{hindi}
जो ख़ुदा की राह से लोगों को रोकते थे और उसमें (ख्वामख्वाह) कज़ी (टेढ़ापन) करना चाहते थे और वह रोज़े आख़ेरत से इन्कार करते थे
\end{hindi}}
\flushright{\begin{Arabic}
\quranayah[7][46]
\end{Arabic}}
\flushleft{\begin{hindi}
और बेहश्त व दोज़ख के दरमियान एक हद फ़ासिल है और कुछ लोग आराफ़ पर होगें जो हर शख्स को (बेहिश्ती हो या जहन्नुमी) उनकी पेशानी से पहचान लेगें और वह जन्नत वालों को आवाज़ देगें कि तुम पर सलाम हो या (आराफ़ वाले) लोग अभी दाख़िले जन्नत नहीं हुए हैं मगर वह तमन्ना ज़रूर रखते हैं
\end{hindi}}
\flushright{\begin{Arabic}
\quranayah[7][47]
\end{Arabic}}
\flushleft{\begin{hindi}
और जब उनकी निगाहें पलटकर जहन्नुमी लोगों की तरफ जा पड़ेगीं (तो उनकी ख़राब हालत देखकर ख़ुदा से अर्ज़ करेगें) ऐ हमारे परवरदिगार हमें ज़ालिम लोगों का साथी न बनाना
\end{hindi}}
\flushright{\begin{Arabic}
\quranayah[7][48]
\end{Arabic}}
\flushleft{\begin{hindi}
और आराफ वाले कुछ (जहन्नुमी) लोगों को जिन्हें उनका चेहरा देखकर पहचान लेगें आवाज़ देगें और और कहेगें अब न तो तुम्हारा जत्था ही तुम्हारे काम आया और न तुम्हारी शेखी बाज़ी ही (सूद मन्द हुई)
\end{hindi}}
\flushright{\begin{Arabic}
\quranayah[7][49]
\end{Arabic}}
\flushleft{\begin{hindi}
जो तुम दुनिया में किया करते थे यही लोग वह हैं जिनकी निस्बत तुम कसमें खाया करते थे कि उन पर ख़ुदा (अपनी) रहमत न करेगा (देखो आज वही लोग हैं जिनसे कहा गया कि बेतकल्लुफ) बेहश्त में चलो जाओ न तुम पर कोई खौफ है और न तुम किसी तरह आर्ज़ुदा ख़ातिर परेशानी होगी
\end{hindi}}
\flushright{\begin{Arabic}
\quranayah[7][50]
\end{Arabic}}
\flushleft{\begin{hindi}
और दोज़ख वाले अहले बेहिश्त को (लजाजत से) आवाज़ देगें कि हम पर थोड़ा सा पानी ही उंडेल दो या जो (नेअमतों) खुदा ने तुम्हें दी है उसमें से कुछ (दे डालो दो तो अहले बेहिश्त जवाब में) कहेंगें कि ख़ुदा ने तो जन्नत का खाना पानी काफिरों पर कतई हराम कर दिया है
\end{hindi}}
\flushright{\begin{Arabic}
\quranayah[7][51]
\end{Arabic}}
\flushleft{\begin{hindi}
जिन लोगों ने अपने दीन को खेल तमाशा बना लिया था और दुनिया की (चन्द रोज़ा) ज़िन्दगी ने उनको फरेब दिया था तो हम भी आज (क़यामत में) उन्हें (क़सदन) भूल जाएगें
\end{hindi}}
\flushright{\begin{Arabic}
\quranayah[7][52]
\end{Arabic}}
\flushleft{\begin{hindi}
जिस तरह यह लोग (हमारी) आज की हुज़ूरी को भूलें बैठे थे और हमारी आयतों से इन्कार करते थे हालांकि हमने उनके पास (रसूल की मारफत किताब भी भेज दी है)
\end{hindi}}
\flushright{\begin{Arabic}
\quranayah[7][53]
\end{Arabic}}
\flushleft{\begin{hindi}
जिसे हर तरह समझ बूझ के तफसीलदार बयान कर दिया है (और वह) ईमानदार लोगों के लिए हिदायत और रहमत है क्या ये लोग बस सिर्फ अन्जाम (क़यामत ही) के मुन्तज़िर है (हालांकि) जिस दिन उसके अन्जाम का (वक्त) आ जाएगा तो जो लोग उसके पहले भूले बैठे थे (बेसाख्ता) बोल उठेगें कि बेशक हमारे परवरदिगार के सब रसूल हक़ लेकर आये थे तो क्या उस वक्त हमारी भी सिफरिश करने वाले हैं जो हमारी सिफारिश करें या हम फिर (दुनिया में) लौटाएं जाएं तो जो जो काम हम करते थे उसको छोड़कर दूसरें काम करें
\end{hindi}}
\flushright{\begin{Arabic}
\quranayah[7][54]
\end{Arabic}}
\flushleft{\begin{hindi}
बेशक उन लोगों ने अपना सख्त घाटा किया और जो इफ़तेरा परदाज़िया किया करते थे वह सब गायब (ग़ल्ला) हो गयीं बेशक तुम्हारा परवरदिगार ख़ुदा ही है जिसके (सिर्फ) 6 दिनों में आसमान और ज़मीन को पैदा किया फिर अर्श के बनाने पर आमादा हुआ वही रात को दिन का लिबास पहनाता है तो (गोया) रात दिन को पीछे पीछे तेज़ी से ढूंढती फिरती है और उसी ने आफ़ताब और माहताब और सितारों को पैदा किया कि ये सब के सब उसी के हुक्म के ताबेदार हैं
\end{hindi}}
\flushright{\begin{Arabic}
\quranayah[7][55]
\end{Arabic}}
\flushleft{\begin{hindi}
देखो हुकूमत और पैदा करना बस ख़ास उसी के लिए है वह ख़ुदा जो सारे जहाँन का परवरदिगार बरक़त वाला है
\end{hindi}}
\flushright{\begin{Arabic}
\quranayah[7][56]
\end{Arabic}}
\flushleft{\begin{hindi}
(लोगों) अपने परवरदिगार से गिड़गिड़ाकर और चुपके - चुपके दुआ करो, वह हद से तजाविज़ करने वालों को हरगिज़ दोस्त नहीं रखता और ज़मीन में असलाह के बाद फसाद न करते फिरो और (अज़ाब) के ख़ौफ से और (रहमत) की आस लगा के ख़ुदा से दुआ मांगो
\end{hindi}}
\flushright{\begin{Arabic}
\quranayah[7][57]
\end{Arabic}}
\flushleft{\begin{hindi}
(क्योंकि) नेकी करने वालों से ख़ुदा की रहमत यक़ीनन क़रीब है और वही तो (वह) ख़ुदा है जो अपनी रहमत (अब्र) से पहले खुशखबरी देने वाली हवाओ को भेजता है यहाँ तक कि जब हवाएं (पानी से भरे) बोझल बादलों के ले उड़े तो हम उनको किसी शहर की की तरफ (जो पानी का नायाबी (कमी) से गोया) मर चुका था हॅका दिया फिर हमने उससे पानी बरसाया, फिर हमने उससे हर तरह के फल ज़मीन से निकाले
\end{hindi}}
\flushright{\begin{Arabic}
\quranayah[7][58]
\end{Arabic}}
\flushleft{\begin{hindi}
हम यूं ही (क़यामत के दिन ज़मीन से) मुर्दों को निकालेंगें ताकि तुम लोग नसीहत व इबरत हासिल करो और उम्दा ज़मीन उसके परवरदिगार के हुक्म से उस सब्ज़ा (अच्छा ही) है और जो ज़मीन बड़ी है उसकी पैदावार ख़राब ही होती है
\end{hindi}}
\flushright{\begin{Arabic}
\quranayah[7][59]
\end{Arabic}}
\flushleft{\begin{hindi}
हम यू अपनी आयतों को उलेटफेर कर शुक्रग़ुजार लोगों के वास्ते बयान करते हैं बेशक हमने नूह को उनकी क़ौम के पास (रसूल बनाकर) भेजा तो उन्होनें (लोगों से ) कहाकि ऐ मेरी क़ौम ख़ुदा की ही इबादत करो उसके सिवा तुम्हारा कोई माबूद नहीं है और मैं तुम्हारी निस्बत (क़यामत जैसे) बड़े ख़ौफनाक दिन के अज़ाब से डरता हूँ
\end{hindi}}
\flushright{\begin{Arabic}
\quranayah[7][60]
\end{Arabic}}
\flushleft{\begin{hindi}
तो उनकी क़ौम के चन्द सरदारों ने कहा हम तो यक़ीनन देखते हैं कि तुम खुल्लम खुल्ला गुमराही में (पड़े) हो
\end{hindi}}
\flushright{\begin{Arabic}
\quranayah[7][61]
\end{Arabic}}
\flushleft{\begin{hindi}
तब नूह ने कहा कि ऐ मेरी क़ौम मुझ में गुमराही (वग़ैरह) तो कुछ नहीं बल्कि मैं तो परवरदिगारे आलम की तरफ से रसूल हूँ
\end{hindi}}
\flushright{\begin{Arabic}
\quranayah[7][62]
\end{Arabic}}
\flushleft{\begin{hindi}
तुम तक अपने परवरदिगार के पैग़ामात पहुचाएं देता हूँ और तुम्हारे लिए तुम्हारी ख़ैर ख्वाही करता हूँ और ख़ुदा की तरफ से जो बातें मै जानता हूँ तुम नहीं जानते
\end{hindi}}
\flushright{\begin{Arabic}
\quranayah[7][63]
\end{Arabic}}
\flushleft{\begin{hindi}
क्या तुम्हें उस बात पर ताअज्जुब है कि तुम्हारे पास तुम्ही में से एक मर्द (आदमी) के ज़रिए से तुम्हारे परवरदिगार का ज़िक्र (हुक्म) आया है ताकि वह तुम्हें (अज़ाब से) डराए और ताकि तुम परहेज़गार बनों और ताकि तुम पर रहम किया जाए
\end{hindi}}
\flushright{\begin{Arabic}
\quranayah[7][64]
\end{Arabic}}
\flushleft{\begin{hindi}
इस पर भी लोगों ने उनकों झुठला दिया तब हमने उनको और जो लोग उनके साथ कश्ती में थे बचा लिया और बाक़ी जितने लोगों ने हमारी आयतों को झुठलाया था सबको डुबो मारा ये सब के सब यक़ीनन अन्धे लोग थे
\end{hindi}}
\flushright{\begin{Arabic}
\quranayah[7][65]
\end{Arabic}}
\flushleft{\begin{hindi}
और (हमने) क़ौम आद की तरफ उनके भाई हूद को (रसूल बनाकर भेजा) तो उन्होनें लोगों से कहा ऐ मेरी क़ौम ख़ुदा ही की इबादत करो उसके सिवा तुम्हारा कोई माबूद नहीं तो क्या तुम (ख़ुदा से) डरते नहीं हो
\end{hindi}}
\flushright{\begin{Arabic}
\quranayah[7][66]
\end{Arabic}}
\flushleft{\begin{hindi}
(तो) उनकी क़ौम के चन्द सरदार जो काफिर थे कहने लगे हम तो बेशक तुमको हिमाक़त में (मुब्तिला) देखते हैं और हम यक़ीनी तुम को झूठा समझते हैं
\end{hindi}}
\flushright{\begin{Arabic}
\quranayah[7][67]
\end{Arabic}}
\flushleft{\begin{hindi}
हूद ने कहा ऐ मेरी क़ौम मुझमें में तो हिमाक़त की कोई बात नहीं बल्कि मैं तो परवरदिगार आलम का रसूल हूँ
\end{hindi}}
\flushright{\begin{Arabic}
\quranayah[7][68]
\end{Arabic}}
\flushleft{\begin{hindi}
मैं तुम्हारे पास तुम्हारे परवरदिगार के पैग़ामात पहँचाए देता हूँ और मैं तुम्हारा सच्चा ख़ैरख्वाह हूँ
\end{hindi}}
\flushright{\begin{Arabic}
\quranayah[7][69]
\end{Arabic}}
\flushleft{\begin{hindi}
क्या तुम्हें इस पर ताअज्जुब है कि तुम्हारे परवरदिगार का हुक्म तुम्हारे पास तुम्ही में एक मर्द (आदमी) के ज़रिए से (आया) कि तुम्हें (अजाब से) डराए और (वह वक्त) याद करो जब उसने तुमको क़ौम नूह के बाद ख़लीफा (व जानशीन) बनाया और तुम्हारी ख़िलाफ़त में भी बहुत ज्यादती कर दी तो ख़ुदा की नेअमतों को याद करो ताकि तुम दिली मुरादे पाओ
\end{hindi}}
\flushright{\begin{Arabic}
\quranayah[7][70]
\end{Arabic}}
\flushleft{\begin{hindi}
तो वह लोग कहने लगे क्या तुम हमारे पास इसलिए आए हो कि सिर्फ ख़ुदा की तो इबादत करें और जिनको हमारे बाप दादा पूजते चले आए छोड़ बैठें पस अगर तुम सच्चे हो तो जिससे तुम हमको डराते हो हमारे पास लाओ
\end{hindi}}
\flushright{\begin{Arabic}
\quranayah[7][71]
\end{Arabic}}
\flushleft{\begin{hindi}
हूद ने जवाब दिया (कि बस समझ लो) कि तुम्हारे परवरदिगार की तरफ से तुम पर अज़ाब और ग़ज़ब नाज़िल हो चुका क्या तुम मुझसे चन्द (बुतो के फर्ज़ी) नामों के बारे में झगड़ते हो जिनको तुमने और तुम्हारे बाप दादाओं ने (ख्वाहमख्वाह) गढ़ लिए हैं हालाकि ख़ुदा ने उनके लिए कोई सनद नहीं नाज़िल की पस तुम ( अज़ाबे ख़ुदा का) इन्तज़ार करो मैं भी तुम्हारे साथ मुन्तिज़र हूँ
\end{hindi}}
\flushright{\begin{Arabic}
\quranayah[7][72]
\end{Arabic}}
\flushleft{\begin{hindi}
आख़िर हमने उनको और जो लोग उनके साथ थे उनको अपनी रहमत से नजात दी और जिन लोगों ने हमारी आयतों को झुठलाया था हमने उनकी जड़ काट दी और वह लोग ईमान लाने वाले थे भी नहीं
\end{hindi}}
\flushright{\begin{Arabic}
\quranayah[7][73]
\end{Arabic}}
\flushleft{\begin{hindi}
और (हमने क़ौम) समूद की तरफ उनके भाई सालेह को रसूल बनाकर भेजा तो उन्होनें (उन लोगों से कहा) ऐ मेरी क़ौम ख़ुदा ही की इबादत करो और उसके सिवा कोई तुम्हारा माबूद नहीं है तुम्हारे पास तो तुम्हारे परवरदिगार की तरफ से वाज़ेए और रौशन दलील आ चुकी है ये ख़ुदा की भेजी हुई ऊँटनी तुम्हारे वास्ते एक मौजिज़ा है तो तुम लोग उसको छोड़ दो कि ख़ुदा की ज़मीन में जहाँ चाहे चरती फिरे और उसे कोई तकलीफ़ ना पहुंचाओ वरना तुम दर्दनाक अज़ाब में गिरफ्तार हो जाआगे
\end{hindi}}
\flushright{\begin{Arabic}
\quranayah[7][74]
\end{Arabic}}
\flushleft{\begin{hindi}
और वह वक्त याद करो जब उसने तुमको क़ौम आद के बाद (ज़मीन में) ख़लीफा (व जानशीन) बनाया और तुम्हें ज़मीन में इस तरह बसाया कि तुम हमवार व नरम ज़मीन में (बड़े-बड़े) महल उठाते हो और पहाड़ों को तराश के घर बनाते हो तो ख़ुदा की नेअमतों को याद करो और रूए ज़मीन में फसाद न करते फिरो
\end{hindi}}
\flushright{\begin{Arabic}
\quranayah[7][75]
\end{Arabic}}
\flushleft{\begin{hindi}
तो उसकी क़ौम के बड़े बड़े लोगों ने बेचारें ग़रीबों से उनमें से जो ईमान लाए थे कहा क्या तुम्हें मालूम है कि सालेह (हक़ीकतन) अपने परवरदिगार के सच्चे रसूल हैं - उन बेचारों ने जवाब दिया कि जिन बातों का वह पैग़ाम लाए हैं हमारा तो उस पर ईमान है
\end{hindi}}
\flushright{\begin{Arabic}
\quranayah[7][76]
\end{Arabic}}
\flushleft{\begin{hindi}
तब जिन लोगों को (अपनी दौलत दुनिया पर) घमन्ड था कहने लगे हम तो जिस पर तुम ईमान लाए हो उसे नहीं मानते
\end{hindi}}
\flushright{\begin{Arabic}
\quranayah[7][77]
\end{Arabic}}
\flushleft{\begin{hindi}
ग़रज़ उन लोगों ने ऊँटनी के कूचें और पैर काट डाले और अपने परवरदिगार के हुक्म से सरताबी की और (बेबाकी से) कहने लगे अगर तुम सच्चे रसूल हो तो जिस (अज़ाब) से हम लोगों को डराते थे अब लाओ
\end{hindi}}
\flushright{\begin{Arabic}
\quranayah[7][78]
\end{Arabic}}
\flushleft{\begin{hindi}
तब उन्हें ज़लज़ले ने ले डाला और वह लोग ज़ानू पर सर किए (जिस तरह) बैठे थे बैठे के बैठे रह गए
\end{hindi}}
\flushright{\begin{Arabic}
\quranayah[7][79]
\end{Arabic}}
\flushleft{\begin{hindi}
उसके बाद सालेह उनसे टल गए और (उनसे मुख़ातिब होकर) कहा मेरी क़ौम (आह) मैनें तो अपने परवरदिगार के पैग़ाम तुम तक पहुचा दिए थे और तुम्हारे ख़ैरख्वाही की थी (और ऊँच नीच समझा दिया था) मगर अफसोस तुम (ख़ैरख्वाह) समझाने वालों को अपना दोस्त ही नहीं समझते
\end{hindi}}
\flushright{\begin{Arabic}
\quranayah[7][80]
\end{Arabic}}
\flushleft{\begin{hindi}
और (लूत को हमने रसूल बनाकर भेजा था) जब उन्होनें अपनी क़ौम से कहा कि (अफसोस) तुम ऐसी बदकारी (अग़लाम) करते हो कि तुमसे पहले सारी ख़ुदाई में किसी ने ऐसी बदकारी नहीं की थी
\end{hindi}}
\flushright{\begin{Arabic}
\quranayah[7][81]
\end{Arabic}}
\flushleft{\begin{hindi}
हाँ तुम औरतों को छोड़कर शहवत परस्ती के वास्ते मर्दों की तरफ माएल होते हो (हालाकि उसकी ज़रूरत नहीं) मगर तुम लोग कुछ हो ही बेहूदा
\end{hindi}}
\flushright{\begin{Arabic}
\quranayah[7][82]
\end{Arabic}}
\flushleft{\begin{hindi}
सिर्फ करनें वालों (को नुत्फे को ज़ाए करते हो उस पर उसकी क़ौम का उसके सिवा और कुछ जवाब नहीं था कि वह आपस में कहने लगे कि उन लोगों को अपनी बस्ती से निकाल बाहर करो क्योंकि ये तो वह लोग हैं जो पाक साफ बनना चाहते हैं)
\end{hindi}}
\flushright{\begin{Arabic}
\quranayah[7][83]
\end{Arabic}}
\flushleft{\begin{hindi}
तब हमने उनको और उनके घर वालों को नजात दी मगर सिर्फ (एक) उनकी बीबी को कि वह (अपनी बदआमाली से) पीछे रह जाने वालों में थी
\end{hindi}}
\flushright{\begin{Arabic}
\quranayah[7][84]
\end{Arabic}}
\flushleft{\begin{hindi}
और हमने उन लोगों पर (पत्थर का) मेह बरसाया-पस ज़रा ग़ौर तो करो कि गुनाहगारों का अन्जाम आखिर क्या हुआ
\end{hindi}}
\flushright{\begin{Arabic}
\quranayah[7][85]
\end{Arabic}}
\flushleft{\begin{hindi}
और (हमने) मदयन (वालों के) पास उनके भाई शुएब को (रसूल बनाकर भेजा) तो उन्होंने (उन लोगों से) कहा ऐ मेरी क़ौम ख़ुदा ही की इबादत करो उसके सिवा कोई दूसरा माबूद नहीं (और) तुम्हारे पास तो तुम्हारे परवरदिगार की तरफ से एक वाजेए व रौशन मौजिज़ा (भी) आ चुका तो नाप और तौल पूरी किया करो और लोगों को उनकी (ख़रीदी हुई) चीज़ में कम न दिया करो और ज़मीन में उसकी असलाह व दुरूस्ती के बाद फसाद न करते फिरो अगर तुम सच्चे ईमानदार हो तो यही तुम्हारे हक़ में बेहतर है
\end{hindi}}
\flushright{\begin{Arabic}
\quranayah[7][86]
\end{Arabic}}
\flushleft{\begin{hindi}
और तुम लोग जो रास्तों पर (बैठकर) जो ख़ुदा पर ईमान लाया है उसको डराते हो और ख़ुदा की राह से रोकते हो और उसकी राह में (ख्वाहमाख्वाह) कज़ी ढूँढ निकालते हो अब न बैठा करो और उसको तो याद करो कि जब तुम (शुमार में) कम थे तो ख़ुदा ही ने तुमको बढ़ाया, और ज़रा ग़ौर तो करो कि (आख़िर) फसाद फैलाने वालों का अन्जाम क्या हुआ
\end{hindi}}
\flushright{\begin{Arabic}
\quranayah[7][87]
\end{Arabic}}
\flushleft{\begin{hindi}
और जिन बातों का मै पैग़ाम लेकर आया हूँ अगर तुममें से एक गिरोह ने उनको मान लिया और एक गिरोह ने नहीं माना तो (कुछ परवाह नहीं) तो तुम सब्र से बैठे (देखते) रहो यहाँ तक कि ख़ुदा (खुद) हमारे दरमियान फैसला कर दे, वह तो सबसे बेहतर फैसला करने वाला है
\end{hindi}}
\flushright{\begin{Arabic}
\quranayah[7][88]
\end{Arabic}}
\flushleft{\begin{hindi}
तो उनकी क़ौम में से जिन लोगों को (अपनी हशमत (दुनिया पर) बड़ा घमण्ड था कहने लगे कि ऐ शुएब हम तुम्हारे साथ ईमान लाने वालों को अपनी बस्ती से निकाल बाहर कर देगें मगर जबकि तुम भी हमारे उसी मज़हब मिल्लत में लौट कर आ जाओ
\end{hindi}}
\flushright{\begin{Arabic}
\quranayah[7][89]
\end{Arabic}}
\flushleft{\begin{hindi}
हम अगरचे तुम्हारे मज़हब से नफरत ही रखते हों (तब भी लौट जाएं माज़अल्लाह) जब तुम्हारे बातिल दीन से ख़ुदा ने मुझे नजात दी उसके बाद भी अब अगर हम तुम्हारे मज़हब मे लौट जाएं तब हमने ख़ुदा पर बड़ा झूठा बोहतान बॉधा (ना) और हमारे वास्ते तो किसी तरह जायज़ नहीं कि हम तुम्हारे मज़हब की तरफ लौट जाएँ मगर हाँ जब मेरा परवरदिगार अल्लाह चाहे तो हमारा परवरदिगार तो (अपने) इल्म से तमाम (आलम की) चीज़ों को घेरे हुए है हमने तो ख़ुदा ही पर भरोसा कर लिया ऐ हमारे परवरदिगार तू ही हमारे और हमारी क़ौम के दरमियान ठीक ठीक फैसला कर दे और तू सबसे बेहतर फ़ैसला करने वाला है
\end{hindi}}
\flushright{\begin{Arabic}
\quranayah[7][90]
\end{Arabic}}
\flushleft{\begin{hindi}
और उनकी क़ौम के चन्द सरदार जो काफिर थे (लोगों से) कहने लगे कि अगर तुम लोगों ने शुएब की पैरवी की तो उसमें शक़ ही नहीं कि तुम सख्त घाटे में रहोगे
\end{hindi}}
\flushright{\begin{Arabic}
\quranayah[7][91]
\end{Arabic}}
\flushleft{\begin{hindi}
ग़रज़ उन लोगों को ज़लज़ले ने ले डाला बस तो वह अपने घरों में औन्धे पड़े रह गए
\end{hindi}}
\flushright{\begin{Arabic}
\quranayah[7][92]
\end{Arabic}}
\flushleft{\begin{hindi}
जिन लोगों ने शुएब को झुठलाया था वह (ऐसे मर मिटे कि) गोया उन बस्तियों में कभी आबाद ही न थे जिन लोगों ने शुएब को झुठलाया वही लोग घाटे में रहे
\end{hindi}}
\flushright{\begin{Arabic}
\quranayah[7][93]
\end{Arabic}}
\flushleft{\begin{hindi}
तब शुएब उन लोगों के सर से टल गए और (उनसे मुख़ातिब हो के) कहा ऐ मेरी क़ौम मैं ने तो अपने परवरदिगार के पैग़ाम तुम तक पहुँचा दिए और तुम्हारी ख़ैर ख्वाही की थी, फिर अब मैं काफिरों पर क्यों कर अफसोस करूँ
\end{hindi}}
\flushright{\begin{Arabic}
\quranayah[7][94]
\end{Arabic}}
\flushleft{\begin{hindi}
और हमने किसी बस्ती में कोई नबी नही भेजा मगर वहाँ के रहने वालों को (कहना न मानने पर) सख्ती और मुसीबत में मुब्तिला किया ताकि वह लोग (हमारी बारगाह में) गिड़गिड़ाए
\end{hindi}}
\flushright{\begin{Arabic}
\quranayah[7][95]
\end{Arabic}}
\flushleft{\begin{hindi}
फिर हमने तकलीफ़ की जगह आराम को बदल दिया यहाँ तक कि वह लोग बढ़ निकले और कहने लगे कि इस तरह की तकलीफ़ व आराम तो हमारे बाप दादाओं को पहुँच चुका है तब हमने (उस बढ़ाने के की सज़ा में (अचानक उनको अज़ाब में) गिरफ्तार किया
\end{hindi}}
\flushright{\begin{Arabic}
\quranayah[7][96]
\end{Arabic}}
\flushleft{\begin{hindi}
और वह बिल्कुल बेख़बर थे और अगर उन बस्तियों के रहने वाले ईमान लाते और परहेज़गार बनते तो हम उन पर आसमान व ज़मीन की बरकतों (के दरवाजे) ख़ोल देते मगर (अफसोस) उन लोगों ने (हमारे पैग़म्बरों को) झूठलाया तो हमने भी उनके करतूतों की बदौलत उन को (अज़ाब में) गिरफ्तार किया
\end{hindi}}
\flushright{\begin{Arabic}
\quranayah[7][97]
\end{Arabic}}
\flushleft{\begin{hindi}
(उन) बस्तियों के रहने वाले उस बात से बेख़ौफ हैं कि उन पर हमारा अज़ाब रातों रात आ जाए जब कि वह पड़े बेख़बर सोते हों
\end{hindi}}
\flushright{\begin{Arabic}
\quranayah[7][98]
\end{Arabic}}
\flushleft{\begin{hindi}
या उन बस्तियों वाले इससे बेख़ौफ हैं कि उन पर दिन दहाड़े हमारा अज़ाब आ पहुँचे जब वह खेल कूद (में मशग़ूल हो)
\end{hindi}}
\flushright{\begin{Arabic}
\quranayah[7][99]
\end{Arabic}}
\flushleft{\begin{hindi}
तो क्या ये लोग ख़ुदा की तद्बीर से ढीट हो गए हैं तो (याद रहे कि) ख़ुदा के दॉव से घाटा उठाने वाले ही निडर हो बैठे हैं
\end{hindi}}
\flushright{\begin{Arabic}
\quranayah[7][100]
\end{Arabic}}
\flushleft{\begin{hindi}
क्या जो लोग अहले ज़मीन के बाद ज़मीन के वारिस (व मालिक) होते हैं उन्हें ये मालूम नहीं कि अगर हम चाहते तो उनके गुनाहों की बदौलत उनको मुसीबत में फँसा देते (मगर ये लोग ऐसे नासमझ हैं कि गोया) उनके दिलों पर हम ख़ुद (मोहर कर देते हैं कि ये लोग कुछ सुनते ही नहीं
\end{hindi}}
\flushright{\begin{Arabic}
\quranayah[7][101]
\end{Arabic}}
\flushleft{\begin{hindi}
(ऐ रसूल) ये चन्द बस्तियाँ हैं जिन के हालात हम तुमसे बयान करते हैं और इसमें तो शक़ ही नहीं कि उनके पैग़म्बर उनके पास वाजेए व रौशन मौजिज़े लेकर आए मगर ये लोग जिसके पहले झुठला चुके थे उस पर भला काहे को ईमान लाने वाले थे ख़ुदा यूं काफिरों के दिलों पर अलामत मुकर्रर कर देता है (कि ये ईमान न लाएँगें)
\end{hindi}}
\flushright{\begin{Arabic}
\quranayah[7][102]
\end{Arabic}}
\flushleft{\begin{hindi}
और हमने तो उसमें से अक्सरों का एहद (ठीक) न पाया और हमने उनमें से अक्सरों को बदकार ही पाया
\end{hindi}}
\flushright{\begin{Arabic}
\quranayah[7][103]
\end{Arabic}}
\flushleft{\begin{hindi}
फिर हमने (उन पैग़म्बरान मज़कूरीन के बाद) मूसा को फिरऔन और उसके सरदारों के पास मौजिज़े अता करके (रसूल बनाकर) भेजा तो उन लोगों ने उन मौजिज़ात के साथ (बड़ी बड़ी) शरारतें की पस ज़रा ग़ौर तो करो कि आख़िर फसादियों का अन्जाम क्या हुआ
\end{hindi}}
\flushright{\begin{Arabic}
\quranayah[7][104]
\end{Arabic}}
\flushleft{\begin{hindi}
और मूसा ने (फिरऔन से) कहा ऐ फिरऔनमें यक़ीनन परवरदिगारे आलम का रसूल हूँ
\end{hindi}}
\flushright{\begin{Arabic}
\quranayah[7][105]
\end{Arabic}}
\flushleft{\begin{hindi}
मुझ पर वाजिब है कि ख़ुदा पर सच के सिवा (एक हुरमत भी झूठ) न कहूँ मै यक़ीनन तुम्हारे पास तुम्हारे परवरदिगार की तरफ से वाजेए व रोशन मौजिज़े लेकर आया हूँ
\end{hindi}}
\flushright{\begin{Arabic}
\quranayah[7][106]
\end{Arabic}}
\flushleft{\begin{hindi}
तो तू बनी ईसराइल को मेरे हमराह करे दे फिरऔन कहने लगा अगर तुम सच्चे हो और वाक़ई कोई मौजिज़ा लेकर आए हो तो उसे दिखाओ
\end{hindi}}
\flushright{\begin{Arabic}
\quranayah[7][107]
\end{Arabic}}
\flushleft{\begin{hindi}
(ये सुनते ही) मूसा ने अपनी छड़ी (ज़मीन पर) डाल दी पस वह यकायक (अच्छा खासा) ज़ाहिर बज़ाहिर अजदहा बन गई
\end{hindi}}
\flushright{\begin{Arabic}
\quranayah[7][108]
\end{Arabic}}
\flushleft{\begin{hindi}
और अपना हाथ बाहर निकाला तो क्या देखते है कि वह हर शख़्श की नज़र मे जगमगा रहा है
\end{hindi}}
\flushright{\begin{Arabic}
\quranayah[7][109]
\end{Arabic}}
\flushleft{\begin{hindi}
तब फिरऔन के क़ौम के चन्द सरदारों ने कहा ये तो अलबत्ता बड़ा माहिर जादूगर है
\end{hindi}}
\flushright{\begin{Arabic}
\quranayah[7][110]
\end{Arabic}}
\flushleft{\begin{hindi}
ये चाहता है कि तुम्हें तुम्हारें मुल्क से निकाल बाहर कर दे तो अब तुम लोग उसके बारे में क्या सलाह देते हो
\end{hindi}}
\flushright{\begin{Arabic}
\quranayah[7][111]
\end{Arabic}}
\flushleft{\begin{hindi}
(आख़िर) सबने मुत्तफिक़ अलफाज़ (एक ज़बान होकर) कहा कि (ऐ फिरऔन) उनको और उनके भाई (हारून) को चन्द दिन कैद में रखिए और (एतराफ़ के) शहरों में हरकारों को भेजिए
\end{hindi}}
\flushright{\begin{Arabic}
\quranayah[7][112]
\end{Arabic}}
\flushleft{\begin{hindi}
कि तमाम बड़े बड़े जादूगरों का जमा करके अपके पास दरबार में हाज़िर करें
\end{hindi}}
\flushright{\begin{Arabic}
\quranayah[7][113]
\end{Arabic}}
\flushleft{\begin{hindi}
ग़रज़ जादूगर सब फिरऔन के पास हाज़िर होकर कहने लगे कि अगर हम (मूसा से) जीत जाएं तो हमको बड़ा भारी इनाम ज़रुर मिलना चाहिए
\end{hindi}}
\flushright{\begin{Arabic}
\quranayah[7][114]
\end{Arabic}}
\flushleft{\begin{hindi}
फिरऔन ने कहा (हॉ इनाम ही नहीं) बल्कि फिर तो तुम हमारे दरबार के मुक़र्रेबीन में से होगें
\end{hindi}}
\flushright{\begin{Arabic}
\quranayah[7][115]
\end{Arabic}}
\flushleft{\begin{hindi}
और मुक़र्रर वक्त पर सब जमा हुए तो बोल उठे कि ऐ मूसा या तो तुम्हें (अपने मुन्तसिर (मंत्र)) या हम ही (अपने अपने मंत्र फेके)
\end{hindi}}
\flushright{\begin{Arabic}
\quranayah[7][116]
\end{Arabic}}
\flushleft{\begin{hindi}
मूसा ने कहा (अच्छा पहले) तुम ही फेक (के अपना हौसला निकालो) तो तब जो ही उन लोगों ने (अपनी रस्सियाँ) डाली तो लोगों की नज़र बन्दी कर दी (कि सब सापँ मालूम होने लगे) और लोगों को डरा दिया
\end{hindi}}
\flushright{\begin{Arabic}
\quranayah[7][117]
\end{Arabic}}
\flushleft{\begin{hindi}
और उन लोगों ने बड़ा (भारी जादू दिखा दिया और हमने मूसा के पास वही भेजी कि (बैठे क्या हो) तुम भी अपनी छड़ी डाल दो तो क्या देखते हैं कि वह छड़ी उनके बनाए हुए (झूठे साँपों को) एक एक करके निगल रही है
\end{hindi}}
\flushright{\begin{Arabic}
\quranayah[7][118]
\end{Arabic}}
\flushleft{\begin{hindi}
अल किस्सा हक़ बात तो जम के बैठी और उनकी सारी कारस्तानी मटियामेट हो गई
\end{hindi}}
\flushright{\begin{Arabic}
\quranayah[7][119]
\end{Arabic}}
\flushleft{\begin{hindi}
पस फिरऔन और उसके तरफदार सब के सब इस अखाड़े मे हारे और ज़लील व रूसवा हो के पलटे
\end{hindi}}
\flushright{\begin{Arabic}
\quranayah[7][120]
\end{Arabic}}
\flushleft{\begin{hindi}
और जादूगर सब मूसा के सामने सजदे में गिर पड़े
\end{hindi}}
\flushright{\begin{Arabic}
\quranayah[7][121]
\end{Arabic}}
\flushleft{\begin{hindi}
और (आजिज़ी से) बोले हम सारे जहाँन के परवरदिगार पर ईमान लाए
\end{hindi}}
\flushright{\begin{Arabic}
\quranayah[7][122]
\end{Arabic}}
\flushleft{\begin{hindi}
जो मूसा व हारून का परवरदिगार है
\end{hindi}}
\flushright{\begin{Arabic}
\quranayah[7][123]
\end{Arabic}}
\flushleft{\begin{hindi}
फिरऔन ने कहा (हाए) तुम लोग मेरी इजाज़त के क़ब्ल (पहले) उस पर ईमान ले आए ये ज़रूर तुम लोगों की मक्कारी है जो तुम लोगों ने उस शहर में फैला रखी है ताकि उसके बाशिन्दों को यहाँ से निकाल कर बाहर करो पस तुम्हें अन क़रीब ही उस शरारत का मज़ा मालूम हो जाएगा
\end{hindi}}
\flushright{\begin{Arabic}
\quranayah[7][124]
\end{Arabic}}
\flushleft{\begin{hindi}
मै तो यक़ीनन तुम्हारे (एक तरफ के) हाथ और दूसरी तरफ के पॉव कटवा डालूंगा फिर तुम सबके सब को सूली दे दूंगा
\end{hindi}}
\flushright{\begin{Arabic}
\quranayah[7][125]
\end{Arabic}}
\flushleft{\begin{hindi}
जादूगर कहने लगे हम को तो (आख़िर एक रोज़) अपने परवरदिगार की तरफ लौट कर जाना (मर जाना) है
\end{hindi}}
\flushright{\begin{Arabic}
\quranayah[7][126]
\end{Arabic}}
\flushleft{\begin{hindi}
तू हमसे उसके सिवा और काहे की अदावत रखता है कि जब हमारे पास ख़ुदा की निशानियाँ आयी तो हम उन पर ईमान लाए (और अब तो हमारी ये दुआ है कि) ऐ हमारे परवरदिगार हम पर सब्र (का मेंह बरसा)
\end{hindi}}
\flushright{\begin{Arabic}
\quranayah[7][127]
\end{Arabic}}
\flushleft{\begin{hindi}
और हमने अपनी फरमाबरदारी की हालत में दुनिया से उठा ले और फिरऔन की क़ौम के चन्द सरदारों ने (फिरऔन) से कहा कि क्या आप मूसा और उसकी क़ौम को उनकी हालत पर छोड़ देंगे कि मुल्क में फ़साद करते फिरे और आपके और आपके ख़ुदाओं (की परसतिश) को छोड़ बैठें- फिरऔन कहने लगा (तुम घबराओ नहीं) हम अनक़रीब ही उनके बेटों की क़त्ल करते हैं और उनकी औरतों को (लौन्डियॉ बनाने के वास्ते) जिन्दा रखते हैं और हम तो उन पर हर तरह क़ाबू रखते हैं
\end{hindi}}
\flushright{\begin{Arabic}
\quranayah[7][128]
\end{Arabic}}
\flushleft{\begin{hindi}
(ये सुनकर) मूसा ने अपनी क़ौम से कहा कि (भाइयों) ख़ुदा से मदद माँगों और सब्र करो सारी ज़मीन तो ख़ुदा ही की है वह अपने बन्दों में जिसकी चाहे उसका वारिस (व मालिक) बनाए और ख़ातमा बिल ख़ैर तो सब परहेज़गार ही का है
\end{hindi}}
\flushright{\begin{Arabic}
\quranayah[7][129]
\end{Arabic}}
\flushleft{\begin{hindi}
वह लोग कहने लगे कि (ऐ मूसा) तुम्हारे आने के क़ब्ल (पहले) ही से और तुम्हारे आने के बाद भी हम को तो बराबर तकलीफ ही पहुँच रही है (आख़िर कहाँ तक सब्र करें) मूसा ने कहा अनकरीब ही तुम्हारा परवरदिगार तुम्हारे दुश्मन को हलाक़ करेगा और तुम्हें (उसका जानशीन) बनाएगा फिर देखेगा कि तुम कैसा काम करते हो
\end{hindi}}
\flushright{\begin{Arabic}
\quranayah[7][130]
\end{Arabic}}
\flushleft{\begin{hindi}
और बेशक हमने फिरऔन के लोगों को बरसों से कहत और फलों की कम पैदावार (के अज़ाब) में गिरफ्तार किया ताकि वह इबरत हासिल करें
\end{hindi}}
\flushright{\begin{Arabic}
\quranayah[7][131]
\end{Arabic}}
\flushleft{\begin{hindi}
तो जब उन्हें कोई राहत मिलती तो कहने लगते कि ये तो हमारे लिए सज़ावार ही है और जब उन्हें कोई मुसीबत पहुँचती तो मूसा और उनके साथियों की बदशुगूनी समझते देखो उनकी बदशुगूनी तो ख़ुदा के हा (लिखी जा चुकी) थी मगर बहुतेरे लोग नही जानते हैं
\end{hindi}}
\flushright{\begin{Arabic}
\quranayah[7][132]
\end{Arabic}}
\flushleft{\begin{hindi}
और फिरऔन के लोग मूसा से एक मरतबा कहने लगे कि तुम हम पर जादू करने के लिए चाहे जितनी निशानियाँ लाओ मगर हम तुम पर किसी तरह ईमान नहीं लाएँगें
\end{hindi}}
\flushright{\begin{Arabic}
\quranayah[7][133]
\end{Arabic}}
\flushleft{\begin{hindi}
तब हमने उन पर (पानी को) तूफान और टिड़डियाँ और जुएं और मेंढ़कों और खून (का अज़ाब भेजा कि सब जुदा जुदा (हमारी कुदरत की) निशानियाँ थी उस पर भी वह लोग तकब्बुर ही करते रहें और वह लोग गुनेहगार तो थे ही
\end{hindi}}
\flushright{\begin{Arabic}
\quranayah[7][134]
\end{Arabic}}
\flushleft{\begin{hindi}
(और जब उन पर अज़ाब आ पड़ता तो कहने लगते कि ऐ मूसा तुम से जो ख़ुदा ने (क़बूल दुआ का) अहद किया है उसी की उम्मीद पर अपने ख़ुदा से दुआ माँगों और अगर तुमने हम से अज़ाब को टाल दिया तो हम ज़रूर भेज देगें
\end{hindi}}
\flushright{\begin{Arabic}
\quranayah[7][135]
\end{Arabic}}
\flushleft{\begin{hindi}
फिर जब हम उनसे उस वक्त क़े वास्ते जिस तक वह ज़रूर पहुँचते अज़ाब को हटा लेते तो फिर फौरन बद अहदी करने लगते
\end{hindi}}
\flushright{\begin{Arabic}
\quranayah[7][136]
\end{Arabic}}
\flushleft{\begin{hindi}
तब आख़िर हमने उनसे (उनकी शरारत का) बदला लिया तो चूंकि वह लोग हमारी आयतों को झुटलाते थे और उनसे ग़ाफिल रहते थे हमने उन्हें दरिया में डुबो दिया
\end{hindi}}
\flushright{\begin{Arabic}
\quranayah[7][137]
\end{Arabic}}
\flushleft{\begin{hindi}
और जिन बेचारों को ये लोग कमज़ोर समझते थे उन्हीं को (मुल्क शाम की) ज़मीन का जिसमें हमने (ज़रखेज़ होने की) बरकत दी थी उसके पूरब पश्चिम (सब) का वारिस (मालिक) बना दिया और चूंकि बनी इसराईल नें (फिरऔन के ज़ालिमों) पर सब्र किया था इसलिए तुम्हारे परवरदिगार का नेक वायदा (जो उसने बनी इसराइल से किया था) पूरा हो गया और जो कुछ फिरऔन और उसकी क़ौम के लोग करते थे और जो ऊँची ऊँची इमारते बनाते थे सब हमने बरबाद कर दी
\end{hindi}}
\flushright{\begin{Arabic}
\quranayah[7][138]
\end{Arabic}}
\flushleft{\begin{hindi}
और हमने बनी ईसराइल को दरिया के उस पार उतार दिया तो एक ऐसे लोगों पर से गुज़रे जो अपने (हाथों से बनाए हुए) बुतों की परसतिश पर जमा बैठे थे (तो उनको देख कर बनी ईसराइल से) कहने लगे ऐ मूसा जैसे उन लोगों के माबूद (बुत) हैं वैसे ही हमारे लिए भी एक माबूद बनाओ मूसा ने जवाब दिया कि तुम लोग जाहिल लोग हो
\end{hindi}}
\flushright{\begin{Arabic}
\quranayah[7][139]
\end{Arabic}}
\flushleft{\begin{hindi}
(अरे कमबख्तो) ये लोग जिस मज़हब पर हैं (वह यक़ीनी बरबाद होकर रहेगा) और जो अमल ये लोग कर रहे हैं (वह सब मिटिया मेट हो जाएगा)
\end{hindi}}
\flushright{\begin{Arabic}
\quranayah[7][140]
\end{Arabic}}
\flushleft{\begin{hindi}
(मूसा ने ये भी) कहा क्या तुम्हारा ये मतलब है कि ख़ुदा को छोड़कर मै दूसरे को तुम्हारा माबूद तलाश करू
\end{hindi}}
\flushright{\begin{Arabic}
\quranayah[7][141]
\end{Arabic}}
\flushleft{\begin{hindi}
हालॉकि उसने तुमको सारी खुदाई पर फज़ीलत दी है (ऐ बनी इसराइल वह वक्त याद करो) जब हमने तुमको फिरऔन के लोगों से नजात दी जब वह लोग तुम्हें बड़ी बड़ी तकलीफें पहुंचाते थे तुम्हारे बेटों को तो (चुन चुन कर) क़त्ल कर डालते थे और तुम्हारी औरतों को (लौन्डियॉ बनाने के वास्ते ज़िन्दा रख छोड़ते) और उसमें तुम्हारे परवरदिगार की तरफ से तुम्हारे (सब्र की) सख्त आज़माइश थी
\end{hindi}}
\flushright{\begin{Arabic}
\quranayah[7][142]
\end{Arabic}}
\flushleft{\begin{hindi}
और हमने मूसा से तौरैत देने के लिए तीस रातों का वायदा किया और हमने उसमें दस रोज़ बढ़ाकर पूरा कर दिया ग़रज़ उसके परवरदिगार का वायदा चालीस रात में पूरा हो गया और (चलते वक्त) मूसा ने अपने भाई हारून कहा कि तुम मेरी क़ौम में मेरे जानशीन रहो और उनकी इसलाह करना और फसाद करने वालों के तरीक़े पर न चलना
\end{hindi}}
\flushright{\begin{Arabic}
\quranayah[7][143]
\end{Arabic}}
\flushleft{\begin{hindi}
और जब मूसा हमारा वायदा पूरा करते (कोहेतूर पर) आए और उनका परवरदिगार उनसे हम कलाम हुआ तो मूसा ने अर्ज़ किया कि ख़ुदाया तू मेझे अपनी एक झलक दिखला दे कि मैं तूझे देखँ ख़ुदा ने फरमाया तुम मुझे हरगिज़ नहीं देख सकते मगर हॉ उस पहाड़ की तरफ देखो (हम उस पर अपनी तजल्ली डालते हैं) पस अगर (पहाड़) अपनी जगह पर क़ायम रहे तो समझना कि अनक़रीब मुझे भी देख लोगे (वरना नहीं) फिर जब उनके परवरदिगार ने पहाड़ पर तजल्ली डाली तो उसको चकनाचूर कर दिया और मूसा बेहोश होकर गिर पड़े फिर जब होश में आए तो कहने लगे ख़ुदा वन्दा तू (देखने दिखाने से) पाक व पाकीज़ा है-मैने तेरी बारगाह में तौबा की और मै सब से पहले तेरी अदम रवायत का यक़ीन करता हूँ
\end{hindi}}
\flushright{\begin{Arabic}
\quranayah[7][144]
\end{Arabic}}
\flushleft{\begin{hindi}
ख़ुदा ने फरमाया ऐ मूसा मैने तुमको तमाम लोगों पर अपनी पैग़म्बरी और हम कलामी (का दरजा देकर) बरगूज़ीदा किया है तब जो (किताब तौरैत) हमने तुमको अता की है उसे लो और शुक्रगुज़ार रहो
\end{hindi}}
\flushright{\begin{Arabic}
\quranayah[7][145]
\end{Arabic}}
\flushleft{\begin{hindi}
और हमने (तौरैत की) तख्तियों में मूसा के लिए हर तरह की नसीहत और हर चीज़ का तफसीलदार बयान लिख दिया था तो (ऐ मूसा) तुम उसे मज़बूती से तो (अमल करो) और अपनी क़ौम को हुक्म दे दो कि उसमें की अच्छी बातों पर अमल करें और बहुत जल्द तुम्हें बदकिरदारों का घर दिखा दूँगा (कि कैसे उजड़ते हैं)
\end{hindi}}
\flushright{\begin{Arabic}
\quranayah[7][146]
\end{Arabic}}
\flushleft{\begin{hindi}
जो लोग (ख़ुदा की) ज़मीन पर नाहक़ अकड़ते फिरते हैं उनको अपनी आयतों से बहुत जल्द फेर दूंगा और मै क्या फेरूंगा ख़ुदा (उसका दिल ऐसा सख्त है कि) अगर दुनिया जहॉन के सारे मौजिज़े भी देखते तो भी ये उन पर ईमान न लाएगें और (अगर) सीधा रास्ता भी देख पाएं तो भी अपनी राह न जाएगें और अगर गुमराही की राह देख लेगें तो झटपट उसको अपना तरीक़ा बना लेगें ये कजरवी इस सबब से हुई कि उन लोगों ने हमारी आयतों को झुठला दिया और उनसे ग़फलत करते रहे
\end{hindi}}
\flushright{\begin{Arabic}
\quranayah[7][147]
\end{Arabic}}
\flushleft{\begin{hindi}
और जिन लोगों ने हमारी आयतों को और आख़िरत की हुज़ूरी को झूठलाया उनका सब किया कराया अकारत हो गया, उनको बस उन्हीं आमाल की जज़ा या सज़ा दी जाएगी जो वह करते थे
\end{hindi}}
\flushright{\begin{Arabic}
\quranayah[7][148]
\end{Arabic}}
\flushleft{\begin{hindi}
और मूसा की क़ौम ने (कोहेतूर पर) उनके जाने के बाद अपने जेवरों को (गलाकर) एक बछड़े की मूरत बनाई (यानि) एक जिस्म जिसमें गाए की सी आवाज़ थी (अफसोस) क्या उन लोगों ने इतना भी न देखा कि वह न तो उनसे बात ही कर सकता और न किसी तरह की हिदायत ही कर सकता है (खुलासा) उन लोगों ने उसे (अपनी माबूद बना लिया)
\end{hindi}}
\flushright{\begin{Arabic}
\quranayah[7][149]
\end{Arabic}}
\flushleft{\begin{hindi}
और आप अपने ऊपर ज़ुल्म करते थे और जब वह पछताए और उन्होने अपने को यक़ीनी गुमराह देख लिया तब कहने लगे कि अगर हमारा परदिगार हम पर रहम नहीं करेगा और हमारा कुसूर न माफ़ करेगा तो हम यक़ीनी घाटा उठाने वालों में हो जाएगें
\end{hindi}}
\flushright{\begin{Arabic}
\quranayah[7][150]
\end{Arabic}}
\flushleft{\begin{hindi}
और जब मूसा पलट कर अपनी क़ौम की तरफ आए तो (ये हालत देखकर) रंज व गुस्से में (अपनी क़ौम से) कहने लगे कि तुम लोगों ने मेरे बाद बहुत बुरी हरकत की-तुम लोग अपने परवरदिगार के हुक्म (मेरे आने में) किस कदर जल्दी कर बैठे और (तौरैत की) तख्तियों को फेंक दिया और अपने भाई (हारून) के सर (के बालों को पकड़ कर अपनी तर फ खींचने लगे) उस पर हारून ने कहा ऐ मेरे मांजाए (भाई) मै क्या करता क़ौम ने मुझे हक़ीर समझा और (मेरा कहना न माना) बल्कि क़रीब था कि मुझे मार डाले तो मुझ पर दुश्मनों को न हॅसवाइए और मुझे उन ज़ालिम लोगों के साथ न करार दीजिए
\end{hindi}}
\flushright{\begin{Arabic}
\quranayah[7][151]
\end{Arabic}}
\flushleft{\begin{hindi}
तब) मूसा ने कहा ऐ मेरे परवरदिगार मुझे और मेरे भाई को बख्श दे और हमें अपनी रहमत में दाख़िल कर और तू सबसे बढ़के रहम करने वाला है
\end{hindi}}
\flushright{\begin{Arabic}
\quranayah[7][152]
\end{Arabic}}
\flushleft{\begin{hindi}
बेशक जिन लोगों ने बछड़े को (अपना माबूद) बना लिया उन पर अनक़रीब ही उनके परवरदिगार की तरफ से अज़ाब नाज़िल होगा और दुनियावी ज़िन्दगी में ज़िल्लत (उसके अलावा) और हम बोहतान बॉधने वालों की ऐसी ही सज़ा करते हैं
\end{hindi}}
\flushright{\begin{Arabic}
\quranayah[7][153]
\end{Arabic}}
\flushleft{\begin{hindi}
और जिन लोगों ने बुरे काम किए फिर उसके बाद तौबा कर ली और ईमान लाए तो बेशक तुम्हारा परवरदिगार तौबा के बाद ज़रूर बख्शने वाला मेहरबान है
\end{hindi}}
\flushright{\begin{Arabic}
\quranayah[7][154]
\end{Arabic}}
\flushleft{\begin{hindi}
और जब मूसा का गुस्सा ठण्डा हुआ तो (तौरैत) की तख्तियों को (ज़मीन से) उठा लिया और तौरैत के नुस्खे में जो लोग अपने परवरदिगार से डरते है उनके लिए हिदायत और रहमत है
\end{hindi}}
\flushright{\begin{Arabic}
\quranayah[7][155]
\end{Arabic}}
\flushleft{\begin{hindi}
और मूसा ने अपनी क़ौम से हमारा वायदा पूरा करने को (कोहतूर पर ले जाने के वास्ते) सत्तर आदमियों को चुना फिर जब उनको ज़लज़ले ने आ पकड़ा तो हज़रत मूसा ने अर्ज़ किया परवरदिगार अगर तू चाहता तो मुझको और उन सबको पहले ही हलाक़ कर डालता क्या हम में से चन्द बेवकूफों की करनी की सज़ा में हमको हलाक़ करता है ये तो सिर्फ तेरी आज़माइश थी तू जिसे चाहे उसे गुमराही में छोड़ दे और जिसको चाहे हिदायत दे तू ही हमारा मालिक है तू ही हमारे कुसूर को माफ कर और हम पर रहम कर और तू तो तमाम बख्शने वालों से कहीें बेहतर है
\end{hindi}}
\flushright{\begin{Arabic}
\quranayah[7][156]
\end{Arabic}}
\flushleft{\begin{hindi}
और तू ही इस दुनिया (फ़ानी) और आख़िरत में हमारे वास्ते भलाई के लिए लिख ले हम तेरी ही तरफ रूझू करते हैं ख़ुदा ने फरमाया जिसको मैं चाहता हूँ (मुस्तहक़ समझकर) अपना अज़ाब पहुँचा देता हूँ और मेरी रहमत हर चीज़ पर छाई हैं मै तो उसे बहुत जल्द ख़ास उन लोगों के लिए लिख दूँगा (जो बुरी बातों से) बचते रहेंगे और ज़कात दिया करेंगे और जो हमारी बातों पर ईमान रखा करेंगें
\end{hindi}}
\flushright{\begin{Arabic}
\quranayah[7][157]
\end{Arabic}}
\flushleft{\begin{hindi}
(यानि) जो लोग हमारे बनी उल उम्मी पैग़म्बर के क़दम बा क़दम चलते हैं जिस (की बशारत) को अपने हॉ तौरैत और इन्जील में लिखा हुआ पाते है (वह नबी) जो अच्छे काम का हुक्म देता है और बुरे काम से रोकता है और जो पाक व पाकीज़ा चीजे तो उन पर हलाल और नापाक गन्दी चीजे उन पर हराम कर देता है और (सख्त एहकाम का) बोझ जो उनकी गर्दन पर था और वह फन्दे जो उन पर (पड़े हुए) थे उनसे हटा देता है पस (याद रखो कि) जो लोग (नबी मोहम्मद) पर ईमान लाए और उसकी इज्ज़त की और उसकी मदद की और उस नूर (क़ुरान) की पैरवी की जो उसके साथ नाज़िल हुआ है तो यही लोग अपनी दिली मुरादे पाएंगें
\end{hindi}}
\flushright{\begin{Arabic}
\quranayah[7][158]
\end{Arabic}}
\flushleft{\begin{hindi}
(ऐ रसूल) तुम (उन लोगों से) कह दो कि लोगों में तुम सब लोगों के पास उस ख़ुदा का भेजा हुआ (पैग़म्बर) हूँ जिसके लिए ख़ास सारे आसमान व ज़मीन की बादशाहत (हुकूमत) है उसके सिवा और कोई माबूद नहीं वही ज़िन्दा करता है वही मार डालता है पस (लोगों) ख़ुदा और उसके रसूल नबी उल उम्मी पर ईमान लाओ जो (ख़ुद भी) ख़ुदा पर और उसकी बातों पर (दिल से) ईमान रखता है और उसी के क़दम बा क़दम चलो ताकि तुम हिदायत पाओ
\end{hindi}}
\flushright{\begin{Arabic}
\quranayah[7][159]
\end{Arabic}}
\flushleft{\begin{hindi}
और मूसा की क़ौम के कुछ लोग ऐसे भी है जो हक़ बात की हिदायत भी करते हैं और हक़ के (मामलात में) इन्साफ़ भी करते हैं
\end{hindi}}
\flushright{\begin{Arabic}
\quranayah[7][160]
\end{Arabic}}
\flushleft{\begin{hindi}
और हमने बनी ईसराइल के एक एक दादा की औलाद को जुदा जुदा बारह गिरोह बना दिए और जब मूसा की क़ौम ने उनसे पानी मॉगा तो हमने उनके पास वही भेजी कि तुम अपनी छड़ी पत्थर पर मारो (छड़ी मारना था) कि उस पत्थर से पानी के बारह चश्मे फूट निकले और ऐसे साफ साफ अलग अलग कि हर क़बीले ने अपना अपना घाट मालूम कर लिया और हमने बनी ईसराइल पर अब्र (बादल) का साया किया और उन पर मन व सलवा (सी नेअमत) नाज़िल की (लोगों) जो पाक (पाकीज़ा) चीज़े तुम्हें दी हैं उन्हें (शौक़ से खाओ पियो) और उन लोगों ने (नाफरमानी करके) कुछ हमारा नहीं बिगाड़ा बल्कि अपना आप ही बिगाड़ते हैं
\end{hindi}}
\flushright{\begin{Arabic}
\quranayah[7][161]
\end{Arabic}}
\flushleft{\begin{hindi}
और जब उनसे कहा गया कि उस गाँव में जाकर रहो सहो और उसके मेवों से जहाँ तुम्हारा जी चाहे (शौक़ से) खाओ (पियो) और मुँह से हुतमा कहते और सजदा करते हुए दरवाजे में दाखिल हो तो हम तुम्हारी ख़ताए बख्श देगें और नेकी करने वालों को हम कुछ ज्यादा ही देगें
\end{hindi}}
\flushright{\begin{Arabic}
\quranayah[7][162]
\end{Arabic}}
\flushleft{\begin{hindi}
तो ज़ालिमों ने जो बात उनसे कही गई थी उसे बदल कर कुछ और कहना शुरू किया तो हमने उनकी शरारतों की बदौलत उन पर आसमान से अज़ाब भेज दिया
\end{hindi}}
\flushright{\begin{Arabic}
\quranayah[7][163]
\end{Arabic}}
\flushleft{\begin{hindi}
और (ऐ रसूल) उनसे ज़रा उस गाँव का हाल तो पूछो जो दरिया के किनारे वाक़ऐ था जब ये लोग उनके बुज़ुर्ग शुम्बे (सनीचर) के दिन ज्यादती करने लगे कि जब उनका शुम्बे (वाला इबादत का) दिन होता तब तो मछलियाँ सिमट कर उनके सामने पानी पर उभर के आ जाती और जब उनका शुम्बा (वाला इबादत का) दिन न होता तो मछलियॉ उनके पास ही न फटकतीं और चॅूकि ये लोग बदचलन थे उस दरजे से हम भी उनकी यूं ही आज़माइश किया करते थे
\end{hindi}}
\flushright{\begin{Arabic}
\quranayah[7][164]
\end{Arabic}}
\flushleft{\begin{hindi}
और जब उनमें से एक जमाअत ने (उन लोगों में से जो शुम्बे के दिन शिकार को मना करते थे) कहा कि जिन्हें ख़ुदा हलाक़ करना या सख्त अज़ाब में मुब्तिला करना चाहता है उन्हें (बेफायदे) क्यो नसीहत करते हो तो वह कहने लगे कि फक़त तुम्हारे परवरदिगार में (अपने को) इल्ज़ाम से बचाने के लिए यायद इसलिए कि ये लोग परहेज़गारी एख्तियार करें
\end{hindi}}
\flushright{\begin{Arabic}
\quranayah[7][165]
\end{Arabic}}
\flushleft{\begin{hindi}
फिर जब वह लोग जिस चीज़ की उन्हें नसीहत की गई थी उसे भूल गए तो हमने उनको तो तजावीज़ (नजात) दे दी जो बुरे काम से लोगों को रोकते थे और जो लोग ज़ालिम थे उनको उनकी बद चलनी की वजह से बड़े अज़ाब में गिरफ्तार किया
\end{hindi}}
\flushright{\begin{Arabic}
\quranayah[7][166]
\end{Arabic}}
\flushleft{\begin{hindi}
फिर जिस बात की उन्हें मुमानिअत (रोक) की गई थी जब उन लोगों ने उसमें सरकशी की तो हमने हुक्म दिया कि तुम ज़लील और ख़्वार (धुत्कारे) हुए बन्दर बन जाओ (और वह बन गए)
\end{hindi}}
\flushright{\begin{Arabic}
\quranayah[7][167]
\end{Arabic}}
\flushleft{\begin{hindi}
(ऐ रसूल वह वक्त याद दिलाओ) जब तुम्हारे परवरदिगार ने पुकार पुकार के (बनी ईसराइल से कह दिया था कि वह क़यामत तक उन पर ऐसे हाक़िम को मुसल्लत देखेगा जो उन्हें बुरी बुरी तकलीफें देता रहे क्योंकि) इसमें तो शक़ ही नहीं कि तुम्हारा परवरदिगार बहुत जल्द अज़ाब करने वाला है और इसमें भी शक़ नहीं कि वह बड़ा बख्शने वाला (मेहरबान) भी है
\end{hindi}}
\flushright{\begin{Arabic}
\quranayah[7][168]
\end{Arabic}}
\flushleft{\begin{hindi}
और हमने उनको रूए ज़मीन में गिरोह गिरोह तितिर बितिर कर दिया उनमें के कुछ लोग तो नेक हैं और कुछ लोग और तरह के (बदकार) हैं और हमने उन्हें सुख और दुख (दोनो तरह) से आज़माया ताकि वह (शरारत से) बाज़ आए
\end{hindi}}
\flushright{\begin{Arabic}
\quranayah[7][169]
\end{Arabic}}
\flushleft{\begin{hindi}
फिर उनके बाद कुछ जानशीन हुए जो किताब (ख़ुदा तौरैत) के तो वारिस बने (मगर लोगों के कहने से एहकामे ख़ुदा को बदलकर (उसके ऐवज़) नापाक कमीनी दुनिया के सामान ले लेते हैं (और लुत्फ तो ये है) कहते हैं कि हम तो अनक़रीब बख्श दिए जाएंगें (और जो लोग उन पर तान करते हैं) अगर उनके पास भी वैसा ही (दूसरा सामान आ जाए तो उसे ये भी न छोड़े और) ले ही लें क्या उनसे किताब (ख़ुदा तौरैत) का एहदो पैमान नहीं लिया गया था कि ख़ुदा पर सच के सिवा (झूठ कभी) नहीं कहेगें और जो कुछ उस किताब में है उन्होनें (अच्छी तरह) पढ़ लिया है और आख़िर का घर तो उन्हीं लोगों के वास्ते ख़ास है जो परहेज़गार हैं तो क्या तुम (इतना भी नहीं समझते)
\end{hindi}}
\flushright{\begin{Arabic}
\quranayah[7][170]
\end{Arabic}}
\flushleft{\begin{hindi}
और जो लोग किताब (ख़ुदा) को मज़बूती से पकड़े हुए हैं और पाबन्दी से नमाज़ अदा करते हैं (उन्हें उसका सवाब ज़रूर मिलेगा क्योंकि) हम हरगिज़ नेकोकारो का सवाब बरबाद नहीं करते
\end{hindi}}
\flushright{\begin{Arabic}
\quranayah[7][171]
\end{Arabic}}
\flushleft{\begin{hindi}
तो (ऐ रसूल यहूद को याद दिलाओ) जब हम ने उन (के सरों) पर पहाड़ को इस तरह लटका दिया कि गोया साएबान (छप्पर) था और वह लोग समझ चुके थे कि उन पर अब गिरा और हमने उनको हुक्म दिया कि जो कुछ हमने तुम्हें अता किया है उसे मज़बूती से पकड़ लो और जो कुछ उसमें लिखा है याद रखो ताकि तुम परहेज़गार बन जाओ
\end{hindi}}
\flushright{\begin{Arabic}
\quranayah[7][172]
\end{Arabic}}
\flushleft{\begin{hindi}
और उसे रसूल वह वक्त भी याद (दिलाओ) जब तुम्हारे परवरदिगार ने आदम की औलाद से बस्तियों से (बाहर निकाल कर) उनकी औलाद से खुद उनके मुक़ाबले में एक़रार कर लिया (पूछा) कि क्या मैं तुम्हारा परवरदिगार नहीं हूँ तो सब के सब बोले हाँ हम उसके गवाह हैं ये हमने इसलिए कहा कि ऐसा न हो कहीं तुम क़यामत के दिन बोल उठो कि हम तो उससे बिल्कुल बे ख़बर थे
\end{hindi}}
\flushright{\begin{Arabic}
\quranayah[7][173]
\end{Arabic}}
\flushleft{\begin{hindi}
या ये कह बैठो कि (हम क्या करें) हमारे तो बाप दादाओं ही ने पहले शिर्क किया था और हम तो उनकी औलाद थे (कि) उनके बाद दुनिया में आए तो क्या हमें उन लोगों के ज़ुर्म की सज़ा में हलाक करेगा जो पहले ही बातिल कर चुके
\end{hindi}}
\flushright{\begin{Arabic}
\quranayah[7][174]
\end{Arabic}}
\flushleft{\begin{hindi}
और हम यूँ अपनी आयतों को तफसीलदार बयान करते हैं और ताकि वह लोग (अपनी ग़लती से) बाज़ आएं
\end{hindi}}
\flushright{\begin{Arabic}
\quranayah[7][175]
\end{Arabic}}
\flushleft{\begin{hindi}
और (ऐ रसूल) तुम उन लोगों को उस शख़्श का हाल पढ़ कर सुना दो जिसे हमने अपनी आयतें अता की थी फिर वह उनसे निकल भागा तो शैतान ने उसका पीछा पकड़ा और आख़िरकार वह गुमराह हो गया
\end{hindi}}
\flushright{\begin{Arabic}
\quranayah[7][176]
\end{Arabic}}
\flushleft{\begin{hindi}
और अगर हम चाहें तो हम उसे उन्हें आयतों की बदौलत बुलन्द मरतबा कर देते मगर वह तो ख़ुद ही पस्ती (नीचे) की तरफ झुक पड़ा और अपनी नफसानी ख्वाहिश का ताबेदार बन बैठा तो उसकी मसल है कि अगर उसको धुत्कार दो तो भी ज़बान निकाले रहे और उसको छोड़ दो तो भी ज़बान निकले रहे ये मसल उन लोगों की है जिन्होंने हमारी आयतों को झुठलाया तो (ऐ रसूल) ये क़िस्से उन लोगों से बयान कर दो ताकि ये लोग खुद भी ग़ौर करें
\end{hindi}}
\flushright{\begin{Arabic}
\quranayah[7][177]
\end{Arabic}}
\flushleft{\begin{hindi}
जिन लोगों ने हमारी आयतों को झुठलाया है उनकी भी क्या बुरी मसल है और अपनी ही जानों पर सितम ढाते रहे
\end{hindi}}
\flushright{\begin{Arabic}
\quranayah[7][178]
\end{Arabic}}
\flushleft{\begin{hindi}
राह पर बस वही शख़्श है जिसकी ख़ुदा हिदायत करे और जिनको गुमराही में छोड़ दे तो वही लोग घाटे में हैं
\end{hindi}}
\flushright{\begin{Arabic}
\quranayah[7][179]
\end{Arabic}}
\flushleft{\begin{hindi}
और गोया हमने (ख़ुदा) बहुतेरे जिन्नात और आदमियों को जहन्नुम के वास्ते पैदा किया और उनके दिल तो हैं (मगर कसदन) उन से देखते ही नहीं और उनके कान भी है (मगर) उनसे सुनने का काम ही नहीं लेते (खुलासा) ये लोग गोया जानवर हैं बल्कि उनसे भी कहीं गए गुज़रे हुए यही लोग (अमूर हक़) से बिल्कुल बेख़बर हैं
\end{hindi}}
\flushright{\begin{Arabic}
\quranayah[7][180]
\end{Arabic}}
\flushleft{\begin{hindi}
और अच्छे (अच्छे) नाम ख़ुदा ही के ख़ास हैं तो उसे उन्हीं नामों में पुकारो और जो लोग उसके नामों में कुफ्र करते हैं उन्हें (उनके हाल पर) छोड़ दो और वह बहुत जल्द अपने करतूत की सज़ाएं पाएंगें
\end{hindi}}
\flushright{\begin{Arabic}
\quranayah[7][181]
\end{Arabic}}
\flushleft{\begin{hindi}
और हमारी मख़लूक़ात से कुछ लोग ऐसे भी हैं जो दीने हक़ की हिदायत करते हैं और हक़ से इन्साफ़ भी करते हैं
\end{hindi}}
\flushright{\begin{Arabic}
\quranayah[7][182]
\end{Arabic}}
\flushleft{\begin{hindi}
और जिन लोगों ने हमारी आयतों को झुठलाया हम उन्हें बहुत जल्द इस तरह आहिस्ता आहिस्ता (जहन्नुम में) ले जाएंगें कि उन्हें ख़बर भी न होगी
\end{hindi}}
\flushright{\begin{Arabic}
\quranayah[7][183]
\end{Arabic}}
\flushleft{\begin{hindi}
और मैं उन्हें (दुनिया में) ढील दूंगा बेशक मेरी तद्बीर (पुख्ता और) मज़बूत है
\end{hindi}}
\flushright{\begin{Arabic}
\quranayah[7][184]
\end{Arabic}}
\flushleft{\begin{hindi}
क्या उन लोगों ने इतना भी ख्याल न किया कि आख़िर उनके रफीक़ (मोहम्मद ) को कुछ जुनून तो नहीं वह तो बस खुल्लम खुल्ला (अज़ाबे ख़ुदा से) डराने वाले हैं
\end{hindi}}
\flushright{\begin{Arabic}
\quranayah[7][185]
\end{Arabic}}
\flushleft{\begin{hindi}
क्या उन लोगों ने आसमान व ज़मीन की हुकूमत और ख़ुदा की पैदा की हुई चीज़ों में ग़ौर नहीं किया और न इस बात में कि यायद उनकी मौत क़रीब आ गई हो फिर इतना समझाने के बाद (भला) किस बात पर ईमान लाएंगें
\end{hindi}}
\flushright{\begin{Arabic}
\quranayah[7][186]
\end{Arabic}}
\flushleft{\begin{hindi}
जिसे ख़ुदा गुमराही में छोड़ दे फिर उसका कोई राहबर नहीं और उन्हीं की सरकशी (व शरारत) में छोड़ देगा कि सरगरदॉ रहें
\end{hindi}}
\flushright{\begin{Arabic}
\quranayah[7][187]
\end{Arabic}}
\flushleft{\begin{hindi}
(ऐ रसूल) तुमसे लोग क़यामत के बारे में पूछा करते हैं कि कहीं उसका कोई वक्त भी तय है तुम कह दो कि उसका इल्म बस फक़त पररवदिगार ही को है वही उसके वक्त मुअय्यन पर उसको ज़ाहिर कर देगा। वह सारे आसमान व ज़मीन में एक कठिन वक्त होगा वह तुम्हारे पास पस अचानक आ जाएगी तुमसे लोग इस तरह पूछते हैं गोया तुम उनसे बखूबी वाक़िफ हो तुम (साफ) कह दो कि उसका इल्म बस ख़ुदा ही को है मगर बहुतेरे लोग नहीं जानते
\end{hindi}}
\flushright{\begin{Arabic}
\quranayah[7][188]
\end{Arabic}}
\flushleft{\begin{hindi}
(ऐ रसूल) तुम कह दो कि मै ख़ुद अपना आप तो एख़तियार रखता ही नहीं न नफे क़ा न ज़रर का मगर बस वही ख़ुदा जो चाहे और अगर (बग़ैर ख़ुदा के बताए) गैब को जानता होता तो यक़ीनन मै अपना बहुत सा फ़ायदा कर लेता और मुझे कभी कोई तकलीफ़ भी न पहुँचती मै तो सिर्फ ईमानदारों को (अज़ाब से डराने वाला) और वेहशत की खुशख़बरी देने वाला हँ
\end{hindi}}
\flushright{\begin{Arabic}
\quranayah[7][189]
\end{Arabic}}
\flushleft{\begin{hindi}
वह ख़ुदा ही तो है जिसने तुमको एक शख़्श (आदम) से पैदा किया और उसकी बची हुई मिट्टी से उसका जोड़ा भी बना डाला ताकि उसके साथ रहे सहे फिर जब इन्सान अपनी बीबी से हम बिस्तरी करता है तो बीबी एक हलके से हमल से हमला हो जाती है फिर उसे लिए चलती फिरती है फिर जब वह (ज्यादा दिन होने से बोझल हो जाती है तो दोनो (मिया बीबी) अपने परवरदिगार ख़ुदा से दुआ करने लगे कि अगर तो हमें नेक फरज़न्द अता फरमा तो हम ज़रूर तेरे शुक्रगुज़ार होगें
\end{hindi}}
\flushright{\begin{Arabic}
\quranayah[7][190]
\end{Arabic}}
\flushleft{\begin{hindi}
फिर जब ख़ुदा ने उनको नेक (फरज़न्द) अता फ़रमा दिया तो जो (औलाद) ख़ुदा ने उन्हें अता किया था लगे उसमें ख़ुदा का शरीक बनाने तो ख़ुदा (की शान) शिर्क से बहुत ऊँची है
\end{hindi}}
\flushright{\begin{Arabic}
\quranayah[7][191]
\end{Arabic}}
\flushleft{\begin{hindi}
क्या वह लोग ख़ुदा का शरीक ऐसों को बनाते हैं जो कुछ भी पैदा नहीं कर सकते बल्कि वह ख़ुद (ख़ुदा के) पैदा किए हुए हैं
\end{hindi}}
\flushright{\begin{Arabic}
\quranayah[7][192]
\end{Arabic}}
\flushleft{\begin{hindi}
और न उनकी मदद की कुदरत रखते हैं और न आप अपनी मदद कर सकते हैं
\end{hindi}}
\flushright{\begin{Arabic}
\quranayah[7][193]
\end{Arabic}}
\flushleft{\begin{hindi}
और अगर तुम उन्हें हिदायत की तरफ बुलाओंगे भी तो ये तुम्हारी पैरवी नहीं करने के तुम्हारे वास्ते बराबर है ख्वाह (चाहे) तुम उनको बुलाओ या तुम चुपचाप बैठे रहो
\end{hindi}}
\flushright{\begin{Arabic}
\quranayah[7][194]
\end{Arabic}}
\flushleft{\begin{hindi}
बेशक वह लोग जिनकी तुम ख़ुदा को छोड़कर हाजत करते हो वह (भी) तुम्हारी तरह (ख़ुदा के) बन्दे हैं भला तुम उन्हें पुकार के देखो तो अगर तुम सच्चे हो तो वह तुम्हारी कुछ सुन लें
\end{hindi}}
\flushright{\begin{Arabic}
\quranayah[7][195]
\end{Arabic}}
\flushleft{\begin{hindi}
क्या उनके ऐसे पॉव भी हैं जिनसे चल सकें या उनके ऐसे हाथ भी हैं जिनसे (किसी चीज़ को) पकड़ सके या उनकी ऐसी ऑखे भी है जिनसे देख सकें या उनके ऐसे कान हैं जिनसे सुन सकें (ऐ रसूल उन लोगों से) कह दो कि तुम अपने बनाए हुए शरीको को बुलाओ फिर सब मिलकर मुझ पर दॉव चले फिर (मुझे) मोहलत न दो
\end{hindi}}
\flushright{\begin{Arabic}
\quranayah[7][196]
\end{Arabic}}
\flushleft{\begin{hindi}
(फिर देखो मेरा क्या बना सकते हो) बेशक मेरा मालिक व मुमताज़ तो बस ख़ुदा है जिस ने किताब क़ुरान को नाज़िल फरमाया और वही (अपने) नेक बन्दों का हाली (मददगार) है
\end{hindi}}
\flushright{\begin{Arabic}
\quranayah[7][197]
\end{Arabic}}
\flushleft{\begin{hindi}
और वह लोग (बुत) जिन्हें तुम ख़ुदा के सिवा (अपनी मदद को) पुकारते हो न तो वह तुम्हारी मदद की कुदरत रखते हैं और न ही अपनी मदद कर सकते हैं
\end{hindi}}
\flushright{\begin{Arabic}
\quranayah[7][198]
\end{Arabic}}
\flushleft{\begin{hindi}
और अगर उन्हें हिदायत की तरफ बुलाएगा भी तो ये सुन ही नहीं सकते और तू तो समझता है कि वह तुझे (ऑखें खोले) देख रहे हैं हालॉकि वह देखते नहीं
\end{hindi}}
\flushright{\begin{Arabic}
\quranayah[7][199]
\end{Arabic}}
\flushleft{\begin{hindi}
(ऐ रसूल) तुम दरगुज़र करना एख्तियार करो और अच्छे काम का हुक्म दो और जाहिलों की तरफ से मुह फेर लो
\end{hindi}}
\flushright{\begin{Arabic}
\quranayah[7][200]
\end{Arabic}}
\flushleft{\begin{hindi}
अगर शैतान की तरफ से तुम्हारी (उम्मत के) दिल में किसी तरह का (वसवसा (शक) पैदा हो तो ख़ुदा से पनाह मॉगों (क्योंकि) उसमें तो शक़ ही नहीं कि वह बड़ा सुनने वाला वाक़िफकार है
\end{hindi}}
\flushright{\begin{Arabic}
\quranayah[7][201]
\end{Arabic}}
\flushleft{\begin{hindi}
बेशक लोग परहेज़गार हैं जब भी उन्हें शैतान का ख्याल छू भी गया तो चौक पड़ते हैं फिर फौरन उनकी ऑखें खुल जाती हैं
\end{hindi}}
\flushright{\begin{Arabic}
\quranayah[7][202]
\end{Arabic}}
\flushleft{\begin{hindi}
उन काफिरों के भाई बन्द शैतान उनको (धर पकड़) गुमराही के तरफ घसीटे जाते हैं फिर किसी तरह की कोताही (भी) नहीं करते
\end{hindi}}
\flushright{\begin{Arabic}
\quranayah[7][203]
\end{Arabic}}
\flushleft{\begin{hindi}
और जब तुम उनके पास कोई (ख़ास) मौजिज़ा नहीं लाते तो कहते हैं कि तुमने उसे क्यों नहीं बना लिया (ऐ रसूल) तुम कह दो कि मै तो बस इसी वही का पाबन्द हूँ जो मेरे परवरदिगार की तरफ से मेरे पास आती है ये (क़ुरान) तुम्हारे परवरदिगार की तरफ से (हक़ीकत) की दलीलें हैं
\end{hindi}}
\flushright{\begin{Arabic}
\quranayah[7][204]
\end{Arabic}}
\flushleft{\begin{hindi}
और ईमानदार लोगों के वास्ते हिदायत और रहमत हैं (लोगों) जब क़ुरान पढ़ा जाए तो कान लगाकर सुनो और चुपचाप रहो ताकि (इसी बहाने) तुम पर रहम किया जाए
\end{hindi}}
\flushright{\begin{Arabic}
\quranayah[7][205]
\end{Arabic}}
\flushleft{\begin{hindi}
और अपने परवरदिगार को अपने जी ही में गिड़गिड़ा के और डर के और बहुत चीख़ के नहीं (धीमी) आवाज़ से सुबह व शाम याद किया करो और (उसकी याद से) ग़ाफिल बिल्कुल न हो जाओ
\end{hindi}}
\flushright{\begin{Arabic}
\quranayah[7][206]
\end{Arabic}}
\flushleft{\begin{hindi}
बेशक जो लोग (फरिशते बग़ैरह) तुम्हारे परवरदिगार के पास मुक़र्रिब हैं और वह उसकी इबादत से सर कशी नही करते और उनकी तसबीह करते हैं और उसका सजदा करते हैं (206) सजदा
\end{hindi}}
\chapter{Al-Anfal (Voluntary Gifts)}
\begin{Arabic}
\Huge{\centerline{\basmalah}}\end{Arabic}
\flushright{\begin{Arabic}
\quranayah[8][1]
\end{Arabic}}
\flushleft{\begin{hindi}
(ऐ रसूल) तुम से लोग अनफाल (माले ग़नीमत) के बारे में पूछा करते हैं तुम कह दो कि अनफाल मख़सूस ख़ुदा और रसूल के वास्ते है तो ख़ुदा से डरो (और) अपने बाहमी (आपसी) मामलात की इसलाह करो और अगर तुम सच्चे (ईमानदार) हो तो ख़ुदा की और उसके रसूल की इताअत करो
\end{hindi}}
\flushright{\begin{Arabic}
\quranayah[8][2]
\end{Arabic}}
\flushleft{\begin{hindi}
सच्चे ईमानदार तो बस वही लोग हैं कि जब (उनके सामने) ख़ुदा का ज़िक्र किया जाता है तो उनके दिल हिल जाते हैं और जब उनके सामने उसकी आयतें पढ़ी जाती हैं तो उनके ईमान को और भी ज्यादा कर देती हैं और वह लोग बस अपने परवरदिगार ही पर भरोसा रखते हैं
\end{hindi}}
\flushright{\begin{Arabic}
\quranayah[8][3]
\end{Arabic}}
\flushleft{\begin{hindi}
नमाज़ को पाबन्दी से अदा करते हैं और जो हम ने उन्हें दिया हैं उसमें से (राहे ख़ुदा में) ख़र्च करते हैं
\end{hindi}}
\flushright{\begin{Arabic}
\quranayah[8][4]
\end{Arabic}}
\flushleft{\begin{hindi}
यही तो सच्चे ईमानदार हैं उन्हीं के लिए उनके परवरदिगार के हॉ (बड़े बड़े) दरजे हैं और बख्शिश और इज्ज़त और आबरू के साथ रोज़ी है (ये माले ग़नीमत का झगड़ा वैसा ही है)
\end{hindi}}
\flushright{\begin{Arabic}
\quranayah[8][5]
\end{Arabic}}
\flushleft{\begin{hindi}
जिस तरह तुम्हारे परवरदिगार ने तुम्हें बिल्कुल ठीक (मसलहत से) तुम्हारे घर से (जंग बदर) में निकाला था और मोमिनीन का एक गिरोह (उससे) नाखुश था
\end{hindi}}
\flushright{\begin{Arabic}
\quranayah[8][6]
\end{Arabic}}
\flushleft{\begin{hindi}
कि वह लोग हक़ के ज़ाहिर होने के बाद भी तुमसे (ख्वाह माख्वाह) सच्ची बात में झगड़तें थें और इस तरह (करने लगे) गोया (ज़बरदस्ती) मौत के मुँह में ढकेले जा रहे हैं
\end{hindi}}
\flushright{\begin{Arabic}
\quranayah[8][7]
\end{Arabic}}
\flushleft{\begin{hindi}
और उसे (अपनी ऑंखों से) देख रहे हैं और (ये वक्त था) जब ख़ुदा तुमसे वायदा कर रहा था कि (कुफ्फार मक्का) दो जमाअतों में से एक तुम्हारे लिए ज़रूरी हैं और तुम ये चाहते थे कि कमज़ोर जमाअत तुम्हारे हाथ लगे (ताकि बग़ैर लड़े भिड़े माले ग़नीमत हाथ आ जाए) और ख़ुदा ये चाहता था कि अपनी बातों से हक़ को साबित (क़दम) करें और काफिरों की जड़ काट डाले
\end{hindi}}
\flushright{\begin{Arabic}
\quranayah[8][8]
\end{Arabic}}
\flushleft{\begin{hindi}
ताकि हक़ को (हक़) साबित कर दे और बातिल का मटियामेट कर दे अगर चे गुनाहगार (कुफ्फार उससे) नाखुश ही क्यों न हो
\end{hindi}}
\flushright{\begin{Arabic}
\quranayah[8][9]
\end{Arabic}}
\flushleft{\begin{hindi}
(ये वह वक्त था) जब तुम अपने परवदिगार से फरियाद कर रहे थे उसने तुम्हारी सुन ली और जवाब दे दिया कि मैं तुम्हारी लगातार हज़ार फ़रिश्तों से मदद करूँगा
\end{hindi}}
\flushright{\begin{Arabic}
\quranayah[8][10]
\end{Arabic}}
\flushleft{\begin{hindi}
और (ये इमदाद ग़ैबी) ख़ुदा ने सिर्फ तुम्हारी ख़ातिर (खुशी) के लिए की थी और तुम्हारे दिल मुतमइन हो जाएं और (याद रखो) मदद ख़ुदा के सिवा और कहीं से (कभी) नहीं होती बेशक ख़ुदा ग़ालिब हिकमत वाला है
\end{hindi}}
\flushright{\begin{Arabic}
\quranayah[8][11]
\end{Arabic}}
\flushleft{\begin{hindi}
ये वह वक्त था जब अपनी तरफ से इत्मिनान देने के लिए तुम पर नींद को ग़ालिब कर रहा था और तुम पर आसमान से पानी बरस रहा था ताकि उससे तुम्हें पाक (पाकीज़ा कर दे और तुम से शैतान की गन्दगी दूर कर दे और तुम्हारे दिल मज़बूत कर दे और पानी से (बालू जम जाए) और तुम्हारे क़दम ब क़दम (अच्छी तरह) जमाए रहे
\end{hindi}}
\flushright{\begin{Arabic}
\quranayah[8][12]
\end{Arabic}}
\flushleft{\begin{hindi}
(ऐ रसूल ये वह वक्त था) जब तुम्हारा परवरदिगार फ़रिश्तों से फरमा रहा था कि मै यकीनन तुम्हारे साथ हूँ तुम ईमानदारों को साबित क़दम रखो मै बहुत जल्द काफिरों के दिलों में (तुम्हारा रौब) डाल दूँगा (पस फिर क्या है अब) तो उन कुफ्फार की गर्दनों पर मारो और उनकी पोर पोर को चटिया कर दो
\end{hindi}}
\flushright{\begin{Arabic}
\quranayah[8][13]
\end{Arabic}}
\flushleft{\begin{hindi}
ये (सज़ा) इसलिए है कि उन लोगों ने ख़ुदा और उसके रसूल की मुख़ालिफ की और जो शख़्स (भी) ख़ुदा और उसके रसूल की मुख़ालफ़त करेगा तो (याद रहें कि) ख़ुदा बड़ा सख्त अज़ाब करने वाला है
\end{hindi}}
\flushright{\begin{Arabic}
\quranayah[8][14]
\end{Arabic}}
\flushleft{\begin{hindi}
(काफिरों दुनिया में तो) लो फिर उस (सज़ा का चखो और (फिर आख़िर में तो) काफिरों के वास्ते जहन्नुम का अज़ाब ही है
\end{hindi}}
\flushright{\begin{Arabic}
\quranayah[8][15]
\end{Arabic}}
\flushleft{\begin{hindi}
ऐ ईमानदारों जब तुमसे कुफ्फ़ार से मैदाने जंग में मुक़ाबला हुआ तो (ख़बरदार) उनकी तरफ पीठ न करना
\end{hindi}}
\flushright{\begin{Arabic}
\quranayah[8][16]
\end{Arabic}}
\flushleft{\begin{hindi}
(याद रहे कि) उस शख़्स के सिवा जो लड़ाई वास्ते कतराए या किसी जमाअत के पास (जाकर) मौके पाए (और) जो शख़्स भी उस दिन उन कुफ्फ़ार की तरफ पीठ फेरेगा वह यक़ीनी (हिर फिर के) ख़ुदा के ग़जब में आ गया और उसका ठिकाना जहन्नुम ही हैं और वह क्या बुरा ठिकाना है
\end{hindi}}
\flushright{\begin{Arabic}
\quranayah[8][17]
\end{Arabic}}
\flushleft{\begin{hindi}
और (मुसलमानों) उन कुफ्फ़ार को कुछ तुमने तो क़त्ल किया नही बल्कि उनको तो ख़ुदा ने क़त्ल किया और (ऐ रसूल) जब तुमने तीर मारा तो कुछ तुमने नही मारा बल्कि ख़ुदा ख़ुदा ने तीर मारा और ताकि अपनी तरफ से मोमिनीन पर खूब एहसान करे बेशक ख़ुदा (सबकी) सुनता और (सब कुछ) जानता है
\end{hindi}}
\flushright{\begin{Arabic}
\quranayah[8][18]
\end{Arabic}}
\flushleft{\begin{hindi}
ये तो ये ख़ुदा तो काफिरों की मक्कारी का कमज़ोर कर देने वाला है
\end{hindi}}
\flushright{\begin{Arabic}
\quranayah[8][19]
\end{Arabic}}
\flushleft{\begin{hindi}
(काफ़िर) अगर तुम ये चाहते हो (कि जो हक़ पर हो उसकी) फ़तेह हो (मुसलमानों की) फ़तेह भी तुम्हारे सामने आ मौजूद हुई अब क्या गुरूर बाक़ी है और अगर तुम (अब भी मुख़तलिफ़ इस्लाम) से बाज़ रहो तो तुम्हारे वास्ते बेहतर है और अगर कहीं तुम पलट पड़े तो (याद रहे) हम भी पलट पड़ेगें (और तुम्हें तबाह कर छोड़ देगें) और तुम्हारी जमाअत अगरचे बहुत ज्यादा भी हो हरगिज़ कुछ काम न आएगी और ख़ुदा तो यक़ीनी मामिनीन के साथ है
\end{hindi}}
\flushright{\begin{Arabic}
\quranayah[8][20]
\end{Arabic}}
\flushleft{\begin{hindi}
(ऐ ईमानदारों खुदा और उसके रसूल की इताअत करो और उससे मुँह न मोड़ो जब तुम समझ रहे हो
\end{hindi}}
\flushright{\begin{Arabic}
\quranayah[8][21]
\end{Arabic}}
\flushleft{\begin{hindi}
और उन लोगों के ऐसे न हो जाओं जो (मुँह से तो) कहते थे कि हम सुन रहे हैं हालाकि वह सुनते (सुनाते ख़ाक) न थे
\end{hindi}}
\flushright{\begin{Arabic}
\quranayah[8][22]
\end{Arabic}}
\flushleft{\begin{hindi}
इसमें शक़ नहीं कि ज़मीन पर चलने वाले तमाम हैवानात से बदतर ख़ुदा के नज़दीक वह बहरे गूँगे (कुफ्फार) हैं जो कुछ नहीं समझते
\end{hindi}}
\flushright{\begin{Arabic}
\quranayah[8][23]
\end{Arabic}}
\flushleft{\begin{hindi}
और अगर ख़ुदा उनमें नेकी (की बू भी) देखता तो ज़रूर उनमें सुनने की क़ाबलियत अता करता मगर ये ऐसे हैं कि अगर उनमें सुनने की क़ाबिलयत भी देता तो मुँह फेर कर भागते।
\end{hindi}}
\flushright{\begin{Arabic}
\quranayah[8][24]
\end{Arabic}}
\flushleft{\begin{hindi}
ऐ ईमानदार जब तुम को हमारा रसूल (मोहम्मद) ऐसे काम के लिए बुलाए जो तुम्हारी रूहानी ज़िन्दगी का बाइस हो तो तुम ख़ुदा और रसूल के हुक्म दिल से कुबूल कर लो और जान लो कि ख़ुदा वह क़ादिर मुतलिक़ है कि आदमी और उसके दिल (इरादे) के दरमियान इस तरह आ जाता है और ये भी समझ लो कि तुम सबके सब उसके सामने हाज़िर किये जाओगे
\end{hindi}}
\flushright{\begin{Arabic}
\quranayah[8][25]
\end{Arabic}}
\flushleft{\begin{hindi}
और उस फितने से डरते रहो जो ख़ास उन्हीं लोगों पर नही पड़ेगा जिन्होने तुम में से ज़ुल्म किया (बल्कि तुम सबके सब उसमें पड़ जाओगे) और यक़ीन मानों कि ख़ुदा बड़ा सख्त अज़ाब करने वाला है
\end{hindi}}
\flushright{\begin{Arabic}
\quranayah[8][26]
\end{Arabic}}
\flushleft{\begin{hindi}
(मुसलमानों वह वक्त याद करो) जब तुम सर ज़मीन (मक्के) में बहुत कम और बिल्कुल बेबस थे उससे सहमे जाते थे कि कहीं लोग तुमको उचक न ले जाए तो ख़ुदा ने तुमको (मदीने में) पनाह दी और ख़ास अपनी मदद से तुम्हारी ताईद की और तुम्हे पाक व पाकीज़ा चीज़े खाने को दी ताकि तुम शुक्र गुज़ारी करो
\end{hindi}}
\flushright{\begin{Arabic}
\quranayah[8][27]
\end{Arabic}}
\flushleft{\begin{hindi}
ऐ ईमानदारों न तो ख़ुदा और रसूल की (अमानत में) ख्यानत करो और न अपनी अमानतों में ख्यानत करो हालॉकि समझते बूझते हो
\end{hindi}}
\flushright{\begin{Arabic}
\quranayah[8][28]
\end{Arabic}}
\flushleft{\begin{hindi}
और यक़ीन जानों कि तुम्हारे माल और तुम्हारी औलाद तुम्हारी आज़माइश (इम्तेहान) की चीज़े हैं कि जो उनकी मोहब्बत में भी ख़ुदा को न भूले और वह दीनदार है
\end{hindi}}
\flushright{\begin{Arabic}
\quranayah[8][29]
\end{Arabic}}
\flushleft{\begin{hindi}
और यक़ीनन ख़ुदा के हॉ बड़ी मज़दूरी है ऐ ईमानदारों अगर तुम ख़ुदा से डरते रहोगे तो वह तुम्हारे वास्ते इम्तियाज़ पैदा करे देगा और तुम्हारी तरफ से तुम्हारे गुनाह का कफ्फ़ारा क़रार देगा और तुम्हें बख्श देगा और ख़ुदा बड़ा साहब फज़ल (व करम) है
\end{hindi}}
\flushright{\begin{Arabic}
\quranayah[8][30]
\end{Arabic}}
\flushleft{\begin{hindi}
और (ऐ रसूल वह वक्त याद करो) जब कुफ्फ़ार तुम से फरेब कर रहे थे ताकि तुमको क़ैद कर लें या तुमको मार डाले तुम्हें (घर से) निकाल बाहर करे वह तो ये तदबीर (चालाकी) कर रहे थे और ख़ुदा भी (उनके ख़िलाफ) तदबीरकर रहा था
\end{hindi}}
\flushright{\begin{Arabic}
\quranayah[8][31]
\end{Arabic}}
\flushleft{\begin{hindi}
और ख़ुदा तो सब तदबीरकरने वालों से बेहतर है और जब उनके सामने हमारी आयते पढ़ी जाती हैं तो बोल उठते हैं कि हमने सुना तो लेकिन अगर हम चाहें तो यक़ीनन ऐसा ही (क़रार) हम भी कह सकते हैं-तो बस अगलों के क़िस्से है
\end{hindi}}
\flushright{\begin{Arabic}
\quranayah[8][32]
\end{Arabic}}
\flushleft{\begin{hindi}
और (ऐ रसूल वह वक्त याद करो) जब उन काफिरों ने दुआएँ माँगीं थी कि ख़ुदा (वन्द) अगर ये (दीन इस्लाम) हक़ है और तेरे पास से (आया है) तो हम पर आसमान से पत्थर बरसा या हम पर कोई और दर्दनाक अज़ाब ही नाज़िल फरमा
\end{hindi}}
\flushright{\begin{Arabic}
\quranayah[8][33]
\end{Arabic}}
\flushleft{\begin{hindi}
हालॉकि जब तक तुम उनके दरमियान मौजूद हो ख़ुदा उन पर अज़ाब नहीं करेगा और अल्लाह ऐसा भी नही कि लोग तो उससे अपने गुनाहो की माफी माँग रहे हैं और ख़ुदा उन पर नाज़िल फरमाए
\end{hindi}}
\flushright{\begin{Arabic}
\quranayah[8][34]
\end{Arabic}}
\flushleft{\begin{hindi}
और जब ये लोग लोगों को मस्जिदुल हराम (ख़ान ए काबा की इबादत) से रोकते है तो फिर उनके लिए कौन सी बात बाक़ी है कि उन पर अज़ाब न नाज़िल करे और ये लोग तो ख़ानाए काबा के मुतवल्ली भी नहीं (फिर क्यों रोकते है) इसके मुतवल्ली तो सिर्फ परहेज़गार लोग हैं मगर इन काफ़िरों में से बहुतेरे नहीं जानते
\end{hindi}}
\flushright{\begin{Arabic}
\quranayah[8][35]
\end{Arabic}}
\flushleft{\begin{hindi}
और ख़ानाए काबे के पास सीटियॉ तालिया बजाने के सिवा उनकी नमाज ही क्या थी तो (ऐ काफिरों) जब तुम कुफ्र किया करते थे उसकी सज़ा में (पड़े) अज़ाब के मज़े चखो
\end{hindi}}
\flushright{\begin{Arabic}
\quranayah[8][36]
\end{Arabic}}
\flushleft{\begin{hindi}
इसमें शक़ नहीं कि ये कुफ्फार अपने माल महज़ इस वास्ते खर्च करेगें फिर उसके बाद उनकी हसरत का बाइस होगा फिर आख़िर ये लोग हार जाएँगें और जिन लोगों ने कुफ्र एख्तियार किया (क़यामत में) सब के सब जहन्नुम की तरफ हाके जाएँगें
\end{hindi}}
\flushright{\begin{Arabic}
\quranayah[8][37]
\end{Arabic}}
\flushleft{\begin{hindi}
ताकि ख़ुदा पाक को नापाक से जुदा कर दे और नापाक लोगों को एक दूसरे पर रखके ढेर बनाए फिर सब को जहन्नुम में झोंक दे यही लोग घाटा उठाने वाले हैं
\end{hindi}}
\flushright{\begin{Arabic}
\quranayah[8][38]
\end{Arabic}}
\flushleft{\begin{hindi}
(ऐ रसूल) तुम काफिरों से कह दो कि अगर वह लोग (अब भी अपनी शरारत से) बाज़ रहें तो उनके पिछले कुसूर माफ कर दिए जाएं और अगर फिर कहीं पलटें तो यक़ीनन अगलों के तरीक़े गुज़र चुके जो, उनकी सज़ा हुई वही इनकी भी होगी
\end{hindi}}
\flushright{\begin{Arabic}
\quranayah[8][39]
\end{Arabic}}
\flushleft{\begin{hindi}
मुसलमानों काफ़िरों से लड़े जाओ यहाँ तक कि कोई फसाद (बाक़ी) न रहे और (बिल्कुल सारी ख़ुदाई में) ख़ुदा की दीन ही दीन हो जाए फिर अगर ये लोग (फ़साद से) न बाज़ आएं तो ख़ुदा उनकी कारवाइयों को ख़ूब देखता है
\end{hindi}}
\flushright{\begin{Arabic}
\quranayah[8][40]
\end{Arabic}}
\flushleft{\begin{hindi}
और अगर सरताबी करें तो (मुसलमानों) समझ लो कि ख़ुदा यक़ीनी तुम्हारा मालिक है और वह क्या अच्छा मालिक है और क्या अच्छा मददगार है
\end{hindi}}
\flushright{\begin{Arabic}
\quranayah[8][41]
\end{Arabic}}
\flushleft{\begin{hindi}
और जान लो कि जो कुछ तुम (माल लड़कर) लूटो तो उनमें का पॉचवॉ हिस्सा मख़सूस ख़ुदा और रसूल और (रसूल के) क़राबतदारों और यतीमों और मिस्कीनों और परदेसियों का है अगर तुम ख़ुदा पर और उस (ग़ैबी इमदाद) पर ईमान ला चुके हो जो हमने ख़ास बन्दे (मोहम्मद) पर फ़ैसले के दिन (जंग बदर में) नाज़िल की थी जिस दिन (मुसलमानों और काफिरों की) दो जमाअतें बाहम गुथ गयी थी और ख़ुदा तो हर चीज़ पर क़ादिर है
\end{hindi}}
\flushright{\begin{Arabic}
\quranayah[8][42]
\end{Arabic}}
\flushleft{\begin{hindi}
(ये वह वक्त था) जब तुम (मैदाने जंग में मदीने के) क़रीब नाके पर थे और वह कुफ्फ़ार बईद (दूर के) के नाके पर और (काफ़िले के) सवार तुम से नशेब में थे और अगर तुम एक दूसरे से (वक्त क़ी तक़रीर का) वायदा कर लेते हो तो और वक्त पर गड़बड़ कर देते मगर (ख़ुदा ने अचानक तुम लोगों को इकट्ठा कर दिया ताकि जो बात यदनी (होनी) थी वह पूरी कर दिखाए ताकि जो शख़्स हलाक (गुमराह) हो वह (हक़ की) हुज्जत तमाम होने के बाद हलाक हो और जो ज़िन्दा रहे वह हिदायत की हुज्जत तमाम होने के बाद ज़िन्दा रहे और ख़ुदा यक़ीनी सुनने वाला ख़बरदार है
\end{hindi}}
\flushright{\begin{Arabic}
\quranayah[8][43]
\end{Arabic}}
\flushleft{\begin{hindi}
(ये वह वक्त था) जब ख़ुदा ने तुम्हें ख्वाब में कुफ्फ़ार को कम करके दिखलाया था और अगर उनको तुम्हें ज्यादा करते दिखलाता तुम यक़ीनन हिम्मत हार देते और लड़ाई के बारे में झगड़ने लगते मगर ख़ुदा ने इसे (बदनामी) से बचाया इसमें तो शक़ ही नहीं कि वह दिली ख्यालात से वाक़िफ़ है
\end{hindi}}
\flushright{\begin{Arabic}
\quranayah[8][44]
\end{Arabic}}
\flushleft{\begin{hindi}
(ये वह वक्त था) जब तुम लोगों ने मुठभेड़ की तो ख़ुदा ने तुम्हारी ऑखों में कुफ्फ़ार को बहुत कम करके दिखलाया और उनकी ऑखों में तुमको थोड़ा कर दिया ताकि ख़ुदा को जो कुछ करना मंज़ूर था वह पूरा हो जाए और कुल बातों का दारोमदार तो ख़ुदा ही पर है
\end{hindi}}
\flushright{\begin{Arabic}
\quranayah[8][45]
\end{Arabic}}
\flushleft{\begin{hindi}
ऐ ईमानदारों जब तुम किसी फौज से मुठभेड़ करो तो ख़बरदार अपने क़दम जमाए रहो और ख़ुदा को बहुंत याद करते रहो ताकि तुम फलाह पाओ
\end{hindi}}
\flushright{\begin{Arabic}
\quranayah[8][46]
\end{Arabic}}
\flushleft{\begin{hindi}
और ख़ुदा की और उसके रसूल की इताअत करो और आपस में झगड़ा न करो वरना तुम हिम्मत हारोगे और तुम्हारी हवा उखड़ जाएगी और (जंग की तकलीफ़ को) झेल जाओ (क्योंकि) ख़ुदा तो यक़ीनन सब्र करने वालों का साथी है
\end{hindi}}
\flushright{\begin{Arabic}
\quranayah[8][47]
\end{Arabic}}
\flushleft{\begin{hindi}
और उन लोगों के ऐसे न हो जाओ जो इतराते हुए और लोगों के दिखलाने के वास्ते अपने घरों से निकल खड़े हुए और लोगों को ख़ुदा की राह से रोकते हैं और जो कुछ भी वह लोग करते हैं ख़ुदा उस पर (हर तरह से) अहाता किए हुए है
\end{hindi}}
\flushright{\begin{Arabic}
\quranayah[8][48]
\end{Arabic}}
\flushleft{\begin{hindi}
और जब शैतान ने उनकी कारस्तानियों को उम्दा कर दिखाया और उनके कान में फूंक दिया कि लोगों में आज कोई ऐसा नहीं जो तुम पर ग़ालिब आ सके और मै तुम्हारा मददगार हूं फिर जब दोनों लश्कर मुकाबिल हुए तो अपने उलटे पॉव भाग निकला और कहने लगा कि मै तो तुम से अलग हूं मै वह चीजें देख रहा हूं जो तुम्हें नहीं सूझती मैं तो ख़ुदा से डरता हूं और ख़ुदा बहुत सख्त अज़ाब वाला है
\end{hindi}}
\flushright{\begin{Arabic}
\quranayah[8][49]
\end{Arabic}}
\flushleft{\begin{hindi}
(ये वक्त था) जब मुनाफिक़ीन और जिन लोगों के दिल में (कुफ्र का) मर्ज़ है कह रहे थे कि उन मुसलमानों को उनके दीन ने धोके में डाल रखा है (कि इतराते फिरते हैं हालॉकि जो शख़्स ख़ुदा पर भरोसा करता है (वह ग़ालिब रहता है क्योंकि) ख़ुदा तो यक़ीनन ग़ालिब और हिकमत वाला है
\end{hindi}}
\flushright{\begin{Arabic}
\quranayah[8][50]
\end{Arabic}}
\flushleft{\begin{hindi}
और काश (ऐ रसूल) तुम देखते जब फ़रिश्ते काफ़िरों की जान निकाल लेते थे और रूख़ और पुश्त (पीठ) पर कोड़े मारते थे और (कहते थे कि) अज़ाब जहन्नुम के मज़े चखों
\end{hindi}}
\flushright{\begin{Arabic}
\quranayah[8][51]
\end{Arabic}}
\flushleft{\begin{hindi}
ये सज़ा उसकी है जो तुम्हारे हाथों ने पहले किया कराया है और ख़ुदा बन्दों पर हरगिज़ ज़ुल्म नहीं किया करता है
\end{hindi}}
\flushright{\begin{Arabic}
\quranayah[8][52]
\end{Arabic}}
\flushleft{\begin{hindi}
(उन लोगों की हालत) क़ौमे फिरऔन और उनके लोगों की सी है जो उन से पहले थे और ख़ुदा की आयतों से इन्कार करते थे तो ख़ुदा ने भी उनके गुनाहों की वजह से उन्हें ले डाला बेशक ख़ुदा ज़बरदस्त और बहुत सख्त अज़ाब देने वाला है
\end{hindi}}
\flushright{\begin{Arabic}
\quranayah[8][53]
\end{Arabic}}
\flushleft{\begin{hindi}
ये सज़ा इस वजह से (दी गई) कि जब कोई नेअमत ख़ुदा किसी क़ौम को देता है तो जब तक कि वह लोग ख़ुद अपनी कलबी हालत (न) बदलें ख़ुदा भी उसे नहीं बदलेगा और ख़ुदा तो यक़ीनी (सब की सुनता) और सब कुछ जानता है
\end{hindi}}
\flushright{\begin{Arabic}
\quranayah[8][54]
\end{Arabic}}
\flushleft{\begin{hindi}
(उन लोगों की हालत) क़ौम फिरऔन और उन लोगों की सी है जो उनसे पहले थे और अपने परवरदिगार की आयतों को झुठलाते थे तो हमने भी उनके गुनाहों की वजह से उनको हलाक़ कर डाला और फिरऔन की क़ौम को डुबा मारा और (ये) सब के सब ज़ालिम थे
\end{hindi}}
\flushright{\begin{Arabic}
\quranayah[8][55]
\end{Arabic}}
\flushleft{\begin{hindi}
इसमें शक़ नहीं कि ख़ुदा के नज़दीक जानवरों में कुफ्फ़ार सबसे बदतरीन तो (बावजूद इसके) फिर ईमान नहीं लाते
\end{hindi}}
\flushright{\begin{Arabic}
\quranayah[8][56]
\end{Arabic}}
\flushleft{\begin{hindi}
ऐ रसूल जिन लोगों से तुम ने एहद व पैमान किया था फिर वह लोग अपने एहद को हर बार तोड़ डालते है और (फिर ख़ुदा से) नहीं डरते
\end{hindi}}
\flushright{\begin{Arabic}
\quranayah[8][57]
\end{Arabic}}
\flushleft{\begin{hindi}
तो अगर वह लड़ाई में तुम्हारे हाथे चढ़ जाएँ तो (ऐसी सख्त गोश्माली दो कि) उनके साथ साथ उन लोगों का तो अगर वह लड़ाई में तुम्हारे हत्थे चढ़ जाएं तो (ऐसी सजा दो की) उनके साथ उन लोगों को भी तितिर बितिर कर दो जो उन के पुश्त पर हो ताकि ये इबरत हासिल करें
\end{hindi}}
\flushright{\begin{Arabic}
\quranayah[8][58]
\end{Arabic}}
\flushleft{\begin{hindi}
और अगर तुम्हें किसी क़ौम की ख्यानत (एहद शिकनी(वादा ख़िलाफी)) का ख़ौफ हो तो तुम भी बराबर उनका एहद उन्हीं की तरफ से फेंक मारो (एहदो शिकन के साथ एहद शिकनी करो ख़ुदा हरगिज़ दग़ाबाजों को दोस्त नहीं रखता
\end{hindi}}
\flushright{\begin{Arabic}
\quranayah[8][59]
\end{Arabic}}
\flushleft{\begin{hindi}
और कुफ्फ़ार ये न ख्याल करें कि वह (मुसलमानों से) आगे बढ़ निकले (क्योंकि) वह हरगिज़ (मुसलमानों को) हरा नहीं सकते
\end{hindi}}
\flushright{\begin{Arabic}
\quranayah[8][60]
\end{Arabic}}
\flushleft{\begin{hindi}
और (मुसलमानों तुम कुफ्फार के मुकाबले के) वास्ते जहाँ तक तुमसे हो सके (अपने बाज़ू के) ज़ोर से और बॅधे हुए घोड़े से लड़ाई का सामान मुहय्या करो इससे ख़ुदा के दुश्मन और अपने दुश्मन और उसके सिवा दूसरे लोगों पर भी अपनी धाक बढ़ा लेगें जिन्हें तुम नहीं जानते हो मगर ख़ुदा तो उनको जानता है और ख़ुदा की राह में तुम जो कुछ भी ख़र्च करोगें वह तुम पूरा पूरा भर पाओगें और तुम पर किसी तरह ज़ुल्म नहीं किया जाएगा
\end{hindi}}
\flushright{\begin{Arabic}
\quranayah[8][61]
\end{Arabic}}
\flushleft{\begin{hindi}
और अगर ये कुफ्फार सुलह की तरफ माएल हो तो तुम भी उसकी तरफ माएल हो और ख़ुदा पर भरोसा रखो (क्योंकि) वह बेशक (सब कुछ) सुनता जानता है
\end{hindi}}
\flushright{\begin{Arabic}
\quranayah[8][62]
\end{Arabic}}
\flushleft{\begin{hindi}
और अगर वह लोग तुम्हें फरेब देना चाहे तो (कुछ परवा नहीं) ख़ुदा तुम्हारे वास्ते यक़ीनी काफी है-(ऐ रसूल) वही तो वह (ख़ुदा) है जिसने अपनी ख़ास मदद और मोमिनीन से तुम्हारी ताईद की
\end{hindi}}
\flushright{\begin{Arabic}
\quranayah[8][63]
\end{Arabic}}
\flushleft{\begin{hindi}
और उसी ने उन मुसलमानों के दिलों में बाहम ऐसी उलफ़त पैदा कर दी कि अगर तुम जो कुछ ज़मीन में है सब का सब खर्च कर डालते तो भी उनके दिलो में ऐसी उलफ़त पैदा न कर सकते मगर ख़ुदा ही था जिसने बाहम उलफत पैदा की बेशक वह ज़बरदस्त हिक़मत वाला है
\end{hindi}}
\flushright{\begin{Arabic}
\quranayah[8][64]
\end{Arabic}}
\flushleft{\begin{hindi}
ऐ रसूल तुमको बस ख़ुदा और जो मोमिनीन तुम्हारे ताबेए फरमान (फरमाबरदार) है काफी है
\end{hindi}}
\flushright{\begin{Arabic}
\quranayah[8][65]
\end{Arabic}}
\flushleft{\begin{hindi}
ऐ रसूल तुम मोमिनीन को जिहाद के वास्ते आमादा करो (वह घबराए नहीं ख़ुदा उनसे वायदा करता है कि) अगर तुम लोगों में के साबित क़दम रहने वाले बीस भी होगें तो वह दो सौ (काफिरों) पर ग़ालिब आ जायेगे और अगर तुम लोगों में से साबित कदम रहने वालों सौ होगें तो हज़ार (काफिरों) पर ग़ालिब आ जाएँगें इस सबब से कि ये लोग ना समझ हैं
\end{hindi}}
\flushright{\begin{Arabic}
\quranayah[8][66]
\end{Arabic}}
\flushleft{\begin{hindi}
अब ख़ुदा ने तुम से (अपने हुक्म की सख्ती में) तख्फ़ीफ (कमी) कर दी और देख लिया कि तुम में यक़ीनन कमज़ोरी है तो अगर तुम लोगों में से साबित क़दम रहने वाले सौ होगें तो दो सौ (काफ़िरों) पर ग़ालिब रहेंगें और अगर तुम लोगों में से (ऐसे) एक हज़ार होगें तो ख़ुदा के हुक्म से दो हज़ार (काफ़िरों) पर ग़ालिब रहेंगे और (जंग की तकलीफों को) झेल जाने वालों का ख़ुदा साथी है
\end{hindi}}
\flushright{\begin{Arabic}
\quranayah[8][67]
\end{Arabic}}
\flushleft{\begin{hindi}
कोई नबी जब कि रूए ज़मीन पर (काफिरों का) खून न बहाए उसके यहाँ कैदियों का रहना मुनासिब नहीं तुम लोग तो दुनिया के साज़ो सामान के ख्वाहॉ (चाहने वाले) हो औॅर ख़ुदा (तुम्हारे लिए) आख़िरत की (भलाई) का ख्वाहॉ है और ख़ुदा ज़बरदस्त हिकमत वाला है
\end{hindi}}
\flushright{\begin{Arabic}
\quranayah[8][68]
\end{Arabic}}
\flushleft{\begin{hindi}
और अगर ख़ुदा की तरफ से पहले ही (उसकी) माफी का हुक्म आ चुका होता तो तुमने जो (बदर के क़ैदियों के छोड़ देने के बदले) फिदिया लिया था
\end{hindi}}
\flushright{\begin{Arabic}
\quranayah[8][69]
\end{Arabic}}
\flushleft{\begin{hindi}
उसकी सज़ा में तुम पर बड़ा अज़ाब नाज़िल होकर रहता तो (ख़ैर जो हुआ सो हुआ) अब तुमने जो माल ग़नीमत हासिल किया है उसे खाओ (और तुम्हारे लिए) हलाल तय्यब है और ख़ुदा से डरते रहो बेशक ख़ुदा बड़ा बख्शने वाला मेहरबान है
\end{hindi}}
\flushright{\begin{Arabic}
\quranayah[8][70]
\end{Arabic}}
\flushleft{\begin{hindi}
ऐ रसूल जो कैदी तुम्हारे कब्जे में है उनसे कह दो कि अगर तुम्हारे दिलों में नेकी देखेगा तो जो (माल) तुम से छीन लिया गया है उससे कहीं बेहतर तुम्हें अता फरमाएगा और तुम्हें बख्श भी देगा और ख़ुदा तो बड़ा बख्शने वाला मेहरबान है
\end{hindi}}
\flushright{\begin{Arabic}
\quranayah[8][71]
\end{Arabic}}
\flushleft{\begin{hindi}
और अगर ये लोग तुमसे फरेब करना चाहते है तो ख़ुदा से पहले ही फरेब कर चुके हैं तो (उसकी सज़ा में) ख़ुदा ने उन पर तुम्हें क़ाबू दे दिया और ख़ुदा तो बड़ा वाक़िफकार हिकमत वाला है
\end{hindi}}
\flushright{\begin{Arabic}
\quranayah[8][72]
\end{Arabic}}
\flushleft{\begin{hindi}
जिन लोगों ने ईमान क़ुबूल किया और हिजरत की और अपने अपने जान माल से ख़ुदा की राह में जिहाद किया और जिन लोगों ने (हिजरत करने वालों को जगह दी और हर (तरह) उनकी ख़बर गीरी (मदद) की यही लोग एक दूसरे के (बाहम) सरपरस्त दोस्त हैं और जिन लोगों ने ईमान क़ुबूल किया और हिजरत नहीं की तो तुम लोगों को उनकी सरपरस्ती से कुछ सरोकार नहीं-यहाँ तक कि वह हिजरत एख्तियार करें और (हॉ) मगर दीनी अम्र में तुम से मदद के ख्वाहॉ हो तो तुम पर (उनकी मदद करना लाज़िम व वाजिब है मगर उन लोगों के मुक़ाबले में (नहीं) जिनमें और तुममें बाहम (सुलह का) एहदो पैमान है और जो कुछ तुम करते हो ख़ुदा (सबको) देख रहा है
\end{hindi}}
\flushright{\begin{Arabic}
\quranayah[8][73]
\end{Arabic}}
\flushleft{\begin{hindi}
और जो लोग काफ़िर हैं वह भी (बाहम) एक दूसरे के सरपरस्त हैं अगर तुम (इस तरह) वायदा न करोगे तो रूए ज़मीन पर फ़ितना (फ़साद) बरपा हो जाएगा और बड़ा फ़साद होगा
\end{hindi}}
\flushright{\begin{Arabic}
\quranayah[8][74]
\end{Arabic}}
\flushleft{\begin{hindi}
और जिन लोगों ने ईमान क़ुबूल किया और हिजरत की और ख़ुदा की राह में लड़े भिड़े और जिन लोगों ने (ऐसे नाज़ुक वक्त में मुहाजिरीन को जगह ही और उनकी हर तरह ख़बरगीरी (मदद) की यही लोग सच्चे ईमानदार हैं उन्हीं के वास्ते मग़फिरत और इज्ज़त व आबरु वाली रोज़ी है
\end{hindi}}
\flushright{\begin{Arabic}
\quranayah[8][75]
\end{Arabic}}
\flushleft{\begin{hindi}
और जिन लोगों ने (सुलह हुदैबिया के) बाद ईमान क़ुबूल किया और हिजरत की और तुम्हारे साथ मिलकर जिहाद किया वह लोग भी तुम्हीं में से हैं और साहबाने क़राबत ख़ुदा की किताब में बाहम एक दूसरे के (बनिस्बत औरों के) ज्यादा हक़दार हैं बेशक ख़ुदा हर चीज़ से ख़ूब वाक़िफ हैं
\end{hindi}}
\chapter{Al-Bara'at / At-Taubah(The Immunity)}
\begin{Arabic}
\Huge{\centerline{\basmalah}}\end{Arabic}
\flushright{\begin{Arabic}
\quranayah[9][1]
\end{Arabic}}
\flushleft{\begin{hindi}
(ऐ मुसलमानों) जिन मुशरिकों से तुम लोगों ने सुलह का एहद किया था अब ख़ुदा और उसके रसूल की तरफ से उनसे (एक दम) बेज़ारी है
\end{hindi}}
\flushright{\begin{Arabic}
\quranayah[9][2]
\end{Arabic}}
\flushleft{\begin{hindi}
तो (ऐ मुशरिकों) बस तुम चार महीने (ज़ीकादा, जिल हिज्जा, मुहर्रम रजब) तो (चैन से बेख़तर) रूए ज़मीन में सैरो सियाहत (घूम फिर) कर लो और ये समझते रहे कि तुम (किसी तरह) ख़ुदा को आजिज़ नहीं कर सकते और ये भी कि ख़ुदा काफ़िरों को ज़रूर रूसवा करके रहेगा
\end{hindi}}
\flushright{\begin{Arabic}
\quranayah[9][3]
\end{Arabic}}
\flushleft{\begin{hindi}
और ख़ुदा और उसके रसूल की तरफ से हज अकबर के दिन (तुम) लोगों को मुनादी की जाती है कि ख़ुदा और उसका रसूल मुशरिकों से बेज़ार (और अलग) है तो (मुशरिकों) अगर तुम लोगों ने (अब भी) तौबा की तो तुम्हारे हक़ में यही बेहतर है और अगर तुम लोगों ने (इससे भी) मुंह मोड़ा तो समझ लो कि तुम लोग ख़ुदा को हरगिज़ आजिज़ नहीं कर सकते और जिन लोगों ने कुफ्र इख्तेयार किया उनको दर्दनाक अज़ाब की ख़ुश ख़बरी दे दो
\end{hindi}}
\flushright{\begin{Arabic}
\quranayah[9][4]
\end{Arabic}}
\flushleft{\begin{hindi}
मगर (हाँ) जिन मुशरिकों से तुमने एहदो पैमान किया था फिर उन लोगों ने भी कभी कुछ तुमसे (वफ़ा एहद में) कमी नहीं की और न तुम्हारे मुक़ाबले में किसी की मदद की हो तो उनके एहद व पैमान को जितनी मुद्द्त के वास्ते मुक़र्रर किया है पूरा कर दो ख़ुदा परहेज़गारों को यक़ीनन दोस्त रखता है
\end{hindi}}
\flushright{\begin{Arabic}
\quranayah[9][5]
\end{Arabic}}
\flushleft{\begin{hindi}
फिर जब हुरमत के चार महीने गुज़र जाएँ तो मुशरिकों को जहाँ पाओ (बे ताम्मुल) कत्ल करो और उनको गिरफ्तार कर लो और उनको कैद करो और हर घात की जगह में उनकी ताक में बैठो फिर अगर वह लोग (अब भी शिर्क से) बाज़ आऎं और नमाज़ पढ़ने लगें और ज़कात दे तो उनकी राह छोड़ दो (उनसे ताअरूज़ न करो) बेशक ख़ुदा बड़ा बख़्शने वाला मेहरबान है
\end{hindi}}
\flushright{\begin{Arabic}
\quranayah[9][6]
\end{Arabic}}
\flushleft{\begin{hindi}
और (ऐ रसूल) अगर मुशरिकीन में से कोई तुमसे पनाह मागें तो उसको पनाह दो यहाँ तक कि वह ख़ुदा का कलाम सुन ले फिर उसे उसकी अमन की जगह वापस पहुँचा दो ये इस वजह से कि ये लोग नादान हैं
\end{hindi}}
\flushright{\begin{Arabic}
\quranayah[9][7]
\end{Arabic}}
\flushleft{\begin{hindi}
(जब) मुशरिकीन ने ख़ुद एहद शिकनी (तोड़ा) की तो उन का कोई एहदो पैमान ख़ुदा के नज़दीक और उसके रसूल के नज़दीक क्योंकर (क़ायम) रह सकता है मगर जिन लोगों से तुमने खानाए काबा के पास मुआहेदा किया था तो वह लोग (अपनी एहदो पैमान) तुमसे क़ायम रखना चाहें तो तुम भी उन से (अपना एहद) क़ायम रखो बेशक ख़ुदा (बद एहदी से) परहेज़ करने वालों को दोस्त रखता है
\end{hindi}}
\flushright{\begin{Arabic}
\quranayah[9][8]
\end{Arabic}}
\flushleft{\begin{hindi}
(उनका एहद) क्योंकर (रह सकता है) जब (उनकी ये हालत है) कि अगर तुम पर ग़लबा पा जाएं तो तुम्हारे में न तो रिश्ते नाते ही का लिहाज़ करेगें और न अपने क़ौल व क़रार का ये लोग तुम्हें अपनी ज़बानी (जमा खर्च में) खुश कर देते हैं हालॉकि उनके दिल नहीं मानते और उनमें के बहुतेरे तो बदचलन हैं
\end{hindi}}
\flushright{\begin{Arabic}
\quranayah[9][9]
\end{Arabic}}
\flushleft{\begin{hindi}
और उन लोगों ने ख़ुदा की आयतों के बदले थोड़ी सी क़ीमत (दुनियावी फायदे) हासिल करके (लोगों को) उसकी राह से रोक दिया बेशक ये लोग जो कुछ करते हैं ये बहुत ही बुरा है
\end{hindi}}
\flushright{\begin{Arabic}
\quranayah[9][10]
\end{Arabic}}
\flushleft{\begin{hindi}
ये लोग किसी मोमिन के बारे में न तो रिश्ता नाता ही कर लिहाज़ करते हैं और न क़ौल का क़रार का और (वाक़ई) यही लोग ज्यादती करते हैं
\end{hindi}}
\flushright{\begin{Arabic}
\quranayah[9][11]
\end{Arabic}}
\flushleft{\begin{hindi}
तो अगर (अभी मुशरिक से) तौबा करें और नमाज़ पढ़ने लगें और ज़कात दें तो तुम्हारे दीनी भाई हैं और हम अपनी आयतों को वाक़िफकार लोगों के वास्ते तफ़सीलन बयान करते हैं
\end{hindi}}
\flushright{\begin{Arabic}
\quranayah[9][12]
\end{Arabic}}
\flushleft{\begin{hindi}
और अगर ये लोग एहद कर चुकने के बाद अपनी क़समें तोड़ डालें और तुम्हारे दीन में तुमको ताना दें तो तुम कुफ्र के सरवर आवारा लोगों से खूब लड़ाई करो उनकी क़समें का हरगिज़ कोई एतबार नहीं ताकि ये लोग (अपनी शरारत से) बाज़ आएँ
\end{hindi}}
\flushright{\begin{Arabic}
\quranayah[9][13]
\end{Arabic}}
\flushleft{\begin{hindi}
(मुसलमानों) भला तुम उन लोगों से क्यों नहीं लड़ते जिन्होंने अपनी क़समों को तोड़ डाला और रसूल को निकाल बाहर करना (अपने दिल में) ठान लिया था और तुमसे पहले छेड़ भी उन्होनें ही शुरू की थी क्या तुम उनसे डरते हो तो अगर तुम सच्चे ईमानदार हो तो ख़ुदा उनसे कहीं बढ़ कर तुम्हारे डरने के क़ाबिल है
\end{hindi}}
\flushright{\begin{Arabic}
\quranayah[9][14]
\end{Arabic}}
\flushleft{\begin{hindi}
इनसे (बेख़ौफ (ख़तर) लड़ो ख़ुदा तुम्हारे हाथों उनकी सज़ा करेगा और उन्हें रूसवा करेगा और तुम्हें उन पर फतेह अता करेगा और ईमानदार लोगों के कलेजे ठन्डे करेगा
\end{hindi}}
\flushright{\begin{Arabic}
\quranayah[9][15]
\end{Arabic}}
\flushleft{\begin{hindi}
और उन मोनिनीन के दिल की क़ुदरतें जो (कुफ्फ़ार से पहुचॅती है) दफ़ा कर देगा और ख़ुदा जिसकी चाहे तौबा क़ुबूल करे और ख़ुदा बड़ा वाक़िफकार (और) हिकमत वाला है
\end{hindi}}
\flushright{\begin{Arabic}
\quranayah[9][16]
\end{Arabic}}
\flushleft{\begin{hindi}
क्या तुमने ये समझ लिया है कि तुम (यूं ही) छोड़ दिए जाओगे और अभी तक तो ख़ुदा ने उन लोगों को मुमताज़ किया ही नहीं जो तुम में के (राहे ख़ुदा में) जिहाद करते हैं और ख़ुदा और उसके रसूल और मोमेनीन के सिवा किसी को अपना राज़दार दोस्त नहीं बनाते और जो कुछ भी तुम करते हो ख़ुदा उससे बाख़बर है
\end{hindi}}
\flushright{\begin{Arabic}
\quranayah[9][17]
\end{Arabic}}
\flushleft{\begin{hindi}
मुशरेकीन का ये काम नहीं कि जब वह अपने कुफ़्र का ख़ुद इक़रार करते है तो ख़ुदा की मस्जिदों को (जाकर) आबाद करे यही वह लोग हैं जिनका किया कराया सब अकारत हुआ और ये लोग हमेशा जहन्नुम में रहेंगे
\end{hindi}}
\flushright{\begin{Arabic}
\quranayah[9][18]
\end{Arabic}}
\flushleft{\begin{hindi}
ख़ुदा की मस्जिदों को बस सिर्फ वहीं शख़्स (जाकर) आबाद कर सकता है जो ख़ुदा और रोजे आख़िरत पर ईमान लाए और नमाज़ पढ़ा करे और ज़कात देता रहे और ख़ुदा के सिवा (और) किसी से न डरो तो अनक़रीब यही लोग हिदायत याफ्ता लोगों मे से हो जाऎंगे
\end{hindi}}
\flushright{\begin{Arabic}
\quranayah[9][19]
\end{Arabic}}
\flushleft{\begin{hindi}
क्या तुम लोगों ने हाजियों की सक़ाई (पानी पिलाने वाले) और मस्जिदुल हराम (ख़ानाए काबा की आबादियों को उस शख़्स के हमसर (बराबर) बना दिया है जो ख़ुदा और रोज़े आख़ेरत के दिन पर ईमान लाया और ख़ुदा के राह में जेहाद किया ख़ुदा के नज़दीक तो ये लोग बराबर नहीं और खुदा ज़ालिम लोगों की हिदायत नहीं करता है
\end{hindi}}
\flushright{\begin{Arabic}
\quranayah[9][20]
\end{Arabic}}
\flushleft{\begin{hindi}
जिन लोगों ने ईमान क़ुबूल किया और (ख़ुदा के लिए) हिजरत एख्तियार की और अपने मालों से और अपनी जानों से ख़ुदा की राह में जिहाद किया वह लोग ख़ुदा के नज़दीक दर्जें में कही बढ़ कर हैं और यही लोग (आला दर्जे पर) फायज़ होने वाले हैं
\end{hindi}}
\flushright{\begin{Arabic}
\quranayah[9][21]
\end{Arabic}}
\flushleft{\begin{hindi}
उनका परवरदिगार उनको अपनी मेहरबानी और ख़ुशनूदी और ऐसे (हरे भरे) बाग़ों की ख़ुशख़बरी देता है जिसमें उनके लिए दाइमी ऐश व (आराम) होगा
\end{hindi}}
\flushright{\begin{Arabic}
\quranayah[9][22]
\end{Arabic}}
\flushleft{\begin{hindi}
और ये लोग उन बाग़ों में हमेशा अब्दआलाबाद (हमेशा हमेशा) तक रहेंगे बेशक ख़ुदा के पास तो बड़ा बड़ा अज्र व (सवाब) है
\end{hindi}}
\flushright{\begin{Arabic}
\quranayah[9][23]
\end{Arabic}}
\flushleft{\begin{hindi}
ऐ ईमानदारों अगर तुम्हारे माँ बाप और तुम्हारे (बहन) भाई ईमान के मुक़ाबले कुफ़्र को तरजीह देते हो तो तुम उनको (अपना) ख़ैर ख्वाह (हमदर्द) न समझो और तुममें जो शख़्स उनसे उलफ़त रखेगा तो यही लोग ज़ालिम है
\end{hindi}}
\flushright{\begin{Arabic}
\quranayah[9][24]
\end{Arabic}}
\flushleft{\begin{hindi}
(ऐ रसूल) तुम कह दो तुम्हारे बाप दादा और तुम्हारे बेटे और तुम्हारे भाई बन्द और तुम्हारी बीबियाँ और तुम्हारे कुनबे वाले और वह माल जो तुमने कमा के रख छोड़ा हैं और वह तिजारत जिसके मन्द पड़ जाने का तुम्हें अन्देशा है और वह मकानात जिन्हें तुम पसन्द करते हो अगर तुम्हें ख़ुदा से और उसके रसूल से और उसकी राह में जिहाद करने से ज्यादा अज़ीज़ है तो तुम ज़रा ठहरो यहाँ तक कि ख़ुदा अपना हुक्म (अज़ाब) मौजूद करे और ख़ुदा नाफरमान लोगों की हिदायत नहीं करता
\end{hindi}}
\flushright{\begin{Arabic}
\quranayah[9][25]
\end{Arabic}}
\flushleft{\begin{hindi}
(मुसलमानों) ख़ुदा ने तुम्हारी बहुतेरे मक़ामात पर (ग़ैबी) इमदाद की और (ख़ासकर) जंग हुनैन के दिन जब तुम्हें अपनी क़सरत (तादाद) ने मग़रूर कर दिया था फिर वह क़सरत तुम्हें कुछ भी काम न आयी और (तुम ऐसे घबराए कि) ज़मीन बावजूद उस वसअत (फैलाव) के तुम पर तंग हो गई तुम पीठ कर भाग निकले
\end{hindi}}
\flushright{\begin{Arabic}
\quranayah[9][26]
\end{Arabic}}
\flushleft{\begin{hindi}
तब ख़ुदा ने अपने रसूल पर और मोमिनीन पर अपनी (तरफ से) तसकीन नाज़िल फरमाई और (रसूल की ख़ातिर से) फ़रिश्तों के लश्कर भेजे जिन्हें तुम देखते भी नहीं थे और कुफ्फ़ार पर अज़ाब नाज़िल फरमाया और काफिरों की यही सज़ा है
\end{hindi}}
\flushright{\begin{Arabic}
\quranayah[9][27]
\end{Arabic}}
\flushleft{\begin{hindi}
उसके बाद ख़ुदा जिसकी चाहे तौबा क़ुबूल करे और ख़ुदा बड़ा बख़्शने वाला मेहरबान है
\end{hindi}}
\flushright{\begin{Arabic}
\quranayah[9][28]
\end{Arabic}}
\flushleft{\begin{hindi}
ऐ ईमानदारों मुशरेकीन तो निरे नजिस हैं तो अब वह उस साल के बाद मस्जिदुल हराम (ख़ाना ए काबा) के पास फिर न फटकने पाएँ और अगर तुम (उनसे जुदा होने में) फक़रों फाक़ा से डरते हो तो अनकरीब ही ख़ुदा तुमको अगर चाहेगा तो अपने फज़ल (करम) से ग़नीकर देगा बेशक ख़ुदा बड़ा वाक़िफकार हिकमत वाला है
\end{hindi}}
\flushright{\begin{Arabic}
\quranayah[9][29]
\end{Arabic}}
\flushleft{\begin{hindi}
अहले किताब में से जो लोग न तो (दिल से) ख़ुदा ही पर ईमान रखते हैं और न रोज़े आख़िरत पर और न ख़ुदा और उसके रसूल की हराम की हुई चीज़ों को हराम समझते हैं और न सच्चे दीन ही को एख्तियार करते हैं उन लोगों से लड़े जाओ यहाँ तक कि वह लोग ज़लील होकर (अपने) हाथ से जज़िया दे
\end{hindi}}
\flushright{\begin{Arabic}
\quranayah[9][30]
\end{Arabic}}
\flushleft{\begin{hindi}
यहूद तो कहते हैं कि अज़ीज़ ख़ुदा के बेटे हैं और ईसाई कहते हैं कि मसीहा (ईसा) ख़ुदा के बेटे हैं ये तो उनकी बात है और (वह ख़ुद) उन्हीं के मुँह से ये लोग भी उन्हीं काफ़िरों की सी बातें बनाने लगे जो उनसे पहले गुज़र चुके हैं ख़ुदा उनको क़त्ल (तहस नहस) करके (देखो तो) कहाँ से कहाँ भटके जा रहे हैं
\end{hindi}}
\flushright{\begin{Arabic}
\quranayah[9][31]
\end{Arabic}}
\flushleft{\begin{hindi}
उन लोगों ने तो अपने ख़ुदा को छोड़कर अपनी आलिमों को और अपने ज़ाहिदों को और मरियम के बेटे मसीह को अपना परवरदिगार बना डाला हालॉकि उन्होनें सिवाए इसके और हुक्म ही नहीं दिया गया कि ख़ुदाए यक़ता (सिर्फ़ ख़ुदा) की इबादत करें उसके सिवा (और कोई क़ाबिले परसतिश नहीं)
\end{hindi}}
\flushright{\begin{Arabic}
\quranayah[9][32]
\end{Arabic}}
\flushleft{\begin{hindi}
जिस चीज़ को ये लोग उसका शरीक़ बनाते हैं वह उससे पाक व पाक़ीजा है ये लोग चाहते हैं कि अपने मुँह से (फ़ूंक मारकर) ख़ुदा के नूर को बुझा दें और ख़ुदा इसके सिवा कुछ मानता ही नहीं कि अपने नूर को पूरा ही करके रहे
\end{hindi}}
\flushright{\begin{Arabic}
\quranayah[9][33]
\end{Arabic}}
\flushleft{\begin{hindi}
अगरचे कुफ्फ़ार बुरा माना करें वही तो (वह ख़ुदा) है कि जिसने अपने रसूल (मोहम्मद) को हिदायत और सच्चे दीन के साथ ( मबऊस करके) भेजता कि उसको तमाम दीनो पर ग़ालिब करे अगरचे मुशरेकीन बुरा माना करे
\end{hindi}}
\flushright{\begin{Arabic}
\quranayah[9][34]
\end{Arabic}}
\flushleft{\begin{hindi}
ऐ ईमानदारों इसमें उसमें शक़ नहीं कि (यहूद व नसारा के) बहुतेरे आलिम ज़ाहिद लोगों के माल (नाहक़) चख जाते है और (लोगों को) ख़ुदा की राह से रोकते हैं और जो लोग सोना और चाँदी जमा करते जाते हैं और उसको ख़ुदा की राह में खर्च नहीं करते तो (ऐ रसूल) उन को दर्दनाक अज़ाब की ख़ुशखबरी सुना दो
\end{hindi}}
\flushright{\begin{Arabic}
\quranayah[9][35]
\end{Arabic}}
\flushleft{\begin{hindi}
(जिस दिन वह (सोना चाँदी) जहन्नुम की आग में गर्म (और लाल) किया जाएगा फिर उससे उनकी पेशानियाँ और उनके पहलू और उनकी पीठें दाग़ी जाऎंगी (और उनसे कहा जाएगा) ये वह है जिसे तुमने अपने लिए (दुनिया में) जमा करके रखा था तो (अब) अपने जमा किए का मज़ा चखो
\end{hindi}}
\flushright{\begin{Arabic}
\quranayah[9][36]
\end{Arabic}}
\flushleft{\begin{hindi}
इसमें तो शक़ ही नहीं कि ख़ुदा ने जिस दिन आसमान व ज़मीन को पैदा किया (उसी दिन से) ख़ुदा के नज़दीक ख़ुदा की किताब (लौहे महफूज़) में महीनों की गिनती बारह महीने है उनमें से चार महीने (अदब व) हुरमत के हैं यही दीन सीधी राह है तो उन चार महीनों में तुम अपने ऊपर (कुश्त व ख़ून (मार काट) करके) ज़ुल्म न करो और मुशरेकीन जिस तरह तुम से सबके बस मिलकर लड़ते हैं तुम भी उसी तरह सबके सब मिलकर उन से लड़ों और ये जान लो कि ख़ुदा तो यक़ीनन परहेज़गारों के साथ है
\end{hindi}}
\flushright{\begin{Arabic}
\quranayah[9][37]
\end{Arabic}}
\flushleft{\begin{hindi}
महीनों का आगे पीछे कर देना भी कुफ़्र ही की ज्यादती है कि उनकी बदौलत कुफ्फ़ार (और) बहक जाते हैं एक बरस तो उसी एक महीने को हलाल समझ लेते हैं और (दूसरे) साल उसी महीने को हराम कहते हैं ताकि ख़ुदा ने जो (चार महीने) हराम किए हैं उनकी गिनती ही पूरी कर लें और ख़ुदा की हराम की हुई चीज़ को हलाल कर लें उनकी बुरी (बुरी) कारस्तानियॉ उन्हें भली कर दिखाई गई हैं और खुदा काफिर लोगो को मंज़िले मक़सूद तक नहीं पहुँचाया करता
\end{hindi}}
\flushright{\begin{Arabic}
\quranayah[9][38]
\end{Arabic}}
\flushleft{\begin{hindi}
ऐ ईमानदारों तुम्हें क्या हो गया है कि जब तुमसे कहा जाता है कि ख़ुदा की राह में (जिहाद के लिए) निकलो तो तुम लदधड़ (ढीले) हो कर ज़मीन की तरफ झुके पड़ते हो क्या तुम आख़िरत के बनिस्बत दुनिया की (चन्द रोज़ा) जिन्दगी को पसन्द करते थे तो (समझ लो कि) दुनिया की ज़िन्दगी का साज़ो सामान (आख़िर के) ऐश व आराम के मुक़ाबले में बहुत ही थोड़ा है
\end{hindi}}
\flushright{\begin{Arabic}
\quranayah[9][39]
\end{Arabic}}
\flushleft{\begin{hindi}
अगर (अब भी) तुम न निकलोगे तो ख़ुदा तुम पर दर्दनाक अज़ाब नाज़िल फरमाएगा और (ख़ुदा कुछ मजबरू तो है नहीं) तुम्हारे बदले किसी दूसरी क़ौम को ले आएगा और तुम उसका कुछ भी बिगाड़ नहीं सकते और ख़ुदा हर चीज़ पर क़ादिर है
\end{hindi}}
\flushright{\begin{Arabic}
\quranayah[9][40]
\end{Arabic}}
\flushleft{\begin{hindi}
अगर तुम उस रसूल की मदद न करोगे तो (कुछ परवाह् नहीं ख़ुदा मददगार है) उसने तो अपने रसूल की उस वक्त मदद की जब उसकी कुफ्फ़ार (मक्का) ने (घर से) निकल बाहर किया उस वक्त सिर्फ (दो आदमी थे) दूसरे रसूल थे जब वह दोनो ग़ार (सौर) में थे जब अपने साथी को (उसकी गिरिया व ज़ारी (रोने) पर) समझा रहे थे कि घबराओ नहीं ख़ुदा यक़ीनन हमारे साथ है तो ख़ुदा ने उन पर अपनी (तरफ से) तसकीन नाज़िल फरमाई और (फ़रिश्तों के) ऐसे लश्कर से उनकी मदद की जिनको तुम लोगों ने देखा तक नहीं और ख़ुदा ने काफिरों की बात नीची कर दिखाई और ख़ुदा ही का बोल बाला है और ख़ुदा तो ग़ालिब हिकमत वाला है
\end{hindi}}
\flushright{\begin{Arabic}
\quranayah[9][41]
\end{Arabic}}
\flushleft{\begin{hindi}
(मुसलमानों) तुम हलके फुलके (हॅसते) हो या भारी भरकम (मसलह) बहर हाल जब तुमको हुक्म दिया जाए फौरन चल खड़े हो और अपनी जानों से अपने मालों से ख़ुदा की राह में जिहाद करो अगर तुम (कुछ जानते हो तो) समझ लो कि यही तुम्हारे हक़ में बेहतर है
\end{hindi}}
\flushright{\begin{Arabic}
\quranayah[9][42]
\end{Arabic}}
\flushleft{\begin{hindi}
(ऐ रसूल) अगर सरे दस्त फ़ायदा और सफर आसान होता तो यक़ीनन ये लोग तुम्हारा साथ देते मगर इन पर मुसाफ़त (सफ़र) की मशक़क़त (सख्ती) तूलानी हो गई और अगर पीछे रह जाने की वज़ह से पूछोगे तो ये लोग फौरन ख़ुदा की क़समें खॉएगें कि अगर हम में सकत होती तो हम भी ज़रूर तुम लोगों के साथ ही चल खड़े होते ये लोग झूठी कसमें खाकर अपनी जान आप हलाक किए डालते हैं और ख़ुदा तो जानता है कि ये लोग बेशक झूठे हैं
\end{hindi}}
\flushright{\begin{Arabic}
\quranayah[9][43]
\end{Arabic}}
\flushleft{\begin{hindi}
(ऐ रसूल) ख़ुदा तुमसे दरगुज़र फरमाए तुमने उन्हें (पीछे रह जाने की) इजाज़त ही क्यों दी ताकि (तुम) अगर ऐसा न करते तो) तुम पर सच बोलने वाले (अलग) ज़ाहिर हो जाते और तुम झूटों को (अलग) मालूम कर लेते
\end{hindi}}
\flushright{\begin{Arabic}
\quranayah[9][44]
\end{Arabic}}
\flushleft{\begin{hindi}
(ऐ रसूल) जो लोग (दिल से) ख़ुदा और रोज़े आख़िरत पर ईमान रखते हैं वह तो अपने माल से और अपनी जानों से जिहाद (न) करने की इजाज़त मॉगने के नहीं (बल्कि वह ख़ुद जाऎंगे) और ख़ुदा परहेज़गारों से खूब वाक़िफ है
\end{hindi}}
\flushright{\begin{Arabic}
\quranayah[9][45]
\end{Arabic}}
\flushleft{\begin{hindi}
(पीछे रह जाने की) इजाज़त तो बस वही लोग मॉगेंगे जो ख़ुदा और रोजे आख़िरत पर ईमान नहीं रखते और उनके दिल (तरह तरह) के शक़ कर रहे है तो वह अपने शक़ में डावॉडोल हो रहे हैं
\end{hindi}}
\flushright{\begin{Arabic}
\quranayah[9][46]
\end{Arabic}}
\flushleft{\begin{hindi}
(कि क्या करें क्या न करें) और अगर ये लोग (घर से) निकलने की ठान लेते तो (कुछ न कुछ सामान तो करते मगर (बात ये है) कि ख़ुदा ने उनके साथ भेजने को नापसन्द किया तो उनको काहिल बना दिया और (गोया) उनसे कह दिया गया कि तुम बैठने वालों के साथ बैठे (मक्खी मारते) रहो
\end{hindi}}
\flushright{\begin{Arabic}
\quranayah[9][47]
\end{Arabic}}
\flushleft{\begin{hindi}
अगर ये लोग तुममें (मिलकर) निकलते भी तो बस तुममे फ़साद ही बरपा कर देते और तुम्हारे हक़ में फ़ितना कराने की ग़रज़ से तुम्हारे दरमियान (इधर उधर) घोड़े दौड़ाते फिरते और तुममें से उनके जासूस भी हैं (जो तुम्हारी उनसे बातें बयान करते हैं) और ख़ुदा शरीरों से ख़ूब वाक़िफ़ है
\end{hindi}}
\flushright{\begin{Arabic}
\quranayah[9][48]
\end{Arabic}}
\flushleft{\begin{hindi}
(ए रसूल) इसमें तो शक़ नहीं कि उन लोगों ने पहले ही फ़साद डालना चाहा था और तुम्हारी बहुत सी बातें उलट पुलट के यहॉ तक कि हक़ आ पहुंचा और ख़ुदा ही का हुक्म ग़ालिब रहा और उनको नागवार ही रहा
\end{hindi}}
\flushright{\begin{Arabic}
\quranayah[9][49]
\end{Arabic}}
\flushleft{\begin{hindi}
उन लोगों में से बाज़ ऐसे भी हैं जो साफ कहते हैं कि मुझे तो (पीछे रह जाने की) इजाज़त दीजिए और मुझ बला में न फॅसाइए (ऐ रसूल) आगाह हो कि ये लोग खुद बला में (औंधे मुँह) गिर पड़े और जहन्नुम तो काफिरों का यक़ीनन घेरे हुए ही हैं
\end{hindi}}
\flushright{\begin{Arabic}
\quranayah[9][50]
\end{Arabic}}
\flushleft{\begin{hindi}
तुमको कोई फायदा पहुंचा तो उन को बुरा मालूम होता है और अगर तुम पर कोई मुसीबत आ पड़ती तो ये लोग कहते हैं कि (इस वजह से) हमने अपना काम पहले ही ठीक कर लिया था और (ये कह कर) ख़ुश (तुम्हारे पास से उठकर) वापस लौटतें है
\end{hindi}}
\flushright{\begin{Arabic}
\quranayah[9][51]
\end{Arabic}}
\flushleft{\begin{hindi}
(ऐ रसूल) तुम कह दो कि हम पर हरगिज़ कोई मुसीबत पड़ नही सकती मगर जो ख़ुदा ने तुम्हारे लिए (हमारी तक़दीर में) लिख दिया है वही हमारा मालिक है और ईमानदारों को चाहिए भी कि ख़ुदा ही पर भरोसा रखें
\end{hindi}}
\flushright{\begin{Arabic}
\quranayah[9][52]
\end{Arabic}}
\flushleft{\begin{hindi}
(ऐ रसूल) तुम मुनाफिकों से कह दो कि तुम तो हमारे वास्ते (फतेह या शहादत) दो भलाइयों में से एक के ख्वाह मख्वाह मुन्तज़िर ही हो और हम तुम्हारे वास्ते उसके मुन्तज़िर हैं कि ख़ुदा तुम पर (ख़ास) अपने ही से कोई अज़ाब नाज़िल करे या हमारे हाथों से फिर (अच्छा) तुम भी इन्तेज़ार करो हम भी तुम्हारे साथ (साथ) इन्तेज़ार करते हैं
\end{hindi}}
\flushright{\begin{Arabic}
\quranayah[9][53]
\end{Arabic}}
\flushleft{\begin{hindi}
(ऐ रसूल) तुम कह दो कि तुम लोग ख्वाह ख़ुशी से खर्च करो या मजबूरी से तुम्हारी ख़ैरात तो कभी कुबूल की नहीं जाएगी तुम यक़ीनन बदकार लोग हो
\end{hindi}}
\flushright{\begin{Arabic}
\quranayah[9][54]
\end{Arabic}}
\flushleft{\begin{hindi}
और उनकी ख़ैरात के क़ुबूल किए जाने में और कोई वजह मायने नहीं मगर यही कि उन लोगों ने ख़ुदा और उसके रसूल की नाफ़रमानी की और नमाज़ को आते भी हैं तो अलकसाए हुए और ख़ुदा की राह में खर्च करते भी हैं तो बे दिली से
\end{hindi}}
\flushright{\begin{Arabic}
\quranayah[9][55]
\end{Arabic}}
\flushleft{\begin{hindi}
(ऐ रसूल) तुम को न तो उनके माल हैरत में डाले और न उनकी औलाद (क्योंकि) ख़ुदा तो ये चाहता है कि उनको आल व माल की वजह से दुनिया की (चन्द रोज़) ज़िन्दगी (ही) में मुबितलाए अज़ाब करे और जब उनकी जानें निकलें तब भी वह काफिर (के काफिर ही) रहें
\end{hindi}}
\flushright{\begin{Arabic}
\quranayah[9][56]
\end{Arabic}}
\flushleft{\begin{hindi}
और (मुसलमानों) ये लोग ख़ुदा की क़सम खाएंगे फिर वह तुममें ही के हैं हालॉकि वह लोग तुममें के नहीं हैं मगर हैं ये लोग बुज़दिल हैं
\end{hindi}}
\flushright{\begin{Arabic}
\quranayah[9][57]
\end{Arabic}}
\flushleft{\begin{hindi}
कि गर कहीं ये लोग पनाह की जगह (क़िले) या (छिपने के लिए) ग़ार या घुस बैठने की कोई (और) जगह पा जाए तो उसी तरफ रस्सियाँ तोड़ाते हुए भाग जाएँ
\end{hindi}}
\flushright{\begin{Arabic}
\quranayah[9][58]
\end{Arabic}}
\flushleft{\begin{hindi}
(ऐ रसूल) उनमें से कुछ तो ऐसे भी हैं जो तुम्हें ख़ैरात (की तक़सीम) में (ख्वाह मा ख्वाह) इल्ज़ाम देते हैं फिर अगर उनमे से कुछ (माक़ूल मिक़दार(हिस्सा)) दे दिया गया तो खुश हो गए और अगर उनकी मर्ज़ी के मुवाफिक़ उसमें से उन्हें कुछ नहीं दिया गया तो बस फौरन ही बिगड़ बैठे
\end{hindi}}
\flushright{\begin{Arabic}
\quranayah[9][59]
\end{Arabic}}
\flushleft{\begin{hindi}
और जो कुछ ख़ुदा ने और उसके रसूल ने उनको अता फरमाया था अगर ये लोग उस पर राज़ी रहते और कहते कि ख़ुदा हमारे वास्ते काफी है (उस वक्त नहीं तो) अनक़रीब ही खुदा हमें अपने फज़ल व करम से उसका रसूल दे ही देगा हम तो यक़ीनन अल्लाह ही की तरफ लौ लगाए बैठे हैं
\end{hindi}}
\flushright{\begin{Arabic}
\quranayah[9][60]
\end{Arabic}}
\flushleft{\begin{hindi}
(तो उनका क्या कहना था) ख़ैरात तो बस ख़ास फकीरों का हक़ है और मोहताजों का और उस (ज़कात वग़ैरह) के कारिन्दों का और जिनकी तालीफ़ क़लब की गई है (उनका) और (जिन की) गर्दनों में (गुलामी का फन्दा पड़ा है उनका) और ग़द्दारों का (जो ख़ुदा से अदा नहीं कर सकते) और खुदा की राह (जिहाद) में और परदेसियों की किफ़ालत में ख़र्च करना चाहिए ये हुकूक़ ख़ुदा की तरफ से मुक़र्रर किए हुए हैं और ख़ुदा बड़ा वाक़िफ कार हिकमत वाला है
\end{hindi}}
\flushright{\begin{Arabic}
\quranayah[9][61]
\end{Arabic}}
\flushleft{\begin{hindi}
और उसमें से बाज़ ऐसे भी हैं जो (हमारे) रसूल को सताते हैं और कहते हैं कि बस ये कान ही (कान) हैं (ऐ रसूल) तुम कह दो कि (कान तो हैं मगर) तुम्हारी भलाई सुन्ने के कान हैं कि ख़ुदा पर ईमान रखते हैं और मोमिनीन की (बातों) का यक़ीन रखते हैं और तुममें से जो लोग ईमान ला चुके हैं उनके लिए रहमत और जो लोग रसूले ख़ुदा को सताते हैं उनके लिए दर्दनाक अज़ाब हैं
\end{hindi}}
\flushright{\begin{Arabic}
\quranayah[9][62]
\end{Arabic}}
\flushleft{\begin{hindi}
(मुसलमानों) ये लोग तुम्हारे सामने ख़ुदा की क़समें खाते हैं ताकि तुम्हें राज़ी कर ले हालॉकि अगर ये लोग सच्चे ईमानदार है
\end{hindi}}
\flushright{\begin{Arabic}
\quranayah[9][63]
\end{Arabic}}
\flushleft{\begin{hindi}
तो ख़ुदा और उसका रसूल कहीं ज्यादा हक़दार है कि उसको राज़ी रखें क्या ये लोग ये भी नहीं जानते कि जिस शख़्स ने ख़ुदा और उसके रसूल की मुख़ालेफ़त की तो इसमें शक़ ही नहीं कि उसके लिए जहन्नुम की आग (तैयार रखी) है
\end{hindi}}
\flushright{\begin{Arabic}
\quranayah[9][64]
\end{Arabic}}
\flushleft{\begin{hindi}
जिसमें वह हमेशा (जलता भुनता) रहेगा यही तो बड़ी रूसवाई है मुनाफेक़ीन इस बात से डरतें हैं कि (कहीं ऐसा न हो) इन मुलसमानों पर (रसूल की माअरफ़त) कोई सूरा नाज़िल हो जाए जो उनको जो कुछ उन मुनाफिक़ीन के दिल में है बता दे (ऐ रसूल) तुम कह दो कि (अच्छा) तुम मसख़रापन किए जाओ
\end{hindi}}
\flushright{\begin{Arabic}
\quranayah[9][65]
\end{Arabic}}
\flushleft{\begin{hindi}
जिससे तुम डरते हो ख़ुदा उसे ज़रूर ज़ाहिर कर देगा और अगर तुम उनसे पूछो (कि ये हरकत थी) तो ज़रूर यूं ही कहेगें कि हम तो यूं ही बातचीत (दिल्लगी) बाज़ी ही कर रहे थे तुम कहो कि हाए क्या तुम ख़ुदा से और उसकी आयतों से और उसके रसूल से हॅसी कर रहे थे
\end{hindi}}
\flushright{\begin{Arabic}
\quranayah[9][66]
\end{Arabic}}
\flushleft{\begin{hindi}
अब बातें न बनाओं हक़ तो ये हैं कि तुम ईमान लाने के बाद काफ़िर हो बैठे अगर हम तुममें से कुछ लोगों से दरगुज़र भी करें तो हम कुछ लोगों को सज़ा ज़रूर देगें इस वजह से कि ये लोग कुसूरवार ज़रूर हैं
\end{hindi}}
\flushright{\begin{Arabic}
\quranayah[9][67]
\end{Arabic}}
\flushleft{\begin{hindi}
मुनाफिक़ मर्द और मुनाफिक़ औरतें एक दूसरे के बाहम जिन्स हैं कि (लोगों को) बुरे काम का तो हुक्म करते हैं और नेक कामों से रोकते हैं और अपने हाथ (राहे ख़ुदा में ख़र्च करने से) बन्द रखते हैं (सच तो यह है कि) ये लोग ख़ुदा को भूल बैठे तो ख़ुदा ने भी (गोया) उन्हें भुला दिया बेशक मुनाफ़िक़ बदचलन है
\end{hindi}}
\flushright{\begin{Arabic}
\quranayah[9][68]
\end{Arabic}}
\flushleft{\begin{hindi}
मुनाफिक़ मर्द और मुनाफिक़ औरतें और काफिरों से ख़ुदा ने जहन्नुम की आग का वायदा तो कर लिया है कि ये लोग हमेशा उसी में रहेगें और यही उन के लिए काफ़ी है और ख़ुदा ने उन पर लानत की है और उन्हीं के लिए दाइमी (हमेशा रहने वाला) अज़ाब है
\end{hindi}}
\flushright{\begin{Arabic}
\quranayah[9][69]
\end{Arabic}}
\flushleft{\begin{hindi}
(मुनाफिक़ो तुम्हारी तो) उनकी मसल है जो तुमसे पहले थे वह लोग तुमसे कूवत में (भी) ज्यादा थे और औलाद में (भी) कही बढ़ कर तो वह अपने हिस्से से भी फ़ायदा उठा हो चुके तो जिस तरह तुम से पहले लोग अपने हिस्से से फायदा उठा चुके हैं इसी तरह तुम ने अपने हिस्से से फायदा उठा लिया और जिस तरह वह बातिल में घुसे रहे उसी तरह तुम भी घुसे रहे ये वह लोग हैं जिन का सब किया धरा दुनिया और आख़िरत (दोनों) में अकारत हुआ और यही लोग घाटे में हैं
\end{hindi}}
\flushright{\begin{Arabic}
\quranayah[9][70]
\end{Arabic}}
\flushleft{\begin{hindi}
क्या इन मुनाफिक़ों को उन लोगों की ख़बर नहीं पहुँची है जो उनसे पहले हो गुज़रे हैं नूह की क़ौम और आद और समूद और इबराहीम की क़ौम और मदियन वाले और उलटी हुई बस्तियों के रहने वाले कि उनके पास उनके रसूल वाजेए (और रौशन) मौजिज़े लेकर आए तो (वह मुब्तिलाए अज़ाब हुए) और ख़ुदा ने उन पर जुल्म नहीं किया मगर ये लोग ख़ुद अपने ऊपर जुल्म करते थे
\end{hindi}}
\flushright{\begin{Arabic}
\quranayah[9][71]
\end{Arabic}}
\flushleft{\begin{hindi}
और ईमानदार मर्द और ईमानदार औरते उनमें से बाज़ के बाज़ रफीक़ है और नामज़ पाबन्दी से पढ़ते हैं और ज़कात देते हैं और ख़ुदा और उसके रसूल की फरमाबरदारी करते हैं यही लोग हैं जिन पर ख़ुदा अनक़रीब रहम करेगा बेशक ख़ुदा ग़ालिब हिकमत वाला है
\end{hindi}}
\flushright{\begin{Arabic}
\quranayah[9][72]
\end{Arabic}}
\flushleft{\begin{hindi}
ख़ुदा ने ईमानदार मर्दों और ईमानदारा औरतों से (बेहश्त के) उन बाग़ों का वायदा कर लिया है जिनके नीचे नहरें जारी हैं और वह उनमें हमेशा रहेगें (बेहश्त) अदन के बाग़ो में उम्दा उम्दा मकानात का (भी वायदा फरमाया) और ख़ुदा की ख़ुशनूदी उन सबसे बालातर है- यही तो बड़ी (आला दर्जे की) कामयाबी है
\end{hindi}}
\flushright{\begin{Arabic}
\quranayah[9][73]
\end{Arabic}}
\flushleft{\begin{hindi}
ऐ रसूल कुफ्फ़ार के साथ (तलवार से) और मुनाफिकों के साथ (ज़बान से) जिहाद करो और उन पर सख्ती करो और उनका ठिकाना तो जहन्नुम ही है और वह (क्या) बुरी जगह है
\end{hindi}}
\flushright{\begin{Arabic}
\quranayah[9][74]
\end{Arabic}}
\flushleft{\begin{hindi}
ये मुनाफेक़ीन ख़ुदा की क़समें खाते है कि (कोई बुरी बात) नहीं कही हालॉकि उन लोगों ने कुफ़्र का कलमा ज़रूर कहा और अपने इस्लाम के बाद काफिर हो गए और जिस बात पर क़ाबू न पा सके उसे ठान बैठे और उन लोगें ने (मुसलमानों से) सिर्फ इस वजह से अदावत की कि अपने फज़ल व करम से ख़ुदा और उसके रसूल ने दौलत मन्द बना दिया है तो उनके लिए उसमें ख़ैर है कि ये लोग अब भी तौबा कर लें और अगर ये न मानेगें तो ख़ुदा उन पर दुनिया और आख़िरत में दर्दनाक अज़ाब नाज़िल फरमाएगा और तमाम दुनिया में उन का न कोई हामी होगा और न मददगार
\end{hindi}}
\flushright{\begin{Arabic}
\quranayah[9][75]
\end{Arabic}}
\flushleft{\begin{hindi}
और इन (मुनाफेक़ीन) में से बाज़ ऐसे भी हैं जो ख़ुदा से क़ौल क़रार कर चुके थे कि अगर हमें अपने फज़ल (व करम) से (कुछ माल) देगा तो हम ज़रूर ख़ैरात किया करेगें और नेकोकार बन्दे हो जाऎंगे
\end{hindi}}
\flushright{\begin{Arabic}
\quranayah[9][76]
\end{Arabic}}
\flushleft{\begin{hindi}
तो जब ख़ुदा ने अपने फज़ल (व करम) से उन्हें अता फरमाया-तो लगे उसमें बुख्ल करने और कतराकर मुंह फेरने
\end{hindi}}
\flushright{\begin{Arabic}
\quranayah[9][77]
\end{Arabic}}
\flushleft{\begin{hindi}
फिर उनसे उनके ख़ामयाजे (बदले) में अपनी मुलाक़ात के दिन (क़यामत) तक उनके दिल में (गोया खुद) निफाक डाल दिया इस वजह से उन लोगों ने जो ख़ुदा से वायदा किया था उसके ख़िलाफ किया और इस वजह से कि झूठ बोला करते थे
\end{hindi}}
\flushright{\begin{Arabic}
\quranayah[9][78]
\end{Arabic}}
\flushleft{\begin{hindi}
क्या वह लोग इतना भी न जानते थे कि ख़ुदा (उनके) सारे भेद और उनकी सरगोशी (सब कुछ) जानता है और ये कि ग़ैब की बातों से ख़ूब आगाह है
\end{hindi}}
\flushright{\begin{Arabic}
\quranayah[9][79]
\end{Arabic}}
\flushleft{\begin{hindi}
जो लोग दिल खोलकर ख़ैरात करने वाले मोमिनीन पर (रियाकारी का) और उन मोमिनीन पर जो सिर्फ अपनी शफ़क़क़्त (मेहनत) की मज़दूरी पाते (शेख़ी का) इल्ज़ाम लगाते हैं फिर उनसे मसख़रापन करते तो ख़ुदा भी उन से मसख़रापन करेगा और उनके लिए दर्दनाक अज़ाब है
\end{hindi}}
\flushright{\begin{Arabic}
\quranayah[9][80]
\end{Arabic}}
\flushleft{\begin{hindi}
(ऐ रसूल) ख्वाह तुम उन (मुनाफिक़ों) के लिए मग़फिरत की दुआ मॉगों या उनके लिए मग़फिरत की दुआ न मॉगों (उनके लिए बराबर है) तुम उनके लिए सत्तर बार भी बख्शिस की दुआ मांगोगे तो भी ख़ुदा उनको हरगिज़ न बख्शेगा ये (सज़ा) इस सबब से है कि उन लोगों ने ख़ुदा और उसके रसूल के साथ कुफ्र किया और ख़ुदा बदकार लोगों को मंज़िलें मकसूद तक नहीं पहुँचाया करता
\end{hindi}}
\flushright{\begin{Arabic}
\quranayah[9][81]
\end{Arabic}}
\flushleft{\begin{hindi}
(जंगे तबूक़ में) रसूले ख़ुदा के पीछे रह जाने वाले अपनी जगह बैठ रहने (और जिहाद में न जाने) से ख़ुश हुए और अपने माल और आपनी जानों से ख़ुदा की राह में जिहाद करना उनको मकरू मालूम हुआ और कहने लगे (इस) गर्मी में (घर से) न निकलो (ऐ रसूल) तुम कह दो कि जहन्नुम की आग (जिसमें तुम चलोगे उससे कहीं ज्यादा गर्म है
\end{hindi}}
\flushright{\begin{Arabic}
\quranayah[9][82]
\end{Arabic}}
\flushleft{\begin{hindi}
अगर वह कुछ समझें जो कुछ वह किया करते थे उसके बदले उन्हें चाहिए कि वह बहुत कम हॅसें और बहुत रोएँ
\end{hindi}}
\flushright{\begin{Arabic}
\quranayah[9][83]
\end{Arabic}}
\flushleft{\begin{hindi}
तो (ऐ रसूल) अगर ख़ुदा तुम इन मुनाफेक़ीन के किसी गिरोह की तरफ (जिहाद से सहीसालिम) वापस लाए फिर तुमसे (जिहाद के वास्ते) निकलने की इजाज़त माँगें तो तुम साफ़ कह दो कि तुम मेरे साथ (जिहाद के वास्ते) हरगिज़ न निकलने पाओगे और न हरगिज़ दुश्मन से मेरे साथ लड़ने पाओगे जब तुमने पहली मरतबा (घर में) बैठे रहना पसन्द किया तो (अब भी) पीछे रह जाने वालों के साथ (घर में) बैठे रहो
\end{hindi}}
\flushright{\begin{Arabic}
\quranayah[9][84]
\end{Arabic}}
\flushleft{\begin{hindi}
और (ऐ रसूल) उन मुनाफिक़ीन में से जो मर जाए तो कभी ना किसी पर नमाजे ज़नाज़ा पढ़ना और न उसकी क़ब्र पर (जाकर) खडे होना इन लोगों ने यक़ीनन ख़ुदा और उसके रसूल के साथ कुफ़्र किया और बदकारी की हालत में मर (भी) गए
\end{hindi}}
\flushright{\begin{Arabic}
\quranayah[9][85]
\end{Arabic}}
\flushleft{\begin{hindi}
और उनके माल और उनकी औलाद (की कसरत) तुम्हें ताज्जुब (हैरत) में न डाले (क्योकि) ख़ुदा तो बस ये चाहता है कि दुनिया में भी उनके माल और औलाद की बदौलत उनको अज़ाब में मुब्तिला करे और उनकी जान निकालने लगे तो उस वक्त भी ये काफ़िर (के काफ़िर ही) रहें
\end{hindi}}
\flushright{\begin{Arabic}
\quranayah[9][86]
\end{Arabic}}
\flushleft{\begin{hindi}
और जब कोई सूरा इस बारे में नाज़िल हुआ कि ख़ुदा को मानों और उसके रसूल के साथ जिहाद करो तो जो उनमें से दौलत वाले हैं वह तुमसे इजाज़त मांगते हैं और कहते हैं कि हमें (यहीं छोड़ दीजिए) कि हम भी (घर बैठने वालो के साथ (बैठे) रहें
\end{hindi}}
\flushright{\begin{Arabic}
\quranayah[9][87]
\end{Arabic}}
\flushleft{\begin{hindi}
ये इस बात से ख़ुश हैं कि पीछे रह जाने वालों (औरतों, बच्चों, बीमारो के साथ बैठे) रहें और (गोया) उनके दिल पर मोहर कर दी गई तो ये कुछ नहीं समझतें
\end{hindi}}
\flushright{\begin{Arabic}
\quranayah[9][88]
\end{Arabic}}
\flushleft{\begin{hindi}
मगर रसूल और जो लोग उनके साथ ईमान लाए हैं उन लोगों ने अपने अपने माल और अपनी अपनी जानों से जिहाद किया- यही वह लोग हैं जिनके लिए (हर तरह की) भलाइयाँ हैं और यही लोग कामयाब होने वाले हैं
\end{hindi}}
\flushright{\begin{Arabic}
\quranayah[9][89]
\end{Arabic}}
\flushleft{\begin{hindi}
ख़ुदा ने उनके वास्ते (बेहश्त) के वह (हरे भरे) बाग़ तैयार कर रखे हैं जिनके (दरख्तों के) नीचे नहरे जारी हैं (और ये) इसमें हमेशा रहेंगें यही तो बड़ी कामयाबी हैं
\end{hindi}}
\flushright{\begin{Arabic}
\quranayah[9][90]
\end{Arabic}}
\flushleft{\begin{hindi}
और (तुम्हारे पास) कुछ हीला करने वाले गवार देहाती (भी) आ मौजदू हुए ताकि उनको भी (पीछे रह जाने की) इजाज़त दी जाए और जिन लोगों ने ख़ुदा और उसके रसूल से झूठ कहा था वह (घर में) बैठ रहे (आए तक नहीं) उनमें से जिन लोगों ने कुफ़्र एख्तेयार किया अनक़रीब ही उन पर दर्दनाक अज़ाब आ पहुँचेगा
\end{hindi}}
\flushright{\begin{Arabic}
\quranayah[9][91]
\end{Arabic}}
\flushleft{\begin{hindi}
(ऐ रसूल जिहाद में न जाने का) न तो कमज़ोरों पर कुछ गुनाह है न बीमारों पर और न उन लोगों पर जो कुछ नहीं पाते कि ख़र्च करें बशर्ते कि ये लोग ख़ुदा और उसके रसूल की ख़ैर ख्वाही करें नेकी करने वालों पर (इल्ज़ाम की) कोई सबील नहीं और ख़ुदा बड़ा बख़्शने वाला मेहरबान है
\end{hindi}}
\flushright{\begin{Arabic}
\quranayah[9][92]
\end{Arabic}}
\flushleft{\begin{hindi}
और न उन्हीं लोगों पर कोई इल्ज़ाम है जो तुम्हारे पास आए कि तुम उनके लिए सवारी बाहम पहुँचा दो और तुमने कहा कि मेरे पास (तो कोई सवारी) मौजूद नहीं कि तुमको उस पर सवार करूँ तो वह लोग (मजबूरन) फिर गए और हसरत (व अफसोस) उसे उस ग़म में कि उन को ख़र्च मयस्सर न आया
\end{hindi}}
\flushright{\begin{Arabic}
\quranayah[9][93]
\end{Arabic}}
\flushleft{\begin{hindi}
उनकी ऑंखों से ऑंसू जारी थे (इल्ज़ाम की) सबील तो सिर्फ उन्हीं लोगों पर है जिन्होंने बावजूद मालदार होने के तुमसे (जिहाद में) न जाने की इजाज़त चाही और उनके पीछे रह जाने वाले (औरतों, बच्चों) के साथ रहना पसन्द आया और ख़ुदा ने उनके दिलों पर (गोया) मोहर कर दी है तो ये लोग कुछ नहीं जानते
\end{hindi}}
\flushright{\begin{Arabic}
\quranayah[9][94]
\end{Arabic}}
\flushleft{\begin{hindi}
जब तुम उनके पास (जिहाद से लौट कर) वापस आओगे तो ये (मुनाफिक़ीन) तुमसे (तरह तरह) की माअज़रत करेंगे (ऐ रसूल) तुम कह दो कि बातें न बनाओ हम हरग़िज़ तुम्हारी बात न मानेंगे (क्योंकि) हमे तो ख़ुदा ने तुम्हारे हालात से आगाह कर दिया है अनक़ीरब ख़ुदा और उसका रसूल तुम्हारी कारस्तानी को मुलाहज़ा फरमाएगें फिर तुम ज़ाहिर व बातिन के जानने वालों (ख़ुदा) की हुज़ूरी में लौटा दिए जाओगे तो जो कुछ तुम (दुनिया में) करते थे (ज़र्रा ज़र्रा) बता देगा
\end{hindi}}
\flushright{\begin{Arabic}
\quranayah[9][95]
\end{Arabic}}
\flushleft{\begin{hindi}
जब तुम उनके पास (जिहाद से) वापस आओगे तो तुम्हारे सामने ख़ुदा की क़समें खाएगें ताकि तुम उनसे दरगुज़र करो तो तुम उनकी तरफ से मुँह फेर लो बेशक ये लोग नापाक हैं और उनका ठिकाना जहन्नुम है (ये) सज़ा है उसकी जो ये (दुनिया में) किया करते थे
\end{hindi}}
\flushright{\begin{Arabic}
\quranayah[9][96]
\end{Arabic}}
\flushleft{\begin{hindi}
तुम्हारे सामने ये लोग क़समें खाते हैं ताकि तुम उनसे राज़ी हो (भी) जाओ तो ख़ुदा बदकार लोगों से हरगिज़ कभी राज़ी नहीं होगा
\end{hindi}}
\flushright{\begin{Arabic}
\quranayah[9][97]
\end{Arabic}}
\flushleft{\begin{hindi}
(ये) अरब के गॅवार देहाती कुफ्र व निफाक़ में बड़े सख्त हैं और इसी क़ाबिल हैं कि जो किताब ख़ुदा ने अपने रसूल पर नाज़िल फरमाई है उसके एहक़ाम न जानें और ख़ुदा तो बड़ा दाना हकीम है
\end{hindi}}
\flushright{\begin{Arabic}
\quranayah[9][98]
\end{Arabic}}
\flushleft{\begin{hindi}
और कुछ गॅवार देहाती (ऐसे भी हैं कि जो कुछ ख़ुदा की) राह में खर्च करते हैं उसे तावान (जुर्माना) समझते हैं और तुम्हारे हक़ में (ज़माने की) गर्दिशों के मुन्तज़िर (इन्तेज़ार में) हैं उन्हीं पर (ज़माने की) बुरी गर्दिश पड़े और ख़ुदा तो सब कुछ सुनता जानता है
\end{hindi}}
\flushright{\begin{Arabic}
\quranayah[9][99]
\end{Arabic}}
\flushleft{\begin{hindi}
और कुछ देहाती तो ऐसे भी हैं जो ख़ुदा और आख़िरत पर ईमान रखते हैं और जो कुछ खर्च करते है उसे ख़ुदा की (बारगाह में) नज़दीकी और रसूल की दुआओं का ज़रिया समझते हैं आगाह रहो वाक़ई ये (ख़ैरात) ज़रूर उनके तक़र्रुब (क़रीब होने का) का बाइस है ख़ुदा उन्हें बहुत जल्द अपनी रहमत में दाख़िल करेगा बेशक ख़ुदा बड़ा बख़्शने वाला मेहरबान है
\end{hindi}}
\flushright{\begin{Arabic}
\quranayah[9][100]
\end{Arabic}}
\flushleft{\begin{hindi}
और मुहाजिरीन व अन्सार में से (ईमान की तरफ) सबक़त (पहल) करने वाले और वह लोग जिन्होंने नेक नीयती से (कुबूले ईमान में उनका साथ दिया ख़ुदा उनसे राज़ी और वह ख़ुदा से ख़ुश और उनके वास्ते ख़ुदा ने (वह हरे भरे) बाग़ जिन के नीचे नहरें जारी हैं तैयार कर रखे हैं वह हमेशा अब्दआलाबाद (हमेशा) तक उनमें रहेगें यही तो बड़ी कामयाबी हैं
\end{hindi}}
\flushright{\begin{Arabic}
\quranayah[9][101]
\end{Arabic}}
\flushleft{\begin{hindi}
और (मुसलमानों) तुम्हारे एतराफ़ (आस पास) के गॅवार देहातियों में से बाज़ मुनाफिक़ (भी) हैं और ख़ुद मदीने के रहने वालों मे से भी (बाज़ मुनाफिक़ हैं) जो निफ़ाक पर अड़ गए हैं (ऐ रसूल) तुम उन को नहीं जानते (मगर) हम उनको (ख़ूब) जानते हैं अनक़रीब हम (दुनिया में) उनकी दोहरी सज़ा करेगें फिर ये लोग (क़यामत में) एक बड़े अज़ाब की तरफ लौटाए जाऎंगे
\end{hindi}}
\flushright{\begin{Arabic}
\quranayah[9][102]
\end{Arabic}}
\flushleft{\begin{hindi}
और कुछ लोग हैं जिन्होंने अपने गुनाहों का (तो) एकरार किया (मगर) उन लोगों ने भले काम को और कुछ बुरे काम को मिला जुला (कर गोलमाल) कर दिया क़रीब है कि ख़ुदा उनकी तौबा कुबूल करे (क्योंकि) ख़ुदा तो यक़ीनी बड़ा बख़्शने वाला मेहरबान हैं
\end{hindi}}
\flushright{\begin{Arabic}
\quranayah[9][103]
\end{Arabic}}
\flushleft{\begin{hindi}
(ऐ रसूल) तुम उनके माल की ज़कात लो (और) इसकी बदौलत उनको (गुनाहो से) पाक साफ करों और उनके वास्ते दुआएं ख़ैर करो क्योंकि तुम्हारी दुआ इन लोगों के हक़ में इत्मेनान (का बाइस है) और ख़ुदा तो (सब कुछ) सुनता (और) जानता है
\end{hindi}}
\flushright{\begin{Arabic}
\quranayah[9][104]
\end{Arabic}}
\flushleft{\begin{hindi}
क्या इन लोगों ने इतने भी नहीं जाना यक़ीनन ख़ुदा बन्दों की तौबा क़ुबूल करता है और वही ख़ैरातें (भी) लेता है और इसमें शक़ नहीं कि वही तौबा का बड़ा कुबूल करने वाला मेहरबान है
\end{hindi}}
\flushright{\begin{Arabic}
\quranayah[9][105]
\end{Arabic}}
\flushleft{\begin{hindi}
और (ऐ रसूल) तुम कह दो कि तुम लोग अपने अपने काम किए जाओ अभी तो ख़ुदा और उसका रसूल और मोमिनीन तुम्हारे कामों को देखेगें और बहुत जल्द (क़यामत में) ज़ाहिर व बातिन के जानने वाले (ख़ुदा) की तरफ लौटाए जाऎंगे तब वह जो कुछ भी तुम करते थे तुम्हें बता देगा
\end{hindi}}
\flushright{\begin{Arabic}
\quranayah[9][106]
\end{Arabic}}
\flushleft{\begin{hindi}
और कुछ लोग हैं जो हुक्मे ख़ुदा के उम्मीदवार किए गए हैं (उसको अख्तेयार है) ख्वाह उन पर अज़ाब करे या उन पर मेहरबानी करे और ख़ुदा (तो) बड़ा वाकिफकार हिकमत वाला है
\end{hindi}}
\flushright{\begin{Arabic}
\quranayah[9][107]
\end{Arabic}}
\flushleft{\begin{hindi}
और (वह लोग भी मुनाफिक़ हैं) जिन्होने (मुसलमानों के) नुकसान पहुंचाने और कुफ़्र करने वाले और मोमिनीन के दरमियान तफरक़ा (फूट) डालते और उस शख़्स की घात में बैठने के वास्ते मस्जिद बनाकर खड़ी की है जो ख़ुदा और उसके रसूल से पहले लड़ चुका है (और लुत्फ़ तो ये है कि) ज़रूर क़समें खाएगें कि हमने भलाई के सिवा कुछ और इरादा ही नहीं किया और ख़ुदा ख़ुद गवाही देता है
\end{hindi}}
\flushright{\begin{Arabic}
\quranayah[9][108]
\end{Arabic}}
\flushleft{\begin{hindi}
ये लोग यक़ीनन झूठे है (ऐ रसूल) तुम इस (मस्जिद) में कभी खड़े भी न होना वह मस्जिद जिसकी बुनियाद अव्वल रोज़ से परहेज़गारी पर रखी गई है वह ज़रूर उसकी ज्यादा हक़दार है कि तुम उसमें खडे होकर (नमाज़ पढ़ो क्योंकि) उसमें वह लोग हैं जो पाक व पाकीज़ा रहने को पसन्द करते हैं और ख़ुदा भी पाक व पाकीज़ा रहने वालों को दोस्त रखता है
\end{hindi}}
\flushright{\begin{Arabic}
\quranayah[9][109]
\end{Arabic}}
\flushleft{\begin{hindi}
क्या जिस शख़्स ने ख़ुदा के ख़ौफ और ख़ुशनूदी पर अपनी इमारत की बुनियाद डाली हो वह ज्यादा अच्छा है या वह शख़्स जिसने अपनी इमारत की बुनियाद इस बोदे किनारे के लब पर रखी हो जिसमें दरार पड़ चुकी हो और अगर वह चाहता हो फिर उसे ले दे के जहन्नुम की आग में फट पडे और ख़ुदा ज़ालिम लोगों को मंज़िलें मक़सूद तक नहीं पहुंचाया करता
\end{hindi}}
\flushright{\begin{Arabic}
\quranayah[9][110]
\end{Arabic}}
\flushleft{\begin{hindi}
(ये इमारत की) बुनियाद जो उन लोगों ने क़ायम की उसके सबब से उनके दिलो में हमेशा धरपकड़ रहेगी यहाँ तक कि उनके दिलों के परख़चे उड़ जाएँ और ख़ुदा तो बड़ा वाक़िफकार हकीम हैं
\end{hindi}}
\flushright{\begin{Arabic}
\quranayah[9][111]
\end{Arabic}}
\flushleft{\begin{hindi}
इसमें तो शक़ ही नहीं कि ख़ुदा ने मोमिनीन से उनकी जानें और उनके माल इस बात पर ख़रीद लिए हैं कि (उनकी क़ीमत) उनके लिए बेहश्त है (इसी वजह से) ये लोग ख़ुदा की राह में लड़ते हैं तो (कुफ्फ़ार को) मारते हैं और ख़ुद (भी) मारे जाते हैं (ये) पक्का वायदा है (जिसका पूरा करना) ख़ुदा पर लाज़िम है और ऐसा पक्का है कि तौरैत और इन्जील और क़ुरान (सब) में (लिखा हुआ है) और अपने एहद का पूरा करने वाला ख़ुदा से बढ़कर कौन है तुम तो अपनी ख़रीद फरोख्त से जो तुमने ख़ुदा से की है खुशियाँ मनाओ यही तो बड़ी कामयाबी है
\end{hindi}}
\flushright{\begin{Arabic}
\quranayah[9][112]
\end{Arabic}}
\flushleft{\begin{hindi}
(ये लोग) तौबा करने वाले इबादत गुज़ार (ख़ुदा की) हम्दो सना (तारीफ़) करने वाले (उस की राह में) सफर करने वाले रूकूउ करने वाले सजदा करने वाले नेक काम का हुक्म करने वाले और बुरे काम से रोकने वाले और ख़ुदा की (मुक़र्रर की हुई) हदो को निगाह रखने वाले हैं और (ऐ रसूल) उन मोमिनीन को (बेहिश्त की) ख़ुशख़बरी दे दो
\end{hindi}}
\flushright{\begin{Arabic}
\quranayah[9][113]
\end{Arabic}}
\flushleft{\begin{hindi}
नबी और मोमिनीन पर जब ज़ाहिर हो चुका कि मुशरेकीन जहन्नुमी है तो उसके बाद मुनासिब नहीं कि उनके लिए मग़फिरत की दुआएं माँगें अगरचे वह मुशरेकीन उनके क़राबतदार हो (क्यों न) हो
\end{hindi}}
\flushright{\begin{Arabic}
\quranayah[9][114]
\end{Arabic}}
\flushleft{\begin{hindi}
और इबराहीम का अपने बाप के लिए मग़फिरत की दुआ माँगना सिर्फ इस वायदे की वजह से था जो उन्होंने अपने बाप से कर लिया था फिर जब उनको मालूम हो गया कि वह यक़ीनी ख़ुदा का दुश्मन है तो उससे बेज़ार हो गए, बेशक इबराहीम यक़ीनन बड़े दर्दमन्द बुर्दबार (सहन करने वाले) थे
\end{hindi}}
\flushright{\begin{Arabic}
\quranayah[9][115]
\end{Arabic}}
\flushleft{\begin{hindi}
ख़ुदा की ये शान नहीं कि किसी क़ौम को जब उनकी हिदायत कर चुका हो उसके बाद बेशक ख़ुदा उन्हें गुमराह कर दे हत्ता (यहां तक) कि वह उन्हीं चीज़ों को बता दे जिससे वह परहेज़ करें बेशक ख़ुदा हर चीज़ से (वाक़िफ है)
\end{hindi}}
\flushright{\begin{Arabic}
\quranayah[9][116]
\end{Arabic}}
\flushleft{\begin{hindi}
इसमें तो शक़ ही नहीं कि सारे आसमान व ज़मीन की हुकूमत ख़ुदा ही के लिए ख़ास है वही (जिसे चाहे) जिलाता है और (जिसे चाहे) मारता है और तुम लोगों का ख़ुदा के सिवा न कोई सरपरस्त है न मददगार
\end{hindi}}
\flushright{\begin{Arabic}
\quranayah[9][117]
\end{Arabic}}
\flushleft{\begin{hindi}
अलबत्ता ख़ुदा ने नबी और उन मुहाजिरीन अन्सार पर बड़ा फज़ल किया जिन्होंने तंगदस्ती के वक्त रसूल का साथ दिया और वह भी उसके बाद कि क़रीब था कि उनमे ंसे कुछ लोगों के दिल जगमगा जाएँ फिर ख़ुदा ने उन पर (भी) फज़ल किया इसमें शक़ नहीं कि वह उन लोगों पर पड़ा तरस खाने वाला मेहरबान है
\end{hindi}}
\flushright{\begin{Arabic}
\quranayah[9][118]
\end{Arabic}}
\flushleft{\begin{hindi}
और उन यमीमों पर (भी फज़ल किया) जो (जिहाद से पीछे रह गए थे और उन पर सख्ती की गई) यहाँ तक कि ज़मीन बावजूद उस वसअत (फैलाव) के उन पर तंग हो गई और उनकी जानें (तक) उन पर तंग हो गई और उन लोगों ने समझ लिया कि ख़ुदा के सिवा और कहीं पनाह की जगह नहीं फिर ख़ुदा ने उनको तौबा की तौफीक दी ताकि वह (ख़ुदा की तरफ) रूजू करें बेशक ख़ुदा ही बड़ा तौबा क़ुबूल करने वाला मेहरबान है
\end{hindi}}
\flushright{\begin{Arabic}
\quranayah[9][119]
\end{Arabic}}
\flushleft{\begin{hindi}
ऐ ईमानदारों ख़ुदा से डरो और सच्चों के साथ हो जाओ
\end{hindi}}
\flushright{\begin{Arabic}
\quranayah[9][120]
\end{Arabic}}
\flushleft{\begin{hindi}
मदीने के रहने वालों और उनके गिर्दोनवॉ (आस पास) देहातियों को ये जायज़ न था कि रसूल ख़ुदा का साथ छोड़ दें और न ये (जायज़ था) कि रसूल की जान से बेपरवा होकर अपनी जानों के बचाने की फ्रिक करें ये हुक्म उसी सबब से था कि उन (जिहाद करने वालों) को ख़ुदा की रूह में जो तकलीफ़ प्यास की या मेहनत या भूख की शिद्दत की पहुँचती है या ऐसी राह चलते हैं जो कुफ्फ़ार के ग़ैज़ (ग़ज़ब का बाइस हो या किसी दुश्मन से कुछ ये लोग हासिल करते हैं तो बस उसके ऐवज़ में (उनके नामए अमल में) एक नेक काम लिख दिया जाएगा बेशक ख़ुदा नेकी करने वालों का अज्र (व सवाब) बरबाद नहीं करता है
\end{hindi}}
\flushright{\begin{Arabic}
\quranayah[9][121]
\end{Arabic}}
\flushleft{\begin{hindi}
और ये लोग (ख़ुदा की राह में) थोड़ा या बहुत माल नहीं खर्च करते और किसी मैदान को नहीं क़तआ करते मगर फौरन (उनके नामाए अमल में) उनके नाम लिख दिया जाता है ताकि ख़ुदा उनकी कारगुज़ारियों का उन्हें अच्छे से अच्छा बदला अता फरमाए
\end{hindi}}
\flushright{\begin{Arabic}
\quranayah[9][122]
\end{Arabic}}
\flushleft{\begin{hindi}
और ये भी मुनासिब नहीं कि मोमिननि कुल के कुल (अपने घरों में) निकल खड़े हों उनमें से हर गिरोह की एक जमाअत (अपने घरों से) क्यों नहीं निकलती ताकि इल्मे दीन हासिल करे और जब अपनी क़ौम की तरफ पलट के आवे तो उनको (अज्र व आख़िरत से) डराए ताकि ये लोग डरें
\end{hindi}}
\flushright{\begin{Arabic}
\quranayah[9][123]
\end{Arabic}}
\flushleft{\begin{hindi}
ऐ ईमानदारों कुफ्फार में से जो लोग तुम्हारे आस पास के है उन से लड़ों और (इस तरह लड़ना) चाहिए कि वह लोग तुम में करारापन महसूस करें और जान रखो कि बेशुबहा ख़ुदा परहेज़गारों के साथ है
\end{hindi}}
\flushright{\begin{Arabic}
\quranayah[9][124]
\end{Arabic}}
\flushleft{\begin{hindi}
और जब कोई सूरा नाज़िल किया गया तो उन मुनाफिक़ीन में से (एक दूसरे से) पूछता है कि भला इस सूरे ने तुममें से किसी का ईमान बढ़ा दिया तो जो लोग ईमान ला चुके हैं उनका तो इस सूरे ने ईमान बढ़ा दिया और वह वहां उसकी ख़ुशियाँ मनाते है
\end{hindi}}
\flushright{\begin{Arabic}
\quranayah[9][125]
\end{Arabic}}
\flushleft{\begin{hindi}
मगर जिन लोगों के दिल में (निफाक़ की) बीमारी है तो उन (पिछली) ख़बासत पर इस सूरो ने एक ख़बासत और बढ़ा दी और ये लोग कुफ़्र ही की हालत में मर गए
\end{hindi}}
\flushright{\begin{Arabic}
\quranayah[9][126]
\end{Arabic}}
\flushleft{\begin{hindi}
क्या वह लोग (इतना भी) नहीं देखते कि हर साल एक मरतबा या दो मरतबा बला में मुबितला किए जाते हैं फिर भी न तो ये लोग तौबा ही करते हैं और न नसीहत ही मानते हैं
\end{hindi}}
\flushright{\begin{Arabic}
\quranayah[9][127]
\end{Arabic}}
\flushleft{\begin{hindi}
और जब कोई सूरा नाज़िल किया गया तो उसमें से एक की तरफ एक देखने लगा (और ये कहकर कि) तुम को कोई मुसलमान देखता तो नहीं है फिर (अपने घर) पलट जाते हैं (ये लोग क्या पलटेगें गोया) ख़ुदा ने उनके दिलों को पलट दिया है इस सबब से कि ये बिल्कुल नासमझ लोग हैं
\end{hindi}}
\flushright{\begin{Arabic}
\quranayah[9][128]
\end{Arabic}}
\flushleft{\begin{hindi}
लोगों तुम ही में से (हमारा) एक रसूल तुम्हारे पास आ चुका (जिसकी शफ़क्क़त (मेहरबानी) की ये हालत है कि) उस पर शाक़ (दुख) है कि तुम तकलीफ उठाओ और उसे तुम्हारी बेहूदी का हौका है ईमानदारो पर हद दर्जे रफ़ीक़ मेहरबान हैं
\end{hindi}}
\flushright{\begin{Arabic}
\quranayah[9][129]
\end{Arabic}}
\flushleft{\begin{hindi}
ऐ रसूल अगर इस पर भी ये लोग (तुम्हारे हुक्म से) मुँह फेरें तो तुम कह दो कि मेरे लिए ख़ुदा काफी है उसके सिवा कोई माबूद नहीं मैने उस पर भरोसा रखा है वही अर्श (ऐसे) बुर्जूग (मख़लूका का) मालिक है
\end{hindi}}
\chapter{Yunus (Jonah)}
\begin{Arabic}
\Huge{\centerline{\basmalah}}\end{Arabic}
\flushright{\begin{Arabic}
\quranayah[10][1]
\end{Arabic}}
\flushleft{\begin{hindi}
अलिफ़ लाम रा ये आयतें उस किताब की हैं जो अज़सरतापा (सर से पैर तक) हिकमत से मलूउ (भरी) है
\end{hindi}}
\flushright{\begin{Arabic}
\quranayah[10][2]
\end{Arabic}}
\flushleft{\begin{hindi}
क्या लोगों को इस बात से बड़ा ताज्जुब हुआ कि हमने उन्हीं लोगों में से एक आदमी के पास वही भेजी कि (बे ईमान) लोगों को डराओ और ईमानदारो को इसकी ख़ुश ख़बरी सुना दो कि उनके लिए उनके परवरदिगार की बारगाह में बुलन्द दर्जे है (मगर) कुफ्फार (उन आयतों को सुनकर) कहने लगे कि ये (शख्स तो यक़ीनन सरीही जादूगर) है
\end{hindi}}
\flushright{\begin{Arabic}
\quranayah[10][3]
\end{Arabic}}
\flushleft{\begin{hindi}
इसमें तो शक़ ही नहीं कि तुमरा परवरदिगार वही ख़ुदा है जिसने सारे आसमान व ज़मीन को 6 दिन में पैदा किया फिर उसने अर्श को बुलन्द किया वही हर काम का इन्तज़ाम करता है (उसके सामने) कोई (किसी का) सिफारिशी नहीं (हो सकता) मगर उसकी इजाज़त के बाद वही ख़ुदा तो तुम्हारा परवरदिगार है तो उसी की इबादत करो तो क्या तुम अब भी ग़ौर नही करते
\end{hindi}}
\flushright{\begin{Arabic}
\quranayah[10][4]
\end{Arabic}}
\flushleft{\begin{hindi}
तुम सबको (आख़िर) उसी की तरफ लौटना है ख़ुदा का वायदा सच्चा है वही यक़ीनन मख़लूक को पहली मरतबा पैदा करता है फिर (मरने के बाद) वही दुबारा जिन्दा करेगा ताकि जिन लोगों ने ईमान कुबूल किया और अच्छे अच्छे काम किए उनको इन्साफ के साथ जज़ाए (ख़ैर) अता फरमाएगा और जिन लोगों ने कुफ्र एख्तियार किया उन के लिए उनके कुफ्र की सज़ा में पीने को खौलता हुआ पानी और दर्दनाक अज़ाब होगा
\end{hindi}}
\flushright{\begin{Arabic}
\quranayah[10][5]
\end{Arabic}}
\flushleft{\begin{hindi}
वही वह (ख़ुदाए क़ादिर) है जिसने आफ़ताब को चमकदार और महताब को रौशन बनाया और उसकी मंज़िलें मुक़र्रर की ताकि तुम लोग बरसों की गिनती और हिसाब मालूम करो ख़ुदा ने उसे हिकमत व मसलहत से बनाया है वह (अपनी) आयतों का वाक़िफ़कार लोगों के लिए तफ़सीलदार बयान करता है
\end{hindi}}
\flushright{\begin{Arabic}
\quranayah[10][6]
\end{Arabic}}
\flushleft{\begin{hindi}
इसमें ज़रा भी शक़ नहीं कि रात दिन के उलट फेर में और जो कुछ ख़ुदा ने आसमानों और ज़मीन में बनाया है (उसमें) परहेज़गारों के वास्ते बहुतेरी निशानियाँ हैं
\end{hindi}}
\flushright{\begin{Arabic}
\quranayah[10][7]
\end{Arabic}}
\flushleft{\begin{hindi}
इसमें भी शक़ नहीं कि जिन लोगों को (क़यामत में) हमारी (बारगाह की) हुज़ूरी का ठिकाना नहीं और दुनिया की (चन्द रोज़) ज़िन्दगी से निहाल हो गए और उसी पर चैन से बैठे हैं और जो लोग हमारी आयतों से ग़ाफिल हैं
\end{hindi}}
\flushright{\begin{Arabic}
\quranayah[10][8]
\end{Arabic}}
\flushleft{\begin{hindi}
यही वह लोग हैं जिनका ठिकाना उनकी करतूत की बदौलत जहन्नुम है
\end{hindi}}
\flushright{\begin{Arabic}
\quranayah[10][9]
\end{Arabic}}
\flushleft{\begin{hindi}
बेशक जिन लोगों ने ईमान कुबूल किया और अच्छे अच्छे काम किए उन्हें उनका परवरदिगार उनके ईमान के सबब से मंज़िल मक़सूद तक पहुँचा देगा कि आराम व आसाइश के बाग़ों में (रहेगें) और उन के नीचे नहरें जारी होगी
\end{hindi}}
\flushright{\begin{Arabic}
\quranayah[10][10]
\end{Arabic}}
\flushleft{\begin{hindi}
उन बाग़ों में उन लोगों का बस ये कौल होगा ऐ ख़ुदा तू पाक व पाकीज़ा है और उनमें उनकी बाहमी (आपसी) खैरसलाही (मुलाक़ात) सलाम से होगी और उनका आख़िरी क़ौल ये होगा कि सब तारीफ ख़ुदा ही को सज़ावार है जो सारे जहाँन का पालने वाला है
\end{hindi}}
\flushright{\begin{Arabic}
\quranayah[10][11]
\end{Arabic}}
\flushleft{\begin{hindi}
और जिस तरह लोग अपनी भलाई के लिए जल्दी कर बैठे हैं उसी तरह अगर ख़ुदा उनकी शरारतों की सज़ा में बुराई में जल्दी कर बैठता है तो उनकी मौत उनके पास कब की आ चुकी होती मगर हम तो उन लोगों को जिन्हें (मरने के बाद) हमारी हुज़ूरी का खटका नहीं छोड़ देते हैं कि वह अपनी सरकशी में आप सरग़िरदा रहें
\end{hindi}}
\flushright{\begin{Arabic}
\quranayah[10][12]
\end{Arabic}}
\flushleft{\begin{hindi}
और इन्सान को जब कोई नुकसान छू भी गया तो अपने पहलू पर (लेटा हो) या बैठा हो या ख़ड़ा (गरज़ हर हालत में) हम को पुकारता है फिर जब हम उससे उसकी तकलीफ को दूर कर देते है तो ऐसा खिसक जाता है जैसे उसने तकलीफ के (दफा करने के) लिए जो उसको पहुँचती थी हमको पुकारा ही न था जो लोग ज्यादती करते हैं उनकी कारस्तानियाँ यूँ ही उन्हें अच्छी कर दिखाई गई हैं
\end{hindi}}
\flushright{\begin{Arabic}
\quranayah[10][13]
\end{Arabic}}
\flushleft{\begin{hindi}
और तुमसे पहली उम्मत वालों को जब उन्होंने शरारत की तो हम ने उन्हें ज़रुर हलाक कर डाला हालॉकि उनके (वक्त क़े) रसूल वाजेए व रौशन मोज़िज़ात लेकर आ चुके थे और वह लोग ईमान (न लाना था) न लाए हम गुनेहगार लोगों की यूँ ही सज़ा किया करते हैं
\end{hindi}}
\flushright{\begin{Arabic}
\quranayah[10][14]
\end{Arabic}}
\flushleft{\begin{hindi}
फिर हमने उनके बाद तुमको ज़मीन में (उनका) जानशीन बनाया ताकि हम (भी) देखें कि तुम किस तरह काम करते हो
\end{hindi}}
\flushright{\begin{Arabic}
\quranayah[10][15]
\end{Arabic}}
\flushleft{\begin{hindi}
और जब उन लोगों के सामने हमारी रौशन आयते पढ़ीं जाती हैं तो जिन लोगों को (मरने के बाद) हमारी हुजूरी का खटका नहीं है वह कहते है कि हमारे सामने इसके अलावा कोई दूसरा (कुरान लाओ या उसका रद्दो बदल कर डालो (ऐ रसूल तुम कह दो कि मुझे ये एख्तेयार नहीं कि मै उसे अपने जी से बदल डालूँ मै तो बस उसी का पाबन्द हूँ जो मेरी तरफ वही की गई है मै तो अगर अपने परवरदिगार की नाफरमानी करु तो बड़े (कठिन) दिन के अज़ाब से डरता हूँ
\end{hindi}}
\flushright{\begin{Arabic}
\quranayah[10][16]
\end{Arabic}}
\flushleft{\begin{hindi}
(ऐ रसूल) कह दो कि ख़ुदा चाहता तो मै न तुम्हारे सामने इसको पढ़ता और न वह तुम्हें इससे आगाह करता क्योंकि मै तो (आख़िर) तुमने इससे पहले मुद्दतों रह चुका हूँ (और कभी 'वही' का नाम भी न लिया)
\end{hindi}}
\flushright{\begin{Arabic}
\quranayah[10][17]
\end{Arabic}}
\flushleft{\begin{hindi}
तो क्या तुम (इतना भी) नहीं समझते तो जो शख़्स ख़ुदा पर झूठ बोहतान बॉधे या उसकी आयतो को झुठलाए उससे बढ़ कर और ज़ालिम कौन होगा इसमें शक़ नहीं कि (ऐसे) गुनाहगार कामयाब नहीं हुआ करते
\end{hindi}}
\flushright{\begin{Arabic}
\quranayah[10][18]
\end{Arabic}}
\flushleft{\begin{hindi}
या लोग ख़ुदा को छोड़ कर ऐसी चीज़ की परसतिश करते है जो न उनको नुकसान ही पहुँचा सकती है न नफा और कहते हैं कि ख़ुदा के यहाँ यही लोग हमारे सिफारिशी होगे (ऐ रसूल) तुम (इनसे) कहो तो क्या तुम ख़ुदा को ऐसी चीज़ की ख़बर देते हो जिसको वह न तो आसमानों में (कहीं) पाता है और न ज़मीन में ये लोग जिस चीज़ को उसका शरीक बनाते है
\end{hindi}}
\flushright{\begin{Arabic}
\quranayah[10][19]
\end{Arabic}}
\flushleft{\begin{hindi}
उससे वह पाक साफ और बरतर है और सब लोग तो (पहले) एक ही उम्मत थे और (ऐ रसूल) अगर तुम्हारे परवरदिगार की तरफ से एक बात (क़यामत का वायदा) पहले न हो चुकी होती जिसमें ये लोग एख्तिलाफ कर रहे हैं उसका फैसला उनके दरमियान (कब न कब) कर दिया गया होता
\end{hindi}}
\flushright{\begin{Arabic}
\quranayah[10][20]
\end{Arabic}}
\flushleft{\begin{hindi}
और कहते हैं कि उस पैग़म्बर पर कोई मोजिज़ा (हमारी ख्वाहिश के मुवाफिक़) क्यों नहीं नाज़िल किया गया तो (ऐ रसूल) तुम कह दो कि ग़ैब (दानी) तो सिर्फ ख़ुदा के वास्ते ख़ास है तो तुम भी इन्तज़ार करो और तुम्हारे साथ मै (भी) यक़ीनन इन्तज़ार करने वालों में हूँ
\end{hindi}}
\flushright{\begin{Arabic}
\quranayah[10][21]
\end{Arabic}}
\flushleft{\begin{hindi}
और लोगों को जो तकलीफ पहुँची उसके बाद जब हमने अपनी रहमत का जाएक़ा चखा दिया तो यकायक उन लोगों से हमारी आयतों में हीले बाज़ी शुरू कर दी (ऐ रसूल) तुम कह दो कि तद्बीर में ख़ुदा सब से ज्यादा तेज़ है तुम जो कुछ मक्कारी करते हो वह हमारे भेजे हुए (फरिश्ते) लिखते जाते हैं
\end{hindi}}
\flushright{\begin{Arabic}
\quranayah[10][22]
\end{Arabic}}
\flushleft{\begin{hindi}
वह वही ख़ुदा है जो तुम्हें ख़ुश्की और दरिया में सैर कराता फिरता है यहाँ तक कि जब (कभी) तुम कश्तियों पर सवार होते हो और वह उन लोगों को बाद मुवाफिक़ (हवा के धारे) की मदद से लेकर चली और लोग उस (की रफ्तार) से ख़ुश हुए (यकायक) कश्ती पर हवा का एक झोंका आ पड़ा और (आना था कि) हर तरफ से उस पर लहरें (बढ़ी चली) आ रही हैं और उन लोगों ने समझ लिया कि अब घिर गए (और जान न बचेगी) तब अपने अक़ीदे को उसके वास्ते निरा खरा करके खुदा से दुआएँ मागँने लगते हैं कि (ख़ुदाया) अगर तूने इस (मुसीबत) से हमें नजात दी तो हम ज़रुर बड़े शुक्र गुज़ार होंगे
\end{hindi}}
\flushright{\begin{Arabic}
\quranayah[10][23]
\end{Arabic}}
\flushleft{\begin{hindi}
फिर जब ख़ुदा ने उन्हें नजात दी तो वह लोग ज़मीन पर (कदम रखते ही) फौरन नाहक़ सरकशी करने लगते हैं (ऐ लोगों तुम्हारी सरकशी का वबाल) तो तुम्हारी ही जान पर है - (ये भी) दुनिया की (चन्द रोज़ा) ज़िन्दगी का फायदा है फिर आख़िर हमारी (ही) तरफ तुमको लौटकर आना है तो (उस वक्त) हम तुमको जो कुछ (दुनिया में) करते थे बता देगे
\end{hindi}}
\flushright{\begin{Arabic}
\quranayah[10][24]
\end{Arabic}}
\flushleft{\begin{hindi}
दुनियावी ज़िदगी की मसल तो बस पानी की सी है कि हमने उसको आसमान से बरसाया फिर ज़मीन के साग पात जिसको लोग और चौपाए खा जाते हैं (उसके साथ मिल जुलकर निकले यहाँ तक कि जब ज़मीन ने (फसल की चीज़ों से) अपना बनाओ सिंगार कर लिया और (हर तरह) आरास्ता हो गई और खेत वालों ने समझ लिया कि अब वह उस पर यक़ीनन क़ाबू पा गए (जब चाहेंगे काट लेगे) यकायक हमारा हुक्म व अज़ाब रात या दिन को आ पहुँचा तो हमने उस खेत को ऐसा साफ कटा हुआ बना दिया कि गोया कुल उसमें कुछ था ही नहीं जो लोग ग़ौर व फिक्र करते हैं उनके वास्ते हम आयतों को यूँ तफसीलदार बयान करते है
\end{hindi}}
\flushright{\begin{Arabic}
\quranayah[10][25]
\end{Arabic}}
\flushleft{\begin{hindi}
और ख़ुदा तो आराम के घर (बेहश्त) की तरफ बुलाता है और जिसको चाहता है सीधे रास्ते की हिदायत करता है
\end{hindi}}
\flushright{\begin{Arabic}
\quranayah[10][26]
\end{Arabic}}
\flushleft{\begin{hindi}
जिन लोगों ने दुनिया में भलाई की उनके लिए (आख़िरत में भी) भलाई है (बल्कि) और कुछ बढ़कर और न (गुनेहगारों की तरह) उनके चेहरों पर कालिक लगी हुई होगी और न (उन्हें ज़िल्लत होगी यही लोग जन्नती हैं कि उसमें हमेशा रहा सहा करेंगे
\end{hindi}}
\flushright{\begin{Arabic}
\quranayah[10][27]
\end{Arabic}}
\flushleft{\begin{hindi}
और जिन लोगों ने बुरे काम किए हैं तो गुनाह की सज़ा उसके बराबर है और उन पर रुसवाई छाई होगी ख़ुदा (के अज़ाब) से उनका कोई बचाने वाला न होगा (उनके मुह ऐसे काले होंगे) गोया उनके चेहरे यबों यज़ूर (अंधेरी रात) के टुकड़े से ढक दिए गए हैं यही लोग जहन्नुमी हैं कि ये उसमें हमेशा रहेंगे
\end{hindi}}
\flushright{\begin{Arabic}
\quranayah[10][28]
\end{Arabic}}
\flushleft{\begin{hindi}
(ऐ रसूल उस दिन से डराओ) जिस दिन सब को इकट्ठा करेगें-फिर मुशरेकीन से कहेंगें कि तुम और तुम्हारे (बनाए हुए ख़ुदा के) शरीक ज़रा अपनी जगह ठहरो फिर हम वाहम उनमें फूट डाल देगें और उनके शरीक उनसे कहेंगे कि तुम तो हमारी परसतिश करते न थे
\end{hindi}}
\flushright{\begin{Arabic}
\quranayah[10][29]
\end{Arabic}}
\flushleft{\begin{hindi}
तो (अब) हमारे और तुम्हारे दरमियान गवाही के वास्ते ख़ुदा ही काफी है हम को तुम्हारी परसतिश की ख़बर ही न थी
\end{hindi}}
\flushright{\begin{Arabic}
\quranayah[10][30]
\end{Arabic}}
\flushleft{\begin{hindi}
(ग़रज़) वहाँ हर शख़्श जो कुछ जिसने पहले (दुनिया में) किया है जाँच लेगा और वह सब के सब अपने सच्चे मालिक ख़ुदा की बारगाह में लौटकर लाए जाएँगें और (दुनिया में) जो कुछ इफ़तेरा परदाज़िया (झूठी बातें) करते थे सब उनके पास से चल चंपत हो जाएगें
\end{hindi}}
\flushright{\begin{Arabic}
\quranayah[10][31]
\end{Arabic}}
\flushleft{\begin{hindi}
ऐ रसूल तुम उने ज़रा पूछो तो कि तुम्हें आसमान व ज़मीन से कौन रोज़ी देता है या (तुम्हारे) कान और (तुम्हारी) ऑंखों का कौन मालिक है और कौन शख़्श मुर्दे से ज़िन्दा को निकालता है और ज़िन्दा से मुर्दे को निकालता है और हर अम्र (काम) का बन्दोबस्त कौन करता है तो फौरन बोल उठेंगे कि ख़ुदा (ऐ रसूल) तुम कहो तो क्या तुम इस पर भी (उससे) नहीं डरते हो
\end{hindi}}
\flushright{\begin{Arabic}
\quranayah[10][32]
\end{Arabic}}
\flushleft{\begin{hindi}
फिर वही ख़ुदा तो तुम्हारा सच्चा रब है फिर हक़ बात के बाद गुमराही के सिवा और क्या है फिर तुम कहाँ फिरे चले जा रहे हो
\end{hindi}}
\flushright{\begin{Arabic}
\quranayah[10][33]
\end{Arabic}}
\flushleft{\begin{hindi}
ये तुम्हारे परवरदिगार की बात बदचलन लोगों पर साबित होकर रही कि ये लोग हरगिज़ ईमान न लाएँगें
\end{hindi}}
\flushright{\begin{Arabic}
\quranayah[10][34]
\end{Arabic}}
\flushleft{\begin{hindi}
(ऐ रसूल) उनसे पूछो तो कि तुम ने जिन लोगों को (ख़ुदा का) शरीक बनाया है कोई भी ऐसा है जो मख़लूकात को पहली बार पैदा करे फिर उन को (मरने के बाद) दोबारा ज़िन्दा करे (तो क्या जवाब देगें) तुम्ही कहो कि ख़ुदा ही पहले भी पैदा करता है फिर वही दोबारा ज़िन्दा करता है तो किधर तुम उल्टे जा रहे हो
\end{hindi}}
\flushright{\begin{Arabic}
\quranayah[10][35]
\end{Arabic}}
\flushleft{\begin{hindi}
(ऐ रसूल उनसे) कहो तो कि तुम्हारे (बनाए हुए) शरीकों में से कोई ऐसा भी है जो तुम्हें (दीन) हक़ की राह दिखा सके तुम ही कह दो कि (ख़ुदा) दीन की राह दिखाता है तो जो तुम्हे दीने हक़ की राह दिखाता है क्या वह ज्यादा हक़दार है कि उसके हुक्म की पैरवी की जाए या वह शख़्श जो (दूसरे) की हिदायत तो दर किनार खुद ही जब तक दूसरा उसको राह न दिखाए राह नही देख पाता तो तुम लोगों को क्या हो गया है
\end{hindi}}
\flushright{\begin{Arabic}
\quranayah[10][36]
\end{Arabic}}
\flushleft{\begin{hindi}
तुम कैसे हुक्म लगाते हो और उनमें के अक्सर तो बस अपने गुमान पर चलते हैं (हालॉकि) गुमान यक़ीन के मुक़ाबले में हरगिज़ कुछ भी काम नहीं आ सकता बेशक वह लोग जो कुछ (भी) कर रहे हैं खुदा उसे खूब जानता है
\end{hindi}}
\flushright{\begin{Arabic}
\quranayah[10][37]
\end{Arabic}}
\flushleft{\begin{hindi}
और ये कुरान ऐसा नहीं कि खुदा के सिवा कोई और अपनी तरफ से झूठ मूठ बना डाले बल्कि (ये तो) जो (किताबें) पहले की उसके सामने मौजूद हैं उसकी तसदीक़ और (उन) किताबों की तफ़सील है उसमें कुछ भी शक़ नहीं कि ये सारे जहाँन के परवरदिगार की तरफ से है
\end{hindi}}
\flushright{\begin{Arabic}
\quranayah[10][38]
\end{Arabic}}
\flushleft{\begin{hindi}
क्या ये लोग कहते हैं कि इसको रसूल ने खुद झूठ मूठ बना लिया है (ऐ रसूल) तुम कहो कि (अच्छा) तो तुम अगर (अपने दावे में) सच्चे हो तो (भला) एक ही सूरा उसके बराबर का बना लाओ और ख़ुदा के सिवा जिसको तुम्हें (मदद के वास्ते) बुलाते बन पड़े बुला लो
\end{hindi}}
\flushright{\begin{Arabic}
\quranayah[10][39]
\end{Arabic}}
\flushleft{\begin{hindi}
(ये लोग लाते तो क्या) बल्कि (उलटे) जिसके जानने पर उनका हाथ न पहुँचा हो लगे उसको झुठलाने हालॉकि अभी तक उनके जेहन में उसके मायने नहीं आए इसी तरह उन लोगों ने भी झुठलाया था जो उनसे पहले थे-तब ज़रा ग़ौर तो करो कि (उन) ज़ालिमों का क्या (बुरा) अन्जाम हुआ
\end{hindi}}
\flushright{\begin{Arabic}
\quranayah[10][40]
\end{Arabic}}
\flushleft{\begin{hindi}
और उनमें से बाज़ तो ऐसे है कि इस कुरान पर आइन्दा ईमान लाएगें और बाज़ ऐसे हैं जो ईमान लाएगें ही नहीं
\end{hindi}}
\flushright{\begin{Arabic}
\quranayah[10][41]
\end{Arabic}}
\flushleft{\begin{hindi}
और (ऐ रसूल) तुम्हारा परवरदिगार फसादियों को खूब जानता है और अगर वह तुम्हे झुठलाए तो तुम कह दो कि हमारे लिए हमारी कार गुजारी है और तुम्हारे लिए तुम्हारी कारस्तानी जो कुछ मै करता हूँ उसके तुम ज़िम्मेदार नहीं और जो कुछ तुम करते हो उससे मै बरी हूँ
\end{hindi}}
\flushright{\begin{Arabic}
\quranayah[10][42]
\end{Arabic}}
\flushleft{\begin{hindi}
और उनमें से बाज़ ऐसे हैं कि तुम्हारी ज़बानों की तरफ कान लगाए रहते हैं तो (क्या) वह तुम्हारी सुन लेगें हरगिज़ नहीं अगरचे वह कुछ समझ भी न सकते हो
\end{hindi}}
\flushright{\begin{Arabic}
\quranayah[10][43]
\end{Arabic}}
\flushleft{\begin{hindi}
तुम कही बहरों को कुछ सुना सकते हो और बाज़ उनमें से ऐसे हैं जो तुम्हारी तरफ (टकटकी बाँधे) देखते हैं तो (क्या वह ईमान लाएँगें हरगिज़ नहीं) अगरचे उन्हें कुछ न सूझता हो तो तुम अन्धे को राहे रास्त दिखा दोगे
\end{hindi}}
\flushright{\begin{Arabic}
\quranayah[10][44]
\end{Arabic}}
\flushleft{\begin{hindi}
ख़ुदा तो हरगिज़ लोगों पर कुछ भी ज़ुल्म नहीं करता मगर लोग खुद अपने ऊपर (अपनी करतूत से) जुल्म किया करते है
\end{hindi}}
\flushright{\begin{Arabic}
\quranayah[10][45]
\end{Arabic}}
\flushleft{\begin{hindi}
और जिस दिन ख़ुदा इन लोगों को (अपनी बारगाह में) जमा करेगा तो गोया ये लोग (समझेगें कि दुनिया में) बस घड़ी दिन भर ठहरे और आपस में एक दूसरे को पहचानेंगे जिन लोगों ने ख़ुदा की बारगाह में हाज़िर होने को झुठलाया वह ज़रुर घाटे में हैं और हिदायत याफता न थे
\end{hindi}}
\flushright{\begin{Arabic}
\quranayah[10][46]
\end{Arabic}}
\flushleft{\begin{hindi}
ऐ रसूल हम जिस जिस (अज़ाब) का उनसे वायदा कर चुके हैं उनमें से बाज़ ख्वाहा तुम्हें दिखा दें या तुमको (पहले ही दुनिया से) उठा ले फिर (आख़िर) तो उन सबको हमारी तरफ लौटना ही है फिर जो कुछ ये लोग कर रहे हैं ख़ुदा तो उस पर गवाह ही है
\end{hindi}}
\flushright{\begin{Arabic}
\quranayah[10][47]
\end{Arabic}}
\flushleft{\begin{hindi}
और हर उम्मत का ख़ास (एक) एक रसूल हुआ है फिर जब उनका रसूल (हमारी बारगाह में) आएगा तो उनके दरमियान इन्साफ़ के साथ फैसला कर दिया जाएगा और उन पर ज़र्रा बराबर ज़ुल्म न किया जाएगा
\end{hindi}}
\flushright{\begin{Arabic}
\quranayah[10][48]
\end{Arabic}}
\flushleft{\begin{hindi}
ये लोग कहा करते हैं कि अगर तुम सच्चे हो तो (आख़िर) ये (अज़ाब का वायदा) कब पूरा होगा
\end{hindi}}
\flushright{\begin{Arabic}
\quranayah[10][49]
\end{Arabic}}
\flushleft{\begin{hindi}
(ऐ रसूल) तुम कह दो कि मै खुद अपने वास्ते नुकसान पर क़ादिर हूँ न नफा पर मगर जो ख़ुदा चाहे हर उम्मत (के रहने) का (उसके इल्म में) एक वक्त मुक़र्रर है-जब उन का वक्त आ जाता है तो न एक घड़ी पीछे हट सकती हैं और न आगे बढ़ सकते हैं
\end{hindi}}
\flushright{\begin{Arabic}
\quranayah[10][50]
\end{Arabic}}
\flushleft{\begin{hindi}
(ऐ रसूल) तुम कह दो कि क्या तुम समझते हो कि अगर उसका अज़ाब तुम पर रात को या दिन को आ जाए तो (तुम क्या करोगे) फिर गुनाहगार लोग आख़िर काहे की जल्दी मचा रहे हैं
\end{hindi}}
\flushright{\begin{Arabic}
\quranayah[10][51]
\end{Arabic}}
\flushleft{\begin{hindi}
फिर क्या जब (तुम पर) आ चुकेगा तब उस पर ईमान लाओगे (आहा) क्या अब (ईमान लाए) हालॉकि तुम तो इसकी जल्दी मचाया करते थे
\end{hindi}}
\flushright{\begin{Arabic}
\quranayah[10][52]
\end{Arabic}}
\flushleft{\begin{hindi}
फिर (क़यामत के दिन) ज़ालिम लोगों से कहा जाएगा कि (अब हमेशा के अज़ाब के मजे चखो (दुनिया में) जैसी तुम्हारी करतूतें तुम्हें (आख़िरत में) वैसा ही बदला दिया जाएगा
\end{hindi}}
\flushright{\begin{Arabic}
\quranayah[10][53]
\end{Arabic}}
\flushleft{\begin{hindi}
(ऐ रसूल) तुम से लोग पूछतें हैं कि क्या (जो कुछ तुम कहते हो) वह सब ठीक है तुम कह दो (हाँ) अपने परवरदिगार की कसम ठीक है और तुम (ख़ुदा को) हरा नहीं सकते
\end{hindi}}
\flushright{\begin{Arabic}
\quranayah[10][54]
\end{Arabic}}
\flushleft{\begin{hindi}
और (दुनिया में) जिस जिसने (हमारी नाफरमानी कर के) ज़ुल्म किया है (क़यामत के दिन) अगर तमाम ख़ज़ाने जो जमीन में हैं उसे मिल जाएँ तो अपने गुनाह के बदले ज़रुर फिदया दे निकले और जब वह लोग अज़ाब को देखेगें तो इज़हारे निदामत करेगें (शर्मिंदा होंगे) और उनमें बाहम इन्साफ़ के साथ हुक्म दिया जाएगा और उन पर ज़र्रा (बराबर ज़ुल्म न किया जाएगा
\end{hindi}}
\flushright{\begin{Arabic}
\quranayah[10][55]
\end{Arabic}}
\flushleft{\begin{hindi}
आगाह रहो कि जो कुछ आसमानों में और ज़मीन में है (ग़रज़ सब कुछ) ख़ुदा ही का है आग़ाह राहे कि ख़ुदा का वायदा यक़ीनी ठीक है मगर उनमें के अक्सर नहीं जानते हैं
\end{hindi}}
\flushright{\begin{Arabic}
\quranayah[10][56]
\end{Arabic}}
\flushleft{\begin{hindi}
वही ज़िन्दा करता है और वही मारता है और तुम सब के सब उसी की तरफ लौटाए जाओगें
\end{hindi}}
\flushright{\begin{Arabic}
\quranayah[10][57]
\end{Arabic}}
\flushleft{\begin{hindi}
लोगों तुम्हारे पास तुम्हारे परवरदिगार की तरफ से नसीहत (किताबे ख़ुदा आ चुकी और जो (मरज़ शिर्क वगैरह) दिल में हैं उनकी दवा और ईमान वालों के लिए हिदायत और रहमत
\end{hindi}}
\flushright{\begin{Arabic}
\quranayah[10][58]
\end{Arabic}}
\flushleft{\begin{hindi}
(ऐ रसूल) तुम कह दो कि (ये क़ुरान) ख़ुदा के फज़ल व करम और उसकी रहमत से तुमको मिला है (ही) तो उन लोगों को इस पर खुश होना चाहिए
\end{hindi}}
\flushright{\begin{Arabic}
\quranayah[10][59]
\end{Arabic}}
\flushleft{\begin{hindi}
और जो कुछ वह जमा कर रहे हैं उससे कहीं बेहतर है (ऐ रसूल) तुम कह दो कि तुम्हारा क्या ख्याल है कि ख़ुदा ने तुम पर रोज़ी नाज़िल की तो अब उसमें से बाज़ को हराम बाज़ को हलाल बनाने लगे (ऐ रसूल) तुम कह दो कि क्या ख़ुदा ने तुम्हें इजाज़त दी है या तुम ख़ुदा पर बोहतान बाँधते हो
\end{hindi}}
\flushright{\begin{Arabic}
\quranayah[10][60]
\end{Arabic}}
\flushleft{\begin{hindi}
और जो लोग ख़ुदा पर झूठ मूठ बोहतान बॉधा करते हैं रोजे क़यामत का क्या ख्याल करते हैं उसमें शक़ नहीं कि ख़ुदा तो लोगों पर बड़ा फज़ल व (करम) है मगर उनमें से बहुतेरे शुक्र गुज़ार नहीं हैं
\end{hindi}}
\flushright{\begin{Arabic}
\quranayah[10][61]
\end{Arabic}}
\flushleft{\begin{hindi}
(और ऐ रसूल) तुम (चाहे) किसी हाल में हो और क़ुरान की कोई सी भी आयत तिलावत करते हो और (लोगों) तुम कोई सा भी अमल कर रहे हो हम (हम सर वक़त) जब तुम उस काम में मशग़ूल होते हो तुम को देखते रहते हैं और तुम्हारे परवरदिगार से ज़र्रा भी कोई चीज़ ग़ायब नहीं रह सकती न ज़मीन में और न आसमान में और न कोई चीज़ ज़र्रे से छोटी है और न उससे बढ़ी चीज़ मगर वह रौशन किताब लौहे महफूज़ में ज़रुर है
\end{hindi}}
\flushright{\begin{Arabic}
\quranayah[10][62]
\end{Arabic}}
\flushleft{\begin{hindi}
आगाह रहो इसमें शक़ नहीं कि दोस्ताने ख़ुदा पर (क़यामत में) न तो कोई ख़ौफ होगा और न वह आजुर्दा (ग़मग़ीन) ख़ातिर होगे
\end{hindi}}
\flushright{\begin{Arabic}
\quranayah[10][63]
\end{Arabic}}
\flushleft{\begin{hindi}
ये वह लोग हैं जो ईमान लाए और (ख़ुदा से) डरते थे
\end{hindi}}
\flushright{\begin{Arabic}
\quranayah[10][64]
\end{Arabic}}
\flushleft{\begin{hindi}
उन्हीं लोगों के वास्ते दीन की ज़िन्दगी में भी और आख़िरत में (भी) ख़ुशख़बरी है ख़ुदा की बातों में अदल बदल नहीं हुआ करता यही तो बड़ी कामयाबी है
\end{hindi}}
\flushright{\begin{Arabic}
\quranayah[10][65]
\end{Arabic}}
\flushleft{\begin{hindi}
और (ऐ रसूल) उन (कुफ्फ़ार) की बातों का तुम रंज न किया करो इसमें तो शक़ नहीं कि सारी इज्ज़त तो सिर्फ ख़ुदा ही के लिए है वही सबकी सुनता जानता है
\end{hindi}}
\flushright{\begin{Arabic}
\quranayah[10][66]
\end{Arabic}}
\flushleft{\begin{hindi}
आगाह रहो इसमें शक़ नहीं कि जो लोग आसमानों में हैं और जो लोग ज़मीन में है (ग़रज़ सब कुछ) ख़ुदा ही के लिए है और जो लोग ख़ुदा को छोड़कर (दूसरों को) पुकारते हैं वह तो (ख़ुदा के फर्ज़ी) शरीकों की राह पर भी नहीं चलते बल्कि वह तो सिर्फ अपनी अटकल पर चलते हैं और वह सिर्फ वहमी और ख्याली बातें किया करते हैं
\end{hindi}}
\flushright{\begin{Arabic}
\quranayah[10][67]
\end{Arabic}}
\flushleft{\begin{hindi}
वह वही (खुदाए क़ादिर तवाना) है जिसने तुम्हारे नफा के वास्ते रात को बनाया ताकि तुम इसमें चैन करो और दिन को (बनाया) कि उसकी रौशनी में देखो भालो उसमें शक़ नहीं जो लोग सुन लेते हैं उनके लिए इसमें (कुदरत की बहुतेरी निशानियाँ हैं)
\end{hindi}}
\flushright{\begin{Arabic}
\quranayah[10][68]
\end{Arabic}}
\flushleft{\begin{hindi}
लोगों ने तो कह दिया कि ख़ुदा ने बेटा बना लिया-ये महज़ लगों वह तमाम नकायस से पाक व पाकीज़ा वह (हर तरह) से बेपरवाह हैं व जो कुछ आसमानों में है और जो कुछ ज़मीन में है (सब) उसी का है (जो कुछ) तुम कहते हो( उसकी कोई दलील तो तुम्हारे पास है नहीं क्या तुम ख़ुदा पर) (यू ही) बे जाने बूझे झूठ बोला करते हो
\end{hindi}}
\flushright{\begin{Arabic}
\quranayah[10][69]
\end{Arabic}}
\flushleft{\begin{hindi}
ऐ रसूल तुम कह दो कि बेशक जो लोग झूठ मूठ ख़ुदा पर बोहतान बाधते हैं वह कभी कामयाब न होगें
\end{hindi}}
\flushright{\begin{Arabic}
\quranayah[10][70]
\end{Arabic}}
\flushleft{\begin{hindi}
(ये) दुनिया के (चन्द रोज़ा) फायदे हैं फिर तो आख़िर हमारी ही तरफ लौट कर आना है तब उनके कुफ्र की सज़ा में हम उनको सख्त अज़ाब के मज़े चखाएँगें
\end{hindi}}
\flushright{\begin{Arabic}
\quranayah[10][71]
\end{Arabic}}
\flushleft{\begin{hindi}
और (ऐ रसूल) तुम उनके सामने नूह का हाल पढ़ दो जब उन्होंने अपनी क़ौम से कहा ऐ मेरी क़ौम अगर मेरा ठहरना और ख़ुदा की आयतों का चर्चा करना तुम पर शाक़ व गिरां (बुरा) गुज़रता है तो मैं सिर्फ ख़ुदा ही पर भरोसा रखता हूँ तो तुम और तुमहारे शरीक़ सब मिलकर अपना काम ठीक कर लो फिर तुम्हारी बात तुम (में से किसी) पर महज़ (छुपी) न रहे फिर (जो तुम्हारा जी चाहे) मेरे साथ कर गुज़रों और गुझे (दम मारने की भी) मोहलत न दो
\end{hindi}}
\flushright{\begin{Arabic}
\quranayah[10][72]
\end{Arabic}}
\flushleft{\begin{hindi}
फिर भी अगर तुम ने (मेरी नसीहत से) मुँह मोड़ा तो मैने तुम से कुछ मज़दूरी तो न माँगी थी-मेरी मज़दूरी तो सिर्फ ख़ुदा ही पर है और (उसी की तरफ से) मुझे हुक्म दिया गया है कि मैं उसके फरमाबरदार बन्दों में से हो जाऊँ
\end{hindi}}
\flushright{\begin{Arabic}
\quranayah[10][73]
\end{Arabic}}
\flushleft{\begin{hindi}
उस पर भी उन लोगों ने उनको झुठलाया तो हमने उनको और जो लोग उनके साथ कश्ती में (सवार) थे (उनको) नजात दी और उनको (अगलों का) जानशीन बनाया और जिन लोगों ने हमारी आयतों को झुठलाया था उनको डुबो मारा
\end{hindi}}
\flushright{\begin{Arabic}
\quranayah[10][74]
\end{Arabic}}
\flushleft{\begin{hindi}
फिर ज़रा ग़ौर तो करो फिर हमने नूह के बाद और रसूलों को अपनी क़ौम के पास भेजा तो वह पैग़म्बर उनके पास वाजेए (खुले हुए) व रौशन मौजिज़े लेकर आए इस पर भी जिस चीज़ को ये लोग पहले झुठला चुके थे उस पर ईमान (न लाना था) न लाए हम यूंही हद से गुज़र जाने वालों के दिलों पर (गोया) खुद मोहर कर देते हैं
\end{hindi}}
\flushright{\begin{Arabic}
\quranayah[10][75]
\end{Arabic}}
\flushleft{\begin{hindi}
फिर हमने इन पैग़म्बरों के बाद मूसा व हारुन को अपनी निशानियाँ (मौजिज़े) लेकर फिरऔन और उस (की क़ौम) के सरदारों के पास भेजा तो वह लोग अकड़ बैठे और ये लोग थे ही कुसूरवार
\end{hindi}}
\flushright{\begin{Arabic}
\quranayah[10][76]
\end{Arabic}}
\flushleft{\begin{hindi}
फिर जब उनके पास हमारी तरफ से हक़ बात (मौजिज़े) पहुँच गए तो कहने लगे कि ये तो यक़ीनी खुल्लम खुल्ला जादू है
\end{hindi}}
\flushright{\begin{Arabic}
\quranayah[10][77]
\end{Arabic}}
\flushleft{\begin{hindi}
मूसा ने कहा क्या जब (दीन) तुम्हारे पास आया तो उसके बारे में कहते हो कि क्या ये जादू है और जादूगर लोग कभी कामयाब न होगें
\end{hindi}}
\flushright{\begin{Arabic}
\quranayah[10][78]
\end{Arabic}}
\flushleft{\begin{hindi}
वह लोग कहने लगे कि (ऐ मूसा) क्यों तुम हमारे पास उस वास्ते आए हो कि जिस दीन पर हमने अपने बाप दादाओं को पाया उससे तुम हमे बहका दो और सारी ज़मीन में ही दोनों की बढ़ाई हो और ये लोग तुम दोनों पर ईमान लाने वाले नहीं
\end{hindi}}
\flushright{\begin{Arabic}
\quranayah[10][79]
\end{Arabic}}
\flushleft{\begin{hindi}
और फिरऔन ने हुक्म दिया कि हमारे हुज़ूर में तमाम खिलाड़ी (वाक़िफकार) जादूगर को तो ले आओ
\end{hindi}}
\flushright{\begin{Arabic}
\quranayah[10][80]
\end{Arabic}}
\flushleft{\begin{hindi}
फिर जब जादूगर लोग (मैदान में) आ मौजूद हुए तो मूसा ने उनसे कहा कि तुमको जो कुछ फेंकना हो फेंको
\end{hindi}}
\flushright{\begin{Arabic}
\quranayah[10][81]
\end{Arabic}}
\flushleft{\begin{hindi}
फिर जब वह लोग (रस्सियों को साँप बनाकर) डाल चुके तू मूसा ने कहा जो कुछ तुम (बनाकर) लाए हो (वह तो सब) जादू है-इसमें तो शक़ ही नहीं कि ख़ुदा उसे फौरन मिटियामेट कर देगा (क्योंकर) ख़ुदा तो हरगिज़ मफ़सिदों (फसाद करने वालों) का काम दुरुस्त नहीं होने देता
\end{hindi}}
\flushright{\begin{Arabic}
\quranayah[10][82]
\end{Arabic}}
\flushleft{\begin{hindi}
और ख़ुदा सच्ची बात को अपने कलाम की बरकत से साबित कर दिखाता है अगरचे गुनाहगारों को ना गॅवार हो
\end{hindi}}
\flushright{\begin{Arabic}
\quranayah[10][83]
\end{Arabic}}
\flushleft{\begin{hindi}
अलग़रज़ मूसा पर उनकी क़ौम की नस्ल के चन्द आदमियों के सिवा फिरऔन और उसके सरदारों के इस ख़ौफ से कि उन पर कोई मुसीबत डाल दे कोई ईमान न लाया और इसमें शक़ नहीं कि फिरऔनरुए ज़मीन में बहुत बढ़ा चढ़ा था और इसमें शक़ नहीं कि वह यक़ीनन ज्यादती करने वालों में से था
\end{hindi}}
\flushright{\begin{Arabic}
\quranayah[10][84]
\end{Arabic}}
\flushleft{\begin{hindi}
और मूसा ने कहा ऐ मेरी क़ौम अगर तुम (सच्चे दिल से) ख़ुदा पर ईमान ला चुके तो अगर तुम फरमाबरदार हो तो बस उसी पर भरोसा करो
\end{hindi}}
\flushright{\begin{Arabic}
\quranayah[10][85]
\end{Arabic}}
\flushleft{\begin{hindi}
उस पर उन लोगों ने अर्ज़ की हमने तो ख़ुदा ही पर भरोसा कर लिया है और दुआ की कि ऐ हमारे पालने वाले तू हमें ज़ालिम लोगों का (ज़रिया) इम्तिहान न बना
\end{hindi}}
\flushright{\begin{Arabic}
\quranayah[10][86]
\end{Arabic}}
\flushleft{\begin{hindi}
और अपनी रहमत से हमें इन काफ़िर लोगों (के नीचे) से नजात दे
\end{hindi}}
\flushright{\begin{Arabic}
\quranayah[10][87]
\end{Arabic}}
\flushleft{\begin{hindi}
और हमने मूसा और उनके भाई (हारुन) के पास 'वही' भेजी कि मिस्र में अपनी क़ौम के (रहने सहने के) लिए घर बना डालो और अपने अपने घरों ही को मस्जिदें क़रार दे लो और पाबन्दी से नमाज़ पढ़ों और मोमिनीन को (नजात का) खुशख़बरी दे दो
\end{hindi}}
\flushright{\begin{Arabic}
\quranayah[10][88]
\end{Arabic}}
\flushleft{\begin{hindi}
और मूसा ने अर्ज़ की ऐ हमारे पालने वाले तूने फिरऔन और उसके सरदारों को दुनिया की ज़िन्दगी में (बड़ी) आराइश और दौलत दे रखी है (क्या तूने ये सामान इस लिए अता किया है) ताकि ये लोग तेरे रास्तें से लोगों को बहकाएं परवरदिगार तू उनके माल (दौलत) को ग़ारत (बरबाद) कर दे और उनके दिलों पर सख्ती कर (क्योंकि) जब तक ये लोग तकलीफ देह अज़ाब न देख लेगें ईमान न लाएगें
\end{hindi}}
\flushright{\begin{Arabic}
\quranayah[10][89]
\end{Arabic}}
\flushleft{\begin{hindi}
(ख़ुदा ने) फरमाया तुम दोनों की दुआ क़ुबूल की गई तो तुम दोनों साबित कदम रहो और नादानों की राह पर न चलो
\end{hindi}}
\flushright{\begin{Arabic}
\quranayah[10][90]
\end{Arabic}}
\flushleft{\begin{hindi}
और हमने बनी इसराइल को दरिया के उस पार कर दिया फिर फिरऔन और उसके लश्कर ने सरकशी की और शरारत से उनका पीछा किया-यहाँ तक कि जब वह डूबने लगा तो कहने लगा कि जिस ख़ुदा पर बनी इसराइल ईमान लाए हैं मै भी उस पर ईमान लाता हूँ उससे सिवा कोई माबूद नहीं और मैं फरमाबरदार बन्दों से हूँ
\end{hindi}}
\flushright{\begin{Arabic}
\quranayah[10][91]
\end{Arabic}}
\flushleft{\begin{hindi}
अब (मरने) के वक्त र्ऌमान लाता है हालॉकि इससे पहले तो नाफ़रमानी कर चुका और तू तो फ़सादियों में से था
\end{hindi}}
\flushright{\begin{Arabic}
\quranayah[10][92]
\end{Arabic}}
\flushleft{\begin{hindi}
तो हम आज तेरी रुह को तो नहीं (मगर) तेरे बदन को (तह नशीन होने से) बचाएँगें ताकि तू अपने बाद वालों के लिए इबरत का (बाइस) हो और इसमें तो शक़ नहीं कि तेरे लोग हमारी निशानियों से यक़ीनन बेख़बर हैं
\end{hindi}}
\flushright{\begin{Arabic}
\quranayah[10][93]
\end{Arabic}}
\flushleft{\begin{hindi}
और हमने बनी इसराइल को (मालिक शाम में) बहुत अच्छी जगह बसाया और उन्हं अच्छी अच्छी चीज़ें खाने को दी तो उन लोगों के पास जब तक इल्म (न) आ चुका उन लोगों ने एख्तेलाफ़ नहीं किया इसमें तो शक़ ही नहीं जिन बातों में ये (दुनिया में) बाहम झगड़े रहे है क़यामत के दिन तुम्हारा परवरदिगार इसमें फैसला कर देगा
\end{hindi}}
\flushright{\begin{Arabic}
\quranayah[10][94]
\end{Arabic}}
\flushleft{\begin{hindi}
पस जो कुरान हमने तुम्हारी तरफ नाज़िल किया है अगर उसके बारे में तुम को कुछ शक़ हो तो जो लोग तुम से पहले से किताब (ख़ुदा) पढ़ा करते हैं उन से पूछ के देखों तुम्हारे पास यक़ीनन तुम्हारे परवरदिगार की तरफ से बरहक़ किताब आ चुकी तो तू न हरगिज़ शक़ करने वालों से होना
\end{hindi}}
\flushright{\begin{Arabic}
\quranayah[10][95]
\end{Arabic}}
\flushleft{\begin{hindi}
न उन लोगों से होना जिन्होंने ख़ुदा की आयतों को झुठलाया (वरना) तुम भी घाटा उठाने वालों से हो जाओगे
\end{hindi}}
\flushright{\begin{Arabic}
\quranayah[10][96]
\end{Arabic}}
\flushleft{\begin{hindi}
(ऐ रसूल) इसमें शक़ नहीं कि जिन लोगों के बारे में तुम्हारे परवरदिगार को बातें पूरी उतर चुकी हैं (कि ये मुस्तहके अज़ाब हैं)
\end{hindi}}
\flushright{\begin{Arabic}
\quranayah[10][97]
\end{Arabic}}
\flushleft{\begin{hindi}
वह लोग जब तक दर्दनाक अज़ाब देख (न) लेगें ईमान न लाएंगें अगरचे इनके सामने सारी (ख़ुदाई के) मौजिज़े आ मौजूद हो
\end{hindi}}
\flushright{\begin{Arabic}
\quranayah[10][98]
\end{Arabic}}
\flushleft{\begin{hindi}
कोई बस्ती ऐसी क्यों न हुई कि ईमान क़ुबूल करती तो उसको उसका ईमान फायदे मन्द होता हाँ यूनूस की क़ौम जब (अज़ाब देख कर) ईमान लाई तो हमने दुनिया की (चन्द रोज़ा) ज़िन्दगी में उनसे रुसवाई का अज़ाब दफा कर दिया और हमने उन्हें एक ख़ास वक्त तक चैन करने दिया
\end{hindi}}
\flushright{\begin{Arabic}
\quranayah[10][99]
\end{Arabic}}
\flushleft{\begin{hindi}
और (ऐ पैग़म्बर) अगर तेरा परवरदिगार चाहता तो जितने लोग रुए ज़मीन पर हैं सबके सब ईमान ले आते तो क्या तुम लोगों पर ज़बरदस्ती करना चाहते हो ताकि सबके सब ईमानदार हो जाएँ हालॉकि किसी शख़्स को ये एख्तेयार नहीं
\end{hindi}}
\flushright{\begin{Arabic}
\quranayah[10][100]
\end{Arabic}}
\flushleft{\begin{hindi}
कि बगैर ख़ुदा की इजाज़त ईमान ले आए और जो लोग (उसूले दीन में) अक़ल से काम नहीं लेते उन्हीं लोगें पर ख़ुदा (कुफ़्र) की गन्दगी डाल देता है
\end{hindi}}
\flushright{\begin{Arabic}
\quranayah[10][101]
\end{Arabic}}
\flushleft{\begin{hindi}
(ऐ रसूल) तुम कहा दो कि ज़रा देखों तो सही कि आसमानों और ज़मीन में (ख़ुदा की निशानियाँ क्या) क्या कुछ हैं (मगर सच तो ये है) और जो लोग ईमान नहीं क़ुबूल करते उनको हमारी निशानियाँ और डरावे कुछ भी मुफीद नहीं
\end{hindi}}
\flushright{\begin{Arabic}
\quranayah[10][102]
\end{Arabic}}
\flushleft{\begin{hindi}
तो ये लोग भी उन्हें सज़ाओं के मुन्तिज़र (इन्तजार में) हैं जो उनसे क़ब्ल (पहले) वालो पर गुज़र चुकी हैं (ऐ रसूल उनसे) कह दो कि अच्छा तुम भी इन्तज़ार करो मैं भी तुम्हारे साथ यक़ीनन इन्तज़ार करता हूँ
\end{hindi}}
\flushright{\begin{Arabic}
\quranayah[10][103]
\end{Arabic}}
\flushleft{\begin{hindi}
फिर (नुज़ूले अज़ाब के वक्त) हम अपने रसूलों को और जो लोग ईमान लाए उनको (अज़ाब से) तलूउ बचा लेते हैं यूँ ही हम पर लाज़िम है कि हम ईमान लाने वालों को भी बचा लें
\end{hindi}}
\flushright{\begin{Arabic}
\quranayah[10][104]
\end{Arabic}}
\flushleft{\begin{hindi}
(ऐ रसूल) तुम कह दो कि अगर तुम लोग मेरे दीन के बारे में शक़ में पड़े हो तो (मैं भी तुमसे साफ कहें देता हूँ) ख़ुदा के सिवा तुम भी जिन लोगों की परसतिश करते हो मै तो उनकी परसतिश नहीं करने का मगर (हाँ) मै उस ख़ुदा की इबादत करता हूँ जो तुम्हें (अपनी कुदरत से दुनिया से) उठा लेगा और मुझे तो ये हुक्म दिया गया है कि मोमिन हूँ
\end{hindi}}
\flushright{\begin{Arabic}
\quranayah[10][105]
\end{Arabic}}
\flushleft{\begin{hindi}
और (मुझे) ये भी (हुक्म है) कि (बातिल) से कतरा के अपना रुख़ दीन की तरफ कायम रख और मुशरेकीन से हरगिज़ न होना
\end{hindi}}
\flushright{\begin{Arabic}
\quranayah[10][106]
\end{Arabic}}
\flushleft{\begin{hindi}
और ख़ुदा को छोड़ ऐसी चीज़ को पुकारना जो न तुझे नफा ही पहुँचा सकती हैं न नुक़सान ही पहुँचा सकती है तो अगर तुमने (कहीं ऐसा) किया तो उस वक्त तुम भी ज़ालिमों में (शुमार) होगें
\end{hindi}}
\flushright{\begin{Arabic}
\quranayah[10][107]
\end{Arabic}}
\flushleft{\begin{hindi}
और (याद रखो कि) अगर ख़ुदा की तरफ से तुम्हें कोई बुराई छू भी गई तो फिर उसके सिवा कोई उसका दफा करने वाला नहीं होगा और अगर तुम्हारे साथ भलाई का इरादा करे तो फिर उसके फज़ल व करम का लपेटने वाला भी कोई नहीं वह अपने बन्दों में से जिसको चाहे फायदा पहुँचाएँ और वह बड़ा बख्शने वाला मेहरबान है
\end{hindi}}
\flushright{\begin{Arabic}
\quranayah[10][108]
\end{Arabic}}
\flushleft{\begin{hindi}
(ऐ रसूल) तुम कह दो कि ऐ लोगों तुम्हारे परवरदिगार की तरफ से तुम्हारे पास हक़ (क़ुरान) आ चुका फिर जो शख़्स सीधी राह पर चलेगा तो वह सिर्फ अपने ही दम के लिए हिदायत एख्तेयार करेगा और जो गुमराही एख्तेयार करेगा वह तो भटक कर कुछ अपना ही खोएगा और मैं कुछ तुम्हारा ज़िम्मेदार तो हूँ नहीं
\end{hindi}}
\flushright{\begin{Arabic}
\quranayah[10][109]
\end{Arabic}}
\flushleft{\begin{hindi}
और (ऐ रसूल) तुम्हारे पास जो 'वही' भेजी जाती है तुम बस उसी की पैरवी करो और सब्र करो यहाँ तक कि ख़ुदा तुम्हारे और काफिरों के दरमियान फैसला फरमाए और वह तो तमाम फैसला करने वालों से बेहतर है
\end{hindi}}
\chapter{Hud (Hud)}
\begin{Arabic}
\Huge{\centerline{\basmalah}}\end{Arabic}
\flushright{\begin{Arabic}
\quranayah[11][1]
\end{Arabic}}
\flushleft{\begin{hindi}
अलिफ़ लाम रा - ये (क़ुरान) वह किताब है जिसकी आयते एक वाकिफ़कार हकीम की तरफ से (दलाएल से) खूब मुस्तहकिम (मज़बूत) कर दी गयीं
\end{hindi}}
\flushright{\begin{Arabic}
\quranayah[11][2]
\end{Arabic}}
\flushleft{\begin{hindi}
फिर तफ़सीलदार बयान कर दी गयी हैं ये कि ख़ुदा के सिवा किसी की परसतिश न करो मै तो उसकी तरफ से तुम्हें (अज़ाब से) डराने वाला और (बेहिश्त की) ख़ुशख़बरी देने वाला (रसूल) हूँ
\end{hindi}}
\flushright{\begin{Arabic}
\quranayah[11][3]
\end{Arabic}}
\flushleft{\begin{hindi}
और ये भी कि अपने परवरदिगार से मग़फिरत की दुआ मॉगों फिर उसकी बारगाह में (गुनाहों से) तौबा करो वही तुम्हें एक मुकर्रर मुद्दत तक अच्छे नुत्फ के फायदे उठाने देगा और वही हर साहबे बुर्ज़गी को उसकी बुर्जुगी (की दाद) अता फरमाएगा और अगर तुमने (उसके हुक्म से) मुँह मोड़ा तो मुझे तुम्हारे बारे में एक बड़े (ख़ौफनाक) दिन के अज़ाब का डर है
\end{hindi}}
\flushright{\begin{Arabic}
\quranayah[11][4]
\end{Arabic}}
\flushleft{\begin{hindi}
(याद रखो) तुम सब को (आख़िरकार) ख़ुदा ही की तरफ लौटना है और वह हर चीज़ पर (अच्छी तरह) क़ादिर है
\end{hindi}}
\flushright{\begin{Arabic}
\quranayah[11][5]
\end{Arabic}}
\flushleft{\begin{hindi}
(ऐ रसूल) देखो ये कुफ़्फ़ार (तुम्हारी अदावत में) अपने सीनों को (गोया) दोहरा किए डालते हैं ताकि ख़ुदा से (अपनी बातों को) छिपाए रहें (मगर) देखो जब ये लोग अपने कपड़े ख़ूब लपेटते हैं (तब भी तो) ख़ुदा (उनकी बातों को) जानता है जो छिपाकर करते हैं और खुल्लम खुल्ला करते हैं इसमें शक़ नहीं कि वह सीनों के भेद तक को खूब जानता है
\end{hindi}}
\flushright{\begin{Arabic}
\quranayah[11][6]
\end{Arabic}}
\flushleft{\begin{hindi}
और ज़मीन पर चलने वालों में कोई ऐसा नहीं जिसकी रोज़ी ख़ुदा के ज़िम्मे न हो और ख़ुदा उनके ठिकाने और (मरने के बाद) उनके सौपे जाने की जगह (क़ब्र) को भी जानता है सब कुछ रौशन किताब (लौहे महफूज़) में मौजूद है
\end{hindi}}
\flushright{\begin{Arabic}
\quranayah[11][7]
\end{Arabic}}
\flushleft{\begin{hindi}
और वह तो वही (क़ादिरे मुत्तलिक़) है जिसने आसमानों और ज़मीन को 6 दिन में पैदा किया और (उस वक्त) उसका अर्श (फलक नहुम) पानी पर था (उसने आसमान व ज़मीन) इस ग़रज़ से बनाया ताकि तुम लोगों को आज़माए कि तुममे ज्यादा अच्छी कार गुज़ारी वाला कौन है और (ऐ रसूल) अगर तुम (उनसे) कहोगे कि मरने के बाद तुम सबके सब दोबारा (क़ब्रों से) उठाए जाओगे तो काफ़िर लोग ज़रुर कह बैठेगें कि ये तो बस खुला हुआ जादू है
\end{hindi}}
\flushright{\begin{Arabic}
\quranayah[11][8]
\end{Arabic}}
\flushleft{\begin{hindi}
और अगर हम गिनती के चन्द रोज़ो तक उन पर अज़ाब करने में देर भी करें तो ये लोग (अपनी शरारत से) बेताम्मुल ज़रुर कहने लगेगें कि (हाए) अज़ाब को कौन सी चीज़ रोक रही है सुन रखो जिस दिन इन पर अज़ाब आ पडे तो (फिर) उनके टाले न टलेगा और जिस (अज़ाब) की ये लोग हँसी उड़ाया करते थे वह उनको हर तरह से घेर लेगा
\end{hindi}}
\flushright{\begin{Arabic}
\quranayah[11][9]
\end{Arabic}}
\flushleft{\begin{hindi}
और अगर हम इन्सान को अपनी रहमत का मज़ा चखाएं फिर उसको हम उससे छीन लें तो (उस वक्त) यक़ीनन बड़ा बेआस और नाशुक्रा हो जाता है
\end{hindi}}
\flushright{\begin{Arabic}
\quranayah[11][10]
\end{Arabic}}
\flushleft{\begin{hindi}
(और हमारी शिकायत करने लगता है) और अगर हम तकलीफ के बाद जो उसे पहुँचती थी राहत व आराम का जाएक़ा चखाए तो ज़रुर कहने लगता है कि अब तो सब सख्तियाँ मुझसे दफा हो गई इसमें शक़ नहीं कि वह बड़ा (जल्दी खुश) होने येख़ी बाज़ है
\end{hindi}}
\flushright{\begin{Arabic}
\quranayah[11][11]
\end{Arabic}}
\flushleft{\begin{hindi}
मगर जिन लोगों ने सब्र किया और अच्छे (अच्छे) काम किए (वह ऐसे नहीं) ये वह लोग हैं जिनके वास्ते (ख़ुदा की) बख़्शिस और बहुत बड़ी (खरी) मज़दूरी है
\end{hindi}}
\flushright{\begin{Arabic}
\quranayah[11][12]
\end{Arabic}}
\flushleft{\begin{hindi}
तो जो चीज़ तुम्हारे पास 'वही' के ज़रिए से भेजी है उनमें से बाज़ को (सुनाने के वक्त) यायद तुम फक़त इस ख्याल से छोड़ देने वाले हो और तुम तंग दिल हो कि मुबादा ये लोग कह बैंठें कि उन पर खज़ाना क्यों नहीं नाज़िल किया गया या (उनके तसदीक के लिए) उनके साथ कोई फरिश्ता क्यों न आया तो तुम सिर्फ (अज़ाब से) डराने वाले हो
\end{hindi}}
\flushright{\begin{Arabic}
\quranayah[11][13]
\end{Arabic}}
\flushleft{\begin{hindi}
तुम्हें उनका ख्याल न करना चाहिए और ख़ुदा हर चीज़ का ज़िम्मेदार है क्या ये लोग कहते हैं कि उस शख़्श (तुम) ने इस (क़ुरान) को अपनी तरफ से गढ़ लिया है तो तुम (उनसे साफ साफ) कह दो कि अगर तुम (अपने दावे में) सच्चे हो तो (ज्यादा नहीं) ऐसे दस सूरे अपनी तरफ से गढ़ के ले आओं
\end{hindi}}
\flushright{\begin{Arabic}
\quranayah[11][14]
\end{Arabic}}
\flushleft{\begin{hindi}
और ख़ुदा के सिवा जिस जिस के तुम्हे बुलाते बन पड़े मदद के वास्ते बुला लो उस पर अगर वह तुम्हारी न सुने तो समझ ले कि (ये क़ुरान) सिर्फ ख़ुदा के इल्म से नाज़िल किया गया है और ये कि ख़ुदा के सिवा कोई माबूद नहीं तो क्या तुम अब भी इस्लाम लाओगे (या नहीं)
\end{hindi}}
\flushright{\begin{Arabic}
\quranayah[11][15]
\end{Arabic}}
\flushleft{\begin{hindi}
नेकी करने वालों में से जो शख़्श दुनिया की ज़िन्दगी और उसके रिज़क़ का तालिब हो तो हम उन्हें उनकी कारगुज़ारियों का बदला दुनिया ही में पूरा पूरा भर देते हैं और ये लोग दुनिया में घाटे में नहीं रहेगें
\end{hindi}}
\flushright{\begin{Arabic}
\quranayah[11][16]
\end{Arabic}}
\flushleft{\begin{hindi}
मगर (हाँ) ये वह लोग हैं जिनके लिए आख़िरत में (जहन्नुम की) आग के सिवा कुछ नहीं और जो कुछ दुनिया में उन लोगों ने किया धरा था सब अकारत (बर्बाद) हो गया और जो कुछ ये लोग करते थे सब मिटियामेट हो गया
\end{hindi}}
\flushright{\begin{Arabic}
\quranayah[11][17]
\end{Arabic}}
\flushleft{\begin{hindi}
तो क्या जो शख़्श अपने परवरदिगार की तरफ से रौशन दलील पर हो और उसके पीछे ही पीछे उनका एक गवाह हो और उसके क़बल मूसा की किताब (तौरैत) जो (लोगों के लिए) पेशवा और रहमत थी (उसकी तसदीक़ करती हो वह बेहतर है या कोई दूसरा) यही लोग सच्चे ईमान लाने वाले और तमाम फिरक़ों में से जो शख़्श भी उसका इन्कार करे तो उसका ठिकाना बस आतिश (जहन्नुम) है तो फिर तुम कहीं उसकी तरफ से शक़ में न पड़े रहना, बेशक ये क़ुरान तुम्हारे परवरदिगार की तरफ़ से बरहक़ है मगर बहुतेरे लोग ईमान नही लाते
\end{hindi}}
\flushright{\begin{Arabic}
\quranayah[11][18]
\end{Arabic}}
\flushleft{\begin{hindi}
और ये जो शख़्श ख़ुदा पर झूठ मूठ बोहतान बॉधे उससे ज्यादा ज़ालिम कौन होगा ऐसे लोग अपने परवरदिगार के हुज़ूर में पेश किए जाएंगें और गवाह इज़हार करेगें कि यही वह लोग हैं जिन्होंने अपने परवरदिगार पर झूट (बोहतान) बाँधा था सुन रखो कि ज़ालिमों पर ख़ुदा की फिटकार है
\end{hindi}}
\flushright{\begin{Arabic}
\quranayah[11][19]
\end{Arabic}}
\flushleft{\begin{hindi}
जो ख़ुदा के रास्ते से लोगों को रोकते हैं और उसमें कज़ी (टेढ़ा पन) निकालना चाहते हैं और यही लोग आख़िरत के भी मुन्किर है
\end{hindi}}
\flushright{\begin{Arabic}
\quranayah[11][20]
\end{Arabic}}
\flushleft{\begin{hindi}
ये लोग रुए ज़मीन में न ख़ुदा को हरा सकते है और न ख़ुदा के सिवा उनका कोई सरपरस्त होगा उनका अज़ाब दूना कर दिया जाएगा ये लोग (हसद के मारे) न तो (हक़ बात) सुन सकते थे न देख सकते थे
\end{hindi}}
\flushright{\begin{Arabic}
\quranayah[11][21]
\end{Arabic}}
\flushleft{\begin{hindi}
ये वह लोग हैं जिन्होंने कुछ अपना ही घाटा किया और जो इफ्तेरा परदाज़ियाँ (झूठी बातें) ये लोग करते थे (क़यामत में सब) उन्हें छोड़ के चल होगी
\end{hindi}}
\flushright{\begin{Arabic}
\quranayah[11][22]
\end{Arabic}}
\flushleft{\begin{hindi}
इसमें शक़ नहीं कि यही लोग आख़िरत में बड़े घाटा उठाने वाले होगें
\end{hindi}}
\flushright{\begin{Arabic}
\quranayah[11][23]
\end{Arabic}}
\flushleft{\begin{hindi}
बेशक जिन लोगों ने ईमान क़ुबूल किया और अच्छे अच्छे काम किए और अपने परवरदिगार के सामने आजज़ी से झुके यही लोग जन्नती हैं कि ये बेहश्त में हमेशा रहेगें
\end{hindi}}
\flushright{\begin{Arabic}
\quranayah[11][24]
\end{Arabic}}
\flushleft{\begin{hindi}
(काफिर, मुसलमान) दोनों फरीक़ की मसल अन्धे और बहरे और देखने वाले और सुनने वाले की सी है क्या ये दोनो मसल में बराबर हो सकते हैं तो क्या तुम लोग ग़ौर नहीं करते और हमने नूह को ज़रुर उन की क़ौम के पास भेजा
\end{hindi}}
\flushright{\begin{Arabic}
\quranayah[11][25]
\end{Arabic}}
\flushleft{\begin{hindi}
(और उन्होने अपनी क़ौम से कहा कि) मैं तो तुम्हारा (अज़ाबे ख़ुदा से) सरीही धमकाने वाला हूँ
\end{hindi}}
\flushright{\begin{Arabic}
\quranayah[11][26]
\end{Arabic}}
\flushleft{\begin{hindi}
(और) ये (समझता हूँ) कि तुम ख़ुदा के सिवा किसी की परसतिश न करो मैं तुम पर एक दर्दनाक दिन (क़यामत) के अज़ाब से डराता हूँ
\end{hindi}}
\flushright{\begin{Arabic}
\quranayah[11][27]
\end{Arabic}}
\flushleft{\begin{hindi}
तो उनके सरदार जो काफ़िर थे कहने लगे कि हम तो तुम्हें अपना ही सा एक आदमी समझते हैं और हम तो देखते हैं कि तुम्हारे पैरोकार हुए भी हैं तो बस सिर्फ हमारे चन्द रज़ील (नीच) लोग (और वह भी बे सोचे समझे सरसरी नज़र में) और हम तो अपने ऊपर तुम लोगों की कोई फज़ीलत नहीं देखते बल्कि तुम को झूठा समझते हैं
\end{hindi}}
\flushright{\begin{Arabic}
\quranayah[11][28]
\end{Arabic}}
\flushleft{\begin{hindi}
(नूह ने) कहा ऐ मेरी क़ौम क्या तुमने ये समझा है कि अगर मैं अपने परवरदिगार की तरफ से एक रौशन दलील पर हूँ और उसने अपनी सरकार से रहमत (नुबूवत) अता फरमाई और वह तुम्हें सुझाई नहीं देती तो क्या मैं उसको (ज़बरदस्ती) तुम्हारे गले मंढ़ सकता हूँ
\end{hindi}}
\flushright{\begin{Arabic}
\quranayah[11][29]
\end{Arabic}}
\flushleft{\begin{hindi}
और तुम हो कि उसको नापसन्द किए जाते हो और ऐ मेरी क़ौम मैं तो तुमसे इसके सिले में कुछ माल का तालिब नहीं मेरी मज़दूरी तो सिर्फ ख़ुदा के ज़िम्मे है और मै तो तुम्हारे कहने से उन लोगों को जो ईमान ला चुके हैं निकाल नहीं सकता (क्योंकि) ये लोग भी ज़रुर अपने परवरदिगार के हुज़ूर में हाज़िर होगें मगर मै तो देखता हूँ कि कुछ तुम ही लोग (नाहक़) जिहालत करते हो
\end{hindi}}
\flushright{\begin{Arabic}
\quranayah[11][30]
\end{Arabic}}
\flushleft{\begin{hindi}
और मेरी क़ौम अगर मै इन (बेचारे ग़रीब) (ईमानदारों) को निकाल दूँ तो ख़ुदा (के अज़ाब) से (बचाने में) मेरी मदद कौन करेगा तो क्या तुम इतना भी ग़ौर नहीं करते
\end{hindi}}
\flushright{\begin{Arabic}
\quranayah[11][31]
\end{Arabic}}
\flushleft{\begin{hindi}
और मै तो तुमसे ये नहीं कहता कि मेरे पास खुदाई ख़ज़ाने हैं और न (ये कहता हूँ कि) मै ग़ैब वॉ हूँ (गैब का जानने वाला) और ये कहता हूँ कि मै फरिश्ता हूँ और जो लोग तुम्हारी नज़रों में ज़लील हैं उन्हें मै ये नहीं कहता कि ख़ुदा उनके साथ हरगिज़ भलाई नहीं करेगा उन लोगों के दिलों की बात ख़ुदा ही खूब जानता है और अगर मै ऐसा कहूँ तो मै भी यक़ीनन ज़ालिम हूँ
\end{hindi}}
\flushright{\begin{Arabic}
\quranayah[11][32]
\end{Arabic}}
\flushleft{\begin{hindi}
वह लोग कहने लगे ऐ नूह तुम हम से यक़ीनन झगड़े और बहुत झगड़े फिर तुम सच्चे हो तो जिस (अज़ाब) की तुम हमें धमकी देते थे हम पर ला चुको
\end{hindi}}
\flushright{\begin{Arabic}
\quranayah[11][33]
\end{Arabic}}
\flushleft{\begin{hindi}
नूह ने कहा अगर चाहेगा तो बस ख़ुदा ही तुम पर अज़ाब लाएगा और तुम लोग किसी तरह उसे हरा नहीं सकते और अगर मै चाहूँ तो तुम्हारी (कितनी ही) ख़ैर ख्वाही (भलाई) करुँ
\end{hindi}}
\flushright{\begin{Arabic}
\quranayah[11][34]
\end{Arabic}}
\flushleft{\begin{hindi}
अगर ख़ुदा को तुम्हारा बहकाना मंज़ूर है तो मेरी ख़ैर ख्वाही कुछ भी तुम्हारे काम नहीं आ सकती वही तुम्हारा परवरदिगार है और उसी की तरफ तुम को लौट जाना है
\end{hindi}}
\flushright{\begin{Arabic}
\quranayah[11][35]
\end{Arabic}}
\flushleft{\begin{hindi}
(ऐ रसूल) क्या (कुफ्फ़ारे मक्का भी) कहते हैं कि क़ुरान को उस (तुम) ने गढ़ लिया है तुम कह दो कि अगर मैने उसको गढ़ा है तो मेरे गुनाह का वबाल मुझ पर होगा और तुम लोग जो (गुनाह करके) मुजरिम होते हो उससे मै बरीउल ज़िम्मा (अलग) हूँ
\end{hindi}}
\flushright{\begin{Arabic}
\quranayah[11][36]
\end{Arabic}}
\flushleft{\begin{hindi}
और नूह के पास ये 'वही' भेज दी गई कि जो ईमान ला चुका उनके सिवा अब कोई शख़्श तुम्हारी क़ौम से हरगिज़ ईमान न लाएगा तो तुम ख्वाहमा ख्वाह उनकी कारस्तानियों का (कुछ) ग़म न खाओ
\end{hindi}}
\flushright{\begin{Arabic}
\quranayah[11][37]
\end{Arabic}}
\flushleft{\begin{hindi}
और (बिस्मिल्लाह करके) हमारे रुबरु और हमारे हुक्म से कश्ती बना डालो और जिन लोगों ने ज़ुल्म किया है उनके बारे में मुझसे सिफारिश न करना क्योंकि ये लोग ज़रुर डुबा दिए जाएँगें
\end{hindi}}
\flushright{\begin{Arabic}
\quranayah[11][38]
\end{Arabic}}
\flushleft{\begin{hindi}
और नूह कश्ती बनाने लगे और जब कभी उनकी क़ौम के सरबर आवुरदा लोग उनके पास से गुज़रते थे तो उनसे मसख़रापन करते नूह (जवाब में) कहते कि अगर इस वक्त तुम हमसे मसखरापन करते हो तो जिस तरह तुम हम पर हँसते हो हम तुम पर एक वक्त हँसेगें
\end{hindi}}
\flushright{\begin{Arabic}
\quranayah[11][39]
\end{Arabic}}
\flushleft{\begin{hindi}
और तुम्हें अनक़रीब ही मालूम हो जाएगा कि किस पर अज़ाब नाज़िल होता है कि (दुनिया में) उसे रुसवा कर दे और किस पर (क़यामत में) दाइमी अज़ाब नाज़िल होता है
\end{hindi}}
\flushright{\begin{Arabic}
\quranayah[11][40]
\end{Arabic}}
\flushleft{\begin{hindi}
यहाँ तक कि जब हमारा हुक्म (अज़ाब) आ पहुँचा और तन्नूर से जोश मारने लगा तो हमने हुक्म दिया (ऐ नूह) हर किस्म के जानदारों में से (नर मादा का) जोड़ा (यानि) दो दो ले लो और जिस (की) हलाकत (तबाही) का हुक्म पहले ही हो चुका हो उसके सिवा अपने सब घर वाले और जो लोग ईमान ला चुके उन सबको कश्ती (नाँव) में बैठा लो और उनके साथ ईमान भी थोड़े ही लोग लाए थे
\end{hindi}}
\flushright{\begin{Arabic}
\quranayah[11][41]
\end{Arabic}}
\flushleft{\begin{hindi}
और नूह ने (अपने साथियों से) कहा बिस्मिल्ला मज़रीहा मुरसाहा (ख़ुदा ही के नाम से उसका बहाओ और ठहराओ है) कश्ती में सवार हो जाओ बेशक मेरा परवरदिगार बड़ा बख्शने वाला मेहरबान है
\end{hindi}}
\flushright{\begin{Arabic}
\quranayah[11][42]
\end{Arabic}}
\flushleft{\begin{hindi}
और कश्ती है कि पहाड़ों की सी (ऊँची) लहरों में उन लोगों को लिए हुए चली जा रही है और नूह ने अपने बेटे को जो उनसे अलग थलग एक गोशे (कोने) में था आवाज़ दी ऐ मेरे फरज़न्द हमारी कश्ती में सवार हो लो और काफिरों के साथ न रह
\end{hindi}}
\flushright{\begin{Arabic}
\quranayah[11][43]
\end{Arabic}}
\flushleft{\begin{hindi}
(मुझे माफ कीजिए) मै तो अभी किसी पहाड़ का सहारा पकड़ता हूँ जो मुझे पानी (में डूबने) से बचा लेगा नूह ने (उससे) कहा (अरे कम्बख्त) आज ख़ुदा के अज़ाब से कोई बचाने वाला नहीं मगर ख़ुदा ही जिस पर रहम फरमाएगा और (ये बात हो रही थी कि) यकायक दोनो बाप बेटे के दरमियान एक मौज हाएल हो गई और वह डूब कर रह गया
\end{hindi}}
\flushright{\begin{Arabic}
\quranayah[11][44]
\end{Arabic}}
\flushleft{\begin{hindi}
और (ग़ैब ख़ुदा की तरफ से) हुक्म दिया गया कि ऐ ज़मीन अपना पानी जज्ब (शोख) करे और ऐ आसमान (बरसने से) थम जा और पानी घट गया और (लोगों का) काम तमाम कर दिया गया और कश्ती जो वही (पहाड़) पर जा ठहरी और (चारो तरफ) पुकार दिया गया कि ज़ालिम लोगों को (ख़ुदा की रहमत से) दूरी हो
\end{hindi}}
\flushright{\begin{Arabic}
\quranayah[11][45]
\end{Arabic}}
\flushleft{\begin{hindi}
और (जिस वक्त नूह का बेटा ग़रक (डूब) हो रहा था तो नूह ने अपने परवरदिगार को पुकारा और अर्ज़ की ऐ मेरे परवरदिगार इसमें तो शक़ नहीं कि मेरा बेटा मेरे अहल (घर वालों) में शामिल है और तूने वायदा किया था कि तेरे अहल को बचा लूँगा) और इसमें शक़ नहीं कि तेरा वायदा सच्चा है और तू सारे (जहान) के हाकिमों से बड़ा हाकिम है
\end{hindi}}
\flushright{\begin{Arabic}
\quranayah[11][46]
\end{Arabic}}
\flushleft{\begin{hindi}
(तू मेरे बेटे को नजात दे) ख़ुदा ने फरमाया ऐ नूह तुम (ये क्या कह रहे हो) हरगिज़ वह तुम्हारे अहल में शामिल नहीं वह बेशक बदचलन है (देखो जिसका तुम्हें इल्म नहीं है मुझसे उसके बारे में (दरख्वास्त न किया करो और नादानों की सी बातें न करो) नूह ने अर्ज़ की ऐ मेरे परवरदिगार मै तुझ ही से पनाह मागँता हूँ कि जिस चीज़ का मुझे इल्म न हो मै उसकी दरख्वास्त करुँ
\end{hindi}}
\flushright{\begin{Arabic}
\quranayah[11][47]
\end{Arabic}}
\flushleft{\begin{hindi}
और अगर तु मुझे (मेरे कसूर न बख्श देगा और मुझ पर रहम न खाएगा तो मैं सख्त घाटा उठाने वालों में हो जाऊँगा (जब तूफान जाता रहा तो) हुक्म दिया गया ऐ नूह हमारी तरफ से सलामती और उन बरकतों के साथ कश्ती से उतरो
\end{hindi}}
\flushright{\begin{Arabic}
\quranayah[11][48]
\end{Arabic}}
\flushleft{\begin{hindi}
जो तुम पर हैं और जो लोग तुम्हारे साथ हैं उनमें से न कुछ लोगों पर और (तुम्हारे बाद) कुछ लोग ऐसे भी हैं जिन्हें हम थोड़े ही दिन बाद बहरावर करेगें फिर हमारी तरफ से उनको दर्दनाक अज़ाब पहुँचेगा
\end{hindi}}
\flushright{\begin{Arabic}
\quranayah[11][49]
\end{Arabic}}
\flushleft{\begin{hindi}
(ऐ रसूल) ये ग़ैब की चन्द ख़बरे हैं जिनको तुम्हारी तरफ वही के ज़रिए पहुँचाते हैं जो उसके क़ब्ल न तुम जानते थे और न तुम्हारी क़ौम ही (जानती थी) तो तुम सब्र करो इसमें शक़ नहीं कि आख़िारत (की खूबियाँ) परहेज़गारों ही के वास्ते हैं
\end{hindi}}
\flushright{\begin{Arabic}
\quranayah[11][50]
\end{Arabic}}
\flushleft{\begin{hindi}
और (हमने) क़ौमे आद के पास उनके भाई हूद को (पैग़म्बर बनाकर भेजा और) उन्होनें अपनी क़ौम से कहा ऐ मेरी क़ौम ख़ुदा ही की परसतिश करों उसके सिवा कोई तुम्हारा माबूद नहीं तुम बस निरे इफ़तेरा परदाज़ (झूठी बात बनाने वाले) हो
\end{hindi}}
\flushright{\begin{Arabic}
\quranayah[11][51]
\end{Arabic}}
\flushleft{\begin{hindi}
ऐ मेरी क़ौम मै उस (समझाने पर तुमसे कुछ मज़दूरी नहीं मॉगता मेरी मज़दूरी तो बस उस शख़्श के ज़िम्मे है जिसने मुझे पैदा किया तो क्या तुम (इतना भी) नहीं समझते
\end{hindi}}
\flushright{\begin{Arabic}
\quranayah[11][52]
\end{Arabic}}
\flushleft{\begin{hindi}
और ऐ मेरी क़ौम अपने परवरदिगार से मग़फिरत की दुआ मॉगों फिर उसकी बारगाह में अपने (गुनाहों से) तौबा करो तो वह तुम पर मूसलाधार मेह आसमान से बरसाएगा ख़ुश्क साली न होगी और तुम्हारी क़ूवत (ताक़त) में और क़ूवत बढ़ा देगा और मुजरिम बन कर उससे मुँह न मोड़ों
\end{hindi}}
\flushright{\begin{Arabic}
\quranayah[11][53]
\end{Arabic}}
\flushleft{\begin{hindi}
वह लोग कहने लगे ऐ हूद तुम हमारे पास कोई दलील लेकर तो आए नहीं और तुम्हारे कहने से अपने ख़ुदाओं को तो छोड़ने वाले नहीं और न हम तुम पर ईमान लाने वाले हैं
\end{hindi}}
\flushright{\begin{Arabic}
\quranayah[11][54]
\end{Arabic}}
\flushleft{\begin{hindi}
हम तो बस ये कहते हैं कि हमारे ख़ुदाओं में से किसने तुम्हें मजनून (दीवाना) बना दिया है (इसी वजह से तुम) बहकी बहकी बातें करते हो हूद ने जवाब दिया बेशक मै ख़ुदा को गवाह करता हूँ और तुम भी गवाह रहो कि तुम ख़ुदा के सिवा (दूसरों को) उसका शरीक बनाते हो
\end{hindi}}
\flushright{\begin{Arabic}
\quranayah[11][55]
\end{Arabic}}
\flushleft{\begin{hindi}
इसमे मै बेज़ार हूँ तो तुम सब के सब मेरे साथ मक्कारी करो और मुझे (दम मारने की) मोहलत भी न दो तो मुझे परवाह नहीं
\end{hindi}}
\flushright{\begin{Arabic}
\quranayah[11][56]
\end{Arabic}}
\flushleft{\begin{hindi}
मै तो सिर्फ ख़ुदा पर भरोसा रखता हूँ जो मेरा भी परवरदिगार है और तुम्हारा भी परवरदिगार है और रुए ज़मीन पर जितने चलने वाले हैं सबकी चोटी उसी के साथ है इसमें तो शक़ ही नहीं कि मेरा परवरदिगार (इन्साफ की) सीधी राह पर है
\end{hindi}}
\flushright{\begin{Arabic}
\quranayah[11][57]
\end{Arabic}}
\flushleft{\begin{hindi}
इस पर भी अगर तुम उसके हुक्म से मुँह फेरे रहो तो जो हुक्म दे कर मैं तुम्हारे पास भेजा गया था उसे तो मैं यक़ीनन पहुँचा चुका और मेरा परवरदिगार (तुम्हारी नाफरमानी पर तुम्हें हलाक करें) तुम्हारे सिवा दूसरी क़ौम को तुम्हारा जानशीन करेगा और तुम उसका कुछ भी बिगाड़ नहीं सकते इसमें तो शक़ नहीं है कि मेरा परवरदिगार हर चीज़ का निगेहबान है
\end{hindi}}
\flushright{\begin{Arabic}
\quranayah[11][58]
\end{Arabic}}
\flushleft{\begin{hindi}
और जब हमारा (अज़ाब का) हुक्म आ पहुँचा तो हमने हूद को और जो लोग उसके साथ ईमान लाए थे अपनी मेहरबानी से नजात दिया और उन सबको सख्त अज़ाब से बचा लिया
\end{hindi}}
\flushright{\begin{Arabic}
\quranayah[11][59]
\end{Arabic}}
\flushleft{\begin{hindi}
(ऐ रसूल) ये हालात क़ौमे आद के हैं जिन्होंने अपने परवरदिगार की आयतों से इन्कार किया और उसके पैग़म्बरों की नाफ़रमानी की और हर सरकश (दुश्मने ख़ुदा) के हुक्म पर चलते रहें
\end{hindi}}
\flushright{\begin{Arabic}
\quranayah[11][60]
\end{Arabic}}
\flushleft{\begin{hindi}
और इस दुनिया में भी लानत उनके पीछे लगा दी गई और क़यामत के दिन भी (लगी रहेगी) देख क़ौमे आद ने अपने परवरदिगार का इन्कार किया देखो हूद की क़ौमे आद (हमारी बारगाह से) धुत्कारी पड़ी है
\end{hindi}}
\flushright{\begin{Arabic}
\quranayah[11][61]
\end{Arabic}}
\flushleft{\begin{hindi}
और (हमने) क़ौमे समूद के पास उनके भाई सालेह को (पैग़म्बर बनाकर भेजा) तो उन्होंने (अपनी क़ौम से) कहा ऐ मेरी क़ौम ख़ुदा ही की परसतिश करो उसके सिवा कोई तुम्हारा माबूद नहीं उसी ने तुमको ज़मीन (की मिट्टी) से पैदा किया और तुमको उसमें बसाया तो उससे मग़फिरत की दुआ मॉगों फिर उसकी बारगाह में तौबा करो (बेशक मेरा परवरदिगार (हर शख़्श के) क़रीब और सबकी सुनता और दुआ क़ुबूल करता है
\end{hindi}}
\flushright{\begin{Arabic}
\quranayah[11][62]
\end{Arabic}}
\flushleft{\begin{hindi}
वह लोग कहने लगे ऐ सालेह इसके पहले तो तुमसे हमारी उम्मीदें वाबस्ता थी तो क्या अब तुम जिस चीज़ की परसतिश हमारे बाप दादा करते थे उसकी परसतिश से हमें रोकते हो और जिस दीन की तरफ तुम हमें बुलाते हो हम तो उसकी निस्बत ऐसे शक़ में पड़े हैं
\end{hindi}}
\flushright{\begin{Arabic}
\quranayah[11][63]
\end{Arabic}}
\flushleft{\begin{hindi}
कि उसने हैरत में डाल दिया है सालेह ने जवाब दिया ऐ मेरी क़ौम भला देखो तो कि अगर मैं अपने परवरदिगार की तरफ से रौशन दलील पर हूँ और उसने मुझे अपनी (बारगाह) मे रहमत (नबूवत) अता की है इस पर भी अगर मै उसकी नाफ़रमानी करुँ तो ख़ुदा (के अज़ाब से बचाने में) मेरी मदद कौन करेगा-फिर तुम सिवा नुक़सान के मेरा कुछ बढ़ा दोगे नहीं
\end{hindi}}
\flushright{\begin{Arabic}
\quranayah[11][64]
\end{Arabic}}
\flushleft{\begin{hindi}
ऐ मेरी क़ौम ये ख़ुदा की (भेजी हुई) ऊँटनी है तुम्हारे वास्ते (मेरी नबूवत का) एक मौजिज़ा है तो इसको (उसके हाल पर) छोड़ दो कि ख़ुदा की ज़मीन में (जहाँ चाहे) खाए और उसे कोई तकलीफ न पहुँचाओ
\end{hindi}}
\flushright{\begin{Arabic}
\quranayah[11][65]
\end{Arabic}}
\flushleft{\begin{hindi}
(वरना) फिर तुम्हें फौरन ही (ख़ुदा का) अज़ाब ले डालेगा इस पर भी उन लोगों ने उसकी कूँचे काटकर (मार) डाला तब सालेह ने कहा अच्छा तीन दिन तक (और) अपने अपने घर में चैन (उड़ा लो)
\end{hindi}}
\flushright{\begin{Arabic}
\quranayah[11][66]
\end{Arabic}}
\flushleft{\begin{hindi}
यही ख़ुदा का वायदा है जो कभी झूठा नहीं होता फिर जब हमारा (अज़ाब का) हुक्म आ पहुँचा तो हमने सालेह और उन लोगों को जो उसके साथ ईमान लाए थे अपनी मेहरबानी से नजात दी और उस दिन की रुसवाई से बचा लिया इसमें शक़ नहीं कि तेरा परवरदिगार ज़बरदस्त ग़ालिब है
\end{hindi}}
\flushright{\begin{Arabic}
\quranayah[11][67]
\end{Arabic}}
\flushleft{\begin{hindi}
और जिन लोगों ने ज़ुल्म किया था उनको एक सख्त चिघाड़ ने ले डाला तो वह लोग अपने अपने घरों में औंधें पड़े रह गये
\end{hindi}}
\flushright{\begin{Arabic}
\quranayah[11][68]
\end{Arabic}}
\flushleft{\begin{hindi}
और ऐसे मर मिटे कि गोया उनमें कभी बसे ही न थे तो देखो क़ौमे समूद ने अपने परवरदिगार की नाफरमानी की और (सज़ा दी गई) सुन रखो कि क़ौमे समूद (उसकी बारगाह से) धुत्कारी हुईहै
\end{hindi}}
\flushright{\begin{Arabic}
\quranayah[11][69]
\end{Arabic}}
\flushleft{\begin{hindi}
और हमारे भेजे हुए (फरिश्ते) इबराहीम के पास खुशख़बरी लेकर आए और उन्होंने (इबराहीम को) सलाम किया (इबराहीम ने) सलाम का जवाब दिया फिर इबराहीम एक बछड़े का भुना हुआ (गोश्त) ले आए
\end{hindi}}
\flushright{\begin{Arabic}
\quranayah[11][70]
\end{Arabic}}
\flushleft{\begin{hindi}
(और साथ खाने बैठें) फिर जब देखा कि उनके हाथ उसकी तरफ नहीं बढ़ते तो उनकी तरफ से बदगुमान हुए और जी ही जी में डर गए (उसको वह फरिश्ते समझे) और कहने लगे आप डरे नहीं हम तो क़ौम लूत की तरफ (उनकी सज़ा के लिए) भेजे गए हैं
\end{hindi}}
\flushright{\begin{Arabic}
\quranayah[11][71]
\end{Arabic}}
\flushleft{\begin{hindi}
और इबराहीम की बीबी (सायरा) खड़ी हुई थी वह (ये ख़बर सुनकर) हॅस पड़ी तो हमने (उन्हेंफ़रिश्तों के ज़रिए से) इसहाक़ के पैदा होने की खुशख़बरी दी और इसहाक़ के बाद याक़ूब की
\end{hindi}}
\flushright{\begin{Arabic}
\quranayah[11][72]
\end{Arabic}}
\flushleft{\begin{hindi}
वह कहने लगी ऐ है क्या अब मै बच्चा जनने बैठॅूगी मैं तो बुढ़िया हूँ और ये मेरे मियॉ भी बूढे है ये तो एक बड़ी ताज्जुब खेज़ बात है
\end{hindi}}
\flushright{\begin{Arabic}
\quranayah[11][73]
\end{Arabic}}
\flushleft{\begin{hindi}
वह फरिश्ते बोले (हाए) तुम ख़ुदा की कुदरत से ताज्जुब करती हो ऐ अहले बैत (नबूवत) तुम पर ख़ुदा की रहमत और उसकी बरकते (नाज़िल हो) इसमें शक़ नहीं कि वह क़ाबिल हम्द (वासना) बुज़ुर्ग हैं
\end{hindi}}
\flushright{\begin{Arabic}
\quranayah[11][74]
\end{Arabic}}
\flushleft{\begin{hindi}
फिर जब इबराहीम (के दिल) से ख़ौफ जाता रहा और उनके पास (औलाद की) खुशख़बरी भी आ चुकी तो हम से क़ौमे लूत के बारे में झगड़ने लगे
\end{hindi}}
\flushright{\begin{Arabic}
\quranayah[11][75]
\end{Arabic}}
\flushleft{\begin{hindi}
बेशक इबराहीम बुर्दबार नरम दिल (हर बात में ख़ुदा की तरफ) रुजू (ध्यान) करने वाले थे
\end{hindi}}
\flushright{\begin{Arabic}
\quranayah[11][76]
\end{Arabic}}
\flushleft{\begin{hindi}
(हमने कहा) ऐ इबराहीम इस बात में हट मत करो (इस बार में) जो हुक्म तुम्हारे परवरदिगार का था वह क़तअन आ चुका और इसमें शक़ नहीं कि उन पर ऐसा अज़ाब आने वाले वाला है
\end{hindi}}
\flushright{\begin{Arabic}
\quranayah[11][77]
\end{Arabic}}
\flushleft{\begin{hindi}
जो किसी तरह टल नहीं सकता और जब हमारे भेजे हुए फरिश्ते (लड़को की सूरत में) लूत के पास आए तो उनके ख्याल से रजीदा हुए और उनके आने से तंग दिल हो गए और कहने लगे कि ये (आज का दिन) सख्त मुसीबत का दिन है
\end{hindi}}
\flushright{\begin{Arabic}
\quranayah[11][78]
\end{Arabic}}
\flushleft{\begin{hindi}
और उनकी क़ौम (लड़को की आवाज़ सुनकर बुरे इरादे से) उनके पास दौड़ती हुई आई और ये लोग उसके क़ब्ल भी बुरे काम किया करते थे लूत ने (जब उनको) आते देखा तो कहा ऐ मेरी क़ौम ये मारी क़ौम की बेटियाँ (मौजूद हैं) उनसे निकाह कर लो ये तुम्हारीे वास्ते जायज़ और ज्यादा साफ सुथरी हैं तो खुदा से डरो और मुझे मेरे मेहमान के बारे में रुसवा न करो क्या तुम में से कोई भी समझदार आदमी नहीं है
\end{hindi}}
\flushright{\begin{Arabic}
\quranayah[11][79]
\end{Arabic}}
\flushleft{\begin{hindi}
उन (कम्बख्तो) न जवाब दिया तुम को खूब मालूम है कि तुम्हारी क़ौम की लड़कियों की हमें कुछ हाजत (जरूरत) नही है और जो बात हम चाहते है वह तो तुम ख़ूब जानते हो
\end{hindi}}
\flushright{\begin{Arabic}
\quranayah[11][80]
\end{Arabic}}
\flushleft{\begin{hindi}
लूत ने कहा काश मुझमें तुम्हारे मुक़ाबले की कूवत होती या मै किसी मज़बूत क़िले मे पनाह ले सकता
\end{hindi}}
\flushright{\begin{Arabic}
\quranayah[11][81]
\end{Arabic}}
\flushleft{\begin{hindi}
वह फरिश्ते बोले ऐ लूत हम तुम्हारे परवरदिगार के भेजे हुए (फरिश्ते हैं तुम घबराओ नहीं) ये लोग तुम तक हरगिज़ (नहीं पहुँच सकते तो तुम कुछ रात रहे अपने लड़कों बालों समैत निकल भागो और तुममें से कोई इधर मुड़ कर भी न देखे मगर तुम्हारी बीबी कि उस पर भी यक़ीनन वह अज़ाब नाज़िल होने वाला है जो उन लोगों पर नाज़िल होगा और उन (के अज़ाब का) वायदा बस सुबह है क्या सुबह क़रीब नहीं
\end{hindi}}
\flushright{\begin{Arabic}
\quranayah[11][82]
\end{Arabic}}
\flushleft{\begin{hindi}
फिर जब हमारा (अज़ाब का) हुक्म आ पहुँचा तो हमने (बस्ती की ज़मीन के तबके) उलट कर उसके ऊपर के हिस्से को नीचे का बना दिया और उस पर हमने खरन्जेदार पत्थर ताबड़ तोड़ बरसाए
\end{hindi}}
\flushright{\begin{Arabic}
\quranayah[11][83]
\end{Arabic}}
\flushleft{\begin{hindi}
जिन पर तुम्हारे परवरदिगार की तरफ से निशान बनाए हुए थे और वह बस्ती (उन) ज़ालिमों (कुफ्फ़ारे मक्का) से कुछ दूर नहीं
\end{hindi}}
\flushright{\begin{Arabic}
\quranayah[11][84]
\end{Arabic}}
\flushleft{\begin{hindi}
और हमने मदयन वालों के पास उनके भाई शुएब को पैग़म्बर बना कर भेजा उन्होंने (अपनी क़ौम से) कहा ऐ मेरी क़ौम ख़ुदा की इबादत करो उसके सिवा तुम्हारा कोई ख़ुदा नहीं और नाप और तौल में कोई कमी न किया करो मै तो तुम को आसूदगी (ख़ुशहाली) में देख रहा हूँ (फिर घटाने की क्या ज़रुरत है) और मै तो तुम पर उस दिन के अज़ाब से डराता हूँ जो (सबको) घेर लेगा
\end{hindi}}
\flushright{\begin{Arabic}
\quranayah[11][85]
\end{Arabic}}
\flushleft{\begin{hindi}
और ऐ मेरी क़ौम पैमाने और तराज़ू ऌन्साफ़ के साथ पूरे पूरे रखा करो और लोगों को उनकी चीज़े कम न दिया करो और रुए ज़मीन में फसाद न फैलाते फिरो
\end{hindi}}
\flushright{\begin{Arabic}
\quranayah[11][86]
\end{Arabic}}
\flushleft{\begin{hindi}
अगर तुम सच्चे मोमिन हो तो ख़ुदा का बक़िया तुम्हारे वास्ते कही अच्छा है और मैं तो कुछ तुम्हारा निगेहबान नहीं
\end{hindi}}
\flushright{\begin{Arabic}
\quranayah[11][87]
\end{Arabic}}
\flushleft{\begin{hindi}
वह लोग कहने लगे ऐ शुएब क्या तुम्हारी नमाज़ (जिसे तुम पढ़ा करते हो) तुम्हें ये सिखाती है कि जिन (बुतों) की परसतिश हमारे बाप दादा करते आए उन्हें हम छोड़ बैठें या हम अपने मालों में जो कुछ चाहे कर बैठें तुम ही तो बस एक बुर्दबार और समझदार (रह गए) हो
\end{hindi}}
\flushright{\begin{Arabic}
\quranayah[11][88]
\end{Arabic}}
\flushleft{\begin{hindi}
शुएब ने कहा ऐ मेरी क़ौम अगर मै अपने परवरदिगार की तरफ से रौशन दलील पर हूँ और उसने मुझे (हलाल) रोज़ी खाने को दी है (तो मै भी तुम्हारी तरह हराम खाने लगूँ) और मै तो ये नहीं चाहता कि जिस काम से तुम को रोकूँ तुम्हारे बर ख़िलाफ (बदले) आप उसको करने लगूं मैं तो जहाँ तक मुझे बन पड़े इसलाह (भलाई) के सिवा (कुछ और) चाहता ही नहीं और मेरी ताईद तो ख़ुदा के सिवा और किसी से हो ही नहीं सकती इस पर मैने भरोसा कर लिया है और उसी की तरफ रुज़ू करता हूँ
\end{hindi}}
\flushright{\begin{Arabic}
\quranayah[11][89]
\end{Arabic}}
\flushleft{\begin{hindi}
और ऐ मेरी क़ौमे मेरी ज़िद कही तुम से ऐसा जुर्म न करा दे जैसी मुसीबत क़ौम नूह या हूद या सालेह पर नाज़िल हुई थी वैसी ही मुसीबत तुम पर भी आ पड़े और लूत की क़ौम (का ज़माना) तो (कुछ ऐसा) तुमसे दूर नहीं (उन्हीं के इबरत हासिल करो
\end{hindi}}
\flushright{\begin{Arabic}
\quranayah[11][90]
\end{Arabic}}
\flushleft{\begin{hindi}
और अपने परवरदिगार से अपनी मग़फिरत की दुआ माँगों फिर उसी की बारगाह में तौबा करो बेशक मेरा परवरदिगार बड़ा मोहब्बत वाला मेहरबान है
\end{hindi}}
\flushright{\begin{Arabic}
\quranayah[11][91]
\end{Arabic}}
\flushleft{\begin{hindi}
और वह लोग कहने लगे ऐ शुएब जो बाते तुम कहते हो उनमें से अक्सर तो हमारी समझ ही में नहीं आयी और इसमें तो शक नहीं कि हम तुम्हें अपने लोगों में बहुत कमज़ोर समझते है और अगर तुम्हारा क़बीला न होता तो हम तुम को (कब का) संगसार कर चुके होते और तुम तो हम पर किसी तरह ग़ालिब नहीं आ सकते
\end{hindi}}
\flushright{\begin{Arabic}
\quranayah[11][92]
\end{Arabic}}
\flushleft{\begin{hindi}
शुएब ने कहा ऐ मेरी क़ौम क्या मेरे कबीले का दबाव तुम पर ख़ुदा से भी बढ़ कर है (कि तुम को उसका ये ख्याल) और ख़ुदा को तुम लोगों ने अपने वास्ते पीछे डाल दिया है बेशक मेरा परवरदिगार तुम्हारे सब आमाल पर अहाता किए हुए है
\end{hindi}}
\flushright{\begin{Arabic}
\quranayah[11][93]
\end{Arabic}}
\flushleft{\begin{hindi}
और ऐ मेरी क़ौम तुम अपनी जगह (जो चाहो) करो मैं भी (बजाए खुद) कुछ करता हू अनक़रीब ही तुम्हें मालूम हो जाएगा कि किस पर अज़ाब नाज़िल होता है जा उसको (लोगों की नज़रों में) रुसवा कर देगा और (ये भी मालूम हो जाएगा कि) कौन झूठा है तुम भी मुन्तिज़र रहो मैं भी तुम्हारे साथ इन्तेज़ार करता हूँ
\end{hindi}}
\flushright{\begin{Arabic}
\quranayah[11][94]
\end{Arabic}}
\flushleft{\begin{hindi}
और जब हमारा (अज़ाब का) हुक्म आ पहुँचा तो हमने शुएब और उन लोगों को जो उसके साथ ईमान लाए थे अपनी मेहरबानी से बचा लिया और जिन लोगों ने ज़ुल्म किया था उनको एक चिंघाड़ ने ले डाला फिर तो वह सबके सब अपने घरों में औंधे पड़े रह गए
\end{hindi}}
\flushright{\begin{Arabic}
\quranayah[11][95]
\end{Arabic}}
\flushleft{\begin{hindi}
(और वह ऐसे मर मिटे) कि गोया उन बस्तियों में कभी बसे ही न थे सुन रखो कि जिस तरह समूद (ख़ुदा की बारगाह से) धुत्कारे गए उसी तरह अहले मदियन की भी धुत्कारी हुई
\end{hindi}}
\flushright{\begin{Arabic}
\quranayah[11][96]
\end{Arabic}}
\flushleft{\begin{hindi}
और बेशक हमने मूसा को अपनी निशानियाँ और रौशन दलील देकर
\end{hindi}}
\flushright{\begin{Arabic}
\quranayah[11][97]
\end{Arabic}}
\flushleft{\begin{hindi}
फिरऔन और उसके अम्र (सरदारों) के पास (पैग़म्बर बना कर) भेजा तो लोगों ने फिरऔन ही का हुक्म मान लिया (और मूसा की एक न सुनी) हालॉकि फिरऔन का हुक्म कुछ जॅचा समझा हुआ न था
\end{hindi}}
\flushright{\begin{Arabic}
\quranayah[11][98]
\end{Arabic}}
\flushleft{\begin{hindi}
क़यामत के दिन वह अपनी क़ौम के आगे आगे चलेगा और उनको दोज़ख़ में ले जाकर झोंक देगा और ये लोग किस क़दर बड़े घाट उतारे गए
\end{hindi}}
\flushright{\begin{Arabic}
\quranayah[11][99]
\end{Arabic}}
\flushleft{\begin{hindi}
और (इस दुनिया) में भी लानत उनके पीछे पीछे लगा दी गई और क़यामत के दिन भी (लगी रहेगी) क्या बुरा इनाम है जो उन्हें मिला
\end{hindi}}
\flushright{\begin{Arabic}
\quranayah[11][100]
\end{Arabic}}
\flushleft{\begin{hindi}
(ऐ रसूल) ये चन्द बस्तियों के हालात हैं जो हम तुम से बयान करते हैं उनमें से बाज़ तो (उस वक्त तक) क़ायम हैं और बाज़ का तहस नहस हो गया
\end{hindi}}
\flushright{\begin{Arabic}
\quranayah[11][101]
\end{Arabic}}
\flushleft{\begin{hindi}
और हमने किसी तरह उन पर ज़ल्म नहीं किया बल्कि उन लोगों ने आप अपने ऊपर (नाफरमानी करके) ज़ुल्म किया फिर जब तुम्हारे परवरदिगार का (अज़ाब का) हुक्म आ पहुँचा तो न उसके वह माबूद ही काम आए जिन्हें ख़ुदा को छोड़कर पुकारा करते थें और न उन माबूदों ने हलाक करने के सिवा कुछ फायदा ही पहुँचाया बल्कि उन्हीं की परसतिश की बदौलत अज़ाब आया
\end{hindi}}
\flushright{\begin{Arabic}
\quranayah[11][102]
\end{Arabic}}
\flushleft{\begin{hindi}
और (ऐ रसूल) बस्तियों के लोगों की सरकशी से जब तुम्हारा परवरदिगार अज़ाब में पकड़ता है तो उसकी पकड़ ऐसी ही होती है बेशक पकड़ तो दर्दनाक (और सख्त) होती है
\end{hindi}}
\flushright{\begin{Arabic}
\quranayah[11][103]
\end{Arabic}}
\flushleft{\begin{hindi}
इसमें तो शक़ नहीं कि उस शख़्श के वास्ते जो अज़ाब आख़िरत से डरता है (हमारी कुदरत की) एक निशानी है ये वह रोज़ होगा कि सारे (जहाँन) के लोग जमा किए जाएंगें और यही वह दिन होगा कि (हमारी बारगाह में) सब हाज़िर किए जाएंगें
\end{hindi}}
\flushright{\begin{Arabic}
\quranayah[11][104]
\end{Arabic}}
\flushleft{\begin{hindi}
और हम बस एक मुअय्युन मुद्दत तक इसमें देर कर रहे है
\end{hindi}}
\flushright{\begin{Arabic}
\quranayah[11][105]
\end{Arabic}}
\flushleft{\begin{hindi}
जिस दिन वह आ पहुँचेगा तो बग़ैर हुक्मे ख़ुदा कोई शख़्श बात भी तो नहीं कर सकेगा फिर कुछ लोग उनमे से बदबख्त होगें और कुछ लोग नेक बख्त
\end{hindi}}
\flushright{\begin{Arabic}
\quranayah[11][106]
\end{Arabic}}
\flushleft{\begin{hindi}
तो जो लोग बदबख्त है वह दोज़ख़ में होगें और उसी में उनकी हाए वाए और चीख़ पुकार होगी
\end{hindi}}
\flushright{\begin{Arabic}
\quranayah[11][107]
\end{Arabic}}
\flushleft{\begin{hindi}
वह लोग जब तक आसमान और ज़मीन में है हमेशा उसी मे रहेगें मगर जब तुम्हारा परवरदिगार (नजात देना) चाहे बेशक तुम्हारा परवरदिगार जो चाहता है कर ही डालता है
\end{hindi}}
\flushright{\begin{Arabic}
\quranayah[11][108]
\end{Arabic}}
\flushleft{\begin{hindi}
और जो लोग नेक बख्त हैं वह तो बेहश्त में होगें (और) जब तक आसमान व ज़मीन (बाक़ी) है वह हमेशा उसी में रहेगें मगर जब तेरा परवरदिगार चाहे (सज़ा देकर आख़िर में जन्नत में ले जाए
\end{hindi}}
\flushright{\begin{Arabic}
\quranayah[11][109]
\end{Arabic}}
\flushleft{\begin{hindi}
ये वह बख़्शिस है जो कभी मनक़तआ (खत्म) न होगी तो ये लोग (ख़ुदा के अलावा) जिसकी परसतिश करते हैं तुम उससे शक़ में न पड़ना ये लोग तो बस वैसी इबादत करते हैं जैसी उनसे पहले उनके बाप दादा करते थे और हम ज़रुर (क़यामत के दिन) उनको (अज़ाब का) पूरा पूरा हिस्सा बग़ैर कम किए देगें
\end{hindi}}
\flushright{\begin{Arabic}
\quranayah[11][110]
\end{Arabic}}
\flushleft{\begin{hindi}
और हमने मूसा को किताब तौरैत अता की तो उसमें (भी) झगड़े डाले गए और अगर तुम्हारे परवरदिगार की तरफ से हुक्म कोइ पहले ही न हो चुका होता तो उनके दरमियान (कब का) फैसला यक़ीनन हो गया होता और ये लोग (कुफ्फ़ारे मक्का) भी इस (क़ुरान) की तरफ से बहुत गहरे शक़ में पड़े हैं
\end{hindi}}
\flushright{\begin{Arabic}
\quranayah[11][111]
\end{Arabic}}
\flushleft{\begin{hindi}
और इसमें तो शक़ ही नहीं कि तुम्हारा परवरदिगार उनकी कारस्तानियों का बदला भरपूर देगा (क्योंकि) जो उनकी करतूतें हैं उससे वह खूब वाक़िफ है
\end{hindi}}
\flushright{\begin{Arabic}
\quranayah[11][112]
\end{Arabic}}
\flushleft{\begin{hindi}
तो (ऐ रसूल) जैसा तुम्हें हुक्म दिया है तुम और वह लोग भी जिन्होंने तुम्हारे साथ (कुफ्र से) तौबा की है ठीक साबित क़दम रहो और सरकशी न करो (क्योंकि) तुम लोग जो कुछ भी करते हो वह यक़ीनन देख रहा है
\end{hindi}}
\flushright{\begin{Arabic}
\quranayah[11][113]
\end{Arabic}}
\flushleft{\begin{hindi}
और (मुसलमानों) जिन लोगों ने (हमारी नाफरमानी करके) अपने ऊपर ज़ुल्म किया है उनकी तरफ माएल (झुकना) न होना और वरना तुम तक भी (दोज़ख़) की आग आ लपटेगी और ख़ुदा के सिवा और लोग तुम्हारे सरपरस्त भी नहीं हैं फिर तुम्हारी मदद कोई भी नहीं करेगा
\end{hindi}}
\flushright{\begin{Arabic}
\quranayah[11][114]
\end{Arabic}}
\flushleft{\begin{hindi}
और (ऐ रसूल) दिन के दोनो किनारे और कुछ रात गए नमाज़ पढ़ा करो (क्योंकि) नेकियाँ यक़ीनन गुनाहों को दूर कर देती हैं और (हमारी) याद करने वालो के लिए ये (बातें) नसीहत व इबरत हैं
\end{hindi}}
\flushright{\begin{Arabic}
\quranayah[11][115]
\end{Arabic}}
\flushleft{\begin{hindi}
और (ऐ रसूल) तुम सब्र करो क्योंकि ख़ुदा नेकी करने वालों का अज्र बरबाद नहीं करता
\end{hindi}}
\flushright{\begin{Arabic}
\quranayah[11][116]
\end{Arabic}}
\flushleft{\begin{hindi}
फिर जो लोग तुमसे पहले गुज़र चुके हैं उनमें कुछ लोग ऐसे अक़ल वाले क्यों न हुए जो (लोगों को) रुए ज़मीन पर फसाद फैलाने से रोका करते (ऐसे लोग थे तो) मगर बहुत थोड़े से और ये उन्हीं लोगों से थे जिनको हमने अज़ाब से बचा लिया और जिन लोगों ने नाफरमानी की थी वह उन्हीं (लज्ज़तों) के पीछे पड़े रहे और जो उन्हें दी गई थी और ये लोग मुजरिम थे ही
\end{hindi}}
\flushright{\begin{Arabic}
\quranayah[11][117]
\end{Arabic}}
\flushleft{\begin{hindi}
और तुम्हारा परवरदिगार ऐसा (बे इन्साफ) कभी न था कि बस्तियों को जबरदस्ती उजाड़ देता और वहाँ के लोग नेक चलन हों
\end{hindi}}
\flushright{\begin{Arabic}
\quranayah[11][118]
\end{Arabic}}
\flushleft{\begin{hindi}
और अगर तुम्हारा परवरदिगार चाहता तो बेशक तमाम लोगों को एक ही (किस्म की) उम्मत बना देता (मगर) उसने न चाहा इसी (वजह से) लोग हमेशा आपस में फूट डाला करेगें
\end{hindi}}
\flushright{\begin{Arabic}
\quranayah[11][119]
\end{Arabic}}
\flushleft{\begin{hindi}
मगर जिस पर तुम्हारा परवरदिगार रहम फरमाए और इसलिए तो उसने उन लोगों को पैदा किया (और इसी वजह से तो) तुम्हारा परवरदिगार का हुक्म क़तई पूरा होकर रहा कि हम यक़ीनन जहन्नुम को तमाम जिन्नात और आदमियों से भर देगें
\end{hindi}}
\flushright{\begin{Arabic}
\quranayah[11][120]
\end{Arabic}}
\flushleft{\begin{hindi}
और (ऐ रसूल) पैग़म्बरों के हालत में से हम उन तमाम क़िस्सों को तुम से बयान किए देते हैं जिनसे हम तुम्हारे दिल को मज़बूत कर देगें और उन्हीं क़िस्सों में तुम्हारे पास हक़ (क़ुरान) और मोमिनीन के लिए नसीहत और याद दहानी भी आ गई
\end{hindi}}
\flushright{\begin{Arabic}
\quranayah[11][121]
\end{Arabic}}
\flushleft{\begin{hindi}
और (ऐ रसूल) जो लोग ईमान नहीं लाते उनसे कहो कि तुम बजाए ख़ुद अमल करो हम भी कुछ (अमल) करते हैं
\end{hindi}}
\flushright{\begin{Arabic}
\quranayah[11][122]
\end{Arabic}}
\flushleft{\begin{hindi}
(नतीजे का) तुम भी इन्तज़ार करो हम (भी) मुन्तिज़िर है
\end{hindi}}
\flushright{\begin{Arabic}
\quranayah[11][123]
\end{Arabic}}
\flushleft{\begin{hindi}
और सारे आसमान व ज़मीन की पोशीदा बातों का इल्म ख़ास ख़ुदा ही को है और उसी की तरफ हर काम हिर फिर कर लौटता है तुम उसी की इबादत करो और उसी पर भरोसा रखो और जो कुछ तुम लोग करते हो उससे ख़ुदा बेख़बर नहीं
\end{hindi}}
\chapter{Yusuf (Joseph)}
\begin{Arabic}
\Huge{\centerline{\basmalah}}\end{Arabic}
\flushright{\begin{Arabic}
\quranayah[12][1]
\end{Arabic}}
\flushleft{\begin{hindi}
अलिफ़ लाम रा ये वाज़ेए व रौशन किताब की आयतें है
\end{hindi}}
\flushright{\begin{Arabic}
\quranayah[12][2]
\end{Arabic}}
\flushleft{\begin{hindi}
हमने इस किताब (क़ुरान) को अरबी में नाज़िल किया है ताकि तुम समझो
\end{hindi}}
\flushright{\begin{Arabic}
\quranayah[12][3]
\end{Arabic}}
\flushleft{\begin{hindi}
(ऐ रसूल) हम तुम पर ये क़ुरान नाज़िल करके तुम से एक निहायत उम्दा क़िस्सा बयान करते हैं अगरचे तुम इसके पहले (उससे) बिल्कुल बेख़बर थे
\end{hindi}}
\flushright{\begin{Arabic}
\quranayah[12][4]
\end{Arabic}}
\flushleft{\begin{hindi}
(वह वक्त याद करो) जब यूसूफ ने अपने बाप से कहा ऐ अब्बा मैने ग्यारह सितारों और सूरज चाँद को (ख्वाब में) देखा है मैने देखा है कि ये सब मुझे सजदा कर रहे हैं
\end{hindi}}
\flushright{\begin{Arabic}
\quranayah[12][5]
\end{Arabic}}
\flushleft{\begin{hindi}
याक़ूब ने कहा ऐ बेटा (देखो ख़बरदार) कहीं अपना ख्वाब अपने भाईयों से न दोहराना (वरना) वह लोग तुम्हारे लिए मक्कारी की तदबीर करने लगेगें इसमें तो शक़ ही नहीं कि शैतान आदमी का खुला हुआ दुश्मन है
\end{hindi}}
\flushright{\begin{Arabic}
\quranayah[12][6]
\end{Arabic}}
\flushleft{\begin{hindi}
और (जो तुमने देखा है) ऐसा ही होगा कि तुम्हारा परवरदिगार तुमको बरगुज़ीदा (इज्ज़तदार) करेगा और तुम्हें ख्वाबो की ताबीर सिखाएगा और जिस तरह इससे पहले तुम्हारे दादा परदादा इबराहीम और इसहाक़ पर अपनी नेअमत पूरी कर चुका है और इसी तरह तुम पर और याक़ूब की औलाद पर अपनी नेअमत पूरी करेगा बेशक तुम्हारा परवरदिगार बड़ा वाक़िफकार हकीम है
\end{hindi}}
\flushright{\begin{Arabic}
\quranayah[12][7]
\end{Arabic}}
\flushleft{\begin{hindi}
(ऐ रसूल) यूसुफ और उनके भाइयों के किस्से में पूछने वाले (यहूद) के लिए (तुम्हारी नुबूवत) की यक़ीनन बहुत सी निशानियाँ हैं
\end{hindi}}
\flushright{\begin{Arabic}
\quranayah[12][8]
\end{Arabic}}
\flushleft{\begin{hindi}
कि जब (यूसूफ के भाइयों ने) कहा कि बावजूद कि हमारी बड़ी जमाअत है फिर भी यूसुफ़ और उसका हकीक़ी भाई (इब्ने यामीन) हमारे वालिद के नज़दीक बहुत ज्यादा प्यारे हैं इसमें कुछ शक़ नहीं कि हमारे वालिद यक़ीनन सरीही (खुली हुई) ग़लती में पड़े हैं
\end{hindi}}
\flushright{\begin{Arabic}
\quranayah[12][9]
\end{Arabic}}
\flushleft{\begin{hindi}
(ख़ैर तो अब मुनासिब ये है कि या तो) युसूफ को मार डालो या (कम से कम) उसको किसी जगह (चल कर) फेंक आओ तो अलबत्ता तुम्हारे वालिद की तवज्जो सिर्फ तुम्हारी तरफ हो जाएगा और उसके बाद तुम सबके सब (बाप की तवजज्जो से) भले आदमी हो जाओगें
\end{hindi}}
\flushright{\begin{Arabic}
\quranayah[12][10]
\end{Arabic}}
\flushleft{\begin{hindi}
उनमें से एक कहने वाला बोल उठा कि यूसुफ को जान से तो न मारो हाँ अगर तुमको ऐसा ही करना है तो उसको किसी अन्धे कुएँ में (ले जाकर) डाल दो कोई राहगीर उसे निकालकर ले जाएगा (और तुम्हारा मतलब हासिल हो जाएगा)
\end{hindi}}
\flushright{\begin{Arabic}
\quranayah[12][11]
\end{Arabic}}
\flushleft{\begin{hindi}
सब ने (याक़ूब से) कहा अब्बा जान आख़िर उसकी क्या वजह है कि आप यूसुफ के बारे में हमारा ऐतबार नहीं करते
\end{hindi}}
\flushright{\begin{Arabic}
\quranayah[12][12]
\end{Arabic}}
\flushleft{\begin{hindi}
हालॉकि हम लोग तो उसके ख़ैर ख्वाह (भला चाहने वाले) हैं आप उसको कुल हमारे साथ भेज दीजिए कि ज़रा (जंगल) से फल वगैरह् खाए और खेले कूदे
\end{hindi}}
\flushright{\begin{Arabic}
\quranayah[12][13]
\end{Arabic}}
\flushleft{\begin{hindi}
और हम लोग तो उसके निगेहबान हैं ही याक़ूब ने कहा तुम्हारा उसको ले जाना मुझे सख्त सदमा पहुँचाना है और मै तो इससे डरता हूँ कि तुम सब के सब उससे बेख़बर हो जाओ और (मुबादा) उसे भेड़िया फाड़ खाए
\end{hindi}}
\flushright{\begin{Arabic}
\quranayah[12][14]
\end{Arabic}}
\flushleft{\begin{hindi}
वह लोग कहने लगे जब हमारी बड़ी जमाअत है (इस पर भी) अगर उसको भेड़िया खा जाए तो हम लोग यक़ीनन बड़े घाटा उठाने वाले (निकलते) ठहरेगें
\end{hindi}}
\flushright{\begin{Arabic}
\quranayah[12][15]
\end{Arabic}}
\flushleft{\begin{hindi}
ग़रज़ यूसुफ को जब ये लोग ले गए और इस पर इत्तेफ़ाक़ कर लिया कि उसको अन्धे कुएँ में डाल दें और (आख़िर ये लोग गुज़रे तो) हमने युसुफ़ के पास 'वही' भेजी कि तुम घबराओ नहीं हम अनक़रीब तुम्हें मरतबे (उँचे मकाम) पर पहुँचाएगे (तब तुम) उनके उस फेल (बद) से तम्बीह (आग़ाह) करोगे
\end{hindi}}
\flushright{\begin{Arabic}
\quranayah[12][16]
\end{Arabic}}
\flushleft{\begin{hindi}
जब उन्हें कुछ ध्यान भी न होगा और ये लोग रात को अपने बाप के पास (बनवट) से रोते पीटते हुए आए
\end{hindi}}
\flushright{\begin{Arabic}
\quranayah[12][17]
\end{Arabic}}
\flushleft{\begin{hindi}
और कहने लगे ऐ अब्बा हम लोग तो जाकर दौड़ने लगे और यूसुफ को अपने असबाब के पास छोड़ दिया इतने में भेड़िया आकर उसे खा गया हम लोग अगर सच्चे भी हो मगर आपको तो हमारी बात का यक़ीन आने का नहीं
\end{hindi}}
\flushright{\begin{Arabic}
\quranayah[12][18]
\end{Arabic}}
\flushleft{\begin{hindi}
और ये लोग यूसुफ के कुरते पर झूठ मूठ (भेड़) का खून भी (लगा के) लाए थे, याक़ूब ने कहा (भेड़िया ने ही खाया (बल्कि) तुम्हारे दिल ने तुम्हारे बचाओ के लिए एक बात गढ़ी वरना कुर्ता फटा हुआ ज़रुर होता फिर सब्र व शुक्र है और जो कुछ तुम बयान करते हो उस पर ख़ुदा ही से मदद माँगी जाती है
\end{hindi}}
\flushright{\begin{Arabic}
\quranayah[12][19]
\end{Arabic}}
\flushleft{\begin{hindi}
और (ख़ुदा की शान देखो) एक काफ़ला (वहाँ) आकर उतरा उन लोगों ने अपने सक्के (पानी भरने वाले) को (पानी भरने) भेजा ग़रज़ उसने अपना डोल डाला ही था (कि यूसुफ उसमें बैठे और उसने ख़ीचा तो निकल आए) वह पुकारा आहा ये तो लड़का है और काफला वालो ने यूसुफ को क़ीमती सरमाया समझकर छिपा रखा हालॉकि जो कुछ ये लोग करते थे ख़ुदा उससे ख़ूब वाकिफ था
\end{hindi}}
\flushright{\begin{Arabic}
\quranayah[12][20]
\end{Arabic}}
\flushleft{\begin{hindi}
(जब यूसुफ के भाइयों को ख़बर लगी तो आ पहुँचे और उनको अपना ग़ुलाम बताया और उन लोगों ने यूसुफ को गिनती के खोटे चन्द दरहम (बहुत थोड़े दाम पर बेच डाला) और वह लोग तो यूसुफ से बेज़ार हो ही रहे थे
\end{hindi}}
\flushright{\begin{Arabic}
\quranayah[12][21]
\end{Arabic}}
\flushleft{\begin{hindi}
(यूसुफ को लेकर मिस्र पहुँचे और वहाँ उसे बड़े नफे में बेच डाला) और मिस्र के लोगों से (अज़ीजे मिस्र) जिसने (उनको ख़रीदा था अपनी बीवी (ज़ुलेख़ा) से कहने लगा इसको इज्ज़त व आबरु से रखो अजब नहीं ये हमें कुछ नफा पहुँचाए या (शायद) इसको अपना बेटा ही बना लें और यू हमने यूसुफ को मुल्क (मिस्र) में (जगह देकर) क़ाबिज़ बनाया और ग़रज़ ये थी कि हमने उसे ख्वाब की बातों की ताबीर सिखायी और ख़ुदा तो अपने काम पर (हर तरह के) ग़ालिब व क़ादिर है मगर बहुतेरे लोग (उसको) नहीं जानते
\end{hindi}}
\flushright{\begin{Arabic}
\quranayah[12][22]
\end{Arabic}}
\flushleft{\begin{hindi}
और जब यूसुफ अपनी जवानी को पहुँचे तो हमने उनको हुक्म (नुबूवत) और इल्म अता किया और नेकी कारों को हम यूँ ही बदला दिया करते हैं
\end{hindi}}
\flushright{\begin{Arabic}
\quranayah[12][23]
\end{Arabic}}
\flushleft{\begin{hindi}
और जिस औरत ज़ुलेखा के घर में यूसुफ रहते थे उसने अपने (नाजायज़) मतलब हासिल करने के लिए ख़ुद उनसे आरज़ू की और सब दरवाज़े बन्द कर दिए और (बे ताना) कहने लगी लो आओ यूसुफ ने कहा माज़अल्लाह वह (तुम्हारे मियाँ) मेरा मालिक हैं उन्होंने मुझे अच्छी तरह रखा है मै ऐसा ज़ुल्म क्यों कर सकता हूँ बेशक ऐसा ज़ुल्म करने वाले फलाह नहीं पाते
\end{hindi}}
\flushright{\begin{Arabic}
\quranayah[12][24]
\end{Arabic}}
\flushleft{\begin{hindi}
ज़ुलेखा ने तो उनके साथ (बुरा) इरादा कर ही लिया था और अगर ये भी अपने परवरदिगार की दलीन न देख चुके होते तो क़स्द कर बैठते (हमने उसको यूँ बचाया) ताकि हम उससे बुराई और बदकारी को दूर रखे बेशक वह हमारे ख़ालिस बन्दों में से था
\end{hindi}}
\flushright{\begin{Arabic}
\quranayah[12][25]
\end{Arabic}}
\flushleft{\begin{hindi}
और दोनों दरवाजे क़ी तरफ झपट पड़े और ज़ुलेख़ा (ने पीछे से उनका कुर्ता पकड़ कर खीचा और) फाड़ डाला और दोनों ने ज़ुलेखा के ख़ाविन्द को दरवाज़े के पास खड़ा पाया ज़ुलेख़ा झट (अपने शौहर से) कहने लगी कि जो तुम्हारी बीबी के साथ बदकारी का इरादा करे उसकी सज़ा इसके सिवा और कुछ नहीं कि या तो कैद कर दिया जाए
\end{hindi}}
\flushright{\begin{Arabic}
\quranayah[12][26]
\end{Arabic}}
\flushleft{\begin{hindi}
या दर्दनाक अज़ाब में मुब्तिला कर दिया जाए यूसुफ ने कहा उसने ख़ुद (मुझसे मेरी आरज़ू की थी और ज़ुलेख़ा) के कुन्बे वालों में से एक गवाही देने वाले (दूध पीते बच्चे) ने गवाही दी कि अगर उनका कुर्ता आगे से फटा हुआ हो तो ये सच्ची और वह झूठे
\end{hindi}}
\flushright{\begin{Arabic}
\quranayah[12][27]
\end{Arabic}}
\flushleft{\begin{hindi}
और अगर उनका कुर्ता पींछे से फटा हुआ हो तो ये झूठी और वह सच्चे
\end{hindi}}
\flushright{\begin{Arabic}
\quranayah[12][28]
\end{Arabic}}
\flushleft{\begin{hindi}
फिर जब अज़ीजे मिस्र ने उनका कुर्ता पीछे से फटा हुआ देखा तो (अपनी औरत से) कहने लगा ये तुम ही लोगों के चलत्तर है उसमें शक़ नहीं कि तुम लोगों के चलत्तर बड़े (ग़ज़ब के) होते हैं
\end{hindi}}
\flushright{\begin{Arabic}
\quranayah[12][29]
\end{Arabic}}
\flushleft{\begin{hindi}
(और यूसुफ से कहा) ऐ यूसुफ इसको जाने दो और (औरत से कहा) कि तू अपने गुनाह की माफी माँग क्योंकि बेशक तू ही सरतापा ख़तावार है
\end{hindi}}
\flushright{\begin{Arabic}
\quranayah[12][30]
\end{Arabic}}
\flushleft{\begin{hindi}
और शहर (मिस्र) में औरतें चर्चा करने लगी कि अज़ीज़ (मिस्र) की बीबी अपने ग़ुलाम से (नाजायज़) मतलब हासिल करने की आरज़ू मन्द है बेशक गुलाम ने उसे उलफत में लुभाया है हम लोग तो यक़ीनन उसे सरीही ग़लती में मुब्तिला देखते हैं
\end{hindi}}
\flushright{\begin{Arabic}
\quranayah[12][31]
\end{Arabic}}
\flushleft{\begin{hindi}
तो जब ज़ुलेख़ा ने उनके ताने सुने तो उस ने उन औरतों को बुला भेजा और उनके लिए एक मजलिस आरास्ता की और उसमें से हर एक के हाथ में एक छुरी और एक (नारंगी) दी (और कह दिया कि जब तुम्हारे सामने आए तो काट के एक फ़ाक उसको दे देना) और यूसुफ़ से कहा कि अब इनके सामने से निकल तो जाओ तो जब उन औरतों ने उसे देखा तो उसके बड़ा हसीन पाया तो सब के सब ने (बे खुदी में) अपने अपने हाथ काट डाले और कहने लगी हाय अल्लाह ये आदमी नहीं है ये तो हो न हो बस एक मुअज़िज़ (इज्ज़त वाला) फ़रिश्ता है
\end{hindi}}
\flushright{\begin{Arabic}
\quranayah[12][32]
\end{Arabic}}
\flushleft{\begin{hindi}
(तब ज़ुलेख़ा उन औरतों से) बोली कि बस ये वही तो है जिसकी बदौलत तुम सब मुझे मलामत (बुरा भला) करती थीं और हाँ बेशक मैं उससे अपना मतलब हासिल करने की खुद उससे आरज़ू मन्द थी मगर ये बचा रहा और जिस काम का मैं हुक्म देती हूँ अगर ये न करेगा तो ज़रुर क़ैद भी किया जाएगा और ज़लील भी होगा (ये सब बातें यूसुफ ने मेरी बारगाह में) अर्ज़ की
\end{hindi}}
\flushright{\begin{Arabic}
\quranayah[12][33]
\end{Arabic}}
\flushleft{\begin{hindi}
ऐ मेरे पालने वाले जिस बात की ये औरते मुझ से ख्वाहिश रखती हैं उसकी निस्वत (बदले में) मुझे क़ैद ख़ानों ज्यादा पसन्द है और अगर तू इन औरतों के फ़रेब मुझसे दफा न फरमाएगा तो (शायद) मै उनकी तरफ माएल (झुक) हो जाँऊ ले तो जाओ और जाहिलों से शुमार किया जाऊँ
\end{hindi}}
\flushright{\begin{Arabic}
\quranayah[12][34]
\end{Arabic}}
\flushleft{\begin{hindi}
तो उनके परवरदिगार ने उनकी सुन ली और उन औरतों के मकर को दफा कर दिया इसमें शक़ नहीं कि वह बड़ा सुनने वाला वाक़िफकार है
\end{hindi}}
\flushright{\begin{Arabic}
\quranayah[12][35]
\end{Arabic}}
\flushleft{\begin{hindi}
फिर (अज़ीज़ मिस्र और उसके लोगों ने) बावजूद के (यूसुफ की पाक दामिनी की) निशानियाँ देख ली थी उसके बाद भी उनको यही मुनासिब मालूम हुआ
\end{hindi}}
\flushright{\begin{Arabic}
\quranayah[12][36]
\end{Arabic}}
\flushleft{\begin{hindi}
कि कुछ मियाद के लिए उनको क़ैद ही करे दें और यूसुफ के साथ और भी दो जवान आदमी (क़ैद ख़ाने) में दाख़िल हुए (चन्द दिन के बाद) उनमें से एक ने कहा कि मैने ख्वाब में देखा है कि मै (शराब बनाने के वास्ते अंगूर) निचोड़ रहा हूँ और दूसरे ने कहा (मै ने भी ख्वाब में) अपने को देखा कि मै अपने सर पर रोटिया उठाए हुए हूँ और चिड़ियाँ उसे खा रही हैं (यूसुफ) हमको उसकी ताबीर (मतलब) बताओ क्योंकि हम तुमको यक़ीनन नेकी कारों से समझते हैं
\end{hindi}}
\flushright{\begin{Arabic}
\quranayah[12][37]
\end{Arabic}}
\flushleft{\begin{hindi}
यूसुफ ने कहा जो खाना तुम्हें (क़ैद ख़ाने से) दिया जाता है वह आने भी न पाएगा कि मै उसके तुम्हारे पास आने के क़ब्ल ही तुम्हे उसकी ताबीर बताऊँगा ये ताबीरे ख्वाब भी उन बातों के साथ है जो मेरे परवरदिगार ने मुझे तालीम फरमाई है मैं उन लोगों का मज़हब छोड़ बैठा हूँ जो ख़ुदा पर ईमान नहीं लाते और वह लोग आख़िरत के भी मुन्किर है
\end{hindi}}
\flushright{\begin{Arabic}
\quranayah[12][38]
\end{Arabic}}
\flushleft{\begin{hindi}
और मैं तो अपने बाप दादा इबराहीम व इसहाक़ व याक़ूब के मज़हब पर चलने वाला हूँ मुनासिब नहीं कि हम ख़ुदा के साथ किसी चीज़ को (उसका) शरीक बनाएँ ये भी ख़ुदा की एक बड़ी मेहरबानी है हम पर भी और तमाम लोगों पर मगर बहुतेरे लोग उसका शुक्रिया (भी) अदा नहीं करते
\end{hindi}}
\flushright{\begin{Arabic}
\quranayah[12][39]
\end{Arabic}}
\flushleft{\begin{hindi}
ऐ मेरे कैद ख़ाने के दोनो रफीक़ों (साथियों) (ज़रा ग़ौर तो करो कि) भला जुदा जुदा माबूद अच्छे या ख़ुदाए यकता ज़बरदस्त (अफसोस)
\end{hindi}}
\flushright{\begin{Arabic}
\quranayah[12][40]
\end{Arabic}}
\flushleft{\begin{hindi}
तुम लोग तो ख़ुदा को छोड़कर बस उन चन्द नामों ही को परसतिश करते हो जिन को तुमने और तुम्हारे बाप दादाओं ने गढ़ लिया है ख़ुदा ने उनके लिए कोई दलील नहीं नाज़िल की हुकूमत तो बस ख़ुदा ही के वास्ते ख़ास है उसने तो हुक्म दिया है कि उसके सिवा किसी की इबादत न करो यही साीधा दीन है मगर (अफसोस) बहुतेरे लोग नहीं जानते हैं
\end{hindi}}
\flushright{\begin{Arabic}
\quranayah[12][41]
\end{Arabic}}
\flushleft{\begin{hindi}
ऐ मेरे क़ैद ख़ाने के दोनो रफीक़ो (अच्छा अब ताबीर सुनो तुममें से एक (जिसने अंगूर देखा रिहा होकर) अपने मालिक को शराब पिलाने का काम करेगा और (दूसरा) जिसने रोटियाँ सर पर (देखी हैं) तो सूली दिया जाएगा और चिड़िया उसके सर से (नोच नोच) कर खाएगी जिस अम्र को तुम दोनों दरयाफ्त करते थे (वह ये है और) फैसला हो चुका है
\end{hindi}}
\flushright{\begin{Arabic}
\quranayah[12][42]
\end{Arabic}}
\flushleft{\begin{hindi}
और उन दोनों में से जिसकी निस्बत यूसुफ ने समझा था वह रिहा हो जाएगा उससे कहा कि अपने मालिक के पास मेरा भी तज़किरा करना (कि मैं बेजुर्म क़ैद हूँ) तो शैतान ने उसे अपने आक़ा से ज़िक्र करना भुला दिया तो यूसुफ क़ैद ख़ाने में कई बरस रहे
\end{hindi}}
\flushright{\begin{Arabic}
\quranayah[12][43]
\end{Arabic}}
\flushleft{\begin{hindi}
और (इसी असना (बीच) में) बादशाह ने (भी ख्वाब देखा और) कहा मैने देखा है कि सात मोटी ताज़ी गाए हैं उनको सात दुबली पतली गाय खाए जाती हैं और सात ताज़ी सब्ज़ बालियॉ (देखीं) और फिर (सात) सूखी बालियॉ ऐ (मेरे दरबार के) सरदारों अगर तुम लोगों को ख्वाब की ताबीर देनी आती हो तो मेरे (इस) ख्वाब के बारे में हुक्म लगाओ
\end{hindi}}
\flushright{\begin{Arabic}
\quranayah[12][44]
\end{Arabic}}
\flushleft{\begin{hindi}
उन लोगों ने अर्ज़ की कि ये तो (कुछ) ख्वाब परेशॉ (सा) है और हम लोग ऐसे ख्वाब (परेशॉ) की ताबीर तो नहीं जानते हैं
\end{hindi}}
\flushright{\begin{Arabic}
\quranayah[12][45]
\end{Arabic}}
\flushleft{\begin{hindi}
और जिसने उन दोनों में से रिहाई पाई थी (साकी) और उसको एक ज़माने के बाद (यूसुफ का क़िस्सा) याद आया बोल उठा कि मुझे (क़ैद ख़ाने तक) जाने दीजिए तो मैं उसकी ताबीर बताए देता हूँ
\end{hindi}}
\flushright{\begin{Arabic}
\quranayah[12][46]
\end{Arabic}}
\flushleft{\begin{hindi}
(ग़रज़ वह गया और यूसुफ से कहने लगा) ऐ यूसुफ ऐ बड़े सच्चे (यूसुफ) ज़रा हमें ये तो बताइए कि सात मोटी ताज़ी गायों को सात पतली गाय खाए जाती है और सात बालियॉ हैं हरी कचवा और फिर (सात) सूखी मुरझाई (इसकी ताबीर क्या है) तो मैं लोगों के पास पलट कर जाऊँ (और बयान करुँ)
\end{hindi}}
\flushright{\begin{Arabic}
\quranayah[12][47]
\end{Arabic}}
\flushleft{\begin{hindi}
ताकि उनको भी (तुम्हारी क़दर) मालूम हो जाए यूसुफ ने कहा (इसकी ताबीर ये है) कि तुम लोग लगातार सात बरस काश्तकारी करते रहोगे तो जो (फसल) तुम काटो उस (के दाने) को बालियों में रहने देना (छुड़ाना नहीं) मगर थोड़ा (बहुत) जो तुम खुद खाओ
\end{hindi}}
\flushright{\begin{Arabic}
\quranayah[12][48]
\end{Arabic}}
\flushleft{\begin{hindi}
उसके बाद बड़े सख्त (खुश्क साली (सूखे) के) सात बरस आएंगें कि जो कुछ तुम लोगों ने उन सातों साल के वास्ते पहले जमा कर रखा होगा सब खा जाएंगें मगर बहुत थोड़ा सा जो तुम (बीज के वास्ते) बचा रखोगे
\end{hindi}}
\flushright{\begin{Arabic}
\quranayah[12][49]
\end{Arabic}}
\flushleft{\begin{hindi}
(बस) फिर उसके बाद एक साल आएगा जिसमें लोगों के लिए खूब मेंह बरसेगी (और अंगूर भी खूब फलेगा) और लोग उस साल (उन्हें) शराब के लिए निचोड़ेगें
\end{hindi}}
\flushright{\begin{Arabic}
\quranayah[12][50]
\end{Arabic}}
\flushleft{\begin{hindi}
(ये ताबीर सुनते ही) बादशाह ने हुक्म दिया कि यूसुफ को मेरे हुज़ूर में तो ले आओ फिर जब (शाही) चौबदार (ये हुक्म लेकर) यूसुफ के पास आया तो युसूफ ने कहा कि तुम अपनी सरकार के पास पलट जाओ और उनसे पूछो कि (आप को) कुछ उन औरतों का हाल भी मालूम है जिन्होने (मुझे देख कर) अपने अपने हाथ काट डाले थे कि या मैं उनका तालिब था
\end{hindi}}
\flushright{\begin{Arabic}
\quranayah[12][51]
\end{Arabic}}
\flushleft{\begin{hindi}
या वह (मेरी) इसमें तो शक़ ही नहीं कि मेरा परवरदिगार ही उनके मक्र से खूब वाक़िफ है चुनान्चे बादशाह ने (उन औरतों को तलब किया) और पूछा कि जिस वक्त तुम लोगों ने यूसुफ से अपना मतलब हासिल करने की खुद उन से तमन्ना की थी तो हमें क्या मामला पेश आया था वह सब की सब अर्ज क़रने लगी हाशा अल्लाह हमने यूसुफ में तो किसी तरह की बुराई नहीं देखी (तब) अज़ीज़ मिस्र की बीबी (ज़ुलेख़ा) बोल उठी अब तू ठीक ठीक हाल सब पर ज़ाहिर हो ही गया (असल बात ये है कि) मैने खुद उससे अपना मतलब हासिल करने की तमन्ना की थी और बेशक वह यक़ीनन सच्चा है
\end{hindi}}
\flushright{\begin{Arabic}
\quranayah[12][52]
\end{Arabic}}
\flushleft{\begin{hindi}
(ये वाक़िया चौबदार ने यूसुफ से बयान किया (यूसुफ ने कहा) ये क़िस्से मैने इसलिए छेड़ा) ताकि तुम्हारे बादशाह को मालूम हो जाए कि मैने अज़ीज़ की ग़ैबत में उसकी (अमानत में ख़यानत नहीं की) और ख़ुदा ख़यानत करने वालों की मक्कारी हरगिज़ चलने नहीं देता
\end{hindi}}
\flushright{\begin{Arabic}
\quranayah[12][53]
\end{Arabic}}
\flushleft{\begin{hindi}
और (यूं तो) मै भी अपने नफ्स को गुनाहो से बे लौस नहीं कहता हूँ क्योंकि (मैं भी बशर हूँ और नफ्स बराबर बुराई की तरफ उभारता ही है मगर जिस पर मेरा परवरदिगार रहम फरमाए (और गुनाह से बचाए)
\end{hindi}}
\flushright{\begin{Arabic}
\quranayah[12][54]
\end{Arabic}}
\flushleft{\begin{hindi}
इसमें शक़ नहीं कि मेरा परवरदिगार बड़ा बख्शने वाला मेहरबान है और बादशाह ने हुक्म दिया कि यूसुफ को मेरे पास ले आओ तो मैं उनको अपने ज़ाती काम के लिए ख़ास कर लूंगा फिर उसने यूसुफ से बातें की तो यूसुफ की आला क़ाबलियत साबित हुई (और) उसने हुक्म दिया कि तुम आज (से) हमारे सरकार में यक़ीन बावक़ार (और) मुअतबर हो
\end{hindi}}
\flushright{\begin{Arabic}
\quranayah[12][55]
\end{Arabic}}
\flushleft{\begin{hindi}
यूसुफ ने कहा (जब अपने मेरी क़दर की है तो) मुझे मुल्की ख़ज़ानों पर मुक़र्रर कीजिए क्योंकि मैं (उसका) अमानतदार ख़ज़ान्ची (और) उसके हिसाब व किताब से भी वाक़िफ हूँ
\end{hindi}}
\flushright{\begin{Arabic}
\quranayah[12][56]
\end{Arabic}}
\flushleft{\begin{hindi}
(ग़रज़ यूसुफ शाही ख़ज़ानो के अफसर मुक़र्रर हुए) और हमने यूसुफ को युं मुल्क (मिस्र) पर क़ाबिज़ बना दिया कि उसमें जहाँ चाहें रहें हम जिस पर चाहते हैं अपना फज़ल करते हैं और हमने नेको कारो के अज्र को अकारत नहीं करते
\end{hindi}}
\flushright{\begin{Arabic}
\quranayah[12][57]
\end{Arabic}}
\flushleft{\begin{hindi}
और जो लोग ईमान लाए और परहेज़गारी करते रहे उनके लिए आख़िरत का अज्र उसी से कही बेहतर है
\end{hindi}}
\flushright{\begin{Arabic}
\quranayah[12][58]
\end{Arabic}}
\flushleft{\begin{hindi}
(और चूंकि कनआन में भी कहत (सूखा) था इस वजह से) यूसुफ के (सौतेले भाई ग़ल्ला ख़रीदने को मिस्र में) आए और यूसुफ के पास गए तो उनको फौरन ही पहचान लिया और वह लोग उनको न पहचान सके
\end{hindi}}
\flushright{\begin{Arabic}
\quranayah[12][59]
\end{Arabic}}
\flushleft{\begin{hindi}
और जब यूसुफ ने उनके (ग़ल्ले का) सामान दुरूस्त कर दिया और वह जाने लगे तो यूसुफ़ ने (उनसे कहा) कि (अबकी आना तो) अपने सौतेले भाई को (जिसे घर छोड़ आए हो) मेरे पास लेते आना क्या तुम नहीं देखते कि मै यक़ीनन नाप भी पूरी देता हूं और बहुत अच्छा मेहमान नवाज़ भी हूँ
\end{hindi}}
\flushright{\begin{Arabic}
\quranayah[12][60]
\end{Arabic}}
\flushleft{\begin{hindi}
पस अगर तुम उसको मेरे पास न लाओगे तो तुम्हारे लिए न मेरे पास कुछ न कुछ (ग़ल्ला वग़ैरह) होगा
\end{hindi}}
\flushright{\begin{Arabic}
\quranayah[12][61]
\end{Arabic}}
\flushleft{\begin{hindi}
न तुम लोग मेरे क़रीब ही चढ़ने पाओगे वह लोग कहने लगे हम उसके वालिद से उसके बारे में जाते ही दरख्वास्त करेंगे
\end{hindi}}
\flushright{\begin{Arabic}
\quranayah[12][62]
\end{Arabic}}
\flushleft{\begin{hindi}
और हम ज़रुर इस काम को पूरा करेंगें और यूसुफ ने अपने मुलाज़िमों (नौकरों) को हुक्म दिया कि उनकी (जमा) पूंजी उनके बोरो में (चूपके से) रख दो ताकि जब ये लोग अपने एहलो (अयाल) के पास लौट कर जाएं तो अपनी पूंजी को पहचान ले
\end{hindi}}
\flushright{\begin{Arabic}
\quranayah[12][63]
\end{Arabic}}
\flushleft{\begin{hindi}
(और इस लालच में) यायद फिर पलट के आएं ग़रज़ जब ये लोग अपने वालिद के पास पलट के आए तो सब ने मिलकर अर्ज़ की ऐ अब्बा हमें (आइन्दा) गल्ले मिलने की मुमानिअत (मना) कर दी गई है तो आप हमारे साथ हमारे भाई (बिन यामीन) को भेज दीजिए
\end{hindi}}
\flushright{\begin{Arabic}
\quranayah[12][64]
\end{Arabic}}
\flushleft{\begin{hindi}
ताकि हम (फिर) गल्ला लाए और हम उसकी पूरी हिफाज़त करेगें याक़ूब ने कहा मै उसके बारे में तुम्हारा ऐतबार नहीं करता मगर वैसा ही जैसा कि उससे पहले उसके मांजाए (भाई) के बारे में किया था तो ख़ुद उसका सबसे बेहतर हिफाज़त करने वाला है और वही सब से ज्यादा रहम करने वाला है
\end{hindi}}
\flushright{\begin{Arabic}
\quranayah[12][65]
\end{Arabic}}
\flushleft{\begin{hindi}
और जब उन लोगों ने अपने अपने असबाब खोले तो अपनी अपनी पूंजी को देखा कि (वैसे ही) वापस कर दी गई है तो (अपने बाप से) कहने लगे ऐ अब्बा हमें (और) क्या चाहिए (देखिए) यह हमारी जमा पूंजी हमें वापस दे दी गयी है और (ग़ल्ला मुफ्त मिला अब इब्ने यामीन को जाने दीजिए तो) हम अपने एहलो अयाल के वास्ते ग़ल्ला लादें और अपने भाई की पूरी हिफाज़त करेगें और एक बार यतर ग़ल्ला और बढ़वा लाएंगें
\end{hindi}}
\flushright{\begin{Arabic}
\quranayah[12][66]
\end{Arabic}}
\flushleft{\begin{hindi}
ये जो अबकी दफा लाए थे थोड़ा सा ग़ल्ला है याकूब ने कहा जब तक तुम लोग मेरे सामने खुदा से एहद न कर लोगे कि तुम उसको ज़रुर मुझ तक (सही व सालिम) ले आओगे मगर हाँ जब तुम खुद घिर जाओ तो मजबूरी है वरना मै तुम्हारे साथ हरगिज़ उसको न भेजूंगा फिर जब उन लोगों ने उनके सामने एहद कर लिया तो याक़ूब ने कहा कि हम लोग जो कह रहे हैं ख़ुदा उसका ज़ामिन है
\end{hindi}}
\flushright{\begin{Arabic}
\quranayah[12][67]
\end{Arabic}}
\flushleft{\begin{hindi}
और याक़ूब ने (नसीहतन चलते वक्त बेटो से) कहा ऐ फरज़न्दों (देखो ख़बरदार) सब के सब एक ही दरवाजे से न दाख़िल होना (कि कहीं नज़र न लग जाए) और मुताफरिक़ (अलग अलग) दरवाज़ों से दाख़िल होना और मै तुमसे (उस बात को जो) ख़ुदा की तरफ से (आए) कुछ टाल भी नहीं सकता हुक्म तो (और असली) ख़ुदा ही के वास्ते है मैने उसी पर भरोसा किया है और भरोसा करने वालों को उसी पर भरोसा करना चाहिए
\end{hindi}}
\flushright{\begin{Arabic}
\quranayah[12][68]
\end{Arabic}}
\flushleft{\begin{hindi}
और जब ये सब भाई जिस तरह उनके वालिद ने हुक्म दिया था उसी तरह (मिस्र में) दाख़िल हुए मगर जो हुक्म ख़ुदा की तरफ से आने को था उसे याक़ूब कुछ भी टाल नहीं सकते थे मगर (हाँ) याक़ूब के दिल में एक तमन्ना थी जिसे उन्होंने भी युं पूरा कर लिया क्योंकि इसमे तो शक़ नहीं कि उसे चूंकि हमने तालीम दी थी साहिबे इल्म ज़रुर था मगर बहुतेरे लोग (उससे भी) वाक़िफ नहीं
\end{hindi}}
\flushright{\begin{Arabic}
\quranayah[12][69]
\end{Arabic}}
\flushleft{\begin{hindi}
और जब ये लोग यूसुफ के पास पहुँचे तो यूसुफ ने अपने हक़ीक़ी (सगे) भाई को अपने पास (बग़ल में) जगह दी और (चुपके से) उस (इब्ने यामीन) से कह दिया कि मै तुम्हारा भाई (यूसुफ) हूँ तो जो कुछ (बदसुलूकियाँ) ये लोग तुम्हारे साथ करते रहे हैं उसका रंज न करो
\end{hindi}}
\flushright{\begin{Arabic}
\quranayah[12][70]
\end{Arabic}}
\flushleft{\begin{hindi}
फिर जब यूसुफ ने उन का साज़ो सामान सफर ग़ल्ला (वग़ैरह) दुरुस्त करा दिया तो अपने भाई के असबाब में पानी पीने का कटोरा (यूसुफ के इशारे) से रखवा दिया फिर एक मुनादी ललकार के बोला कि ऐ क़ाफ़िले वालों (हो न हो) यक़ीनन तुम्ही लोग ज़रुर चोर हो
\end{hindi}}
\flushright{\begin{Arabic}
\quranayah[12][71]
\end{Arabic}}
\flushleft{\begin{hindi}
ये सुन कर ये लोग पुकारने वालों की तरफ भिड़ पड़े और कहने लगे (आख़िर) तुम्हारी क्या चीज़ गुम हो गई है
\end{hindi}}
\flushright{\begin{Arabic}
\quranayah[12][72]
\end{Arabic}}
\flushleft{\begin{hindi}
उन लोगों ने जवाब दिया कि हमें बादशाह का प्याला नहीं मिलता है और मै उसका ज़ामिन हूँ कि जो शख़्श उसको ला हाज़िर करेगा उसको एक ऊँट के बोझ बराबर (ग़ल्ला इनाम) मिलेगा
\end{hindi}}
\flushright{\begin{Arabic}
\quranayah[12][73]
\end{Arabic}}
\flushleft{\begin{hindi}
तब ये लोग कहने लगे ख़ुदा की क़सम तुम तो जानते हो कि (तुम्हारे) मुल्क में हम फसाद करने की ग़रज़ से नहीं आए थे और हम लोग तो कुछ चोर तो हैं नहीं
\end{hindi}}
\flushright{\begin{Arabic}
\quranayah[12][74]
\end{Arabic}}
\flushleft{\begin{hindi}
तब वह मुलाज़िमीन बोले कि अगर तुम झूठे निकले तो फिर चोर की क्या सज़ा होगी
\end{hindi}}
\flushright{\begin{Arabic}
\quranayah[12][75]
\end{Arabic}}
\flushleft{\begin{hindi}
(वे धड़क) बोल उठे कि उसकी सज़ा ये है कि जिसके बोरे में वह (माल) निकले तो वही उसका बदला है (तो वह माल के बदले में ग़ुलाम बना लिया जाए)
\end{hindi}}
\flushright{\begin{Arabic}
\quranayah[12][76]
\end{Arabic}}
\flushleft{\begin{hindi}
हम लोग तो (अपने यहाँ) ज़ालिमों (चोरों) को इसी तरह सज़ा दिया करते हैं ग़रज़ यूसुफ ने अपने भाई का थैला खोलने ने से क़ब्ल दूसरे भाइयों के थैलों से (तलाशी) शुरू की उसके बाद (आख़िर में) उस प्याले को यूसुफ ने अपने भाई के थैले से बरामद किया यूसुफ को भाई के रोकने की हमने यू तदबीर बताइ वरना (बादशाह मिस्र) के क़ानून के मुवाफिक़ अपने भाई को रोक नहीं सकते थे मगर हाँ जब ख़ुदा चाहे हम जिसे चाहते हैं उसके दर्जे बुलन्द कर देते हैं और (दुनिया में) हर साहबे इल्म से बढ़कर एक और आलिम है
\end{hindi}}
\flushright{\begin{Arabic}
\quranayah[12][77]
\end{Arabic}}
\flushleft{\begin{hindi}
(ग़रज़) इब्ने यामीन रोक लिए गए तो ये लोग कहने लगे अगर उसने चोरी की तो (कौन ताज्जुब है) इसके पहले इसका भाई (यूसुफ) चोरी कर चुका है तो यूसुफ ने (उसका कुछ जवाब न दिया) उसको अपने दिल में पोशीदा (छुपाये) रखा और उन पर ज़ाहिर न होने दिया मगर ये कह दिया कि तुम लोग बड़े ख़ाना ख़राब (बुरे आदमी) हो
\end{hindi}}
\flushright{\begin{Arabic}
\quranayah[12][78]
\end{Arabic}}
\flushleft{\begin{hindi}
और जो (उसके भाई की चोरी का हाल बयान करते हो ख़ुदा खूब बवाक़िफ है (इस पर) उन लोगों ने कहा ऐ अज़ीज़ उस (इब्ने यामीन) के वालिद बहुत बूढ़े (आदमी) हैं (और इसको बहुत चाहते हैं) तो आप उसके ऐवज़ (बदले) हम में से किसी को ले लीजिए और उसको छोड़ दीजिए
\end{hindi}}
\flushright{\begin{Arabic}
\quranayah[12][79]
\end{Arabic}}
\flushleft{\begin{hindi}
क्योंकि हम आपको बहुत नेको कार बुर्जुग़ समझते हैं यूसुफ ने कहा माज़ अल्लाह (ये क्यों कर हो सकता है कि) हमने जिसकी पास अपनी चीज़ पाई है उसे छोड़कर दूसरे को पकड़ लें (अगर हम ऐसा करें) तो हम ज़रुर बड़े बेइन्साफ ठहरे
\end{hindi}}
\flushright{\begin{Arabic}
\quranayah[12][80]
\end{Arabic}}
\flushleft{\begin{hindi}
फिर जब यूसुफ की तरफ से मायूस हुए तो बाहम मशवरा करने के लिए अलग खड़े हुए तो जो शख़्श उन सब में बड़ा था (यहूदा) कहने लगा (भाइयों) क्या तुम को मालूम नहीं कि तुम्हार वालिद ने तुम लोगों से ख़ुदा का एहद करा लिया था और उससे तुम लोग यूसुफ के बारे में क्या कुछ ग़लती कर ही चुके हो तो (भाई) जब तक मेरे वालिद मुझे इजाज़त (न) दें या खुद ख़ुदा मुझे कोई हुक्म (न) दे मै उस सर ज़मीन से हरगिज़ न हटूंगा और ख़ुदा तो सब हुक्म देने वालो से कहीं बेहतर है
\end{hindi}}
\flushright{\begin{Arabic}
\quranayah[12][81]
\end{Arabic}}
\flushleft{\begin{hindi}
तुम लोग अपने वालिद के पास पलट के जाओ और उनसे जाकर अर्ज़ करो ऐ अब्बा अपके साहबज़ादे ने चोरी की और हम लोगों ने तो अपनी समझ के मुताबिक़ (उसके ले आने का एहद किया था और हम कुछ (अर्ज़) ग़ैबी (आफत) के निगेहबान थे नहीं
\end{hindi}}
\flushright{\begin{Arabic}
\quranayah[12][82]
\end{Arabic}}
\flushleft{\begin{hindi}
और आप इस बस्ती (मिस्र) के लोगों से जिसमें हम लोग थे दरयाफ्त कर लीजिए और इस क़ाफले से भी जिसमें आए हैं (पूछ लीजिए) और हम यक़ीनन बिल्कुल सच्चे हैं
\end{hindi}}
\flushright{\begin{Arabic}
\quranayah[12][83]
\end{Arabic}}
\flushleft{\begin{hindi}
(ग़रज़ जब उन लोगों ने जाकर बयान किया तो) याक़ूब न कहा (उसने चोरी नहीं की) बल्कि ये बात तुमने अपने दिल से गढ़ ली है तो (ख़ैर) सब्र (और ख़ुदा का) शुक्र ख़ुदा से तो (मुझे) उम्मीद है कि मेरे सब (लड़कों) को मेरे पास पहुँचा दे बेशक वह बड़ा वाकिफ़ कार हकीम है
\end{hindi}}
\flushright{\begin{Arabic}
\quranayah[12][84]
\end{Arabic}}
\flushleft{\begin{hindi}
और याक़ूब ने उन लोगों की तरफ से मुँह फेर लिया और (रोकर) कहने लगे हाए अफसोस यूसुफ पर और (इस क़दर रोए कि) उनकी ऑंखें सदमे से सफेद हो गई वह तो बड़े रंज के ज़ाबित (झेलने वाले) थे
\end{hindi}}
\flushright{\begin{Arabic}
\quranayah[12][85]
\end{Arabic}}
\flushleft{\begin{hindi}
(ये देखकर उनके बेटे) कहने लगे कि आप तो हमेशा यूसुफ को याद ही करते रहिएगा यहाँ तक कि बीमार हो जाएगा या जान ही दे दीजिएगा
\end{hindi}}
\flushright{\begin{Arabic}
\quranayah[12][86]
\end{Arabic}}
\flushleft{\begin{hindi}
याक़ूब ने कहा (मै तुमसे कुछ नहीं कहता) मैं तो अपनी बेक़रारी व रंज की शिकायत ख़ुदा ही से करता हूँ और ख़ुदा की तरफ से जो बातें मै जानता हूँ तुम नहीं जानते हो
\end{hindi}}
\flushright{\begin{Arabic}
\quranayah[12][87]
\end{Arabic}}
\flushleft{\begin{hindi}
ऐ मेरी फरज़न्द (एक बार फिर मिस्र) जाओ और यूसुफ और उसके भाई को (जिस तरह बने) ढूँढ के ले आओ और ख़ुदा की रहमत से ना उम्मीद न हो क्योंकि ख़ुदा की रहमत से काफिर लोगो के सिवा और कोई ना उम्मीद नहीं हुआ करता
\end{hindi}}
\flushright{\begin{Arabic}
\quranayah[12][88]
\end{Arabic}}
\flushleft{\begin{hindi}
फिर जब ये लोग यूसुफ के पास गए तो (बहुत गिड़गिड़ाकर) अर्ज़ की कि ऐ अज़ीज़ हमको और हमारे (सारे) कुनबे को कहत की वजह से बड़ी तकलीफ हो रही है और हम कुछ थोड़ी सी पूंजी लेकर आए हैं तो हम को (उसके ऐवज़ पर पूरा ग़ल्ला दिलवा दीजिए और (क़ीमत ही पर नहीं) हम को (अपना) सदक़ा खैरात दीजिए इसमें तो शक़ नहीं कि ख़ुदा सदक़ा ख़ैरात देने वालों को जजाए ख़ैर देता है
\end{hindi}}
\flushright{\begin{Arabic}
\quranayah[12][89]
\end{Arabic}}
\flushleft{\begin{hindi}
(अब तो यूसुफ से न रहा गया) कहा तुम्हें कुछ मालूम है कि जब तुम जाहिल हो रहे थे तो तुम ने यूसुफ और उसके भाई के साथ क्या क्या सुलूक किए
\end{hindi}}
\flushright{\begin{Arabic}
\quranayah[12][90]
\end{Arabic}}
\flushleft{\begin{hindi}
(उस पर वह लोग चौके) और कहने लगे (हाए) क्या तुम ही यूसुफ हो, यूसुफ ने कहा हाँ मै ही यूसुफ हूँ और यह मेरा भाई है बेशक ख़ुदा ने मुझ पर अपना फज़ल व (करम) किया है क्या इसमें शक़ नहीं कि जो शख़्श (उससे) डरता है (और मुसीबत में) सब्र करे तो ख़ुदा हरगिज़ (ऐसे नेको कारों का) अज्र बरबाद नहीं करता
\end{hindi}}
\flushright{\begin{Arabic}
\quranayah[12][91]
\end{Arabic}}
\flushleft{\begin{hindi}
वह लोग कहने लगे ख़ुदा की क़सम तुम्हें ख़ुदा ने यक़ीनन हम पर फज़ीलत दी है और बेशक हम ही यक़ीनन (अज़सरतापा) ख़तावार थे
\end{hindi}}
\flushright{\begin{Arabic}
\quranayah[12][92]
\end{Arabic}}
\flushleft{\begin{hindi}
यूसुफ ने कहा अब आज से तुम पर कुछ इल्ज़ाम नहीं ख़ुदा तुम्हारे गुनाह माफ फरमाए वह तो सबसे ज्यादा रहीम है ये मेरा कुर्ता ले जाओ
\end{hindi}}
\flushright{\begin{Arabic}
\quranayah[12][93]
\end{Arabic}}
\flushleft{\begin{hindi}
और उसको अब्बा जान के चेहरे पर डाल देना कि वह फिर बीना हो जाएंगें (देखने लगेंगे) और तुम लोग अपने सब लड़के बालों को लेकर मेरे पास चले आओ
\end{hindi}}
\flushright{\begin{Arabic}
\quranayah[12][94]
\end{Arabic}}
\flushleft{\begin{hindi}
और जो ही ये काफ़िला मिस्र से चला था कि उन लोगों के वालिद (याक़ूब) ने कहा दिया था कि अगर मुझे सठिया या हुआ न कहो तो बात कहूँ कि मुझे यूसुफ की बू मालूम हो रही है
\end{hindi}}
\flushright{\begin{Arabic}
\quranayah[12][95]
\end{Arabic}}
\flushleft{\begin{hindi}
वह लोग कुनबे वाले (पोते वग़ैराह) कहने लगे आप यक़ीनन अपने पुराने ख़याल (मोहब्बत) में (पड़े हुए) हैं
\end{hindi}}
\flushright{\begin{Arabic}
\quranayah[12][96]
\end{Arabic}}
\flushleft{\begin{hindi}
फिर (यूसुफ की) खुशखबरी देने वाला आया और उनके कुर्ते को उनके चेहरे पर डाल दिया तो याक़ूब फौरन फिर दोबारा ऑंख वाले हो गए (तब याक़ूब ने बेटों से) कहा क्यों मै तुमसे न कहता था जो बातें खुदा की तरफ से मै जानता हूँ तुम नहीं जानते
\end{hindi}}
\flushright{\begin{Arabic}
\quranayah[12][97]
\end{Arabic}}
\flushleft{\begin{hindi}
उन लोगों ने अर्ज़ की ऐ अब्बा हमारे गुनाहों की मग़फिरत की (ख़ुदा की बारगाह में) हमारे वास्ते दुआ मॉगिए हम बेशक अज़सरतापा गुनेहगार हैं
\end{hindi}}
\flushright{\begin{Arabic}
\quranayah[12][98]
\end{Arabic}}
\flushleft{\begin{hindi}
याक़ूब ने कहा मै बहुत जल्द अपने परवरदिगार से तुम्हारी मग़फिरत की दुआ करुगाँ बेशक वह बड़ा बख्शने वाला मेहरबान है
\end{hindi}}
\flushright{\begin{Arabic}
\quranayah[12][99]
\end{Arabic}}
\flushleft{\begin{hindi}
(ग़रज़) जब फिर ये लोग (मय याकूब के) चले और यूसुफ शहर के बाहर लेने आए तो जब ये लोग यूसुफ के पास पहुँचे तो यूसुफ ने अपने माँ बाप को अपने पास जगह दी और (उनसे) कहा कि अब इन्शा अल्लाह बड़े इत्मिनान से मिस्र में चलिए
\end{hindi}}
\flushright{\begin{Arabic}
\quranayah[12][100]
\end{Arabic}}
\flushleft{\begin{hindi}
(ग़रज़) पहुँचकर यूसुफ ने अपने माँ बाप को तख्त पर बिठाया और सब के सब यूसुफ की ताज़ीम के वास्ते उनके सामने सजदे में गिर पड़े (उस वक्त) यूसुफ ने कहा ऐ अब्बा ये ताबीर है मेरे उस पहले ख्वाब की कि मेरे परवरदिगार ने उसे सच कर दिखाया बेशक उसने मेरे साथ एहसान किया जब उसने मुझे क़ैद ख़ाने से निकाला और बावजूद कि मुझ में और मेरे भाईयों में शैतान ने फसाद डाल दिया था उसके बाद भी आप लोगों को गाँव से (शहर में) ले आया (और मुझसे मिला दिया) बेशक मेरा परवरदिगार जो कुछ करता है उसकी तद्बीर खूब जानता है बेशक वह बड़ा वाकिफकार हकीम है
\end{hindi}}
\flushright{\begin{Arabic}
\quranayah[12][101]
\end{Arabic}}
\flushleft{\begin{hindi}
(उसके बाद यूसुफ ने दुआ की ऐ परवरदिगार तूने मुझे मुल्क भी अता फरमाया और मुझे ख्वाब की बातों की ताबीर भी सिखाई ऐ आसमान और ज़मीन के पैदा करने वाले तू ही मेरा मालिक सरपरस्त है दुनिया में भी और आख़िरत में भी तू मुझे (दुनिया से) मुसलमान उठाये और मुझे नेको कारों में शामिल फरमा
\end{hindi}}
\flushright{\begin{Arabic}
\quranayah[12][102]
\end{Arabic}}
\flushleft{\begin{hindi}
(ऐ रसूल) ये किस्सा ग़ैब की ख़बरों में से है जिसे हम तुम्हारे पास वही के ज़रिए भेजते हैं (और तुम्हें मालूम होता है वरना जिस वक्त यूसुफ के भाई बाहम अपने काम का मशवरा कर रहे थे और (हलाक की) तदबीरे कर रहे थे
\end{hindi}}
\flushright{\begin{Arabic}
\quranayah[12][103]
\end{Arabic}}
\flushleft{\begin{hindi}
तुम उनके पास मौजूद न थे और कितने ही चाहो मगर बहुतेरे लोग ईमान लाने वाले नहीं हैं
\end{hindi}}
\flushright{\begin{Arabic}
\quranayah[12][104]
\end{Arabic}}
\flushleft{\begin{hindi}
हालॉकि तुम उनसे (तबलीगे रिसालत का) कोई सिला नहीं मॉगते और ये (क़ुरान) तो सारे जहाँन के वास्ते नसीहत (ही नसीहत) है
\end{hindi}}
\flushright{\begin{Arabic}
\quranayah[12][105]
\end{Arabic}}
\flushleft{\begin{hindi}
और आसमानों और ज़मीन में (ख़ुदा की क़ुदरत की) कितनी निशानियाँ हैं जिन पर ये लोग (दिन रात) ग़ुज़ारा करते हैं और उससे मुँह फेरे रहते हैं
\end{hindi}}
\flushright{\begin{Arabic}
\quranayah[12][106]
\end{Arabic}}
\flushleft{\begin{hindi}
और अक्सर लोगों की ये हालत है कि वह ख़ुदा पर ईमान तो नहीं लाते मगर शिर्क किए जाते हैं
\end{hindi}}
\flushright{\begin{Arabic}
\quranayah[12][107]
\end{Arabic}}
\flushleft{\begin{hindi}
तो क्या ये लोग इस बात से मुतमइन हो बैठें हैं कि उन पर ख़ुदा का अज़ाब आ पड़े जो उन पर छा जाए या उन पर अचानक क़यामत ही आ जाए और उनको कुछ ख़बर भी न हो
\end{hindi}}
\flushright{\begin{Arabic}
\quranayah[12][108]
\end{Arabic}}
\flushleft{\begin{hindi}
(ऐ रसूल) उन से कह दो कि मेरा तरीका तो ये है कि मै (लोगों) को ख़ुदा की तरफ बुलाता हूँ मैं और मेरा पैरव (पीछे चलने वाले) (दोनों) मज़बूत दलील पर हैं और ख़ुदा (हर ऐब व नुक़स से) पाक व पाकीज़ा है और मै मुशरेकीन से नहीं हूँ
\end{hindi}}
\flushright{\begin{Arabic}
\quranayah[12][109]
\end{Arabic}}
\flushleft{\begin{hindi}
और (ऐ रसूल) तुमसे पहले भी हम गाँव ही के रहने वाले कुछ मर्दों को (पैग़म्बर बनाकर) भेजा किए है कि हम उन पर वही नाज़िल करते थे तो क्या ये लोग रुए ज़मीन पर चले फिरे नहीं कि ग़ौर करते कि जो लोग उनसे पहले हो गुज़रे हैं उनका अन्जाम क्या हुआ और जिन लोगों ने परहेज़गारी एख्तेयार की उनके लिए आख़िरत का घर (दुनिया से) यक़ीनन कहीं ज्यादा बेहतर है क्या ये लोग नहीं समझते
\end{hindi}}
\flushright{\begin{Arabic}
\quranayah[12][110]
\end{Arabic}}
\flushleft{\begin{hindi}
पहले के पैग़म्बरो ने तबलीग़े रिसालत यहाँ वक कि जब (क़ौम के ईमान लाने से) पैग़म्बर मायूस हो गए और उन लोगों ने समझ लिया कि वह झुठलाए गए तो उनके पास हमारी (ख़ास) मदद आ पहुँची तो जिसे हमने चाहा नजात दी और हमारा अज़ाब गुनेहगार लोगों के सर से तो टाला नहीं जाता
\end{hindi}}
\flushright{\begin{Arabic}
\quranayah[12][111]
\end{Arabic}}
\flushleft{\begin{hindi}
इसमें शक़ नहीं कि उन लोगों के किस्सों में अक़लमन्दों के वास्ते (अच्छी ख़ासी) इबरत (व नसीहत) है ये (क़ुरान) कोई ऐसी बात नहीं है जो (ख्वाहामा ख्वाह) गढ़ ली जाए बल्कि (जो आसमानी किताबें) इसके पहले से मौजूद हैं उनकी तसदीक़ है और हर चीज़ की तफसील और ईमानदारों के वास्ते (अज़सरतापा) हिदायत व रहमत है
\end{hindi}}
\chapter{Ar-Ra'd (The Thunder)}
\begin{Arabic}
\Huge{\centerline{\basmalah}}\end{Arabic}
\flushright{\begin{Arabic}
\quranayah[13][1]
\end{Arabic}}
\flushleft{\begin{hindi}
अलिफ़ लाम मीम रा ये किताब (क़ुरान) की आयतें है और तुम्हारे परवरदिगार की तरफ से जो कुछ तुम्हारे पास नाज़िल किया गया है बिल्कुल ठीक है मगर बहुतेरे लोग ईमान नहीं लाते
\end{hindi}}
\flushright{\begin{Arabic}
\quranayah[13][2]
\end{Arabic}}
\flushleft{\begin{hindi}
ख़ुदा वही तो है जिसने आसमानों को जिन्हें तुम देखते हो बग़ैर सुतून (खम्बों) के उठाकर खड़ा कर दिया फिर अर्श (के बनाने) पर आमादा हुआ और सूरज और चाँद को (अपना) ताबेदार बनाया कि हर एक वक्त मुक़र्ररा तक चला करते है वही (दुनिया के) हर एक काम का इन्तेज़ाम करता है और इसी ग़रज़ से कि तुम लोग अपने परवरदिगार के सामने हाज़िर होने का यक़ीन करो
\end{hindi}}
\flushright{\begin{Arabic}
\quranayah[13][3]
\end{Arabic}}
\flushleft{\begin{hindi}
(अपनी) आयतें तफसीलदार बयान करता है और वह वही है जिसने ज़मीन को बिछाया और उसमें (बड़े बड़े) अटल पहाड़ और दरिया बनाए और उसने हर तरह के मेवों की दो दो किस्में पैदा की (जैसे खट्टे मीठे) वही रात (के परदे) से दिन को ढाक देता है इसमें शक़ नहीं कि जो लोग और ग़ौर व फिक्र करते हैं उनके लिए इसमें (कुदरत खुदा की) बहुतेरी निशानियाँ हैं
\end{hindi}}
\flushright{\begin{Arabic}
\quranayah[13][4]
\end{Arabic}}
\flushleft{\begin{hindi}
और खुरमों (खजूर) के दरख्त की एक जड़ और दो याखें और बाज़ अकेला (एक ही याख़ का) हालॉकि सब एक ही पानी से सीचे जाते हैं और फलों में बाज़ को बाज़ पर हम तरजीह देते हैं बेशक जो लोग अक़ल वाले हैं उनके लिए इसमें (कुदरत खुदा की) बहुतेरी निशानियाँ हैं
\end{hindi}}
\flushright{\begin{Arabic}
\quranayah[13][5]
\end{Arabic}}
\flushleft{\begin{hindi}
और अगर तुम्हें (किसी बात पर) ताज्जुब होता है तो उन कुफ्फारों को ये क़ौल ताज्जुब की बात है कि जब हम (सड़गल कर) मिट्टी हो जाएंगें तो क्या हम (फिर दोबारा) एक नई जहन्नुम में आंऎंगें ये वही लोग हैं जिन्होंने अपने परवरदिगार के साथ कुफ्र किया और यही वह लोग हैं जिनकी गर्दनों में (क़यामत के दिन) तौक़ पड़े होगें और यही लोग जहन्नुमी हैं कि ये इसमें हमेशा रहेगें
\end{hindi}}
\flushright{\begin{Arabic}
\quranayah[13][6]
\end{Arabic}}
\flushleft{\begin{hindi}
और (ऐ रसूल) ये लोग तुम से भलाई के क़ब्ल ही बुराई (अज़ाब) की जल्दी मचा रहे हैं हालॉकि उनके पहले (बहुत से लोगों की) सज़ाएं हो चुकी हैं और इसमें शक़ नहीं की तुम्हारा परवरदिगार बावजूद उनकी शरारत के लोगों पर बड़ा बख़शिश (करम) वाला है और इसमें भी शक़ नहीं कि तुम्हारा परवरदिगार यक़ीनन सख्त अज़ाब वाला है
\end{hindi}}
\flushright{\begin{Arabic}
\quranayah[13][7]
\end{Arabic}}
\flushleft{\begin{hindi}
और वो लोग काफिर हैं कहते हैं कि इस शख़्श (मोहम्मद) पर उसके परवरदिगार की तरफ से कोई निशानी (हमारी मर्ज़ी के मुताबिक़) क्यों नहीं नाज़िल की जाती ऐ रसूल तुम तो सिर्फ (ख़ौफे ख़ुदा से) डराने वाले हो
\end{hindi}}
\flushright{\begin{Arabic}
\quranayah[13][8]
\end{Arabic}}
\flushleft{\begin{hindi}
और हर क़ौम के लिए एक हिदायत करने वाला है हर मादा जो कि पेट में लिए हुए है और उसको ख़ुदा ही जानता है व बच्चा दानियों का घटना बढ़ना (भी वही जानता है) और हर चीज़ उसके नज़दीक़ एक अन्दाजे से है
\end{hindi}}
\flushright{\begin{Arabic}
\quranayah[13][9]
\end{Arabic}}
\flushleft{\begin{hindi}
(वही) बातिन (छुपे हुवे) व ज़ाहिर का जानने वाला (सब से) बड़ा और आलीशान है
\end{hindi}}
\flushright{\begin{Arabic}
\quranayah[13][10]
\end{Arabic}}
\flushleft{\begin{hindi}
तुम लोगों में जो कोई चुपके से बात कहे और जो शख़्श ज़ोर से पुकार के बोले और जो शख़्श रात की तारीक़ी (अंधेरे) में छुपा बैठा हो और जो शख़्श दिन दहाडें चला जा रहा हो
\end{hindi}}
\flushright{\begin{Arabic}
\quranayah[13][11]
\end{Arabic}}
\flushleft{\begin{hindi}
(उसके नज़दीक) सब बराबर हैं (आदमी किसी हालत में हो मगर) उस अकेले के लिए उसके आगे उसके पीछे उसके निगेहबान (फरिश्ते) मुक़र्रर हैं कि उसको हुक्म ख़ुदा से हिफाज़त करते हैं जो (नेअमत) किसी क़ौम को हासिल हो बेशक वह लोग खुद अपनी नफ्सानी हालत में तग्य्युर न डालें ख़ुदा हरगिज़ तग्य्युर नहीं डाला करता और जब ख़ुदा किसी क़ौम पर बुराई का इरादा करता है तो फिर उसका कोई टालने वाला नहीं और न उसका उसके सिवा कोई वाली और (सरपरस्त) है
\end{hindi}}
\flushright{\begin{Arabic}
\quranayah[13][12]
\end{Arabic}}
\flushleft{\begin{hindi}
वह वही तो है जो तुम्हें डराने और लालच देने के वास्ते बिजली की चमक दिखाता है और पानी से भरे बोझल बादलों को पैदा करता है
\end{hindi}}
\flushright{\begin{Arabic}
\quranayah[13][13]
\end{Arabic}}
\flushleft{\begin{hindi}
और ग़र्ज और फरिश्ते उसके ख़ौफ से उसकी हम्दो सना की तस्बीह किया करते हैं वही (आसमान से) बिजलियों को भेजता है फिर उसे जिस पर चाहता है गिरा भी देता है और ये लोग ख़ुदा के बारे में (ख्वामाख्वाह) झगड़े करते हैं हालॉकि वह बड़ा सख्त क़ूवत वाला है
\end{hindi}}
\flushright{\begin{Arabic}
\quranayah[13][14]
\end{Arabic}}
\flushleft{\begin{hindi}
(मुसीबत के वक्त) उसी का (पुकारना) ठीक पुकारना है और जो लोग उसे छोड़कर (दूसरों को) पुकारते हैं वह तो उनकी कुछ सुनते तक नहीं मगर जिस तरह कोई शख़्श (बग़ैर उंगलियां मिलाए) अपनी दोनों हथेलियाँ पानी की तरफ फैलाए ताकि पानी उसके मुँह में पहुँच जाए हालॉकि वह किसी तरह पहुँचने वाला नहीं और (इसी तरह) काफिरों की दुआ गुमराही में (पड़ी बहकी फिरा करती है)
\end{hindi}}
\flushright{\begin{Arabic}
\quranayah[13][15]
\end{Arabic}}
\flushleft{\begin{hindi}
और आसमानों और ज़मीन में (मख़लूक़ात से) जो कोई भी है खुशी से या ज़बरदस्ती सब (अल्लाह के आगे सर बसजूद हैं और (इसी तरह) उनके साए भी सुबह व शाम (सजदा करते हैं) (15) (सजदा)
\end{hindi}}
\flushright{\begin{Arabic}
\quranayah[13][16]
\end{Arabic}}
\flushleft{\begin{hindi}
(ऐ रसूल) तुम पूछो कि (आख़िर) आसमान और ज़मीन का परवरदिगार कौन है (ये क्या जवाब देगें) तुम कह दो कि अल्लाह है (ये भी कह दो कि क्या तुमने उसके सिवा दूसरे कारसाज़ बना रखे हैं जो अपने लिए आप न तो नफे पर क़ाबू रखते हैं न ज़रर (नुकसान) पर (ये भी तो) पूछो कि भला (कहीं) अन्धा और ऑंखों वाला बराबर हो सकता है (हरगिज़ नहीं) (या कहीं) अंधेरा और उजाला बराबर हो सकता है (हरगिज़ नहीं) इन लोगों ने ख़ुदा के कुछ शरीक़ ठहरा रखे हैं क्या उन्होनें ख़ुदा ही की सी मख़लूक़ पैदा कर रखी है जिनके सबब मख़लूकात उन पर मुशतबा हो गई है (और उनकी खुदाई के क़ायल हो गए) तुम कह दो कि ख़ुदा ही हर चीज़ का पैदा करने वाला और वही यकता और सिपर (सब पर) ग़ालिब है
\end{hindi}}
\flushright{\begin{Arabic}
\quranayah[13][17]
\end{Arabic}}
\flushleft{\begin{hindi}
उसी ने आसमान से पानी बरसाया फिर अपने अपने अन्दाज़े से नाले बह निकले फिर पानी के रेले पर (जोश खाकर) फूला हुआ झाग (फेन) आ गया और उस चीज़ (धातु) से भी जिसे ये लोग ज़ेवर या कोई असबाब बनाने की ग़रज़ से आग में तपाते हैं इसी तरह फेन आ जाता है (फिर अलग हो जाता है) युं ख़ुदा हक़ व बातिल की मसले बयान फरमाता है (कि पानी हक़ की मिसाल और फेन बातिल की) ग़रज़ फेन तो खुश्क होकर ग़ायब हो जाता है जिससे लोगों को नफा पहुँचता है (पानी) वह ज़मीन में ठहरा रहता है युं ख़ुदा (लोगों के समझाने के वास्ते) मसले बयान फरमाता है
\end{hindi}}
\flushright{\begin{Arabic}
\quranayah[13][18]
\end{Arabic}}
\flushleft{\begin{hindi}
जिन लोगों ने अपने परवरदिगार का कहना माना उनके लिए बहुत बेहतरी है और जिन लोगों ने उसका कहा न माना (क़यामत में उनकी ये हालत होगी) कि अगर उन्हें रुए ज़मीन के सब ख़ज़ाने बल्कि उसके साथ इतना और मिल जाए तो ये लोग अपनी नजात के बदले उसको (ये खुशी) दे डालें (मगर फिर भी कोई फायदा नहीं) यही लोग हैं जिनसे बुरी तरह हिसाब लिया जाएगा और आख़िर उन का ठिकाना जहन्नुम है और वह क्या बुरी जगह है
\end{hindi}}
\flushright{\begin{Arabic}
\quranayah[13][19]
\end{Arabic}}
\flushleft{\begin{hindi}
(ऐ रसूल) भला वह शख़्श जो ये जानता है कि जो कुछ तुम्हारे परवरदिगार की तरफ से तुम पर नाज़िल हुआ है बिल्कुल ठीक है कभी उस शख़्श के बराबर हो सकता है जो मुत्तलिक़ (पूरा) अंधा है (हरगिज़ नहीं)
\end{hindi}}
\flushright{\begin{Arabic}
\quranayah[13][20]
\end{Arabic}}
\flushleft{\begin{hindi}
इससे तो बस कुछ समझदार लोग ही नसीहत हासिल करते हैं वह लोग है कि ख़ुदा से जो एहद किया उसे पूरा करते हैं और अपने पैमान को नहीं तोड़ते
\end{hindi}}
\flushright{\begin{Arabic}
\quranayah[13][21]
\end{Arabic}}
\flushleft{\begin{hindi}
(ये) वह लोग हैं कि जिन (ताल्लुक़ात) के क़ायम रखने का ख़ुदा ने हुक्म दिया उन्हें क़ायम रखते हैं और अपने परवरदिगार से डरते हैं और (क़यामत के दिन) बुरी तरह हिसाब लिए जाने से ख़ौफ खाते हैं
\end{hindi}}
\flushright{\begin{Arabic}
\quranayah[13][22]
\end{Arabic}}
\flushleft{\begin{hindi}
और (ये) वह लोग हैं जो अपने परवरदिगार की खुशनूदी हासिल करने की ग़रज़ से (जो मुसीबत उन पर पड़ी है) झेल गए और पाबन्दी से नमाज़ अदा की और जो कुछ हमने उन्हें रोज़ी दी थी उसमें से छिपाकर और खुल कर ख़ुदा की राह में खर्च किया और ये लोग बुराई को भी भलाई स दफा करते हैं -यही लोग हैं जिनके लिए आख़िरत की खूबी मख़सूस है
\end{hindi}}
\flushright{\begin{Arabic}
\quranayah[13][23]
\end{Arabic}}
\flushleft{\begin{hindi}
(यानि) हमेशा रहने के बाग़ जिनमें वह आप जाएॅगे और उनके बाप, दादाओं, बीवियों और उनकी औलाद में से जो लोग नेको कार है (वह सब भी) और फरिश्ते बेहश्त के हर दरवाजे से उनके पास आएगें
\end{hindi}}
\flushright{\begin{Arabic}
\quranayah[13][24]
\end{Arabic}}
\flushleft{\begin{hindi}
और सलाम अलैकुम (के बाद कहेगें) कि (दुनिया में) तुमने सब्र किया (ये उसी का सिला है देखो) तो आख़िरत का घर कैसा अच्छा है
\end{hindi}}
\flushright{\begin{Arabic}
\quranayah[13][25]
\end{Arabic}}
\flushleft{\begin{hindi}
और जो लोग ख़ुदा से एहद व पैमान को पक्का करने के बाद तोड़ डालते हैं और जिन (तालुकात बाहमी) के क़ायम रखने का ख़ुदा ने हुक्म दिया है उन्हें क़तआ (तोड़ते) करते हैं और रुए ज़मीन पर फ़साद फैलाते फिरते हैं ऐसे ही लोग हैं जिनके लिए लानत है और ऐसे ही लोगों के वास्ते बड़ा घर (जहन्नुम) है
\end{hindi}}
\flushright{\begin{Arabic}
\quranayah[13][26]
\end{Arabic}}
\flushleft{\begin{hindi}
और ख़ुदा ही जिसके लिए चाहता है रोज़ी को बढ़ा देता है और जिसके लिए चाहता है तंग करता है और ये लोग दुनिया की (चन्द रोज़ा) ज़िन्दगी पर बहुत निहाल हैं हालॉकि दुनियावी ज़िन्दगी (नईम) आख़िरत के मुक़ाबिल में बिल्कुल बेहकीक़त चीज़ है
\end{hindi}}
\flushright{\begin{Arabic}
\quranayah[13][27]
\end{Arabic}}
\flushleft{\begin{hindi}
और जिन लोगों ने कुफ्र एख़तियार किया वह कहते हैं कि उस (शख्स यानि तुम) पर हमारी ख्वाहिश के मुवाफिक़ कोई मौजिज़ा उसके परवरदिगार की तरफ से क्यों नहीं नाज़िल होता तुम उनसे कह दो कि इसमें शक़ नहीं कि ख़ुदा जिसे चाहता है गुमराही में छोड़ देता है
\end{hindi}}
\flushright{\begin{Arabic}
\quranayah[13][28]
\end{Arabic}}
\flushleft{\begin{hindi}
और जिसने उसकी तरफ रुज़ू की उसे अपनी तरफ पहुँचने की राह दिखाता है (ये) वह लोग हैं जिन्होंने ईमान कुबूल किया और उनके दिलों को ख़ुदा की चाह से तसल्ली हुआ करती है
\end{hindi}}
\flushright{\begin{Arabic}
\quranayah[13][29]
\end{Arabic}}
\flushleft{\begin{hindi}
जिन लोगों ने ईमान क़ुबूल किया और अच्छे अच्छे काम किए उनके वास्ते (बेहश्त में) तूबा (दरख्त) और ख़ुशहाली और अच्छा अन्जाम है
\end{hindi}}
\flushright{\begin{Arabic}
\quranayah[13][30]
\end{Arabic}}
\flushleft{\begin{hindi}
(ऐ रसूल जिस तरह हमने और पैग़म्बर भेजे थे) उसी तरह हमने तुमको उस उम्मत में भेजा है जिससे पहले और भी बहुत सी उम्मते गुज़र चुकी हैं -ताकि तुम उनके सामने जो क़ुरान हमने वही के ज़रिए से तुम्हारे पास भेजा है उन्हें पढ़ कर सुना दो और ये लोग (कुछ तुम्हारे ही नहीं बल्कि सिरे से) ख़ुदा ही के मुन्किर हैं तुम कह दो कि वही मेरा परवरदिगार है उसके सिवा कोई माबूद नहीं मै उसी पर भरोसा रखता हूँ और उसी तरफ रुज़ू करता हूँ
\end{hindi}}
\flushright{\begin{Arabic}
\quranayah[13][31]
\end{Arabic}}
\flushleft{\begin{hindi}
और अगर कोई ऐसा क़ुरान (भी नाज़िल होता) जिसकी बरकत से पहाड़ (अपनी जगह) चल खड़े होते या उसकी वजह से ज़मीन (की मुसाफ़त (दूरी)) तय की जाती और उसकी बरकत से मुर्दे बोल उठते (तो भी ये लोग मानने वाले न थे) बल्कि सच यूँ है कि सब काम का एख्तेयार ख़ुदा ही को है तो क्या अभी तक ईमानदारों को चैन नहीं आया कि अगर ख़ुदा चाहता तो सब लोगों की हिदायत कर देता और जिन लोगों ने कुफ्र एख्तेयार किया उन पर उनकी करतूत की सज़ा में कोई (न कोई) मुसीबत पड़ती ही रहेगी या (उन पर पड़ी) तो उनके घरों के आस पास (ग़रज़) नाज़िल होगी (ज़रुर) यहाँ तक कि ख़ुदा का वायदा (फतेह मक्का) पूरा हो कर रहे और इसमें शक़ नहीं कि ख़ुदा हरगिज़ ख़िलाफ़े वायदा नहीं करता
\end{hindi}}
\flushright{\begin{Arabic}
\quranayah[13][32]
\end{Arabic}}
\flushleft{\begin{hindi}
और (ऐ रसूल) तुमसे पहले भी बहुतेरे पैग़म्बरों की हॅसी उड़ाई जा चुकी है तो मैने (चन्द रोज़) काफिरों को मोहलत दी फिर (आख़िर कार) हमने उन्हें ले डाला फिर (तू क्या पूछता है कि) हमारा अज़ाब कैसा था
\end{hindi}}
\flushright{\begin{Arabic}
\quranayah[13][33]
\end{Arabic}}
\flushleft{\begin{hindi}
क्या जो (ख़ुदा) हर एक शख़्श के आमाल की ख़बर रखता है (उनको युं ही छोड़ देगा हरगिज़ नहीं) और उन लोगों ने ख़ुदा के (दूसरे दूसरे) शरीक ठहराए (ऐ रसूल तुम उनसे कह दो कि तुम आख़िर उनके नाम तो बताओं या तुम ख़ुदा को ऐसे शरीक़ो की ख़बर देते हो जिनको वह जानता तक नहीं कि वह ज़मीन में (किधर बसते) हैं या (निरी ऊपर से बातें बनाते हैं बल्कि (असल ये है कि) काफिरों को उनकी मक्कारियाँ भली दिखाई गई है और वह (गोया) राहे रास्त से रोक दिए गए हैं और जिस शख़्श को ख़ुदा गुमराही में छोड़ दे तो उसका कोई हिदायत करने वाला नहीं
\end{hindi}}
\flushright{\begin{Arabic}
\quranayah[13][34]
\end{Arabic}}
\flushleft{\begin{hindi}
इन लोगों के वास्ते दुनियावी ज़िन्दगी में (भी) अज़ाब है और आख़िरत का अज़ाब तो यक़ीनी और बहुत सख्त खुलने वाला है (और) (फिर) ख़ुदा (के ग़ज़ब) से उनको कोई बचाने वाला (भी) नहीं
\end{hindi}}
\flushright{\begin{Arabic}
\quranayah[13][35]
\end{Arabic}}
\flushleft{\begin{hindi}
जिस बाग़ (बेहश्त) का परहेज़गारों से वायदा किया गया है उसकी सिफत ये है कि उसके नीचे नहरें जारी होगी उसके मेवे सदाबहार और ऐसे ही उसकी छॉव भी ये अन्जाम है उन लोगों को जो (दुनिया में) परहेज़गार थे और काफिरों का अन्जाम (जहन्नुम की) आग है
\end{hindi}}
\flushright{\begin{Arabic}
\quranayah[13][36]
\end{Arabic}}
\flushleft{\begin{hindi}
और (ए रसूल) जिन लोगों को हमने किताब दी है वह तो जो (एहकाम) तुम्हारे पास नाज़िल किए गए हैं सब ही से खुश होते हैं और बाज़ फिरके उसकी बातों से इन्कार करते हैं तुम (उनसे) कह दो कि (तुम मानो या न मानो) मुझे तो ये हुक्म दिया गया है कि मै ख़ुदा ही की इबादत करु और किसी को उसका शरीक न बनाऊ मै (सब को) उसी की तरफ बुलाता हूँ और हर शख़्श को हिर फिर कर उसकी तरफ जाना है
\end{hindi}}
\flushright{\begin{Arabic}
\quranayah[13][37]
\end{Arabic}}
\flushleft{\begin{hindi}
और यूँ हमने उस क़ुरान को अरबी (ज़बान) का फरमान नाज़िल फरमाया और (ऐ रसूल) अगर कहीं तुमने इसके बाद को तुम्हारे पास इल्म (क़ुरान) आ चुका उन की नफसियानी ख्वाहिशों की पैरवी कर ली तो (याद रखो कि) फिर ख़ुदा की तरफ से न कोई तुम्हारा सरपरस्त होगा न कोई बचाने वाला
\end{hindi}}
\flushright{\begin{Arabic}
\quranayah[13][38]
\end{Arabic}}
\flushleft{\begin{hindi}
और हमने तुमसे पहले और (भी) बहुतेरे पैग़म्बर भेजे और हमने उनको बीवियाँ भी दी और औलाद (भी अता की) और किसी पैग़म्बर की ये मजाल न थी कि कोई मौजिज़ा ख़ुदा की इजाज़त के बगैर ला दिखाए हर एक वक्त (मौऊद) के लिए (हमारे यहाँ) एक (क़िस्म की) तहरीर (होती) है
\end{hindi}}
\flushright{\begin{Arabic}
\quranayah[13][39]
\end{Arabic}}
\flushleft{\begin{hindi}
फिर इसमें से ख़ुदा जिसको चाहता है मिटा देता है और (जिसको चाहता है बाक़ी रखता है और उसके पास असल किताब (लौहे महफूज़) मौजूद है
\end{hindi}}
\flushright{\begin{Arabic}
\quranayah[13][40]
\end{Arabic}}
\flushleft{\begin{hindi}
और (ए रसूल) जो जो वायदे (अज़ाब वगैरह के) हम उन कुफ्फारों से करते हैं चाहे, उनमें से बाज़ तुम्हारे सामने पूरे कर दिखाएँ या तुम्हें उससे पहले उठा लें बहर हाल तुम पर तो सिर्फ एहकाम का पहुचा देना फर्ज है
\end{hindi}}
\flushright{\begin{Arabic}
\quranayah[13][41]
\end{Arabic}}
\flushleft{\begin{hindi}
और उनसे हिसाब लेना हमारा काम है क्या उन लोगों ने ये बात न देखी कि हम ज़मीन को (फ़ुतुहाते इस्लाम से) उसके तमाम एतराफ (चारो ओर) से (सवाह कुफ्र में) घटाते चले आते हैं और ख़ुदा जो चाहता है हुक्म देता है उसके हुक्म का कोई टालने वाला नहीं और बहुत जल्द हिसाब लेने वाला है
\end{hindi}}
\flushright{\begin{Arabic}
\quranayah[13][42]
\end{Arabic}}
\flushleft{\begin{hindi}
और जो लोग उन (कुफ्फार मक्के) से पहले हो गुज़रे हैं उन लोगों ने भी पैग़म्बरों की मुख़ालफत में बड़ी बड़ी तदबीरे की तो (ख़ाक न हो सका क्योंकि) सब तदबीरे तो ख़ुदा ही के हाथ में हैं जो शख़्श जो कुछ करता है वह उसे खूब जानता है और अनक़रीब कुफ्फार को भी मालूम हो जाएगा कि आख़िरत की खूबी किस के लिए है
\end{hindi}}
\flushright{\begin{Arabic}
\quranayah[13][43]
\end{Arabic}}
\flushleft{\begin{hindi}
और (ऐ रसूल) काफिर लोग कहते हैं कि तुम पैग़म्बर नही हो तो तुम (उनसे) कह दो कि मेरे और तुम्हारे दरमियान मेरी रिसालत की गवाही के वास्ते ख़ुदा और वह शख़्श जिसके पास (आसमानी) किताब का इल्म है काफी है
\end{hindi}}
\chapter{Ibrahim (Abraham)}
\begin{Arabic}
\Huge{\centerline{\basmalah}}\end{Arabic}
\flushright{\begin{Arabic}
\quranayah[14][1]
\end{Arabic}}
\flushleft{\begin{hindi}
अलिफ़ लाम रा ऐ रसूल ये (क़ुरान वह) किताब है जिसकों हमने तुम्हारे पास इसलिए नाज़िल किया है कि तुम लोगों को परवरदिगार के हुक्म से (कुफ्र की) तारीकी से (ईमान की) रौशनी में निकाल लाओ ग़रज़ उसकी राह पर लाओ जो सब पर ग़ालिब और सज़ावार हम्द है
\end{hindi}}
\flushright{\begin{Arabic}
\quranayah[14][2]
\end{Arabic}}
\flushleft{\begin{hindi}
वह ख़ुदा को कुछ आसमानों में और जो कुछ ज़मीन में है (ग़रज़ सब कुछ) उसी का है और (आख़िरत में) काफिरों को लिए जो सख्त अज़ाब (मुहय्या किया गया) है अफसोस नाक है
\end{hindi}}
\flushright{\begin{Arabic}
\quranayah[14][3]
\end{Arabic}}
\flushleft{\begin{hindi}
वह कुफ्फार जो दुनिया की (चन्द रोज़ा) ज़िन्दगी को आख़िरत पर तरजीह देते हैं और (लोगों) को ख़ुदा की राह (पर चलने) से रोकते हैं और इसमें ख्वाह मा ख्वाह कज़ी पैदा करना चाहते हैं यही लोग बड़े पल्ले दर्जे की गुमराही में हैं
\end{hindi}}
\flushright{\begin{Arabic}
\quranayah[14][4]
\end{Arabic}}
\flushleft{\begin{hindi}
और हमने जब कभी कोई पैग़म्बर भेजा तो उसकी क़ौम की ज़बान में बातें करता हुआ (ताकि उसके सामने (हमारे एहक़ाम) बयान कर सके तो यही ख़ुदा जिसे चाहता है गुमराही में छोड़ देता है और जिस की चाहता है हिदायत करता है वही सब पर ग़ालिब हिकमत वाला है
\end{hindi}}
\flushright{\begin{Arabic}
\quranayah[14][5]
\end{Arabic}}
\flushleft{\begin{hindi}
और हमने मूसा को अपनी निशानियाँ देकर भेजा (और ये हुक्म दिया) कि अपनी क़ौम को (कुफ्र की) तारिकियों से (ईमान की) रौशनी में निकाल लाओ और उन्हें ख़ुदा के (वह) दिन याद दिलाओ (जिनमें ख़ुदा की बड़ी बड़ी कुदरतें ज़ाहिर हुई) इसमें शक़ नहीं इसमें तमाम सब्र शुक्र करने वालों के वास्ते (कुदरते ख़ुदा की) बहुत सी निशानियाँ हैं
\end{hindi}}
\flushright{\begin{Arabic}
\quranayah[14][6]
\end{Arabic}}
\flushleft{\begin{hindi}
और वह (वक्त याद दिलाओ) जब मूसा ने अपनी क़ौम से कहा कि ख़ुदा ने जो एहसान तुम पर किए हैं उनको याद करो जब अकेले तुमको फिरऔन के लोगों (के ज़ुल्म) से नजात दी कि वह तुम को बहुत बड़े बड़े दुख दे के सताते थे तुम्हारा लड़कों को जबाह कर डालते थे और तुम्हारी औरतों को (अपनी ख़िदमत के वास्ते) जिन्दा रहने देते थे और इसमें तुम्हारा परवरदिगार की तरफ से (तुम्हारा सब्र की) बड़ी (सख्त) आज़माइश थी
\end{hindi}}
\flushright{\begin{Arabic}
\quranayah[14][7]
\end{Arabic}}
\flushleft{\begin{hindi}
और (वह वक्त याद दिलाओ) जब तुम्हारे परवरदिगार ने तुम्हें जता दिया कि अगर (मेरा) शुक्र करोगें तो मै यक़ीनन तुम पर (नेअमत की) ज्यादती करुँगा और अगर कहीं तुमने नाशुक्री की तो (याद रखो कि) यक़ीनन मेरा अज़ाब सख्त है
\end{hindi}}
\flushright{\begin{Arabic}
\quranayah[14][8]
\end{Arabic}}
\flushleft{\begin{hindi}
और मूसा ने (अपनी क़ौम से) कह दिया कि अगर और (तुम्हारे साथ) जितने रुए ज़मीन पर हैं सब के सब (मिलकर भी ख़ुदा की) नाशुक्री करो तो ख़ुदा (को ज़रा भी परवाह नहीं क्योंकि वह तो बिल्कुल) बे नियाज़ है
\end{hindi}}
\flushright{\begin{Arabic}
\quranayah[14][9]
\end{Arabic}}
\flushleft{\begin{hindi}
और हम्द है क्या तुम्हारे पास उन लोगों की ख़बर नहीं पहुँची जो तुमसे पहले थे (जैसे) नूह की क़ौम और आद व समूद और (दूसरे लोग) जो उनके बाद हुए (क्योकर ख़बर होती) उनको ख़ुदा के सिवा कोई जानता ही नहीं उनके पास उनके (वक्त क़े) पैग़म्बर मौजिज़े लेकर आए (और समझाने लगे) तो उन लोगों ने उन पैग़म्बरों के हाथों को उनके मुँह पर उलटा मार दिया और कहने लगे कि जो (हुक्म लेकर) तुम ख़ुदा की तरफ से भेजे गए हो हम तो उसको नहीं मानते और जिस (दीन) की तरफ तुम हमको बुलाते हो बड़े गहरे शक़ में पड़े है
\end{hindi}}
\flushright{\begin{Arabic}
\quranayah[14][10]
\end{Arabic}}
\flushleft{\begin{hindi}
(तब) उनके पैग़म्बरों ने (उनसे) कहा क्या तुम को ख़ुदा के बारे में शक़ है जो सारे आसमान व ज़मीन का पैदा करने वाला (और) वह तुमको अपनी तरफ बुलाता भी है तो इसलिए कि तुम्हारे गुनाह माफ कर दे और एक वक्त मुक़र्रर तक तुमको (दुनिया में चैन से) रहने दे वह लोग बोल उठे कि तुम भी बस हमारे ही से आदमी हो (अच्छा) अब समझे तुम ये चाहते हो कि जिन माबूदों की हमारे बाप दादा परसतिश करते थे तुम हमको उनसे बाज़ रखो अच्छा अगर तुम सच्चे हो तो कोई साफ खुला हुआ सरीही मौजिज़ा हमे ला दिखाओ
\end{hindi}}
\flushright{\begin{Arabic}
\quranayah[14][11]
\end{Arabic}}
\flushleft{\begin{hindi}
उनके पैग़म्बरों ने उनके जवाब में कहा कि इसमें शक़ नहीं कि हम भी तुम्हारे ही से आदमी हैं मगर ख़ुदा अपने बन्दों में जिस पर चाहता है अपना फज़ल (व करम) करता है (और) रिसालत अता करता है और हमारे एख्तियार मे ये बात नही कि बे हुक्मे ख़ुदा (तुम्हारी फरमाइश के मुवाफिक़) हम कोई मौजिज़ा तुम्हारे सामने ला सकें और ख़ुदा ही पर सब ईमानदारों को भरोसा रखना चाहिए
\end{hindi}}
\flushright{\begin{Arabic}
\quranayah[14][12]
\end{Arabic}}
\flushleft{\begin{hindi}
और हमें (आख़िर) क्या है कि हम उस पर भरोसा न करें हालॉकि हमे (निजात की) आसान राहें दिखाई और जो तूने अज़ियतें हमें पहुँचाइ (उन पर हमने सब्र किया और आइन्दा भी सब्र करेगें और तवक्कल भरोसा करने वालो को ख़ुदा ही पर तवक्कल करना चाहिए
\end{hindi}}
\flushright{\begin{Arabic}
\quranayah[14][13]
\end{Arabic}}
\flushleft{\begin{hindi}
और जिन लोगों नें कुफ्र एख्तियार किया था अपने (वक्त क़े) पैग़म्बरों से कहने लगे हम तो तुमको अपनी सरज़मीन से ज़रुर निकाल बाहर कर देगें यहाँ तक कि तुम फिर हमारे मज़हब की तरफ पलट आओ-तो उनके परवरदिगार ने उनकी तरफ वही भेजी कि तुम घबराओं नहीं हम उन सरकश लोगों को ज़रुर बर्बाद करेगें
\end{hindi}}
\flushright{\begin{Arabic}
\quranayah[14][14]
\end{Arabic}}
\flushleft{\begin{hindi}
और उनकी हलाकत के बाद ज़रुर तुम्ही को इस सरज़मीन में बसाएगें ये (वायदा) महज़ उस शख़्श से जो हमारी बारगाह में (आमाल की जवाब देही में) खड़े होने से डरे
\end{hindi}}
\flushright{\begin{Arabic}
\quranayah[14][15]
\end{Arabic}}
\flushleft{\begin{hindi}
और हमारे अज़ाब से ख़ौफ खाए और उन पैग़म्बरों हम से अपनी फतेह की दुआ माँगी (आख़िर वह पूरी हुई)
\end{hindi}}
\flushright{\begin{Arabic}
\quranayah[14][16]
\end{Arabic}}
\flushleft{\begin{hindi}
और हर एक सरकश अदावत रखने वाला हलाक हुआ (ये तो उनकी सज़ा थी और उसके पीछे ही पीछे जहन्नुम है और उसमें) से पीप लहू भरा हुआ पानी पीने को दिया जाएगा
\end{hindi}}
\flushright{\begin{Arabic}
\quranayah[14][17]
\end{Arabic}}
\flushleft{\begin{hindi}
(ज़बरदस्ती) उसे घूँट घँट करके पीना पड़ेगा और उसे हलक़ से आसानी से न उतार सकेगा और (वह मुसीबत है कि) उसे हर तरफ से मौत ही मौत आती दिखाई देती है हालॉकि वह मारे न मर सकेगा-और फिर उसके पीछे अज़ाब सख्त होगा
\end{hindi}}
\flushright{\begin{Arabic}
\quranayah[14][18]
\end{Arabic}}
\flushleft{\begin{hindi}
जो लोग अपने परवरदिगार से काफिर हो बैठे हैं उनकी मसल ऐसी है कि उनकी कारस्तानियाँ गोया (राख का एक ढेर) है जिसे (अन्धड़ के रोज़ हवा का बड़े ज़ोरों का झोंका उड़ा लेगा जो कुछ उन लोगों ने (दुनिया में) किया कराया उसमें से कुछ भी उनके क़ाबू में न होगा यही तो पल्ले दर्जे की गुमराही है
\end{hindi}}
\flushright{\begin{Arabic}
\quranayah[14][19]
\end{Arabic}}
\flushleft{\begin{hindi}
क्या तूने नहीं देखा कि ख़ुदा ही ने सारे आसमान व ज़मीन ज़रुर मसलहत से पैदा किए अगर वह चाहे तो सबको मिटाकर एक नई खिलक़त (बस्ती) ला बसाए
\end{hindi}}
\flushright{\begin{Arabic}
\quranayah[14][20]
\end{Arabic}}
\flushleft{\begin{hindi}
औ ये ख़ुदा पर कुछ भी दुशवार नहीं
\end{hindi}}
\flushright{\begin{Arabic}
\quranayah[14][21]
\end{Arabic}}
\flushleft{\begin{hindi}
और (क़यामत के दिन) लोग सबके सब ख़ुदा के सामने निकल खड़े होगें जो लोग (दुनिया में कमज़ोर थे बड़ी इज्ज़त रखने वालो से (उस वक्त) क़हेंगें कि हम तो बस तुम्हारे क़दम ब क़दम चलने वाले थे तो क्या (आज) तुम ख़ुदा के अज़ाब से कुछ भी हमारे आड़े आ सकते हो वह जवाब देगें काश ख़ुदा हमारी हिदायत करता तो हम भी तुम्हारी हिदायत करते हम ख्वाह बेक़रारी करें ख्वाह सब्र करे (दोनो) हमारे लिए बराबर है (क्योंकि अज़ाब से) हमें तो अब छुटकारा नहीं
\end{hindi}}
\flushright{\begin{Arabic}
\quranayah[14][22]
\end{Arabic}}
\flushleft{\begin{hindi}
और जब (लोगों का) ख़ैर फैसला हो चुकेगा (और लोग शैतान को इल्ज़ाम देगें) तो शैतान कहेगा कि ख़ुदा ने तुम से सच्चा वायदा किया था (तो वह पूरा हो गया) और मैने भी वायदा तो किया था फिर मैने वायदा ख़िलाफ़ी की और मुझे कुछ तुम पर हुकूमत तो थी नहीं मगर इतनी बात थी कि मैने तुम को (बुरे कामों की तरफ) बुलाया और तुमने मेरा कहा मान लिया तो अब तुम मुझे बुंरा (भला) न कहो बल्कि (अगर कहना है तो) अपने नफ्स को बुरा कहो (आज) न तो मैं तुम्हारी फरियाद को पहुँचा सकता हूँ और न तुम मेरी फरियाद कर सकते हो मै तो उससे पहले ही बेज़ार हूँ कि तुमने मुझे (ख़ुदा का) शरीक बनाया बेशक जो लोग नाफरमान हैं उनके लिए दर्दनाक अज़ाब है
\end{hindi}}
\flushright{\begin{Arabic}
\quranayah[14][23]
\end{Arabic}}
\flushleft{\begin{hindi}
और जिन लोगों ने (सदक़ दिल से) ईमान क़ुबूल किया और अच्छे (अच्छे) काम किए वह (बेहश्त के) उन बाग़ों में दाख़िल किए जाएँगें जिनके नीचे नहरे जारी होगी और वह अपने परवरदिगार के हुक्म से हमेशा उसमें रहेगें वहाँ उन (की मुलाक़ात) का तोहफा सलाम का हो
\end{hindi}}
\flushright{\begin{Arabic}
\quranayah[14][24]
\end{Arabic}}
\flushleft{\begin{hindi}
(ऐ रसूल) क्या तुमने नहीं देखा कि ख़ुदा ने अच्छी बात (मसलन कलमा तौहीद की) वैसी अच्छी मिसाल बयान की है कि (अच्छी बात) गोया एक पाकीज़ा दरख्त है कि उसकी जड़ मज़बूत है और उसकी टहनियाँ आसमान में लगी हो
\end{hindi}}
\flushright{\begin{Arabic}
\quranayah[14][25]
\end{Arabic}}
\flushleft{\begin{hindi}
अपने परवरदिगार के हुक्म से हर वक्त फ़ला (फूला) रहता है और ख़ुदा लोगों के वास्ते (इसलिए) मिसालें बयान फरमाता है ताकि लोग नसीहत व इबरत हासिल करें
\end{hindi}}
\flushright{\begin{Arabic}
\quranayah[14][26]
\end{Arabic}}
\flushleft{\begin{hindi}
और गन्दी बात (जैसे कलमाए शिर्क) की मिसाल गोया एक गन्दे दरख्त की सी है (जिसकी जड़ ऐसी कमज़ोर हो) कि ज़मीन के ऊपर ही से उखाड़ फेंका जाए (क्योंकि) उसको कुछ ठहराओ तो है नहीं
\end{hindi}}
\flushright{\begin{Arabic}
\quranayah[14][27]
\end{Arabic}}
\flushleft{\begin{hindi}
जो लोग पक्की बात (कलमा तौहीद) पर (सदक़ दिल से ईमान ला चुके उनको ख़ुदा दुनिया की ज़िन्दगी में भी साबित क़दम रखता है और आख़िरत में भी साबित क़दम रखेगा (और) उन्हें सवाल व जवाब में कोई वक्त न होगा और सरकशों को ख़ुदा गुमराही में छोड़ देता है और ख़ुदा जो चाहता है करता है
\end{hindi}}
\flushright{\begin{Arabic}
\quranayah[14][28]
\end{Arabic}}
\flushleft{\begin{hindi}
(ऐ रसूल) क्या तुमने उन लोगों के हाल पर ग़ौर नहीं किया जिन्होंने मेरे एहसान के बदले नाशुक्री की एख्तियार की और अपनी क़ौम को हलाकत के घरवाहे (जहन्नुम) में झोंक दिया
\end{hindi}}
\flushright{\begin{Arabic}
\quranayah[14][29]
\end{Arabic}}
\flushleft{\begin{hindi}
कि सबके सब जहन्नुम वासिल होगें और वह (क्या) बुरा ठिकाना है
\end{hindi}}
\flushright{\begin{Arabic}
\quranayah[14][30]
\end{Arabic}}
\flushleft{\begin{hindi}
और वह लोग दूसरो को ख़ुदा का हमसर (बराबर) बनाने लगे ताकि (लोगों को) उसकी राह से बहका दे (ऐ रसूल) तुम कह दो कि (ख़ैर चन्द रोज़ तो) चैन कर लो फिर तो तुम्हें दोज़ख की तरफ लौट कर जाना ही है
\end{hindi}}
\flushright{\begin{Arabic}
\quranayah[14][31]
\end{Arabic}}
\flushleft{\begin{hindi}
(ऐ रसूल) मेरे वह बन्दे जो ईमान ला चुके उन से कह दो कि पाबन्दी से नमाज़ पढ़ा करें और जो कुछ हमने उन्हें रोज़ी दी है उसमें से (ख़ुदा की राह में) छिपाकर या दिखा कर ख़र्च किया करे उस दिन (क़यामत) के आने से पहल जिसमें न तो (ख़रीदो) फरोख्त ही (काम आएगी) न दोस्ती मोहब्बत काम (आएगी)
\end{hindi}}
\flushright{\begin{Arabic}
\quranayah[14][32]
\end{Arabic}}
\flushleft{\begin{hindi}
ख़ुदा ही ऐसा (क़ादिर तवाना) है जिसने सारे आसमान व ज़मीन पैदा कर डाले और आसमान से पानी बरसाया फिर उसके ज़रिए से (मुख्तलिफ दरख्तों से) तुम्हारी रोज़ा के वास्ते (तरह तरह) के फल पैदा किए और तुम्हारे वास्ते कश्तियां तुम्हारे बस में कर दी-ताकि उसके हुक्म से दरिया में चलें और तुम्हारे वास्ते नदियों को तुम्हारे एख्तियार में कर दिया
\end{hindi}}
\flushright{\begin{Arabic}
\quranayah[14][33]
\end{Arabic}}
\flushleft{\begin{hindi}
और सूरज और चाँद को तुम्हारा ताबेदार बना दिया कि सदा फेरी किया करते हैं और रात और दिन को तुम्हारे क़ब्ज़े में कर दिया कि हमेशा हाज़िर रहते हैं
\end{hindi}}
\flushright{\begin{Arabic}
\quranayah[14][34]
\end{Arabic}}
\flushleft{\begin{hindi}
(और अपनी ज़रुरत के मुवाफिक) जो कुछ तुमने उससे माँगा उसमें से (तुम्हारी ज़रूरत भर) तुम्हे दिया और तुम ख़ुदा की नेमतो गिनती करना चाहते हो तो गिन नहीं सकते हो तू बड़ा बे इन्साफ नाशुक्रा है
\end{hindi}}
\flushright{\begin{Arabic}
\quranayah[14][35]
\end{Arabic}}
\flushleft{\begin{hindi}
और (वह वक्त याद करो) जब इबराहीम ने (ख़ुदा से) अर्ज़ की थी कि परवरदिगार इस शहर (मक्के) को अमन व अमान की जगह बना दे और मुझे और मेरी औलाद को इस बात को बचा ले कि बुतों की परसतिश करने लगे
\end{hindi}}
\flushright{\begin{Arabic}
\quranayah[14][36]
\end{Arabic}}
\flushleft{\begin{hindi}
ऐ मेरे पालने वाले इसमें शक़ नहीं कि इन बुतों ने बहुतेरे लोगों को गुमराह बना छोड़ा तो जो शख़्श मेरी पैरवी करे तो वह मुझ से है और जिसने मेरी नाफ़रमानी की (तो तुझे एख्तेयार है) तू तो बड़ा बख्शने वपला मेहरबान है)
\end{hindi}}
\flushright{\begin{Arabic}
\quranayah[14][37]
\end{Arabic}}
\flushleft{\begin{hindi}
ऐ हमारे पालने वाले मैने तेरे मुअज़िज़ (इज्ज़त वाले) घर (काबे) के पास एक बेखेती के (वीरान) बियाबान (मक्का) में अपनी कुछ औलाद को (लाकर) बसाया है ताकि ऐ हमारे पालने वाले ये लोग बराबर यहाँ नमाज़ पढ़ा करें तो तू कुछ लोगों के दिलों को उनकी तरफ माएल कर (ताकि वह यहाँ आकर आबाद हों) और उन्हें तरह तरह के फलों से रोज़ी अता कर ताकि ये लोग (तेरा) शुक्र करें
\end{hindi}}
\flushright{\begin{Arabic}
\quranayah[14][38]
\end{Arabic}}
\flushleft{\begin{hindi}
ऐ हमारे पालने वाले जो कुछ हम छिपाते हैं और जो कुछ ज़ाहिर करते हैं तू (सबसे) खूब वाक़िफ है और ख़ुदा से तो कोई चीज़ छिपी नहीं (न) ज़मीन में और न आसमान में उस ख़ुदा का (लाख लाख) शुक्र है
\end{hindi}}
\flushright{\begin{Arabic}
\quranayah[14][39]
\end{Arabic}}
\flushleft{\begin{hindi}
जिसने मुझे बुढ़ापा आने पर इस्माईल व इसहाक़ (दो फरज़न्द) अता किए इसमें तो शक़ नहीं कि मेरा परवरदिगार दुआ का सुनने वाला है
\end{hindi}}
\flushright{\begin{Arabic}
\quranayah[14][40]
\end{Arabic}}
\flushleft{\begin{hindi}
(ऐ मेरे पालने वाले मुझे और मेरी औलाद को (भी) नमाज़ का पाबन्द बना दे और ऐ मेरे पालने वाले मेरी दुआ क़ुबूल फरमा
\end{hindi}}
\flushright{\begin{Arabic}
\quranayah[14][41]
\end{Arabic}}
\flushleft{\begin{hindi}
ऐ हमारे पालने वाले जिस दिन (आमाल का) हिसाब होने लगे मुझको और मेरे माँ बाप को और सारे ईमानदारों को तू बख्श दे
\end{hindi}}
\flushright{\begin{Arabic}
\quranayah[14][42]
\end{Arabic}}
\flushleft{\begin{hindi}
और जो कुछ ये कुफ्फ़ार (कुफ्फ़ारे मक्का) किया करते हैं उनसे ख़ुदा को ग़ाफिल न समझना (और उन पर फौरन अज़ाब न करने की) सिर्फ ये वजह है कि उस दिन तक की मोहलत देता है जिस दिन लोगों की ऑंखों के ढेले (ख़ौफ के मारे) पथरा जाएँगें
\end{hindi}}
\flushright{\begin{Arabic}
\quranayah[14][43]
\end{Arabic}}
\flushleft{\begin{hindi}
(और अपने अपने सर उठाए भागे चले जा रहे हैं (टकटकी बँधी है कि) उनकी तरफ उनकी नज़र नहीं लौटती (जिधर देख रहे हैं) और उनके दिल हवा हवा हो रहे हैं
\end{hindi}}
\flushright{\begin{Arabic}
\quranayah[14][44]
\end{Arabic}}
\flushleft{\begin{hindi}
और (ऐ रसूल) लोगों को उस दिन से डराओ (जिस दिन) उन पर अज़ाब नाज़िल होगा तो जिन लोगों ने नाफरमानी की थी (गिड़गिड़ा कर) अर्ज़ करेगें कि ऐ हमारे पालने वाले हम को थोड़ी सी मोहलत और दे दे (अबकी बार) हम तेरे बुलाने पर ज़रुर उठ खड़े होगें और सब रसूलों की पैरवी करेगें (तो उनको जवाब मिलेगा) क्या तुम वह लोग नहीं हो जो उसके पहले (उस पर) क़समें खाया करते थे कि तुम को किसी तरह का ज़व्वाल (नुक्सान) नहीं
\end{hindi}}
\flushright{\begin{Arabic}
\quranayah[14][45]
\end{Arabic}}
\flushleft{\begin{hindi}
(और क्या तुम वह लोग नहीं कि) जिन लोगों ने (हमारी नाफ़रमानी करके) आप अपने ऊपर जुल्म किया उन्हीं के घरों में तुम भी रहे हालॉकि तुम पर ये भी ज़ाहिर हो चुका था कि हमने उनके साथ क्या (बरताओ) किया और हमने (तुम्हारे समझाने के वास्ते) मसले भी बयान कर दी थीं
\end{hindi}}
\flushright{\begin{Arabic}
\quranayah[14][46]
\end{Arabic}}
\flushleft{\begin{hindi}
और वह लोग अपनी चालें चलते हैं (और कभी बाज़ न आए) हालॉकि उनकी सब हालतें खुदा की नज़र में थी और अगरचे उनकी मक्कारियाँ (उस गज़ब की) थीं कि उन से पहाड़ (अपनी जगह से) हट जाये
\end{hindi}}
\flushright{\begin{Arabic}
\quranayah[14][47]
\end{Arabic}}
\flushleft{\begin{hindi}
तो तुम ये ख्याल (भी) न करना कि ख़ुदा अपने रसूलों से ख़िलाफ वायदा करेगा इसमें शक़ नहीं कि ख़ुदा (सबसे) ज़बरदस्त बदला लेने वाला है
\end{hindi}}
\flushright{\begin{Arabic}
\quranayah[14][48]
\end{Arabic}}
\flushleft{\begin{hindi}
(मगर कब) जिस दिन ये ज़मीन बदलकर दूसरी ज़मीन कर दी जाएगी और (इसी तरह) आसमान (भी बदल दिए जाएँगें) और सब लोग यकता क़हार (ज़बरदस्त) ख़ुदा के रुबरु (अपनी अपनी जगह से) निकल खड़े होगें
\end{hindi}}
\flushright{\begin{Arabic}
\quranayah[14][49]
\end{Arabic}}
\flushleft{\begin{hindi}
और तुम उस दिन गुनेहगारों को देखोगे कि ज़ज़ीरों मे जकड़े हुए होगें
\end{hindi}}
\flushright{\begin{Arabic}
\quranayah[14][50]
\end{Arabic}}
\flushleft{\begin{hindi}
उनके (बदन के) कपड़े क़तरान (तारकोल) के होगे और उनके चेहरों को आग (हर तरफ से) ढाके होगी
\end{hindi}}
\flushright{\begin{Arabic}
\quranayah[14][51]
\end{Arabic}}
\flushleft{\begin{hindi}
ताकि ख़ुदा हर शख़्श को उसके किए का बदला दे (अच्छा तो अच्छा बुरा तो बुरा) बेशक ख़ुदा बहुत जल्द हिसाब लेने वाला है
\end{hindi}}
\flushright{\begin{Arabic}
\quranayah[14][52]
\end{Arabic}}
\flushleft{\begin{hindi}
ये (क़ुरान) लोगों के लिए एक क़िस्म की इत्तेला (जानकारी) है ताकि लोग उसके ज़रिये से (अज़ाबे ख़ुदा से) डराए जाए और ताकि ये भी ये यक़ीन जान लें कि बस वही (ख़ुदा) एक माबूद है और ताकि जो लोग अक्ल वाले हैं नसीहत व इबरत हासिल करें
\end{hindi}}
\chapter{Al-Hijr (The Rock)}
\begin{Arabic}
\Huge{\centerline{\basmalah}}\end{Arabic}
\flushright{\begin{Arabic}
\quranayah[15][1]
\end{Arabic}}
\flushleft{\begin{hindi}
अलिफ़ लाम रा ये किताब (ख़ुदा) और वाजेए व रौशन क़ुरान की (चन्द) आयते हैं
\end{hindi}}
\flushright{\begin{Arabic}
\quranayah[15][2]
\end{Arabic}}
\flushleft{\begin{hindi}
(एक दिन वह भी आने वाला है कि) जो लोग काफ़िर हो बैठे हैं अक्सर दिल से चाहेंगें
\end{hindi}}
\flushright{\begin{Arabic}
\quranayah[15][3]
\end{Arabic}}
\flushleft{\begin{hindi}
काश (हम भी) मुसलमान होते (ऐ रसूल) उन्हें उनकी हालत पर रहने दो कि खा पी लें और (दुनिया के चन्द रोज़) चैन कर लें और उनकी तमन्नाएँ उन्हें खेल तमाशे में लगाए रहीं
\end{hindi}}
\flushright{\begin{Arabic}
\quranayah[15][4]
\end{Arabic}}
\flushleft{\begin{hindi}
अनक़रीब ही (इसका नतीजा) उन्हें मालूम हो जाएगा और हमने कभी कोई बस्ती तबाह नहीं की मगर ये कि उसकी तबाही के लिए (पहले ही से) समझी बूझी मियाद मुक़र्रर लिखी हुई थी
\end{hindi}}
\flushright{\begin{Arabic}
\quranayah[15][5]
\end{Arabic}}
\flushleft{\begin{hindi}
कोई उम्मत अपने वक्त से न आगे बढ़ सकती है न पीछे हट सकती है
\end{hindi}}
\flushright{\begin{Arabic}
\quranayah[15][6]
\end{Arabic}}
\flushleft{\begin{hindi}
(ऐ रसूल कुफ्फ़ारे मक्का तुमसे) कहते हैं कि ऐ शख़्श (जिसको ये भरम है) कि उस पर 'वही' व किताब नाज़िल हुईहै तो (अच्छा ख़ासा) सिड़ी है
\end{hindi}}
\flushright{\begin{Arabic}
\quranayah[15][7]
\end{Arabic}}
\flushleft{\begin{hindi}
अगर तू अपने दावे में सच्चा है तो फरिश्तों को हमारे सामने क्यों नहीं ला खड़ा करता
\end{hindi}}
\flushright{\begin{Arabic}
\quranayah[15][8]
\end{Arabic}}
\flushleft{\begin{hindi}
(हालॉकि) हम फरिश्तों को खुल्लम खुल्ला (जिस अज़ाब के साथ) फैसले ही के लिए भेजा करते हैं और (अगर फरिश्ते नाज़िल हो जाए तो) फिर उनको (जान बचाने की) मोहलत भी न मिले
\end{hindi}}
\flushright{\begin{Arabic}
\quranayah[15][9]
\end{Arabic}}
\flushleft{\begin{hindi}
बेशक हम ही ने क़ुरान नाज़िल किया और हम ही तो उसके निगेहबान भी हैं
\end{hindi}}
\flushright{\begin{Arabic}
\quranayah[15][10]
\end{Arabic}}
\flushleft{\begin{hindi}
(ऐ रसूल) हमने तो तुमसे पहले भी अगली उम्मतों में (और भी बहुत से) रसूल भेजे
\end{hindi}}
\flushright{\begin{Arabic}
\quranayah[15][11]
\end{Arabic}}
\flushleft{\begin{hindi}
और (उनकी भी यही हालत थी कि) उनके पास कोई रसूल न आया मगर उन लोगों ने उसकी हँसी ज़रुर उड़ाई
\end{hindi}}
\flushright{\begin{Arabic}
\quranayah[15][12]
\end{Arabic}}
\flushleft{\begin{hindi}
हम (गोया खुद) इसी तरह इस (गुमराही) को (उन) गुनाहगारों के दिल में डाल देते हैं
\end{hindi}}
\flushright{\begin{Arabic}
\quranayah[15][13]
\end{Arabic}}
\flushleft{\begin{hindi}
ये कुफ्फ़ार इस (क़ुरान) पर ईमान न लाएँगें और (ये कुछ अनोखी बात नहीं) अगलों के तरीक़े भी (ऐसे ही) रहें है
\end{hindi}}
\flushright{\begin{Arabic}
\quranayah[15][14]
\end{Arabic}}
\flushleft{\begin{hindi}
और अगर हम अपनी कुदरत से आसमान का एक दरवाज़ा भी खोल दें और ये लोग दिन दहाड़े उस दरवाज़े से (आसमान पर) चढ़ भी जाएँ
\end{hindi}}
\flushright{\begin{Arabic}
\quranayah[15][15]
\end{Arabic}}
\flushleft{\begin{hindi}
तब भी यहीं कहेगें कि हो न हो हमारी ऑंखें (नज़र बन्दी से) मतवाली कर दी गई हैं या नहीं तो हम लोगों पर जादू किया गया है
\end{hindi}}
\flushright{\begin{Arabic}
\quranayah[15][16]
\end{Arabic}}
\flushleft{\begin{hindi}
और हम ही ने आसमान में बुर्ज बनाए और देखने वालों के वास्ते उनके (सितारों से) आरास्ता (सजाया) किया
\end{hindi}}
\flushright{\begin{Arabic}
\quranayah[15][17]
\end{Arabic}}
\flushleft{\begin{hindi}
और हर शैतान मरदूद की आमद रफत (आने जाने) से उन्हें महफूज़ रखा
\end{hindi}}
\flushright{\begin{Arabic}
\quranayah[15][18]
\end{Arabic}}
\flushleft{\begin{hindi}
मगर जो शैतान चोरी छिपे (वहाँ की किसी बात पर) कान लगाए तो यहाब का दहकता हुआ योला उसके (खदेड़ने को) पीछे पड़ जाता है
\end{hindi}}
\flushright{\begin{Arabic}
\quranayah[15][19]
\end{Arabic}}
\flushleft{\begin{hindi}
और ज़मीन को (भी अपने मख़लूक़ात के रहने सहने को) हम ही ने फैलाया और इसमें (कील की तरह) पहाड़ो के लंगर डाल दिए और हमने उसमें हर किस्म की मुनासिब चीज़े उगाई
\end{hindi}}
\flushright{\begin{Arabic}
\quranayah[15][20]
\end{Arabic}}
\flushleft{\begin{hindi}
और हम ही ने उन्हें तुम्हारे वास्ते ज़िन्दगी के साज़ों सामान बना दिए और उन जानवरों के लिए भी जिन्हें तुम रोज़ी नहीं देते
\end{hindi}}
\flushright{\begin{Arabic}
\quranayah[15][21]
\end{Arabic}}
\flushleft{\begin{hindi}
और हमारे यहाँ तो हर चीज़ के बेशुमार खज़ाने (भरे) पड़े हैं और हम (उसमें से) एक जची तली मिक़दार भेजते रहते है
\end{hindi}}
\flushright{\begin{Arabic}
\quranayah[15][22]
\end{Arabic}}
\flushleft{\begin{hindi}
और हम ही ने वह हवाएँ भेजी जो बादलों को पानी से (भरे हुए) है फिर हम ही ने आसमान से पानी बरसाया फिर हम ही ने तुम लोगों को वह पानी पिलाया और तुम लोगों ने तो कुछ उसको जमा करके नहीं रखा था
\end{hindi}}
\flushright{\begin{Arabic}
\quranayah[15][23]
\end{Arabic}}
\flushleft{\begin{hindi}
और इसमें शक़ नहीं कि हम ही (लोगों को) जिलाते और हम ही मार डालते हैं और (फिर) हम ही (सब के) वाली वारिस हैं
\end{hindi}}
\flushright{\begin{Arabic}
\quranayah[15][24]
\end{Arabic}}
\flushleft{\begin{hindi}
और बेशक हम ही ने तुममें से उन लोगों को भी अच्छी तरह समझ लिया जो पहले हो गुज़रे और हमने उनको भी जान लिया जो बाद को आने वाले हैं
\end{hindi}}
\flushright{\begin{Arabic}
\quranayah[15][25]
\end{Arabic}}
\flushleft{\begin{hindi}
और इसमें शक़ नहीं कि तेरा परवरदिगार वही है जो उन सब को (क़यामत में कब्रों से) उठाएगा बेशक वह हिक़मत वाला वाक़िफकार है
\end{hindi}}
\flushright{\begin{Arabic}
\quranayah[15][26]
\end{Arabic}}
\flushleft{\begin{hindi}
और बेशक हम ही ने आदमी को ख़मीर (गुंधी) दी हुईसड़ी मिट्टी से जो (सूखकर) खन खन बोलने लगे पैदा किया
\end{hindi}}
\flushright{\begin{Arabic}
\quranayah[15][27]
\end{Arabic}}
\flushleft{\begin{hindi}
और हम ही ने जिन्नात को आदमी से (भी) पहले वे धुएँ की तेज़ आग से पैदा किया
\end{hindi}}
\flushright{\begin{Arabic}
\quranayah[15][28]
\end{Arabic}}
\flushleft{\begin{hindi}
और (ऐ रसूल वह वक्त याद करो) जब तुम्हारे परवरदिगार ने फरिश्तों से कहा कि मैं एक आदमी को खमीर दी हुई मिट्टी से (जो सूखकर) खन खन बोलने लगे पैदा करने वाला हूँ
\end{hindi}}
\flushright{\begin{Arabic}
\quranayah[15][29]
\end{Arabic}}
\flushleft{\begin{hindi}
तो जिस वक्त मै उसको हर तरह से दुरुस्त कर चुके और उसमें अपनी (तरफ से) रुह फूँक दूँ तो सब के सब उसके सामने सजदे में गिर पड़ना
\end{hindi}}
\flushright{\begin{Arabic}
\quranayah[15][30]
\end{Arabic}}
\flushleft{\begin{hindi}
ग़रज़ फरिश्ते तो सब के सब सर ब सजूद हो गए
\end{hindi}}
\flushright{\begin{Arabic}
\quranayah[15][31]
\end{Arabic}}
\flushleft{\begin{hindi}
मगर इबलीस (मलऊन) की उसने सजदा करने वालों के साथ शामिल होने से इन्कार किया
\end{hindi}}
\flushright{\begin{Arabic}
\quranayah[15][32]
\end{Arabic}}
\flushleft{\begin{hindi}
(इस पर ख़ुदा ने) फरमाया आओ शैतान आख़िर तुझे क्या हुआ कि तू सजदा करने वालों के साथ शामिल न हुआ
\end{hindi}}
\flushright{\begin{Arabic}
\quranayah[15][33]
\end{Arabic}}
\flushleft{\begin{hindi}
वह (ढिठाई से) कहने लगा मैं ऐसा गया गुज़रा तो हूँ नहीं कि ऐसे आदमी को सजदा कर बैठूँ जिसे तूने सड़ी हुई खन खन बोलने वाली मिट्टी से पैदा किया है
\end{hindi}}
\flushright{\begin{Arabic}
\quranayah[15][34]
\end{Arabic}}
\flushleft{\begin{hindi}
ख़ुदा ने फरमाया (नहीं तू) तो बेहश्त से निकल जा (दूर हो) कि बेशक तू मरदूद है
\end{hindi}}
\flushright{\begin{Arabic}
\quranayah[15][35]
\end{Arabic}}
\flushleft{\begin{hindi}
और यक़ीनन तुझ पर रोज़े में जज़ा तक फिटकार बरसा करेगी
\end{hindi}}
\flushright{\begin{Arabic}
\quranayah[15][36]
\end{Arabic}}
\flushleft{\begin{hindi}
शैतान ने कहा ऐ मेरे परवरदिगार ख़ैर तू मुझे उस दिन तक की मोहलत दे जबकि (लोग दोबारा ज़िन्दा करके) उठाए जाएँगें
\end{hindi}}
\flushright{\begin{Arabic}
\quranayah[15][37]
\end{Arabic}}
\flushleft{\begin{hindi}
ख़ुदा ने फरमाया वक्त मुक़र्रर
\end{hindi}}
\flushright{\begin{Arabic}
\quranayah[15][38]
\end{Arabic}}
\flushleft{\begin{hindi}
के दिन तक तुझे मोहलत दी गई
\end{hindi}}
\flushright{\begin{Arabic}
\quranayah[15][39]
\end{Arabic}}
\flushleft{\begin{hindi}
उन शैतान ने कहा ऐ मेरे परवरदिगार चूंकि तूने मुझे रास्ते से अलग किया मैं भी उनके लिए दुनिया में (साज़ व सामान को) उम्दा कर दिखाऊँगा और सबको ज़रुर बहकाऊगा
\end{hindi}}
\flushright{\begin{Arabic}
\quranayah[15][40]
\end{Arabic}}
\flushleft{\begin{hindi}
मगर उनमें से तेरे निरे खुरे ख़ास बन्दे (कि वह मेरे बहकाने में न आएँगें)
\end{hindi}}
\flushright{\begin{Arabic}
\quranayah[15][41]
\end{Arabic}}
\flushleft{\begin{hindi}
ख़ुदा ने फरमाया कि यही राह सीधी है कि मुझ तक (पहुँचती) है
\end{hindi}}
\flushright{\begin{Arabic}
\quranayah[15][42]
\end{Arabic}}
\flushleft{\begin{hindi}
जो मेरे मुख़लिस (ख़ास बन्दे) बन्दे हैं उन पर तुझसे किसी तरह की हुकूमत न होगी मगर हाँ गुमराहों में से जो तेरी पैरवी करे (उस पर तेरा वार चल जाएगा)
\end{hindi}}
\flushright{\begin{Arabic}
\quranayah[15][43]
\end{Arabic}}
\flushleft{\begin{hindi}
और हाँ ये भी याद रहे कि उन सब के वास्ते (आख़िरी) वायदा बस जहन्न ुम है जिसके सात दरवाजे होगे
\end{hindi}}
\flushright{\begin{Arabic}
\quranayah[15][44]
\end{Arabic}}
\flushleft{\begin{hindi}
हर (दरवाज़े में जाने) के लिए उन गुमराहों की अलग अलग टोलियाँ होगीं
\end{hindi}}
\flushright{\begin{Arabic}
\quranayah[15][45]
\end{Arabic}}
\flushleft{\begin{hindi}
और परहेज़गार तो बेहश्त के बाग़ों और चश्मों मे यक़ीनन होंगे
\end{hindi}}
\flushright{\begin{Arabic}
\quranayah[15][46]
\end{Arabic}}
\flushleft{\begin{hindi}
(दाख़िले के वक्त फ़रिश्ते कहेगें कि) उनमें सलामती इत्मिनान से चले चलो
\end{hindi}}
\flushright{\begin{Arabic}
\quranayah[15][47]
\end{Arabic}}
\flushleft{\begin{hindi}
और (दुनिया की तकलीफों से) जो कुछ उनके दिल में रंज था उसको भी हम निकाल देगें और ये बाहम एक दूसरे के आमने सामने तख्तों पर इस तरह बैठे होगें जैसे भाई भाई
\end{hindi}}
\flushright{\begin{Arabic}
\quranayah[15][48]
\end{Arabic}}
\flushleft{\begin{hindi}
उनको बेहश्त में तकलीफ छुएगी भी तो नहीं और न कभी उसमें से निकाले जाएँगें
\end{hindi}}
\flushright{\begin{Arabic}
\quranayah[15][49]
\end{Arabic}}
\flushleft{\begin{hindi}
(ऐ रसूल) मेरे बन्दों को आगाह करो कि बेशक मै बड़ा बख्शने वाला मेहरबान हूँ
\end{hindi}}
\flushright{\begin{Arabic}
\quranayah[15][50]
\end{Arabic}}
\flushleft{\begin{hindi}
मगर साथ ही इसके (ये भी याद रहे कि) बेशक मेरा अज़ाब भी बड़ा दर्दनाक अज़ाब है
\end{hindi}}
\flushright{\begin{Arabic}
\quranayah[15][51]
\end{Arabic}}
\flushleft{\begin{hindi}
और उनको इबराहीम के मेहमान का हाल सुना दो
\end{hindi}}
\flushright{\begin{Arabic}
\quranayah[15][52]
\end{Arabic}}
\flushleft{\begin{hindi}
कि जब ये इबराहीम के पास आए तो (पहले) उन्होंने सलाम किया इबराहीम ने (जवाब सलाम के बाद) कहा हमको तो तुम से डर मालूम होता है
\end{hindi}}
\flushright{\begin{Arabic}
\quranayah[15][53]
\end{Arabic}}
\flushleft{\begin{hindi}
उन्होंने कहा आप मुत्तालिक़ ख़ौफ न कीजिए (क्योंकि) हम तो आप को एक (दाना व बीना) फरज़न्द (के पैदाइश) की खुशख़बरी देते हैं
\end{hindi}}
\flushright{\begin{Arabic}
\quranayah[15][54]
\end{Arabic}}
\flushleft{\begin{hindi}
इब्राहिम ने कहा क्या मुझे ख़ुशख़बरी (बेटा होने की) देते हो जब मुझे बुढ़ापा छा गया
\end{hindi}}
\flushright{\begin{Arabic}
\quranayah[15][55]
\end{Arabic}}
\flushleft{\begin{hindi}
तो फिर अब काहे की खुशख़बरी देते हो वह फरिश्ते बोले हमने आप को बिल्कुल ठीक खुशख़बरी दी है तो आप (बारगाह ख़ुदा बन्दी से) ना उम्मीद न हो
\end{hindi}}
\flushright{\begin{Arabic}
\quranayah[15][56]
\end{Arabic}}
\flushleft{\begin{hindi}
इबराहीम ने कहा गुमराहों के सिवा और ऐसा कौन है जो अपने परवरदिगार की रहमत से ना उम्मीद हो
\end{hindi}}
\flushright{\begin{Arabic}
\quranayah[15][57]
\end{Arabic}}
\flushleft{\begin{hindi}
(फिर) इबराहीम ने कहा ऐ (ख़ुदा के) भेजे हुए (फरिश्तों) तुम्हें आख़िर क्या मुहिम दर पेश है
\end{hindi}}
\flushright{\begin{Arabic}
\quranayah[15][58]
\end{Arabic}}
\flushleft{\begin{hindi}
उन्होंने कहा कि हम तो एक गुनाहगार क़ौम की तरफ (अज़ाब नाज़िल करने के लिए) भेजे गए हैं
\end{hindi}}
\flushright{\begin{Arabic}
\quranayah[15][59]
\end{Arabic}}
\flushleft{\begin{hindi}
मगर लूत के लड़के वाले कि हम उन सबको ज़रुर बचा लेगें मगर उनकी बीबी जिसे हमने ताक लिया है
\end{hindi}}
\flushright{\begin{Arabic}
\quranayah[15][60]
\end{Arabic}}
\flushleft{\begin{hindi}
कि वह ज़रुर (अपने लड़के बालों के) पीछे (अज़ाब में) रह जाएगी
\end{hindi}}
\flushright{\begin{Arabic}
\quranayah[15][61]
\end{Arabic}}
\flushleft{\begin{hindi}
ग़रज़ जब (ख़ुदा के) भेजे हुए (फरिश्ते) लूत के बाल बच्चों के पास आए तो लूत ने कहा कि तुम तो (कुछ) अजनबी लोग (मालूम होते हो)
\end{hindi}}
\flushright{\begin{Arabic}
\quranayah[15][62]
\end{Arabic}}
\flushleft{\begin{hindi}
फरिश्तों ने कहा (नहीं) बल्कि हम तो आपके पास वह (अज़ाब) लेकर आए हैं
\end{hindi}}
\flushright{\begin{Arabic}
\quranayah[15][63]
\end{Arabic}}
\flushleft{\begin{hindi}
जिसके बारे में आपकी क़ौम के लोग शक़ रखते थे
\end{hindi}}
\flushright{\begin{Arabic}
\quranayah[15][64]
\end{Arabic}}
\flushleft{\begin{hindi}
(कि आए न आए) और हम आप के पास (अज़ाब का) कलई (सही) हुक्म लेकर आए हैं और हम बिल्कुल सच कहते हैं
\end{hindi}}
\flushright{\begin{Arabic}
\quranayah[15][65]
\end{Arabic}}
\flushleft{\begin{hindi}
बस तो आप कुछ रात रहे अपने लड़के बालों को लेकर निकल जाइए और आप सब के सब पीछे रहिएगा और उन लोगों में से कोई मुड़कर पीछे न देखे और जिधर (जाने) का हुक्म दिया गया है (शाम) उधर (सीधे) चले जाओ और हमने लूत के पास इस अम्र का क़तई फैसला कहला भेजा
\end{hindi}}
\flushright{\begin{Arabic}
\quranayah[15][66]
\end{Arabic}}
\flushleft{\begin{hindi}
कि बस सुबह होते होते उन लोगों की जड़ काट डाली जाएगी
\end{hindi}}
\flushright{\begin{Arabic}
\quranayah[15][67]
\end{Arabic}}
\flushleft{\begin{hindi}
और (ये बात हो रही थीं कि) शहर के लोग (मेहमानों की ख़बर सुन कर बुरी नीयत से) खुशियाँ मनाते हुए आ पहुँचे
\end{hindi}}
\flushright{\begin{Arabic}
\quranayah[15][68]
\end{Arabic}}
\flushleft{\begin{hindi}
लूत ने (उनसे कहा) कि ये लोग मेरे मेहमान है तो तुम (इन्हें सताकर) मुझे रुसवा बदनाम न करो
\end{hindi}}
\flushright{\begin{Arabic}
\quranayah[15][69]
\end{Arabic}}
\flushleft{\begin{hindi}
और ख़ुदा से डरो और मुझे ज़लील न करो
\end{hindi}}
\flushright{\begin{Arabic}
\quranayah[15][70]
\end{Arabic}}
\flushleft{\begin{hindi}
वह लोग कहने लगे क्यों जी हमने तुम को सारे जहाँन के लोगों (के आने) की मनाही नहीं कर दी थी
\end{hindi}}
\flushright{\begin{Arabic}
\quranayah[15][71]
\end{Arabic}}
\flushleft{\begin{hindi}
लूत ने कहा अगर तुमको (ऐसा ही) करना है तो ये मेरी क़ौम की बेटियाँ मौजूद हैं
\end{hindi}}
\flushright{\begin{Arabic}
\quranayah[15][72]
\end{Arabic}}
\flushleft{\begin{hindi}
(इनसे निकाह कर लो) ऐ रसूल तुम्हारी जान की कसम ये लोग (क़ौम लूत) अपनी मस्ती में मदहोश हो रहे थे
\end{hindi}}
\flushright{\begin{Arabic}
\quranayah[15][73]
\end{Arabic}}
\flushleft{\begin{hindi}
(लूत की सुनते काहे को) ग़रज़ सूरज निकलते निकलते उनको (बड़े ज़ोरो की) चिघाड़ न ले डाला
\end{hindi}}
\flushright{\begin{Arabic}
\quranayah[15][74]
\end{Arabic}}
\flushleft{\begin{hindi}
फिर हमने उसी बस्ती को उलट कर उसके ऊपर के तबके क़ो नीचे का तबक़ा बना दिया और उसके ऊपर उन पर खरन्जे के पत्थर बरसा दिए इसमें शक़ नहीं कि इसमें (असली बात के) ताड़ जाने वालों के लिए (कुदरते ख़ुदा की) बहुत सी निशानियाँ हैं
\end{hindi}}
\flushright{\begin{Arabic}
\quranayah[15][75]
\end{Arabic}}
\flushleft{\begin{hindi}
और वह उलटी हुई बस्ती हमेशा (की आमदरफ्त)
\end{hindi}}
\flushright{\begin{Arabic}
\quranayah[15][76]
\end{Arabic}}
\flushleft{\begin{hindi}
के रास्ते पर है
\end{hindi}}
\flushright{\begin{Arabic}
\quranayah[15][77]
\end{Arabic}}
\flushleft{\begin{hindi}
इसमें तो शक हीं नहीं कि इसमें ईमानदारों के वास्ते (कुदरते ख़ुदा की) बहुत बड़ी निशानी है
\end{hindi}}
\flushright{\begin{Arabic}
\quranayah[15][78]
\end{Arabic}}
\flushleft{\begin{hindi}
और एैका के रहने वाले (क़ौमे शुएब की तरह बड़े सरकश थे)
\end{hindi}}
\flushright{\begin{Arabic}
\quranayah[15][79]
\end{Arabic}}
\flushleft{\begin{hindi}
तो उन से भी हमने (नाफरमानी का) बदला लिया और ये दो बस्तियाँ (क़ौमे लूत व शुएब की) एक खुली हुई यह राह पर (अभी तक मौजूद) हैं
\end{hindi}}
\flushright{\begin{Arabic}
\quranayah[15][80]
\end{Arabic}}
\flushleft{\begin{hindi}
और इसी तरह हिज्र के रहने वालों (क़ौम सालेह ने भी) पैग़म्बरों को झुठलाया
\end{hindi}}
\flushright{\begin{Arabic}
\quranayah[15][81]
\end{Arabic}}
\flushleft{\begin{hindi}
और (बावजूद कि) हमने उन्हें अपनी निशानियाँ दी उस पर भी वह लोग उनसे रद गिरदानी करते रहे
\end{hindi}}
\flushright{\begin{Arabic}
\quranayah[15][82]
\end{Arabic}}
\flushleft{\begin{hindi}
और बहुत दिल जोई से पहाड़ों को तराश कर घर बनाते रहे
\end{hindi}}
\flushright{\begin{Arabic}
\quranayah[15][83]
\end{Arabic}}
\flushleft{\begin{hindi}
आख़िर उनके सुबह होते होते एक बड़ी (जोरों की) चिंघाड़ ने ले डाला
\end{hindi}}
\flushright{\begin{Arabic}
\quranayah[15][84]
\end{Arabic}}
\flushleft{\begin{hindi}
फिर जो कुछ वह अपनी हिफाज़त की तदबीर किया करते थे (अज़ाब ख़ुदा से बचाने में) कि कुछ भी काम न आयीं
\end{hindi}}
\flushright{\begin{Arabic}
\quranayah[15][85]
\end{Arabic}}
\flushleft{\begin{hindi}
और हमने आसमानों और ज़मीन को और जो कुछ उन दोनों के दरमियान में है हिकमत व मसलहत से पैदा किया है और क़यामत यक़ीनन ज़रुर आने वाली है तो तुम (ऐ रसूल) उन काफिरों से शाइस्ता उनवान (अच्छे बरताव) के साथ दर गुज़र करो
\end{hindi}}
\flushright{\begin{Arabic}
\quranayah[15][86]
\end{Arabic}}
\flushleft{\begin{hindi}
इसमें शक़ नहीं कि तुम्हारा परवरदिगार बड़ा पैदा करने वाला है
\end{hindi}}
\flushright{\begin{Arabic}
\quranayah[15][87]
\end{Arabic}}
\flushleft{\begin{hindi}
(बड़ा दाना व बीना है) और हमने तुमको सबए मसानी (सूरे हम्द) और क़ुरान अज़ीम अता किया है
\end{hindi}}
\flushright{\begin{Arabic}
\quranayah[15][88]
\end{Arabic}}
\flushleft{\begin{hindi}
और हमने जो उन कुफ्फारों में से कुछ लोगों को (दुनिया की) माल व दौलत से निहाल कर दिया है तुम उसकी तरफ हरगिज़ नज़र भी न उठाना और न उनकी (बेदीनी) पर कुछ अफसोस करना और ईमानदारों से (अगरचे ग़रीब हो) झुककर मिला करो और कहा दो कि मै तो (अज़ाबे ख़ुदा से) सरीही तौर से डराने वाला हूँ
\end{hindi}}
\flushright{\begin{Arabic}
\quranayah[15][89]
\end{Arabic}}
\flushleft{\begin{hindi}
(ऐ रसूल) उन कुफ्फारों पर इस तरह अज़ाब नाज़िल करेगें जिस तरह हमने उन लोगों पर नाज़िल किया
\end{hindi}}
\flushright{\begin{Arabic}
\quranayah[15][90]
\end{Arabic}}
\flushleft{\begin{hindi}
जिन्होंने क़ुरान को बॉट कर टुकडे टुकड़े कर डाला
\end{hindi}}
\flushright{\begin{Arabic}
\quranayah[15][91]
\end{Arabic}}
\flushleft{\begin{hindi}
(बाज़ को माना बाज को नहीं) तो ऐ रसूल तुम्हारे ही परवरदिगार की (अपनी) क़सम
\end{hindi}}
\flushright{\begin{Arabic}
\quranayah[15][92]
\end{Arabic}}
\flushleft{\begin{hindi}
कि हम उनसे जो कुछ ये (दुनिया में) किया करते थे (बहुत सख्ती से) ज़रुर बाज़ पुर्स (पुछताछ) करेंगे
\end{hindi}}
\flushright{\begin{Arabic}
\quranayah[15][93]
\end{Arabic}}
\flushleft{\begin{hindi}
पस जिसका तुम्हें हुक्म दिया गया है उसे वाजेए करके सुना दो
\end{hindi}}
\flushright{\begin{Arabic}
\quranayah[15][94]
\end{Arabic}}
\flushleft{\begin{hindi}
और मुशरेकीन की तरफ से मुँह फेर लो
\end{hindi}}
\flushright{\begin{Arabic}
\quranayah[15][95]
\end{Arabic}}
\flushleft{\begin{hindi}
जो लोग तुम्हारी हँसी उड़ाते है
\end{hindi}}
\flushright{\begin{Arabic}
\quranayah[15][96]
\end{Arabic}}
\flushleft{\begin{hindi}
और ख़ुदा के साथ दूसरे परवरदिगार को (शरीक) ठहराते हैं हम तुम्हारी तरफ से उनके लिए काफी हैं तो अनक़रीब ही उन्हें मालूम हो जाएगा
\end{hindi}}
\flushright{\begin{Arabic}
\quranayah[15][97]
\end{Arabic}}
\flushleft{\begin{hindi}
कि तुम जो इन (कुफ्फारों मुनाफिक़ीन) की बातों से दिल तंग होते हो उसको हम ज़रुर जानते हैं
\end{hindi}}
\flushright{\begin{Arabic}
\quranayah[15][98]
\end{Arabic}}
\flushleft{\begin{hindi}
तो तुम अपने परवरदिगार की हम्दो सना से उसकी तस्बीह करो और (उसकी बारगाह में) सजदा करने वालों में हो जाओ
\end{hindi}}
\flushright{\begin{Arabic}
\quranayah[15][99]
\end{Arabic}}
\flushleft{\begin{hindi}
और जब तक तुम्हारे पास मौत आए अपने परवरदिगार की इबादत में लगे रहो
\end{hindi}}
\chapter{An-Nahl (The Bee)}
\begin{Arabic}
\Huge{\centerline{\basmalah}}\end{Arabic}
\flushright{\begin{Arabic}
\quranayah[16][1]
\end{Arabic}}
\flushleft{\begin{hindi}
ऐ कुफ्फ़ारे मक्का (ख़ुदा का हुक्म (क़यामत गोया) आ पहुँचा तो (ऐ काफिरों बे फायदे) तुम इसकी जल्दी न मचाओ जिस चीज़ को ये लोग शरीक क़रार देते हैं उससे वह ख़ुदा पाक व पाकीज़ा और बरतर है
\end{hindi}}
\flushright{\begin{Arabic}
\quranayah[16][2]
\end{Arabic}}
\flushleft{\begin{hindi}
वही अपने हुक्म से अपने बन्दों में से जिसके पास चाहता है 'वहीं' देकर फ़रिश्तों को भेजता है कि लोगों को इस बात से आगाह कर दें कि मेरे सिवा कोई माबूद नहीं तो मुझी से डरते रहो
\end{hindi}}
\flushright{\begin{Arabic}
\quranayah[16][3]
\end{Arabic}}
\flushleft{\begin{hindi}
उसी ने सारे आसमान और ज़मीन मसलहत व हिकमत से पैदा किए तो ये लोग जिसको उसका यशरीक बनाते हैं उससे कहीं बरतर है
\end{hindi}}
\flushright{\begin{Arabic}
\quranayah[16][4]
\end{Arabic}}
\flushleft{\begin{hindi}
उसने इन्सान को नुत्फे से पैदा किया फिर वह यकायक (हम ही से) खुल्लम खुल्ला झगड़ने वाला हो गया
\end{hindi}}
\flushright{\begin{Arabic}
\quranayah[16][5]
\end{Arabic}}
\flushleft{\begin{hindi}
उसी ने चरपायों को भी पैदा किया कि तुम्हारे लिए ऊन (ऊन की खाल और ऊन) से जाडे क़ा सामान है
\end{hindi}}
\flushright{\begin{Arabic}
\quranayah[16][6]
\end{Arabic}}
\flushleft{\begin{hindi}
इसके अलावा और भी फायदें हैं और उनमें से बाज़ को तुम खाते हो और जब तुम उन्हें सिरे शाम चराई पर से लाते हो जब सवेरे ही सवेरे चराई पर ले जाते हो
\end{hindi}}
\flushright{\begin{Arabic}
\quranayah[16][7]
\end{Arabic}}
\flushleft{\begin{hindi}
तो उनकी वजह से तुम्हारी रौनक़ भी है और जिन शहरों तक बग़ैर बड़ी जान ज़ोख़म में डाले बगैर के पहुँच न सकते थे वहाँ तक ये चौपाए भी तुम्हारे बोझे भी उठा लिए फिरते हैं इसमें शक़ नहीं कि तुम्हारा परवरदिगार बड़ा शफीक़ मेहरबान है
\end{hindi}}
\flushright{\begin{Arabic}
\quranayah[16][8]
\end{Arabic}}
\flushleft{\begin{hindi}
और (उसी ने) घोड़ों ख़च्चरों और गधों को (पैदा किया) ताकि तुम उन पर सवार हो और (इसमें) ज़ीनत (भी) है
\end{hindi}}
\flushright{\begin{Arabic}
\quranayah[16][9]
\end{Arabic}}
\flushleft{\begin{hindi}
(उसके अलावा) और चीज़े भी पैदा करेगा जिनको तुम नहीं जानते हो और (ख़ुश्क व तर में) सीधी राह (की हिदायत तो ख़ुदा ही के ज़िम्मे हैं और बाज़ रस्ते टेढे हैं
\end{hindi}}
\flushright{\begin{Arabic}
\quranayah[16][10]
\end{Arabic}}
\flushleft{\begin{hindi}
और अगर ख़ुदा चाहता तो तुम सबको मज़िले मक़सूद तक पहुँचा देता वह वही (ख़ुदा) है जिसने आसमान से पानी बरसाया जिसमें से तुम सब पीते हो और इससे दरख्त शादाब होते हैं
\end{hindi}}
\flushright{\begin{Arabic}
\quranayah[16][11]
\end{Arabic}}
\flushleft{\begin{hindi}
जिनमें तुम (अपने मवेशियों को) चराते हो इसी पानी से तुम्हारे वास्ते खेती और जैतून और खुरमें और अंगूर उगाता है और हर तरह के फल (पैदा करता है) इसमें शक़ नहीं कि इसमें ग़ौर करने वालो के वास्ते (कुदरते ख़ुदा की) बहुत बड़ी निशानी है
\end{hindi}}
\flushright{\begin{Arabic}
\quranayah[16][12]
\end{Arabic}}
\flushleft{\begin{hindi}
उसी ने तुम्हारे वास्ते रात को और दिन को और सूरज और चाँद को तुम्हारा ताबेए बना दिया है और सितारे भी उसी के हुक्म से (तुम्हारे) फरमाबरदार हैं कुछ शक़ ही नहीं कि (इसमें) समझदार लोगों के वास्ते यक़ीनन (कुदरत खुदा की) बहुत सी निशानियाँ हैं
\end{hindi}}
\flushright{\begin{Arabic}
\quranayah[16][13]
\end{Arabic}}
\flushleft{\begin{hindi}
और जो तरह तरह के रंगों की चीज़े उसमें ज़मीन में तुम्हारे नफे के वास्ते पैदा की
\end{hindi}}
\flushright{\begin{Arabic}
\quranayah[16][14]
\end{Arabic}}
\flushleft{\begin{hindi}
कुछ शक़ नहीं कि इसमें भी इबरत व नसीहत हासिल करने वालों के वास्ते (कुदरते ख़ुदा की) बहुत सी निशानी है और वही (वह ख़ुदा है जिसने दरिया को (भी तुम्हारे) क़ब्ज़े में कर दिया ताकि तुम इसमें से (मछलियों का) ताज़ा ताज़ा गोश्त खाओ और इसमें से जेवर (की चीज़े मोती वगैरह) निकालो जिन को तुम पहना करते हो और तू कश्तियों को देखता है कि (आमद व रफत में) दरिया में (पानी को) चीरती फाड़ती आती है
\end{hindi}}
\flushright{\begin{Arabic}
\quranayah[16][15]
\end{Arabic}}
\flushleft{\begin{hindi}
और (दरिया को तुम्हारे ताबेए) इसलिए (कर दिया) कि तुम लोग उसके फज़ल (नफा तिजारत) की तलाश करो और ताकि तुम शुक्र करो और उसी ने ज़मीन पर (भारी भारी) पहाड़ों को गाड़ दिया
\end{hindi}}
\flushright{\begin{Arabic}
\quranayah[16][16]
\end{Arabic}}
\flushleft{\begin{hindi}
ताकि (ऐसा न हों) कि ज़मीन तुम्हें लेकर झुक जाए (और तुम्हारे क़दम न जमें) और (उसी ने) नदियाँ और रास्ते (बनाए)
\end{hindi}}
\flushright{\begin{Arabic}
\quranayah[16][17]
\end{Arabic}}
\flushleft{\begin{hindi}
ताकि तुम (अपनी अपनी मंज़िले मक़सूद तक पहुँचों (उसके अलावा रास्तों में) और बहुत सी निशानियाँ (पैदा की हैं) और बहुत से लोग सितारे से भी राह मालूम करते हैं
\end{hindi}}
\flushright{\begin{Arabic}
\quranayah[16][18]
\end{Arabic}}
\flushleft{\begin{hindi}
तो क्या जो (ख़ुदा इतने मख़लूकात को) पैदा करता है वह उन (बुतों के बराबर हो सकता है जो कुछ भी पैदा नहीं कर सकते तो क्या तुम (इतनी बात भी) नहीं समझते और अगर तुम ख़ुदा की नेअमतों को गिनना चाहो तो (इस कसरत से हैं कि) तुम नहीं गिन सकते हो
\end{hindi}}
\flushright{\begin{Arabic}
\quranayah[16][19]
\end{Arabic}}
\flushleft{\begin{hindi}
बेशक ख़ुदा बड़ा बख्शने वाला मेहरबान है कि (तुम्हारी नाफरमानी पर भी नेअमत देता है)
\end{hindi}}
\flushright{\begin{Arabic}
\quranayah[16][20]
\end{Arabic}}
\flushleft{\begin{hindi}
और जो कुछ तुम छिपाकर करते हो और जो कुछ ज़ाहिर करते हो (ग़रज़) ख़ुदा (सब कुछ) जानता है और (ये कुफ्फार) ख़ुदा को छोड़कर जिन लोगों को (हाजत के वक्त) पुकारते हैं वह कुछ भी पैदा नहीं कर सकते
\end{hindi}}
\flushright{\begin{Arabic}
\quranayah[16][21]
\end{Arabic}}
\flushleft{\begin{hindi}
(बल्कि) वह ख़ुद दूसरों के बनाए हुए मुर्दे बेजान हैं और इतनी भी ख़बर नहीं कि कब (क़यामत) होगी और कब मुर्दे उठाए जाएगें
\end{hindi}}
\flushright{\begin{Arabic}
\quranayah[16][22]
\end{Arabic}}
\flushleft{\begin{hindi}
(फिर क्या काम आएंगीं)तुम्हारा परवरदिगार यकता खुदा है तो जो लोग आख़िरत पर ईमान नहीं रखते उनके दिल ही (इस वजह के हैं कि हर बात का) इन्कार करते हैं और वह बड़े मग़रुर हैं
\end{hindi}}
\flushright{\begin{Arabic}
\quranayah[16][23]
\end{Arabic}}
\flushleft{\begin{hindi}
ये लोग जो कुछ छिपा कर करते हैं और जो कुछ ज़ाहिर बज़ाहिर करते हैं (ग़रज़ सब कुछ) ख़ुदा ज़रुर जानता है वह हरगिज़ तकब्बुर करने वालों को पसन्द नहीं करता
\end{hindi}}
\flushright{\begin{Arabic}
\quranayah[16][24]
\end{Arabic}}
\flushleft{\begin{hindi}
और जब उनसे कहा जाता है कि तुम्हारे परवरदिगार ने क्या नाज़िल किया है तो वह कहते हैं कि (अजी कुछ भी नहीं बस) अगलो के किस्से हैं
\end{hindi}}
\flushright{\begin{Arabic}
\quranayah[16][25]
\end{Arabic}}
\flushleft{\begin{hindi}
(उनको बकने दो ताकि क़यामत के दिन) अपने (गुनाहों) के पूरे बोझ और जिन लोगों को उन्होंने बे समझे बूझे गुमराह किया है उनके (गुनाहों के) बोझ भी उन्हीं को उठाने पड़ेगें ज़रा देखो तो कि ये लोग कैसा बुरा बोझ अपने ऊपर लादे चले जा रहें हैं
\end{hindi}}
\flushright{\begin{Arabic}
\quranayah[16][26]
\end{Arabic}}
\flushleft{\begin{hindi}
बेशक जो लोग उनसे पहले थे उन्होंने भी मक्कारियाँ की थीं तो (ख़ुदा का हुक्म) उनके ख्यालात की इमारत की जड़ की तरफ से आ पड़ा (पस फिर क्या था) इस ख्याली इमारत की छत उन पर उनके ऊपर से धम से गिर पड़ी (और सब ख्याल हवा हो गए) और जिधर से उन पर अज़ाब आ पहुँचा उसकी उनको ख़बर तक न थी
\end{hindi}}
\flushright{\begin{Arabic}
\quranayah[16][27]
\end{Arabic}}
\flushleft{\begin{hindi}
(फिर उसी पर इकतिफा नहीं) उसके बाद क़यामत के दिन ख़ुदा उनको रुसवा करेगा और फरमाएगा कि (अब बताओ) जिसको तुमने मेरा यशरीक बना रखा था और जिनके बारे में तुम (ईमानदारों से) झगड़ते थे कहाँ हैं (वह तो कुछ जवाब देगें नहीं मगर) जिन लोगों को (ख़ुदा की तरफ से) इल्म दिया गया है कहेगें कि आज के दिन रुसवाई और ख़राबी (सब कुछ) काफिरों पर ही है
\end{hindi}}
\flushright{\begin{Arabic}
\quranayah[16][28]
\end{Arabic}}
\flushleft{\begin{hindi}
वह लोग हैं कि जब फरिश्ते उनकी रुह क़ब्ज़ करने लगते हैं (और) ये लोग (कुफ्र करके) आप अपने ऊपर सितम ढ़ाते रहे तो इताअत पर आमादा नज़र आते हैं और (कहते हैं कि) हम तो (अपने ख्याल में) कोई बुराई नहीं करते थे (तो फरिश्ते कहते हैं) हाँ जो कुछ तुम्हारी करतूते थी ख़ुदा उससे खूब अच्छी तरह वाक़िफ हैं
\end{hindi}}
\flushright{\begin{Arabic}
\quranayah[16][29]
\end{Arabic}}
\flushleft{\begin{hindi}
(अच्छा तो लो) जहन्नुम के दरवाज़ों में दाख़िल हो और इसमें हमेशा रहोगे ग़रज़ तकब्बुर करने वालो का भी क्या बुरा ठिकाना है
\end{hindi}}
\flushright{\begin{Arabic}
\quranayah[16][30]
\end{Arabic}}
\flushleft{\begin{hindi}
और जब परहेज़गारों से पूछा जाता है कि तुम्हारे परवरदिगार ने क्या नाज़िल किया है तो बोल उठते हैं सब अच्छे से अच्छा जिन लोगों ने नेकी की उनके लिए इस दुनिया में (भी) भलाई (ही भलाई) है और आख़िरत का घर क्या उम्दा है
\end{hindi}}
\flushright{\begin{Arabic}
\quranayah[16][31]
\end{Arabic}}
\flushleft{\begin{hindi}
सदा बहार (हरे भरे) बाग़ हैं जिनमें (वे तकल्लुफ) जा पहुँचेगें उनके नीचे नहरें जारी होंगी और ये लोग जो चाहेगें उनके लिए मुयय्या (मौजूद) है यूँ ख़ुदा परहेज़गारों (को उनके किए) की जज़ा अता फरमाता है
\end{hindi}}
\flushright{\begin{Arabic}
\quranayah[16][32]
\end{Arabic}}
\flushleft{\begin{hindi}
(ये) वह लोग हैं जिनकी रुहें फरिश्ते इस हालत में क़ब्ज़ करतें हैं कि वह (नजासते कुफ्र से) पाक व पाकीज़ा होते हैं तो फरिश्ते उनसे (निहायत तपाक से) कहते है सलामुन अलैकुम जो नेकियाँ दुनिया में तुम करते थे उसके सिले में जन्नत में (बेतकल्लुफ) चले जाओ
\end{hindi}}
\flushright{\begin{Arabic}
\quranayah[16][33]
\end{Arabic}}
\flushleft{\begin{hindi}
(ऐ रसूल) क्या ये (अहले मक्का) इसी बात के मुन्तिज़र है कि उनके पास फरिश्ते (क़ब्ज़े रुह को) आ ही जाएँ या (उनके हलाक करने को) तुम्हारे परवरदिगार का अज़ाब ही आ पहुँचे जो लोग उनसे पहले हो गुज़रे हैं वह ऐसी बातें कर चुके हैं और ख़ुदा ने उन पर (ज़रा भी) जुल्म नहीं किया बल्कि वह लोग ख़ुद कुफ़्र की वजह से अपने ऊपर आप जुल्म करते रहे
\end{hindi}}
\flushright{\begin{Arabic}
\quranayah[16][34]
\end{Arabic}}
\flushleft{\begin{hindi}
फिर जो जो करतूतें उनकी थी उसकी सज़ा में बुरे नतीजे उनको मिले और जिस (अज़ाब) की वह हँसी उड़ाया करते थे उसने उन्हें (चारो तरफ से) घेर लिया
\end{hindi}}
\flushright{\begin{Arabic}
\quranayah[16][35]
\end{Arabic}}
\flushleft{\begin{hindi}
और मुशरेकीन कहते हैं कि अगर ख़ुदा चाहता तो न हम ही उसके सिवा किसी और चीज़ की इबादत करते और न हमारे बाप दादा और न हम बग़ैर उस (की मर्ज़ी) के किसी चीज़ को हराम कर बैठते जो लोग इनसे पहले हो गुज़रे हैं वह भी ऐसे (हीला हवाले की) बातें कर चुके हैं तो (कहा करें) पैग़म्बरों पर तो उसके सिवा कि एहकाम को साफ साफ पहुँचा दे और कुछ भी नहीं
\end{hindi}}
\flushright{\begin{Arabic}
\quranayah[16][36]
\end{Arabic}}
\flushleft{\begin{hindi}
और हमने तो हर उम्मत में एक (न एक) रसूल इस बात के लिए ज़रुर भेजा कि लोगों ख़ुदा की इबादत करो और बुतों (की इबादत) से बचे रहो ग़रज़ उनमें से बाज़ की तो ख़ुदा ने हिदायत की और बाज़ के (सर) पर गुमराही सवार हो गई तो ज़रा तुम लोग रुए ज़मीन पर चल फिर कर देखो तो कि (पैग़म्बराने ख़ुदा के) झुठलाने वालों को क्या अन्जाम हुआ
\end{hindi}}
\flushright{\begin{Arabic}
\quranayah[16][37]
\end{Arabic}}
\flushleft{\begin{hindi}
(ऐ रसूल) अगर तुमको इन लोगों के राहे रास्त पर जाने का हौका है (तो बे फायदा) क्योंकि ख़ुदा तो हरगिज़ उस शख़्श की हिदायत नहीं करेगा जिसको (नाज़िल होने की वजह से) गुमराही में छोड़ देता है और न उनका कोई मददगार है
\end{hindi}}
\flushright{\begin{Arabic}
\quranayah[16][38]
\end{Arabic}}
\flushleft{\begin{hindi}
(कि अज़ाब से बचाए) और ये कुफ्फार ख़ुदा की जितनी क़समें उनके इमकान में तुम्हें खा (कर कहते) हैं कि जो शख़्श मर जाता है फिर उसको ख़ुदा दोबारा ज़िन्दा नहीं करेगा (ऐ रसूल) तुम कह दो कि हाँ ज़रुर ऐसा करेगा इस पर अपने वायदे की (वफा) लाज़िम व ज़रुरी है मगर बहुतेरे आदमी नहीं जानते हैं
\end{hindi}}
\flushright{\begin{Arabic}
\quranayah[16][39]
\end{Arabic}}
\flushleft{\begin{hindi}
(दोबारा ज़िन्दा करना इसलिए ज़रुरी है) कि जिन बातों पर ये लोग झगड़ा करते हैं उन्हें उनके सामने साफ वाज़ेए कर देगा और ताकि कुफ्फार ये समझ लें कि ये लोग (दुनिया में) झूठे थे
\end{hindi}}
\flushright{\begin{Arabic}
\quranayah[16][40]
\end{Arabic}}
\flushleft{\begin{hindi}
हम जब किसी चीज़ (के पैदा करने) का इरादा करते हैं तो हमारा कहना उसके बारे में इतना ही होता है कि हम कह देते हैं कि 'हो जा' बस फौरन हो जाती है (तो फिर मुर्दों का जिलाना भी कोई बात है)
\end{hindi}}
\flushright{\begin{Arabic}
\quranayah[16][41]
\end{Arabic}}
\flushleft{\begin{hindi}
और जिन लोगों ने (कुफ्फ़ार के ज़ुल्म पर ज़ुल्म सहने के बाद ख़ुदा की खुशी के लिए घर बार छोड़ा हिजरत की हम उनको ज़रुर दुनिया में भी अच्छी जगह बिठाएँगें और आख़िरत की जज़ा तो उससे कहीं बढ़ कर है काश ये लोग जिन्होंने ख़ुदा की राह में सख्तियों पर सब्र किया
\end{hindi}}
\flushright{\begin{Arabic}
\quranayah[16][42]
\end{Arabic}}
\flushleft{\begin{hindi}
और अपने परवरदिगार ही पर भरोसा रखते हैं (आख़िरत का सवाब) जानते होते
\end{hindi}}
\flushright{\begin{Arabic}
\quranayah[16][43]
\end{Arabic}}
\flushleft{\begin{hindi}
और (ऐ रसूल) तुम से पहले आदमियों ही को पैग़म्बर बना बनाकर भेजा किए जिन की तरफ हम वहीं भेजते थे तो (तुम अहले मक्का से कहो कि) अगर तुम खुद नहीं जानते हो तो अहले ज़िक्र (आलिमों से) पूछो (और उन पैग़म्बरों को भेजा भी तो) रौशन दलीलों और किताबों के साथ
\end{hindi}}
\flushright{\begin{Arabic}
\quranayah[16][44]
\end{Arabic}}
\flushleft{\begin{hindi}
और तुम्हारे पास क़ुरान नाज़िल किया है ताकि जो एहकाम लोगों के लिए नाज़िल किए गए है तुम उनसे साफ साफ बयान कर दो ताकि वह लोग खुद से कुछ ग़ौर फिक्र करें
\end{hindi}}
\flushright{\begin{Arabic}
\quranayah[16][45]
\end{Arabic}}
\flushleft{\begin{hindi}
तो क्या जो लोग बड़ी बड़ी मक्कारियाँ (शिर्क वग़ैरह) करते थे (उनको इस बात का इत्मिनान हो गया है (और मुत्तलिक़ ख़ौफ नहीं) कि ख़ुदा उन्हें ज़मीन में धसा दे या ऐसी तरफ से उन पर अज़ाब आ पहुँचे कि उसकी उनको ख़बर भी न हो
\end{hindi}}
\flushright{\begin{Arabic}
\quranayah[16][46]
\end{Arabic}}
\flushleft{\begin{hindi}
उनके चलते फिरते (ख़ुदा का अज़ाब) उन्हें गिरफ्तार करे तो वह लोग उसे ज़ेर नहीं कर सकते
\end{hindi}}
\flushright{\begin{Arabic}
\quranayah[16][47]
\end{Arabic}}
\flushleft{\begin{hindi}
या वह अज़ाब से डरते हो तो (उसी हालत में) धर पकड़ करे इसमें तो शक नहीं कि तुम्हारा परवरदिगार बड़ा शफीक़ रहम वाला है
\end{hindi}}
\flushright{\begin{Arabic}
\quranayah[16][48]
\end{Arabic}}
\flushleft{\begin{hindi}
क्या उन लोगों ने ख़ुदा की मख़लूक़ात में से कोई ऐसी चीज़ नहीं देखी जिसका साया (कभी) दाहिनी तरफ और कभी बायीं तरफ पलटा रहता है कि (गोया) ख़ुदा के सामने सर सजदा है और सब इताअत का इज़हार करते हैं
\end{hindi}}
\flushright{\begin{Arabic}
\quranayah[16][49]
\end{Arabic}}
\flushleft{\begin{hindi}
और जितनी चीज़ें (चादँ सूरज वग़ैरह) आसमानों में हैं और जितने जानवर ज़मीन में हैं सब ख़ुदा ही के आगे सर सजूद हैं और फरिश्ते तो (है ही) और वह हुक्में ख़ुदा से सरकशी नहीं करते (49) (सजदा)
\end{hindi}}
\flushright{\begin{Arabic}
\quranayah[16][50]
\end{Arabic}}
\flushleft{\begin{hindi}
और अपने परवरदिगार से जो उनसे (कहीं) बरतर (बड़ा) व आला है डरते हैं और जो हुक्में दिया जाता है फौरन बजा लाते हैं
\end{hindi}}
\flushright{\begin{Arabic}
\quranayah[16][51]
\end{Arabic}}
\flushleft{\begin{hindi}
और ख़ुदा ने फरमाया था कि दो दो माबूद न बनाओ माबूद तो बस वही यकता ख़ुदा है सिर्फ मुझी से डरते रहो
\end{hindi}}
\flushright{\begin{Arabic}
\quranayah[16][52]
\end{Arabic}}
\flushleft{\begin{hindi}
और जो कुछ आसमानों में हैं और जो कुछ ज़मीन में हैं (ग़रज़) सब कुछ उसी का है और ख़ालिस फरमाबरदारी हमेशा उसी को लाज़िम (जरूरी) है तो क्या तुम लोग ख़ुदा के सिवा (किसी और से भी) डरते हो
\end{hindi}}
\flushright{\begin{Arabic}
\quranayah[16][53]
\end{Arabic}}
\flushleft{\begin{hindi}
और जितनी नेअमतें तुम्हारे साथ हैं (सब ही की तरफ से हैं) फिर जब तुमको तकलीफ छू भी गई तो तुम उसी के आगे फरियाद करने लगते हो
\end{hindi}}
\flushright{\begin{Arabic}
\quranayah[16][54]
\end{Arabic}}
\flushleft{\begin{hindi}
फिर जब वह तुमसे तकलीफ को दूर कर देता है तो बस फौरन तुममें से कुछ लोग अपने परवरदिगार को शरीक ठहराते हैं
\end{hindi}}
\flushright{\begin{Arabic}
\quranayah[16][55]
\end{Arabic}}
\flushleft{\begin{hindi}
ताकि जो (नेअमतें) हमने उनको दी है उनकी नाशुक्री करें तो (ख़ैर दुनिया में चन्द रोज़ चैन कर लो फिर तो अनक़रीब तुमको मालूम हो जाएगा
\end{hindi}}
\flushright{\begin{Arabic}
\quranayah[16][56]
\end{Arabic}}
\flushleft{\begin{hindi}
और हमने जो रोज़ी उनको दी है उसमें से ये लोग उन बुतों का हिस्सा भी क़रार देते है जिनकी हक़ीकत नहीं जानते तो ख़ुदा की (अपनी) क़िस्म जो इफ़तेरा परदाज़ियाँ तुम करते थे (क़यामत में) उनकी बाज़पुर्स (पूछ गछ) तुम से ज़रुर की जाएगी
\end{hindi}}
\flushright{\begin{Arabic}
\quranayah[16][57]
\end{Arabic}}
\flushleft{\begin{hindi}
और ये लोग ख़ुदा के लिए बेटियाँ तजवीज़ करते हैं (सुबान अल्लाह) वह उस से पाक व पाकीज़ा है
\end{hindi}}
\flushright{\begin{Arabic}
\quranayah[16][58]
\end{Arabic}}
\flushleft{\begin{hindi}
और अपने लिए (बेटे) जो मरग़ूब (दिल पसन्द) हैं और जब उसमें से किसी एक को लड़की पैदा होने की जो खुशख़बरी दीजिए रंज के मारे मुँह काला हो जाता है
\end{hindi}}
\flushright{\begin{Arabic}
\quranayah[16][59]
\end{Arabic}}
\flushleft{\begin{hindi}
और वह ज़हर का सा घूँट पीकर रह जाता है (बेटी की) जिसकी खुशखबरी दी गई है अपनी क़ौम के लोगों से छिपा फिरता है (और सोचता है) कि क्या इसको ज़िल्लत उठाके ज़िन्दा रहने दे या (ज़िन्दा ही) इसको ज़मीन में गाड़ दे-देखो तुम लोग किस क़दर बुरा एहकाम (हुक्म) लगाते हैं
\end{hindi}}
\flushright{\begin{Arabic}
\quranayah[16][60]
\end{Arabic}}
\flushleft{\begin{hindi}
बुरी (बुरी) बातें तो उन्हीं लोगों के लिए ज्यादा मुनासिब हैं जो आख़िरत का यक़ीन नहीं रखते और ख़ुदा की शान के लायक़ तो आला सिफते (बहुत बड़ी अच्छाइया) हैं और वही तो ग़ालिब है
\end{hindi}}
\flushright{\begin{Arabic}
\quranayah[16][61]
\end{Arabic}}
\flushleft{\begin{hindi}
और अगर (कहीं) ख़ुदा अपने बन्दों की नाफरमानियों की गिरफ़त करता तो रुए ज़मीन पर किसी एक जानदार को बाक़ी न छोड़ता मगर वह तो एक मुक़र्रर वक्त तक उन सबको मोहलत देता है फिर जब उनका (वह) वक्त अा पहुँचेगा तो न एक घड़ी पीछे हट सकते हैं और न आगे बढ़ सकते हैं
\end{hindi}}
\flushright{\begin{Arabic}
\quranayah[16][62]
\end{Arabic}}
\flushleft{\begin{hindi}
और ये लोग खुद जिन बातों को पसन्द नहीं करते उनको ख़ुदा के वास्ते क़रार देते हैं और अपनी ही ज़बान से ये झूठे दावे भी करते हैं कि (आख़िरत में भी) उन्हीं के लिए भलाई है (भलाई तो नहीं मगर) हाँ उनके लिए जहन्नुम की आग ज़रुरी है और यही लोग सबसे पहले (उसमें) झोंके जाएँगें
\end{hindi}}
\flushright{\begin{Arabic}
\quranayah[16][63]
\end{Arabic}}
\flushleft{\begin{hindi}
(ऐ रसूल) ख़ुदा की (अपनी) कसम तुमसे पहले उम्मतों के पास बहुतेरे पैग़म्बर भेजे तो शैतान ने उनकी कारस्तानियों को उम्दा कर दिखाया तो वही (शैतान) आज भी उन लोगों का सरपरस्त बना हुआ है हालॉकि उनके वास्ते दर्दनाक अज़ाब है
\end{hindi}}
\flushright{\begin{Arabic}
\quranayah[16][64]
\end{Arabic}}
\flushleft{\begin{hindi}
और हमने तुम पर किताब (क़ुरान) तो इसी लिए नाज़िल की ताकि जिन बातों में ये लोग बाहम झगड़ा किए हैं उनको तुम साफ साफ बयान करो (और यह किताब) ईमानदारों के लिए तो (अज़सरतापा) हिदायत और रहमत है
\end{hindi}}
\flushright{\begin{Arabic}
\quranayah[16][65]
\end{Arabic}}
\flushleft{\begin{hindi}
और ख़ुदा ही ने आसमान से पानी बरसाया तो उसके ज़रिए ज़मीन को मुर्दा होने के बाद ज़िन्दा (शादाब) (हरी भरी) किया क्या कुछ शक नहीं कि इसमें जो लोग बसते हैं उनके वास्ते (कुदरते ख़ुदा की) बहुत बड़ी निशानी है
\end{hindi}}
\flushright{\begin{Arabic}
\quranayah[16][66]
\end{Arabic}}
\flushleft{\begin{hindi}
और इसमें शक़ नहीं कि चौपायों में भी तुम्हारे लिए (इबरत की बात) है कि उनके पेट में ख़ाक, बला, गोबर और ख़ून (जो कुछ भरा है) उसमें से हमे तुमको ख़ालिस दूध पिलाते हैं जो पीने वालों के लिए खुशगवार है
\end{hindi}}
\flushright{\begin{Arabic}
\quranayah[16][67]
\end{Arabic}}
\flushleft{\begin{hindi}
और इसी तरह खुरमें और अंगूर के फल से (हम तुमको शीरा पिलाते हैं) जिसकी (कभी तो) तुम शराब बना लिया करते हो और कभी अच्छी रोज़ी (सिरका वगैरह) इसमें शक नहीं कि इसमें भी समझदार लोगों के लिए (क़ुदरत ख़ुदा की) बड़ी निशानी है
\end{hindi}}
\flushright{\begin{Arabic}
\quranayah[16][68]
\end{Arabic}}
\flushleft{\begin{hindi}
और (ऐ रसूल) तुम्हारे परवरदिगार ने शहद की मक्खियों के दिल में ये बात डाली कि तू पहाड़ों और दरख्तों और ऊँची ऊँची ट्ट्टियाँ (और मकानात पाट कर) बनाते हैं
\end{hindi}}
\flushright{\begin{Arabic}
\quranayah[16][69]
\end{Arabic}}
\flushleft{\begin{hindi}
उनमें अपने छत्ते बना फिर हर तरह के फलों (के पूर से) (उनका अर्क़) चूस कर फिर अपने परवरदिगार की राहों में ताबेदारी के साथ चली मक्खियों के पेट से पीने की एक चीज़ निकलती है (शहद) जिसके मुख्तलिफ रंग होते हैं इसमें लोगों (के बीमारियों) की शिफ़ा (भी) है इसमें शक़ नहीं कि इसमें ग़ौर व फ़िक्र करने वालों के वास्ते (क़ुदरते ख़ुदा की बहुत बड़ी निशानी है)
\end{hindi}}
\flushright{\begin{Arabic}
\quranayah[16][70]
\end{Arabic}}
\flushleft{\begin{hindi}
और ख़ुदा ही ने तुमको पैदा किया फिर वही तुमको (दुनिया से) उठा लेगा और तुममें से बाज़ ऐसे भी हैं जो ज़लील ज़िन्दगी (सख्त बुढ़ापे) की तरफ लौटाए जाते हैं (बहुत कुछ) जानने के बाद (ऐसे सड़ीले हो गए कि) कुछ नहीं जान सके बेशक ख़ुदा (सब कुछ) जानता और (हर तरह की) कुदरत रखता है
\end{hindi}}
\flushright{\begin{Arabic}
\quranayah[16][71]
\end{Arabic}}
\flushleft{\begin{hindi}
और ख़ुदा ही ने तुममें से बाज़ को बाज़ पर रिज़क (दौलत में) तरजीह दी है फिर जिन लोगों को रोज़ी ज्यादा दी गई है वह लोग अपनी रोज़ी में से उन लोगों को जिन पर उनका दस्तरस (इख्तेयार) है (लौन्डी ग़ुलाम वग़ैरह) देने वाला नहीं (हालॉकि इस माल में तो सब के सब मालिक गुलाम वग़ैरह) बराबर हैं तो क्या ये लोग ख़ुदा की नेअमत के मुन्किर हैं
\end{hindi}}
\flushright{\begin{Arabic}
\quranayah[16][72]
\end{Arabic}}
\flushleft{\begin{hindi}
और ख़ुदा ही ने तुम्हारे लिए बीबियाँ तुम ही में से बनाई और उसी ने तुम्हारे वास्ते तुम्हारी बीवियों से बेटे और पोते पैदा किए और तुम्हें पाक व पाकीज़ा रोज़ी खाने को दी तो क्या ये लोग बिल्कुल बातिल (सुल्तान बुत वग़ैरह) पर ईमान रखते हैं और ख़ुदा की नेअमत (वहदानियत वग़ैरह से) इन्कार करते हैं
\end{hindi}}
\flushright{\begin{Arabic}
\quranayah[16][73]
\end{Arabic}}
\flushleft{\begin{hindi}
और ख़ुदा को छोड़कर ऐसी चीज़ की परसतिश करते हैं जो आसमानों और ज़मीन में से उनको रिज़क़ देने का कुछ भी न एख्तियार रखते हैं और न कुदरत हासिल कर सकते हैं
\end{hindi}}
\flushright{\begin{Arabic}
\quranayah[16][74]
\end{Arabic}}
\flushleft{\begin{hindi}
तो तुम (दुनिया के लोगों पर कयास करके ख़ुद) मिसालें न गढ़ो बेशक ख़ुदा (सब कुछ) जानता है और तुम नहीं जानते
\end{hindi}}
\flushright{\begin{Arabic}
\quranayah[16][75]
\end{Arabic}}
\flushleft{\begin{hindi}
एक मसल ख़ुदा ने बयान फरमाई कि एक ग़ुलाम है जो दूसरे के कब्जे में है (और) कुछ भी एख्तियार नहीं रखता और एक शख़्श वह है कि हमने उसे अपनी बारगाह से अच्छी (खासी) रोज़ी दे रखी है तो वह उसमें से छिपा के दिखा के (ग़रज़ हर तरह ख़ुदा की राह में) ख़र्च करता है क्या ये दोनो बराबर हो सकते हैं (हरगिज़ नहीं) अल्हमदोलिल्लाह (कि वह भी उसके मुक़र्रर हैं) मगर उनके बहुतेरे (इतना भी) नहीं जानते
\end{hindi}}
\flushright{\begin{Arabic}
\quranayah[16][76]
\end{Arabic}}
\flushleft{\begin{hindi}
और ख़ुदा एक दूसरी मसल बयान फरमाता है दो आदमी हैं कि एक उनमें से बिल्कुल गूँगा उस पर गुलाम जो कुछ भी (बात वग़ैरह की) कुदरत नहीं रखता और (इस वजह से) वह अपने मालिक को दूभर हो रहा है कि उसको जिधर भेजता है (ख़ैर से) कभी भलाई नहीं लाता क्या ऐसा ग़ुलाम और वह शख़्श जो (लोगों को) अदल व मियाना रवी का हुक्म करता है वह खुद भी ठीक सीधी राह पर क़ायम है (दोनों बराबर हो सकते हैं (हरगिज़ नहीं)
\end{hindi}}
\flushright{\begin{Arabic}
\quranayah[16][77]
\end{Arabic}}
\flushleft{\begin{hindi}
और सारे आसमान व ज़मीन की ग़ैब की बातें ख़ुदा ही के लिए मख़सूस हैं और ख़ुदा क़यामत का वाकेए होना तो ऐसा है जैसे पलक का झपकना बल्कि इससे भी जल्दी बेशक ख़ुदा हर चीज़ पर क़ुदरत कामेला रखता है
\end{hindi}}
\flushright{\begin{Arabic}
\quranayah[16][78]
\end{Arabic}}
\flushleft{\begin{hindi}
और ख़ुदा ही ने तुम्हें तुम्हारी माओं के पेट से निकाला (जब) तुम बिल्कुल नासमझ थे और तुम को कान दिए और ऑंखें (अता की) दिल (इनायत किए) ताकि तुम शुक्र करो
\end{hindi}}
\flushright{\begin{Arabic}
\quranayah[16][79]
\end{Arabic}}
\flushleft{\begin{hindi}
क्या उन लोगों ने परिन्दों की तरफ ग़ौर नहीं किया जो आसमान के नीचे हवा में घिरे हुए (उड़ते रहते हैं) उनको तो बस ख़ुदा ही (गिरने से ) रोकता है बेशक इसमें भी (कुदरते ख़ुदा की) ईमानदारों के लिए बहुत सी निशानियाँ हैं
\end{hindi}}
\flushright{\begin{Arabic}
\quranayah[16][80]
\end{Arabic}}
\flushleft{\begin{hindi}
और ख़ुदा ही ने तुम्हारे घरों को तुम्हारा ठिकाना क़रार दिया और उसी ने तुम्हारे वास्ते चौपायों की खालों से तुम्हारे लिए (हलके फुलके) डेरे (ख़ेमे) बना जिन्हें तुम हलका पाकर अपने सफर और हजर (ठहरने) में काम लाते हो और इन चारपायों की ऊन और रुई और बालो से एक वक्त ख़ास (क़यामत तक) के लिए तुम्हारे बहुत से असबाब और अबकार आमद (काम की) चीज़े बनाई
\end{hindi}}
\flushright{\begin{Arabic}
\quranayah[16][81]
\end{Arabic}}
\flushleft{\begin{hindi}
और ख़ुदा ही ने तुम्हारे आराम के लिए अपनी पैदा की हुईचीज़ों के साए बनाए और उसी ने तुम्हारे (छिपने बैठने) के वास्ते पहाड़ों में घरौन्दे (ग़ार वग़ैरह) बनाए और उसी ने तुम्हारे लिए कपड़े बनाए जो तुम्हें (सर्दी) गर्मी से महफूज़ रखें और (लोहे) के कुर्ते जो तुम्हें हाथियों की ज़द (मार) से बचा लंग़ यूँ ख़ुदा अपनी नेअमत तुम पर पूरी करता है
\end{hindi}}
\flushright{\begin{Arabic}
\quranayah[16][82]
\end{Arabic}}
\flushleft{\begin{hindi}
तुम उसकी फरमाबरदारी करो उस पर भी अगर ये लोग (ईमान से) मुँह फेरे तो तुम्हारा फर्ज सिर्फ (एहकाम का) साफ पहुँचा देना है
\end{hindi}}
\flushright{\begin{Arabic}
\quranayah[16][83]
\end{Arabic}}
\flushleft{\begin{hindi}
(बस) ये लोग ख़ुदा की नेअमतों को पहचानते हैं फिर (जानबुझ कर) उनसे मुकर जाते हैं और इन्हीं में से बहुतेरे नाशुक्रे हैं
\end{hindi}}
\flushright{\begin{Arabic}
\quranayah[16][84]
\end{Arabic}}
\flushleft{\begin{hindi}
और जिस दिन हम एक उम्मत में से (उसके पैग़म्बरों को) गवाह बनाकर क़ब्रों से उठा खड़ा करेगे फिर तो काफिरों को न (बात करने की) इजाज़त दी जाएगी और न उनका उज्र (जवाब) ही सुना जाएगा
\end{hindi}}
\flushright{\begin{Arabic}
\quranayah[16][85]
\end{Arabic}}
\flushleft{\begin{hindi}
और जिन लोगों ने (दुनिया में) शरारतें की थी जब वह अज़ाब को देख लेगें तो उनके अज़ाब ही में तख़फ़ीफ़ की जाएगी और न उनको मोहलत ही दी जाएगी
\end{hindi}}
\flushright{\begin{Arabic}
\quranayah[16][86]
\end{Arabic}}
\flushleft{\begin{hindi}
और जिन लोगों ने दुनिया में ख़ुदा का शरीक (दूसरे को) ठहराया था जब अपने (उन) शरीकों को (क़यामत में) देखेंगे तो बोल उठेगें कि ऐ हमारे परवरदिगार यही तो हमारे शरीक हैं जिन्हें हम (मुसीबत) के वक्त तुझे छोड़ कर पुकारा करते थे तो वह शरीक उन्हीं की तरफ बात को फेंक मारेगें कि तुम निरे झूठे हो
\end{hindi}}
\flushright{\begin{Arabic}
\quranayah[16][87]
\end{Arabic}}
\flushleft{\begin{hindi}
और उस दिन ख़ुदा के सामने सर झुका देंगे और जो इफ़तेरा परदाज़ियाँ दुनिया में किया करते थे उनसे गायब हो जाएँगें
\end{hindi}}
\flushright{\begin{Arabic}
\quranayah[16][88]
\end{Arabic}}
\flushleft{\begin{hindi}
जिन लोगों ने कुफ्र एख्तेयार किया और लोगों को ख़ुदा की राह से रोका उनके फ़साद वाले कामों के बदले में उनके लिए हम अज़ाब पर अज़ाब बढ़ाते जाएँगें
\end{hindi}}
\flushright{\begin{Arabic}
\quranayah[16][89]
\end{Arabic}}
\flushleft{\begin{hindi}
और वह दिन याद करो जिस दिन हम हर एक गिरोह में से उन्हीं में का एक गवाह उनके मुक़ाबिल ला खड़ा करेगें और (ऐ रसूल) तुम को उन लोगों पर (उनके) सामने गवाह बनाकर ला खड़ा करेंगें और हमने तुम पर किताब (क़ुरान) नाज़िल की जिसमें हर चीज़ का (शाफी) बयान है और मुसलमानों के लिए (सरमापा) हिदायत और रहमत और खुशख़बरी है
\end{hindi}}
\flushright{\begin{Arabic}
\quranayah[16][90]
\end{Arabic}}
\flushleft{\begin{hindi}
इसमें शक़ नहीं कि ख़ुदा इन्साफ और (लोगों के साथ) नेकी करने और क़राबतदारों को (कुछ) देने का हुक्म करता है और बदकारी और नाशाएस्ता हरकतों और सरकशी करने को मना करता है (और) तुम्हें नसीहत करता है ताकि तुम नसीहत हासिल करो
\end{hindi}}
\flushright{\begin{Arabic}
\quranayah[16][91]
\end{Arabic}}
\flushleft{\begin{hindi}
और जब तुम लोग बाहम क़ौल व क़रार कर लिया करो तो ख़ुदा के एहदो पैमान को पूरा करो और क़समों को उनके पक्का हो जाने के बाद न तोड़ा करो हालॉकि तुम तो ख़ुदा को अपना ज़ामिन बना चुके हो जो कुछ भी तुम करते हो ख़ुदा उसे ज़रुर जानता है
\end{hindi}}
\flushright{\begin{Arabic}
\quranayah[16][92]
\end{Arabic}}
\flushleft{\begin{hindi}
और तुम लोग (क़समों के तोड़ने में) उस औरत के ऐसे न हो जो अपना सूत मज़बूत कातने के बाद टुकड़े टुकड़े करके तोड़ डाले कि अपने एहदो को आपस में उस बात की मक्कारी का ज़रिया बनाने लगो कि एक गिरोह दूसरे गिरोह से (ख्वामख़वाह) बढ़ जाए इससे बस ख़ुदा तुमको आज़माता है (कि तुम किसी की पालाइश करते हो) और जिन बातों में तुम दुनिया में झगड़ते थे क़यामत के दिन ख़ुदा खुद तुम से साफ साफ बयान कर देगा
\end{hindi}}
\flushright{\begin{Arabic}
\quranayah[16][93]
\end{Arabic}}
\flushleft{\begin{hindi}
और अगर ख़ुदा चाहता तो तुम सबको एक ही (किस्म के) गिरोह बना देता मगर वह तो जिसको चाहता है गुमराही में छोड़ देता है और जिसकी चाहता है हिदायत करता है और जो कुछ तुम लोग दुनिया में किया करते थे उसकी बाज़ पुर्स (पुछ गछ) तुमसे ज़रुर की जाएगी
\end{hindi}}
\flushright{\begin{Arabic}
\quranayah[16][94]
\end{Arabic}}
\flushleft{\begin{hindi}
और तुम अपनी क़समों को आपस में के फसाद का सबब न बनाओ ताकि (लोगों के) क़दम जमने के बाद (इस्लाम से) उखड़ जाएँ और फिर आख़िरकार क़यामत में तुम्हें लोगों को ख़ुदा की राह से रोकने की पादाश (रोकने के बदले) में अज़ाब का मज़ा चखना पड़े और तुम्हारे वास्ते बड़ा सख्त अज़ाब हो
\end{hindi}}
\flushright{\begin{Arabic}
\quranayah[16][95]
\end{Arabic}}
\flushleft{\begin{hindi}
और ख़ुदा के एहदो पैमान के बदले थोड़ी क़ीमत (दुनयावी नफा) न लो अगर तुम जानते (बूझते) हो तो (समझ लो कि) जो कुछ ख़ुदा के पास है वह उससे कहीं बेहतर है
\end{hindi}}
\flushright{\begin{Arabic}
\quranayah[16][96]
\end{Arabic}}
\flushleft{\begin{hindi}
(क्योंकि माल दुनिया से) जो कुछ तुम्हारे पास है एक न एक दिन ख़त्म हो जाएगा और (अज्र) ख़ुदा के पास है वह हमेशा बाक़ी रहेगा और जिन लोगों ने दुनिया में सब्र किया था उनको (क़यामत में) उनके कामों का हम अच्छे से अच्छा अज्र व सवाब अता करेंगें
\end{hindi}}
\flushright{\begin{Arabic}
\quranayah[16][97]
\end{Arabic}}
\flushleft{\begin{hindi}
मर्द हो या औरत जो शख़्श नेक काम करेगा और वह ईमानदार भी हो तो हम उसे (दुनिया में भी) पाक व पाकीज़ा जिन्दगी बसर कराएँगें और (आख़िरत में भी) जो कुछ वह करते थे उसका अच्छे से अच्छा अज्र व सवाब अता फरमाएँगें
\end{hindi}}
\flushright{\begin{Arabic}
\quranayah[16][98]
\end{Arabic}}
\flushleft{\begin{hindi}
और जब तुम क़ुरान पढ़ने लगो तो शैतान मरदूद (के वसवसो) से ख़ुदा की पनाह तलब कर लिया करो
\end{hindi}}
\flushright{\begin{Arabic}
\quranayah[16][99]
\end{Arabic}}
\flushleft{\begin{hindi}
इसमें शक नहीं कि जो लोग ईमानदार हैं और अपने परवरदिगार पर भरोसा रखते हैं उन पर उसका क़ाबू नहीं चलता
\end{hindi}}
\flushright{\begin{Arabic}
\quranayah[16][100]
\end{Arabic}}
\flushleft{\begin{hindi}
उसका क़ाबू चलता है तो बस उन्हीं लोगों पर जो उसको दोस्त बनाते हैं और जो लोग उसको ख़ुदा का शरीक बनाते हैं
\end{hindi}}
\flushright{\begin{Arabic}
\quranayah[16][101]
\end{Arabic}}
\flushleft{\begin{hindi}
और (ऐ रसूल) हम जब एक आयत के बदले दूसरी आयत नाज़िल करते हैं तो हालॉकि ख़ुदा जो चीज़ नाज़िल करता है उस (की मसलहतों) से खूब वाक़िफ है मगर ये लोग (तुम को) कहने लगते हैं कि तुम बस बिल्कुल मुज़तरी (ग़लत बयान करने वाले) हो बल्कि खुद उनमें के बहुतेरे (मसालेह को) नहीं जानते
\end{hindi}}
\flushright{\begin{Arabic}
\quranayah[16][102]
\end{Arabic}}
\flushleft{\begin{hindi}
(ऐ रसूल) तुम (साफ) कह दो कि इस (क़ुरान) को तो रुहलकुदूस (जिबरील) ने तुम्हारे परवरदिगार की तरफ से हक़ नाज़िल किया है ताकि जो लोग ईमान ला चुके हैं उनको साबित क़दम रखे और मुसलमानों के लिए (अज़सरतापा) खुशखबरी है
\end{hindi}}
\flushright{\begin{Arabic}
\quranayah[16][103]
\end{Arabic}}
\flushleft{\begin{hindi}
और (ऐ रसूल) हम तहक़ीक़तन जानते हैं कि ये कुफ्फार तुम्हारी निस्बत कहा करते है कि उनको (तुम को) कोई आदमी क़ुरान सिखा दिया करता है हालॉकि बिल्कुल ग़लत है क्योंकि जिस शख्स की तरफ से ये लोग निस्बत देते हैं उसकी ज़बान तो अजमी है और ये तो साफ साफ अरबी ज़बान है
\end{hindi}}
\flushright{\begin{Arabic}
\quranayah[16][104]
\end{Arabic}}
\flushleft{\begin{hindi}
इसमें तो शक ही नहीं कि जो लोग ख़ुदा की आयतों पर ईमान नहीं लाते ख़ुदा भी उनको मंज़िले मक़सूद तक नहीं पहुँचाएगा और उनके लिए दर्दनाक अज़ाब है
\end{hindi}}
\flushright{\begin{Arabic}
\quranayah[16][105]
\end{Arabic}}
\flushleft{\begin{hindi}
झूठ बोहतान तो बस वही लोग बॉधा करते हैं जो खुदा की आयतों पर ईमान नहीं रखतें
\end{hindi}}
\flushright{\begin{Arabic}
\quranayah[16][106]
\end{Arabic}}
\flushleft{\begin{hindi}
और हक़ीक़त अम्र ये है कि यही लोग झूठे हैं उस शख्स के सिवा जो (कलमाए कुफ्र) पर मजबूर किया जाए और उसका दिल ईमान की तरफ से मुतमइन हो जो शख्स भी ईमान लाने के बाद कुफ्र एख्तियार करे बल्कि खूब सीना कुशादा (जी खोलकर) कुफ्र करे तो उन पर ख़ुदा का ग़ज़ब है और उनके लिए बड़ा (सख्त) अज़ाब है
\end{hindi}}
\flushright{\begin{Arabic}
\quranayah[16][107]
\end{Arabic}}
\flushleft{\begin{hindi}
ये इस वजह से कि उन लोगों ने दुनिया की चन्द रोज़ा ज़िन्दगी को आख़िरत पर तरजीह दी और इस वजह से की ख़ुदा काफिरों को हरगिज़ मंज़िले मक़सूद तक नहीं पहुँचाया करता
\end{hindi}}
\flushright{\begin{Arabic}
\quranayah[16][108]
\end{Arabic}}
\flushleft{\begin{hindi}
ये वही लोग हैं जिनके दिलों पर और उनके कानों पर और उनकी ऑंखों पर खुदा ने अलामात मुक़र्रर कर दी है
\end{hindi}}
\flushright{\begin{Arabic}
\quranayah[16][109]
\end{Arabic}}
\flushleft{\begin{hindi}
(कि ईमान न लाएँगे) और यही लोग (इस सिरे के) बेखबर हैं कुछ शक़ नहीं कि यही लोग आख़िरत में भी यक़ीनन घाटा उठाने वाले हैं
\end{hindi}}
\flushright{\begin{Arabic}
\quranayah[16][110]
\end{Arabic}}
\flushleft{\begin{hindi}
फिर इसमें शक़ नहीं कि तुम्हारा परवरदिगार उन लोगों को जिन्होने मुसीबत में मुब्तिला होने क बाद घर बार छोडे फ़िर (ख़ुदा की राह में) जिहाद किए और तकलीफों पर सब्र किया इसमें शक़ नहीं कि तुम्हारा परवरदिगार इन सब बातों के बाद अलबत्ता बड़ा बख्शने वाला मेहरबान है
\end{hindi}}
\flushright{\begin{Arabic}
\quranayah[16][111]
\end{Arabic}}
\flushleft{\begin{hindi}
और (उस दिन को याद) करो जिस दिन हर शख़्श अपनी ज़ात के बारे में झगड़ने को आ मौजूद होगा
\end{hindi}}
\flushright{\begin{Arabic}
\quranayah[16][112]
\end{Arabic}}
\flushleft{\begin{hindi}
और हर शख़्श को जो कुछ भी उसने किया था उसका पूरा पूरा बदला मिलेगा और उन पर किसी तरह का जुल्म न किया जाएगा ख़ुदा ने एक गाँव की मसल बयान फरमाई जिसके रहने वाले हर तरह के चैन व इत्मेनान में थे हर तरफ से बाफराग़त (बहुत ज्यादा) उनकी रोज़ी उनके पास आई थी फिर उन लोगों ने ख़ुदा की नूअमतों की नाशुक्री की तो ख़ुदा ने उनकी करतूतों की बदौलत उनको मज़ा चखा दिया
\end{hindi}}
\flushright{\begin{Arabic}
\quranayah[16][113]
\end{Arabic}}
\flushleft{\begin{hindi}
कि भूक और ख़ौफ को ओढ़ना (बिछौना) बना दिया और उन्हीं लोगों में का एक रसूल भी उनके पास आया तो उन्होंने उसे झुठलाया
\end{hindi}}
\flushright{\begin{Arabic}
\quranayah[16][114]
\end{Arabic}}
\flushleft{\begin{hindi}
फिर अज़ाब (ख़ुदा) ने उन्हें ले डाला और वह ज़ालिम थे ही तो ख़ुदा ने जो कुछ तुम्हें हलाल तय्यब (ताहिर) रोज़ी दी है उसको (शौक़ से) खाओ और अगर तुम खुदा ही की परसतिश (का दावा करते हो)
\end{hindi}}
\flushright{\begin{Arabic}
\quranayah[16][115]
\end{Arabic}}
\flushleft{\begin{hindi}
उसकी नेअमत का शुक्र अदा किया करो तुम पर उसने मुरदार और खून और सूअर का गोश्त और वह जानवर जिस पर (ज़बाह के वक्त) ख़ुदा के सिवा (किसी) और का नाम लिया जाए हराम किया है फिर जो शख़्श (मारे भूक के) मजबूर हो ख़ुदा से सरतापी (नाफरमानी) करने वाला हो और न (हद ज़रुरत से) बढ़ने वाला हो और (हराम खाए) तो बेशक ख़ुदा बख्शने वाला मेहरबान है
\end{hindi}}
\flushright{\begin{Arabic}
\quranayah[16][116]
\end{Arabic}}
\flushleft{\begin{hindi}
और झूट मूट जो कुछ तुम्हारी ज़बान पर आए (बे समझे बूझे) न कह बैठा करों कि ये हलाल है और हराम है ताकि इसकी बदौलत ख़ुदा पर झूठ बोहतान बाँधने लगो इसमें शक़ नहीं कि जो लोग ख़ुदा पर झूठ बोहतान बाधते हैं वह कभी कामयाब न होगें
\end{hindi}}
\flushright{\begin{Arabic}
\quranayah[16][117]
\end{Arabic}}
\flushleft{\begin{hindi}
(दुनिया में) फायदा तो ज़रा सा है और (आख़िरत में) दर्दनाक अज़ाब है
\end{hindi}}
\flushright{\begin{Arabic}
\quranayah[16][118]
\end{Arabic}}
\flushleft{\begin{hindi}
और यहूदियों पर हमने वह चीज़े हराम कर दीं थी जो तुमसे पहले बयान कर चुके हैं और हमने तो (इस की वजह से) उन पर कुछ ज़ुल्म नहीं किया
\end{hindi}}
\flushright{\begin{Arabic}
\quranayah[16][119]
\end{Arabic}}
\flushleft{\begin{hindi}
मगर वह लोग खुद अपने ऊपर सितम तोड़ रहे हैं फिर इसमे शक़ नहीं कि जो लोग नादानी से गुनाह कर बैठे उसके बाद सदक़ दिल से तौबा कर ली और अपने को दुरुस्त कर लिया तो (ऐ रसूल) इसमें शक़ नहीं कि तुम्हारा परवरदिगार इसके बाद बख्शने वाला मेहरबान है
\end{hindi}}
\flushright{\begin{Arabic}
\quranayah[16][120]
\end{Arabic}}
\flushleft{\begin{hindi}
इसमें शक़ नहीं कि इबराहीम (लोगों के) पेशवा ख़ुदा के फरमाबरदार बन्दे और बातिल से कतरा कर चलने वाले थे और मुशरेकीन से हरगिज़ न थे
\end{hindi}}
\flushright{\begin{Arabic}
\quranayah[16][121]
\end{Arabic}}
\flushleft{\begin{hindi}
उसकी नेअमतों के शुक्र गुज़ार उनको ख़ुदा ने मुनतख़िब कर लिया है और (अपनी) सीधी राह की उन्हें हिदायत की थी
\end{hindi}}
\flushright{\begin{Arabic}
\quranayah[16][122]
\end{Arabic}}
\flushleft{\begin{hindi}
और हमने उन्हें दुनिया में भी (हर तरह की) बेहतरी अता की थी
\end{hindi}}
\flushright{\begin{Arabic}
\quranayah[16][123]
\end{Arabic}}
\flushleft{\begin{hindi}
और वह आख़िरत में भी यक़ीनन नेको कारों से होगें ऐ रसूल फिर तुम्हारे पास वही भेजी कि इबराहीम के तरीक़े की पैरवी करो जो बातिल से कतरा के चलते थे और मुशरेकीन से नहीं थे
\end{hindi}}
\flushright{\begin{Arabic}
\quranayah[16][124]
\end{Arabic}}
\flushleft{\begin{hindi}
(ऐ रसूल) हफ्ते (के दिन) की ताज़ीम तो बस उन्हीं लोगों पर लाज़िम की गई थी (यहूद व नसारा इसके बारे में) एख्तिलाफ करते थे और कुछ शक़ नहीं कि तुम्हारा परवरदिगार उनके दरमियान जिस अग्र में वह झगड़ा करते थे क़यामत के दिन फैसला कर देगा
\end{hindi}}
\flushright{\begin{Arabic}
\quranayah[16][125]
\end{Arabic}}
\flushleft{\begin{hindi}
(ऐ रसूल) तुम (लोगों को) अपने परवरदिगार की राह पर हिकमत और अच्छी अच्छी नसीहत के ज़रिए से बुलाओ और बहस व मुबाशा करो भी तो इस तरीक़े से जो लोगों के नज़दीक सबसे अच्छा हो इसमें शक़ नहीं कि जो लोग ख़ुदा की राह से भटक गए उनको तुम्हारा परवरदिगार खूब जानता है
\end{hindi}}
\flushright{\begin{Arabic}
\quranayah[16][126]
\end{Arabic}}
\flushleft{\begin{hindi}
और हिदायत याफ़ता लोगों से भी खूब वाक़िफ है और अगर (मुख़ालिफीन के साथ) सख्ती करो भी तो वैसी ही सख्ती करो जैसे सख्ती उन लोगों ने तुम पर की थी और अगर तुम सब्र करो तो सब्र करने वालों के वास्ते बेहतर हैं
\end{hindi}}
\flushright{\begin{Arabic}
\quranayah[16][127]
\end{Arabic}}
\flushleft{\begin{hindi}
और (ऐ रसूल) तुम सब्र ही करो (और ख़ुदा की (मदद) बग़ैर तो तुम सब्र कर भी नहीं सकते और उन मुख़ालिफीन के हाल पर तुम रंज न करो और जो मक्कारीयाँ ये लोग करते हैं उससे तुम तंग दिल न हो
\end{hindi}}
\flushright{\begin{Arabic}
\quranayah[16][128]
\end{Arabic}}
\flushleft{\begin{hindi}
जो लोग परहेज़गार हैं और जो लोग नेको कार हैं ख़ुदा उनका साथी है
\end{hindi}}
\chapter{Bani Isra'il (The Israelites)}
\begin{Arabic}
\Huge{\centerline{\basmalah}}\end{Arabic}
\flushright{\begin{Arabic}
\quranayah[17][1]
\end{Arabic}}
\flushleft{\begin{hindi}
वह ख़ुदा (हर ऐब से) पाक व पाकीज़ा है जिसने अपने बन्दों को रातों रात मस्जिदुल हराम (ख़ान ऐ काबा) से मस्जिदुल अक़सा (आसमानी मस्जिद) तक की सैर कराई जिसके चौगिर्द हमने हर किस्म की बरकत मुहय्या कर रखी हैं ताकि हम उसको (अपनी कुदरत की) निशानियाँ दिखाए इसमें शक़ नहीं कि (वह सब कुछ) सुनता (और) देखता है
\end{hindi}}
\flushright{\begin{Arabic}
\quranayah[17][2]
\end{Arabic}}
\flushleft{\begin{hindi}
और हमने मूसा को किताब (तौरैत) अता की और उस को बनी इसराईल की रहनुमा क़रार दिया (और हुक्म दे दिया) कि ऐ उन लोगों की औलाद जिन्हें हम ने नूह के साथ कश्ती में सवार किया था
\end{hindi}}
\flushright{\begin{Arabic}
\quranayah[17][3]
\end{Arabic}}
\flushleft{\begin{hindi}
मेरे सिवा किसी को अपना कारसाज़ न बनाना बेशक नूह बड़ा शुक्र गुज़ार बन्दा था
\end{hindi}}
\flushright{\begin{Arabic}
\quranayah[17][4]
\end{Arabic}}
\flushleft{\begin{hindi}
और हमने बनी इसराईल से इसी किताब (तौरैत) में साफ साफ बयान कर दिया था कि तुम लोग रुए ज़मीन पर दो मरतबा ज़रुर फसाद फैलाओगे और बड़ी सरकशी करोगे
\end{hindi}}
\flushright{\begin{Arabic}
\quranayah[17][5]
\end{Arabic}}
\flushleft{\begin{hindi}
फिर जब उन दो फसादों में पहले का वक्त अा पहुँचा तो हमने तुम पर कुछ अपने बन्दों (नजतुलनस्र) और उसकी फौज को मुसल्लत (ग़ालिब) कर दिया जो बड़े सख्त लड़ने वाले थे तो वह लोग तुम्हारे घरों के अन्दर घुसे (और खूब क़त्ल व ग़ारत किया) और ख़ुदा के अज़ाब का वायदा जो पूरा होकर रहा
\end{hindi}}
\flushright{\begin{Arabic}
\quranayah[17][6]
\end{Arabic}}
\flushleft{\begin{hindi}
फिर हमने तुमको दोबारा उन पर ग़लबा देकर तुम्हारे दिन फेरे और माल से और बेटों से तुम्हारी मदद की और तुमको बड़े जत्थे वाला बना दिया
\end{hindi}}
\flushright{\begin{Arabic}
\quranayah[17][7]
\end{Arabic}}
\flushleft{\begin{hindi}
अगर तुम अच्छे काम करोगे तो अपने फायदे के लिए अच्छे काम करोगे और अगर तुम बुरे काम करोगे तो (भी) अपने ही लिए फिर जब दूसरे वक्त क़ा वायदा आ पहुँचा तो (हमने तैतूस रोगी को तुम पर मुसल्लत किया) ताकि वह लोग (मारते मारते) तुम्हारे चेहरे बिगाड़ दें (कि पहचाने न जाओ) और जिस तरह पहली दफा मस्जिद बैतुल मुक़द्दस में घुस गये थे उसी तरह फिर घुस पड़ें और जिस चीज़ पर क़ाबू पाए खूब अच्छी तरह बरबाद कर दी
\end{hindi}}
\flushright{\begin{Arabic}
\quranayah[17][8]
\end{Arabic}}
\flushleft{\begin{hindi}
(अब भी अगर तुम चैन से रहो तो) उम्मीद है कि तुम्हारा परवरदिगार तुम पर तरस खाए और अगर (कहीं) वही शरारत करोगे तो हम भी फिर पकड़ेंगे और हमने तो काफिरों के लिए जहन्नुम को क़ैद खाना बना ही रखा है
\end{hindi}}
\flushright{\begin{Arabic}
\quranayah[17][9]
\end{Arabic}}
\flushleft{\begin{hindi}
इसमें शक़ नहीं कि ये क़ुरान उस राह की हिदायत करता है जो सबसे ज्यादा सीधी है और जो ईमानदार अच्छे अच्छे काम करते हैं उनको ये खुशख़बरी देता है कि उनके लिए बहुत बड़ा अज्र और सवाब (मौजूद) है
\end{hindi}}
\flushright{\begin{Arabic}
\quranayah[17][10]
\end{Arabic}}
\flushleft{\begin{hindi}
और ये भी कि बेशक जो लोग आख़िरत पर ईमान नहीं रखते हैं उनके लिए हमने दर्दनाक अज़ाब तैयार कर रखा है
\end{hindi}}
\flushright{\begin{Arabic}
\quranayah[17][11]
\end{Arabic}}
\flushleft{\begin{hindi}
और आदमी कभी (आजिज़ होकर अपने हक़ में) बुराई (अज़ाब वग़ैरह की दुआ) इस तरह माँगता है जिस तरह अपने लिए भलाई की दुआ करता है और आदमी तो बड़ा जल्दबाज़ है
\end{hindi}}
\flushright{\begin{Arabic}
\quranayah[17][12]
\end{Arabic}}
\flushleft{\begin{hindi}
और हमने रात और दिन को (अपनी क़ुदरत की) दो निशानियाँ क़रार दिया फिर हमने रात की निशानी (चाँद) को धुँधला बनाया और दिन की निशानी (सूरज) को रौशन बनाया (कि सब चीज़े दिखाई दें) ताकि तुम लोग अपने परवरदिगार का फज़ल ढूँढते फिरों और ताकि तुम बरसों की गिनती और हिसाब को जानो (बूझों) और हमने हर चीज़ को खूब अच्छी तरह तफसील से बयान कर दिया है
\end{hindi}}
\flushright{\begin{Arabic}
\quranayah[17][13]
\end{Arabic}}
\flushleft{\begin{hindi}
और हमने हर आदमी के नामए अमल को उसके गले का हार बना दिया है (कि उसकी किस्मत उसके साथ रहे) और क़यामत के दिन हम उसे उसके सामने निकल के रख देगें कि वह उसको एक खुली हुई किताब अपने रुबरु पाएगा
\end{hindi}}
\flushright{\begin{Arabic}
\quranayah[17][14]
\end{Arabic}}
\flushleft{\begin{hindi}
और हम उससे कहेंगें कि अपना नामए अमल पढ़ले और आज अपने हिसाब के लिए तू आप ही काफी हैं
\end{hindi}}
\flushright{\begin{Arabic}
\quranayah[17][15]
\end{Arabic}}
\flushleft{\begin{hindi}
जो शख़्श रुबरु होता है तो बस अपने फायदे के लिए यह पर आता है और जो शख़्श गुमराह होता है तो उसने भटक कर अपना आप बिगाड़ा और कोई शख़्श किसी दूसरे (के गुनाह) का बोझ अपने सर नहीं लेगा और हम तो जब तक रसूल को भेजकर तमाम हुज्जत न कर लें किसी पर अज़ाब नहीं किया करते
\end{hindi}}
\flushright{\begin{Arabic}
\quranayah[17][16]
\end{Arabic}}
\flushleft{\begin{hindi}
और हमको जब किसी बस्ती का वीरान करना मंज़ूर होता है तो हम वहाँ के खुशहालों को (इताअत का) हुक्म देते हैं तो वह लोग उसमें नाफरमानियाँ करने लगे तब वह बस्ती अज़ाब की मुस्तहक़ होगी उस वक्त हमने उसे अच्छी तरह तबाह व बरबाद कर दिया
\end{hindi}}
\flushright{\begin{Arabic}
\quranayah[17][17]
\end{Arabic}}
\flushleft{\begin{hindi}
और नूह के बाद से (उस वक्त तक) हमने कितनी उम्मतों को हलाक कर मारा और (ऐ रसूल) तुम्हारा परवरदिगार अपने बन्दों के गुनाहों को जानने और देखने के लिए काफी है
\end{hindi}}
\flushright{\begin{Arabic}
\quranayah[17][18]
\end{Arabic}}
\flushleft{\begin{hindi}
(और गवाह याहिद की ज़रुरत नहीं) और जो शख़्श दुनिया का ख्वाहाँ हो तो हम जिसे चाहते और जो चाहते हैं उसी दुनिया में सिरदस्त (फ़ौरन) उसे अता करते हैं (मगर) फिर हमने उसके लिए तो जहन्नुम ठहरा ही रखा है कि वह उसमें बुरी हालत से रौंदा हुआ दाख़िल होगा
\end{hindi}}
\flushright{\begin{Arabic}
\quranayah[17][19]
\end{Arabic}}
\flushleft{\begin{hindi}
और जो शख़्श आख़िर का मुतमइनी हो और उसके लिए खूब जैसी चाहिए कोशिश भी की और वह ईमानदार भी है तो यही वह लोग हैं जिनकी कोशिश मक़बूल होगी
\end{hindi}}
\flushright{\begin{Arabic}
\quranayah[17][20]
\end{Arabic}}
\flushleft{\begin{hindi}
(ऐ रसूल) उनको (ग़रज़ सबको) हम ही तुम्हारे परवरदिगार की (अपनी) बख़्शिस से मदद देते हैं और तुम्हारे परवरदिगार की बख़्शिस तो (आम है) किसी पर बन्द नहीं
\end{hindi}}
\flushright{\begin{Arabic}
\quranayah[17][21]
\end{Arabic}}
\flushleft{\begin{hindi}
(ऐ रसूल) ज़रा देखो तो कि हमने बाज़ लोगों को बाज़ पर कैसी फज़ीलत दी है और आख़िरत के दर्जे तो यक़ीनन (यहाँ से) कहीं बढ़के है और वहाँ की फज़ीलत भी तो कैसी बढ़ कर है
\end{hindi}}
\flushright{\begin{Arabic}
\quranayah[17][22]
\end{Arabic}}
\flushleft{\begin{hindi}
और देखो कहीं ख़ुदा के साथ दूसरे को (उसका) शरीक न बनाना वरना तुम बुरे हाल में ज़लील रुसवा बैठै के बैठें रह जाओगे
\end{hindi}}
\flushright{\begin{Arabic}
\quranayah[17][23]
\end{Arabic}}
\flushleft{\begin{hindi}
और तुम्हारे परवरदिगार ने तो हुक्म ही दिया है कि उसके सिवा किसी दूसरे की इबादत न करना और माँ बाप से नेकी करना अगर उनमें से एक या दोनों तेरे सामने बुढ़ापे को पहुँचे (और किसी बात पर खफा हों) तो (ख़बरदार उनके जवाब में उफ तक) न कहना और न उनको झिड़कना और जो कुछ कहना सुनना हो तो बहुत अदब से कहा करो
\end{hindi}}
\flushright{\begin{Arabic}
\quranayah[17][24]
\end{Arabic}}
\flushleft{\begin{hindi}
और उनके सामने नियाज़ (रहमत) से ख़ाकसारी का पहलू झुकाए रखो और उनके हक़ में दुआ करो कि मेरे पालने वाले जिस तरह इन दोनों ने मेरे छोटेपन में मेरी मेरी परवरिश की है
\end{hindi}}
\flushright{\begin{Arabic}
\quranayah[17][25]
\end{Arabic}}
\flushleft{\begin{hindi}
इसी तरह तू भी इन पर रहम फरमा तुम्हारे दिल की बात तुम्हारा परवरदिगार ख़ूब जानता है अगर तुम (वाक़ई) नेक होगे और भूले से उनकी ख़ता की है तो वह तुमको बख्श देगा क्योंकि वह तो तौबा करने वालों का बड़ा बख़शने वाला है
\end{hindi}}
\flushright{\begin{Arabic}
\quranayah[17][26]
\end{Arabic}}
\flushleft{\begin{hindi}
और क़राबतदारों और मोहताज और परदेसी को उनका हक़ दे दो और ख़बरदार फुज़ूल ख़र्ची मत किया करो
\end{hindi}}
\flushright{\begin{Arabic}
\quranayah[17][27]
\end{Arabic}}
\flushleft{\begin{hindi}
क्योंकि फुज़ूलख़र्ची करने वाले यक़ीनन शैतानों के भाई है और शैतान अपने परवरदिगार का बड़ा नाशुक्री करने वाला है
\end{hindi}}
\flushright{\begin{Arabic}
\quranayah[17][28]
\end{Arabic}}
\flushleft{\begin{hindi}
और तुमको अपने परवरदिगार के फज़ल व करम के इन्तज़ार में जिसकी तुम को उम्मीद हो (मजबूरन) उन (ग़रीबों) से मुँह मोड़ना पड़े तो नरमी से उनको समझा दो
\end{hindi}}
\flushright{\begin{Arabic}
\quranayah[17][29]
\end{Arabic}}
\flushleft{\begin{hindi}
और अपने हाथ को न तो गर्दन से बँधा हुआ (बहुत तंग) कर लो (कि किसी को कुछ दो ही नहीं) और न बिल्कुल खोल दो कि सब कुछ दे डालो और आख़िर तुम को मलामत ज़दा हसरत से बैठना पड़े
\end{hindi}}
\flushright{\begin{Arabic}
\quranayah[17][30]
\end{Arabic}}
\flushleft{\begin{hindi}
इसमें शक़ नहीं कि तुम्हारा परवरदिगार जिसके लिए चाहता है रोज़ी को फराख़ (बढ़ा) देता है और जिसकी रोज़ी चाहता है तंग रखता है इसमें शक़ नहीं कि वह अपने बन्दों से बहुत बाख़बर और देखभाल रखने वाला है
\end{hindi}}
\flushright{\begin{Arabic}
\quranayah[17][31]
\end{Arabic}}
\flushleft{\begin{hindi}
और (लोगों) मुफलिसी (ग़रीबी) के ख़ौफ से अपनी औलाद को क़त्ल न करो (क्योंकि) उनको और तुम को (सबको) तो हम ही रोज़ी देते हैं बेशक औलाद का क़त्ल करना बहुत सख्त गुनाह है
\end{hindi}}
\flushright{\begin{Arabic}
\quranayah[17][32]
\end{Arabic}}
\flushleft{\begin{hindi}
और (देखो) ज़िना के पास भी न फटकना क्योंकि बेशक वह बड़ी बेहयाई का काम है और बहुत बुरा चलन है
\end{hindi}}
\flushright{\begin{Arabic}
\quranayah[17][33]
\end{Arabic}}
\flushleft{\begin{hindi}
और जिस जान का मारना ख़ुदा ने हराम कर दिया है उसके क़त्ल न करना मगर जायज़ तौर पर और जो शख़्श नाहक़ मारा जाए तो हमने उसके वारिस को (क़ातिल पर क़सास का क़ाबू दिया है तो उसे चाहिए कि क़त्ल (ख़ून का बदला लेने) में ज्यादती न करे बेशक वह मदद दिया जाएगा
\end{hindi}}
\flushright{\begin{Arabic}
\quranayah[17][34]
\end{Arabic}}
\flushleft{\begin{hindi}
(कि क़त्ल ही करे और माफ न करे) और यतीम जब तक जवानी को पहुँचे उसके माल के क़रीब भी न पहुँच जाना मगर हाँ इस तरह पर कि (यतीम के हक़ में) बेहतर हो और एहद को पूरा करो क्योंकि (क़यामत में) एहद की ज़रुर पूछ गछ होगी
\end{hindi}}
\flushright{\begin{Arabic}
\quranayah[17][35]
\end{Arabic}}
\flushleft{\begin{hindi}
और जब नाप तौल कर देना हो तो पैमाने को पूरा भर दिया करो और (जब तौल कर देना हो तो) बिल्कुल ठीक तराजू से तौला करो (मामले में) यही (तरीक़ा) बेहतर है और अन्जाम (भी उसका) अच्छा है
\end{hindi}}
\flushright{\begin{Arabic}
\quranayah[17][36]
\end{Arabic}}
\flushleft{\begin{hindi}
और जिस चीज़ का कि तुम्हें यक़ीन न हो (ख्वाह मा ख्वाह) उसके पीछे न पड़ा करो (क्योंकि) कान और ऑंख और दिल इन सबकी (क़यामत के दिना यक़ीनन बाज़पुर्स होती है
\end{hindi}}
\flushright{\begin{Arabic}
\quranayah[17][37]
\end{Arabic}}
\flushleft{\begin{hindi}
और (देखो) ज़मीन पर अकड़ कर न चला करो क्योंकि तू (अपने इस धमाके की चाल से) न तो ज़मीन को हरगिज़ फाड़ डालेगा और न (तनकर चलने से) हरगिज़ लम्बाई में पहाड़ों के बराबर पहुँच सकेगा
\end{hindi}}
\flushright{\begin{Arabic}
\quranayah[17][38]
\end{Arabic}}
\flushleft{\begin{hindi}
(ऐ रसूल) इन सब बातों में से जो बुरी बात है वह तुम्हारे परवरदिगार के नज़दीक नापसन्द है
\end{hindi}}
\flushright{\begin{Arabic}
\quranayah[17][39]
\end{Arabic}}
\flushleft{\begin{hindi}
ये बात तो हिकमत की उन बातों में से जो तुम्हारे परवरदिगार ने तुम्हारे पास 'वही' भेजी और ख़ुदा के साथ कोई दूसरा माबूद न बनाना और न तू मलामत ज़दा राइन्द (धुत्कारा) होकर जहन्नुम में झोंक दिया जाएगा
\end{hindi}}
\flushright{\begin{Arabic}
\quranayah[17][40]
\end{Arabic}}
\flushleft{\begin{hindi}
(ऐ मुशरेकीन मक्का) क्या तुम्हारे परवरदिगार ने तुम्हें चुन चुन कर बेटे दिए हैं और खुद बेटियाँ ली हैं (यानि) फरिश्ते इसमें शक़ नहीं कि बड़ी (सख्त) बात कहते हो
\end{hindi}}
\flushright{\begin{Arabic}
\quranayah[17][41]
\end{Arabic}}
\flushleft{\begin{hindi}
और हमने तो इसी क़ुरान में तरह तरह से बयान कर दिया ताकि लोग किसी तरह समझें मगर उससे तो उनकी नफरत ही बढ़ती गई
\end{hindi}}
\flushright{\begin{Arabic}
\quranayah[17][42]
\end{Arabic}}
\flushleft{\begin{hindi}
(ऐ रसूल उनसे) तुम कह दो कि अगर ख़ुदा के साथ जैसा ये लोग कहते हैं और माबूद भी होते तो अब तक उन माबूदों ने अर्श तक (पहुँचाने की कोई न कोई राह निकाल ली होती
\end{hindi}}
\flushright{\begin{Arabic}
\quranayah[17][43]
\end{Arabic}}
\flushleft{\begin{hindi}
जो बेहूदा बातें ये लोग (ख़ुदा की निस्बत) कहा करते हैं वह उनसे बहुत बढ़के पाक व पाकीज़ा और बरतर है
\end{hindi}}
\flushright{\begin{Arabic}
\quranayah[17][44]
\end{Arabic}}
\flushleft{\begin{hindi}
सातों आसमान और ज़मीन और जो लोग इनमें (सब) उसकी तस्बीह करते हैं और (सारे जहाँन) में कोई चीज़ ऐसी नहीं जो उसकी (हम्द व सना) की तस्बीह न करती हो मगर तुम लोग उनकी तस्बीह नहीं समझते इसमें शक़ नहीं कि वह बड़ा बुर्दबार बख्शने वाला है
\end{hindi}}
\flushright{\begin{Arabic}
\quranayah[17][45]
\end{Arabic}}
\flushleft{\begin{hindi}
और जब तुम क़ुरान पढ़ते हो तो हम तुम्हारे और उन लोगों के दरमियान जो आख़िरत का यक़ीन नहीं रखते एक गहरा पर्दा डाल देते हैं
\end{hindi}}
\flushright{\begin{Arabic}
\quranayah[17][46]
\end{Arabic}}
\flushleft{\begin{hindi}
और (गोया) हम उनके कानों में गरानी पैदा कर देते हैं कि न सुन सकें जब तुम क़ुरान में अपने परवरदिगार का तन्हा ज़िक्र करते हो तो कुफ्फार उलटे पावँ नफरत करके (तुम्हारे पास से) भाग खड़े होते हैं
\end{hindi}}
\flushright{\begin{Arabic}
\quranayah[17][47]
\end{Arabic}}
\flushleft{\begin{hindi}
जब ये लोग तुम्हारी तरफ कान लगाते हैं तो जो कुछ ये ग़ौर से सुनते हैं हम तो खूब जानते हैं और जब ये लोग बाहम कान में बात करते हैं तो उस वक्त ये ज़ालिम (ईमानदारों से) कहते हैं कि तुम तो बस एक (दीवाने) आदमी के पीछे पड़े हो जिस पर किसी ने जादू कर दिया है
\end{hindi}}
\flushright{\begin{Arabic}
\quranayah[17][48]
\end{Arabic}}
\flushleft{\begin{hindi}
(ऐ रसूल) ज़रा देखो तो ये कम्बख्त तुम्हारी निस्बत कैसी कैसी फब्तियाँ कहते हैं तो (इसी वजह से) ऐसे गुमराह हुए कि अब (हक़ की) राह किसी तरह पा ही नहीं सकते
\end{hindi}}
\flushright{\begin{Arabic}
\quranayah[17][49]
\end{Arabic}}
\flushleft{\begin{hindi}
और ये लोग कहते हैं कि जब हम (मरने के बाद सड़ गल कर) हड्डियाँ रह जाएँगें और रेज़ा रेज़ा हो जाएँगें तो क्या नये सिरे से पैदा करके उठा खड़े किए जाएँगें
\end{hindi}}
\flushright{\begin{Arabic}
\quranayah[17][50]
\end{Arabic}}
\flushleft{\begin{hindi}
(ऐ रसूल) तुम कह दो कि तुम (मरने के बाद) चाहे पत्थर बन जाओ या लोहा या कोई और चीज़ जो तुम्हारे ख्याल में बड़ी (सख्त) हो
\end{hindi}}
\flushright{\begin{Arabic}
\quranayah[17][51]
\end{Arabic}}
\flushleft{\begin{hindi}
और उसका ज़िन्दा होना दुश्वार हो वह भी ज़रुर ज़िन्दा हो गई तो ये लोग अनक़रीब ही तुम से पूछेगें भला हमें दोबारा कौन ज़िन्दा करेगा तुम कह दो कि वही (ख़ुदा) जिसने तुमको पहली मरतबा पैदा किया (जब तुम कुछ न थे) इस पर ये लोग तुम्हारे सामने अपने सर मटकाएँगें और कहेगें (अच्छा अगर होगा) तो आख़िर कब तुम कह दो कि बहुत जल्द अनक़रीब ही होगा
\end{hindi}}
\flushright{\begin{Arabic}
\quranayah[17][52]
\end{Arabic}}
\flushleft{\begin{hindi}
जिस दिन ख़ुदा तुम्हें (इसराफील के ज़रिए से) बुंलाएगा तो उसकी हम्दो सना करते हुए उसकी तामील करोगे (और क़ब्रों से निकलोगे) और तुम ख्याल करोगे कि (मरने के बाद क़ब्रों में) बहुत ही कम ठहरे
\end{hindi}}
\flushright{\begin{Arabic}
\quranayah[17][53]
\end{Arabic}}
\flushleft{\begin{hindi}
और (ऐ रसूल) मेरे (सच्चे) बन्दों (मोमिनों से कह दो कि वह (काफिरों से) बात करें तो अच्छे तरीक़े से (सख्त कलामी न करें) क्योंकि शैतान तो (ऐसी ही) बातों से फसाद डलवाता है इसमें तो शक़ ही नहीं कि शैतान आदमी का खुला हुआ दुश्मन है
\end{hindi}}
\flushright{\begin{Arabic}
\quranayah[17][54]
\end{Arabic}}
\flushleft{\begin{hindi}
तुम्हारा परवरदिगार तुम्हारे हाल से खूब वाक़िफ है अगर चाहेगा तुम पर रहम करेगा और अगर चाहेगा तुम पर अज़ाब करेगा और (ऐ रसूल) हमने तुमको कुछ उन लोगों का ज़िम्मेदार बनाकर नहीं भेजा है
\end{hindi}}
\flushright{\begin{Arabic}
\quranayah[17][55]
\end{Arabic}}
\flushleft{\begin{hindi}
और जो लोग आसमानों में है और ज़मीन पर हैं (सब को) तुम्हारा परवरदिगार खूब जानता है और हम ने यक़ीनन बाज़ पैग़म्बरों को बाज़ पर फज़ीलत दी और हम ही ने दाऊद को जूबूर अता की
\end{hindi}}
\flushright{\begin{Arabic}
\quranayah[17][56]
\end{Arabic}}
\flushleft{\begin{hindi}
(ऐ रसूल) तुम उनसे कह दों कि ख़ुदा के सिवा और जिन लोगों को माबूद समझते हो उनको (वक्त पडे) पुकार के तो देखो कि वह न तो तुम से तुम्हारी तकलीफ ही दफा कर सकते हैं और न उसको बदल सकते हैं
\end{hindi}}
\flushright{\begin{Arabic}
\quranayah[17][57]
\end{Arabic}}
\flushleft{\begin{hindi}
ये लोग जिनको मुशरेकीन (अपना ख़ुदा समझकर) इबादत करते हैं वह खुद अपने परवरदिगार की क़ुरबत के ज़रिए ढूँढते फिरते हैं कि (देखो) इनमें से कौन ज्यादा कुरबत रखता है और उसकी रहमत की उम्मीद रखते और उसके अज़ाब से डरते हैं इसमें शक़ नहीं कि तेरे परवरदिगार का अज़ाब डरने की चीज़ है
\end{hindi}}
\flushright{\begin{Arabic}
\quranayah[17][58]
\end{Arabic}}
\flushleft{\begin{hindi}
और कोई बस्ती नहीं है मगर रोज़ क़यामत से पहले हम उसे तबाह व बरबाद कर छोड़ेगें या (नाफरमानी) की सज़ा में उस पर सख्त से सख्त अज़ाब करेगें (और) ये बात किताब (लौहे महफूज़) में लिखी जा चुकी है
\end{hindi}}
\flushright{\begin{Arabic}
\quranayah[17][59]
\end{Arabic}}
\flushleft{\begin{hindi}
और हमें मौजिज़ात भेजने से किसी चीज़ ने नहीं रोका मगर इसके सिवा कि अगलों ने उन्हें झुठला दिया और हमने क़ौमे समूद को (मौजिज़े से) ऊँटनी अता की जो (हमारी कुदरत की) दिखाने वाली थी तो उन लोगों ने उस पर ज़ुल्म किया यहाँ तक कि मार डाला और हम तो मौजिज़े सिर्फ डराने की ग़रज़ से भेजा करते हैं
\end{hindi}}
\flushright{\begin{Arabic}
\quranayah[17][60]
\end{Arabic}}
\flushleft{\begin{hindi}
और (ऐ रसूल) वह वक्त याद करो जब तुमसे हमने कह दिया था कि तुम्हारे परवरदिगार ने लोगों को (हर तरफ से) रोक रखा है कि (तुम्हारा कुछ बिगाड़ नहीं सकते और हमने जो ख्वाब तुमाको दिखलाया था तो बस उसे लोगों (के ईमान) की आज़माइश का ज़रिया ठहराया था और (इसी तरह) वह दरख्त जिस पर क़ुरान में लानत की गई है और हम बावजूद कि उन लोगों को (तरह तरह) से डराते हैं मगर हमारा डराना उनकी सख्त सरकशी को बढ़ाता ही गया
\end{hindi}}
\flushright{\begin{Arabic}
\quranayah[17][61]
\end{Arabic}}
\flushleft{\begin{hindi}
और जब हम ने फरिश्तों से कहा कि आदम को सजदा करो तो सबने सजदा किया मगर इबलीस वह (गुरुर से) कहने लगा कि क्या मै ऐसे शख़्श को सजदा करुँ जिसे तूने मिट्टी से पैदा किया है
\end{hindi}}
\flushright{\begin{Arabic}
\quranayah[17][62]
\end{Arabic}}
\flushleft{\begin{hindi}
और (शेख़ी से) बोला भला देखो तो सही यही वह शख़्श है जिसको तूने मुझ पर फज़ीलत दी है अगर तू मुझ को क़यामत तक की मोहलत दे तो मैं (दावे से कहता हूँ कि) कम लोगों के सिवा इसकी नस्ल की जड़ काटता रहूँगा
\end{hindi}}
\flushright{\begin{Arabic}
\quranayah[17][63]
\end{Arabic}}
\flushleft{\begin{hindi}
ख़ुदा ने फरमाया चल (दूर हो) उनमें से जो शख़्श तेरी पैरवी करेगा तो (याद रहे कि) तुम सबकी सज़ा जहन्नुम है और वह भी पूरी पूरी सज़ा है
\end{hindi}}
\flushright{\begin{Arabic}
\quranayah[17][64]
\end{Arabic}}
\flushleft{\begin{hindi}
और इसमें से जिस पर अपनी (चिकनी चुपड़ी) बात से क़ाबू पा सके वहॉ और अपने (चेलों के लश्कर) सवार और पैदल (सब) से चढ़ाई कर और माल और औलाद में उनके साथ साझा करे और उनसे (खूब झूठे) वायदे कर और शैतान तो उनसे जो वायदे करता है धोखे के सिवा कुछ नहीं होता
\end{hindi}}
\flushright{\begin{Arabic}
\quranayah[17][65]
\end{Arabic}}
\flushleft{\begin{hindi}
बेशक जो मेरे (ख़ास) बन्दें हैं उन पर तेरा ज़ोर नहीं चल (सकता) और कारसाज़ी में तेरा परवरदिगार काफी है
\end{hindi}}
\flushright{\begin{Arabic}
\quranayah[17][66]
\end{Arabic}}
\flushleft{\begin{hindi}
लोगों) तुम्हारा परवरदिगार वह (क़ादिरे मुत्तलिक़) है जो तुम्हारे लिए समन्दर में जहाज़ों को चलाता है ताकि तुम उसके फज़ल व करम (रोज़ी) की तलाश करो इसमें शक़ नहीं कि वह तुम पर बड़ा मेहरबान है
\end{hindi}}
\flushright{\begin{Arabic}
\quranayah[17][67]
\end{Arabic}}
\flushleft{\begin{hindi}
और जब समन्दर में कभी तुम को कोई तकलीफ पहुँचे तो जिनकी तुम इबादत किया करते थे ग़ायब हो गए मगर बस वही (एक ख़ुदा याद रहता है) उस पर भी जब ख़ुदा ने तुम को छुटकारा देकर खुशकी तक पहुँचा दिया तो फिर तुम इससे मुँह मोड़ बैठें और इन्सान बड़ा ही नाशुक्रा है
\end{hindi}}
\flushright{\begin{Arabic}
\quranayah[17][68]
\end{Arabic}}
\flushleft{\begin{hindi}
तो क्या तुम उसको इस का भी इत्मिनान हो गया कि वह तुम्हें खुश्की की तरफ (ले जाकर) (क़ारुन की तरह) ज़मीन में धंसा दे या तुम पर (क़ौम) लूत की तरह पत्थरों का मेंह बरसा दे फिर (उस वक्त) तुम किसी को अपना कारसाज़ न पाओगे
\end{hindi}}
\flushright{\begin{Arabic}
\quranayah[17][69]
\end{Arabic}}
\flushleft{\begin{hindi}
या तुमको इसका भी इत्मेनान हो गया कि फिर तुमको दोबारा इसी समन्दर में ले जाएगा उसके बाद हवा का एक ऐसा झोका जो (जहाज़ के) परख़चे उड़ा दे तुम पर भेजे फिर तुम्हें तुम्हारे कुफ्र की सज़ा में डुबा मारे फिर तुम किसी को (ऐसा हिमायती) न पाओगे जो हमारा पीछा करे और (तुम्हें छोड़ा जाए)
\end{hindi}}
\flushright{\begin{Arabic}
\quranayah[17][70]
\end{Arabic}}
\flushleft{\begin{hindi}
और हमने यक़ीनन आदम की औलाद को इज्ज़त दी और खुश्की और तरी में उनको (जानवरों कश्तियों के ज़रिए) लिए लिए फिरे और उन्हें अच्छी अच्छी चीज़ें खाने को दी और अपने बहुतेरे मख़लूक़ात पर उनको अच्छी ख़ासी फज़ीलत दी
\end{hindi}}
\flushright{\begin{Arabic}
\quranayah[17][71]
\end{Arabic}}
\flushleft{\begin{hindi}
उस दिन (को याद करो) जब हम तमाम लोगों को उन पेशवाओं के साथ बुलाएँगें तो जिसका नामए अमल उनके दाहिने हाथ में दिया जाएगा तो वह लोग (खुश खुश) अपना नामए अमल पढ़ने लगेगें और उन पर रेशा बराबर ज़ुल्म नहीं किया जाएगा
\end{hindi}}
\flushright{\begin{Arabic}
\quranayah[17][72]
\end{Arabic}}
\flushleft{\begin{hindi}
और जो शख़्श इस (दुनिया) में (जान बूझकर) अंधा बना रहा तो वह आख़िरत में भी अंधा ही रहेगा और (नजात) के रास्ते से बहुत दूर भटका सा हुआ
\end{hindi}}
\flushright{\begin{Arabic}
\quranayah[17][73]
\end{Arabic}}
\flushleft{\begin{hindi}
और (ऐ रसूल) हमने तो (क़ुरान) तुम्हारे पास 'वही' के ज़रिए भेजा अगर चे लोग तो तुम्हें इससे बहकाने ही लगे थे ताकि तुम क़ुरान के अलावा फिर (दूसरी बातों का) इफ़तेरा बाँधों और (जब तुम ये कर गुज़रते उस वक्त ये लोग तुम को अपना सच्चा दोस्त बना लेते
\end{hindi}}
\flushright{\begin{Arabic}
\quranayah[17][74]
\end{Arabic}}
\flushleft{\begin{hindi}
और अगर हम तुमको साबित क़दम न रखते तो ज़रुर तुम भी ज़रा (ज़हूर) झुकने ही लगते
\end{hindi}}
\flushright{\begin{Arabic}
\quranayah[17][75]
\end{Arabic}}
\flushleft{\begin{hindi}
और (अगर तुम ऐसा करते तो) उस वक्त हम तुमको ज़िन्दगी में भी और मरने पर भी दोहरे (अज़ाब) का मज़ा चखा देते और फिर तुम को हमारे मुक़ाबले में कोई मददगार भी न मिलता
\end{hindi}}
\flushright{\begin{Arabic}
\quranayah[17][76]
\end{Arabic}}
\flushleft{\begin{hindi}
और ये लोग तो तुम्हें (सर ज़मीन मक्के) से दिल बर्दाश्त करने ही लगे थे ताकि तुम को वहाँ से (शाम की तरफ) निकाल बाहर करें और ऐसा होता तो तुम्हारे पीछे में ये लोग चन्द रोज़ के सिवा ठहरने भी न पाते
\end{hindi}}
\flushright{\begin{Arabic}
\quranayah[17][77]
\end{Arabic}}
\flushleft{\begin{hindi}
तुमसे पहले जितने रसूल हमने भेजे हैं उनका बराबर यही दस्तूर रहा है और जो दस्तूर हमारे (ठहराए हुए) हैं उनमें तुम तग्य्युर तबद्दुल (रद्दो बदल) न पाओगे
\end{hindi}}
\flushright{\begin{Arabic}
\quranayah[17][78]
\end{Arabic}}
\flushleft{\begin{hindi}
(ऐ रसूल) सूरज के ढलने से रात के अंधेरे तक नमाज़े ज़ोहर, अस्र, मग़रिब, इशा पढ़ा करो और नमाज़ सुबह (भी) क्योंकि सुबह की नमाज़ पर (दिन और रात दोनों के फरिश्तों की) गवाही होती है
\end{hindi}}
\flushright{\begin{Arabic}
\quranayah[17][79]
\end{Arabic}}
\flushleft{\begin{hindi}
और रात के ख़ास हिस्से में नमाजे तहज्जुद पढ़ा करो ये सुन्नत तुम्हारी खास फज़ीलत हैं क़रीब है कि क़यामत के दिन ख़ुदा तुमको मक़ामे महमूद तक पहुँचा दे
\end{hindi}}
\flushright{\begin{Arabic}
\quranayah[17][80]
\end{Arabic}}
\flushleft{\begin{hindi}
और ये दुआ माँगा करो कि ऐ मेरे परवरदिगार मुझे (जहाँ) पहुँचा अच्छी तरह पहुँचा और मुझे (जहाँ से निकाल) तो अच्छी तरह निकाल और मुझे ख़ास अपनी बारगाह से एक हुकूमत अता फरमा जिस से (हर क़िस्म की) मदद पहुँचे
\end{hindi}}
\flushright{\begin{Arabic}
\quranayah[17][81]
\end{Arabic}}
\flushleft{\begin{hindi}
और (ऐ रसूल) कह दो कि (दीन) हक़ आ गया और बातिल नेस्तनाबूद हुआ इसमें शक़ नहीं कि बातिल मिटने वाला ही था
\end{hindi}}
\flushright{\begin{Arabic}
\quranayah[17][82]
\end{Arabic}}
\flushleft{\begin{hindi}
और हम तो क़ुरान में वही चीज़ नाज़िल करते हैं जो मोमिनों के लिए (सरासर) शिफा और रहमत है (मगर) नाफरमानों को तो घाटे के सिवा कुछ बढ़ाता ही नहीं
\end{hindi}}
\flushright{\begin{Arabic}
\quranayah[17][83]
\end{Arabic}}
\flushleft{\begin{hindi}
और जब हमने आदमी को नेअमत अता फरमाई तो (उल्टे) उसने (हमसे) मुँह फेरा और पहलू बचाने लगा और जब उसे कोई तकलीफ छू भी गई तो मायूस हो बैठा
\end{hindi}}
\flushright{\begin{Arabic}
\quranayah[17][84]
\end{Arabic}}
\flushleft{\begin{hindi}
(ऐ रसूल) तुम कह दो कि हर (एक अपने तरीक़े पर कारगुज़ारी करता है फिर तुम में से जो शख़्श बिल्कुल ठीक सीधी राह पर है तुम्हारा परवरदिगार (उससे) खूब वाक़िफ है
\end{hindi}}
\flushright{\begin{Arabic}
\quranayah[17][85]
\end{Arabic}}
\flushleft{\begin{hindi}
और (ऐ रसूल) तुमसे लोग रुह के बारे में सवाल करते हैं तुम (उनके जवाब में) कह दो कि रूह (भी) मेरे परदिगार के हुक्म से (पैदा हुईहै) और तुमको बहुत थोड़ा सा इल्म दिया गया है
\end{hindi}}
\flushright{\begin{Arabic}
\quranayah[17][86]
\end{Arabic}}
\flushleft{\begin{hindi}
(इसकी हक़ीकत नहीं समझ सकते) और (ऐ रसूल) अगर हम चाहे तो जो (क़ुरान) हमने तुम्हारे पास 'वही' के ज़रिए भेजा है (दुनिया से) उठा ले जाएँ फिर तुम अपने वास्ते हमारे मुक़ाबले में कोई मददगार न पाओगे
\end{hindi}}
\flushright{\begin{Arabic}
\quranayah[17][87]
\end{Arabic}}
\flushleft{\begin{hindi}
मगर ये सिर्फ तुम्हारे परवरदिगार की रहमत है (कि उसने ऐसा किया) इसमें शक़ नहीं कि उसका तुम पर बड़ा फज़ल व करम है
\end{hindi}}
\flushright{\begin{Arabic}
\quranayah[17][88]
\end{Arabic}}
\flushleft{\begin{hindi}
(ऐ रसूल) तुम कह दो कि (अगर सारे दुनिया जहाँन के) आदमी और जिन इस बात पर इकट्ठे हो कि उस क़ुरान का मिसल ले आएँ तो (ना मुमकिन) उसके बराबर नहीं ला सकते अगरचे (उसको कोशिश में) एक का एक मददगार भी बने
\end{hindi}}
\flushright{\begin{Arabic}
\quranayah[17][89]
\end{Arabic}}
\flushleft{\begin{hindi}
और हमने तो लोगों (के समझाने) के वास्ते इस क़ुरान में हर क़िस्म की मसलें अदल बदल के बयान कर दीं उस पर भी अक्सर लोग बग़ैर नाशुक्री किए नहीं रहते
\end{hindi}}
\flushright{\begin{Arabic}
\quranayah[17][90]
\end{Arabic}}
\flushleft{\begin{hindi}
(ऐ रसूल कुफ्फार मक्के ने) तुमसे कहा कि जब तक तुम हमारे वास्ते ज़मीन से चश्मा (न) बहा निकालोगे हम तो तुम पर हरगिज़ ईमान न लाएँगें
\end{hindi}}
\flushright{\begin{Arabic}
\quranayah[17][91]
\end{Arabic}}
\flushleft{\begin{hindi}
या (ये नहीं तो) खजूरों और अंगूरों का तुम्हारा कोई बाग़ हो उसमें तुम बीच बीच में नहरे जारी करके दिखा दो
\end{hindi}}
\flushright{\begin{Arabic}
\quranayah[17][92]
\end{Arabic}}
\flushleft{\begin{hindi}
या जैसा तुम गुमान रखते थे हम पर आसमान ही को टुकड़े (टुकड़े) करके गिराओ या ख़ुदा और फरिश्तों को (अपने क़ौल की तस्दीक़) में हमारे सामने (गवाही में ला खड़ा कर दिया
\end{hindi}}
\flushright{\begin{Arabic}
\quranayah[17][93]
\end{Arabic}}
\flushleft{\begin{hindi}
और जब तक तुम हम पर ख़ुदा के यहाँ से एक किताब न नाज़िल करोगे कि हम उसे खुद पढ़ भी लें उस वक्त तक हम तुम्हारे (आसमान पर चढ़ने के भी) क़ायल न होगें (ऐ रसूल) तुम कह दो कि सुबहान अल्लाह मै एक आदमी (ख़ुदा के) रसूल के सिवा आख़िर और क्या हूँ
\end{hindi}}
\flushright{\begin{Arabic}
\quranayah[17][94]
\end{Arabic}}
\flushleft{\begin{hindi}
(जो ये बेहूदा बातें करते हो) और जब लोगों के पास हिदायत आ चुकी तो उनको ईमान लाने से इसके सिवा किसी चीज़ ने न रोका कि वह कहने लगे कि क्या ख़ुदा ने आदमी को रसूल बनाकर भेजा है
\end{hindi}}
\flushright{\begin{Arabic}
\quranayah[17][95]
\end{Arabic}}
\flushleft{\begin{hindi}
(ऐ रसूल) तुम कह दो कि अगर ज़मीन पर फ़रिश्ते (बसे हुये) होते कि इत्मेनान से चलते फिरते तो हम उन लोगों के पास फ़रिश्ते ही को रसूल बनाकर नाज़िल करते
\end{hindi}}
\flushright{\begin{Arabic}
\quranayah[17][96]
\end{Arabic}}
\flushleft{\begin{hindi}
(ऐ रसूल) तुम कह दो कि हमारे तुम्हारे दरमियान गवाही के वास्ते बस ख़ुदा काफी है इसमें शक़ नहीं कि वह अपने बन्दों के हाल से खूब वाक़िफ और देखता रहता है
\end{hindi}}
\flushright{\begin{Arabic}
\quranayah[17][97]
\end{Arabic}}
\flushleft{\begin{hindi}
और ख़ुदा जिसकी हिदायत करे वही हिदायत याफता है और जिसको गुमराही में छोड़ दे तो (याद रखो कि) फिर उसके सिवा किसी को उसका सरपरस्त न पाआगे और क़यामत के दिन हम उन लोगों का मुँह के बल औंधे और गूँगें और बहरे क़ब्रों से उठाएँगें उनका ठिकाना जहन्नुम है कि जब कभी बुझने को होगी तो हम उन लोगों पर (उसे) और भड़का देंगे
\end{hindi}}
\flushright{\begin{Arabic}
\quranayah[17][98]
\end{Arabic}}
\flushleft{\begin{hindi}
ये सज़ा उनकी इस वज़ह से है कि उन लोगों ने हमारी आयतों से इन्कार किया और कहने लगे कि जब हम (मरने के बाद सड़ गल) कर हड्डियाँ और रेज़ा रेज़ा हो जाएँगीं तो क्या फिर हम नये सिरे से पैदा करके उठाए जाएँगें
\end{hindi}}
\flushright{\begin{Arabic}
\quranayah[17][99]
\end{Arabic}}
\flushleft{\begin{hindi}
क्या उन लोगों ने इस पर भी नहीं ग़ौर किया कि वह ख़ुदा जिसने सारे आसमान और ज़मीन बनाए इस पर भी (ज़रुर) क़ादिर है कि उनके ऐसे आदमी दोबारा पैदा करे और उसने उन (की मौत) की एक मियाद मुक़र्रर कर दी है जिसमें ज़रा भी शक़ नहीं उस पर भी ये ज़ालिम इन्कार किए बग़ैर न रहे
\end{hindi}}
\flushright{\begin{Arabic}
\quranayah[17][100]
\end{Arabic}}
\flushleft{\begin{hindi}
(ऐ रसूल) इनसे कहो कि अगर मेरे परवरदिगार के रहमत के ख़ज़ाने भी तुम्हारे एख़तियार में होते तो भी तुम खर्च हो जाने के डर से (उनको) बन्द रखते और आदमी बड़ा ही तंग दिल है
\end{hindi}}
\flushright{\begin{Arabic}
\quranayah[17][101]
\end{Arabic}}
\flushleft{\begin{hindi}
और हमने यक़ीनन मूसा को खुले हुए नौ मौजिज़े अता किए तो (ऐ रसूल) बनी इसराईल से (यही) पूछ देखो कि जब मूसा उनके पास आए तो फिरऔन ने उनसे कहा कि ऐ मूसा मै तो समझता हूँ कि किसी ने तुम पर जादू करके दीवाना बना दिया है
\end{hindi}}
\flushright{\begin{Arabic}
\quranayah[17][102]
\end{Arabic}}
\flushleft{\begin{hindi}
मूसा ने कहा तुम ये ज़रुर जानते हो कि ये मौजिज़े सारे आसमान व ज़मीन के परवरदिगार ने नाज़िल किए (और वह भी लोगों की) सूझ बूझ की बातें हैं और ऐ फिरऔन मै तो ख्याल करता हूँ कि तुम पर यामत आई है
\end{hindi}}
\flushright{\begin{Arabic}
\quranayah[17][103]
\end{Arabic}}
\flushleft{\begin{hindi}
फिर फिरऔन ने ये ठान लिया कि बनी इसराईल को (सर ज़मीने) मिò से निकाल बाहर करे तो हमने फिरऔन और जो लोग उसके साथ थे सब को डुबो मारा
\end{hindi}}
\flushright{\begin{Arabic}
\quranayah[17][104]
\end{Arabic}}
\flushleft{\begin{hindi}
और उसके बाद हमने बनी इसराईल से कहा कि (अब तुम ही) इस मुल्क में (खूब आराम से) रहो सहो फिर जब आख़िरत का वायदा आ पहुँचेगा तो हम तुम सबको समेट कर ले आएँगें
\end{hindi}}
\flushright{\begin{Arabic}
\quranayah[17][105]
\end{Arabic}}
\flushleft{\begin{hindi}
और (ऐ रसूल) हमने इस क़ुरान को बिल्कुल ठीक नाज़िल किया और बिल्कुल ठीक नाज़िल हुआ और तुमको तो हमने (जन्नत की) खुशखबरी देने वाला और (अज़ाब से) डराने वाला (रसूल) बनाकर भेजा है
\end{hindi}}
\flushright{\begin{Arabic}
\quranayah[17][106]
\end{Arabic}}
\flushleft{\begin{hindi}
और क़ुरान को हमने थोड़ा थोड़ा करके इसलिए नाज़िल किया कि तुम लोगों के सामने (ज़रुरत पड़ने पर) मोहलत दे देकर उसको पढ़ दिया करो
\end{hindi}}
\flushright{\begin{Arabic}
\quranayah[17][107]
\end{Arabic}}
\flushleft{\begin{hindi}
और (इसी वजह से) हमने उसको रफ्ता रफ्ता नाज़िल किया तुम कह दो कि ख्वाह तुम इस पर ईमान लाओ या न लाओ इसमें शक़ नहीं कि जिन लोगों को उसके क़ब्ल ही (आसमानी किताबों का इल्म अता किया गया है उनके सामने जब ये पढ़ा जाता है तो ठुडडियों से (मुँह के बल) सजदे में गिर पड़तें हैं
\end{hindi}}
\flushright{\begin{Arabic}
\quranayah[17][108]
\end{Arabic}}
\flushleft{\begin{hindi}
और कहते हैं कि हमारा परवरदिगार (हर ऐब से) पाक व पाकीज़ा है बेशक हमारे परवरदिगार का वायदा पूरा होना ज़रुरी था
\end{hindi}}
\flushright{\begin{Arabic}
\quranayah[17][109]
\end{Arabic}}
\flushleft{\begin{hindi}
और ये लोग (सजदे के लिए) मुँह के बल गिर पड़तें हैं और रोते चले जाते हैं और ये क़ुरान उन की ख़ाकसारी के बढ़ाता जाता है (109) (सजदा)
\end{hindi}}
\flushright{\begin{Arabic}
\quranayah[17][110]
\end{Arabic}}
\flushleft{\begin{hindi}
(ऐ रसूल) तुम (उनसे) कह दो कि (तुम को एख़तियार है) ख्वाह उसे अल्लाह (कहकर) पुकारो या रहमान कह कर पुकारो (ग़रज़) जिस नाम को भी पुकारो उसके तो सब नाम अच्छे (से अच्छे) हैं और (ऐ रसूल) न तो अपनी नमाज़ बहुत चिल्ला कर पढ़ो न और न बिल्कुल चुपके से बल्कि उसके दरमियान एक औसत तरीका एख्तेयार कर लो
\end{hindi}}
\flushright{\begin{Arabic}
\quranayah[17][111]
\end{Arabic}}
\flushleft{\begin{hindi}
और कहो कि हर तरह की तारीफ उसी ख़ुदा को (सज़ावार) है जो न तो कोई औलाद रखता है और न (सारे जहाँन की) सल्तनत में उसका कोई साझेदार है और न उसे किसी तरह की कमज़ोरी है न कोई उसका सरपरस्त हो और उसकी बड़ाई अच्छी तरह करते रहा करो
\end{hindi}}
\chapter{Al-Kahf (The Cave)}
\begin{Arabic}
\Huge{\centerline{\basmalah}}\end{Arabic}
\flushright{\begin{Arabic}
\quranayah[18][1]
\end{Arabic}}
\flushleft{\begin{hindi}
हर तरह की तारीफ ख़ुदा ही को (सज़ावार) है जिसने अपने बन्दे (मोहम्मद) पर किताब (क़ुरान) नाज़िल की और उसमें किसी तरह की कज़ी (ख़राबी) न रखी
\end{hindi}}
\flushright{\begin{Arabic}
\quranayah[18][2]
\end{Arabic}}
\flushleft{\begin{hindi}
बल्कि हर तरह से सधा ताकि जो सख्त अज़ाब ख़ुदा की बारगाह से काफिरों पर नाज़िल होने वाला है उससे लोगों को डराए और जिन मोमिनीन ने अच्छे अच्छे काम किए हैं उनको इस बात की खुशख़बरी दे की उनके लिए बहुत अच्छा अज्र (व सवाब) मौजूद है
\end{hindi}}
\flushright{\begin{Arabic}
\quranayah[18][3]
\end{Arabic}}
\flushleft{\begin{hindi}
जिसमें वह हमेशा (बाइत्मेनान) तमाम रहेगें
\end{hindi}}
\flushright{\begin{Arabic}
\quranayah[18][4]
\end{Arabic}}
\flushleft{\begin{hindi}
और जो लोग इसके क़ाएल हैं कि ख़ुदा औलाद रखता है उनको (अज़ाब से) डराओ
\end{hindi}}
\flushright{\begin{Arabic}
\quranayah[18][5]
\end{Arabic}}
\flushleft{\begin{hindi}
न तो उन्हीं को उसकी कुछ खबर है और न उनके बाप दादाओं ही को थी (ये) बड़ी सख्त बात है जो उनके मुँह से निकलती है ये लोग झूठ मूठ के सिवा (कुछ और) बोलते ही नहीं
\end{hindi}}
\flushright{\begin{Arabic}
\quranayah[18][6]
\end{Arabic}}
\flushleft{\begin{hindi}
तो (ऐ रसूल) अगर ये लोग इस बात को न माने तो यायद तुम मारे अफसोस के उनके पीछे अपनी जान दे डालोगे
\end{hindi}}
\flushright{\begin{Arabic}
\quranayah[18][7]
\end{Arabic}}
\flushleft{\begin{hindi}
और जो कुछ रुए ज़मीन पर है हमने उसकी ज़ीनत (रौनक़) क़रार दी ताकि हम लोगों का इम्तिहान लें कि उनमें से कौन सबसे अच्छा चलन का है
\end{hindi}}
\flushright{\begin{Arabic}
\quranayah[18][8]
\end{Arabic}}
\flushleft{\begin{hindi}
और (फिर) हम एक न एक दिन जो कुछ भी इस पर है (सबको मिटा करके) चटियल मैदान बना देगें
\end{hindi}}
\flushright{\begin{Arabic}
\quranayah[18][9]
\end{Arabic}}
\flushleft{\begin{hindi}
(ऐ रसूल) क्या तुम ये ख्याल करते हो कि असहाब कहफ व रक़ीम (खोह) और (तख्ती वाले) हमारी (क़ुदरत की) निशानियों में से एक अजीब (निशानी) थे
\end{hindi}}
\flushright{\begin{Arabic}
\quranayah[18][10]
\end{Arabic}}
\flushleft{\begin{hindi}
कि एक बारगी कुछ जवान ग़ार में आ पहुँचे और दुआ की-ऐ हमारे परवरदिगार हमें अपनी बारगाह से रहमत अता फरमा-और हमारे वास्ते हमारे काम में कामयाबी इनायत कर
\end{hindi}}
\flushright{\begin{Arabic}
\quranayah[18][11]
\end{Arabic}}
\flushleft{\begin{hindi}
तब हमने कई बरस तक ग़ार में उनके कानों पर पर्दे डाल दिए (उन्हें सुला दिया)
\end{hindi}}
\flushright{\begin{Arabic}
\quranayah[18][12]
\end{Arabic}}
\flushleft{\begin{hindi}
फिर हमने उन्हें चौकाया ताकि हम देखें कि दो गिरोहों में से किसी को (ग़ार में) ठहरने की मुद्दत खूब याद है
\end{hindi}}
\flushright{\begin{Arabic}
\quranayah[18][13]
\end{Arabic}}
\flushleft{\begin{hindi}
(ऐ रसूल) अब हम उनका हाल तुमसे बिल्कुल ठीक तहक़ीक़ातन (यक़ीन के साथ) बयान करते हैं वह चन्द जवान थे कि अपने (सच्चे) परवरदिगार पर ईमान लाए थे और हम ने उनकी सोच समझ और ज्यादा कर दी है
\end{hindi}}
\flushright{\begin{Arabic}
\quranayah[18][14]
\end{Arabic}}
\flushleft{\begin{hindi}
और हमने उनकी दिलों पर (सब्र व इस्तेक़लाल की) गिराह लगा दी (कि जब दक़ियानूस बादशाह ने कुफ्र पर मजबूर किया) तो उठ खड़े हुए (और बे ताम्मुल (खटके)) कहने लगे हमारा परवरदिगार तो बस सारे आसमान व ज़मीन का मालिक है हम तो उसके सिवा किसी माबूद की हरगिज़ इबादत न करेगें
\end{hindi}}
\flushright{\begin{Arabic}
\quranayah[18][15]
\end{Arabic}}
\flushleft{\begin{hindi}
अगर हम ऐसा करे तो यक़ीनन हमने अक़ल से दूर की बात कही (अफसोस एक) ये हमारी क़ौम के लोग हैं कि जिन्होनें ख़ुदा को छोड़कर (दूसरे) माबूद बनाए हैं (फिर) ये लोग उनके (माबूद होने) की कोई सरीही (खुली) दलील क्यों नहीं पेश करते और जो शख़्श ख़ुदा पर झूट बोहतान बाँधे उससे ज्यादा ज़ालिम और कौन होगा
\end{hindi}}
\flushright{\begin{Arabic}
\quranayah[18][16]
\end{Arabic}}
\flushleft{\begin{hindi}
(फिर बाहम कहने लगे कि) जब तुमने उन लोगों से और ख़ुदा के सिवा जिन माबूदों की ये लोग परसतिश करते हैं उनसे किनारा कशी करली तो चलो (फलॉ) ग़ार में जा बैठो और तुम्हारा परवरदिगार अपनी रहमत तुम पर वसीह कर देगा और तुम्हारा काम में तुम्हारे लिए आसानी के सामान मुहय्या करेगा
\end{hindi}}
\flushright{\begin{Arabic}
\quranayah[18][17]
\end{Arabic}}
\flushleft{\begin{hindi}
(ग़रज़ ये ठान कर ग़ार में जा पहुँचे) कि जब सूरज निकलता है तो देखेगा कि वह उनके ग़ार से दाहिनी तरफ झुक कर निकलता है और जब ग़ुरुब (डुबता) होता है तो उनसे बायीं तरफ कतरा जाता है और वह लोग (मजे से) ग़ार के अन्दर एक वसीइ (बड़ी) जगह में (लेटे) हैं ये ख़ुदा (की कुदरत) की निशानियों में से (एक निशानी) है जिसको हिदायत करे वही हिदायत याफ्ता है और जिस को गुमराह करे तो फिर उसका कोई सरपरस्त रहनुमा हरगिज़ न पाओगे
\end{hindi}}
\flushright{\begin{Arabic}
\quranayah[18][18]
\end{Arabic}}
\flushleft{\begin{hindi}
तू उनको समझेगा कि वह जागते हैं हालॉकि वह (गहरी नींद में) सो रहे हैं और हम कभी दाहिनी तरफ और कभी बायीं तरफ उनकी करवट बदलवा देते हैं और उनका कुत्ताा अपने आगे के दोनो पाँव फैलाए चौखट पर डटा बैठा है (उनकी ये हालत है कि) अगर कहीं तू उनको झाक कर देखे तो उलटे पाँव ज़रुर भाग खड़े हो और तेरे दिल में दहशत समा जाए
\end{hindi}}
\flushright{\begin{Arabic}
\quranayah[18][19]
\end{Arabic}}
\flushleft{\begin{hindi}
और (जिस तरह अपनी कुदरत से उनको सुलाया) उसी तरह (अपनी कुदरत से) उनको (जगा) उठाया ताकि आपस में कुछ पूछ गछ करें (ग़रज़) उनमें एक बोलने वाला बोल उठा कि (भई आख़िर इस ग़ार में) तुम कितनी मुद्दत ठहरे कहने लगे (अरे ठहरे क्या बस) एक दिन से भी कम उसके बाद कहने लगे कि जितनी देर तुम ग़ार में ठहरे उसको तुम्हारे परवरदिगार ही (कुछ तुम से) बेहतर जानता है (अच्छा) तो अब अपने में से किसी को अपना ये रुपया देकर शहर की तरफ भेजो तो वह (जाकर) देखभाल ले कि वहाँ कौन सा खाना बहुत अच्छा है फिर उसमें से (ज़रुरत भर) खाना तुम्हारे वास्ते ले आए और उसे चाहिए कि वह आहिस्ता चुपके से आ जाए और किसी को तुम्हारी ख़बर न होने दे
\end{hindi}}
\flushright{\begin{Arabic}
\quranayah[18][20]
\end{Arabic}}
\flushleft{\begin{hindi}
इसमें शक़ नहीं कि अगर उन लोगों को तुम्हारी इत्तेलाअ हो गई तो बस फिर तुम को संगसार ही कर देंगें या फिर तुम को अपने दीन की तरफ फेर कर ले जाएँगे और अगर ऐसा हुआ तो फिर तुम कभी कामयाब न होगे
\end{hindi}}
\flushright{\begin{Arabic}
\quranayah[18][21]
\end{Arabic}}
\flushleft{\begin{hindi}
और हमने यूँ उनकी क़ौम के लोगों को उनकी हालत पर इत्तेलाअ (ख़बर) कराई ताकि वह लोग देख लें कि ख़ुदा को वायदा यक़ीनन सच्चा है और ये (भी समझ लें) कि क़यामत (के आने) में कुछ भी शुबहा नहीं अब (इत्तिलाआ होने के बाद) उनके बारे में लोग बाहम झगड़ने लगे तो कुछ लोगों ने कहा कि उनके (ग़ार) पर (बतौर यादगार) कोई इमारत बना दो उनका परवरदिगार तो उनके हाल से खूब वाक़िफ है ही और उनके बारे में जिन (मोमिनीन) की राए ग़ालिब रही उन्होंने कहा कि हम तो उन (के ग़ार) पर एक मस्जिद बनाएँगें
\end{hindi}}
\flushright{\begin{Arabic}
\quranayah[18][22]
\end{Arabic}}
\flushleft{\begin{hindi}
क़रीब है कि लोग (नुसैरे नज़रान) कहेगें कि वह तीन आदमी थे चौथा उनका कुत्ताा (क़तमीर) है और कुछ लोग (आक़िब वग़ैरह) कहते हैं कि वह पाँच आदमी थे छठा उनका कुत्ताा है (ये सब) ग़ैब में अटकल लगाते हैं और कुछ लोग कहते हैं कि सात आदमी हैं और आठवाँ उनका कुत्ताा है (ऐ रसूल) तुम कह दो की उनका सुमार मेरा परवरदिगार ही ख़ब जानता है उन (की गिनती) के थोडे ही लोग जानते हैं तो (ऐ रसूल) तुम (उन लोगों से) असहाब कहफ के बारे में सरसरी गुफ्तगू के सिवा (ज्यादा) न झगड़ों और उनके बारे में उन लोगों से किसी से कुछ पूछ गछ नहीं
\end{hindi}}
\flushright{\begin{Arabic}
\quranayah[18][23]
\end{Arabic}}
\flushleft{\begin{hindi}
और किसी काम की निस्बत न कहा करो कि मै इसको कल करुँगा
\end{hindi}}
\flushright{\begin{Arabic}
\quranayah[18][24]
\end{Arabic}}
\flushleft{\begin{hindi}
मगर इन्शा अल्लाह कह कर और अगर (इन्शा अल्लाह कहना) भूल जाओ तो (जब याद आए) अपने परवरदिगार को याद कर लो (इन्शा अल्लाह कह लो) और कहो कि उम्मीद है कि मेरा परवरदिगार मुझे ऐसी बात की हिदायत फरमाए जो रहनुमाई में उससे भी ज्यादा क़रीब हो
\end{hindi}}
\flushright{\begin{Arabic}
\quranayah[18][25]
\end{Arabic}}
\flushleft{\begin{hindi}
और असहाब कहफ अपने ग़ार में नौ ऊपर तीन सौ बरस रहे
\end{hindi}}
\flushright{\begin{Arabic}
\quranayah[18][26]
\end{Arabic}}
\flushleft{\begin{hindi}
(ऐ रसूल) अगर वह लोग इस पर भी न मानें तो तुम कह दो कि ख़ुदा उनके ठहरने की मुद्दत से बखूबी वाक़िफ है सारे आसमान और ज़मीन का ग़ैब उसी के वास्ते ख़ास है (अल्लाह हो अकबर) वो कैसा देखने वाला क्या ही सुनने वाला है उसके सिवा उन लोगों का कोई सरपरस्त नहीं और वह अपने हुक्म में किसी को अपना दख़ील (शरीक) नहीं बनाता
\end{hindi}}
\flushright{\begin{Arabic}
\quranayah[18][27]
\end{Arabic}}
\flushleft{\begin{hindi}
और (ऐ रसूल) जो किताब तुम्हारे परवरदिगार की तरफ से वही के ज़रिए से नाज़िल हुईहै उसको पढ़ा करो उसकी बातों को कोई बदल नहीं सकता और तुम उसके सिवा कहीं कोई हरगिज़ पनाह की जगह (भी) न पाओगे
\end{hindi}}
\flushright{\begin{Arabic}
\quranayah[18][28]
\end{Arabic}}
\flushleft{\begin{hindi}
और (ऐ रसूल) जो लोग अपने परवरदिगार को सुबह सवेरे और झटपट वक्त शाम को याद करते हैं और उसकी खुशनूदी के ख्वाहाँ हैं उनके उनके साथ तुम खुद (भी) अपने नफस पर जब्र करो और उनकी तरफ से अपनी नज़र (तवज्जो) न फेरो कि तुम दुनिया में ज़िन्दगी की आराइश चाहने लगो और जिसके दिल को हमने (गोया खुद) अपने ज़िक्र से ग़ाफिल कर दिया है और वह अपनी ख्वाहिशे नफसानी के पीछे पड़ा है और उसका काम सरासर ज्यादती है उसका कहना हरगिज़ न मानना
\end{hindi}}
\flushright{\begin{Arabic}
\quranayah[18][29]
\end{Arabic}}
\flushleft{\begin{hindi}
और (ऐ रसूल) तुम कह दों कि सच्ची बात (कलमए तौहीद) तुम्हारे परवरदिगार की तरफ से (नाज़िल हो चुकी है) बस जो चाहे माने और जो चाहे न माने (मगर) हमने ज़ालिमों के लिए वह आग (दहका के) तैयार कर रखी है जिसकी क़नातें उन्हें घेर लेगी और अगर वह लोग दोहाई करेगें तो उनकी फरियाद रसी खौलते हुए पानी से की जाएगी जो मसलन पिघले हुए ताबें की तरह होगा (और) वह मुँह को भून डालेगा क्या बुरा पानी है और (जहन्नुम भी) क्या बुरी जगह है
\end{hindi}}
\flushright{\begin{Arabic}
\quranayah[18][30]
\end{Arabic}}
\flushleft{\begin{hindi}
इसमें शक़ नहीं कि जिन लोगों ने ईमान कुबूल किया और अच्छे अच्छे काम करते रहे तो हम हरगिज़ अच्छे काम वालो के अज्र को अकारत नहीं करते
\end{hindi}}
\flushright{\begin{Arabic}
\quranayah[18][31]
\end{Arabic}}
\flushleft{\begin{hindi}
ये वही लोग हैं जिनके (रहने सहने के) लिए सदाबहार (बेहश्त के) बाग़ात हैं उनके (मकानात के) नीचे नहरें जारी होगीं वह उन बाग़ात में दमकते हुए कुन्दन के कंगन से सँवारे जाँएगें और उन्हें बारीक रेशम (क्रेब) और दबीज़ रेश्म (वाफते)के धानी जोड़े पहनाए जाएँगें और तख्तों पर तकिए लगाए (बैठे) होगें क्या ही अच्छा बदला है और (बेहश्त भी आसाइश की) कैसी अच्छी जगह है
\end{hindi}}
\flushright{\begin{Arabic}
\quranayah[18][32]
\end{Arabic}}
\flushleft{\begin{hindi}
और (ऐ रसूल) इन लोगों से उन दो शख़्शों की मिसाल बयान करो कि उनमें से एक को हमने अंगूर के दो बाग़ दे रखे है और हमने चारो ओर खजूर के पेड़ लगा दिये है और उन दोनों बाग़ के दरमियान खेती भी लगाई है
\end{hindi}}
\flushright{\begin{Arabic}
\quranayah[18][33]
\end{Arabic}}
\flushleft{\begin{hindi}
वह दोनों बाग़ खूब फल लाए और फल लाने में कुछ कमी नहीं की और हमने उन दोनों बाग़ों के दरमियान नहर भी जारी कर दी है
\end{hindi}}
\flushright{\begin{Arabic}
\quranayah[18][34]
\end{Arabic}}
\flushleft{\begin{hindi}
और उसे फल मिला तो अपने साथी से जो उससे बातें कर रहा था बोल उठा कि मै तो तुझसे माल में (भी) ज्यादा हूँ और जत्थे में भी बढ़ कर हूँ
\end{hindi}}
\flushright{\begin{Arabic}
\quranayah[18][35]
\end{Arabic}}
\flushleft{\begin{hindi}
और ये बातें करता हुआ अपने बाग़ मे भी जा पहुँचा हालॉकि उसकी आदत ये थी कि (कुफ्र की वजह से) अपने ऊपर आप ज़ुल्म कर रहा था (ग़रज़ वह कह बैठा) कि मुझे तो इसका गुमान नहीं तो कि कभी भी ये बाग़ उजड़ जाए
\end{hindi}}
\flushright{\begin{Arabic}
\quranayah[18][36]
\end{Arabic}}
\flushleft{\begin{hindi}
और मै तो ये भी नहीं ख्याल करता कि क़यामत क़ायम होगी और (बिलग़रज़ हुई भी तो) जब मै अपने परवरदिगार की तरफ लौटाया जाऊँगा तो यक़ीनन इससे कहीं अच्छी जगह पाऊँगा
\end{hindi}}
\flushright{\begin{Arabic}
\quranayah[18][37]
\end{Arabic}}
\flushleft{\begin{hindi}
उसका साथी जो उससे बातें कर रहा था कहने लगा कि क्या तू उस परवरदिगार का मुन्किर है जिसने (पहले) तुझे मिट्टी से पैदा किया फिर नुत्फे से फिर तुझे बिल्कुल ठीक मर्द (आदमी) बना दिया
\end{hindi}}
\flushright{\begin{Arabic}
\quranayah[18][38]
\end{Arabic}}
\flushleft{\begin{hindi}
हम तो (कहते हैं कि) वही ख़ुदा मेरा परवरदिगार है और मै तो अपने परवरदिगार का किसी को शरीक नहीं बनाता
\end{hindi}}
\flushright{\begin{Arabic}
\quranayah[18][39]
\end{Arabic}}
\flushleft{\begin{hindi}
और जब तू अपने बाग़ में आया तो (ये) क्यों न कहा कि ये सब (माशा अल्लाह ख़ुदा ही के चाहने से हुआ है (मेरा कुछ भी नहीं क्योंकि) बग़ैर ख़ुदा की (मदद) के (किसी में) कुछ सकत नहीं अगर माल और औलाद की राह से तू मुझे कम समझता है
\end{hindi}}
\flushright{\begin{Arabic}
\quranayah[18][40]
\end{Arabic}}
\flushleft{\begin{hindi}
तो अनक़ीरब ही मेरा परवरदिगार मुझे वह बाग़ अता फरमाएगा जो तेरे बाग़ से कहीं बेहतर होगा और तेरे बाग़ पर कोई ऐसी आफत आसमान से नाज़िल करे कि (ख़ाक सियाह) होकर चटियल चिकना सफ़ाचट मैदान हो जाए
\end{hindi}}
\flushright{\begin{Arabic}
\quranayah[18][41]
\end{Arabic}}
\flushleft{\begin{hindi}
उसका पानी नीचे उतर (के खुश्क) हो जाए फिर तो उसको किसी तरह तलब न कर सके
\end{hindi}}
\flushright{\begin{Arabic}
\quranayah[18][42]
\end{Arabic}}
\flushleft{\begin{hindi}
(चुनान्चे अज़ाब नाज़िल हुआ) और उसके (बाग़ के) फल (आफत में) घेर लिए गए तो उस माल पर जो बाग़ की तैयारी में सर्फ (ख़र्च) किया था (अफसोस से) हाथ मलने लगा और बाग़ की ये हालत थी कि अपनी टहनियों पर औंधा गिरा हुआ पड़ा था तो कहने लगा काश मै अपने परवरदिगार का किसी को शरीक न बनाता
\end{hindi}}
\flushright{\begin{Arabic}
\quranayah[18][43]
\end{Arabic}}
\flushleft{\begin{hindi}
और ख़ुदा के सिवा उसका कोई जत्था भी न था कि उसकी मदद करता और न वह बदला ले सकता था इसी जगह से (साबित हो गया
\end{hindi}}
\flushright{\begin{Arabic}
\quranayah[18][44]
\end{Arabic}}
\flushleft{\begin{hindi}
कि सरपरस्ती ख़ास ख़ुदा ही के लिए है जो सच्चा है वही बेहतर सवाब (देने) वाला है और अन्जाम के जंगल से भी वही बेहतर है
\end{hindi}}
\flushright{\begin{Arabic}
\quranayah[18][45]
\end{Arabic}}
\flushleft{\begin{hindi}
और (ऐ रसूल) इनसे दुनिया की ज़िन्दगी की मसल भी बयान कर दो कि उसके हालत पानी की सी है जिसे हमने आसमान से बरसाया तो ज़मीन की उगाने की ताक़त उसमें मिल गई और (खूब फली फूली) फिर आख़िर रेज़ा रेज़ा (भूसा) हो गई कि उसको हवाएँ उड़ाए फिरती है और ख़ुदा हर चीज़ पर क़ादिर है
\end{hindi}}
\flushright{\begin{Arabic}
\quranayah[18][46]
\end{Arabic}}
\flushleft{\begin{hindi}
(ऐ रसूल) माल और औलाद (इस ज़रा सी) दुनिया की ज़िन्दगी की ज़ीनत हैं और बाक़ी रहने वाली नेकियाँ तुम्हारे परवरदिगार के नज़दीक सवाब में उससे कही ज्यादा अच्छी हैं और तमन्नाएँ व आरजू की राह से (भी) बेहतर हैं
\end{hindi}}
\flushright{\begin{Arabic}
\quranayah[18][47]
\end{Arabic}}
\flushleft{\begin{hindi}
और (उस दिन से डरो) जिस दिन हम पहाड़ों को चलाएँगें और तुम ज़मीन को खुला मैदान (पहाड़ों से) खाली देखोगे और हम इन सभी को इकट्ठा करेगे तो उनमें से एक को न छोड़ेगें
\end{hindi}}
\flushright{\begin{Arabic}
\quranayah[18][48]
\end{Arabic}}
\flushleft{\begin{hindi}
सबके सब तुम्हारे परवरदिगार के सामने कतार पे क़तार पेश किए जाएँगें और (उस वक्त हम याद दिलाएँगे कि जिस तरह हमने तुमको पहली बार पैदा किया था (उसी तरह) तुम लोगों को (आख़िर) हमारे पास आना पड़ा मगर तुम तो ये ख्याल करते थे कि हम तुम्हारे (दोबारा पैदा करने के) लिए कोई वक्त ही न ठहराएँगें
\end{hindi}}
\flushright{\begin{Arabic}
\quranayah[18][49]
\end{Arabic}}
\flushleft{\begin{hindi}
और लोगों के आमाल की किताब (सामने) रखी जाएँगी तो तुम गुनेहगारों को देखोगे कि जो कुछ उसमें (लिखा) है (देख देख कर) सहमे हुए हैं और कहते जाते हैं हाए हमारी यामत ये कैसी किताब है कि न छोटे ही गुनाह को बे क़लमबन्द किए छोड़ती है न बड़े गुनाह को और जो कुछ इन लोगों ने (दुनिया में) किया था वह सब (लिखा हुआ) मौजूद पाएँगें और तेरा परवरदिगार किसी पर (ज़र्रा बराबर) ज़ुल्म न करेगा
\end{hindi}}
\flushright{\begin{Arabic}
\quranayah[18][50]
\end{Arabic}}
\flushleft{\begin{hindi}
और (वह वक्त याद करो) जब हमने फ़रिश्तों को हुक्म दिया कि आदम को सजदा करो तो इबलीस के सिवा सबने सजदा किया (ये इबलीस) जिन्नात से था तो अपने परवरदिगार के हुक्म से निकल भागा तो (लोगों) क्या मुझे छोड़कर उसको और उसकी औलाद को अपना दोस्त बनाते हो हालॉकि वह तुम्हारा (क़दीमी) दुश्मन हैं ज़ालिमों (ने ख़ुदा के बदले शैतान को अपना दोस्त बनाया ये उन) का क्या बुरा ऐवज़ है
\end{hindi}}
\flushright{\begin{Arabic}
\quranayah[18][51]
\end{Arabic}}
\flushleft{\begin{hindi}
मैने न तो आसमान व ज़मीन के पैदा करने के वक्त उनको (मदद के लिए) बुलाया था और न खुद उनके पैदा करने के वक्त अौर मै (ऐसा गया गुज़रा) न था कि मै गुमराह करने वालों को मददगार बनाता
\end{hindi}}
\flushright{\begin{Arabic}
\quranayah[18][52]
\end{Arabic}}
\flushleft{\begin{hindi}
और (उस दिन से डरो) जिस दिन ख़ुदा फरमाएगा कि अब तुम जिन लोगों को मेरा शरीक़ ख्याल करते थे उनको (मदद के लिए) पुकारो तो वह लोग उनको पुकारेगें मगर वह लोग उनकी कुछ न सुनेगें और हम उन दोनों के बीच में महलक (खतरनाक) आड़ बना देंगे
\end{hindi}}
\flushright{\begin{Arabic}
\quranayah[18][53]
\end{Arabic}}
\flushleft{\begin{hindi}
और गुनेहगार लोग (देखकर समझ जाएँगें कि ये इसमें सोके जाएँगे और उससे गरीज़ (बचने की) की राह न पाएँगें
\end{hindi}}
\flushright{\begin{Arabic}
\quranayah[18][54]
\end{Arabic}}
\flushleft{\begin{hindi}
और हमने तो इस क़ुरान में लोगों (के समझाने) के वास्ते हर तरह की मिसालें फेर बदल कर बयान कर दी है मगर इन्सान तो तमाम मख़लूक़ात से ज्यादा झगड़ालू है
\end{hindi}}
\flushright{\begin{Arabic}
\quranayah[18][55]
\end{Arabic}}
\flushleft{\begin{hindi}
और जब लोगों के पास हिदायत आ चुकी तो (फिर) उनको ईमान लाने और अपने परवरदिगार से मग़फिरत की दुआ माँगने से (उसके सिवा और कौन) अम्र मायने है कि अगलों की सी रीत रस्म उनको भी पेश आई या हमारा अज़ाब उनके सामने से (मौजूद) हो
\end{hindi}}
\flushright{\begin{Arabic}
\quranayah[18][56]
\end{Arabic}}
\flushleft{\begin{hindi}
और हम तो पैग़म्बरों को सिर्फ इसलिए भेजते हैं कि (अच्छों को निजात की) खुशख़बरी सुनाएंऔर (बदों को अज़ाब से) डराएंऔर जो लोग काफिर हैं झूटी झूटी बातों का सहारा पकड़ के झगड़ा करते है ताकि उसकी बदौलत हक़ को (उसकी जगह से उखाड़ फेकें और उन लोगों ने मेरी आयतों को जिस (अज़ाब से) ये लोग डराए गए हॅसी ठ्ठ्ठा (मज़ाक) बना रखा है
\end{hindi}}
\flushright{\begin{Arabic}
\quranayah[18][57]
\end{Arabic}}
\flushleft{\begin{hindi}
और उससे बढ़कर और कौन ज़ालिम होगा जिसको ख़ुदा की आयतें याद दिलाई जाए और वह उनसे रद गिरदानी (मुँह फेर ले) करे और अपने पहले करतूतों को जो उसके हाथों ने किए हैं भूल बैठे (गोया) हमने खुद उनके दिलों पर परदे डाल दिए हैं कि वह (हक़ बात को) न समझ सकें और (गोया) उनके कानों में गिरानी पैदा कर दी है कि (सुन न सकें) और अगर तुम उनको राहे रास्त की तरफ़ बुलाओ भी तो ये हरगिज़ कभी रुबरु होने वाले नहीं हैं
\end{hindi}}
\flushright{\begin{Arabic}
\quranayah[18][58]
\end{Arabic}}
\flushleft{\begin{hindi}
और (ऐ रसूल) तुम्हारा परवरदिगार तो बड़ा बख्शने वाला मेहरबान है अगर उनकी करतूतों की सज़ा में धर पकड़ करता तो फौरन (दुनिया ही में) उन पर अज़ाब नाज़िल कर देता मगर उनके लिए तो एक मियाद (मुक़र्रर) है जिससे खुदा के सिवा कहीें पनाह की जगह न पाएंगें
\end{hindi}}
\flushright{\begin{Arabic}
\quranayah[18][59]
\end{Arabic}}
\flushleft{\begin{hindi}
और ये बस्तियाँ (जिन्हें तुम अपनी ऑंखों से देखते हो) जब उन लोगों ने सरकशी तो हमने उन्हें हलाक कर मारा और हमने उनकी हलाकत की मियाद मुक़र्रर कर दी थी
\end{hindi}}
\flushright{\begin{Arabic}
\quranayah[18][60]
\end{Arabic}}
\flushleft{\begin{hindi}
(ऐ रसूल) वह वाक़या याद करो जब मूसा खिज़्र की मुलाक़ात को चले तो अपने जवान वसी यूशा से बोले कि जब तक में दोनों दरियाओं के मिलने की जगह न पहुँच जाऊँ (चलने से) बाज़ न आऊँगा
\end{hindi}}
\flushright{\begin{Arabic}
\quranayah[18][61]
\end{Arabic}}
\flushleft{\begin{hindi}
ख्वाह (अगर मुलाक़ात न हो तो) बरसों यूँ ही चलता जाऊँगा फिर जब ये दोनों उन दोनों दरियाओं के मिलने की जगह पहुँचे तो अपनी (भुनी हुई) मछली छोड़ चले तो उसने दरिया में सुरंग बनाकर अपनी राह ली
\end{hindi}}
\flushright{\begin{Arabic}
\quranayah[18][62]
\end{Arabic}}
\flushleft{\begin{hindi}
फिर जब कुछ और आगे बढ़ गए तो मूसा ने अपने जवान (वसी) से कहा (अजी हमारा नाश्ता तो हमें दे दो हमारे (आज के) इस सफर से तो हमको बड़ी थकन हो गई
\end{hindi}}
\flushright{\begin{Arabic}
\quranayah[18][63]
\end{Arabic}}
\flushleft{\begin{hindi}
(यूशा ने) कहा क्या आप ने देखा भी कि जब हम लोग (दरिया के किनारे) उस पत्थर के पास ठहरे तो मै (उसी जगह) मछली छोड़ आया और मुझे आप से उसका ज़िक्र करना शैतान ने भुला दिया और मछली ने अजीब तरह से दरिया में अपनी राह ली
\end{hindi}}
\flushright{\begin{Arabic}
\quranayah[18][64]
\end{Arabic}}
\flushleft{\begin{hindi}
मूसा ने कहा वही तो वह (जगह) है जिसकी हम जुस्तजू (तलाश) में थे फिर दोनों अपने क़दम के निशानों पर देखते देखते उलटे पॉव फिरे
\end{hindi}}
\flushright{\begin{Arabic}
\quranayah[18][65]
\end{Arabic}}
\flushleft{\begin{hindi}
तो (जहाँ मछली थी) दोनों ने हमारे बन्दों में से एक (ख़ास) बन्दा खिज्र को पाया जिसको हमने अपनी बारगाह से रहमत (विलायत) का हिस्सा अता किया था
\end{hindi}}
\flushright{\begin{Arabic}
\quranayah[18][66]
\end{Arabic}}
\flushleft{\begin{hindi}
और हमने उसे इल्म लदुन्नी (अपने ख़ास इल्म) में से कुछ सिखाया था मूसा ने उन (ख़िज्र) से कहा क्या (आपकी इजाज़त है कि) मै इस ग़रज़ से आपके साथ साथ रहूँ
\end{hindi}}
\flushright{\begin{Arabic}
\quranayah[18][67]
\end{Arabic}}
\flushleft{\begin{hindi}
कि जो रहनुमाई का इल्म आपको है (ख़ुदा की तरफ से) सिखाया गया है उसमें से कुछ मुझे भी सिखा दीजिए खिज्र ने कहा (मै सिखा दूँगा मगर) आपसे मेरे साथ सब्र न हो सकेगा
\end{hindi}}
\flushright{\begin{Arabic}
\quranayah[18][68]
\end{Arabic}}
\flushleft{\begin{hindi}
और (सच तो ये है) जो चीज़ आपके इल्मी अहाते से बाहर हो
\end{hindi}}
\flushright{\begin{Arabic}
\quranayah[18][69]
\end{Arabic}}
\flushleft{\begin{hindi}
उस पर आप सब्र क्योंकर कर सकते हैं मूसा ने कहा (आप इत्मिनान रखिए) अगर ख़ुदा ने चाहा तो आप मुझे साबिर आदमी पाएँगें
\end{hindi}}
\flushright{\begin{Arabic}
\quranayah[18][70]
\end{Arabic}}
\flushleft{\begin{hindi}
और मै आपके किसी हुक्म की नाफरमानी न करुँगा खिज्र ने कहा अच्छा तो अगर आप को मेरे साथ रहना है तो जब तक मै खुद आपसे किसी बात का ज़िक्र न छेडँ
\end{hindi}}
\flushright{\begin{Arabic}
\quranayah[18][71]
\end{Arabic}}
\flushleft{\begin{hindi}
आप मुझसे किसी चीज़ के बारे में न पूछियेगा ग़रज़ ये दोनो (मिलकर) चल खड़े हुए यहाँ तक कि (एक दरिया में) जब दोनों कश्ती में सवार हुए तो ख़िज्र ने कश्ती में छेद कर दिया मूसा ने कहा (आप ने तो ग़ज़ब कर दिया) क्या कश्ती में इस ग़रज़ से सुराख़ किया है
\end{hindi}}
\flushright{\begin{Arabic}
\quranayah[18][72]
\end{Arabic}}
\flushleft{\begin{hindi}
कि लोगों को डुबा दीजिए ये तो आप ने बड़ी अजीब बात की है-ख़िज्र ने कहा क्या मैने आप से (पहले ही) न कह दिया था
\end{hindi}}
\flushright{\begin{Arabic}
\quranayah[18][73]
\end{Arabic}}
\flushleft{\begin{hindi}
कि आप मेरे साथ हरगिज़ सब्र न कर सकेगे-मूसा ने कहा अच्छा जो हुआ सो हुआ आप मेरी गिरफत न कीजिए और मुझ पर मेरे इस मामले में इतनी सख्ती न कीजिए
\end{hindi}}
\flushright{\begin{Arabic}
\quranayah[18][74]
\end{Arabic}}
\flushleft{\begin{hindi}
(ख़ैर ये तो हो गया) फिर दोनों के दोनों आगे चले यहाँ तक कि दोनों एक लड़के से मिले तो उस बन्दे ख़ुदा ने उसे जान से मार डाला मूसा ने कहा (ऐ माज़ अल्लाह) क्या आपने एक मासूम शख़्श को मार डाला और वह भी किसी के (ख़ौफ के) बदले में नहीं आपने तो यक़ीनी एक अजीब हरकत की
\end{hindi}}
\flushright{\begin{Arabic}
\quranayah[18][75]
\end{Arabic}}
\flushleft{\begin{hindi}
खिज्र ने कहा कि मैंने आपसे (मुक़र्रर) न कह दिया था कि आप मेरे साथ हरगिज़ नहीं सब्र कर सकेगें
\end{hindi}}
\flushright{\begin{Arabic}
\quranayah[18][76]
\end{Arabic}}
\flushleft{\begin{hindi}
मूसा ने कहा (ख़ैर जो हुआ वह हुआ) अब अगर मैं आप से किसी चीज़ के बारे में पूछगछ करूँगा तो आप मुझे अपने साथ न रखियेगा बेशक आप मेरी तरफ से माज़रत (की हद को) पहुँच गए
\end{hindi}}
\flushright{\begin{Arabic}
\quranayah[18][77]
\end{Arabic}}
\flushleft{\begin{hindi}
ग़रज़ (ये सब हो हुआ कर फिर) दोनों आगे चले यहाँ तक कि जब एक गाँव वालों के पास पहुँचे तो वहाँ के लोगों से कुछ खाने को माँगा तो उन लोगों ने दोनों को मेहमान बनाने से इन्कार कर दिया फिर उन दोनों ने उसी गाँव में एक दीवार को देखा कि गिरा ही चाहती थी तो खिज्र ने उसे सीधा खड़ा कर दिया उस पर मूसा ने कहा अगर आप चाहते तो (इन लोगों से) इसकी मज़दूरी ले सकते थे
\end{hindi}}
\flushright{\begin{Arabic}
\quranayah[18][78]
\end{Arabic}}
\flushleft{\begin{hindi}
(ताकि खाने का सहारा होता) खिज्र ने कहा मेरे और आपके दरमियान छुट्टम छुट्टा अब जिन बातों पर आप से सब्र न हो सका मैं अभी आप को उनकी असल हक़ीकत बताए देता हूँ
\end{hindi}}
\flushright{\begin{Arabic}
\quranayah[18][79]
\end{Arabic}}
\flushleft{\begin{hindi}
(लीजिए सुनिये) वह कश्ती (जिसमें मैंने सुराख़ कर दिया था) तो चन्द ग़रीबों की थी जो दरिया में मेहनत करके गुज़ारा करते थे मैंने चाहा कि उसे ऐबदार बना दूँ (क्योंकि) उनके पीछे-पीछे एक (ज़ालिम) बादशाह (आता) था कि तमाम कश्तियां ज़बरदस्ती बेगार में पकड़ लेता था
\end{hindi}}
\flushright{\begin{Arabic}
\quranayah[18][80]
\end{Arabic}}
\flushleft{\begin{hindi}
और वह जो लड़का जिसको मैंने मार डाला तो उसके माँ बाप दोनों (सच्चे) ईमानदार हैं तो मुझको ये अन्देशा हुआ कि (ऐसा न हो कि बड़ा होकर) उनको भी अपने सरकशी और कुफ़्र में फँसा दे
\end{hindi}}
\flushright{\begin{Arabic}
\quranayah[18][81]
\end{Arabic}}
\flushleft{\begin{hindi}
तो हमने चाहा कि (हम उसको मार डाले और) उनका परवरदिगार इसके बदले में ऐसा फरज़न्द अता फरमाए जो उससे पाक नफ़सी और पाक कराबत में बेहतर हो
\end{hindi}}
\flushright{\begin{Arabic}
\quranayah[18][82]
\end{Arabic}}
\flushleft{\begin{hindi}
और वह जो दीवार थी (जिसे मैंने खड़ा कर दिया) तो वह शहर के दो यतीम लड़कों की थी और उसके नीचे उन्हीं दोनों लड़कों का ख़ज़ाना (गड़ा हुआ था) और उन लड़कों का बाप एक नेक आदमी था तो तुम्हारे परवरदिगार ने चाहा कि दोनों लड़के अपनी जवानी को पहुँचे तो तुम्हारे परवरदिगार की मेहरबानी से अपना ख़ज़ाने निकाल ले और मैंने (जो कुछ किया) कुछ अपने एख्तियार से नहीं किया (बल्कि खुदा के हुक्म से) ये हक़ीक़त है उन वाक़यात की जिन पर आपसे सब्र न हो सका
\end{hindi}}
\flushright{\begin{Arabic}
\quranayah[18][83]
\end{Arabic}}
\flushleft{\begin{hindi}
और (ऐ रसूल) तुमसे लोग ज़ुलक़रनैन का हाल (इम्तेहान) पूछा करते हैं तुम उनके जवाब में कह दो कि मैं भी तुम्हें उसका कुछ हाल बता देता हूँ
\end{hindi}}
\flushright{\begin{Arabic}
\quranayah[18][84]
\end{Arabic}}
\flushleft{\begin{hindi}
(ख़ुदा फरमाता है कि) बेशक हमने उनको ज़मीन पर कुदरतें हुकूमत अता की थी और हमने उसे हर चीज़ के साज़ व सामान दे रखे थे
\end{hindi}}
\flushright{\begin{Arabic}
\quranayah[18][85]
\end{Arabic}}
\flushleft{\begin{hindi}
वह एक सामान (सफर के) पीछे पड़ा
\end{hindi}}
\flushright{\begin{Arabic}
\quranayah[18][86]
\end{Arabic}}
\flushleft{\begin{hindi}
यहाँ तक कि जब (चलते-चलते) आफताब के ग़ुरूब होने की जगह पहुँचा तो आफताब उनको ऐसा दिखाई दिया कि (गोया) वह काली कीचड़ के चश्में में डूब रहा है और उसी चश्में के क़रीब एक क़ौम को भी आबाद पाया हमने कहा ऐ जुलकरनैन (तुमको एख्तियार है) ख्वाह इनके कुफ्र की वजह से इनकी सज़ा करो (कि ईमान लाए) या इनके साथ हुस्ने सुलूक का शेवा एख्तियार करो (कि खुद ईमान क़ुबूल करें)
\end{hindi}}
\flushright{\begin{Arabic}
\quranayah[18][87]
\end{Arabic}}
\flushleft{\begin{hindi}
जुलकरनैन ने अर्ज़ की जो शख्स सरकशी करेगा तो हम उसकी फौरन सज़ा कर देगें (आख़िर) फिर वह (क़यामत में) अपने परवरदिगार के सामने लौटाकर लाया ही जाएगा और वह बुरी से बुरी सज़ा देगा
\end{hindi}}
\flushright{\begin{Arabic}
\quranayah[18][88]
\end{Arabic}}
\flushleft{\begin{hindi}
और जो शख्स ईमान कुबूल करेगा और अच्छे काम करेगा तो (वैसा ही) उसके लिए अच्छे से अच्छा बदला है और हम बहुत जल्द उसे अपने कामों में से आसान काम (करने) को कहेंगे
\end{hindi}}
\flushright{\begin{Arabic}
\quranayah[18][89]
\end{Arabic}}
\flushleft{\begin{hindi}
फिर उस ने एक दूसरी राह एख्तियार की
\end{hindi}}
\flushright{\begin{Arabic}
\quranayah[18][90]
\end{Arabic}}
\flushleft{\begin{hindi}
यहाँ तक कि जब चलते-चलते आफताब के तूलूउ होने की जगह पहुँचा तो (आफताब) से ऐसा ही दिखाई दिया (गोया) कुछ लोगों के सर पर उस तरह तुलूउ कर रहा है जिन के लिए हमने आफताब के सामने कोई आड़ नहीं बनाया था
\end{hindi}}
\flushright{\begin{Arabic}
\quranayah[18][91]
\end{Arabic}}
\flushleft{\begin{hindi}
और था भी ऐसा ही और जुलक़रनैन के पास वो कुछ भी था हमको उससे पूरी वाकफ़ियत थी
\end{hindi}}
\flushright{\begin{Arabic}
\quranayah[18][92]
\end{Arabic}}
\flushleft{\begin{hindi}
(ग़रज़) उसने फिर एक और राह एख्तियार की
\end{hindi}}
\flushright{\begin{Arabic}
\quranayah[18][93]
\end{Arabic}}
\flushleft{\begin{hindi}
यहाँ तक कि जब चलते-चलते रोम में एक पहाड़ के (कंगुरों के) दीवारों के बीचो बीच पहुँच गया तो उन दोनों दीवारों के इस तरफ एक क़ौम को (आबाद) पाया तो बात चीत कुछ समझ ही नहीं सकती थी
\end{hindi}}
\flushright{\begin{Arabic}
\quranayah[18][94]
\end{Arabic}}
\flushleft{\begin{hindi}
उन लोगों ने मुतरज्जिम के ज़रिए से अर्ज़ की ऐ ज़ुलकरनैन (इसी घाटी के उधर याजूज माजूज की क़ौम है जो) मुल्क में फ़साद फैलाया करते हैं तो अगर आप की इजाज़त हो तो हम लोग इस ग़र्ज़ से आपसे पास चन्दा जमा करें कि आप हमारे और उनके दरमियान कोई दीवार बना दें
\end{hindi}}
\flushright{\begin{Arabic}
\quranayah[18][95]
\end{Arabic}}
\flushleft{\begin{hindi}
जुलकरनैन ने कहा कि मेरे परवरदिगार ने ख़र्च की जो कुदरत मुझे दे रखी है वह (तुम्हारे चन्दे से) कहीं बेहतर है (माल की ज़रूरत नहीं) तुम फक़त मुझे क़ूवत से मदद दो तो मैं तुम्हारे और उनके दरमियान एक रोक बना दूँ
\end{hindi}}
\flushright{\begin{Arabic}
\quranayah[18][96]
\end{Arabic}}
\flushleft{\begin{hindi}
(अच्छा तो) मुझे (कहीं से) लोहे की सिले ला दो (चुनान्चे वह लोग) लाए और एक बड़ी दीवार बनाई यहाँ तक कि जब दोनो कंगूरो के दरमेयान (दीवार) को बुलन्द करके उनको बराबर कर दिया तो उनको हुक्म दिया कि इसके गिर्द आग लगाकर धौको यहां तक उसको (धौंकते-धौंकते) लाल अंगारा बना दिया
\end{hindi}}
\flushright{\begin{Arabic}
\quranayah[18][97]
\end{Arabic}}
\flushleft{\begin{hindi}
तो कहा कि अब हमको ताँबा दो कि इसको पिघलाकर इस दीवार पर उँडेल दें (ग़रज़) वह ऐसी ऊँची मज़बूत दीवार बनी कि न तो याजूज व माजूज उस पर चढ़ ही सकते थे और न उसमें नक़ब लगा सकते थे
\end{hindi}}
\flushright{\begin{Arabic}
\quranayah[18][98]
\end{Arabic}}
\flushleft{\begin{hindi}
जुलक़रनैन ने (दीवार को देखकर) कहा ये मेरे परवरदिगार की मेहरबानी है मगर जब मेरे परवरदिगार का वायदा (क़यामत) आयेगा तो इसे ढहा कर हमवार कर देगा और मेरे परवरदिगार का वायदा सच्चा है
\end{hindi}}
\flushright{\begin{Arabic}
\quranayah[18][99]
\end{Arabic}}
\flushleft{\begin{hindi}
और हम उस दिन (उन्हें उनकी हालत पर) छोड़ देंगे कि एक दूसरे में (टकरा के दरिया की) लहरों की तरह गुड़मुड़ हो जाएँ और सूर फूँका जाएगा तो हम सब को इकट्ठा करेंगे
\end{hindi}}
\flushright{\begin{Arabic}
\quranayah[18][100]
\end{Arabic}}
\flushleft{\begin{hindi}
और उसी दिन जहन्नुम को उन काफिरों के सामने खुल्लम खुल्ला पेश करेंगे
\end{hindi}}
\flushright{\begin{Arabic}
\quranayah[18][101]
\end{Arabic}}
\flushleft{\begin{hindi}
और उसी (रसूल की दुश्मनी की सच्ची बात) कुछ भी सुन ही न सकते थे
\end{hindi}}
\flushright{\begin{Arabic}
\quranayah[18][102]
\end{Arabic}}
\flushleft{\begin{hindi}
तो क्या जिन लोगों ने कुफ्र एख्तियार किया इस ख्याल में हैं कि हमको छोड़कर हमारे बन्दों को अपना सरपरस्त बना लें (कुछ पूछगछ न होगी) (अच्छा सुनो) हमने काफिरों की मेहमानदारी के लिए जहन्नुम तैयार कर रखी है
\end{hindi}}
\flushright{\begin{Arabic}
\quranayah[18][103]
\end{Arabic}}
\flushleft{\begin{hindi}
(ऐ रसूल) तुम कह दो कि क्या हम उन लोगों का पता बता दें जो लोग आमाल की हैसियत से बहुत घाटे में हैं
\end{hindi}}
\flushright{\begin{Arabic}
\quranayah[18][104]
\end{Arabic}}
\flushleft{\begin{hindi}
(ये) वह लोग (हैं) जिन की दुनियावी ज़िन्दगी की राई (कोशिश सब) अकारत हो गई और वह उस ख़ाम ख्याल में हैं कि वह यक़ीनन अच्छे-अच्छे काम कर रहे हैं
\end{hindi}}
\flushright{\begin{Arabic}
\quranayah[18][105]
\end{Arabic}}
\flushleft{\begin{hindi}
यही वह लोग हैं जिन्होंने अपने परवरदिगार की आयातों से और (क़यामत के दिन) उसके सामने हाज़िर होने से इन्कार किया तो उनका सब किया कराया अकारत हुआ तो हम उसके लिए क़यामत के दिन मीजान हिसाब भी क़ायम न करेंगे
\end{hindi}}
\flushright{\begin{Arabic}
\quranayah[18][106]
\end{Arabic}}
\flushleft{\begin{hindi}
(और सीधे जहन्नुम में झोंक देगें) ये जहन्नुम उनकी करतूतों का बदला है कि उन्होंने कुफ्र एख्तियार किया और मेरी आयतों और मेरे रसूलों को हँसी ठठ्ठा बना लिया
\end{hindi}}
\flushright{\begin{Arabic}
\quranayah[18][107]
\end{Arabic}}
\flushleft{\begin{hindi}
बेशक जिन लोगों ने ईमान क़ुबूल किया और अच्छे-अच्छे काम किये उनकी मेहमानदारी के लिए फिरदौस (बरी) के बाग़ात होंगे जिनमें वह हमेशा रहेंगे
\end{hindi}}
\flushright{\begin{Arabic}
\quranayah[18][108]
\end{Arabic}}
\flushleft{\begin{hindi}
और वहाँ से हिलने की भी ख्वाहिश न करेंगे
\end{hindi}}
\flushright{\begin{Arabic}
\quranayah[18][109]
\end{Arabic}}
\flushleft{\begin{hindi}
(ऐ रसूल उन लोगों से) कहो कि अगर मेरे परवरदिगार की बातों के (लिखने के) वास्ते समन्दर (का पानी) भी सियाही बन जाए तो क़ब्ल उसके कि मेरे परवरदिगार की बातें ख़त्म हों समन्दर ही ख़त्म हो जाएगा अगरचे हम वैसा ही एक समन्दर उस की मदद को लाँए
\end{hindi}}
\flushright{\begin{Arabic}
\quranayah[18][110]
\end{Arabic}}
\flushleft{\begin{hindi}
(ऐ रसूल) कह दो कि मैं भी तुम्हारा ही ऐसा एक आदमी हूँ (फर्क़ इतना है) कि मेरे पास ये वही आई है कि तुम्हारे माबूद यकता माबूद हैं तो वो शख्स आरज़ूमन्द होकर अपने परवरदिगार के सामने हाज़िर होगा तो उसे अच्छे काम करने चाहिए और अपने परवरदिगार की इबादत में किसी को शरीक न करें
\end{hindi}}
\chapter{Maryam (Mary)}
\begin{Arabic}
\Huge{\centerline{\basmalah}}\end{Arabic}
\flushright{\begin{Arabic}
\quranayah[19][1]
\end{Arabic}}
\flushleft{\begin{hindi}
काफ़ हा या ऐन साद
\end{hindi}}
\flushright{\begin{Arabic}
\quranayah[19][2]
\end{Arabic}}
\flushleft{\begin{hindi}
ये तुम्हारे परवरदिगार की मेहरबानी का ज़िक्र है जो (उसने) अपने ख़ास बन्दे ज़करिया के साथ की थी
\end{hindi}}
\flushright{\begin{Arabic}
\quranayah[19][3]
\end{Arabic}}
\flushleft{\begin{hindi}
कि जब ज़करिया ने अपने परवरदिगार को धीमी आवाज़ से पुकारा
\end{hindi}}
\flushright{\begin{Arabic}
\quranayah[19][4]
\end{Arabic}}
\flushleft{\begin{hindi}
(और) अर्ज़ की ऐ मेरे पालने वाले मेरी हड्डियां कमज़ोर हो गई और सर है कि बुढ़ापे की (आग से) भड़क उठा (सेफद हो गया) है और ऐ मेरे पालने वाले मैं तेरी बारगाह में दुआ कर के कभी महरूम नहीं रहा हूँ
\end{hindi}}
\flushright{\begin{Arabic}
\quranayah[19][5]
\end{Arabic}}
\flushleft{\begin{hindi}
और मैं अपने (मरने के) बाद अपने वारिसों से सहम जाता हूँ (कि मुबादा दीन को बरबाद करें) और मेरी बीबी उम्मे कुलसूम बिनते इमरान बांझ है पस तू मुझको अपनी बारगाह से एक जाँनशीन फरज़न्द अता फ़रमा
\end{hindi}}
\flushright{\begin{Arabic}
\quranayah[19][6]
\end{Arabic}}
\flushleft{\begin{hindi}
जो मेरी और याकूब की नस्ल की मीरास का मालिक हो ऐ मेरे परवरदिगार और उसको अपना पसन्दीदा बन्दा बना
\end{hindi}}
\flushright{\begin{Arabic}
\quranayah[19][7]
\end{Arabic}}
\flushleft{\begin{hindi}
खुदा ने फरमाया हम तुमको एक लड़के की खुशख़बरी देते हैं जिसका नाम यहया होगा और हमने उससे पहले किसी को उसका हमनाम नहीं पैदा किया
\end{hindi}}
\flushright{\begin{Arabic}
\quranayah[19][8]
\end{Arabic}}
\flushleft{\begin{hindi}
ज़करिया ने अर्ज़ की या इलाही (भला) मुझे लड़का क्योंकर होगा और हालत ये है कि मेरी बीवी बाँझ है और मैं खुद हद से ज्यादा बुढ़ापे को पहुँच गया हूँ
\end{hindi}}
\flushright{\begin{Arabic}
\quranayah[19][9]
\end{Arabic}}
\flushleft{\begin{hindi}
(खुदा ने) फ़रमाया ऐसा ही होगा तुम्हारा परवरदिगार फ़रमाता है कि ये बात हम पर (कुछ दुशवार नहीं) आसान है और (तुम अपने को तो ख्याल करो कि) इससे पहले तुमको पैदा किया हालाँकि तुम कुछ भी न थे
\end{hindi}}
\flushright{\begin{Arabic}
\quranayah[19][10]
\end{Arabic}}
\flushleft{\begin{hindi}
ज़करिया ने अर्ज़ की इलाही मेरे लिए कोई अलामत मुक़र्रर कर दें हुक्म हुआ तुम्हारी पहचान ये है कि तुम तीन रात (दिन) बराबर लोगों से बात नहीं कर सकोगे
\end{hindi}}
\flushright{\begin{Arabic}
\quranayah[19][11]
\end{Arabic}}
\flushleft{\begin{hindi}
फिर ज़करिया (अपने इबादत के) हुजरे से अपनी क़ौम के पास (हिदायत देने के लिए) निकले तो उन से इशारा किया कि तुम लोग सुबह व शाम बराबर उसकी तसबीह (व तक़दीस) किया करो
\end{hindi}}
\flushright{\begin{Arabic}
\quranayah[19][12]
\end{Arabic}}
\flushleft{\begin{hindi}
(ग़रज़ यहया पैदा हुए और हमने उनसे कहा) ऐ यहया किताब (तौरेत) मज़बूती के साथ लो
\end{hindi}}
\flushright{\begin{Arabic}
\quranayah[19][13]
\end{Arabic}}
\flushleft{\begin{hindi}
और हमने उन्हें बचपन ही में अपनी बारगाह से नुबूवत और रहमदिली और पाक़ीज़गी अता फरमाई
\end{hindi}}
\flushright{\begin{Arabic}
\quranayah[19][14]
\end{Arabic}}
\flushleft{\begin{hindi}
और वह (ख़ुद भी) परहेज़गार और अपने माँ बाप के हक़ में सआदतमन्द थे और सरकश नाफरमान न थे
\end{hindi}}
\flushright{\begin{Arabic}
\quranayah[19][15]
\end{Arabic}}
\flushleft{\begin{hindi}
और (हमारी तरफ से) उन पर (बराबर) सलाम है जिस दिन पैदा हुए और जिस दिन मरेंगे और जिस दिन (दोबारा) ज़िन्दा उठा खड़े किए जाएँगे
\end{hindi}}
\flushright{\begin{Arabic}
\quranayah[19][16]
\end{Arabic}}
\flushleft{\begin{hindi}
और (ऐ रसूल) कुरान में मरियम का भी तज़किरा करो कि जब वह अपने लोगों से अलग होकर पूरब की तरफ़ वाले मकान में (गुस्ल के वास्ते) जा बैठें
\end{hindi}}
\flushright{\begin{Arabic}
\quranayah[19][17]
\end{Arabic}}
\flushleft{\begin{hindi}
फिर उसने उन लोगों से परदा कर लिया तो हमने अपनी रूह (जिबरील) को उन के पास भेजा तो वह अच्छे ख़ासे आदमी की सूरत बनकर उनके सामने आ खड़ा हुआ
\end{hindi}}
\flushright{\begin{Arabic}
\quranayah[19][18]
\end{Arabic}}
\flushleft{\begin{hindi}
(वह उसको देखकर घबराई और) कहने लगी अगर तू परहेज़गार है तो मैं तुझ से खुदा की पनाह माँगती हूँ
\end{hindi}}
\flushright{\begin{Arabic}
\quranayah[19][19]
\end{Arabic}}
\flushleft{\begin{hindi}
(मेरे पास से हट जा) जिबरील ने कहा मैं तो साफ़ तुम्हारे परवरदिगार का पैग़मबर (फ़रिश्ता) हूँ ताकि तुमको पाक व पाकीज़ा लड़का अता करूँ
\end{hindi}}
\flushright{\begin{Arabic}
\quranayah[19][20]
\end{Arabic}}
\flushleft{\begin{hindi}
मरियम ने कहा मुझे लड़का क्योंकर हो सकता है हालाँकि किसी मर्द ने मुझे छुआ तक नहीं है औ मैं न बदकार हूँ
\end{hindi}}
\flushright{\begin{Arabic}
\quranayah[19][21]
\end{Arabic}}
\flushleft{\begin{hindi}
जिबरील ने कहा तुमने कहा ठीक (मगर) तुम्हारे परवरदिगार ने फ़रमाया है कि ये बात (बे बाप के लड़का पैदा करना) मुझ पर आसान है ताकि इसको (पैदा करके) लोगों के वास्ते (अपनी क़ुदरत की) निशानी क़रार दें और अपनी ख़ास रहमत का ज़रिया बनायें
\end{hindi}}
\flushright{\begin{Arabic}
\quranayah[19][22]
\end{Arabic}}
\flushleft{\begin{hindi}
और ये बात फैसला शुदा है ग़रज़ लड़के के साथ वह आप ही आप हामेला हो गई फिर इसकी वजह से लोगों से अलग एक दूर के मकान में चली गई
\end{hindi}}
\flushright{\begin{Arabic}
\quranayah[19][23]
\end{Arabic}}
\flushleft{\begin{hindi}
फिर (जब जनने का वक्त ़क़रीब आया तो दरदे ज़ह) उन्हें एक खजूर के (सूखे) दरख्त की जड़ में ले आया और (बेकसी में शर्म से) कहने लगीं काश मैं इससे पहले मर जाती और (न पैद होकर)
\end{hindi}}
\flushright{\begin{Arabic}
\quranayah[19][24]
\end{Arabic}}
\flushleft{\begin{hindi}
बिल्कुल भूली बिसरी हो जाती तब जिबरील ने मरियम के पाईन की तरफ़ से आवाज़ दी कि तुम कुढ़ों नहीं देखो तो तुम्हारे परवरदिगार ने तुम्हारे क़रीब ही नीचे एक चश्मा जारी कर दिया है
\end{hindi}}
\flushright{\begin{Arabic}
\quranayah[19][25]
\end{Arabic}}
\flushleft{\begin{hindi}
और खुरमे की जड़ (पकड़ कर) अपनी तरफ़ हिलाओ तुम पर पक्के-पक्के ताजे खुरमे झड़ पड़ेगें फिर (शौक़ से खुरमे) खाओ
\end{hindi}}
\flushright{\begin{Arabic}
\quranayah[19][26]
\end{Arabic}}
\flushleft{\begin{hindi}
और (चश्मे का पानी) पियो और (लड़के से) अपनी ऑंख ठन्डी करो फिर अगर तुम किसी आदमी को देखो (और वह तुमसे कुछ पूछे) तो तुम इशारे से कह देना कि मैंने खुदा के वास्ते रोजे क़ी नज़र की थी तो मैं आज हरगिज़ किसी से बात नहीं कर सकती
\end{hindi}}
\flushright{\begin{Arabic}
\quranayah[19][27]
\end{Arabic}}
\flushleft{\begin{hindi}
फिर मरियम उस लड़के को अपनी गोद में लिए हुए अपनी क़ौम के पास आयीं वह लोग देखकर कहने लगे ऐ मरियम तुमने तो यक़ीनन बहुत बुरा काम किया
\end{hindi}}
\flushright{\begin{Arabic}
\quranayah[19][28]
\end{Arabic}}
\flushleft{\begin{hindi}
ऐ हारून की बहन न तो तेरा बाप ही बुरा आदमी था और न तो तेरी माँ ही बदकार थी (ये तूने क्या किया)
\end{hindi}}
\flushright{\begin{Arabic}
\quranayah[19][29]
\end{Arabic}}
\flushleft{\begin{hindi}
तो मरियम ने उस लड़के की तरफ इशारा किया ( कि जो कुछ पूछना है इससे पूछ लो) और वह लोग बोले भला हम गोद के बच्चे से क्योंकर बात करें
\end{hindi}}
\flushright{\begin{Arabic}
\quranayah[19][30]
\end{Arabic}}
\flushleft{\begin{hindi}
(इस पर वह बच्चा कुदरते खुदा से) बोल उठा कि मैं बेशक खुदा का बन्दा हूँ मुझ को उसी ने किताब (इन्जील) अता फरमाई है और मुझ को नबी बनाया
\end{hindi}}
\flushright{\begin{Arabic}
\quranayah[19][31]
\end{Arabic}}
\flushleft{\begin{hindi}
और मै (चाहे) कहीं रहूँ मुझ को मुबारक बनाया और मुझ को जब तक ज़िन्दा रहूँ नमाज़ पढ़ने ज़कात देने की ताकीद की है और मुझ को अपनी वालेदा का फ़रमाबरदार बनाया
\end{hindi}}
\flushright{\begin{Arabic}
\quranayah[19][32]
\end{Arabic}}
\flushleft{\begin{hindi}
और (अलहमदोलिल्लाह कि) मुझको सरकश नाफरमान नहीं बनाया
\end{hindi}}
\flushright{\begin{Arabic}
\quranayah[19][33]
\end{Arabic}}
\flushleft{\begin{hindi}
और (खुदा की तरफ़ से) जिस दिन मैं पैदा हुआ हूँ और जिस दिन मरूँगा मुझ पर सलाम है और जिस दिन (दोबारा) ज़िन्दा उठा कर खड़ा किया जाऊँगा
\end{hindi}}
\flushright{\begin{Arabic}
\quranayah[19][34]
\end{Arabic}}
\flushleft{\begin{hindi}
ये है कि मरियम के बेटे ईसा का सच्चा (सच्चा) क़िस्सा जिसमें ये लोग (ख्वाहमख्वाह) शक किया करते हैं
\end{hindi}}
\flushright{\begin{Arabic}
\quranayah[19][35]
\end{Arabic}}
\flushleft{\begin{hindi}
खुदा कि लिए ये किसी तरह सज़ावार नहीं कि वह (किसी को) बेटा बनाए वह पाक व पकीज़ा है जब वह किसी काम का करना ठान लेता है तो बस उसको कह देता है कि ''हो जा'' तो वह हो जाता है
\end{hindi}}
\flushright{\begin{Arabic}
\quranayah[19][36]
\end{Arabic}}
\flushleft{\begin{hindi}
और इसमें तो शक ही नहीं कि खुदा (ही) मेरा (भी) परवरदिगार है और तुम्हारा (भी) परवरदिगार है तो सब के सब उसी की इबादत करो यही (तौहीद) सीधा रास्ता है
\end{hindi}}
\flushright{\begin{Arabic}
\quranayah[19][37]
\end{Arabic}}
\flushleft{\begin{hindi}
(और यही दीन ईसा लेकर आए थे) फिर (काफिरों के) फ़िरकों ने बहम एख़तेलाफ किया तो जिन लोगों ने कुफ्र इख़्तियार किया उनके लिए बड़े (सख्त दिन खुदा के हुज़ूर) हाज़िर होने से ख़राबी है
\end{hindi}}
\flushright{\begin{Arabic}
\quranayah[19][38]
\end{Arabic}}
\flushleft{\begin{hindi}
जिस दिन ये लोग हमारे हुज़ूर में हाज़िर होंगे क्या कुछ सुनते देखते होंगे मगर आज तो नफ़रमान लोग खुल्लम खुल्ला गुमराही में हैं
\end{hindi}}
\flushright{\begin{Arabic}
\quranayah[19][39]
\end{Arabic}}
\flushleft{\begin{hindi}
और (ऐ रसूल) तुम उनको हसरत (अफ़सोस) के दिन से डराओ जब क़तई फैसला कर दिया जाएगा और (इस वक्त तो) ये लोग ग़फलत में (पड़े हैं)
\end{hindi}}
\flushright{\begin{Arabic}
\quranayah[19][40]
\end{Arabic}}
\flushleft{\begin{hindi}
और ईमान नहीं लाते इसमें शक नहीं कि (एक दिन) ज़मीन के और जो कुछ उस पर है (उसके) हम ही वारिस होंगे
\end{hindi}}
\flushright{\begin{Arabic}
\quranayah[19][41]
\end{Arabic}}
\flushleft{\begin{hindi}
(और सब नेस्त व नाबूद हो जाएँगे) और सब के सब हमारी तरफ़ लौटाए जाएँगे और (ऐ रसूल) क़ुरान में इबराहीम का (भी) तज़किरा करो
\end{hindi}}
\flushright{\begin{Arabic}
\quranayah[19][42]
\end{Arabic}}
\flushleft{\begin{hindi}
इसमें शक नहीं कि वह बड़े सच्चे नबी थे जब उन्होंने अपने चचा और मुँह बोले बाप से कहा कि ऐ अब्बा आप क्यों, ऐसी चीज़ (बुत) की परसतिश करते हैं जो ने सुन सकता है और न देख सकता है
\end{hindi}}
\flushright{\begin{Arabic}
\quranayah[19][43]
\end{Arabic}}
\flushleft{\begin{hindi}
और न कुछ आपके काम ही आ सकता है ऐ मेरे अब्बा यक़ीनन मेरे पास वह इल्म आ चुका है जो आपके पास नहीं आया तो आप मेरी पैरवी कीजिए मैं आपको (दीन की) सीधी राह दिखा दूँगा
\end{hindi}}
\flushright{\begin{Arabic}
\quranayah[19][44]
\end{Arabic}}
\flushleft{\begin{hindi}
ऐ अब्बा आप शैतान की परसतिश न कीजिए (क्योंकि) शैतान यक़ीनन खुदा का नाफ़रमान (बन्दा) है।
\end{hindi}}
\flushright{\begin{Arabic}
\quranayah[19][45]
\end{Arabic}}
\flushleft{\begin{hindi}
ऐ अब्बा मैं यक़ीनन इससे डरता हूँ कि (मुबादा) खुदा की तरफ से आप पर कोई अज़ाब नाज़िल हो तो (आख़िर) आप शैतान के साथी बन जाईए
\end{hindi}}
\flushright{\begin{Arabic}
\quranayah[19][46]
\end{Arabic}}
\flushleft{\begin{hindi}
(आज़र ने) कहा (क्यों) इबराहीम क्या तू मेरे माबूदों को नहीं मानता है अगर तू (इन बातों से) किसी तरह बाज़ न आएगा तो (याद रहे) मैं तुझे संगसार कर दूँगा और तू मेरे पास से हमेशा के लिए दूर हो जा
\end{hindi}}
\flushright{\begin{Arabic}
\quranayah[19][47]
\end{Arabic}}
\flushleft{\begin{hindi}
इबराहीम ने कहा (अच्छा तो) मेरा सलाम लीजिए (मगर इस पर भी) मैं अपने परवरदिगार से आपकी बख्शिश की दुआ करूँगा
\end{hindi}}
\flushright{\begin{Arabic}
\quranayah[19][48]
\end{Arabic}}
\flushleft{\begin{hindi}
(क्योंकि) बेशक वह मुझ पर बड़ा मेहरबान है और मैंने आप को (भी) और इन बुतों को (भी) जिन्हें आप लोग खुदा को छोड़कर पूजा करते हैं (सबको) छोड़ा और अपने परवरदिगार ही की इबादत करूँगा उम्मीद है कि मैं अपने परवरदिगार की इबादत से महरूम न रहूँगा
\end{hindi}}
\flushright{\begin{Arabic}
\quranayah[19][49]
\end{Arabic}}
\flushleft{\begin{hindi}
ग़रज़ इबराहीम ने उन लोगों को और जिसे ये लोग खुदा को छोड़कर परसतिश किया करते थे छोड़ा तो हमने उन्हें इसहाक़ व याकूब (सी औलाद) अता फ़रमाई और हर एक को नुबूवत के दर्जे पर फ़ायज़ किया
\end{hindi}}
\flushright{\begin{Arabic}
\quranayah[19][50]
\end{Arabic}}
\flushleft{\begin{hindi}
और उन सबको अपनी रहमत से कुछ इनायत फ़रमाया और हमने उनके लिए आला दर्जे का ज़िक्रे ख़ैर (दुनिया में भी) क़रार दिया
\end{hindi}}
\flushright{\begin{Arabic}
\quranayah[19][51]
\end{Arabic}}
\flushleft{\begin{hindi}
और (ऐ रसूल) कुरान में (कुछ) मूसा का (भी) तज़किरा करो इसमें शक नहीं कि वह (मेरा) बन्दा और साहिबे किताब व शरीयत नबी था
\end{hindi}}
\flushright{\begin{Arabic}
\quranayah[19][52]
\end{Arabic}}
\flushleft{\begin{hindi}
और हमने उनको (कोहे तूर) की दाहिनी तरफ़ से आवाज़ दी और हमने उन्हें राज़ व नियाज़ की बातें करने के लिए अपने क़रीब बुलाया
\end{hindi}}
\flushright{\begin{Arabic}
\quranayah[19][53]
\end{Arabic}}
\flushleft{\begin{hindi}
और हमने उन्हें अपनी ख़ास मेहरबानी से उनके भाई हारून को (उनका वज़ीर बनाकर) इनायत फ़रमाया
\end{hindi}}
\flushright{\begin{Arabic}
\quranayah[19][54]
\end{Arabic}}
\flushleft{\begin{hindi}
(ऐ रसूल) कुरान में इसमाईल का (भी) तज़किरा करो इसमें शक नहीं कि वह वायदे के सच्चे थे और भेजे हुए पैग़म्बर थे
\end{hindi}}
\flushright{\begin{Arabic}
\quranayah[19][55]
\end{Arabic}}
\flushleft{\begin{hindi}
और अपने घर के लोगों को नमाज़ पढ़ने और ज़कात देने की ताकीद किया करते थे और अपने परवरदिगार की बारगाह में पसन्दीदा थे
\end{hindi}}
\flushright{\begin{Arabic}
\quranayah[19][56]
\end{Arabic}}
\flushleft{\begin{hindi}
और (ऐ रसूल) कुरान में इदरीस का भी तज़किरा करो इसमें शक नहीं कि वह बड़े सच्चे (बन्दे और) नबी थे
\end{hindi}}
\flushright{\begin{Arabic}
\quranayah[19][57]
\end{Arabic}}
\flushleft{\begin{hindi}
और हमने उनको बहुत ऊँची जगह (बेहिश्त में) बुलन्द कर (के पहुँचा) दिया
\end{hindi}}
\flushright{\begin{Arabic}
\quranayah[19][58]
\end{Arabic}}
\flushleft{\begin{hindi}
ये अम्बिया लोग जिन्हें खुदा ने अपनी नेअमत दी आदमी की औलाद से हैं और उनकी नस्ल से जिन्हें हमने (तूफ़ान के वक्त) नूह के साथ (कश्ती पर) सवारकर लिया था और इबराहीम व याकूब की औलाद से हैं और उन लोगों में से हैं जिनकी हमने हिदायत की और मुन्तिख़ब किया जब उनके सामने खुदा की (नाज़िल की हुई) आयतें पढ़ी जाती थीं तो सजदे में ज़ारोक़तार रोते हुए गिर पड़ते थे (58) सजदा
\end{hindi}}
\flushright{\begin{Arabic}
\quranayah[19][59]
\end{Arabic}}
\flushleft{\begin{hindi}
फिर उनके बाद कुछ नाख़लफ (उनके) जानशीन हुए जिन्होंने नमाज़ें खोयी और नफ़सानी ख्वाहिशों के चेले बन बैठे अनक़रीब ही ये लोग (अपनी) गुमराही (के ख़ामयाजे) से जा मिलेंगे
\end{hindi}}
\flushright{\begin{Arabic}
\quranayah[19][60]
\end{Arabic}}
\flushleft{\begin{hindi}
मगर (हाँ) जिसने तौबा कर लिया और अच्छे-अच्छे काम किए तो ऐसे लोग बेहिश्त में दाख़िल होंगे और उन पर कुछ भी जुल्म नहीं किया जाएगा वह सदाबहार बाग़ात में रहेंगे
\end{hindi}}
\flushright{\begin{Arabic}
\quranayah[19][61]
\end{Arabic}}
\flushleft{\begin{hindi}
जिनका खुदा ने अपने बन्दों से ग़ाएबाना वायदा कर लिया है बेशक उसका वायदा पूरा होने वाला है
\end{hindi}}
\flushright{\begin{Arabic}
\quranayah[19][62]
\end{Arabic}}
\flushleft{\begin{hindi}
वह लोग वहाँ सलाम के सिवा कोई बेहूदा बात सुनेंगे ही नहीं मगर हर तरफ से इस्लाम ही इस्लाम (की आवाज़ आएगी) और वहाँ उनका खाना सुबह व शाम (जिस वक्त चाहेंगे) उनके लिए (तैयार) रहेगा
\end{hindi}}
\flushright{\begin{Arabic}
\quranayah[19][63]
\end{Arabic}}
\flushleft{\begin{hindi}
यही वह बेिहश्त है कि हमारे बन्दों में से जो परहेज़गार होगा हम उसे उसका वारिस बनायेगे
\end{hindi}}
\flushright{\begin{Arabic}
\quranayah[19][64]
\end{Arabic}}
\flushleft{\begin{hindi}
और (ऐ रसूल) हम लोग फ़रिश्ते आप के परवरदिगार के हुक्म के बग़ैर (दुनिया में) नहीं नाज़िल होते जो कुछ हमारे सामने है और जो कुछ हमारे पीठ पीछे है और जो कुछ उनके दरमियान में है (ग़रज़ सबकुछ) उसी का है
\end{hindi}}
\flushright{\begin{Arabic}
\quranayah[19][65]
\end{Arabic}}
\flushleft{\begin{hindi}
और तुम्हारा परवरदिगार कुछ भूलने वाला नहीं है सारे आसमान और ज़मीन का मालिक है और उन चीज़ों का भी जो दोनों के दरमियान में है तो तुम उसकी इबादत करो (और उसकी इबादत पर साबित) क़दम रहो भला तुम्हारे इल्म में उसका कोई हमनाम भी है
\end{hindi}}
\flushright{\begin{Arabic}
\quranayah[19][66]
\end{Arabic}}
\flushleft{\begin{hindi}
और (बाज़) आदमी अबी बिन ख़लफ ताज्जुब से कहा करते हैं कि क्या जब मैं मर जाऊँगा तो जल्दी ही जीता जागता (क़ब्र से) निकाला जाऊँगा
\end{hindi}}
\flushright{\begin{Arabic}
\quranayah[19][67]
\end{Arabic}}
\flushleft{\begin{hindi}
क्या वह (आदमी) उसको नहीं याद करता कि उसको इससे पहले जब वह कुछ भी न था पैदा किया था
\end{hindi}}
\flushright{\begin{Arabic}
\quranayah[19][68]
\end{Arabic}}
\flushleft{\begin{hindi}
तो वह (ऐ रसूल) तुम्हारे परवरदिगार की (अपनी) क़िस्म हम उनको और शैतान को इकट्ठा करेगे फिर उन सब को जहन्नुम के गिर्दागिर्द घुटनों के बल हाज़िर करेंगे
\end{hindi}}
\flushright{\begin{Arabic}
\quranayah[19][69]
\end{Arabic}}
\flushleft{\begin{hindi}
फिर हर गिरोह में से ऐसे लोगों को अलग निकाल लेंगे (जो दुनिया में) खुदा से औरों की निस्बत अकड़े-अकड़े फिरते थे
\end{hindi}}
\flushright{\begin{Arabic}
\quranayah[19][70]
\end{Arabic}}
\flushleft{\begin{hindi}
फिर जो लोग जहन्नुम में झोंके जाएँगे ज्यादा सज़ावार हैं हम उनसे खूब वाक़िफ हैं
\end{hindi}}
\flushright{\begin{Arabic}
\quranayah[19][71]
\end{Arabic}}
\flushleft{\begin{hindi}
और तुममे से कोई ऐसा नहीं जो जहन्नुम पर से होकर न गुज़रे (क्योंकि पुल सिरात उसी पर है) ये तुम्हारे परवरदिगार पर हेतेमी और लाज़मी (वायदा) है
\end{hindi}}
\flushright{\begin{Arabic}
\quranayah[19][72]
\end{Arabic}}
\flushleft{\begin{hindi}
फिर हम परहेज़गारों को बचाएँगे और नाफ़रमानों को घुटने के भल उसमें छोड़ देंगे
\end{hindi}}
\flushright{\begin{Arabic}
\quranayah[19][73]
\end{Arabic}}
\flushleft{\begin{hindi}
और जब हमारी वाज़ेए रौशन आयतें उनके सामने पढ़ी जाती हैं तो जिन लोगों ने कुफ़्र किया ईमानवालों से पूछते हैं भला ये तो बताओ कि हम तुम दोनों फरीक़ो में से मरतबे में कौन ज्यादा बेहतर है और किसकी महफिल ज्यादा अच्छी है
\end{hindi}}
\flushright{\begin{Arabic}
\quranayah[19][74]
\end{Arabic}}
\flushleft{\begin{hindi}
हालाँकि हमने उनसे पहले बहुत सी जमाअतों को हलाक कर छोड़ा जो उनसे साज़ो सामान और ज़ाहिरी नमूद में कहीं बढ़ चढ़ के थी
\end{hindi}}
\flushright{\begin{Arabic}
\quranayah[19][75]
\end{Arabic}}
\flushleft{\begin{hindi}
(ऐ रसूल) कह दो कि जो शख्स गुमराही में पड़ा है तो खुदा उसको ढ़ील ही देता चला जाता है यहाँ तक कि उस चीज़ को (अपनी ऑंखों से) देख लेंगे जिनका उनसे वायदा किया गया है या अज़ाब या क़यामत तो उस वक्त उन्हें मालूम हो जाएगा कि मरतबे में कौन बदतर है और लश्कर (जत्थे) में कौन कमज़ोर है (बेकस) है
\end{hindi}}
\flushright{\begin{Arabic}
\quranayah[19][76]
\end{Arabic}}
\flushleft{\begin{hindi}
और जो लोग राहे रास्त पर हैं खुदा उनकी हिदायत और ज्यादा करता जाता है और बाक़ी रह जाने वाली नेकियाँ तुम्हारे परवरदिगार के नज़दीक सवाब की राह से भी बेहतर है और अन्जाम के ऐतबार से (भी) बेहतर है
\end{hindi}}
\flushright{\begin{Arabic}
\quranayah[19][77]
\end{Arabic}}
\flushleft{\begin{hindi}
(ऐ रसूल) क्या तुमने उस शख्स पर भी नज़र की जिसने हमारी आयतों से इन्कार किया और कहने लगा कि (अगर क़यामत हुईतो भी) मुझे माल और औलाद ज़रूर मिलेगी
\end{hindi}}
\flushright{\begin{Arabic}
\quranayah[19][78]
\end{Arabic}}
\flushleft{\begin{hindi}
क्या उसे ग़ैब का हाल मालूम हो गया है या उसने खुदा से कोई अहद (व पैमान) ले रखा है हरग़िज नहीं
\end{hindi}}
\flushright{\begin{Arabic}
\quranayah[19][79]
\end{Arabic}}
\flushleft{\begin{hindi}
जो कुछ ये बकता है (सब) हम सभी से लिखे लेते हैं और उसके लिए और ज्यादा अज़ाब बढ़ाते हैं
\end{hindi}}
\flushright{\begin{Arabic}
\quranayah[19][80]
\end{Arabic}}
\flushleft{\begin{hindi}
और वो माल व औलाद की निस्बत बक रहा है हम ही उसके मालिक हो बैठेंगे और ये हमारे पास तनहा आयेगा
\end{hindi}}
\flushright{\begin{Arabic}
\quranayah[19][81]
\end{Arabic}}
\flushleft{\begin{hindi}
और उन लोगों ने खुदा को छोड़कर दूसरे-दूसरे माबूद बना रखे हैं ताकि वह उनकी इज्ज़त के बाएस हों हरग़िज़ नहीं
\end{hindi}}
\flushright{\begin{Arabic}
\quranayah[19][82]
\end{Arabic}}
\flushleft{\begin{hindi}
(बल्कि) वह माबूद खुद उनकी इबादत से इन्कार करेंगे और (उल्टे) उनके दुशमन हो जाएँगे
\end{hindi}}
\flushright{\begin{Arabic}
\quranayah[19][83]
\end{Arabic}}
\flushleft{\begin{hindi}
(ऐ रसूल) क्या तुमने इसी बात को नहीं देखा कि हमने शैतान को काफ़िरों पर छोड़ रखा है कि वह उन्हें बहकाते रहते हैं
\end{hindi}}
\flushright{\begin{Arabic}
\quranayah[19][84]
\end{Arabic}}
\flushleft{\begin{hindi}
तो (ऐ रसूल) तुम उन काफिरों पर (नुज़ूले अज़ाब की) जल्दी न करो हम तो बस उनके लिए (अज़ाब) का दिन गिन रहे हैं
\end{hindi}}
\flushright{\begin{Arabic}
\quranayah[19][85]
\end{Arabic}}
\flushleft{\begin{hindi}
कि जिस दिन परहेज़गारों को (खुदाए) रहमान के (अपने) सामने मेहमानों की तरह तरह जमा करेंगे
\end{hindi}}
\flushright{\begin{Arabic}
\quranayah[19][86]
\end{Arabic}}
\flushleft{\begin{hindi}
और गुनेहगारों को जहन्नुम की तरफ प्यासे (जानवरो की तरह हकाँएगे
\end{hindi}}
\flushright{\begin{Arabic}
\quranayah[19][87]
\end{Arabic}}
\flushleft{\begin{hindi}
(उस दिन) ये लोग सिफारिश पर भी क़ादिर न होंगे मगर (वहाँ) जिस शख्स ने ख़ुदा से (सिफारिश का) एक़रा ले लिया हो
\end{hindi}}
\flushright{\begin{Arabic}
\quranayah[19][88]
\end{Arabic}}
\flushleft{\begin{hindi}
और (यहूदी) लोग कहते हैं कि खुदा ने (अज़ीज़ को) बेटा बना लिया है
\end{hindi}}
\flushright{\begin{Arabic}
\quranayah[19][89]
\end{Arabic}}
\flushleft{\begin{hindi}
(ऐ रसूल) तुम कह दो कि तुमने इतनी बड़ी सख्त बात अपनी तरफ से गढ़ के की है
\end{hindi}}
\flushright{\begin{Arabic}
\quranayah[19][90]
\end{Arabic}}
\flushleft{\begin{hindi}
कि क़रीब है कि आसमान उससे फट पड़े और ज़मीन शिगाफता हो जाए और पहाड़ टुकड़े-टुकडे होकर गिर पड़े
\end{hindi}}
\flushright{\begin{Arabic}
\quranayah[19][91]
\end{Arabic}}
\flushleft{\begin{hindi}
इस बात से कि उन लोगों ने खुदा के लिए बेटा क़रार दिया
\end{hindi}}
\flushright{\begin{Arabic}
\quranayah[19][92]
\end{Arabic}}
\flushleft{\begin{hindi}
हालाँकि खुदा के लिए ये किसी तरह शायाँ ही नहीं कि वह (किसी को अपना) बेटा बना ले
\end{hindi}}
\flushright{\begin{Arabic}
\quranayah[19][93]
\end{Arabic}}
\flushleft{\begin{hindi}
सारे आसमान व ज़मीन में जितनी चीज़े हैं सब की सब खुदा के सामने बन्दा ही बनकर आने वाली हैं उसने यक़ीनन सबको अपने (इल्म) के अहाते में घेर लिया है
\end{hindi}}
\flushright{\begin{Arabic}
\quranayah[19][94]
\end{Arabic}}
\flushleft{\begin{hindi}
और सबको अच्छी तरह गिन लिया है
\end{hindi}}
\flushright{\begin{Arabic}
\quranayah[19][95]
\end{Arabic}}
\flushleft{\begin{hindi}
और ये सब उसके सामने क़यामत के दिन अकेले (अकेले) हाज़िर होंगे
\end{hindi}}
\flushright{\begin{Arabic}
\quranayah[19][96]
\end{Arabic}}
\flushleft{\begin{hindi}
बेशक जिन लोगों ने ईमान कुबूल किया और अच्छे-अच्छे काम किए अनक़रीब ही खुदा उन की मोहब्बत (लोगों के दिलों में) पैदा कर देगा
\end{hindi}}
\flushright{\begin{Arabic}
\quranayah[19][97]
\end{Arabic}}
\flushleft{\begin{hindi}
(ऐ रसूल) हमने उस कुरान को तुम्हारी (अरबी) जुबान में सिर्फ इसलिए आसान कर दिया है कि तुम उसके ज़रिए से परहेज़गारों को (जन्नत की) खुशख़बरी दो और (अरब की) झगड़ालू क़ौम को (अज़ाबे खुदा से) डराओ
\end{hindi}}
\flushright{\begin{Arabic}
\quranayah[19][98]
\end{Arabic}}
\flushleft{\begin{hindi}
और हमने उनसे पहले कितनी जमाअतों को हलाक कर डाला भला तुम उनमें से किसी को (कहीं देखते हो) उसकी कुछ भनक भी सुनते हो
\end{hindi}}
\chapter{Ta Ha (Ta Ha)}
\begin{Arabic}
\Huge{\centerline{\basmalah}}\end{Arabic}
\flushright{\begin{Arabic}
\quranayah[20][1]
\end{Arabic}}
\flushleft{\begin{hindi}
ऐ ता हा (रसूलअल्लाह)
\end{hindi}}
\flushright{\begin{Arabic}
\quranayah[20][2]
\end{Arabic}}
\flushleft{\begin{hindi}
हमने तुम पर कुरान इसलिए नाज़िल नहीं किया कि तुम (इस क़दर) मशक्क़त उठाओ
\end{hindi}}
\flushright{\begin{Arabic}
\quranayah[20][3]
\end{Arabic}}
\flushleft{\begin{hindi}
मगर जो शख्स खुदा से डरता है उसके लिए नसीहत (क़रार दिया है)
\end{hindi}}
\flushright{\begin{Arabic}
\quranayah[20][4]
\end{Arabic}}
\flushleft{\begin{hindi}
(ये) उस शख्स की तरफ़ से नाज़िल हुआ है जिसने ज़मीन और ऊँचे-ऊँचे आसमानों को पैदा किया
\end{hindi}}
\flushright{\begin{Arabic}
\quranayah[20][5]
\end{Arabic}}
\flushleft{\begin{hindi}
वही रहमान है जो अर्श पर (हुक्मरानी के लिए) आमादा व मुस्तईद है
\end{hindi}}
\flushright{\begin{Arabic}
\quranayah[20][6]
\end{Arabic}}
\flushleft{\begin{hindi}
जो कुछ आसमानों में है और जो कुछ ज़मीन में है और जो कुछ दोनों के बीच में है और जो कुछ ज़मीन के नीचे है (ग़रज़ सब कुछ) उसी का है
\end{hindi}}
\flushright{\begin{Arabic}
\quranayah[20][7]
\end{Arabic}}
\flushleft{\begin{hindi}
और अगर तू पुकार कर बात करे (तो भी आहिस्ता करे तो भी) वह यक़ीनन भेद और उससे ज्यादा पोशीदा चीज़ को जानता है
\end{hindi}}
\flushright{\begin{Arabic}
\quranayah[20][8]
\end{Arabic}}
\flushleft{\begin{hindi}
अल्लाह (वह माबूद है कि) उसके सिवा कोइ माबूद नहीं है (अच्छे-अच्छे) उसी के नाम हैं
\end{hindi}}
\flushright{\begin{Arabic}
\quranayah[20][9]
\end{Arabic}}
\flushleft{\begin{hindi}
और (ऐ रसूल) क्या तुम तक मूसा की ख़बर पहुँची है कि जब उन्होंने दूर से आग देखी
\end{hindi}}
\flushright{\begin{Arabic}
\quranayah[20][10]
\end{Arabic}}
\flushleft{\begin{hindi}
तो अपने घर के लोगों से कहने लगे कि तुम लोग (ज़रा यहीं) ठहरो मैंने आग देखी है क्या अजब है कि मैं वहाँ (जाकर) उसमें से एक अंगारा तुम्हारे पास ले आऊँ या आग के पास किसी राह का पता पा जाऊँ
\end{hindi}}
\flushright{\begin{Arabic}
\quranayah[20][11]
\end{Arabic}}
\flushleft{\begin{hindi}
फिर जब मूसा आग के पास आए तो उन्हें आवाज आई
\end{hindi}}
\flushright{\begin{Arabic}
\quranayah[20][12]
\end{Arabic}}
\flushleft{\begin{hindi}
कि ऐ मूसा बेशक मैं ही तुम्हारा परवरदिगार हूँ तो तुम अपनी जूतियाँ उतार डालो क्योंकि तुम (इस वक्त) तुआ (नामी) पाक़ीज़ा चटियल मैदान में हो
\end{hindi}}
\flushright{\begin{Arabic}
\quranayah[20][13]
\end{Arabic}}
\flushleft{\begin{hindi}
और मैंने तुमको पैग़म्बरी के वास्ते मुन्तख़िब किया (चुन लिया) है तो जो कुछ तुम्हारी तरफ़ वही की जाती है उसे कान लगा कर सुनो
\end{hindi}}
\flushright{\begin{Arabic}
\quranayah[20][14]
\end{Arabic}}
\flushleft{\begin{hindi}
इसमें शक नहीं कि मैं ही वह अल्लाह हूँ कि मेरे सिवा कोई माबूद नहीं तो मेरी ही इबादत करो और मेरी याद के लिए नमाज़ बराबर पढ़ा करो
\end{hindi}}
\flushright{\begin{Arabic}
\quranayah[20][15]
\end{Arabic}}
\flushleft{\begin{hindi}
(क्योंकि) क़यामत ज़रूर आने वाली है और मैं उसे लामहौला छिपाए रखूँगा ताकि हर शख्स (उसके ख़ौफ से नेकी करे) और वैसी कोशिश की है उसका उसे बदला दिया जाए
\end{hindi}}
\flushright{\begin{Arabic}
\quranayah[20][16]
\end{Arabic}}
\flushleft{\begin{hindi}
तो (कहीं) ऐसा न हो कि जो शख्स उसे दिल से नहीं मानता और अपनी नफ़सियानी ख्वाहिश के पीछे पड़ा वह तुम्हें इस (फिक्र) से रोक दे तो तुम तबाह हो जाओगे
\end{hindi}}
\flushright{\begin{Arabic}
\quranayah[20][17]
\end{Arabic}}
\flushleft{\begin{hindi}
और ऐ मूसा ये तुम्हारे दाहिने हाथ में क्या चीज़ है
\end{hindi}}
\flushright{\begin{Arabic}
\quranayah[20][18]
\end{Arabic}}
\flushleft{\begin{hindi}
अर्ज़ की ये तो मेरी लाठी है मैं उस पर सहारा करता हूँ और इससे अपनी बकरियों पर (और दरख्तों की) पत्तियाँ झाड़ता हूँ और उसमें मेरे और भी मतलब हैं
\end{hindi}}
\flushright{\begin{Arabic}
\quranayah[20][19]
\end{Arabic}}
\flushleft{\begin{hindi}
फ़रमाया ऐ मूसा उसको ज़रा ज़मीन पर डाल तो दो मूसा ने उसे डाल दिया
\end{hindi}}
\flushright{\begin{Arabic}
\quranayah[20][20]
\end{Arabic}}
\flushleft{\begin{hindi}
तो फ़ौरन वह साँप बनकर दौड़ने लगा (ये देखकर मूसा भागे)
\end{hindi}}
\flushright{\begin{Arabic}
\quranayah[20][21]
\end{Arabic}}
\flushleft{\begin{hindi}
तो फ़रमाया कि तुम इसको पकड़ लो और डरो नहीं मैं अभी इसकी पहली सी सूरत फिर किए देता हूँ
\end{hindi}}
\flushright{\begin{Arabic}
\quranayah[20][22]
\end{Arabic}}
\flushleft{\begin{hindi}
और अपने हाथ को समेंट कर अपने बग़ल में तो कर लो (फिर देखो कि) वह बग़ैर किसी बीमारी के सफेद चमकता दमकता हुआ निकलेगा ये दूसरा मौजिज़ा है
\end{hindi}}
\flushright{\begin{Arabic}
\quranayah[20][23]
\end{Arabic}}
\flushleft{\begin{hindi}
(ये) ताकि हम तुमको अपनी (कुदरत की) बड़ी-बड़ी निशानियाँ दिखाएँ
\end{hindi}}
\flushright{\begin{Arabic}
\quranayah[20][24]
\end{Arabic}}
\flushleft{\begin{hindi}
अब तुम फिरऔन के पास जाओ उसने बहुत सर उठाया है
\end{hindi}}
\flushright{\begin{Arabic}
\quranayah[20][25]
\end{Arabic}}
\flushleft{\begin{hindi}
मूसा ने अर्ज़ की परवरदिगार (मैं जाता तो हूँ)
\end{hindi}}
\flushright{\begin{Arabic}
\quranayah[20][26]
\end{Arabic}}
\flushleft{\begin{hindi}
मगर तू मेरे लिए मेरे सीने को कुशादा फरमा
\end{hindi}}
\flushright{\begin{Arabic}
\quranayah[20][27]
\end{Arabic}}
\flushleft{\begin{hindi}
और दिलेर बना और मेरा काम मेरे लिए आसान कर दे और मेरी ज़बान से लुक़नत की गिरह खोल दे
\end{hindi}}
\flushright{\begin{Arabic}
\quranayah[20][28]
\end{Arabic}}
\flushleft{\begin{hindi}
ताकि लोग मेरी बात अच्छी तरह समझें और
\end{hindi}}
\flushright{\begin{Arabic}
\quranayah[20][29]
\end{Arabic}}
\flushleft{\begin{hindi}
मेरे कीनेवालों में से मेरे भाई हारून
\end{hindi}}
\flushright{\begin{Arabic}
\quranayah[20][30]
\end{Arabic}}
\flushleft{\begin{hindi}
को मेरा वज़ीर बोझ बटाने वाला बना दे
\end{hindi}}
\flushright{\begin{Arabic}
\quranayah[20][31]
\end{Arabic}}
\flushleft{\begin{hindi}
उसके ज़रिए से मेरी पुश्त मज़बूत कर दे
\end{hindi}}
\flushright{\begin{Arabic}
\quranayah[20][32]
\end{Arabic}}
\flushleft{\begin{hindi}
और मेरे काम में उसको मेरा शरीक बना
\end{hindi}}
\flushright{\begin{Arabic}
\quranayah[20][33]
\end{Arabic}}
\flushleft{\begin{hindi}
ताकि हम दोनों (मिलकर) कसरत से तेरी तसबीह करें
\end{hindi}}
\flushright{\begin{Arabic}
\quranayah[20][34]
\end{Arabic}}
\flushleft{\begin{hindi}
और कसरत से तेरी याद करें
\end{hindi}}
\flushright{\begin{Arabic}
\quranayah[20][35]
\end{Arabic}}
\flushleft{\begin{hindi}
तू तो हमारी हालत देख ही रहा है
\end{hindi}}
\flushright{\begin{Arabic}
\quranayah[20][36]
\end{Arabic}}
\flushleft{\begin{hindi}
फ़रमाया ऐ मूसा तुम्हारी सब दरख्वास्तें मंज़ूर की गई
\end{hindi}}
\flushright{\begin{Arabic}
\quranayah[20][37]
\end{Arabic}}
\flushleft{\begin{hindi}
और हम तो तुम पर एक बार और एहसान कर चुके हैं
\end{hindi}}
\flushright{\begin{Arabic}
\quranayah[20][38]
\end{Arabic}}
\flushleft{\begin{hindi}
जब हमने तुम्हारी माँ को इलहाम किया जो अब तुम्हें ''वही'' के ज़रिए से बताया जाता है
\end{hindi}}
\flushright{\begin{Arabic}
\quranayah[20][39]
\end{Arabic}}
\flushleft{\begin{hindi}
कि तुम इसे (मूसा को) सन्दूक़ में रखकर सन्दूक़ को दरिया में डाल दो फिर दरिया उसे ढकेल कर किनारे डाल देगा कि मूसा को मेरा दुशमन और मूसा का दुशमन (फिरऔन) उठा लेगा और मैंने तुम पर अपनी मोहब्बत को डाल दिया जो देखता (प्यार करता) ताकि तुम मेरी ख़ास निगरानी में पाले पोसे जाओ
\end{hindi}}
\flushright{\begin{Arabic}
\quranayah[20][40]
\end{Arabic}}
\flushleft{\begin{hindi}
(उस वक्त) ज़ब तुम्हारी बहन चली (और फिर उनके घर में आकर) कहने लगी कि कहो तो मैं तुम्हें ऐसी दाया बताऊँ कि जो इसे अच्छी तरह पाले तो (इस तदबीर से) हमने फिर तुमको तुम्हारी माँ के पास पहुँचा दिया ताकि उसकी ऑंखें ठन्डी रहें और तुम्हारी (जुदाई पर) कुढ़े नहीं और तुमने एक शख्स (क़िबती) को मार डाला था और सख्त परेशान थे तो हमने तुमको (इस) ग़म से नजात दी और हमने तुम्हारा अच्छी तरह इम्तिहान कर लिया फिर तुम कई बरस तक मदयन के लोगों में जाकर रहे ऐ मूसा फिर तुम (उम्र के) एक अन्दाजे पर आ गए नबूवत के क़ायल हुए
\end{hindi}}
\flushright{\begin{Arabic}
\quranayah[20][41]
\end{Arabic}}
\flushleft{\begin{hindi}
और मैंने तुमको अपनी रिसालत के वास्ते मुन्तख़िब किया
\end{hindi}}
\flushright{\begin{Arabic}
\quranayah[20][42]
\end{Arabic}}
\flushleft{\begin{hindi}
तुम अपने भाई समैत हमारे मौजिज़े लेकर जाओ और (देखो) मेरी याद में सुस्ती न करना
\end{hindi}}
\flushright{\begin{Arabic}
\quranayah[20][43]
\end{Arabic}}
\flushleft{\begin{hindi}
तुम दोनों फिरऔन के पास जाओ बेशक वह बहुत सरकश हो गया है
\end{hindi}}
\flushright{\begin{Arabic}
\quranayah[20][44]
\end{Arabic}}
\flushleft{\begin{hindi}
फिर उससे (जाकर) नरमी से बातें करो ताकि वह नसीहत मान ले या डर जाए
\end{hindi}}
\flushright{\begin{Arabic}
\quranayah[20][45]
\end{Arabic}}
\flushleft{\begin{hindi}
दोनों ने अर्ज़ की ऐ हमारे पालने वाले हम डरते हैं कि कहीं वह हम पर ज्यादती (न) कर बैठे या ज्यादा सरकशी कर ले
\end{hindi}}
\flushright{\begin{Arabic}
\quranayah[20][46]
\end{Arabic}}
\flushleft{\begin{hindi}
फ़रमाया तुम डरो नहीं मैं तुम्हारे साथ हूँ (सब कुछ) सुनता और देखता हूँ
\end{hindi}}
\flushright{\begin{Arabic}
\quranayah[20][47]
\end{Arabic}}
\flushleft{\begin{hindi}
ग़रज़ तुम दोनों उसके पास जाओ और कहो कि हम आप के परवरदिगार के रसूल हैं तो बनी इसराइल को हमारे साथ भेज दीजिए और उन्हें सताइए नहीं हम आपके पास आपके परवरदिगार का मौजिज़ा लेकर आए हैं और जो राहे रास्त की पैरवी करे उसी के लिए सलामती है
\end{hindi}}
\flushright{\begin{Arabic}
\quranayah[20][48]
\end{Arabic}}
\flushleft{\begin{hindi}
हमारे पास खुदा की तरफ से ये ''वही'' नाज़िल हुईहै कि यक़ीनन अज़ाब उसी शख्स पर है जो (खुदा की आयतों को) झुठलाए
\end{hindi}}
\flushright{\begin{Arabic}
\quranayah[20][49]
\end{Arabic}}
\flushleft{\begin{hindi}
और उसके हुक्म से मुँह मोड़े (ग़रज़ गए और कहा) फिरऔन ने पूछा ऐ मूसा आख़िर तुम दोनों का परवरदिगार कौन है
\end{hindi}}
\flushright{\begin{Arabic}
\quranayah[20][50]
\end{Arabic}}
\flushleft{\begin{hindi}
मूसा ने कहा हमारा परवरदिगार वह है जिसने हर चीज़ को उसके (मुनासिब) सूरत अता फरमाई
\end{hindi}}
\flushright{\begin{Arabic}
\quranayah[20][51]
\end{Arabic}}
\flushleft{\begin{hindi}
फिर उसी ने ज़िन्दगी बसर करने के तरीक़े बताए फिरऔन ने पूछा भला अगले लोगों का हाल (तो बताओ) कि क्या हुआ
\end{hindi}}
\flushright{\begin{Arabic}
\quranayah[20][52]
\end{Arabic}}
\flushleft{\begin{hindi}
मूसा ने कहा इन बातों का इल्म मेरे परवरदिगार के पास एक किताब (लौहे महफूज़) में (लिखा हुआ) है मेरा परवरदिगार न बहकता है न भूलता है
\end{hindi}}
\flushright{\begin{Arabic}
\quranayah[20][53]
\end{Arabic}}
\flushleft{\begin{hindi}
वह वही है जिसने तुम्हारे (फ़ायदे के) वास्ते ज़मीन को बिछौना बनाया और तुम्हारे लिए उसमें राहें निकाली और उसी ने आसमान से पानी बरसाया फिर (खुदा फरमाता है कि) हम ही ने उस पानी के ज़रिए से मुख्तलिफ़ क़िस्मों की घासे निकाली
\end{hindi}}
\flushright{\begin{Arabic}
\quranayah[20][54]
\end{Arabic}}
\flushleft{\begin{hindi}
(ताकि) तुम खुद भी खाओ और अपने चारपायों को भी चराओ कुछ शक नहीं कि इसमें अक्लमन्दों के वास्ते (क़ुदरते खुदा की) बहुत सी निशानियाँ हैं
\end{hindi}}
\flushright{\begin{Arabic}
\quranayah[20][55]
\end{Arabic}}
\flushleft{\begin{hindi}
हमने इसी ज़मीन से तुम को पैदा किया और (मरने के बाद) इसमें लौटा कर लाएँगे और उसी से दूसरी बार (क़यामत के दिन) तुमको निकाल खड़ा करेंगे
\end{hindi}}
\flushright{\begin{Arabic}
\quranayah[20][56]
\end{Arabic}}
\flushleft{\begin{hindi}
और मैंने फिरऔन को अपनी सारी निशानियाँ दिखा दी
\end{hindi}}
\flushright{\begin{Arabic}
\quranayah[20][57]
\end{Arabic}}
\flushleft{\begin{hindi}
इस पर भी उसने सबको झुठला दिया और न माना (और) कहने लगा ऐ मूसा क्या तुम हमारे पास इसलिए आए हो
\end{hindi}}
\flushright{\begin{Arabic}
\quranayah[20][58]
\end{Arabic}}
\flushleft{\begin{hindi}
कि हम को हमारे मुल्क (मिस्र से) अपने जादू के ज़ोर से निकाल बाहर करो अच्छा तो (रहो) हम भी तुम्हारे सामने ऐसा जादू पेश करते हैं फिर तुम अपने और हमारे दरमियान एक वक्त मुक़र्रर करो कि न हम उसके ख़िलाफ करे और न तुम और मुक़ाबला एक साफ़ खुले मैदान में हो
\end{hindi}}
\flushright{\begin{Arabic}
\quranayah[20][59]
\end{Arabic}}
\flushleft{\begin{hindi}
मूसा ने कहा तुम्हारे (मुक़ाबले) की मीयाद ज़ीनत (ईद) का दिन है और उस रोज़ सब लोग दिन चढ़े जमा किए जाँए
\end{hindi}}
\flushright{\begin{Arabic}
\quranayah[20][60]
\end{Arabic}}
\flushleft{\begin{hindi}
उसके बाद फिरऔन (अपनी जगह) लौट गया फिर अपने चलत्तर (जादू के सामान) जमा करने लगा
\end{hindi}}
\flushright{\begin{Arabic}
\quranayah[20][61]
\end{Arabic}}
\flushleft{\begin{hindi}
फिर (मुक़ाबले को) आ मौजूद हुआ मूसा ने (फ़िरऔनियों से) कहा तुम्हारा नास हो खुदा पर झूठी-झूठी इफ्तेरा परदाज़ियाँ न करो वरना वह अज़ाब (नाज़िल करके) इससे तुम्हारा मटियामेट कर छोड़ेगा
\end{hindi}}
\flushright{\begin{Arabic}
\quranayah[20][62]
\end{Arabic}}
\flushleft{\begin{hindi}
और (याद रखो कि) जिसने इफ्तेरा परदाज़ियाँ न की वह यकीनन नामुराद रहा उस पर वह लोग अपने काम में बाहम झगड़ने और सरगोशियाँ करने लगे
\end{hindi}}
\flushright{\begin{Arabic}
\quranayah[20][63]
\end{Arabic}}
\flushleft{\begin{hindi}
(आख़िर) वह लोग बोल उठे कि ये दोनों यक़ीनन जादूगर हैं और चाहते हैं कि अपने जादू (के ज़ोर) से तुम लोगों को तुम्हारे मुल्क से निकाल बाहर करें और तुम्हारे अच्छे ख़ासे मज़हब को मिटा छोड़ें
\end{hindi}}
\flushright{\begin{Arabic}
\quranayah[20][64]
\end{Arabic}}
\flushleft{\begin{hindi}
तो तुम भी खूब अपने चलत्तर (जादू वग़ैरह) दुरूस्त कर लो फिर परा (सफ़) बाँध के (उनके मुक़ाबले में) आ पड़ो और जो आज डर रहा हो वही फायज़ुलहराम रहा
\end{hindi}}
\flushright{\begin{Arabic}
\quranayah[20][65]
\end{Arabic}}
\flushleft{\begin{hindi}
ग़रज़ जादूगरों ने कहा (ऐ मूसा) या तो तुम ही (अपने जादू) फेंको और या ये कि पहले जो जादू फेंके वह हम ही हों
\end{hindi}}
\flushright{\begin{Arabic}
\quranayah[20][66]
\end{Arabic}}
\flushleft{\begin{hindi}
मूसा ने कहा (मैं नहीं डालूँगा) बल्कि तुम ही पहले डालो (ग़रज़ उन्होंने अपने करतब दिखाए) तो बस मूसा को उनके जादू (के ज़ोर से) ऐसा मालूम हुआ कि उनकी रस्सियाँ और उनकी छड़ियाँ दौड़ रही हैं
\end{hindi}}
\flushright{\begin{Arabic}
\quranayah[20][67]
\end{Arabic}}
\flushleft{\begin{hindi}
तो मूसा ने अपने दिल में कुछ दहशत सी पाई
\end{hindi}}
\flushright{\begin{Arabic}
\quranayah[20][68]
\end{Arabic}}
\flushleft{\begin{hindi}
हमने कहा (मूसा) इस से डरो नहीं यक़ीनन तुम ही वर रहोगे
\end{hindi}}
\flushright{\begin{Arabic}
\quranayah[20][69]
\end{Arabic}}
\flushleft{\begin{hindi}
और तुम्हारे दाहिने हाथ में जो (लाठी) है उसे डाल तो दो कि जो करतब उन लोगों ने की है उसे हड़प कर जाए क्योंकि उन लोगों ने जो कुछ करतब की वह एक जादूगर की करतब है और जादूगर जहाँ जाए कामयाब नहीं हो सकता
\end{hindi}}
\flushright{\begin{Arabic}
\quranayah[20][70]
\end{Arabic}}
\flushleft{\begin{hindi}
(ग़रज मूसा की लाटी ने) सब हड़प कर लिया (ये देखते ही) वह सब जादूगर सजदे में गिर पड़े (और कहने लगे) कि हम मूसा और हारून के परवरदिगार पर ईमान ले आए
\end{hindi}}
\flushright{\begin{Arabic}
\quranayah[20][71]
\end{Arabic}}
\flushleft{\begin{hindi}
फिरऔन ने उन लोगों से कहा (हाए) इससे पहले कि हम तुमको इजाज़त दें तुम उस पर ईमान ले आए इसमें शक नहीं कि ये तुम सबका बड़ा (गुरू) है जिसने तुमको जादू सिखाया है तो मैं तुम्हारा एक तरफ़ के हाथ और दूसरी तरफ़ के पाँव ज़रूर काट डालूँगा और तुम्हें यक़ीनन खुरमे की शाख़ों पर सूली चढ़ा दूँगा और उस वक्त तुमको (अच्छी तरह) मालूम हो जाएगा कि हम (दोनों) फरीक़ों से अज़ाब में ज्यादा बढ़ा हुआ कौन और किसको क़याम ज्यादा है
\end{hindi}}
\flushright{\begin{Arabic}
\quranayah[20][72]
\end{Arabic}}
\flushleft{\begin{hindi}
जादूगर बोले कि ऐसे वाजेए व रौशन मौजिज़ात जो हमारे सामने आए उन पर और जिस (खुदा) ने हमको पैदा किया उस पर तो हम तुमको किसी तरह तरजीह नहीं दे सकते तो जो तुझे करना हो कर गुज़र तो बस दुनिया की (इसी ज़रा) ज़िन्दगी पर हुकूमत कर सकता है
\end{hindi}}
\flushright{\begin{Arabic}
\quranayah[20][73]
\end{Arabic}}
\flushleft{\begin{hindi}
(और कहा) हम तो अपने परवरदिगार पर इसलिए ईमान लाए हैं ताकि हमारे वास्ते सारे गुनाह माफ़ कर दे और (ख़ास कर) जिस पर तूने हमें मजबूर किया था और खुदा ही सबसे बेहतर है
\end{hindi}}
\flushright{\begin{Arabic}
\quranayah[20][74]
\end{Arabic}}
\flushleft{\begin{hindi}
और (उसी को) सबसे ज्यादा क़याम है इसमें शक नहीं कि जो शख्स मुजरिम होकर अपने परवरदिगार के सामने हाज़िर होगा तो उसके लिए यक़ीनन जहन्नुम (धरा हुआ) है जिसमें न तो वह मरे ही गा और न ज़िन्दा ही रहेगा
\end{hindi}}
\flushright{\begin{Arabic}
\quranayah[20][75]
\end{Arabic}}
\flushleft{\begin{hindi}
(सिसकता रहेगा) और जो शख्स उसके सामने ईमानदार हो कर हाज़िर होगा और उसने अच्छे-अच्छे काम भी किए होंगे तो ऐसे ही लोग हैं जिनके लिए बड़े-बड़े बुलन्द रूतबे हैं
\end{hindi}}
\flushright{\begin{Arabic}
\quranayah[20][76]
\end{Arabic}}
\flushleft{\begin{hindi}
वह सदाबहार बाग़ात जिनके नीचे नहरें जारी हैं वह लोग उसमें हमेशा रहेंगे और जो गुनाह से पाक व पाकीज़ा रहे उस का यही सिला है
\end{hindi}}
\flushright{\begin{Arabic}
\quranayah[20][77]
\end{Arabic}}
\flushleft{\begin{hindi}
और हमने मूसा के पास ''वही'' भेजी कि मेरे बन्दों (बनी इसराइल) को (मिस्र से) रातों रात निकाल ले जाओ फिर दरिया में (लाठी मारकर) उनके लिए एक सूखी राह निकालो और तुमको पीछा करने का न कोई खौफ़ रहेगा न (डूबने की) कोई दहशत
\end{hindi}}
\flushright{\begin{Arabic}
\quranayah[20][78]
\end{Arabic}}
\flushleft{\begin{hindi}
ग़रज़ फिरऔन ने अपने लशकर समैत उनका पीछा किया फिर दरिया (के पानी का रेला) जैसा कुछ उन पर छाया गया वह छा गया
\end{hindi}}
\flushright{\begin{Arabic}
\quranayah[20][79]
\end{Arabic}}
\flushleft{\begin{hindi}
और फिरऔन ने अपनी क़ौम को गुमराह (करके हलाक) कर डाला और उनकी हिदायत न की
\end{hindi}}
\flushright{\begin{Arabic}
\quranayah[20][80]
\end{Arabic}}
\flushleft{\begin{hindi}
ऐ बनी इसराइल हमने तुमको तुम्हारे दुश्मन (के पंजे) से छुड़ाया और तुम से (कोहेतूर) के दाहिने तरफ का वायदा किया और हम ही ने तुम पर मन व सलवा नाज़िल किया
\end{hindi}}
\flushright{\begin{Arabic}
\quranayah[20][81]
\end{Arabic}}
\flushleft{\begin{hindi}
और (फ़रमाया) कि हमने जो पाक व पाक़ीज़ा रोज़ी तुम्हें दे रखी है उसमें से खाओ (पियो) और उसमें (किसी क़िस्म की) शरारत न करो वरना तुम पर मेरा अज़ाब नाज़िल हो जाएगा और (याद रखो कि) जिस पर मेरा ग़ज़ब नाज़िल हुआ तो वह यक़ीनन गुमराह (हलाक) हुआ
\end{hindi}}
\flushright{\begin{Arabic}
\quranayah[20][82]
\end{Arabic}}
\flushleft{\begin{hindi}
और जो शख्स तौबा करे और ईमान लाए और अच्छे काम करे फिर साबित क़दम रहे तो हम उसको ज़रूर बख्शने वाले हैं
\end{hindi}}
\flushright{\begin{Arabic}
\quranayah[20][83]
\end{Arabic}}
\flushleft{\begin{hindi}
फिर जब मूसा सत्तर आदमियों को लेकर चले और खुद बढ़ आए तो हमने कहा कि (ऐ मूसा तुमने अपनी क़ौम से आगे चलने में क्यों जल्दी की)
\end{hindi}}
\flushright{\begin{Arabic}
\quranayah[20][84]
\end{Arabic}}
\flushleft{\begin{hindi}
ग़रज़ की वह भी तो मेरे ही पीछे चले आ रहे हैं और इसी लिए मैं जल्दी करके तेरे पास इसलिए आगे बढ़ आया हूँ ताकि तू (मुझसे) खुश रहे
\end{hindi}}
\flushright{\begin{Arabic}
\quranayah[20][85]
\end{Arabic}}
\flushleft{\begin{hindi}
फ़रमाया तो हमने तुम्हारे (आने के बाद) तुम्हारी क़ौम का इम्तिहान लिया और सामरी ने उनको गुमराह कर छोड़ा
\end{hindi}}
\flushright{\begin{Arabic}
\quranayah[20][86]
\end{Arabic}}
\flushleft{\begin{hindi}
(तो मूसा) गुस्से में भरे पछताए हुए अपनी क़ौम की तरफ पलटे और आकर कहने लगे ऐ मेरी (क़म्बख्त) क़ौम क्या तुमसे तुम्हारे परवरदिगार ने एक अच्छा वायदा (तौरेत देने का) न किया था तुम्हारे वायदे में अरसा लग गया या तुमने ये चाहा कि तुम पर तुम्हारे परवरदिगार का ग़ज़ब टूंट पड़े कि तुमने मेरे वायदे (खुदा की परसतिश) के ख़िलाफ किया
\end{hindi}}
\flushright{\begin{Arabic}
\quranayah[20][87]
\end{Arabic}}
\flushleft{\begin{hindi}
वह लोग कहने लगे हमने आपके वायदे के ख़िलाफ नहीं किया बल्कि (बात ये हुईकि फिरऔन की) क़ौम के ज़ेवर के बोझे जो (मिस्र से निकलते वक्त) हम पर लोग गए थे उनको हम लोगों ने (सामरी के कहने से आग में) डाल दिया फिर सामरी ने भी डाल दिया
\end{hindi}}
\flushright{\begin{Arabic}
\quranayah[20][88]
\end{Arabic}}
\flushleft{\begin{hindi}
फिर सामरी ने उन लोगों के लिए (उसी जेवर से) एक बछड़े की मूरत बनाई जिसकी आवाज़ भी बछड़े की सी थी उस पर बाज़ लोग कहने लगे यही तुम्हारा (भी) माबूद और मूसा का (भी) माबूद है मगर वह भूल गया है
\end{hindi}}
\flushright{\begin{Arabic}
\quranayah[20][89]
\end{Arabic}}
\flushleft{\begin{hindi}
भला इनको इतनी भी न सूझी कि ये बछड़ा न तो उन लोगों को पलट कर उन की बात का जवाब ही देता है और न उनका ज़रर ही उस के हाथ में है और न नफ़ा
\end{hindi}}
\flushright{\begin{Arabic}
\quranayah[20][90]
\end{Arabic}}
\flushleft{\begin{hindi}
और हारून ने उनसे पहले कहा भी था कि ऐ मेरी क़ौम तुम्हारा सिर्फ़ इसके ज़रिये से इम्तिहान किया जा रहा है और इसमें शक नहीं कि तम्हारा परवरदिगार (बस खुदाए रहमान है) तो तुम मेरी पैरवी करो और मेरा कहा मानो
\end{hindi}}
\flushright{\begin{Arabic}
\quranayah[20][91]
\end{Arabic}}
\flushleft{\begin{hindi}
तो वह लोग कहने लगे जब तक मूसा हमारे पास पलट कर न आएँ हम तो बराबर इसकी परसतिश पर डटे बैठे रहेंगे
\end{hindi}}
\flushright{\begin{Arabic}
\quranayah[20][92]
\end{Arabic}}
\flushleft{\begin{hindi}
मूसा ने हारून की तरफ ख़िताब करके कहा ऐ हारून जब तुमने उनको देख लिया था गमुराह हो गए हैं तो तुम्हें मेरी पैरवी (क़ताल) करने को किसने मना किया
\end{hindi}}
\flushright{\begin{Arabic}
\quranayah[20][93]
\end{Arabic}}
\flushleft{\begin{hindi}
तो क्या तुमने मेरे हुक्म की नाफ़रमानी की
\end{hindi}}
\flushright{\begin{Arabic}
\quranayah[20][94]
\end{Arabic}}
\flushleft{\begin{hindi}
हारून ने कहा ऐ मेरे माँजाए (भाई) मेरी दाढ़ी न पकडिऐ और न मेरे सर (के बाल) मैं तो उससे डरा कि (कहीं) आप (वापस आकर) ये (न) कहिए कि तुमने बनी इसराईल में फूट डाल दी और मेरी बात का भी ख्याल न रखा
\end{hindi}}
\flushright{\begin{Arabic}
\quranayah[20][95]
\end{Arabic}}
\flushleft{\begin{hindi}
तब सामरी से कहने लगे कि ओ सामरी तेरा क्या हाल है
\end{hindi}}
\flushright{\begin{Arabic}
\quranayah[20][96]
\end{Arabic}}
\flushleft{\begin{hindi}
उसने (जवाब में) कहा मुझे वह चीज़ दिखाई दी जो औरों को न सूझी (जिबरील घोड़े पर सवार जा रहे थे) तो मैंने जिबरील फरिश्ते (के घोड़े) के निशाने क़दम की एक मुट्ठी (ख़ाक) की उठा ली फिर मैंने (बछड़ों के क़ालिब में) डाल दी (तो वह बोलेने लगा
\end{hindi}}
\flushright{\begin{Arabic}
\quranayah[20][97]
\end{Arabic}}
\flushleft{\begin{hindi}
और उस वक्त मुझे मेरे नफ्स ने यही सुझाया मूसा ने कहा चल (दूर हो) तेरे लिए (इस दुनिया की) ज़िन्दगी में तो (ये सज़ा है) तू कहता फ़िरेगा कि मुझे न छूना (वरना बुख़ार चढ़ जाएगा) और (आख़िरत में भी) यक़ीनी तेरे लिए (अज़ाब का) वायदा है कि हरगिज़ तुझसे ख़िलाफ़ न किया जाएगा और तू अपने माबूद को तो देख जिस (की इबादत) पर तू डट बैठा था कि हम उसे यक़ीनन जलाकर (राख) कर डालेंगे फिर हम उसे तितिर बितिर करके दरिया में उड़ा देगें
\end{hindi}}
\flushright{\begin{Arabic}
\quranayah[20][98]
\end{Arabic}}
\flushleft{\begin{hindi}
तुम्हारा माबूद तो बस वही ख़ुदा है जिसके सिवा कोई और माबूद बरहक़ नहीं कि उसका इल्म हर चीज़ पर छा गया है
\end{hindi}}
\flushright{\begin{Arabic}
\quranayah[20][99]
\end{Arabic}}
\flushleft{\begin{hindi}
(ऐ रसूल) हम तुम्हारे सामने यूँ वाक़ेयात बयान करते हैं जो गुज़र चुके और हमने ही तुम्हारे पास अपनी बारगाह से कुरान अता किया
\end{hindi}}
\flushright{\begin{Arabic}
\quranayah[20][100]
\end{Arabic}}
\flushleft{\begin{hindi}
जिसने उससे मुँह फेरा वह क़यामत के दिन यक़ीनन (अपने बुरे आमाल का) बोझ उठाएगा
\end{hindi}}
\flushright{\begin{Arabic}
\quranayah[20][101]
\end{Arabic}}
\flushleft{\begin{hindi}
और उसी हाल में हमेशा रहेंगे और क्या ही बुरा बोझ है क़यामत के दिन ये लोग उठाए होंगे
\end{hindi}}
\flushright{\begin{Arabic}
\quranayah[20][102]
\end{Arabic}}
\flushleft{\begin{hindi}
जिस दिन सूर फूँका जाएगा और हम उस दिन गुनाहगारों को (उनकी) ऑंखें पुतली (अन्धी) करके (आमने-सामने) जमा करेंगे
\end{hindi}}
\flushright{\begin{Arabic}
\quranayah[20][103]
\end{Arabic}}
\flushleft{\begin{hindi}
(और) आपस में चुपके-चुपके कहते होंगे कि (दुनिया या क़ब्र में) हम लोग (बहुत से बहुत) नौ दस दिन ठहरे होंगे
\end{hindi}}
\flushright{\begin{Arabic}
\quranayah[20][104]
\end{Arabic}}
\flushleft{\begin{hindi}
जो कुछ ये लोग (उस दिन) कहेंगे हम खूब जानते हैं कि जो इनमें सबसे ज्यादा होशियार होगा बोल उठेगा कि तुम बस (बहुत से बहुत) एक दिन ठहरे होगे
\end{hindi}}
\flushright{\begin{Arabic}
\quranayah[20][105]
\end{Arabic}}
\flushleft{\begin{hindi}
(और ऐ रसूल) तुम से लोग पहाड़ों के बारे में पूछा करते हैं (कि क़यामत के रोज़ क्या होगा)
\end{hindi}}
\flushright{\begin{Arabic}
\quranayah[20][106]
\end{Arabic}}
\flushleft{\begin{hindi}
तो तुम कह दो कि मेरा परवरदिगार बिल्कुल रेज़ा रेज़ा करके उड़ा डालेगा और ज़मीन को एक चटियल मैदान कर छोड़ेगा
\end{hindi}}
\flushright{\begin{Arabic}
\quranayah[20][107]
\end{Arabic}}
\flushleft{\begin{hindi}
कि (ऐ शख्स) न तो उसमें मोड़ देखेगा और न ऊँच-नीच
\end{hindi}}
\flushright{\begin{Arabic}
\quranayah[20][108]
\end{Arabic}}
\flushleft{\begin{hindi}
उस दिन लोग एक पुकारने वाले इसराफ़ील की आवाज़ के पीछे (इस तरह सीधे) दौड़ पड़ेगे कि उसमें कुछ भी कज़ी न होगी और आवाज़े उस दिन खुदा के सामने (इस तरह) घिघियाएगें कि तू घुनघुनाहट के सिवा और कुछ न सुनेगा
\end{hindi}}
\flushright{\begin{Arabic}
\quranayah[20][109]
\end{Arabic}}
\flushleft{\begin{hindi}
उस दिन किसी की सिफ़ारिश काम न आएगी मगर जिसको खुदा ने इजाज़त दी हो और उसका बोलना पसन्द करे जो कुछ उन लोगों के सामने है और जो कुछ उनके पीछे है
\end{hindi}}
\flushright{\begin{Arabic}
\quranayah[20][110]
\end{Arabic}}
\flushleft{\begin{hindi}
(ग़रज़ सब कुछ) वह जानता है और ये लोग अपने इल्म से उसपर हावी नहीं हो सकते
\end{hindi}}
\flushright{\begin{Arabic}
\quranayah[20][111]
\end{Arabic}}
\flushleft{\begin{hindi}
और (क़यामत में) सारी (खुदाई के) का मुँह ज़िन्दा और बाक़ी रहने वाले खुदा के सामने झुक जाएँगे और जिसने जुल्म का बोझ (अपने सर पर) उठाया वह यक़ीनन नाकाम रहा
\end{hindi}}
\flushright{\begin{Arabic}
\quranayah[20][112]
\end{Arabic}}
\flushleft{\begin{hindi}
और जिसने अच्छे-अच्छे काम किए और वह मोमिन भी है तो उसको न किसी तरह की बेइन्साफ़ी का डर है और न किसी नुक़सान का
\end{hindi}}
\flushright{\begin{Arabic}
\quranayah[20][113]
\end{Arabic}}
\flushleft{\begin{hindi}
हमने उसको उसी तरह अरबी ज़बान का कुरान नाज़िल फ़रमाया और उसमें अज़ाब के तरह-तरह के वायदे बयान किए ताकि ये लोग परहेज़गार बनें या उनके मिजाज़ में इबरत पैदा कर दे
\end{hindi}}
\flushright{\begin{Arabic}
\quranayah[20][114]
\end{Arabic}}
\flushleft{\begin{hindi}
पस (दो जहाँ का) सच्चा बादशाह खुदा बरतर व आला है और (ऐ रसूल) कुरान के (पढ़ने) में उससे पहले कि तुम पर उसकी ''वही'' पूरी कर दी जाए जल्दी न करो और दुआ करो कि ऐ मेरे पालने वाले मेरे इल्म को और ज्यादा फ़रमा
\end{hindi}}
\flushright{\begin{Arabic}
\quranayah[20][115]
\end{Arabic}}
\flushleft{\begin{hindi}
और हमने आदम से पहले ही एहद ले लिया था कि उस दरख्त के पास न जाना तो आदम ने उसे तर्क कर दिया
\end{hindi}}
\flushright{\begin{Arabic}
\quranayah[20][116]
\end{Arabic}}
\flushleft{\begin{hindi}
और हमने उनमें साबित व इस्तक़लाल न पाया और जब हमने फ़रिश्तों से कहा कि आदम को सजदा करो तो सबने सजदा किया मगर शैतान ने इन्कार किया
\end{hindi}}
\flushright{\begin{Arabic}
\quranayah[20][117]
\end{Arabic}}
\flushleft{\begin{hindi}
तो मैंने (आदम से कहा) कि ऐ आदम ये यक़ीनी तुम्हारा और तुम्हारी बीवी का दुशमन है तो कहीं तुम दोनों को बेहिश्त से निकलवा न छोड़े तो तुम (दुनिया की) मुसीबत में फँस जाओ
\end{hindi}}
\flushright{\begin{Arabic}
\quranayah[20][118]
\end{Arabic}}
\flushleft{\begin{hindi}
कुछ शक नहीं कि (बेहिश्त में) तुम्हें ये आराम है कि न तो तुम यहाँ भूके रहोगे और न नँगे
\end{hindi}}
\flushright{\begin{Arabic}
\quranayah[20][119]
\end{Arabic}}
\flushleft{\begin{hindi}
और न यहाँ प्यासे रहोगे और न धूप खाओगे
\end{hindi}}
\flushright{\begin{Arabic}
\quranayah[20][120]
\end{Arabic}}
\flushleft{\begin{hindi}
तो शैतान ने उनके दिल में वसवसा डाला (और) कहा ऐ आदम क्या मैं तम्हें (हमेशगी की ज़िन्दगी) का दरख्त और वह सल्तनत जो कभी ज़ाएल न हो बता दूँ
\end{hindi}}
\flushright{\begin{Arabic}
\quranayah[20][121]
\end{Arabic}}
\flushleft{\begin{hindi}
चुनान्चे दोनों मियाँ बीबी ने उसी में से कुछ खाया तो उनका आगा पीछा उनपर ज़ाहिर हो गया और दोनों बेहिश्त के (दरख्त के) पत्ते अपने आगे पीछे पर चिपकाने लगे और आदम ने अपने परवरदिगार की नाफ़रमानी की
\end{hindi}}
\flushright{\begin{Arabic}
\quranayah[20][122]
\end{Arabic}}
\flushleft{\begin{hindi}
तो (राहे सवाब से) बेराह हो गए इसके बाद उनके परवरदिगार ने बर गुज़ीदा किया
\end{hindi}}
\flushright{\begin{Arabic}
\quranayah[20][123]
\end{Arabic}}
\flushleft{\begin{hindi}
फिर उनकी तौबा कुबूल की और उनकी हिदायत की फरमाया कि तुम दोनों बेहश्त से नीचे उतर जाओ तुम में से एक का एक दुशमन है फिर अगर तुम्हारे पास मेरी तरफ से हिदायत पहुँचे तो (तुम) उसकी पैरवी करना क्योंकि जो शख्स मेरी हिदायत पर चलेगा न तो गुमराह होगा और न मुसीबत में फँसेगा
\end{hindi}}
\flushright{\begin{Arabic}
\quranayah[20][124]
\end{Arabic}}
\flushleft{\begin{hindi}
और जिस शख्स ने मेरी याद से मुँह फेरा तो उसकी ज़िन्दगी बहुत तंगी में बसर होगी और हम उसको क़यामत के दिन अंधा बना के उठाएँगे
\end{hindi}}
\flushright{\begin{Arabic}
\quranayah[20][125]
\end{Arabic}}
\flushleft{\begin{hindi}
वह कहेगा इलाही मैं तो (दुनिया में) ऑंख वाला था तूने मुझे अन्धा करके क्यों उठाया
\end{hindi}}
\flushright{\begin{Arabic}
\quranayah[20][126]
\end{Arabic}}
\flushleft{\begin{hindi}
खुदा फरमाएगा ऐसा ही (होना चाहिए) हमारी आयतें भी तो तेरे पास आई तो तू उन्हें भुला बैठा और इसी तरह आज तू भी भूला दिया जाएगा
\end{hindi}}
\flushright{\begin{Arabic}
\quranayah[20][127]
\end{Arabic}}
\flushleft{\begin{hindi}
और जिसने (हद से) तजाविज़ किया और अपने परवरदिगार की आयतों पर ईमान न लाया उसको ऐसी ही बदला देगें और आख़िरत का अज़ाब तो यक़ीनी बहुत सख्त और बहुत देर पा है
\end{hindi}}
\flushright{\begin{Arabic}
\quranayah[20][128]
\end{Arabic}}
\flushleft{\begin{hindi}
तो क्या उन (अहले मक्का) को उस (खुदा) ने ये नहीं बता दिया था कि हमने उनके पहले कितने लोगों को हलाक कर डाला जिनके घरों में ये लोग चलते फिरते हैं इसमें शक नहीं कि उसमें अक्लमंदों के लिए (कुदरते खुदा की) यक़ीनी बहुत सी निशानियाँ हैं
\end{hindi}}
\flushright{\begin{Arabic}
\quranayah[20][129]
\end{Arabic}}
\flushleft{\begin{hindi}
और (ऐ रसूल) अगर तुम्हारे परवरदिगार की तरफ से पहले ही एक वायदा और अज़ाब का) एक वक्त मुअय्युन न होता तो (उनकी हरकतों से) फ़ौरन अज़ाब का आना लाज़मी बात थी
\end{hindi}}
\flushright{\begin{Arabic}
\quranayah[20][130]
\end{Arabic}}
\flushleft{\begin{hindi}
(ऐ रसूल) जो कुछ ये कुफ्फ़ार बका करते हैं तुम उस पर सब्र करो और आफ़ताब निकलने के क़ब्ल और उसके ग़ुरूब होने के क़ब्ल अपने परवरदिगार की हम्दोसना के साथ तसबीह किया करो और कुछ रात के वक़्तों में और दिन के किनारों में तस्बीह करो ताकि तुम निहाल हो जाओ
\end{hindi}}
\flushright{\begin{Arabic}
\quranayah[20][131]
\end{Arabic}}
\flushleft{\begin{hindi}
और (ऐ रसूल) जो उनमें से कुछ लोगों को दुनिया की इस ज़रा सी ज़िन्दगी की रौनक़ से निहाल कर दिया है ताकि हम उनको उसमें आज़माएँ तुम अपनी नज़रें उधर न बढ़ाओ और (इससे) तुम्हारे परवरदिगार की रोज़ी (सवाब) कहीं बेहतर और ज्यादा पाएदार है
\end{hindi}}
\flushright{\begin{Arabic}
\quranayah[20][132]
\end{Arabic}}
\flushleft{\begin{hindi}
और अपने घर वालों को नमाज़ का हुक्म दो और तुम खुद भी उसके पाबन्द रहो हम तुम से रोज़ी तो तलब करते नहीं (बल्कि) हम तो खुद तुमको रोज़ी देते हैं और परहेज़गारी ही का तो अन्जाम बखैर है
\end{hindi}}
\flushright{\begin{Arabic}
\quranayah[20][133]
\end{Arabic}}
\flushleft{\begin{hindi}
और (अहले मक्का) कहते हैं कि अपने परवरदिगार की तरफ से हमारे पास कोई मौजिज़ा हमारी मर्ज़ी के मुवाफिक़ क्यों नहीं लाते तो क्या जो (पेशीव गोइयाँ) अगली किताबों (तौरेत, इन्जील) में (इसकी) गवाह हैं वह भी उनके पास नहीं पहुँची
\end{hindi}}
\flushright{\begin{Arabic}
\quranayah[20][134]
\end{Arabic}}
\flushleft{\begin{hindi}
और अगर हम उनको इस रसूल से पहले अज़ाब से हलाक कर डालते तो ज़रूर कहते कि ऐ हमारे पालने वाले तूने हमारे पास (अपना) रसूल क्यों न भेजा तो हम अपने ज़लील व रूसवा होने से पहले तेरी आयतों की पैरवी ज़रूर करते
\end{hindi}}
\flushright{\begin{Arabic}
\quranayah[20][135]
\end{Arabic}}
\flushleft{\begin{hindi}
रसूल तुम कह दो कि हर शख्स (अपने अन्जामकार का) मुन्तिज़र है तो तुम भी इन्तिज़ार करो फिर तो तुम्हें बहुत जल्द मालूम हो जाएगा कि सीधी राह वाले कौन हैं (और कज़ी पर कौन हैं) हिदायत याफ़ता कौन है और गुमराह कौन है।
\end{hindi}}
\chapter{Al-Anbiya' (The Prophets)}
\begin{Arabic}
\Huge{\centerline{\basmalah}}\end{Arabic}
\flushright{\begin{Arabic}
\quranayah[21][1]
\end{Arabic}}
\flushleft{\begin{hindi}
लोगों के पास उनका हिसाब (उसका वक्त) अा पहुँचा और वह है कि गफ़लत में पड़े मुँह मोड़े ही जाते हैं
\end{hindi}}
\flushright{\begin{Arabic}
\quranayah[21][2]
\end{Arabic}}
\flushleft{\begin{hindi}
जब उनके परवरदिगार की तरफ से उनके पास कोई नया हुक्म आता है तो उसे सिर्फ कान लगाकर सुन लेते हैं और (फिर उसका) हँसी खेल उड़ाते हैं
\end{hindi}}
\flushright{\begin{Arabic}
\quranayah[21][3]
\end{Arabic}}
\flushleft{\begin{hindi}
उनके दिल (आख़िरत के ख्याल से) बिल्कुल बेख़बर हैं और ये ज़ालिम चुपके-चुपके कानाफूसी किया करते हैं कि ये शख्स (मोहम्मद कुछ भी नहीं) बस तुम्हारे ही सा आदमी है तो क्या तुम दीन व दानिस्ता जादू में फँसते हो
\end{hindi}}
\flushright{\begin{Arabic}
\quranayah[21][4]
\end{Arabic}}
\flushleft{\begin{hindi}
(तो उस पर) रसूल ने कहा कि मेरा परवरदिगार जितनी बातें आसमान और ज़मीन में होती हैं खूब जानता है (फिर क्यों कानाफूसी करते हो) और वह तो बड़ा सुनने वाला वाक़िफ़कार है
\end{hindi}}
\flushright{\begin{Arabic}
\quranayah[21][5]
\end{Arabic}}
\flushleft{\begin{hindi}
(उस पर भी उन लोगों ने इक्तिफ़ा न की) बल्कि कहने लगे (ये कुरान तो) ख़ाबहाय परीशाँ का मजमूआ है बल्कि उसने खुद अपने जी से झूट-मूट गढ़ लिया है बल्कि ये शख्स शायर है और अगर हक़ीकतन रसूल है) तो जिस तरह अगले पैग़म्बर मौजिज़ों के साथ भेजे गए थे
\end{hindi}}
\flushright{\begin{Arabic}
\quranayah[21][6]
\end{Arabic}}
\flushleft{\begin{hindi}
उसी तरह ये भी कोई मौजिज़ा (जैसा हम कहें) हमारे पास भला लाए तो सही इनसे पहले जिन बस्तियों को तबाह कर डाला (मौजिज़े भी देखकर तो) ईमान न लाए
\end{hindi}}
\flushright{\begin{Arabic}
\quranayah[21][7]
\end{Arabic}}
\flushleft{\begin{hindi}
तो क्या ये लोग ईमान लाएँगे और ऐ रसूल हमने तुमसे पहले भी आदमियों ही को (रसूल बनाकर) भेजा था कि उनके पास ''वही'' भेजा करते थे तो अगर तुम लोग खुद नहीं जानते हो तो आलिमों से पूँछकर देखो
\end{hindi}}
\flushright{\begin{Arabic}
\quranayah[21][8]
\end{Arabic}}
\flushleft{\begin{hindi}
और हमने उन (पैग़म्बरों) के बदन ऐसे नहीं बनाए थे कि वह खाना न खाएँ और न वह (दुनिया में) हमेशा रहने सहने वाले थे
\end{hindi}}
\flushright{\begin{Arabic}
\quranayah[21][9]
\end{Arabic}}
\flushleft{\begin{hindi}
फिर हमने उन्हें (अपना अज़ाब का) वायदा सच्चा कर दिखाया (और जब अज़ाब आ पहुँचा) तो हमने उन पैग़म्बरों को और जिस जिसको चाहा छुटकारा दिया और हद से बढ़ जाने वालों को हलाक कर डाला
\end{hindi}}
\flushright{\begin{Arabic}
\quranayah[21][10]
\end{Arabic}}
\flushleft{\begin{hindi}
हमने तो तुम लोगों के पास वह किताब (कुरान) नाज़िल की है जिसमें तुम्हारा (भी) ज़िक्रे (ख़ैर) है तो क्या तुम लोग (इतना भी) समझते
\end{hindi}}
\flushright{\begin{Arabic}
\quranayah[21][11]
\end{Arabic}}
\flushleft{\begin{hindi}
और हमने कितनी बस्तियों को जिनके रहने वाले सरकश थे बरबाद कर दिया और उनके बाद दूसरे लोगों को पैदा किया
\end{hindi}}
\flushright{\begin{Arabic}
\quranayah[21][12]
\end{Arabic}}
\flushleft{\begin{hindi}
तो जब उन लोगों ने हमारे अज़ाब की आहट पाई तो एका एकी भागने लगे
\end{hindi}}
\flushright{\begin{Arabic}
\quranayah[21][13]
\end{Arabic}}
\flushleft{\begin{hindi}
(हमने कहा) भागो नहीं और उन्हीं बस्तियों और घरों में लौट जाओ जिनमें तुम चैन करते थे ताकि तुमसे कुछ पूछगछ की जाए
\end{hindi}}
\flushright{\begin{Arabic}
\quranayah[21][14]
\end{Arabic}}
\flushleft{\begin{hindi}
वह लोग कहने लगे हाए हमारी शामत बेशक हम सरकश तो ज़रूर थे
\end{hindi}}
\flushright{\begin{Arabic}
\quranayah[21][15]
\end{Arabic}}
\flushleft{\begin{hindi}
ग़रज़ वह बराबर यही पड़े पुकारा किए यहाँ तक कि हमने उन्हें कटी हुई खेती की तरह बिछा के ठन्डा करके ढेर कर दिया
\end{hindi}}
\flushright{\begin{Arabic}
\quranayah[21][16]
\end{Arabic}}
\flushleft{\begin{hindi}
और हमने आसमान और ज़मीन को और जो कुछ इन दोनों के दरमियान है बेकार लगो नहीं पैदा किया
\end{hindi}}
\flushright{\begin{Arabic}
\quranayah[21][17]
\end{Arabic}}
\flushleft{\begin{hindi}
अगर हम कोई खेल बनाना चाहते तो बेशक हम उसे अपनी तजवीज़ से बना लेते अगर हमको करना होता (मगर हमें शायान ही न था)
\end{hindi}}
\flushright{\begin{Arabic}
\quranayah[21][18]
\end{Arabic}}
\flushleft{\begin{hindi}
बल्कि हम तो हक़ को नाहक़ (के सर) पर खींच मारते हैं तो वह बिल्कुल के सर को कुचल देता है फिर वह उसी वक्त नेस्तवेनाबूद हो जाता है और तुम पर अफ़सोस है कि ऐसी-ऐसी नाहक़ बातें बनाये करते हो
\end{hindi}}
\flushright{\begin{Arabic}
\quranayah[21][19]
\end{Arabic}}
\flushleft{\begin{hindi}
हालाँकि जो लोग (फरिश्ते) आसमान और ज़मीन में हैं (सब) उसी के (बन्दे) हैं और जो (फरिश्ते) उस सरकार में हैं न तो वह उसकी इबादत की शेख़ी करते हैं और न थकते हैं
\end{hindi}}
\flushright{\begin{Arabic}
\quranayah[21][20]
\end{Arabic}}
\flushleft{\begin{hindi}
रात और दिन उसकी तस्बीह किया करते हैं (और) कभी काहिली नहीं करते
\end{hindi}}
\flushright{\begin{Arabic}
\quranayah[21][21]
\end{Arabic}}
\flushleft{\begin{hindi}
उन लोगों जो माबूद ज़मीन में बना रखे हैं क्या वही (लोगों को) ज़िन्दा करेंगे
\end{hindi}}
\flushright{\begin{Arabic}
\quranayah[21][22]
\end{Arabic}}
\flushleft{\begin{hindi}
अगर बफ़रने मुहाल) ज़मीन व आसमान में खुदा कि सिवा चन्द माबूद होते तो दोनों कब के बरबाद हो गए होते तो जो बातें ये लोग अपने जी से (उसके बारे में) बनाया करते हैं खुदा जो अर्श का मालिक है उन तमाम ऐबों से पाक व पाकीज़ा है
\end{hindi}}
\flushright{\begin{Arabic}
\quranayah[21][23]
\end{Arabic}}
\flushleft{\begin{hindi}
जो कुछ वह करता है उसकी पूछगछ नहीं हो सकती
\end{hindi}}
\flushright{\begin{Arabic}
\quranayah[21][24]
\end{Arabic}}
\flushleft{\begin{hindi}
(हाँ) और उन लोगों से बाज़पुर्स होगी क्या उन लोगों ने खुदा को छोड़कर कुछ और माबूद बना रखे हैं (ऐ रसूल) तुम कहो कि भला अपनी दलील तो पेश करो जो मेरे (ज़माने में) साथ है उनकी किताब (कुरान) और जो लोग मुझ से पहले थे उनकी किताबें (तौरेत वग़ैरह) ये (मौजूद) हैं (उनमें खुदा का शरीक बता दो) बल्कि उनमें से अक्सर तो हक़ (बात) को तो जानते ही नहीं
\end{hindi}}
\flushright{\begin{Arabic}
\quranayah[21][25]
\end{Arabic}}
\flushleft{\begin{hindi}
तो (जब हक़ का ज़िक्र आता है) ये लोग मुँह फेर लेते हैं और (ऐ रसूल) हमने तुमसे पहले जब कभी कोई रसूल भेजा तो उसके पास ''वही'' भेजते रहे कि बस हमारे सिवा कोई माबूद क़ाबिले परसतिश नहीं तो मेरी इबादत किया करो
\end{hindi}}
\flushright{\begin{Arabic}
\quranayah[21][26]
\end{Arabic}}
\flushleft{\begin{hindi}
और (अहले मक्का) कहते हैं कि खुदा ने (फरिश्तों को) अपनी औलाद (बेटियाँ) बना रखा है (हालाँकि) वह उससे पाक व पकीज़ा हैं बल्कि (वह फ़रिश्ते) (खुदा के) मोअज्ज़ि बन्दे हैं
\end{hindi}}
\flushright{\begin{Arabic}
\quranayah[21][27]
\end{Arabic}}
\flushleft{\begin{hindi}
ये लोग उसके सामने बढ़कर बोल नहीं सकते और ये लोग उसी के हुक्म पर चलते हैं
\end{hindi}}
\flushright{\begin{Arabic}
\quranayah[21][28]
\end{Arabic}}
\flushleft{\begin{hindi}
जो कुछ उनके सामने है और जो कुछ उनके पीछे है (ग़रज़ सब कुछ) वह (खुदा) जानता है और ये लोग उस शख्स के सिवा जिससे खुदा राज़ी हो किसी की सिफारिश भी नहीं करते और ये लोग खुद उसके ख़ौफ से (हर वक्त) ड़रते रहते हैं
\end{hindi}}
\flushright{\begin{Arabic}
\quranayah[21][29]
\end{Arabic}}
\flushleft{\begin{hindi}
और उनमें से जो कोई ये कह दे कि खुदा नहीं (बल्कि) मैं माबूद हूँ तो वह (मरदूद बारगाह हुआ) हम उसको जहन्नुम की सज़ा देंगे और सरकशों को हम ऐसी ही सज़ा देते हैं
\end{hindi}}
\flushright{\begin{Arabic}
\quranayah[21][30]
\end{Arabic}}
\flushleft{\begin{hindi}
जो लोग काफिर हो बैठे क्या उन लोगों ने इस बात पर ग़ौर नहीं किया कि आसमान और ज़मीन दोनों बस्ता (बन्द) थे तो हमने दोनों को शिगाफ़ता किया (खोल दिया) और हम ही ने जानदार चीज़ को पानी से पैदा किया इस पर भी ये लोग ईमान न लाएँगे
\end{hindi}}
\flushright{\begin{Arabic}
\quranayah[21][31]
\end{Arabic}}
\flushleft{\begin{hindi}
और हम ही ने ज़मीन पर भारी बोझल पहाड़ बनाए ताकि ज़मीन उन लोगों को लेकर किसी तरफ झुक न पड़े और हम ने ही उसमें लम्बे-चौड़े रास्ते बनाए ताकि ये लोग अपने-अपने मंज़िलें मक़सूद को जा पहुँचे
\end{hindi}}
\flushright{\begin{Arabic}
\quranayah[21][32]
\end{Arabic}}
\flushleft{\begin{hindi}
और हम ही ने आसमान को छत बनाया जो हर तरह महफूज़ है और ये लोग उसकी आसमानी निशानियों से मुँह फेर रहे हैं
\end{hindi}}
\flushright{\begin{Arabic}
\quranayah[21][33]
\end{Arabic}}
\flushleft{\begin{hindi}
और वही वह (क़ादिरे मुत्तलिक़) है जिसने रात और दिन और आफ़ताब और माहताब को पैदा किया कि सब के सब एक (एक) आसमान में पैर कर चक्मर लगा रहे हैं
\end{hindi}}
\flushright{\begin{Arabic}
\quranayah[21][34]
\end{Arabic}}
\flushleft{\begin{hindi}
और (ऐ रसूल) हमने तुमसे पहले भी किसी फ़र्दे बशर को सदा की ज़िन्दगी नहीं दी तो क्या अगर तुम मर जाओगे तो ये लोग हमेशा जिया ही करेंगे
\end{hindi}}
\flushright{\begin{Arabic}
\quranayah[21][35]
\end{Arabic}}
\flushleft{\begin{hindi}
(हर शख्स एक न एक दिन) मौत का मज़ा चखने वाला है और हम तुम्हें मुसीबत व राहत में इम्तिहान की ग़रज़ से आज़माते हैं और (आख़िकार) हमारी ही तरफ लौटाए जाओगे
\end{hindi}}
\flushright{\begin{Arabic}
\quranayah[21][36]
\end{Arabic}}
\flushleft{\begin{hindi}
और (ऐ रसूल) जब तुम्हें कुफ्फ़ार देखते हैं तो बस तुमसे मसखरापन करते हैं कि क्या यही हज़रत हैं जो तुम्हारे माबूदों को (बुरी तरह) याद करते हैं हालाँकि ये लोग खुद खुदा की याद से इन्कार करते हैं (तो इनकी बेवकूफ़ी पर हँसना चाहिए)
\end{hindi}}
\flushright{\begin{Arabic}
\quranayah[21][37]
\end{Arabic}}
\flushleft{\begin{hindi}
आदमी तो बड़ा जल्दबाज़ पैदा किया गया है मैं अनक़रीब ही तुम्हें अपनी (कुदरत की) निशानियाँ दिखाऊँगा तो तुम मुझसे जल्दी की (द्दूम) न मचाओ
\end{hindi}}
\flushright{\begin{Arabic}
\quranayah[21][38]
\end{Arabic}}
\flushleft{\begin{hindi}
और लुत्फ़ तो ये है कि कहते हैं कि अगर सच्चे हो तो ये क़यामत का वायदा कब (पूरा) होगा
\end{hindi}}
\flushright{\begin{Arabic}
\quranayah[21][39]
\end{Arabic}}
\flushleft{\begin{hindi}
और जो लोग काफ़िर हो बैठे काश उस वक्त क़ी हालत से आगाह होते (कि जहन्नुम की आग में खडे होंगे) और न अपने चेहरों से आग को हटा सकेंगे और न अपनी पीठ से और न उनकी मदद की जाएगी
\end{hindi}}
\flushright{\begin{Arabic}
\quranayah[21][40]
\end{Arabic}}
\flushleft{\begin{hindi}
(क़यामत कुछ जता कर तो आने से रही) बल्कि वह तो अचानक उन पर आ पड़ेगी और उन्हें हक्का बक्का कर देगी फिर उस वक्त उसमें न उसके हटाने की मजाल होगी और न उन्हें ही दी जाएगी
\end{hindi}}
\flushright{\begin{Arabic}
\quranayah[21][41]
\end{Arabic}}
\flushleft{\begin{hindi}
और (ऐ रसूल) कुछ तुम ही नहीं तुमसे पहले पैग़म्बरों के साथ मसख़रापन किया जा चुका है तो उन पैग़म्बरों से मसखरापन करने वालों को उस सख्त अज़ाब ने आ घेर लिया जिसकी वह हँसी उड़ाया करते थे
\end{hindi}}
\flushright{\begin{Arabic}
\quranayah[21][42]
\end{Arabic}}
\flushleft{\begin{hindi}
(ऐ रसूल) तुम उनसे पूछो तो कि खुदा (के अज़ाब) से (बचाने में) रात को या दिन को तुम्हारा कौन पहरा दे सकता है उस पर डरना तो दर किनार बल्कि ये लोग अपने परवरदिगार की याद से मुँह फेरते हैं
\end{hindi}}
\flushright{\begin{Arabic}
\quranayah[21][43]
\end{Arabic}}
\flushleft{\begin{hindi}
क्या हमरो सिवा उनके कुछ और परवरदिगार हैं कि जो उनको (हमारे अज़ाब से) बचा सकते हैं (वह क्या बचाएँगे) ये लोग खुद अपनी आप तो मदद कर ही नहीं सकते और न हमारे अज़ाब से उन्हें पनाह दी जाएगी
\end{hindi}}
\flushright{\begin{Arabic}
\quranayah[21][44]
\end{Arabic}}
\flushleft{\begin{hindi}
बल्कि हम ही ने उनको और उनके बुर्जुग़ों को आराम व चैन रहा यहाँ तक कि उनकी उम्रें बढ़ गई तो फिर क्या ये लोग नहीं देखते कि हम रूए ज़मीन को चारों तरफ से क़ब्ज़ा करते और उसको फतेह करते चले आते हैं तो क्या (अब भी यही लोग कुफ्फ़ारे मक्का) ग़ालिब और वर हैं
\end{hindi}}
\flushright{\begin{Arabic}
\quranayah[21][45]
\end{Arabic}}
\flushleft{\begin{hindi}
(ऐ रसूल) तुम कह दो कि मैं तो बस तुम लोगों को ''वही'' के मुताबिक़ (अज़ाब से) डराता हूँ (मगर तुम लोग गोया बहरे हो) और बहरों को जब डराया जाता है तो वह पुकारने ही को नहीं सुनते (डरें क्या ख़ाक)
\end{hindi}}
\flushright{\begin{Arabic}
\quranayah[21][46]
\end{Arabic}}
\flushleft{\begin{hindi}
और (ऐ रसूल) अगर कहीं उनको तुम्हारे परवरदिगार के अज़ाब की ज़रा सी हवा भी लग गई तो वे सख्त! बोल उठे हाय अफसोस वाक़ई हम ही ज़ालिम थे
\end{hindi}}
\flushright{\begin{Arabic}
\quranayah[21][47]
\end{Arabic}}
\flushleft{\begin{hindi}
और क़यामत के दिन तो हम (बन्दों के भले बुरे आमाल तौलने के लिए) इन्साफ़ की तराज़ू में खड़ी कर देंगे तो फिर किसी शख्स पर कुछ भी ज़ुल्म न किया जाएगा और अगर राई के दाने के बराबर भी किसी का (अमल) होगा तो तुम उसे ला हाज़िर करेंगे और हम हिसाब करने के वास्ते बहुत काफ़ी हैं
\end{hindi}}
\flushright{\begin{Arabic}
\quranayah[21][48]
\end{Arabic}}
\flushleft{\begin{hindi}
और हम ही ने यक़ीनन मूसा और हारून को (हक़ व बातिल की) जुदा करने वाली किताब (तौरेत) और परहेज़गारों के लिए अज़सरतापा बनूँ और नसीहत अता की
\end{hindi}}
\flushright{\begin{Arabic}
\quranayah[21][49]
\end{Arabic}}
\flushleft{\begin{hindi}
जो बे देखे अपने परवरदिगार से ख़ौफ खाते हैं और ये लोग रोज़े क़यामत से भी डरते हैं
\end{hindi}}
\flushright{\begin{Arabic}
\quranayah[21][50]
\end{Arabic}}
\flushleft{\begin{hindi}
और ये (कुरान भी) एक बाबरकत तज़किरा है जिसको हमने उतारा है तो क्या तुम लोग इसको नहीं मानते
\end{hindi}}
\flushright{\begin{Arabic}
\quranayah[21][51]
\end{Arabic}}
\flushleft{\begin{hindi}
और इसमें भी शक नहीं कि हमने इबराहीम को पहले ही से फ़हेम सलीम अता की थी और हम उन (की हालत) से खूब वाक़िफ थे
\end{hindi}}
\flushright{\begin{Arabic}
\quranayah[21][52]
\end{Arabic}}
\flushleft{\begin{hindi}
जब उन्होंने अपने (मुँह बोले) बाप और अपनी क़ौम से कहा ये मूर्ते जिनकी तुम लोग मुजाबिरी करते हो आख़िर क्या (बला) है
\end{hindi}}
\flushright{\begin{Arabic}
\quranayah[21][53]
\end{Arabic}}
\flushleft{\begin{hindi}
वह लोग बोले (और तो कुछ नहीं जानते मगर) अपने बडे बूढ़ों को इनही की परसतिश करते देखा है
\end{hindi}}
\flushright{\begin{Arabic}
\quranayah[21][54]
\end{Arabic}}
\flushleft{\begin{hindi}
इबराहीम ने कहा यक़ीनन तुम भी और तुम्हारे बुर्जुग़ भी खुली हुईगुमराही में पड़े हुए थे
\end{hindi}}
\flushright{\begin{Arabic}
\quranayah[21][55]
\end{Arabic}}
\flushleft{\begin{hindi}
वह लोग कहने लगे तो क्या तुम हमारे पास हक़ बात लेकर आए हो या तुम भी (यूँ ही) दिल्लगी करते हो
\end{hindi}}
\flushright{\begin{Arabic}
\quranayah[21][56]
\end{Arabic}}
\flushleft{\begin{hindi}
इबराहीम ने कहा मज़ाक नहीं ठीक कहता हूँ कि तुम्हारे माबूद व बुत नहीं बल्कि तुम्हारा परवरदिगार आसमान व ज़मीन का मालिक है जिसने उनको पैदा किया और मैं खुद इस बात का तुम्हारे सामने गवाह हूँ
\end{hindi}}
\flushright{\begin{Arabic}
\quranayah[21][57]
\end{Arabic}}
\flushleft{\begin{hindi}
और अपने जी में कहा खुदा की क़सम तुम्हारे पीठ फेरने के बाद में तुम्हारे बुतों के साथ एक चाल चलूँगा
\end{hindi}}
\flushright{\begin{Arabic}
\quranayah[21][58]
\end{Arabic}}
\flushleft{\begin{hindi}
चुनान्चे इबराहीम ने उन बुतों को (तोड़कर) चकनाचूर कर डाला मगर उनके बड़े बुत को (इसलिए रहने दिया) ताकि ये लोग ईद से पलटकर उसकी तरफ रूजू करें
\end{hindi}}
\flushright{\begin{Arabic}
\quranayah[21][59]
\end{Arabic}}
\flushleft{\begin{hindi}
(जब कुफ्फ़ार को मालूम हुआ) तो कहने लगे जिसने ये गुस्ताख़ी हमारे माबूदों के साथ की है उसने यक़ीनी बड़ा ज़ुल्म किया
\end{hindi}}
\flushright{\begin{Arabic}
\quranayah[21][60]
\end{Arabic}}
\flushleft{\begin{hindi}
(कुछ लोग) कहने लगे हमने एक नौजवान को जिसको लोग इबराहीम कहते हैं उन बुतों का (बुरी तरह) ज़िक्र करते सुना था
\end{hindi}}
\flushright{\begin{Arabic}
\quranayah[21][61]
\end{Arabic}}
\flushleft{\begin{hindi}
लोगों ने कहा तो अच्छा उसको सब लोगों के सामने (गिरफ्तार करके) ले आओ ताकि वह (जो कुछ कहें) लोग उसके गवाह रहें
\end{hindi}}
\flushright{\begin{Arabic}
\quranayah[21][62]
\end{Arabic}}
\flushleft{\begin{hindi}
(ग़रज़ इबराहीम आए) और लोगों ने उनसे पूछा कि क्यों इबराहीम क्या तुमने माबूदों के साथ ये हरकत की है
\end{hindi}}
\flushright{\begin{Arabic}
\quranayah[21][63]
\end{Arabic}}
\flushleft{\begin{hindi}
इबराहीम ने कहा बल्कि ये हरकत इन बुतों (खुदाओं) के बड़े (खुदा) ने की है तो अगर ये बुत बोल सकते हों तो उनही से पूछ देखो
\end{hindi}}
\flushright{\begin{Arabic}
\quranayah[21][64]
\end{Arabic}}
\flushleft{\begin{hindi}
इस पर उन लोगों ने अपने जी में सोचा तो (एक दूसरे से) कहने लगे बेशक तुम ही लोग खुद बर सरे नाहक़ हो
\end{hindi}}
\flushright{\begin{Arabic}
\quranayah[21][65]
\end{Arabic}}
\flushleft{\begin{hindi}
फिर उन लोगों के सर इसी गुमराही में झुका दिए गए (और तो कुछ बन न पड़ा मगर ये बोले) तुमको तो अच्छी तरह मालूम है कि ये बुत बोला नहीं करते
\end{hindi}}
\flushright{\begin{Arabic}
\quranayah[21][66]
\end{Arabic}}
\flushleft{\begin{hindi}
(फिर इनसे क्या पूछे) इबराहीम ने कहा तो क्या तुम लोग खुदा को छोड़कर ऐसी चीज़ों की परसतिश करते हो जो न तुम्हें कुछ नफा ही पहुँचा सकती है और न तुम्हारा कुछ नुक़सान ही कर सकती है
\end{hindi}}
\flushright{\begin{Arabic}
\quranayah[21][67]
\end{Arabic}}
\flushleft{\begin{hindi}
तफ है तुम पर उस चीज़ पर जिसे तुम खुदा के सिवा पूजते हो तो क्या तुम इतना भी नहीं समझते
\end{hindi}}
\flushright{\begin{Arabic}
\quranayah[21][68]
\end{Arabic}}
\flushleft{\begin{hindi}
(आख़िर) वह लोग (बाहम) कहने लगे कि अगर तुम कुछ कर सकते हो तो इबराहीम को आग में जला दो और अपने खुदाओं की मदद करो
\end{hindi}}
\flushright{\begin{Arabic}
\quranayah[21][69]
\end{Arabic}}
\flushleft{\begin{hindi}
(ग़रज़) उन लोगों ने इबराहीम को आग में डाल दिया तो हमने फ़रमाया ऐ आग तू इबराहीम पर बिल्कुल ठन्डी और सलामती का बाइस हो जा
\end{hindi}}
\flushright{\begin{Arabic}
\quranayah[21][70]
\end{Arabic}}
\flushleft{\begin{hindi}
(कि उनको कोई तकलीफ़ न पहुँचे) और उन लोगों में इबराहीम के साथ चालबाज़ी करनी चाही थी तो हमने इन सब को नाकाम कर दिया
\end{hindi}}
\flushright{\begin{Arabic}
\quranayah[21][71]
\end{Arabic}}
\flushleft{\begin{hindi}
और हम ने ही इबराहीम और लूत को (सरकशों से) सही व सालिम निकालकर इस सर ज़मीन (शाम बैतुलमुक़द्दस) में जा पहुँचाया जिसमें हमने सारे जहाँन के लिए तरह-तरह की बरकत अता की थी
\end{hindi}}
\flushright{\begin{Arabic}
\quranayah[21][72]
\end{Arabic}}
\flushleft{\begin{hindi}
और हमने इबराहीम को इनाम में इसहाक़ (जैसा बैटा) और याकूब (जैसा पोता) इनायत फरमाया हमने सबको नेक बख्त बनाया
\end{hindi}}
\flushright{\begin{Arabic}
\quranayah[21][73]
\end{Arabic}}
\flushleft{\begin{hindi}
और उन सबको (लोगों का) पेशवा बनाया कि हमारे हुक्म से (उनकी) हिदायत करते थे और हमने उनके पास नेक काम करने और नमाज़ पढ़ने और ज़कात देने की ''वही'' भेजी थी और ये सब के सब हमारी ही इबादत करते थे
\end{hindi}}
\flushright{\begin{Arabic}
\quranayah[21][74]
\end{Arabic}}
\flushleft{\begin{hindi}
और लूत को भी हम ही के फ़हमे सलीम और नबूवत अता की और हम ही ने उस बस्ती से जहाँ के लोग बदकारियाँ करते थे नजात दी इसमें शक नहीं कि वह लोग बड़े बदकार आदमी थे
\end{hindi}}
\flushright{\begin{Arabic}
\quranayah[21][75]
\end{Arabic}}
\flushleft{\begin{hindi}
और हमने लूत को अपनी रहमत में दाख़िल कर लिया इसमें शक नहीं कि वह नेकोंकार बन्दों में से थे
\end{hindi}}
\flushright{\begin{Arabic}
\quranayah[21][76]
\end{Arabic}}
\flushleft{\begin{hindi}
और (ऐ रसूल लूत से भी) पहले (हमने) नूह को नबूवत पर फ़ायज़ किया जब उन्होंने (हमको) आवाज़ दी तो हमने उनकी (दुआ) सुन ली फिर उनको और उनके साथियों को (तूफ़ान की) बड़ी सख्त मुसीबत से नजात दी
\end{hindi}}
\flushright{\begin{Arabic}
\quranayah[21][77]
\end{Arabic}}
\flushleft{\begin{hindi}
और जिन लोगों ने हमारी आयतों को झुठलाया था उनके मुक़ाबले में उनकी मदद की बेशक ये लोग (भी) बहुत बुरे लोग थे तो हमने उन सबको डुबो मारा
\end{hindi}}
\flushright{\begin{Arabic}
\quranayah[21][78]
\end{Arabic}}
\flushleft{\begin{hindi}
और (ऐ रसूल इनको) दाऊद और सुलेमान का (वाक्या याद दिलाओ) जब ये दोनों एक खेती के बारे में जिसमें रात के वक्त क़ुछ लोगों की बकरियाँ (घुसकर) चर गई थी फैसला करने बैठे और हम उन लोगों के क़िस्से को देख रहे थे (कि बाहम इख़तेलाफ़ हुआ)
\end{hindi}}
\flushright{\begin{Arabic}
\quranayah[21][79]
\end{Arabic}}
\flushleft{\begin{hindi}
तो हमने सुलेमान को (इसका सही फ़ैसला समझा दिया) और (यूँ तो) सबको हम ही ने फहमे सलीम और इल्म अता किया और हम ही ने पहाड़ों को दाऊद का ताबेए बना दिया था कि उनके साथ (खुदा की) तस्बीह किया करते थे और परिन्दों को (भी ताबेए कर दिया था) और हम ही (ये अज़ाब) किया करते थे
\end{hindi}}
\flushright{\begin{Arabic}
\quranayah[21][80]
\end{Arabic}}
\flushleft{\begin{hindi}
और हम ही ने उनको तुम्हारी जंगी पोशिश (ज़िराह) का बनाना सिखा दिया ताकि तुम्हें (एक दूसरे के) वार से बचाए तो क्या तुम (अब भी) उसके शुक्रगुज़ार बनोगे
\end{hindi}}
\flushright{\begin{Arabic}
\quranayah[21][81]
\end{Arabic}}
\flushleft{\begin{hindi}
और (हम ही ने) बड़े ज़ोरों की हवा को सुलेमान का (ताबेए कर दिया था) कि वह उनके हुक्म से इस सरज़मीन (बैतुलमुक़द्दस) की तरफ चला करती थी जिसमें हमने तरह-तरह की बरकतें अता की थी और हम तो हर चीज़ से खूब वाक़िफ़ थे (और) है
\end{hindi}}
\flushright{\begin{Arabic}
\quranayah[21][82]
\end{Arabic}}
\flushleft{\begin{hindi}
और जिन्नात में से जो लोग (समन्दर में) ग़ोता लगाकर (जवाहरात निकालने वाले) थे और उसके अलावा और काम भी करते थे (सुलेमान का ताबेए कर दिया था) और हम ही उनके निगेहबान थे
\end{hindi}}
\flushright{\begin{Arabic}
\quranayah[21][83]
\end{Arabic}}
\flushleft{\begin{hindi}
(कि भाग न जाएँ) और (ऐ रसूल) अय्यूब (का क़िस्सा याद करो) जब उन्होंने अपने परवरदिगार से दुआ की कि (ख़ुदा वन्द) बीमारी तो मेरे पीछे लग गई है और तू तो सब रहम करने वालोंसे (बढ़ कर है मुझ पर तरस खा)
\end{hindi}}
\flushright{\begin{Arabic}
\quranayah[21][84]
\end{Arabic}}
\flushleft{\begin{hindi}
तो हमने उनकी दुआ कुबूल की तो हमने उनका जो कुछ दर्द दुख था दफ़ा कर दिया और उन्हें उनके लड़के वाले बल्कि उनके साथ उतनी ही और भी महज़ अपनी ख़ास मेहरबानी से और इबादत करने वालों की इबरत के वास्ते अता किए
\end{hindi}}
\flushright{\begin{Arabic}
\quranayah[21][85]
\end{Arabic}}
\flushleft{\begin{hindi}
और (ऐ रसूल) इसमाईल और इदरीस और जुलकिफ्ल (के वाक़यात से याद करो) ये सब साबिर बन्दे थे
\end{hindi}}
\flushright{\begin{Arabic}
\quranayah[21][86]
\end{Arabic}}
\flushleft{\begin{hindi}
और हमने उन सबको अपनी (ख़ास) रहमत में दाख़िल कर लिया बेशक ये लोग नेक बन्दे थे
\end{hindi}}
\flushright{\begin{Arabic}
\quranayah[21][87]
\end{Arabic}}
\flushleft{\begin{hindi}
और जुन्नून (यूनुस को याद करो) जबकि गुस्से में आकर चलते हुए और ये ख्याल न किया कि हम उन पर रोज़ी तंग न करेंगे (तो हमने उन्हें मछली के पेट में पहुँचा दिया) तो (घटाटोप) अंधेरे में (घबराकर) चिल्ला उठा कि (परवरदिगार) तेरे सिवा कोई माबूद नहीं तू (हर ऐब से) पाक व पाकीज़ा है बेशक मैं कुसूरवार हूँ
\end{hindi}}
\flushright{\begin{Arabic}
\quranayah[21][88]
\end{Arabic}}
\flushleft{\begin{hindi}
तो हमने उनकी दुआ कुबूल की और उन्हें रंज से नजात दी और हम तो ईमानवालों को यूँ ही नजात दिया करते हैं
\end{hindi}}
\flushright{\begin{Arabic}
\quranayah[21][89]
\end{Arabic}}
\flushleft{\begin{hindi}
और ज़करिया (को याद करो) जब उन्होंने (मायूस की हालत में) अपने परवरदिगार से दुआ की ऐ मेरे पालने वाले मुझे तन्हा (बे औलाद) न छोड़ और तू तो सब वारिसों से बेहतर है
\end{hindi}}
\flushright{\begin{Arabic}
\quranayah[21][90]
\end{Arabic}}
\flushleft{\begin{hindi}
तो हमने उनकी दुआ सुन ली और उन्हें यहया सा बेटा अता किया और हमने उनके लिए उनकी बीबी को अच्छी बता दिया इसमें शक नहीं कि ये सब नेक कामों में जल्दी करते थे और हमको बड़ी रग़बत और ख़ौफ के साथ पुकारा करते थे और हमारे आगे गिड़गिड़ाया करते थे
\end{hindi}}
\flushright{\begin{Arabic}
\quranayah[21][91]
\end{Arabic}}
\flushleft{\begin{hindi}
और (ऐ रसूल) उस बीबी को (याद करो) जिसने अपनी अज़मत की हिफाज़त की तो हमने उन (के पेट) में अपनी तरफ से रूह फूँक दी और उनको और उनके बेटे (ईसा) को सारे जहाँन के वास्ते (अपनी क़ुदरत की) निशानी बनाया
\end{hindi}}
\flushright{\begin{Arabic}
\quranayah[21][92]
\end{Arabic}}
\flushleft{\begin{hindi}
बेशक ये तुम्हारा दीन (इस्लाम) एक ही दीन है और मैं तुम्हारा परवरदिगार हूँ तो मेरी ही इबादत करो
\end{hindi}}
\flushright{\begin{Arabic}
\quranayah[21][93]
\end{Arabic}}
\flushleft{\begin{hindi}
और लोगों ने बाहम (इख़तेलाफ़ करके) अपने दीन को टुकड़े -टुकड़े कर डाला (हालाँकि) वह सब के सब हिरफिर के हमारे ही पास आने वाले हैं
\end{hindi}}
\flushright{\begin{Arabic}
\quranayah[21][94]
\end{Arabic}}
\flushleft{\begin{hindi}
(उस वक्त फ़ैसला हो जाएगा कि) तो जो शख्स अच्छे-अच्छे काम करेगा और वह ईमानवाला भी हो तो उसकी कोशिश अकारत न की जाएगी और हम उसके आमाल लिखते जाते हैं
\end{hindi}}
\flushright{\begin{Arabic}
\quranayah[21][95]
\end{Arabic}}
\flushleft{\begin{hindi}
और जिस बस्ती को हमने तबाह कर डाला मुमकिन नहीं कि वह लोग क़यामत के दिन हिरफिर के से (हमारे पास) न लौटे
\end{hindi}}
\flushright{\begin{Arabic}
\quranayah[21][96]
\end{Arabic}}
\flushleft{\begin{hindi}
बस इतना (तवक्कुफ़ तो ज़रूर होगा) कि जब याजूज माजूज (सद्दे सिकन्दरी) की कैद से खोल दिए जाएँगे और ये लोग (ज़मीन की) हर बुलन्दी से दौड़ते हुए निकल पड़े
\end{hindi}}
\flushright{\begin{Arabic}
\quranayah[21][97]
\end{Arabic}}
\flushleft{\begin{hindi}
और क़यामत का सच्चा वायदा नज़दीक आ जाए तो फिर काफिरों की ऑंखें एक दम से पथरा दी जाएँ (और कहने लगे) हाय हमारी शामत कि हम तो इस (दिन) से ग़फलत ही में (पड़े) रहे बल्कि (सच तो यूँ है कि अपने ऊपर) हम आप ज़ालिम थे
\end{hindi}}
\flushright{\begin{Arabic}
\quranayah[21][98]
\end{Arabic}}
\flushleft{\begin{hindi}
(उस दिन किहा जाएगा कि ऐ कुफ्फ़ार) तुम और जिस चीज़ की तुम खुदा के सिवा परसतिश करते थे यक़ीनन जहन्नुम की ईधन (जलावन) होंगे (और) तुम सबको उसमें उतरना पड़ेगा
\end{hindi}}
\flushright{\begin{Arabic}
\quranayah[21][99]
\end{Arabic}}
\flushleft{\begin{hindi}
अगर ये (सच्चे) माबूद होते तो उन्हें दोज़ख़ में न जाना पड़ता और (अब तो) सबके सब उसी में हमेशा रहेंगे
\end{hindi}}
\flushright{\begin{Arabic}
\quranayah[21][100]
\end{Arabic}}
\flushleft{\begin{hindi}
उन लोगों की दोज़ख़ में चिंघाड़ होगी और ये लोग (अपने शोर व ग़ुल में) किसी की बात भी न सुन सकेंगे
\end{hindi}}
\flushright{\begin{Arabic}
\quranayah[21][101]
\end{Arabic}}
\flushleft{\begin{hindi}
ज़बान अलबत्ता जिन लोगों के वास्ते हमारी तरफ से पहले ही भलाई (तक़दीर में लिखी जा चुकी) वह लोग दोज़ख़ से दूर ही दूर रखे जाएँगे
\end{hindi}}
\flushright{\begin{Arabic}
\quranayah[21][102]
\end{Arabic}}
\flushleft{\begin{hindi}
(यहाँ तक) कि ये लोग उसकी भनक भी न सुनेंगे और ये लोग हमेशा अपनी मनमाँगी मुरादों में (चैन से) रहेंगे
\end{hindi}}
\flushright{\begin{Arabic}
\quranayah[21][103]
\end{Arabic}}
\flushleft{\begin{hindi}
और उनको (क़यामत का) बड़े से बड़ा ख़ौफ़ भी दहशत में न लाएगा और फ़रिश्ते उन से खुशी-खुशी मुलाक़ात करेंगे और ये खुशख़बरी देंगे कि यही वह तुम्हारा (खुशी का) दिन है जिसका (दुनिया में) तुमसे वायदा किया जाता था
\end{hindi}}
\flushright{\begin{Arabic}
\quranayah[21][104]
\end{Arabic}}
\flushleft{\begin{hindi}
(ये) वह दिन (होगा) जब हम आसमान को इस तरह लपेटेगे जिस तरह ख़तों का तूमार लपेटा जाता है जिस तरह हमने (मख़लूक़ात को) पहली बार पैदा किया था (उसी तरह) दोबारा (पैदा) कर छोड़ेगें (ये वह) वायदा (है जिसका करना) हम पर (लाज़िम) है और हम उसे ज़रूर करके रहेंगे
\end{hindi}}
\flushright{\begin{Arabic}
\quranayah[21][105]
\end{Arabic}}
\flushleft{\begin{hindi}
और हमने तो नसीहत (तौरेत) के बाद यक़ीनन जुबूर में लिख ही दिया था कि रूए ज़मीन के वारिस हमारे नेक बन्दे होंगे
\end{hindi}}
\flushright{\begin{Arabic}
\quranayah[21][106]
\end{Arabic}}
\flushleft{\begin{hindi}
इसमें शक नहीं कि इसमें इबादत करने वालों के लिए (एहकामें खुदा की) तबलीग़ है
\end{hindi}}
\flushright{\begin{Arabic}
\quranayah[21][107]
\end{Arabic}}
\flushleft{\begin{hindi}
और (ऐ रसूल) हमने तो तुमको सारे दुनिया जहाँन के लोगों के हक़ में अज़सरतापा रहमत बनाकर भेजा
\end{hindi}}
\flushright{\begin{Arabic}
\quranayah[21][108]
\end{Arabic}}
\flushleft{\begin{hindi}
तुम कह दो कि मेरे पास तो बस यही ''वही'' आई है कि तुम लोगों का माबूद बस यकता खुदा है तो क्या तुम (उसके) फरमाबरदार बन्दे बनते हो
\end{hindi}}
\flushright{\begin{Arabic}
\quranayah[21][109]
\end{Arabic}}
\flushleft{\begin{hindi}
फिर अगर ये लोग (उस पर भी) मुँह फेरें तो तुम कह दो कि मैंने तुम सबको यकसाँ ख़बर कर दी है और मैं नहीं जानता कि जिस (अज़ाब) का तुमसे वायदा किया गया है क़रीब आ पहुँचा या (अभी) दूर है
\end{hindi}}
\flushright{\begin{Arabic}
\quranayah[21][110]
\end{Arabic}}
\flushleft{\begin{hindi}
इसमें शक नहीं कि वह उस बात को भी जानता है जो पुकार कर कही जाती है और जिसे तुम लोग छिपाते हो उससे भी खूब वाक़िफ है
\end{hindi}}
\flushright{\begin{Arabic}
\quranayah[21][111]
\end{Arabic}}
\flushleft{\begin{hindi}
और मैं ये भी नहीं जानता कि शायद ये (ताख़ीरे अज़ाब तुम्हारे) वास्ते इम्तिहान हो और एक मुअय्युन मुद्दत तक (तुम्हारे लिए) चैन हो
\end{hindi}}
\flushright{\begin{Arabic}
\quranayah[21][112]
\end{Arabic}}
\flushleft{\begin{hindi}
(आख़िर) रसूल ने दुआ की ऐ मेरे पालने वाले तू ठीक-ठीक मेरे और काफिरों के दरमियान फैसला कर दे और हमार परवरदिगार बड़ा मेहरबान है कि उसी से इन बातों में मदद माँगी जाती है जो तुम लोग बयान करते हो
\end{hindi}}
\chapter{Al-Hajj (The Pilgrimage)}
\begin{Arabic}
\Huge{\centerline{\basmalah}}\end{Arabic}
\flushright{\begin{Arabic}
\quranayah[22][1]
\end{Arabic}}
\flushleft{\begin{hindi}
ऐ लोगों अपने परवरदिगार से डरते रहो (क्योंकि) क़यामत का ज़लज़ला (कोई मामूली नहीं) एक बड़ी (सख्त) चीज़ है
\end{hindi}}
\flushright{\begin{Arabic}
\quranayah[22][2]
\end{Arabic}}
\flushleft{\begin{hindi}
जिस दिन तुम उसे देख लोगे तो हर दूध पिलाने वाली (डर के मारे) अपने दूध पीते (बच्चे) को भूल जायेगी और सारी हामला औरते अपने-अपने हमल (बेहिश्त से) गिरा देगी और (घबराहट में) लोग तुझे मतवाले मालूम होंगे हालाँकि वह मतवाले नहीं हैं बल्कि खुदा का अज़ाब बहुत सख्त है कि लोग बदहवास हो रहे हैं
\end{hindi}}
\flushright{\begin{Arabic}
\quranayah[22][3]
\end{Arabic}}
\flushleft{\begin{hindi}
और कुछ लोग ऐसे भी हैं जो बग़ैर जाने खुदा के बारे में (ख्वाह म ख्वाह) झगड़ते हैं और हर सरकश शैतान के पीछे हो लेते हैं
\end{hindi}}
\flushright{\begin{Arabic}
\quranayah[22][4]
\end{Arabic}}
\flushleft{\begin{hindi}
जिन (की पेशानी) के ऊपर (ख़ते तक़दीर से) लिखा जा चुका है कि जिसने उससे दोस्ती की हो तो ये यक़ीनन उसे गुमराह करके छोड़ेगा और दोज़ख़ के अज़ाब तक पहुँचा देगा
\end{hindi}}
\flushright{\begin{Arabic}
\quranayah[22][5]
\end{Arabic}}
\flushleft{\begin{hindi}
लोगों अगर तुमको (मरने के बाद) दोबारा जी उठने में किसी तरह का शक है तो इसमें शक नहीं कि हमने तुम्हें शुरू-शुरू मिट्टी से उसके बाद नुत्फे से उसके बाद जमे हुए ख़ून से फिर उस लोथड़े से जो पूरा (सूडौल हो) या अधूरा हो पैदा किया ताकि तुम पर (अपनी कुदरत) ज़ाहिर करें (फिर तुम्हारा दोबारा ज़िन्दा) करना क्या मुश्किल है और हम औरतों के पेट में जिस (नुत्फे) को चाहते हैं एक मुद्दत मुअय्यन तक ठहरा रखते हैं फिर तुमको बच्चा बनाकर निकालते हैं फिर (तुम्हें पालते हैं) ताकि तुम अपनी जवानी को पहुँचो और तुममें से कुछ लोग तो ऐसे हैं जो (क़ब्ल बुढ़ापे के) मर जाते हैं और तुम में से कुछ लोग ऐसे हैं जो नाकारा ज़िन्दगी बुढ़ापे तक फेर लाए हैं जातें ताकि समझने के बाद सठिया के कुछ भी (ख़ाक) न समझ सके और तो ज़मीन को मुर्दा (बेकार उफ़तादा) देख रहा है फिर जब हम उस पर पानी बरसा देते हैं तो लहलहाने और उभरने लगती है और हर तरह की ख़ुशनुमा चीज़ें उगती है तो ये क़ुदरत के तमाशे इसलिए दिखाते हैं ताकि तुम जानो
\end{hindi}}
\flushright{\begin{Arabic}
\quranayah[22][6]
\end{Arabic}}
\flushleft{\begin{hindi}
कि बेशक खुदा बरहक़ है और (ये भी कि) बेशक वही मुर्दों को जिलाता है और वह यक़ीनन हर चीज़ पर क़ादिर है
\end{hindi}}
\flushright{\begin{Arabic}
\quranayah[22][7]
\end{Arabic}}
\flushleft{\begin{hindi}
और क़यामत यक़ीनन आने वाली है इसमें कोई शक नहीं और बेशक जो लोग क़ब्रों में हैं उनको खुदा दोबारा ज़िन्दा करेगा
\end{hindi}}
\flushright{\begin{Arabic}
\quranayah[22][8]
\end{Arabic}}
\flushleft{\begin{hindi}
और लोगों में से कुछ ऐसे भी है जो बेजाने बूझे बे हिदायत पाए बगैर रौशन किताब के (जो उसे राह बताए) खुदा की आयतों से मुँह मोडे
\end{hindi}}
\flushright{\begin{Arabic}
\quranayah[22][9]
\end{Arabic}}
\flushleft{\begin{hindi}
(ख्वाहमख्वाह) खुदा के बारे में लड़ने मरने पर तैयार है ताकि (लोगों को) ख़ुदा की राह बहका दे ऐसे (नाबकार) के लिए दुनिया में (भी) रूसवाई है और क़यामत के दिन (भी) हम उसे जहन्नुम के अज़ाब (का मज़ा) चखाएँगे
\end{hindi}}
\flushright{\begin{Arabic}
\quranayah[22][10]
\end{Arabic}}
\flushleft{\begin{hindi}
और उस वक्त उससे कहा जाएगा कि ये उन आमाल की सज़ा है जो तेरे हाथों ने पहले से किए हैं और बेशक खुदा बन्दों पर हरगिज़ जुल्म नहीं करता
\end{hindi}}
\flushright{\begin{Arabic}
\quranayah[22][11]
\end{Arabic}}
\flushleft{\begin{hindi}
और लोगों में से कुछ ऐसे भी हैं जो एक किनारे पर (खड़े होकर) खुदा की इबादत करता है तो अगर उसको कोई फायदा पहुँच गया तो उसकी वजह से मुतमईन हो गया और अगर कहीं उस कोई मुसीबत छू भी गयी तो (फौरन) मुँह फेर के (कुफ़्र की तरफ़) पलट पड़ा उसने दुनिया और आखेरत (दोनों) का घाटा उठाया यही तो सरीही घाटा है
\end{hindi}}
\flushright{\begin{Arabic}
\quranayah[22][12]
\end{Arabic}}
\flushleft{\begin{hindi}
खुदा को छोड़कर उन चीज़ों को (हाजत के वक्त) बुलाता है जो न उसको नुक़सान ही पहुँचा सकते हैं और न कुछ नफा ही पहुँचा सकते हैं
\end{hindi}}
\flushright{\begin{Arabic}
\quranayah[22][13]
\end{Arabic}}
\flushleft{\begin{hindi}
यही तो पल्ले दरने की गुमराही है और उसको अपनी हाजत रवाई के लिए पुकारता है जिस का नुक़सान उसके नफे से ज्यादा क़रीब है बेशक ऐसा मालिक भी बुरा और ऐसा रफीक़ भी बुरा
\end{hindi}}
\flushright{\begin{Arabic}
\quranayah[22][14]
\end{Arabic}}
\flushleft{\begin{hindi}
बेशक जिन लोगों ने ईमान कुबूल किया और अच्छे अच्छे काम किए उनको (खुदा बेहश्त के) उन (हरे-भरे) बाग़ात में ले जाकर दाख़िल करेगा जिनके नीचे नहरें जारी होगीं बेशक खुदा जो चाहता है करता है
\end{hindi}}
\flushright{\begin{Arabic}
\quranayah[22][15]
\end{Arabic}}
\flushleft{\begin{hindi}
जो शख्स (गुस्से में) ये बदगुमानी करता है कि दुनिया और आख़ेरत में खुदा उसकी हरग़िज मदद न करेगा तो उसे चाहिए कि आसमान तक रस्सी ताने (और अपने गले में फाँसी डाल दे) फिर उसे काट दे (ताकि घुट कर मर जाए) फिर देखिए कि जो चीज़ उसे गुस्से में ला रही थी उसे उसकी तद्बीर दूर दफ़ा कर देती है
\end{hindi}}
\flushright{\begin{Arabic}
\quranayah[22][16]
\end{Arabic}}
\flushleft{\begin{hindi}
(या नहीं) और हमने इस कुरान को यूँ ही वाजेए व रौशन निशानियाँ (बनाकर) नाज़िल किया और बेशक खुदा जिसकी चाहता है हिदायत करता है
\end{hindi}}
\flushright{\begin{Arabic}
\quranayah[22][17]
\end{Arabic}}
\flushleft{\begin{hindi}
इसमें शक नहीं कि जिन लोगों ने ईमान कुबूल किया (मुसलमान) और यहूदी और लामज़हब लोग और ईसाई और मजूसी (आतिशपरस्त) और मुशरेकीन (कुफ्फ़ार) यक़ीनन खुदा उन लोगों के दरमियान क़यामत के दिन (ठीक ठीक) फ़ैसला कर देगा इसमें शक नहीं कि खुदा हर चीज़ को देख रहा है
\end{hindi}}
\flushright{\begin{Arabic}
\quranayah[22][18]
\end{Arabic}}
\flushleft{\begin{hindi}
क्या तुमने इसको भी नहीं देखा कि जो लोग आसमानों में हैं और जो लोग ज़मीन में हैं और आफताब और माहताब और सितारे और पहाड़ और दरख्त और चारपाए (ग़रज़ कुल मख़लूक़ात) और आदमियों में से बहुत से लोग सब खुदा ही को सजदा करते हैं और बहुतेरे ऐसे भी हैं जिन पर नाफ़रमानी की वजह से अज़ाब का (का आना) लाज़िम हो चुका है और जिसको खुदा ज़लील करे फिर उसका कोई इज्ज़त देने वाला नहीं कुछ शक नहीं कि खुदा जो चाहता है करता है (18) सजदा
\end{hindi}}
\flushright{\begin{Arabic}
\quranayah[22][19]
\end{Arabic}}
\flushleft{\begin{hindi}
ये दोनों (मोमिन व काफिर) दो फरीक़ हैं आपस में अपने परवरदिगार के बारे में लड़ते हैं ग़रज़ जो लोग काफ़िर हो बैठे उनके लिए तो आग के कपड़े केता किए गए हैं (वह उन्हें पहनाए जाएँगें और) उनके सरों पर खौलता हुआ पानी उँडेला जाएगा
\end{hindi}}
\flushright{\begin{Arabic}
\quranayah[22][20]
\end{Arabic}}
\flushleft{\begin{hindi}
जिस (की गर्मी) से जो कुछ उनके पेट में है (ऑंतें वग़ैरह) और खालें सब गल जाएँगी
\end{hindi}}
\flushright{\begin{Arabic}
\quranayah[22][21]
\end{Arabic}}
\flushleft{\begin{hindi}
और उनके (मारने के) लिए लोहे के गुर्ज़ होंगे
\end{hindi}}
\flushright{\begin{Arabic}
\quranayah[22][22]
\end{Arabic}}
\flushleft{\begin{hindi}
कि जब सदमें के मारे चाहेंगे कि दोज़ख़ से निकल भागें तो (ग़ुर्ज मार के) फिर उसके अन्दर ढकेल दिए जाएँगे और (उनसे कहा जाएगा कि) जलाने वाले अज़ाब के मज़े चखो
\end{hindi}}
\flushright{\begin{Arabic}
\quranayah[22][23]
\end{Arabic}}
\flushleft{\begin{hindi}
जो लोग ईमान लाए और उन्होंने अच्छे अच्छे काम भी किए उनको खुदा बेहश्त के ऐसे हरे-भरे बाग़ों में दाख़िल फरमाएगा जिनके नीचे नहरे जारी होगी उन्हें वहाँ सोने के कंगन और मोती (के हार) से सँवारा जाएगा और उनका लिबास वहाँ रेशमी होगा
\end{hindi}}
\flushright{\begin{Arabic}
\quranayah[22][24]
\end{Arabic}}
\flushleft{\begin{hindi}
और (ये इस वजह से कि दुनिया में) उन्हें अच्छी बात (कलमाए तौहीद) की हिदायत की गई और उन्हें सज़ावारे हम्द (खुदा) का रास्ता दिखाया गया
\end{hindi}}
\flushright{\begin{Arabic}
\quranayah[22][25]
\end{Arabic}}
\flushleft{\begin{hindi}
बेशक जो लोग काफिर हो बैठे और खुदा की राह से और मस्जिदें मोहतरम (ख़ानए काबा) से जिसे हमने सब लोगों के लिए (माबद) बनाया है (और) इसमें शहरी और बेरूनी सबका हक़ बराबर है (लोगों को) रोकते हैं (उनको) और जो शख्स इसमें शरारत से गुमराही करे उसको हम दर्दनाक अज़ाब का मज़ा चखा देंगे
\end{hindi}}
\flushright{\begin{Arabic}
\quranayah[22][26]
\end{Arabic}}
\flushleft{\begin{hindi}
और (ऐ रसूल वह वक्त याद करो) जब हमने इबराहीम के ज़रिये से इबरहीम के वास्ते ख़ानए काबा की जगह ज़ाहिर कर दी (और उनसे कहा कि) मेरा किसी चीज़ को शरीक न बनाना और मेरे घर को तवाफ और क़याम और रूकू सुजूद करने वालों के वास्ते साफ सुथरा रखना
\end{hindi}}
\flushright{\begin{Arabic}
\quranayah[22][27]
\end{Arabic}}
\flushleft{\begin{hindi}
और लोगों को हज की ख़बर कर दो कि लोग तुम्हारे पास (ज़ूक दर ज़ूक) ज्यादा और हर तरह की दुबली (सवारियों पर जो राह दूर दराज़ तय करके आयी होगी चढ़-चढ़ के) आ पहुँचेगें
\end{hindi}}
\flushright{\begin{Arabic}
\quranayah[22][28]
\end{Arabic}}
\flushleft{\begin{hindi}
ताकि अपने (दुनिया व आखेरत के) फायदो पर फायज़ हों और खुदा ने जो जानवर चारपाए उन्हें अता फ़रमाए उनपर (ज़िबाह के वक्त) चन्द मुअय्युन दिनों में खुदा का नाम लें तो तुम लोग कुरबानी के गोश्त खुद भी खाओ और भूखे मोहताज को भी खिलाओ
\end{hindi}}
\flushright{\begin{Arabic}
\quranayah[22][29]
\end{Arabic}}
\flushleft{\begin{hindi}
फिर लोगों को चाहिए कि अपनी-अपनी (बदन की) कसाफ़त दूर करें और अपनी नज़रें पूरी करें और क़दीम (इबादत) ख़ानए काबा का तवाफ करें यही हुक्म है
\end{hindi}}
\flushright{\begin{Arabic}
\quranayah[22][30]
\end{Arabic}}
\flushleft{\begin{hindi}
और इसके अलावा जो शख्स खुदा की हुरमत वाली चीज़ों की ताज़ीम करेगा तो ये उसके पवरदिगार के यहाँ उसके हक़ में बेहतर है और उन जानवरों के अलावा जो तुमसे बयान किए जाँएगे कुल चारपाए तुम्हारे वास्ते हलाल किए गए तो तुम नापाक बुतों से बचे रहो और लग़ो बातें गाने वग़ैरह से बचे रहो
\end{hindi}}
\flushright{\begin{Arabic}
\quranayah[22][31]
\end{Arabic}}
\flushleft{\begin{hindi}
निरे खरे अल्लाह के होकर (रहो) उसका किसी को शरीक न बनाओ और जिस शख्स ने (किसी को) खुदा का शरीक बनाया तो गोया कि वह आसमान से गिर पड़ा फिर उसको (या तो दरमियान ही से) कोई (मुरदा ख्ववार) चिड़िया उचक ले गई या उसे हवा के झोंके ने बहुत दूर जा फेंका
\end{hindi}}
\flushright{\begin{Arabic}
\quranayah[22][32]
\end{Arabic}}
\flushleft{\begin{hindi}
ये (याद रखो) और जिस शख्स ने खुदा की निशानियों की ताज़ीम की तो कुछ शक नहीं कि ये भी दिलों की परहेज़गारी से हासिल होती है
\end{hindi}}
\flushright{\begin{Arabic}
\quranayah[22][33]
\end{Arabic}}
\flushleft{\begin{hindi}
और इन चार पायों में एक मुअय्युन मुद्दत तक तुम्हार लिये बहुत से फायदें हैं फिर उनके ज़िबाह होने की जगह क़दीम (इबादत) ख़ानए काबा है
\end{hindi}}
\flushright{\begin{Arabic}
\quranayah[22][34]
\end{Arabic}}
\flushleft{\begin{hindi}
और हमने तो हर उम्मत के वास्ते क़ुरबानी का तरीक़ा मुक़र्रर कर दिया है ताकि जो मवेशी चारपाए खुदा ने उन्हें अता किए हैं उन पर (ज़िबाह के वक्त) ख़ुदा का नाम ले ग़रज़ तुम लोगों का माबूद (वही) यकता खुदा है तो उसी के फरमाबरदार बन जाओ
\end{hindi}}
\flushright{\begin{Arabic}
\quranayah[22][35]
\end{Arabic}}
\flushleft{\begin{hindi}
और (ऐ रसूल हमारे) गिड़गिड़ाने वाले बन्दों को (बेहश्त की) खुशख़बरी दे दो ये वह हैं कि जब (उनके सामने) खुदा का नाम लिया जाता है तो उनके दिल सहम जाते हैं और जब उनपर कोई मुसीबत आ पड़े तो सब्र करते हैं और नमाज़ पाबन्दी से अदा करते हैं और जो कुछ हमने उन्हें दे रखा है उसमें से (राहे खुदा में) ख़र्च करते हैं
\end{hindi}}
\flushright{\begin{Arabic}
\quranayah[22][36]
\end{Arabic}}
\flushleft{\begin{hindi}
और कुरबानी (मोटे गदबदे) ऊँट भी हमने तुम्हारे वास्ते खुदा की निशानियों में से क़रार दिया है इसमें तुम्हारी बहुत सी भलाईयाँ हैं फिर उनका तांते का तांता बाँध कर ज़िबाह करो और उस वक्त उन पर खुदा का नाम लो फिर जब उनके दस्त व बाजू काटकर गिर पड़े तो उन्हीं से तुम खुद भी खाओ और केनाअत पेशा फक़ीरों और माँगने वाले मोहताजों (दोनों) को भी खिलाओ हमने यूँ इन जानवरों को तुम्हारा ताबेए कर दिया ताकि तुम शुक्रगुज़ार बनो
\end{hindi}}
\flushright{\begin{Arabic}
\quranayah[22][37]
\end{Arabic}}
\flushleft{\begin{hindi}
खुदा तक न तो हरगिज़ उनके गोश्त ही पहुँचेगे और न खून मगर (हाँ) उस तक तुम्हारी परहेज़गारी अलबत्ता पहुँचेगी ख़ुदा ने जानवरों को (इसलिए) यूँ तुम्हारे क़ाबू में कर दिया है ताकि जिस तरह खुदा ने तुम्हें बनाया है उसी तरह उसकी बड़ाई करो
\end{hindi}}
\flushright{\begin{Arabic}
\quranayah[22][38]
\end{Arabic}}
\flushleft{\begin{hindi}
और (ऐ रसूल) नेकी करने वालों को (हमेशा की) ख़ुशख़बरी दे दो इसमें शक नहीं कि खुदा ईमानवालों से कुफ्फ़ार को दूर दफा करता रहता है खुदा किसी बददयानत नाशुक्रे को हरगिज़ दोस्त नहीं रखता
\end{hindi}}
\flushright{\begin{Arabic}
\quranayah[22][39]
\end{Arabic}}
\flushleft{\begin{hindi}
जिन (मुसलमानों) से (कुफ्फ़ार) लड़ते थे चूँकि वह (बहुत) सताए गए उस वजह से उन्हें भी (जिहाद) की इजाज़त दे दी गई और खुदा तो उन लोगों की मदद पर यक़ीनन क़ादिर (वत वाना) है
\end{hindi}}
\flushright{\begin{Arabic}
\quranayah[22][40]
\end{Arabic}}
\flushleft{\begin{hindi}
ये वह (मज़लूम हैं जो बेचारे) सिर्फ इतनी बात कहने पर कि हमारा परवरदिगार खुदा है (नाहक़) अपने-अपने घरों से निकाल दिए गये और अगर खुदा लोगों को एक दूसरे से दूर दफा न करता रहता तो गिरजे और यहूदियों के इबादत ख़ाने और मजूस के इबादतख़ाने और मस्जिद जिनमें कसरत से खुदा का नाम लिया जाता है कब के कब ढहा दिए गए होते और जो शख्स खुदा की मदद करेगा खुदा भी अलबत्ता उसकी मदद ज़रूर करेगा बेशक खुदा ज़रूर ज़बरदस्त ग़ालिब है
\end{hindi}}
\flushright{\begin{Arabic}
\quranayah[22][41]
\end{Arabic}}
\flushleft{\begin{hindi}
ये वह लोग हैं कि अगर हम इन्हें रूए ज़मीन पर क़ाबू दे दे तो भी यह लोग पाबन्दी से नमाजे अदा करेंगे और ज़कात देंगे और अच्छे-अच्छे काम का हुक्म करेंगे और बुरी बातों से (लोगों को) रोकेंगे और (यूँ तो) सब कामों का अन्जाम खुदा ही के एख्तेयार में है
\end{hindi}}
\flushright{\begin{Arabic}
\quranayah[22][42]
\end{Arabic}}
\flushleft{\begin{hindi}
और (ऐ रसूल) अगर ये (कुफ्फ़ार) तुमको झुठलाते हैं तो कोइ ताज्जुब की बात नहीं उनसे पहले नूह की क़ौम और (क़ौमे आद और समूद)
\end{hindi}}
\flushright{\begin{Arabic}
\quranayah[22][43]
\end{Arabic}}
\flushleft{\begin{hindi}
और इबराहीम की क़ौम और लूत की क़ौम
\end{hindi}}
\flushright{\begin{Arabic}
\quranayah[22][44]
\end{Arabic}}
\flushleft{\begin{hindi}
और मदियन के रहने वाले (अपने-अपने पैग़म्बरों को) झुठला चुके हैं और मूसा (भी) झुठलाए जा चुके हैं तो मैंने काफिरों को चन्द ढील दे दी फिर (आख़िर) उन्हें ले डाला तो तुमने देखा मेरा अज़ाब कैसा था
\end{hindi}}
\flushright{\begin{Arabic}
\quranayah[22][45]
\end{Arabic}}
\flushleft{\begin{hindi}
ग़रज़ कितनी बस्तियाँ हैं कि हम ने उन्हें बरबाद कर दिया और वह सरकश थीं पस वह अपनी छतों पर ढही पड़ी हैं और कितने बेकार (उजडे क़ुएँ और कितने) मज़बूत बड़े-बड़े ऊँचे महल (वीरान हो गए)
\end{hindi}}
\flushright{\begin{Arabic}
\quranayah[22][46]
\end{Arabic}}
\flushleft{\begin{hindi}
क्या ये लोग रूए ज़मीन पर चले फिरे नहीं ताकि उनके लिए ऐसे दिल होते हैं जैसे हक़ बातों को समझते या उनके ऐसे कान होते जिनके ज़रिए से (सच्ची बातों को) सुनते क्योंकि ऑंखें अंधी नहीं हुआ करती बल्कि दिल जो सीने में है वही अन्धे हो जाया करते हैं
\end{hindi}}
\flushright{\begin{Arabic}
\quranayah[22][47]
\end{Arabic}}
\flushleft{\begin{hindi}
और (ऐ रसूल) तुम से ये लोग अज़ाब के जल्द आने की तमन्ना रखते हैं और खुदा तो हरगिज़ अपने वायदे के ख़िलाफ नहीं करेगा और बेशक (क़यामत का) एक दिन तुम्हारे परवरदिगार के नज़दीक तुम्हारी गिनती के हिसाब से एक हज़ार बरस के बराबर है
\end{hindi}}
\flushright{\begin{Arabic}
\quranayah[22][48]
\end{Arabic}}
\flushleft{\begin{hindi}
और कितनी बस्तियाँ हैं कि मैंने उन्हें (चन्द) मोहलत दी हालाँकि वह सरकश थी फिर (आख़िर) मैंने उन्हें ले डाला और (सबको) मेरी तरफ लौटना है
\end{hindi}}
\flushright{\begin{Arabic}
\quranayah[22][49]
\end{Arabic}}
\flushleft{\begin{hindi}
(ऐ रसूल) तुम कह दो कि लोगों में तो सिर्फ तुमको खुल्लम-खुल्ला (अज़ाब से) डराने वाला हूँ
\end{hindi}}
\flushright{\begin{Arabic}
\quranayah[22][50]
\end{Arabic}}
\flushleft{\begin{hindi}
पस जिन लोगों ने ईमान कुबूल किया और अच्छे-अच्छे काम किए (आख़िरत में) उनके लिए बख्शिश है और बेहिश्त की बहुत उम्दा रोज़ी
\end{hindi}}
\flushright{\begin{Arabic}
\quranayah[22][51]
\end{Arabic}}
\flushleft{\begin{hindi}
और जिन लोगों ने हमारी आयतों (के झुठलाने में हमारे) आजिज़ करने के वास्ते कोशिश की यही लोग तो जहन्नुमी हैं
\end{hindi}}
\flushright{\begin{Arabic}
\quranayah[22][52]
\end{Arabic}}
\flushleft{\begin{hindi}
और (ऐ रसूल) हमने तो तुमसे पहले जब कभी कोई रसूल और नबी भेजा तो ये ज़रूर हुआ कि जिस वक्त उसने (तबलीग़े एहकाम की) आरज़ू की तो शैतान ने उसकी आरज़ू में (लोंगों को बहका कर) ख़लल डाल दिया फिर जो वस वसा शैतान डालता है खुदा उसे बेट देता है फिर अपने एहकाम को मज़बूत करता है और खुदा तो बड़ा वाक़िफकार दाना है
\end{hindi}}
\flushright{\begin{Arabic}
\quranayah[22][53]
\end{Arabic}}
\flushleft{\begin{hindi}
और शैतान जो (वसवसा) डालता (भी) है तो इसलिए ताकि खुदा उसे उन लोगों के आज़माइश (का ज़रिया) क़रार दे जिनके दिलों में (कुफ्र का) मर्ज़ है और जिनके दिल सख्त हैं और बेशक (ये) ज़ालिम मुशरेकीन पल्ले दरजे की मुख़ालेफ़त में पड़े हैं
\end{hindi}}
\flushright{\begin{Arabic}
\quranayah[22][54]
\end{Arabic}}
\flushleft{\begin{hindi}
और (इसलिए भी) ताकि जिन लोगों को (कुतूबे समावी का) इल्म अता हुआ है वह जान लें कि ये (वही) बेशक तुम्हारे परवरदिगार की तरफ से ठीक ठीक (नाज़िल) हुईहै फिर (ये ख्याल करके) इस पर वह लोग ईमान लाए फिर उनके दिल खुदा के सामने आजिज़ी करें और इसमें तो शक ही नहीं कि जिन लोगों ने ईमान कुबूल किया उनकी खुदा सीधी राह तक पहुँचा देता है
\end{hindi}}
\flushright{\begin{Arabic}
\quranayah[22][55]
\end{Arabic}}
\flushleft{\begin{hindi}
और जो लोग काफिर हो बैठे वह तो कुरान की तरफ से हमेशा शक ही में पड़े रहेंगे यहाँ तक कि क़यामत यकायक उनके सर पर आ मौजूद हो या (यूँ कहो कि) उनपर एक सख्त मनहूस दिन का अज़ाब नाज़िल हुआ
\end{hindi}}
\flushright{\begin{Arabic}
\quranayah[22][56]
\end{Arabic}}
\flushleft{\begin{hindi}
उस दिन की हुकूमत तो ख़ास खुदा ही की होगी वह लोगों (के बाहमी एख्तेलाफ) का फ़ैसला कर देगा तो जिन लोगों ने ईमान कुबूल किया और अच्छे काम किए हैं वह नेअमतों के (भरे) हुए बाग़ात (बेहश्त) में रहेंगे
\end{hindi}}
\flushright{\begin{Arabic}
\quranayah[22][57]
\end{Arabic}}
\flushleft{\begin{hindi}
और जिन लोगों ने कुफ्र एख्तियार किया और हमारी आयतों को झुठलाया तो यही वह (कम्बख्त) लोग हैं
\end{hindi}}
\flushright{\begin{Arabic}
\quranayah[22][58]
\end{Arabic}}
\flushleft{\begin{hindi}
जिनके लिए ज़लील करने वाला अज़ाब है जिन लोगों ने खुदा की राह में अपने देस छोडे फ़िर शहीद किए गए या (आप अपनी मौत से) मर गए खुदा उन्हें (आख़िरत में) ज़रूर उम्दा रोज़ी अता फ़रमाएगा
\end{hindi}}
\flushright{\begin{Arabic}
\quranayah[22][59]
\end{Arabic}}
\flushleft{\begin{hindi}
और बेशक तमाम रोज़ी देने वालों में खुदा ही सबसे बेहतर है वह उन्हें ज़रूर ऐसी जगह (बेहिश्त) पहुँचा देगा जिससे वह निहाल हो जाएँगे
\end{hindi}}
\flushright{\begin{Arabic}
\quranayah[22][60]
\end{Arabic}}
\flushleft{\begin{hindi}
और खुदा तो बेशक बड़ा वाक़िफकार बुर्दवार है यही (ठीक) है और जो शख्स (अपने दुश्मन को) उतना ही सताए जितना ये उसके हाथों से सताया गया था उसके बाद फिर (दोबारा दुशमन की तरफ़ से) उस पर ज्यादती की जाए तो खुदा उस मज़लूम की ज़रूर मदद करेगा
\end{hindi}}
\flushright{\begin{Arabic}
\quranayah[22][61]
\end{Arabic}}
\flushleft{\begin{hindi}
बेशक खुदा बड़ा माफ करने वाला बख़शने वाला है ये (मदद) इस वजह से दी जाएगी कि खुदा (बड़ा क़ादिर है वही) तो रात को दिन में दाख़िल करता है और दिन को रात में दाख़िल करता है और इसमें भी शक नहीं कि खुदा सब कुछ जानता है
\end{hindi}}
\flushright{\begin{Arabic}
\quranayah[22][62]
\end{Arabic}}
\flushleft{\begin{hindi}
(और) इस वजह से (भी) कि यक़ीनन खुदा ही बरहक़ है और उसके सिवा जिनको लोग (वक्ते मुसीबत) पुकारा करते हैं (सबके सब) बातिल हैं और (ये भी) यक़ीनी (है कि) खुदा ही (सबसे) बुलन्द मर्तबा बुर्जुग़ है
\end{hindi}}
\flushright{\begin{Arabic}
\quranayah[22][63]
\end{Arabic}}
\flushleft{\begin{hindi}
अरे क्या तूने इतना भी नहीं देखा कि खुदा ही आसमान से पानी बरसाता है तो ज़मीन सर सब्ज़ (व शादाब) हो जाती है बेशक खुदा (बन्दों के हाल पर) बड़ा मेहरबान वाक़िफ़कार है
\end{hindi}}
\flushright{\begin{Arabic}
\quranayah[22][64]
\end{Arabic}}
\flushleft{\begin{hindi}
जो कुछ आसमानों में है और जो कुछ ज़मीन में है (ग़रज़ सब कुछ) उसी का है और इसमें तो शक ही नहीं कि खुदा (सबसे) बेपरवाह (और) सज़ावार हम्द है
\end{hindi}}
\flushright{\begin{Arabic}
\quranayah[22][65]
\end{Arabic}}
\flushleft{\begin{hindi}
क्या तूने उस पर भी नज़र न डाली कि जो कुछ रूए ज़मीन में है सबको खुदा ही ने तुम्हारे क़ाबू में कर दिया है और कश्ती को (भी) जो उसके हुक्म से दरिया में चलती है और वही तो आसमान को रोके हुए है कि ज़मीन पर न गिर पड़े मगर (जब) उसका हुक्म होगा (तो गिर पडेग़ा) इसमें शक नहीं कि खुदा लोगों पर बड़ा मेहरबान व रहमवाला है
\end{hindi}}
\flushright{\begin{Arabic}
\quranayah[22][66]
\end{Arabic}}
\flushleft{\begin{hindi}
और वही तो क़ादिर मुत्तलिक़ है जिसने तुमको (पहली बार माँ के पेट में) जिला उठाया फिर वही तुमको मार डालेगा फिर वही तुमको दोबारा ज़िन्दगी देगा
\end{hindi}}
\flushright{\begin{Arabic}
\quranayah[22][67]
\end{Arabic}}
\flushleft{\begin{hindi}
इसमें शक नहीं कि इन्सान बड़ा ही नाशुक्रा है (ऐ रसूल) हमने हर उम्मत के वास्ते एक तरीक़ा मुक़र्रर कर दिया कि वह इस पर चलते हैं फिर तो उन्हें इस दीन (इस्लाम) में तुम से झगड़ा न करना चाहिए और तुम (लोगों को) अपने परवरदिगार की तरफ बुलाए जाओ
\end{hindi}}
\flushright{\begin{Arabic}
\quranayah[22][68]
\end{Arabic}}
\flushleft{\begin{hindi}
बेशक तुम सीधे रास्ते पर हो और अगर (इस पर भी) लोग तुमसे झगड़ा करें तो तुक कह दो कि जो कुछ तुम कर रहे हो खुदा उससे खूब वाक़िफ़ है
\end{hindi}}
\flushright{\begin{Arabic}
\quranayah[22][69]
\end{Arabic}}
\flushleft{\begin{hindi}
जिन बातों में तुम बाहम झगड़ा करते थे क़यामत के दिन ख़ुदा तुम लोगों के दरमियान (ठीक) फ़ैसला कर देगा
\end{hindi}}
\flushright{\begin{Arabic}
\quranayah[22][70]
\end{Arabic}}
\flushleft{\begin{hindi}
(ऐ रसूल) क्या तुम नहीं जानते कि जो कुछ आसमान और ज़मीन में है खुदा यक़ीनन जानता है उसमें तो शक नहीं कि ये सब (बातें) किताब (लौहे महफूज़) में (लिखी हुईमौजूद) हैं
\end{hindi}}
\flushright{\begin{Arabic}
\quranayah[22][71]
\end{Arabic}}
\flushleft{\begin{hindi}
बेशक ये (सब कुछ) खुदा पर आसान है और ये लोग खुदा को छोड़कर उन लोगों की इबादत करते हैं जिनके लिए न तो ख़ुदा ही ने कोई सनद नाज़िल की है और न उस (के हक़ होने) का खुद उन्हें इल्म है और क़यामत में तो ज़ालिमों का कोई मददगार भी नहीं होगा
\end{hindi}}
\flushright{\begin{Arabic}
\quranayah[22][72]
\end{Arabic}}
\flushleft{\begin{hindi}
और (ऐ रसूल) जब हमारी वाज़ेए व रौशन आयतें उनके सामने पढ़ कर सुनाई जाती हैं तो तुम (उन) काफिरों के चेहरों पर नाखुशी के (आसार) देखते हो (यहाँ तक कि) क़रीब होता है कि जो लोग उनको हमारी आयातें पढ़कर सुनाते हैं उन पर ये लोग हमला कर बैठे (ऐ रसूल) तुम कह दो (कि) तो क्या मैं तुम्हें इससे भी कहीं बदतर चीज़ बता दूँ (अच्छा) तो सुन लो वह जहन्नुम है जिसमें झोंकने का वायदा खुदा ने काफ़िरों से किया है
\end{hindi}}
\flushright{\begin{Arabic}
\quranayah[22][73]
\end{Arabic}}
\flushleft{\begin{hindi}
और वह क्या बुरा ठिकाना है लोगों एक मसल बयान की जाती है तो उसे कान लगा के सुनो कि खुदा को छोड़कर जिन लोगों को तुम पुकारते हो वह लोग अगरचे सब के सब इस काम के लिए इकट्ठे भी हो जाएँ तो भी एक मक्खी तक पैदा नहीं कर सकते और कहीं मक्खी कुछ उनसे छीन ले जाए तो उससे उसको छुड़ा नहीं सकते (अजब लुत्फ है) कि माँगने वाला (आबिद) और जिससे माँग लिया (माबूद) दोनों कमज़ोर हैं
\end{hindi}}
\flushright{\begin{Arabic}
\quranayah[22][74]
\end{Arabic}}
\flushleft{\begin{hindi}
खुदा की जैसे क़द्र करनी चाहिए उन लोगों ने न की इसमें शक नहीं कि खुदा तो बड़ा ज़बरदस्त ग़ालिब है
\end{hindi}}
\flushright{\begin{Arabic}
\quranayah[22][75]
\end{Arabic}}
\flushleft{\begin{hindi}
खुदा फरिश्तों में से बाज़ को अपने एहकाम पहुँचाने के लिए मुन्तख़िब कर लेता है
\end{hindi}}
\flushright{\begin{Arabic}
\quranayah[22][76]
\end{Arabic}}
\flushleft{\begin{hindi}
और (इसी तरह) आदमियों में से भी बेशक खुदा (सबकी) सुनता देखता है जो कुछ उनके सामने है और जो कुछ उनके पीछे (हो चुका है) (खुदा सब कुछ) जानता है
\end{hindi}}
\flushright{\begin{Arabic}
\quranayah[22][77]
\end{Arabic}}
\flushleft{\begin{hindi}
और तमाम उमूर की रूजू खुदा ही की तरफ होती है ऐ ईमानवालों रूकू करो और सजदे करो और अपने परवरदिगार की इबादत करो और नेकी करो
\end{hindi}}
\flushright{\begin{Arabic}
\quranayah[22][78]
\end{Arabic}}
\flushleft{\begin{hindi}
ताकि तुम कामयाब हो और जो हक़ जिहाद करने का है खुदा की राह में जिहाद करो उसी नें तुमको बरगुज़ीदा किया और उमूरे दीन में तुम पर किसी तरह की सख्ती नहीं की तुम्हारे बाप इबराहीम ने मजहब को (तुम्हारा मज़हब बना दिया उसी (खुदा) ने तुम्हारा पहले ही से मुसलमान (फरमाबरदार बन्दे) नाम रखा और कुरान में भी (तो जिहाद करो) ताकि रसूल तुम्हारे मुक़ाबले में गवाह बने और तुम पाबन्दी से नामज़ पढ़ा करो और ज़कात देते रहो और खुदा ही (के एहकाम) को मज़बूत पकड़ो वही तुम्हारा सरपरस्त है तो क्या अच्छा सरपरस्त है और क्या अच्छा मददगार है
\end{hindi}}
\chapter{Al-Mu'minun (The Believers)}
\begin{Arabic}
\Huge{\centerline{\basmalah}}\end{Arabic}
\flushright{\begin{Arabic}
\quranayah[23][1]
\end{Arabic}}
\flushleft{\begin{hindi}
अलबत्ता वह ईमान लाने वाले रस्तगार हुए
\end{hindi}}
\flushright{\begin{Arabic}
\quranayah[23][2]
\end{Arabic}}
\flushleft{\begin{hindi}
जो अपनी नमाज़ों में (खुदा के सामने) गिड़गिड़ाते हैं
\end{hindi}}
\flushright{\begin{Arabic}
\quranayah[23][3]
\end{Arabic}}
\flushleft{\begin{hindi}
और जो बेहूदा बातों से मुँह फेरे रहते हैं
\end{hindi}}
\flushright{\begin{Arabic}
\quranayah[23][4]
\end{Arabic}}
\flushleft{\begin{hindi}
और जो ज़कात (अदा) किया करते हैं
\end{hindi}}
\flushright{\begin{Arabic}
\quranayah[23][5]
\end{Arabic}}
\flushleft{\begin{hindi}
और जो (अपनी) शर्मगाहों को (हराम से) बचाते हैं
\end{hindi}}
\flushright{\begin{Arabic}
\quranayah[23][6]
\end{Arabic}}
\flushleft{\begin{hindi}
मगर अपनी बीबियों से या अपनी ज़र ख़रीद लौनडियों से कि उन पर हरगिज़ इल्ज़ाम नहीं हो सकता
\end{hindi}}
\flushright{\begin{Arabic}
\quranayah[23][7]
\end{Arabic}}
\flushleft{\begin{hindi}
पस जो शख्स उसके सिवा किसी और तरीके से शहवत परस्ती की तमन्ना करे तो ऐसे ही लोग हद से बढ़ जाने वाले हैं
\end{hindi}}
\flushright{\begin{Arabic}
\quranayah[23][8]
\end{Arabic}}
\flushleft{\begin{hindi}
और जो अपनी अमानतों और अपने एहद का लिहाज़ रखते हैं
\end{hindi}}
\flushright{\begin{Arabic}
\quranayah[23][9]
\end{Arabic}}
\flushleft{\begin{hindi}
और जो अपनी नमाज़ों की पाबन्दी करते हैं
\end{hindi}}
\flushright{\begin{Arabic}
\quranayah[23][10]
\end{Arabic}}
\flushleft{\begin{hindi}
(आदमी की औलाद में) यही लोग सच्चे वारिस है
\end{hindi}}
\flushright{\begin{Arabic}
\quranayah[23][11]
\end{Arabic}}
\flushleft{\begin{hindi}
जो बेहश्त बरी का हिस्सा लेंगे (और) यही लोग इसमें हमेशा (जिन्दा) रहेंगे
\end{hindi}}
\flushright{\begin{Arabic}
\quranayah[23][12]
\end{Arabic}}
\flushleft{\begin{hindi}
और हमने आदमी को गीली मिट्टी के जौहर से पैदा किया
\end{hindi}}
\flushright{\begin{Arabic}
\quranayah[23][13]
\end{Arabic}}
\flushleft{\begin{hindi}
फिर हमने उसको एक महफूज़ जगह (औरत के रहम में) नुत्फ़ा बना कर रखा
\end{hindi}}
\flushright{\begin{Arabic}
\quranayah[23][14]
\end{Arabic}}
\flushleft{\begin{hindi}
फिर हम ही ने नुतफ़े को जमा हुआ ख़ून बनाया फिर हम ही ने मुनजमिद खून को गोश्त का लोथड़ा बनाया हम ही ने लोथडे क़ी हड्डियाँ बनायीं फिर हम ही ने हड्डियों पर गोश्त चढ़ाया फिर हम ही ने उसको (रुह डालकर) एक दूसरी सूरत में पैदा किया तो (सुबहान अल्लाह) ख़ुदा बा बरकत है जो सब बनाने वालो से बेहतर है
\end{hindi}}
\flushright{\begin{Arabic}
\quranayah[23][15]
\end{Arabic}}
\flushleft{\begin{hindi}
फिर इसके बाद यक़ीनन तुम सब लोगों को (एक न एक दिन) मरना है
\end{hindi}}
\flushright{\begin{Arabic}
\quranayah[23][16]
\end{Arabic}}
\flushleft{\begin{hindi}
इसके बाद कयामत के दिन तुम सब के सब कब्रों से उठाए जाओगे
\end{hindi}}
\flushright{\begin{Arabic}
\quranayah[23][17]
\end{Arabic}}
\flushleft{\begin{hindi}
और हम ही ने तुम्हारे ऊपर तह ब तह आसमान बनाए और हम मख़लूक़ात से बेखबर नही है
\end{hindi}}
\flushright{\begin{Arabic}
\quranayah[23][18]
\end{Arabic}}
\flushleft{\begin{hindi}
और हमने आसमान से एक अन्दाजे क़े साथ पानी बरसाया फिर उसको ज़मीन में (हसब मसलेहत) ठहराए रखा और हम तो यक़ीनन उसके ग़ाएब कर देने पर भी क़ाबू रखते है
\end{hindi}}
\flushright{\begin{Arabic}
\quranayah[23][19]
\end{Arabic}}
\flushleft{\begin{hindi}
फिर हमने उस पानी से तुम्हारे वास्ते खजूरों और अंगूरों के बाग़ात बनाए कि उनमें तुम्हारे वास्ते (तरह तरह के) बहुतेरे मेवे (पैदा होते) हैं उनमें से बाज़ को तुम खाते हो
\end{hindi}}
\flushright{\begin{Arabic}
\quranayah[23][20]
\end{Arabic}}
\flushleft{\begin{hindi}
और (हम ही ने ज़ैतून का) दरख्त (पैदा किया) जो तूरे सैना (पहाड़) में (कसरत से) पैदा होता है जिससे तेल भी निकलता है और खाने वालों के लिए सालन भी है
\end{hindi}}
\flushright{\begin{Arabic}
\quranayah[23][21]
\end{Arabic}}
\flushleft{\begin{hindi}
और उसमें भी शक नहीं कि तुम्हारे वास्ते चौपायों में भी इबरत की जगह है और (ख़ाक बला) जो कुछ उनके पेट में है उससे हम तुमको दूध पिलाते हैं और जानवरों में तो तुम्हारे और भी बहुत से फायदे हैं और उन्हीं में से बाज़ तुम खाते हो
\end{hindi}}
\flushright{\begin{Arabic}
\quranayah[23][22]
\end{Arabic}}
\flushleft{\begin{hindi}
और उन्हें जानवरों और कश्तियों पर चढे चढ़े फिरते भी हो
\end{hindi}}
\flushright{\begin{Arabic}
\quranayah[23][23]
\end{Arabic}}
\flushleft{\begin{hindi}
और हमने नूह को उनकी काैम के पास पैग़म्बर बनाकर भेजा तो नूह ने (उनसे) कहा ऐ मेरी क़ौम खुदा ही की इबादत करो उसके सिवा कोई तुम्हारा माबूद नहीं तो क्या तुम (उससे) डरते नहीं हो
\end{hindi}}
\flushright{\begin{Arabic}
\quranayah[23][24]
\end{Arabic}}
\flushleft{\begin{hindi}
तो उनकी क़ौम के सरदारों ने जो काफिर थे कहा कि ये भी तो बस (आख़िर) तुम्हारे ही सा आदमी है (मगर) इसकी तमन्ना ये है कि तुम पर बुर्जुगी हासिल करे और अगर खुदा (पैग़म्बर ही न भेजना) चाहता तो फरिश्तों को नाज़िल करता हम ने तो (भाई) ऐसी बात अपने अगले बाप दादाओं में (भी होती) नहीं सुनी
\end{hindi}}
\flushright{\begin{Arabic}
\quranayah[23][25]
\end{Arabic}}
\flushleft{\begin{hindi}
हो न हों बस ये एक आदमी है जिसे जुनून हो गया है ग़रज़ तुम लोग एक (ख़ास) वक्त तक (इसके अन्जाम का) इन्तेज़ार देखो
\end{hindi}}
\flushright{\begin{Arabic}
\quranayah[23][26]
\end{Arabic}}
\flushleft{\begin{hindi}
नूह ने (ये बातें सुनकर) दुआ की ऐ मेरे पलने वाले मेरी मदद कर
\end{hindi}}
\flushright{\begin{Arabic}
\quranayah[23][27]
\end{Arabic}}
\flushleft{\begin{hindi}
इस वजह से कि उन लोगों ने मुझे झुठला दिया तो हमने नूह के पास 'वही' भेजी कि तुम हमारे सामने हमारे हुक्म के मुताबिक़ कश्ती बनाना शुरु करो फिर जब कल हमारा अज़ाब आ जाए और तन्नूर (से पानी) उबलने लगे तो तुम उसमें हर किस्म (के जानवरों में) से (नर मादा) दो दो का जोड़ा और अपने लड़के बालों को बिठा लो मगर उन में से जिसकी निस्बत (ग़रक़ होने का) पहले से हमारा हुक्म हो चुका है (उन्हें छोड़ दो) और जिन लोगों ने (हमारे हुकम से) सरकशी की है उनके बारे में मुझसे कुछ कहना (सुनना) नहीं क्योंकि ये लोग यकीनन डूबने वाले है
\end{hindi}}
\flushright{\begin{Arabic}
\quranayah[23][28]
\end{Arabic}}
\flushleft{\begin{hindi}
ग़रज़ जब तुम अपने हमराहियों के साथ कश्ती पर दुरुस्त बैठो तो कहो तमाम हम्दो सना की सज़ावार खुदा ही है जिसने हमको ज़ालिम लोगों से नजात दी
\end{hindi}}
\flushright{\begin{Arabic}
\quranayah[23][29]
\end{Arabic}}
\flushleft{\begin{hindi}
और दुआ करो कि ऐ मेरे पालने वाले तू मुझको (दरख्त के पानी की) बा बरकत जगह में उतारना और तू तो सब उतारने वालो ंसे बेहतर है
\end{hindi}}
\flushright{\begin{Arabic}
\quranayah[23][30]
\end{Arabic}}
\flushleft{\begin{hindi}
इसमें शक नहीं कि हसमें (हमारी क़ुदरत की) बहुत सी निशानियाँ हैं और हमको तो बस उनका इम्तिहान लेना मंज़ूर था
\end{hindi}}
\flushright{\begin{Arabic}
\quranayah[23][31]
\end{Arabic}}
\flushleft{\begin{hindi}
फिर हमने उनके बाद एक और क़ौम को (समूद) को पैदा किया
\end{hindi}}
\flushright{\begin{Arabic}
\quranayah[23][32]
\end{Arabic}}
\flushleft{\begin{hindi}
और हमने उनही में से (एक आदमी सालेह को) रसूल बनाकर उन लोगों में भेजा (और उन्होंने अपनी क़ौम से कहा) कि खुदा की इबादत करो उसके सिवा कोई तुम्हारा माबूद नहीं तो क्या तुम (उससे डरते नही हो)
\end{hindi}}
\flushright{\begin{Arabic}
\quranayah[23][33]
\end{Arabic}}
\flushleft{\begin{hindi}
और उनकी क़ौम के चन्द सरदारों ने जो काफिर थे और (रोज़) आख़िरत की हाज़िरी को भी झुठलाते थे और दुनिया की (चन्द रोज़ा) ज़िन्दगी में हमने उन्हें शहवत भी दे रखी थी आपस में कहने लगे (अरे) ये तो बस तुम्हारा ही सा आदमी है जो चीज़े तुम खाते वही ये भी खाता है और जो चीज़े तुम पीते हो उन्हीं में से ये भी पीता है
\end{hindi}}
\flushright{\begin{Arabic}
\quranayah[23][34]
\end{Arabic}}
\flushleft{\begin{hindi}
और अगर कहीं तुम लोगों ने अपने ही से आदमी की इताअत कर ली तो तुम ज़रुर घाटे में रहोगे
\end{hindi}}
\flushright{\begin{Arabic}
\quranayah[23][35]
\end{Arabic}}
\flushleft{\begin{hindi}
क्या ये शख्स तुमसे वायदा करता है कि जब तुम मर जाओगे और (मर कर) सिर्फ मिट्टी और हड्डियाँ (बनकर) रह जाओगे तो तुम दुबारा ज़िन्दा करके कब्रों से निकाले जाओगे (है है अरे) जिसका तुमसे वायदा किया जाता है
\end{hindi}}
\flushright{\begin{Arabic}
\quranayah[23][36]
\end{Arabic}}
\flushleft{\begin{hindi}
बिल्कुल (अक्ल से) दूर और क़यास से बईद है (दो बार ज़िन्दा होना कैसा) बस यही तुम्हारी दुनिया की ज़िन्दगी है
\end{hindi}}
\flushright{\begin{Arabic}
\quranayah[23][37]
\end{Arabic}}
\flushleft{\begin{hindi}
कि हम मरते भी हैं और जीते भी हैं और हम तो फिर (दुबारा) उठाए नहीं जाएँगे हो न हो ये (सालेह) वह शख्स है जिसने खुदा पर झूठ मूठ बोहतान बाँधा है
\end{hindi}}
\flushright{\begin{Arabic}
\quranayah[23][38]
\end{Arabic}}
\flushleft{\begin{hindi}
और हम तो कभी उस पर ईमान लाने वाले नहीं (ये हालत देखकर) सालेह ने दुआ की ऐ मेरे पालने वाले चूँकि इन लोगों ने मुझे झुठला दिया
\end{hindi}}
\flushright{\begin{Arabic}
\quranayah[23][39]
\end{Arabic}}
\flushleft{\begin{hindi}
तू मेरी मदद कर ख़ुदा ने फरमाया (एक ज़रा ठहर जाओ)
\end{hindi}}
\flushright{\begin{Arabic}
\quranayah[23][40]
\end{Arabic}}
\flushleft{\begin{hindi}
अनक़रीब ही ये लोग नादिम व परेशान हो जाएँगे
\end{hindi}}
\flushright{\begin{Arabic}
\quranayah[23][41]
\end{Arabic}}
\flushleft{\begin{hindi}
ग़रज़ उन्हें यक़ीनन एक सख्त चिंघाड़ ने ले डाला तो हमने उन्हें कूडे क़रकट (का ढेर) बना छोड़ा पस ज़ालिमों पर (खुदा की) लानत है
\end{hindi}}
\flushright{\begin{Arabic}
\quranayah[23][42]
\end{Arabic}}
\flushleft{\begin{hindi}
फिर हमने उनके बाद दूसरी क़ौमों को पैदा किया
\end{hindi}}
\flushright{\begin{Arabic}
\quranayah[23][43]
\end{Arabic}}
\flushleft{\begin{hindi}
कोई उम्मत अपने वक्त मुर्करर से न आगे बढ़ सकती है न (उससे) पीछे हट सकती है
\end{hindi}}
\flushright{\begin{Arabic}
\quranayah[23][44]
\end{Arabic}}
\flushleft{\begin{hindi}
फिर हमने लगातार बहुत से पैग़म्बर भेजे (मगर) जब जब किसी उम्मत का पैग़म्बर उन के पास आता तो ये लोग उसको झुठलाते थे तो हम थी (आगे पीछे) एक को दूसरे के बाद (हलाक) करते गए और हमने उन्हें (नेस्त व नाबूद करके) अफसाना बना दिया तो ईमान न लाने वालो पर ख़ुदा की लानत है
\end{hindi}}
\flushright{\begin{Arabic}
\quranayah[23][45]
\end{Arabic}}
\flushleft{\begin{hindi}
फिर हमने मूसा और उनके भाई हारुन को अपनी निशानियों और वाज़ेए व रौशन दलील के साथ फिरऔन और उसके दरबार के उमराओ के पास रसूल बना कर भेजा
\end{hindi}}
\flushright{\begin{Arabic}
\quranayah[23][46]
\end{Arabic}}
\flushleft{\begin{hindi}
तो उन लोगो ने शेख़ी की और वह थे ही बड़े सरकश लोग
\end{hindi}}
\flushright{\begin{Arabic}
\quranayah[23][47]
\end{Arabic}}
\flushleft{\begin{hindi}
आपस मे कहने लगे क्या हम अपने ही ऐसे दो आदमियों पर ईमान ले आएँ हालाँकि इन दोनों की (क़ौम की) क़ौम हमारी ख़िदमत गारी करती है
\end{hindi}}
\flushright{\begin{Arabic}
\quranayah[23][48]
\end{Arabic}}
\flushleft{\begin{hindi}
गरज़ उन लोगों ने इन दोनों को झुठलाया तो आख़िर ये सब के सब हलाक कर डाले गए
\end{hindi}}
\flushright{\begin{Arabic}
\quranayah[23][49]
\end{Arabic}}
\flushleft{\begin{hindi}
और हमने मूसा को किताब (तौरैत) इसलिए अता की थी कि ये लोग हिदायत पाएँ
\end{hindi}}
\flushright{\begin{Arabic}
\quranayah[23][50]
\end{Arabic}}
\flushleft{\begin{hindi}
और हमने मरियम के बेटे (ईसा) और उनकी माँ को (अपनी कुदरत की निशानी बनाया था) और उन दोनों को हमने एक ऊँची हमवार ठहरने के क़ाबिल चश्में वाली ज़मीन पर (रहने की) जगह दी
\end{hindi}}
\flushright{\begin{Arabic}
\quranayah[23][51]
\end{Arabic}}
\flushleft{\begin{hindi}
और मेरा आम हुक्म था कि ऐ (मेरे पैग़म्बर) पाक व पाकीज़ा चीज़ें खाओ और अच्छे अच्छे काम करो (क्योंकि) तुम जो कुछ करते हो मैं उससे बख़ूबी वाक़िफ हूँ
\end{hindi}}
\flushright{\begin{Arabic}
\quranayah[23][52]
\end{Arabic}}
\flushleft{\begin{hindi}
(लोगों ये दीन इस्लाम) तुम सबका मज़हब एक ही मज़हब है और मै तुम लोगों का परवरदिगार हूँ
\end{hindi}}
\flushright{\begin{Arabic}
\quranayah[23][53]
\end{Arabic}}
\flushleft{\begin{hindi}
तो बस मुझी से डरते रहो फिर लोगों ने अपने काम (में एख़तिलाफ करके उस) को टुकड़े टुकड़े कर डाला हर गिरो जो कुछ उसके पास है उसी में निहाल निहाल है
\end{hindi}}
\flushright{\begin{Arabic}
\quranayah[23][54]
\end{Arabic}}
\flushleft{\begin{hindi}
तो (ऐ रसूल) तुम उन लोगों को उन की ग़फलत में एक ख़ास वक्त तक (पड़ा) छोड़ दो
\end{hindi}}
\flushright{\begin{Arabic}
\quranayah[23][55]
\end{Arabic}}
\flushleft{\begin{hindi}
क्या ये लोग ये ख्याल करते है कि हम जो उन्हें माल और औलाद में तरक्क़ी दे रहे है तो हम उनके साथ भलाईयाँ करने में जल्दी कर रहे है
\end{hindi}}
\flushright{\begin{Arabic}
\quranayah[23][56]
\end{Arabic}}
\flushleft{\begin{hindi}
(ऐसा नहीं) बल्कि ये लोग समझते नहीं
\end{hindi}}
\flushright{\begin{Arabic}
\quranayah[23][57]
\end{Arabic}}
\flushleft{\begin{hindi}
उसमें शक नहीं कि जो लोग अपने परवरदिगार की वहशत से लरज़ रहे है
\end{hindi}}
\flushright{\begin{Arabic}
\quranayah[23][58]
\end{Arabic}}
\flushleft{\begin{hindi}
और जो लोग अपने परवरदिगार की (क़ुदरत की) निशानियों पर ईमान रखते हैं
\end{hindi}}
\flushright{\begin{Arabic}
\quranayah[23][59]
\end{Arabic}}
\flushleft{\begin{hindi}
और अपने परवरदिगार का किसी को शरीक नही बनाते
\end{hindi}}
\flushright{\begin{Arabic}
\quranayah[23][60]
\end{Arabic}}
\flushleft{\begin{hindi}
और जो लोग (ख़ुदा की राह में) जो कुछ बन पड़ता है देते हैं और फिर उनके दिल को इस बात का खटका लगा हुआ है कि उन्हें अपने परवरदिगार के सामने लौट कर जाना है
\end{hindi}}
\flushright{\begin{Arabic}
\quranayah[23][61]
\end{Arabic}}
\flushleft{\begin{hindi}
(देखिये क्या होता है) यही लोग अलबत्ता नेकियों में जल्दी करते हैं और भलाई की तरफ (दूसरों से) लपक के आगे बढ़ जाते हैं
\end{hindi}}
\flushright{\begin{Arabic}
\quranayah[23][62]
\end{Arabic}}
\flushleft{\begin{hindi}
और हम तो किसी शख्स को उसकी क़ूवत से बढ़के तकलीफ देते ही नहीं और हमारे पास तो (लोगों के आमाल की) किताब है जो बिल्कुल ठीक (हाल बताती है) और उन लोगों की (ज़र्रा बराबर) हक़ तलफी नहीं की जाएगी
\end{hindi}}
\flushright{\begin{Arabic}
\quranayah[23][63]
\end{Arabic}}
\flushleft{\begin{hindi}
उनके दिल उसकी तरफ से ग़फलत में पडे हैं इसके अलावा उन के बहुत से आमाल हैं जिन्हें ये (बराबर किया करते है) और बाज़ नहीं आते
\end{hindi}}
\flushright{\begin{Arabic}
\quranayah[23][64]
\end{Arabic}}
\flushleft{\begin{hindi}
यहाँ तक कि जब हम उनके मालदारों को अज़ाब में गिरफ््तार करेंगे तो ये लोग वावैला करने लगेंगें
\end{hindi}}
\flushright{\begin{Arabic}
\quranayah[23][65]
\end{Arabic}}
\flushleft{\begin{hindi}
(उस वक्त क़हा जाएगा) आज वावैला मत करों तुमको अब हमारी तरफ से मदद नहीं मिल सकती
\end{hindi}}
\flushright{\begin{Arabic}
\quranayah[23][66]
\end{Arabic}}
\flushleft{\begin{hindi}
(जब) हमारी आयतें तुम्हारे सामने पढ़ी जाती थीं तो तुम अकड़ते किस्सा कहते बकते हुए उन से उलटे पाँव फिर जाते
\end{hindi}}
\flushright{\begin{Arabic}
\quranayah[23][67]
\end{Arabic}}
\flushleft{\begin{hindi}
तो क्या उन लोगों ने (हमारी) बात (कुरान) पर ग़ौर नहीं किया
\end{hindi}}
\flushright{\begin{Arabic}
\quranayah[23][68]
\end{Arabic}}
\flushleft{\begin{hindi}
उनके पास कोई ऐसी नयी चीज़ आयी जो उनके अगले बाप दादाओं के पास नहीं आयी थी
\end{hindi}}
\flushright{\begin{Arabic}
\quranayah[23][69]
\end{Arabic}}
\flushleft{\begin{hindi}
या उन लोगों ने अपने रसूल ही को नहीं पहचाना तो इस वजह से इन्कार कर बैठे
\end{hindi}}
\flushright{\begin{Arabic}
\quranayah[23][70]
\end{Arabic}}
\flushleft{\begin{hindi}
या कहते हैं कि इसको जुनून हो गया है (हरगिज़ उसे जुनून नहीं) बल्कि वह तो उनके पास हक़ बात लेकर आया है और उनमें के अक्सर हक़ बात से नफरत रखते हैं
\end{hindi}}
\flushright{\begin{Arabic}
\quranayah[23][71]
\end{Arabic}}
\flushleft{\begin{hindi}
और अगर कहीं हक़ उनकी नफसियानी ख्वाहिश की पैरवी करता है तो सारे आसमान व ज़मीन और जो लोग उनमें हैं (सबके सब) बरबाद हो जाते बल्कि हम तो उन्हीं के तज़किरे (जिबरील के वास्ते से) उनके पास लेकर आए तो यह लोग अपने ही तज़किरे से मुँह मोड़तें हैं
\end{hindi}}
\flushright{\begin{Arabic}
\quranayah[23][72]
\end{Arabic}}
\flushleft{\begin{hindi}
(ऐ रसूल) क्या तुम उनसे (अपनी रिसालत की) कुछ उजरत माँगतें हों तो तुम्हारे परवरदिगार की उजरत उससे कही बेहतर है और वह तो सबसे बेहतर रोज़ी देने वाला है
\end{hindi}}
\flushright{\begin{Arabic}
\quranayah[23][73]
\end{Arabic}}
\flushleft{\begin{hindi}
और तुम तो यक़ीनन उनको सीधी राह की तरफ बुलाते हो
\end{hindi}}
\flushright{\begin{Arabic}
\quranayah[23][74]
\end{Arabic}}
\flushleft{\begin{hindi}
और इसमें शक नहीं कि जो लोग आख़िरत पर ईमान नहीं रखते वह सीधी राह से हटे हुए हैं
\end{hindi}}
\flushright{\begin{Arabic}
\quranayah[23][75]
\end{Arabic}}
\flushleft{\begin{hindi}
और अगर हम उन पर तरस खायें और जो तकलीफें उनको (कुफ्र की वजह से) पहुँच रही हैं उन को दफा कर दें तो यक़ीनन ये लोग (और भी) अपनी सरकशी पर अड़ जाए और भटकते फिरें
\end{hindi}}
\flushright{\begin{Arabic}
\quranayah[23][76]
\end{Arabic}}
\flushleft{\begin{hindi}
और हमने उनको अज़ाब में गिरफ्तार किया तो भी वे लोग न तो अपने परवरदिगार के सामने झुके और गिड़गिड़ाएँ
\end{hindi}}
\flushright{\begin{Arabic}
\quranayah[23][77]
\end{Arabic}}
\flushleft{\begin{hindi}
यहाँ तक कि जब हमने उनके सामने एक सख्त अज़ाब का दरवाज़ा खोल दिया तो उस वक्त फ़ौरन ये लोग बेआस होकर बैठ रहे
\end{hindi}}
\flushright{\begin{Arabic}
\quranayah[23][78]
\end{Arabic}}
\flushleft{\begin{hindi}
हालाँकि वही वह (मेहरबान खुदा) है जिसने तुम्हारे लिए कान और ऑंखें और दिल पैदा किये (मगर) तुम लोग हो ही बहुत कम शुक्र करने वाले
\end{hindi}}
\flushright{\begin{Arabic}
\quranayah[23][79]
\end{Arabic}}
\flushleft{\begin{hindi}
और वह वही (ख़ुदा) है जिसने तुम को रुए ज़मीन में (हर तरफ) फैला दिया और फिर (एक दिन) सब के सब उसी के सामने इकट्ठे किये जाओगे
\end{hindi}}
\flushright{\begin{Arabic}
\quranayah[23][80]
\end{Arabic}}
\flushleft{\begin{hindi}
और वही वह (ख़ुदा) है जो जिलाता और मारता है कि और रात दिन का फेर बदल भी उसी के एख्तियार में है तो क्या तुम (इतना भी) नहीं समझते
\end{hindi}}
\flushright{\begin{Arabic}
\quranayah[23][81]
\end{Arabic}}
\flushleft{\begin{hindi}
(इन बातों को समझें ख़ाक नहीं) बल्कि जो अगले लोग कहते आए वैसी ही बात ये भी कहने लगे
\end{hindi}}
\flushright{\begin{Arabic}
\quranayah[23][82]
\end{Arabic}}
\flushleft{\begin{hindi}
कि जब हम मर जाएँगें और (मरकर) मिट्टी (का ढ़ेर) और हड्डियाँ हो जाएँगें तो क्या हम फिर दोबारा (क्रबों से ज़िन्दा करके) निकाले जाएँगे
\end{hindi}}
\flushright{\begin{Arabic}
\quranayah[23][83]
\end{Arabic}}
\flushleft{\begin{hindi}
इसका वायदा तो हमसे और हमसे पहले हमारे बाप दादाओं से भी (बार हा) किया जा चुका है ये तो बस सिर्फ अगले लोगों के ढकोसले हैं
\end{hindi}}
\flushright{\begin{Arabic}
\quranayah[23][84]
\end{Arabic}}
\flushleft{\begin{hindi}
(ऐ रसूल) तुम कह दो कि भला अगर तुम लोग कुछ जानते हो (तो बताओ) ये ज़मीन और जो लोग इसमें हैं किस के हैं वह फौरन जवाब देगें ख़ुदा के
\end{hindi}}
\flushright{\begin{Arabic}
\quranayah[23][85]
\end{Arabic}}
\flushleft{\begin{hindi}
तुम कह दो कि तो क्या तुम अब भी ग़ौर न करोगे
\end{hindi}}
\flushright{\begin{Arabic}
\quranayah[23][86]
\end{Arabic}}
\flushleft{\begin{hindi}
(ऐ रसूल) तुम उनसे पूछो तो कि सातों आसमानों का मालिक और (इतने) बड़े अर्श का मालिक कौन है तो फौरन जवाब देगें कि (सब कुछ) खुदा ही का है
\end{hindi}}
\flushright{\begin{Arabic}
\quranayah[23][87]
\end{Arabic}}
\flushleft{\begin{hindi}
अब तुम कहो तो क्या तुम अब भी (उससे) नहीं डरोगे
\end{hindi}}
\flushright{\begin{Arabic}
\quranayah[23][88]
\end{Arabic}}
\flushleft{\begin{hindi}
(ऐ रसूल) तुम उनसे पूछो कि भला अगर तुम कुछ जानते हो (तो बताओ) कि वह कौन शख्स है- जिसके एख्तेयार में हर चीज़ की बादशाहत है वह (जिसे चाहता है) पनाह देता है और उस (के अज़ाब) से पनाह नहीं दी जा सकती
\end{hindi}}
\flushright{\begin{Arabic}
\quranayah[23][89]
\end{Arabic}}
\flushleft{\begin{hindi}
तो ये लोग फौरन बोल उठेंगे कि (सब एख्तेयार) ख़ुदा ही को है- अब तुम कह दो कि तुम पर जादू कहाँ किया जाता है
\end{hindi}}
\flushright{\begin{Arabic}
\quranayah[23][90]
\end{Arabic}}
\flushleft{\begin{hindi}
बात ये है कि हमने उनके पास हक़ बात पहुँचा दी और ये लोग यक़ीनन झूठे हैं
\end{hindi}}
\flushright{\begin{Arabic}
\quranayah[23][91]
\end{Arabic}}
\flushleft{\begin{hindi}
न तो अल्लाह ने किसी को (अपना) बेटा बनाया है और न उसके साथ कोई और ख़ुदा है (अगर ऐसा होता) उस वक्त हर खुदा अपने अपने मख़लूक़ को लिए लिए फिरता और यक़ीनन एक दूसरे पर चढ़ाई करता
\end{hindi}}
\flushright{\begin{Arabic}
\quranayah[23][92]
\end{Arabic}}
\flushleft{\begin{hindi}
(और ख़ूब जंग होती) जो जो बाते ये लोग (ख़ुदा की निस्बत) बयान करते हैं उस से ख़ुदा पाक व पाकीज़ा है वह पोशीदा और हाज़िर (सबसे) खुदा वाक़िफ है ग़रज़ वह उनके शिर्क से (बिल्कुल पाक और) बालातर है
\end{hindi}}
\flushright{\begin{Arabic}
\quranayah[23][93]
\end{Arabic}}
\flushleft{\begin{hindi}
(ऐ रसूल) तुम दुआ करो कि ऐ मेरे पालने वाले जिस (अज़ाब) का तूने उनसे वायदा किया है अगर शायद तू मुझे दिखाए
\end{hindi}}
\flushright{\begin{Arabic}
\quranayah[23][94]
\end{Arabic}}
\flushleft{\begin{hindi}
तो परवरदिगार मुझे उन ज़ालिम लोगों के हमराह न करना
\end{hindi}}
\flushright{\begin{Arabic}
\quranayah[23][95]
\end{Arabic}}
\flushleft{\begin{hindi}
और (ऐ रसूल) हम यक़ीनन इस पर क़ादिर हैं कि जिस (अज़ाब) का हम उनसे वायदा करते हैं तुम्हें दिखा दें
\end{hindi}}
\flushright{\begin{Arabic}
\quranayah[23][96]
\end{Arabic}}
\flushleft{\begin{hindi}
और बुरी बात के जवाब में ऐसी बात कहो जो निहायत अच्छी हो जो कुछ ये लोग (तुम्हारी निस्बत) बयान करते हैं उससे हम ख़ूब वाक़िफ हैं
\end{hindi}}
\flushright{\begin{Arabic}
\quranayah[23][97]
\end{Arabic}}
\flushleft{\begin{hindi}
और (ये भी) दुआ करो कि ऐ मेरे पालने वाले मै शैतान के वसवसों से तेरी पनाह माँगता हूँ
\end{hindi}}
\flushright{\begin{Arabic}
\quranayah[23][98]
\end{Arabic}}
\flushleft{\begin{hindi}
और ऐ मेरे परवरदिगार इससे भी तेरी पनाह माँगता हूँ कि शयातीन मेरे पास आएँ
\end{hindi}}
\flushright{\begin{Arabic}
\quranayah[23][99]
\end{Arabic}}
\flushleft{\begin{hindi}
(और कुफ्फ़ार तो मानेगें नहीं) यहाँ तक कि जब उनमें से किसी को मौत आयी तो कहने लगे परवरदिगार तू मुझे (एक बार) उस मुक़ाम (दुनिया) में छोड़ आया हूँ फिर वापस कर दे ताकि मै (अपकी दफ़ा) अच्छे अच्छे काम करूं
\end{hindi}}
\flushright{\begin{Arabic}
\quranayah[23][100]
\end{Arabic}}
\flushleft{\begin{hindi}
(जवाब दिया जाएगा) हरगिज़ नहीं ये एक लग़ो बात है- जिसे वह बक रहा और उनके (मरने के) बाद (आलमे) बरज़ख़ है
\end{hindi}}
\flushright{\begin{Arabic}
\quranayah[23][101]
\end{Arabic}}
\flushleft{\begin{hindi}
(जहाँ) क़ब्रों से उठाए जाएँगें (रहना होगा) फिर जिस वक्त सूर फूँका जाएगा तो उस दिन न लोगों में क़राबत दारियाँ रहेगी और न एक दूसरे की बात पूछेंगे
\end{hindi}}
\flushright{\begin{Arabic}
\quranayah[23][102]
\end{Arabic}}
\flushleft{\begin{hindi}
फिर जिन (के नेकियों) के पल्लें भारी होगें तो यही लोग कामयाब होंगे
\end{hindi}}
\flushright{\begin{Arabic}
\quranayah[23][103]
\end{Arabic}}
\flushleft{\begin{hindi}
और जिन (के नेकियों) के पल्लें हल्के होंगे तो यही लोग है जिन्होंने अपना नुक़सान किया कि हमेशा जहन्नुम में रहेंगे
\end{hindi}}
\flushright{\begin{Arabic}
\quranayah[23][104]
\end{Arabic}}
\flushleft{\begin{hindi}
और (उनकी ये हालत होगी कि) जहन्नुम की आग उनके मुँह को झुलसा देगी और लोग मुँह बनाए हुए होगें
\end{hindi}}
\flushright{\begin{Arabic}
\quranayah[23][105]
\end{Arabic}}
\flushleft{\begin{hindi}
(उस वक्त हम पूछेंगें) क्या तुम्हारे सामने मेरी आयतें न पढ़ी गयीं थीं (ज़रुर पढ़ी गयी थीं) तो तुम उन्हें झुठलाया करते थे
\end{hindi}}
\flushright{\begin{Arabic}
\quranayah[23][106]
\end{Arabic}}
\flushleft{\begin{hindi}
वह जवाब देगें ऐ हमारे परवरदिगार हमको हमारी कम्बख्ती ने आज़माया और हम गुमराह लोग थे
\end{hindi}}
\flushright{\begin{Arabic}
\quranayah[23][107]
\end{Arabic}}
\flushleft{\begin{hindi}
परवरदिगार हमको (अबकी दफा) किसी तरह इस जहन्नुम से निकाल दे फिर अगर दोबारा हम ऐसा करें तो अलबत्ता हम कुसूरवार हैं
\end{hindi}}
\flushright{\begin{Arabic}
\quranayah[23][108]
\end{Arabic}}
\flushleft{\begin{hindi}
ख़ुदा फरमाएगा दूर हो इसी में (तुम को रहना होगा) और (बस) मुझ से बात न करो
\end{hindi}}
\flushright{\begin{Arabic}
\quranayah[23][109]
\end{Arabic}}
\flushleft{\begin{hindi}
मेरे बन्दों में से एक गिरोह ऐसा भी था जो (बराबर) ये दुआ करता था कि ऐ हमारे पालने वाले हम ईमान लाए तो तू हमको बख्श दे और हम पर रहम कर तू तो तमाम रहम करने वालों से बेहतर है
\end{hindi}}
\flushright{\begin{Arabic}
\quranayah[23][110]
\end{Arabic}}
\flushleft{\begin{hindi}
तो तुम लोगों ने उन्हें मसखरा बना लिया-यहाँ तक कि (गोया) उन लोगों ने तुम से मेरी याद भुला दी और तुम उन पर (बराबर) हँसते रहे
\end{hindi}}
\flushright{\begin{Arabic}
\quranayah[23][111]
\end{Arabic}}
\flushleft{\begin{hindi}
मैने आज उनको उनके सब्र का अच्छा बदला दिया कि यही लोग अपनी (ख़ातिरख्वाह) मुराद को पहुँचने वाले हैं
\end{hindi}}
\flushright{\begin{Arabic}
\quranayah[23][112]
\end{Arabic}}
\flushleft{\begin{hindi}
(फिर उनसे) ख़ुदा पूछेगा कि (आख़िर) तुम ज़मीन पर कितने बरस रहे
\end{hindi}}
\flushright{\begin{Arabic}
\quranayah[23][113]
\end{Arabic}}
\flushleft{\begin{hindi}
वह कहेंगें (बरस कैसा) हम तो बस पूरा एक दिन रहे या एक दिन से भी कम
\end{hindi}}
\flushright{\begin{Arabic}
\quranayah[23][114]
\end{Arabic}}
\flushleft{\begin{hindi}
तो तुम शुमार करने वालों से पूछ लो ख़ुदा फरमाएगा बेशक तुम (ज़मीन में) बहुत ही कम ठहरे काश तुम (इतनी बात भी दुनिया में) समझे होते
\end{hindi}}
\flushright{\begin{Arabic}
\quranayah[23][115]
\end{Arabic}}
\flushleft{\begin{hindi}
तो क्या तुम ये ख्याल करते हो कि हमने तुमको (यूँ ही) बेकार पैदा किया और ये कि तुम हमारे हुज़ूर में लौटा कर न लाए जाओगे
\end{hindi}}
\flushright{\begin{Arabic}
\quranayah[23][116]
\end{Arabic}}
\flushleft{\begin{hindi}
तो ख़ुदा जो सच्चा बादशाह (हर चीज़ से) बरतर व आला है उसके सिवा कोई माबूद नहीं (वहीं) अर्श बुर्जुग़ का मालिक है
\end{hindi}}
\flushright{\begin{Arabic}
\quranayah[23][117]
\end{Arabic}}
\flushleft{\begin{hindi}
और जो शख्स ख़ुदा के साथ दूसरे माबूद की भी परसतिश करेगा उसके पास इस शिर्क की कोई दलील तो है नहीं तो बस उसका हिसाब (किताब) उसके परवरदिगार ही के पास होगा (मगर याद रहे कि कुफ्फ़ार हरगिज़ फलॉह पाने वाले नहीं)
\end{hindi}}
\flushright{\begin{Arabic}
\quranayah[23][118]
\end{Arabic}}
\flushleft{\begin{hindi}
और (ऐ रसूल) तुम कह दो परवरदिगार तू (मेरी उम्मत को) बख्श दे और तरस खा और तू तो सब रहम करने वालों से बेहतर है
\end{hindi}}
\chapter{An-Nur (The Light)}
\begin{Arabic}
\Huge{\centerline{\basmalah}}\end{Arabic}
\flushright{\begin{Arabic}
\quranayah[24][1]
\end{Arabic}}
\flushleft{\begin{hindi}
(ये) एक सूरा है जिसे हमने नाज़िल किया है और उस (के एहक़ाम) को फर्ज क़र दिया है और इसमें हमने वाज़ेए व रौशन आयतें नाज़िल की हैं ताकि तुम (ग़ौर करके) नसीहत हासिल करो
\end{hindi}}
\flushright{\begin{Arabic}
\quranayah[24][2]
\end{Arabic}}
\flushleft{\begin{hindi}
़ज़िना करने वाली औरत और ज़िना करने वाले मर्द इन दोनों में से हर एक को सौ (सौ) कोडे मारो और अगर तुम ख़ुदा और रोजे अाख़िरत पर ईमान रखते हो तो हुक्मे खुदा के नाफिज़ करने में तुमको उनके बारे में किसी तरह की तरस का लिहाज़ न होने पाए और उन दोनों की सज़ा के वक्त मोमिन की एक जमाअत को मौजूद रहना चाहिए
\end{hindi}}
\flushright{\begin{Arabic}
\quranayah[24][3]
\end{Arabic}}
\flushleft{\begin{hindi}
ज़िना करने वाला मर्द तो ज़िना करने वाली औरत या मुशरिका से निकाह करेगा और ज़िना करने वाली औरत भी बस ज़िना करने वाले ही मर्द या मुशरिक से निकाह करेगी और सच्चे ईमानदारों पर तो इस क़िस्म के ताल्लुक़ात हराम हैं
\end{hindi}}
\flushright{\begin{Arabic}
\quranayah[24][4]
\end{Arabic}}
\flushleft{\begin{hindi}
और जो लोग पाक दामन औरतों पर (ज़िना की) तोहमत लगाएँ फिर (अपने दावे पर) चार गवाह पेश न करें तो उन्हें अस्सी कोड़ें मारो और फिर (आइन्दा) कभी उनकी गवाही कुबूल न करो और (याद रखो कि) ये लोग ख़ुद बदकार हैं
\end{hindi}}
\flushright{\begin{Arabic}
\quranayah[24][5]
\end{Arabic}}
\flushleft{\begin{hindi}
मगर हाँ जिन लोगों ने उसके बाद तौबा कर ली और अपनी इसलाह की तो बेशक ख़ुदा बड़ा बख्शने वाला मेहरबान है
\end{hindi}}
\flushright{\begin{Arabic}
\quranayah[24][6]
\end{Arabic}}
\flushleft{\begin{hindi}
और जो लोग अपनी बीवियों पर (ज़िना) का ऐब लगाएँ और (इसके सुबूत में) अपने सिवा उनका कोई गवाह न हो तो ऐसे लोगों में से एक की गवाही चार मरतबा इस तरह होगी कि वह (हर मरतबा) ख़ुदा की क़सम खाकर बयान करे कि वह (अपने दावे में) जरुर सच्चा है
\end{hindi}}
\flushright{\begin{Arabic}
\quranayah[24][7]
\end{Arabic}}
\flushleft{\begin{hindi}
और पाँचवी (मरतबा) यूँ (कहेगा) अगर वह झूट बोलता हो तो उस पर ख़ुदा की लानत
\end{hindi}}
\flushright{\begin{Arabic}
\quranayah[24][8]
\end{Arabic}}
\flushleft{\begin{hindi}
और औरत (के सर से) इस तरह सज़ा टल सकती है कि वह चार मरतबा ख़ुदा की क़सम खा कर बयान कर दे कि ये शख्स (उसका शौहर अपने दावे में) ज़रुर झूठा है
\end{hindi}}
\flushright{\begin{Arabic}
\quranayah[24][9]
\end{Arabic}}
\flushleft{\begin{hindi}
और पाँचवी मरतबा यूँ करेगी कि अगर ये शख्स (अपने दावे में) सच्चा हो तो मुझ पर खुदा का ग़ज़ब पड़े
\end{hindi}}
\flushright{\begin{Arabic}
\quranayah[24][10]
\end{Arabic}}
\flushleft{\begin{hindi}
और अगर तुम पर ख़ुदा का फज़ल (व करम) और उसकी मेहरबानी न होती तो देखते कि तोहमत लगाने वालों का क्या हाल होता और इसमें शक ही नहीं कि ख़ुदा बड़ा तौबा क़ुबूल करने वाला हकीम है
\end{hindi}}
\flushright{\begin{Arabic}
\quranayah[24][11]
\end{Arabic}}
\flushleft{\begin{hindi}
बेशक जिन लोगों ने झूठी तोहमत लगायी वह तुम्ही में से एक गिरोह है तुम अपने हक़ में इस तोहमत को बड़ा न समझो बल्कि ये तुम्हारे हक़ में बेहतर है इनमें से जिस शख्स ने जितना गुनाह समेटा वह उस (की सज़ा) को खुद भुगतेगा और उनमें से जिस शख्स ने तोहमत का बड़ा हिस्सा लिया उसके लिए बड़ी (सख्त) सज़ा होगी
\end{hindi}}
\flushright{\begin{Arabic}
\quranayah[24][12]
\end{Arabic}}
\flushleft{\begin{hindi}
और जब तुम लोगो ने उसको सुना था तो उसी वक्त ईमानदार मर्दों और ईमानदार औरतों ने अपने लोगों पर भलाई का गुमान क्यो न किया और ये क्यों न बोल उठे कि ये तो खुला हुआ बोहतान है
\end{hindi}}
\flushright{\begin{Arabic}
\quranayah[24][13]
\end{Arabic}}
\flushleft{\begin{hindi}
और जिन लोगों ने तोहमत लगायी थी अपने दावे के सुबूत में चार गवाह क्यों न पेश किए फिर जब इन लोगों ने गवाह न पेश किये तो ख़ुदा के नज़दीक यही लोग झूठे हैं
\end{hindi}}
\flushright{\begin{Arabic}
\quranayah[24][14]
\end{Arabic}}
\flushleft{\begin{hindi}
और अगर तुम लोगों पर दुनिया और आख़िरत में ख़ुदा का फज़ल (व करम) और उसकी रहमत न होती तो जिस बात का तुम लोगों ने चर्चा किया था उस की वजह से तुम पर कोई बड़ा (सख्त) अज़ाब आ पहुँचता
\end{hindi}}
\flushright{\begin{Arabic}
\quranayah[24][15]
\end{Arabic}}
\flushleft{\begin{hindi}
कि तुम अपनी ज़बानों से इसको एक दूसरे से बयान करने लगे और अपने मुँह से ऐसी बात कहते थे जिसका तुम्हें इल्म व यक़ीन न था (और लुत्फ ये है कि) तुमने इसको एक आसान बात समझी थी हॉलाकि वह ख़ुदा के नज़दीक बड़ी सख्त बात थी
\end{hindi}}
\flushright{\begin{Arabic}
\quranayah[24][16]
\end{Arabic}}
\flushleft{\begin{hindi}
और जब तुमने ऐसी बात सुनी थी तो तुमने लोगों से क्यों न कह दिया कि हमको ऐसी बात मुँह से निकालनी मुनासिब नहीं सुबहान अल्लाह ये बड़ा भारी बोहतान है
\end{hindi}}
\flushright{\begin{Arabic}
\quranayah[24][17]
\end{Arabic}}
\flushleft{\begin{hindi}
ख़ुदा तुम्हारी नसीहत करता है कि अगर तुम सच्चे ईमानदार हो तो ख़बरदार फिर कभी ऐसा न करना
\end{hindi}}
\flushright{\begin{Arabic}
\quranayah[24][18]
\end{Arabic}}
\flushleft{\begin{hindi}
और ख़ुदा तुम से (अपने) एहकाम साफ साफ बयान करता है और ख़ुदा तो बड़ा वाक़िफकार हकीम है
\end{hindi}}
\flushright{\begin{Arabic}
\quranayah[24][19]
\end{Arabic}}
\flushleft{\begin{hindi}
जो लोग ये चाहते हैं कि ईमानदारों में बदकारी का चर्चा फैल जाए बेशक उनके लिए दुनिया और आख़िरत में दर्दनाक अज़ाब है और ख़ुदा (असल हाल को) ख़ूब जानता है और तुम लोग नहीं जानते हो
\end{hindi}}
\flushright{\begin{Arabic}
\quranayah[24][20]
\end{Arabic}}
\flushleft{\begin{hindi}
और अगर ये बात न होती कि तुम पर ख़ुदा का फ़ज़ल (व करम) और उसकी रहमत से और ये कि ख़ुदा (अपने बन्दों पर) बड़ा शफीक़ मेहरबान है
\end{hindi}}
\flushright{\begin{Arabic}
\quranayah[24][21]
\end{Arabic}}
\flushleft{\begin{hindi}
(तो तुम देखते क्या होता) ऐ ईमानदारों शैतान के क़दम ब क़दम न चलो और जो शख्स शैतान के क़दम ब क़दम चलेगा तो वह यक़ीनन उसे बदकारी और बुरी बात (करने) का हुक्म देगा और अगर तुम पर ख़ुदा का फ़ज़ल (व करम) और उसकी रहमत न होती तो तुममें से कोई भी कभी पाक साफ न होता मगर ख़ुदा तो जिसे चहता है पाक साफ कर देता है और ख़ुदा बड़ा सुनने वाला वाकिफकार है
\end{hindi}}
\flushright{\begin{Arabic}
\quranayah[24][22]
\end{Arabic}}
\flushleft{\begin{hindi}
और तुममें से जो लोग ज्यादा दौलत और मुक़द्दर वालें है क़राबतदारों और मोहताजों और ख़ुदा की राह में हिजरत करने वालों को कुछ देने (लेने) से क़सम न खा बैठें बल्कि उन्हें चाहिए कि (उनकी ख़ता) माफ कर दें और दरगुज़र करें क्या तुम ये नहीं चाहते हो कि ख़ुदा तुम्हारी ख़ता माफ करे और खुदा तो बड़ा बख्शने वाला मेहरबान है
\end{hindi}}
\flushright{\begin{Arabic}
\quranayah[24][23]
\end{Arabic}}
\flushleft{\begin{hindi}
बेशक जो लोग पाक दामन बेख़बर और ईमानदार औरतों पर (ज़िना की) तोहमत लगाते हैं उन पर दुनिया और आख़िरत में (ख़ुदा की) लानत है और उन पर बड़ा (सख्त) अज़ाब होगा
\end{hindi}}
\flushright{\begin{Arabic}
\quranayah[24][24]
\end{Arabic}}
\flushleft{\begin{hindi}
जिस दिन उनके ख़िलाफ उनकी ज़बानें और उनके हाथ उनके पावँ उनकी कारस्तानियों की गवाही देगें
\end{hindi}}
\flushright{\begin{Arabic}
\quranayah[24][25]
\end{Arabic}}
\flushleft{\begin{hindi}
उस दिन ख़ुदा उनको ठीक उनका पूरा पूरा बदला देगा और जान जाएँगें कि ख़ुदा बिल्कुल बरहक़ और (हक़ का) ज़ाहिर करने वाला है
\end{hindi}}
\flushright{\begin{Arabic}
\quranayah[24][26]
\end{Arabic}}
\flushleft{\begin{hindi}
गन्दी औरते गन्दें मर्दों के लिए (मुनासिब) हैं और गन्दे मर्द गन्दी औरतो के लिए और पाक औरतें पाक मर्दों के लिए (मौज़ूँ) हैं और पाक मर्द पाक औरतों के लिए लोग जो कुछ उनकी निस्बत बका करते हैं उससे ये लोग बुरी उल ज़िम्मा हैं उन ही (पाक लोगों) के लिए (आख़िरत में) बख़्शिस है
\end{hindi}}
\flushright{\begin{Arabic}
\quranayah[24][27]
\end{Arabic}}
\flushleft{\begin{hindi}
और इज्ज़त की रोज़ी ऐ ईमानदारों अपने घरों के सिवा दूसरे घरों में (दर्राना) न चले जाओ यहाँ तक कि उनसे इजाज़त ले लो और उन घरों के रहने वालों से साहब सलामत कर लो यही तुम्हारे हक़ में बेहतर है
\end{hindi}}
\flushright{\begin{Arabic}
\quranayah[24][28]
\end{Arabic}}
\flushleft{\begin{hindi}
(ये नसीहत इसलिए है) ताकि तुम याद रखो पस अगर तुम उन घरों में किसी को न पाओ तो तावाक़फियत कि तुम को (ख़ास तौर पर) इजाज़त न हासिल हो जाए उन में न जाओ और अगर तुम से कहा जाए कि फिर जाओ तो तुम (बे ताम्मुल) फिर जाओ यही तुम्हारे वास्ते ज्यादा सफाई की बात है और तुम जो कुछ भी करते हो ख़ुदा उससे खूब वाकिफ़ है
\end{hindi}}
\flushright{\begin{Arabic}
\quranayah[24][29]
\end{Arabic}}
\flushleft{\begin{hindi}
इसमें अलबत्ता तुम पर इल्ज़ाम नहीं कि ग़ैर आबाद मकानात में जिसमें तुम्हारा कोई असबाब हो (बे इजाज़त) चले जाओ और जो कुछ खुल्लम खुल्ला करते हो और जो कुछ छिपाकर करते हो खुदा (सब कुछ) जानता है
\end{hindi}}
\flushright{\begin{Arabic}
\quranayah[24][30]
\end{Arabic}}
\flushleft{\begin{hindi}
(ऐ रसूल) ईमानदारों से कह दो कि अपनी नज़रों को नीची रखें और अपनी शर्मगाहों की हिफाज़त करें यही उनके वास्ते ज्यादा सफाई की बात है ये लोग जो कुछ करते हैं ख़ुदा उससे यक़ीनन ख़ूब वाक़िफ है
\end{hindi}}
\flushright{\begin{Arabic}
\quranayah[24][31]
\end{Arabic}}
\flushleft{\begin{hindi}
और (ऐ रसूल) ईमानदार औरतों से भी कह दो कि वह भी अपनी नज़रें नीची रखें और अपनी शर्मगाहों की हिफाज़त करें और अपने बनाव सिंगार (के मक़ामात) को (किसी पर) ज़ाहिर न होने दें मगर जो खुद ब खुद ज़ाहिर हो जाता हो (छुप न सकता हो) (उसका गुनाह नही) और अपनी ओढ़नियों को (घूँघट मारके) अपने गरेबानों (सीनों) पर डाले रहें और अपने शौहर या अपने बाप दादाओं या आपने शौहर के बाप दादाओं या अपने बेटों या अपने शौहर के बेटों या अपने भाइयों या अपने भतीजों या अपने भांजों या अपने (क़िस्म की) औरतों या अपनी या अपनी लौंडियों या (घर के) नौकर चाकर जो मर्द सूरत हैं मगर (बहुत बूढे होने की वजह से) औरतों से कुछ मतलब नहीं रखते या वह कमसिन लड़के जो औरतों के पर्दे की बात से आगाह नहीं हैं उनके सिवा (किसी पर) अपना बनाव सिंगार ज़ाहिर न होने दिया करें और चलने में अपने पाँव ज़मीन पर इस तरह न रखें कि लोगों को उनके पोशीदा बनाव सिंगार (झंकार वग़ैरह) की ख़बर हो जाए और ऐ ईमानदारों तुम सबके सब ख़ुदा की बारगाह में तौबा करो ताकि तुम फलाह पाओ
\end{hindi}}
\flushright{\begin{Arabic}
\quranayah[24][32]
\end{Arabic}}
\flushleft{\begin{hindi}
और अपनी (क़ौम की) बेशौहर औरतों और अपने नेक बख्त गुलामों और लौंडियों का निकाह कर दिया करो अगर ये लोग मोहताज होंगे तो खुदा अपने फज़ल व (करम) से उन्हें मालदार बना देगा और ख़ुदा तो बड़ी गुन्जाइश वाला वाक़िफ कार है
\end{hindi}}
\flushright{\begin{Arabic}
\quranayah[24][33]
\end{Arabic}}
\flushleft{\begin{hindi}
और जो लोग निकाह करने का मक़दूर नहीं रखते उनको चाहिए कि पाक दामिनी एख्तियार करें यहाँ तक कि ख़ुदा उनको अपने फज़ल व (करम) से मालदार बना दे और तुम्हारी लौन्डी ग़ुलामों में से जो मकातबत होने (कुछ रुपए की शर्त पर आज़ादी का सरख़त लेने) की ख्वाहिश करें तो तुम अगर उनमें कुछ सलाहियत देखो तो उनको मकातिब कर दो और ख़ुदा के माल से जो उसने तुम्हें अता किया है उनका भी दो और तुम्हारी लौन्डियाँ जो पाक दामन ही रहना चाहती हैं उनको दुनियावी ज़िन्दगी के फायदे हासिल करने की ग़रज़ से हराम कारी पर मजबूर न करो और जो शख्स उनको मजबूर करेगा तो इसमें शक नहीं कि ख़ुदा उसकी बेबसी के बाद बड़ा बख्शने वाले मेहरबान है
\end{hindi}}
\flushright{\begin{Arabic}
\quranayah[24][34]
\end{Arabic}}
\flushleft{\begin{hindi}
और (ईमानदारों) हमने तो तुम्हारे पास (अपनी) वाज़ेए व रौशन आयतें और जो लोग तुमसे पहले गुज़र चुके हैं उनकी हालतें और परहेज़गारों के लिए नसीहत (की बाते) नाज़िल की
\end{hindi}}
\flushright{\begin{Arabic}
\quranayah[24][35]
\end{Arabic}}
\flushleft{\begin{hindi}
ख़ुदा तो सारे आसमान और ज़मीन का नूर है उसके नूर की मिसल (ऐसी) है जैसे एक ताक़ (सीना) है जिसमे एक रौशन चिराग़ (इल्मे शरीयत) हो और चिराग़ एक शीशे की क़न्दील (दिल) में हो (और) क़न्दील (अपनी तड़प में) गोया एक जगमगाता हुआ रौशन सितारा (वह चिराग़) जैतून के मुबारक दरख्त (के तेल) से रौशन किया जाए जो न पूरब की तरफ हो और न पश्चिम की तरफ (बल्कि बीचों बीच मैदान में) उसका तेल (ऐसा) शफ्फाफ हो कि अगरचे आग उसे छुए भी नही ताहम ऐसा मालूम हो कि आप ही आप रौशन हो जाएगा (ग़रज़ एक नूर नहीं बल्कि) नूर आला नूर (नूर की नूर पर जोत पड़ रही है) ख़ुदा अपने नूर की तरफ जिसे चाहता है हिदायत करता है और ख़ुदा तो हर चीज़ से खूब वाक़िफ है
\end{hindi}}
\flushright{\begin{Arabic}
\quranayah[24][36]
\end{Arabic}}
\flushleft{\begin{hindi}
(वह क़न्दील) उन घरों में रौशन है जिनकी निस्बत ख़ुदा ने हुक्म दिया कि उनकी ताज़ीम की जाए और उनमें उसका नाम लिया जाए जिनमें सुबह व शाम वह लोग उसकी तस्बीह किया करते हैं
\end{hindi}}
\flushright{\begin{Arabic}
\quranayah[24][37]
\end{Arabic}}
\flushleft{\begin{hindi}
ऐसे लोग जिनको ख़ुदा के ज़िक्र और नमाज़ पढ़ने और ज़कात अदा करने से न तो तिजारत ही ग़ाफिल कर सकती है न (ख़रीद फरोख्त) (का मामला क्योंकि) वह लोग उस दिन से डरते हैं जिसमें ख़ौफ के मारे दिल और ऑंखें उलट जाएँगी
\end{hindi}}
\flushright{\begin{Arabic}
\quranayah[24][38]
\end{Arabic}}
\flushleft{\begin{hindi}
(उसकी इबादत इसलिए करते हैं) ताकि ख़ुदा उन्हें उनके आमाल का बेहतर से बेहतर बदला अता फरमाए और अपने फज़ल व करम से कुछ और ज्यादा भी दे और ख़ुदा तो जिसे चाहता है बेहिसाब रोज़ी देता है
\end{hindi}}
\flushright{\begin{Arabic}
\quranayah[24][39]
\end{Arabic}}
\flushleft{\begin{hindi}
और जिन लोगों ने कुफ्र एख्तेयार किया उनकी कारस्तानियाँ (ऐसी है) जैसे एक चटियल मैदान का चमकता हुआ बालू कि प्यासा उस को दूर से देखे तो पानी ख्याल करता है यहाँ तक कि जब उसके पास आया तो उसको कुछ भी न पाया (और प्यास से तड़प कर मर गया) और ख़ुदा को अपने पास मौजूद पाया तो उसने उसका हिसाब (किताब) पूरा पूरा चुका दिया और ख़ुदा तो बहुत जल्द हिसाब लेने वाला है
\end{hindi}}
\flushright{\begin{Arabic}
\quranayah[24][40]
\end{Arabic}}
\flushleft{\begin{hindi}
(या काफिरों के आमाल की मिसाल) उस बड़े गहरे दरिया की तारिकियों की सी है- जैसे एक लहर उसके ऊपर दूसरी लहर उसके ऊपर अब्र (तह ब तह) ढॉके हुए हो (ग़रज़) तारिकियाँ है कि एक से ऊपर एक (उमड़ी) चली आती हैं (इसी तरह से) कि अगर कोइ शख्स अपना हाथ निकाले तो (शिद्दत तारीकी से) उसे देख न सके और जिसे खुद ख़ुदा ही ने (हिदायत की) रौशनी न दी हो तो (समझ लो कि) उसके लिए कहीं कोई रौशनी नहीं है
\end{hindi}}
\flushright{\begin{Arabic}
\quranayah[24][41]
\end{Arabic}}
\flushleft{\begin{hindi}
(ऐ शख्स) क्या तूने इतना भी नहीं देखा कि जितनी मख़लूक़ात सारे आसमान और ज़मीन में हैं और परिन्दें पर फैलाए (ग़रज़ सब) उसी को तस्बीह किया करते हैं सब के सब अपनी नमाज़ और अपनी तस्बीह का तरीक़ा खूब जानते हैं और जो कुछ ये किया करते हैं ख़ुदा उससे खूब वाक़िफ है
\end{hindi}}
\flushright{\begin{Arabic}
\quranayah[24][42]
\end{Arabic}}
\flushleft{\begin{hindi}
और सारे आसमान व ज़मीन की सल्तनत ख़ास ख़ुदा ही की है और ख़ुदा ही की तरफ (सब को) लौट कर जाना है
\end{hindi}}
\flushright{\begin{Arabic}
\quranayah[24][43]
\end{Arabic}}
\flushleft{\begin{hindi}
क्या तूने उस पर भी नज़र नहीं की कि यक़ीनन ख़ुदा ही अब्र को चलाता है फिर वही बाहम उसे जोड़ता है-फिर वही उसे तह ब तह रखता है तब तू तो बारिश उसके दरमियान से निकलते हुए देखता है और आसमान में जो (जमे हुए बादलों के) पहाड़ है उनमें से वही उसे बरसाता है- फिर उन्हें जिस (के सर) पर चाहता है पहुँचा देता है- और जिस (के सर) से चाहता है टाल देता है- क़रीब है कि उसकी बिजली की कौन्द आखों की रौशनी उचके लिये जाती है
\end{hindi}}
\flushright{\begin{Arabic}
\quranayah[24][44]
\end{Arabic}}
\flushleft{\begin{hindi}
ख़ुदा ही रात और दिन को फेर बदल करता रहता है- बेशक इसमें ऑंख वालों के लिए बड़ी इबरत है
\end{hindi}}
\flushright{\begin{Arabic}
\quranayah[24][45]
\end{Arabic}}
\flushleft{\begin{hindi}
और ख़ुदा ही ने तमाम ज़मीन पर चलने वाले (जानवरों) को पानी से पैदा किया उनमें से बाज़ तो ऐसे हैं जो अपने पेट के बल चलते हैं और बाज़ उनमें से ऐसे हैं जो दो पाँव पर चलते हैं और बाज़ उनमें से ऐसे हैं जो चार पावों पर चलते हैं- ख़ुदा जो चाहता है पैदा करता है इसमें शक नहीं कि खुदा हर चीज़ पर क़ादिर है
\end{hindi}}
\flushright{\begin{Arabic}
\quranayah[24][46]
\end{Arabic}}
\flushleft{\begin{hindi}
हम ही ने यक़ीनन वाजेए व रौशन आयतें नाज़िल की और खुदा ही जिसको चाहता है सीधी राह की हिदायत करता है
\end{hindi}}
\flushright{\begin{Arabic}
\quranayah[24][47]
\end{Arabic}}
\flushleft{\begin{hindi}
और (जो लोग ऐसे भी है जो) कहते हैं कि ख़ुदा पर और रसूल पर ईमान लाए और हमने इताअत क़ुबूल की- फिर उसके बाद उन में से कुछ लोग (ख़ुदा के हुक्म से) मुँह फेर लेते हैं और (सच यूँ है कि) ये लोग ईमानदार थे ही नहीं
\end{hindi}}
\flushright{\begin{Arabic}
\quranayah[24][48]
\end{Arabic}}
\flushleft{\begin{hindi}
और जब वह लोग ख़ुदा और उसके रसूल की तरफ बुलाए जाते हैं ताकि रसूल उनके आपस के झगड़े का फैसला कर दें तो उनमें का एक फरीक रदगिरदानी करता है
\end{hindi}}
\flushright{\begin{Arabic}
\quranayah[24][49]
\end{Arabic}}
\flushleft{\begin{hindi}
और (असल ये है कि) अगर हक़ उनकी तरफ होता तो गर्दन झुकाए (चुपके) रसूल के पास दौड़े हुए आते
\end{hindi}}
\flushright{\begin{Arabic}
\quranayah[24][50]
\end{Arabic}}
\flushleft{\begin{hindi}
क्या उन के दिल में (कुफ्र का) मर्ज़ (बाक़ी) है या शक में पड़े हैं या इस बात से डरते हैं कि (मुबादा) ख़ुदा और उसका रसूल उन पर ज़ुल्म कर बैठेगा- (ये सब कुछ नहीं) बल्कि यही लोग ज़ालिम हैं
\end{hindi}}
\flushright{\begin{Arabic}
\quranayah[24][51]
\end{Arabic}}
\flushleft{\begin{hindi}
ईमानदारों का क़ौल तो बस ये है कि जब उनको ख़ुदा और उसके रसूल के पास बुलाया जाता है ताकि उनके बाहमी झगड़ों का फैसला करो तो कहते हैं कि हमने (हुक्म) सुना और (दिल से) मान लिया और यही लोग (आख़िरत में) कामयाब होने वाले हैं
\end{hindi}}
\flushright{\begin{Arabic}
\quranayah[24][52]
\end{Arabic}}
\flushleft{\begin{hindi}
और जो शख्स ख़ुदा और उसके रसूल का हुक्म माने और ख़ुदा से डरे और उस (की नाफरमानी) से बचता रहेगा तो ऐसे ही लोग अपनी मुराद को पहुँचेगें
\end{hindi}}
\flushright{\begin{Arabic}
\quranayah[24][53]
\end{Arabic}}
\flushleft{\begin{hindi}
और (ऐ रसूल) उन (मुनाफेक़ीन) ने तुम्हारी इताअत की ख़ुदा की सख्त से सख्त क़समें खाई कि अगर तुम उन्हें हुक्म दो तो बिला उज़्र (घर बार छोड़कर) निकल खडे हों- तुम कह दो कि क़समें न खाओ दस्तूर के मुवाफिक़ इताअत (इससे बेहतर) और बेशक तुम जो कुछ करते हो ख़ुदा उससे ख़बरदार है
\end{hindi}}
\flushright{\begin{Arabic}
\quranayah[24][54]
\end{Arabic}}
\flushleft{\begin{hindi}
(ऐ रसूल) तुम कह दो कि ख़ुदा की इताअत करो और रसूल की इताअत करो इस पर भी अगर तुम सरताबी करोगे तो बस रसूल पर इतना ही (तबलीग़) वाजिब है जिसके वह ज़िम्मेदार किए गए हैं और जिसके ज़िम्मेदार तुम बनाए गए हो तुम पर वाजिब है और अगर तुम उसकी इताअत करोगे तो हिदायत पाओगे और रसूल पर तो सिर्फ साफ तौर पर (एहकाम का) पहुँचाना फर्ज है
\end{hindi}}
\flushright{\begin{Arabic}
\quranayah[24][55]
\end{Arabic}}
\flushleft{\begin{hindi}
(ऐ ईमानदारों) तुम में से जिन लोगों ने ईमान क़ुबूल किया और अच्छे अच्छे काम किए उन से ख़ुदा ने वायदा किया कि उन को (एक न एक) दिन रुए ज़मीन पर ज़रुर (अपना) नाएब मुक़र्रर करेगा जिस तरह उन लोगों को नाएब बनाया जो उनसे पहले गुज़र चुके हैं और जिस दीन को उसने उनके लिए पसन्द फरमाया है (इस्लाम) उस पर उन्हें ज़रुर ज़रुर पूरी क़ुदरत देगा और उनके ख़ाएफ़ होने के बाद (उनकी हर आस को) अमन से ज़रुर बदल देगा कि वह (इत्मेनान से) मेरी ही इबादत करेंगे और किसी को हमारा शरीक न बनाएँगे और जो शख्स इसके बाद भी नाशुक्री करे तो ऐसे ही लोग बदकार हैं
\end{hindi}}
\flushright{\begin{Arabic}
\quranayah[24][56]
\end{Arabic}}
\flushleft{\begin{hindi}
और (ऐ ईमानदारों) नमाज़ पाबन्दी से पढ़ा करो और ज़कात दिया करो और (दिल से) रसूल की इताअत करो ताकि तुम पर रहम किया जाए
\end{hindi}}
\flushright{\begin{Arabic}
\quranayah[24][57]
\end{Arabic}}
\flushleft{\begin{hindi}
और (ऐ रसूल) तुम ये ख्याल न करो कि कुफ्फार (इधर उधर) ज़मीन मे (फैल कर हमें) आजिज़ कर देगें (ये ख़ुद आजिज़ हो जाएगें) और उनका ठिकाना तो जहन्नुम है और क्या बुरा ठिकाना है
\end{hindi}}
\flushright{\begin{Arabic}
\quranayah[24][58]
\end{Arabic}}
\flushleft{\begin{hindi}
ऐ ईमानदारों तुम्हारी लौन्डी ग़ुलाम और वह लड़के जो अभी तक बुलूग़ की हद तक नहीं पहुँचे हैं उनको भी चाहिए कि (दिन रात में) तीन मरतबा (तुम्हारे पास आने की) तुमसे इजाज़त ले लिया करें तब आएँ (एक) नमाज़ सुबह से पहले और (दूसरे) जब तुम (गर्मी से) दोपहर को (सोने के लिए मामूलन) कपड़े उतार दिया करते हो (तीसरी) नमाजे इशा के बाद (ये) तीन (वक्त) तुम्हारे परदे के हैं इन अवक़ात के अलावा (बे अज़न आने मे) न तुम पर कोई इल्ज़ाम है-न उन पर (क्योंकि) उन अवक़ात के अलावा (ब ज़रुरत या बे ज़रुरत) लोग एक दूसरे के पास चक्कर लगाया करते हैं- यँ ख़ुदा (अपने) एहकाम तुम से साफ साफ बयान करता है और ख़ुदा तो बड़ा वाकिफ़कार हकीम है
\end{hindi}}
\flushright{\begin{Arabic}
\quranayah[24][59]
\end{Arabic}}
\flushleft{\begin{hindi}
और (ऐ ईमानदारों) जब तुम्हारे लड़के हदे बुलूग को पहुँचें तो जिस तरह उन के कब्ल (बड़ी उम्र) वाले (घर में आने की) इजाज़त ले लिया करते थे उसी तरह ये लोग भी इजाज़त ले लिया करें-यूँ ख़ुदा अपने एहकाम साफ साफ बयान करता है और ख़ुदा तो बड़ा वाकिफकार हकीम है
\end{hindi}}
\flushright{\begin{Arabic}
\quranayah[24][60]
\end{Arabic}}
\flushleft{\begin{hindi}
और बूढ़ी बूढ़ी औरतें जो (बुढ़ापे की वजह से) निकाह की ख्वाहिश नही रखती वह अगर अपने कपड़े (दुपट्टे वगैराह) उतारकर (सर नंगा) कर डालें तो उसमें उन पर कुछ गुनाह नही है- बशर्ते कि उनको अपना बनाव सिंगार दिखाना मंज़ूर न हो और (इस से भी) बचें तो उनके लिए और बेहतर है और ख़ुदा तो (सबकी सब कुछ) सुनता और जानता है
\end{hindi}}
\flushright{\begin{Arabic}
\quranayah[24][61]
\end{Arabic}}
\flushleft{\begin{hindi}
इस बात में न तो अंधे आदमी के लिए मज़ाएक़ा है और न लँगड़ें आदमी पर कुछ इल्ज़ाम है- और न बीमार पर कोई गुनाह है और न ख़ुद तुम लोगो पर कि अपने घरों से खाना खाओ या अपने बाप दादा नाना बग़ैरह के घरों से या अपनी माँ दादी नानी वगैरह के घरों से या अपने भाइयों के घरों से या अपनी बहनों के घरों से या अपने चचाओं के घरों से या अपनी फूफ़ियों के घरों से या अपने मामूओं के घरों से या अपनी खालाओं के घरों से या उस घर से जिसकी कुन्जियाँ तुम्हारे हाथ में है या अपने दोस्तों (के घरों) से इस में भी तुम पर कोई इल्ज़ाम नहीं कि सब के सब मिलकर खाओ या अलग अलग फिर जब तुम घर वालों में जाने लगो (और वहाँ किसी का न पाओ) तो ख़ुद अपने ही ऊपर सलाम कर लिया करो जो ख़ुदा की तरफ से एक मुबारक पाक व पाकीज़ा तोहफा है- ख़ुदा यूँ (अपने) एहकाम तुमसे साफ साफ बयान करता है ताकि तुम समझो
\end{hindi}}
\flushright{\begin{Arabic}
\quranayah[24][62]
\end{Arabic}}
\flushleft{\begin{hindi}
सच्चे ईमानदार तो सिर्फ वह लोग हैं जो ख़ुदा और उसके रसूल पर ईमान लाए और जब किसी ऐसे काम के लिए जिसमें लोंगों के जमा होने की ज़रुरत है- रसूल के पास होते हैं जब तक उससे इजाज़त न ले ली न गए (ऐ रसूल) जो लोग तुम से (हर बात में) इजाज़त ले लेते हैं वे ही लोग (दिल से) ख़ुदा और उसके रसूल पर ईमान लाए हैं तो जब ये लोग अपने किसी काम के लिए तुम से इजाज़त माँगें तो तुम उनमें से जिसको (मुनासिब ख्याल करके) चाहो इजाज़त दे दिया करो और खुदा उसे उसकी बख़्शिस की दुआ भी करो बेशक खुदा बड़ा बख्शने वाला मेहरबान है
\end{hindi}}
\flushright{\begin{Arabic}
\quranayah[24][63]
\end{Arabic}}
\flushleft{\begin{hindi}
(ऐ ईमानदारों) जिस तरह तुम में से एक दूसरे को (नाम ले कर) बुलाया करते हैं उस तरह आपस में रसूल का बुलाना न समझो ख़ुदा उन लोगों को खूब जानता है जो तुम में से ऑंख बचा के (पैग़म्बर के पास से) खिसक जाते हैं- तो जो लोग उसके हुक्म की मुख़ालफत करते हैं उनको इस बात से डरते रहना चाहिए कि (मुबादा) उन पर कोई मुसीबत आ पडे या उन पर कोई दर्दनाक अज़ाब नाज़िल हो
\end{hindi}}
\flushright{\begin{Arabic}
\quranayah[24][64]
\end{Arabic}}
\flushleft{\begin{hindi}
ख़बरदार जो कुछ सारे आसमान व ज़मीन में है (सब) यक़ीनन ख़ुदा ही का है जिस हालत पर तुम हो ख़ुदा ख़ूब जानता है और जिस दिन उसके पास ये लोग लौटा कर लाएँ जाएँगें तो जो कुछ उन लोगों ने किया कराया है बता देगा और ख़ुदा तो हर चीज़ से खूब वाकिफ है
\end{hindi}}
\chapter{Al-Furqan (The Discrimination)}
\begin{Arabic}
\Huge{\centerline{\basmalah}}\end{Arabic}
\flushright{\begin{Arabic}
\quranayah[25][1]
\end{Arabic}}
\flushleft{\begin{hindi}
(ख़ुदा) बहुत बाबरकत है जिसने अपने बन्दे (मोहम्मद) पर कुरान नाज़िल किया ताकि सारे जहॉन के लिए (ख़ुदा के अज़ाब से) डराने वाला हो
\end{hindi}}
\flushright{\begin{Arabic}
\quranayah[25][2]
\end{Arabic}}
\flushleft{\begin{hindi}
वह खुदा कि सारे आसमान व ज़मीन की बादशाहत उसी की है और उसने (किसी को) न अपना लड़का बनाया और न सल्तनत में उसका कोई शरीक है और हर चीज़ को (उसी ने पैदा किया) फिर उस अन्दाज़े से दुरुस्त किया
\end{hindi}}
\flushright{\begin{Arabic}
\quranayah[25][3]
\end{Arabic}}
\flushleft{\begin{hindi}
और लोगों ने उसके सिवा दूसरे दूसरे माबूद बना रखें हैं जो कुछ भी पैदा नहीं कर सकते बल्कि वह खुद दूसरे के पैदा किए हुए हैं और वह खुद अपने लिए भी न नुक़सान पर क़ाबू रखते हैं न नफा पर और न मौत ही पर एख़्तियार रखते हैं और न ज़िन्दगी पर और न मरने बाद जी उठने पर
\end{hindi}}
\flushright{\begin{Arabic}
\quranayah[25][4]
\end{Arabic}}
\flushleft{\begin{hindi}
और जो लोग काफिर हो गए बोल उठे कि ये (क़ुरान) तो निरा झूठ है जिसे उस शख्स (रसूल) ने अपने जी से गढ़ लिया और कुछ और लोगों ने इस इफतिरा परवाज़ी में उसकी मदद भी की है
\end{hindi}}
\flushright{\begin{Arabic}
\quranayah[25][5]
\end{Arabic}}
\flushleft{\begin{hindi}
तो यक़ीनन ख़ुद उन ही लोगों ने ज़ुल्म व फरेब किया है और (ये भी) कहा कि (ये तो) अगले लोगों के ढकोसले हैं जिसे उसने किसी से लिखवा लिया है पस वही सुबह व शाम उसके सामने पढ़ा जाता है
\end{hindi}}
\flushright{\begin{Arabic}
\quranayah[25][6]
\end{Arabic}}
\flushleft{\begin{hindi}
(ऐ रसूल) तुम कह दो कि इसको उस शख्स ने नाज़िल किया है जो सारे आसमान व ज़मीन की पोशीदा बातों को ख़ूब जानता है बेशक वह बड़ा बख्शने वाला मेहरबान है
\end{hindi}}
\flushright{\begin{Arabic}
\quranayah[25][7]
\end{Arabic}}
\flushleft{\begin{hindi}
और उन लोगों ने (ये भी) कहा कि ये कैसा रसूल है जो खाना खाता है और बाज़ारों में चलता है फिरता है उसके पास कोई फरिश्ता क्यों नहीं नाज़िल होता कि वह भी उसके साथ (ख़ुदा के अज़ाब से) डराने वाला होता
\end{hindi}}
\flushright{\begin{Arabic}
\quranayah[25][8]
\end{Arabic}}
\flushleft{\begin{hindi}
(कम से कम) इसके पास ख़ज़ाना ही ख़ज़ाना ही (आसमान से) गिरा दिया जाता या (और नहीं तो) उसके पास कोई बाग़ ही होता कि उससे खाता (पीता) और ये ज़ालिम (कुफ्फ़ार मोमिनों से) कहते हैं कि तुम लोग तो बस ऐसे आदमी की पैरवी करते हो जिस पर जादू कर दिया गया है
\end{hindi}}
\flushright{\begin{Arabic}
\quranayah[25][9]
\end{Arabic}}
\flushleft{\begin{hindi}
(ऐ रसूल) ज़रा देखो तो कि इन लोगों ने तुम्हारे वास्ते कैसी कैसी फबत्तियां गढ़ी हैं और गुमराह हो गए तो अब ये लोग किसी तरह राह पर आ ही नहीं सकते
\end{hindi}}
\flushright{\begin{Arabic}
\quranayah[25][10]
\end{Arabic}}
\flushleft{\begin{hindi}
ख़ुदा तो ऐसा बारबरकत है कि अगर चाहे तो (एक बाग़ क्या चीज़ है) इससे बेहतर बहुतेरे ऐसे बाग़ात तुम्हारे वास्ते पैदा करे जिन के नीचे नहरें जारी हों और (बाग़ात के अलावा उनमें) तुम्हारे वास्ते महल बना दे
\end{hindi}}
\flushright{\begin{Arabic}
\quranayah[25][11]
\end{Arabic}}
\flushleft{\begin{hindi}
(ये सब कुछ नहीं) बल्कि (सच यूँ है कि) उन लोगों ने क़यामत ही को झूठ समझा है और जिस शख्स ने क़यामत को झूठ समझा उसके लिए हमने जहन्नुम को (दहका के) तैयार कर रखा है
\end{hindi}}
\flushright{\begin{Arabic}
\quranayah[25][12]
\end{Arabic}}
\flushleft{\begin{hindi}
जब जहन्नुम इन लोगों को दूर से दखेगी तो (जोश खाएगी और) ये लोग उसके जोश व ख़रोश की आवाज़ सुनेंगें
\end{hindi}}
\flushright{\begin{Arabic}
\quranayah[25][13]
\end{Arabic}}
\flushleft{\begin{hindi}
और जब ये लोग ज़ज़ीरों से जकड़कर उसकी किसी तंग जगह मे झोंक दिए जाएँगे तो उस वक्त मौत को पुकारेंगे
\end{hindi}}
\flushright{\begin{Arabic}
\quranayah[25][14]
\end{Arabic}}
\flushleft{\begin{hindi}
(उस वक्त उनसे कहा जाएगा कि) आज एक मौत को न पुकारो बल्कि बहुतेरी मौतों को पुकारो (मगर इससे भी कुछ होने वाला नहीं)
\end{hindi}}
\flushright{\begin{Arabic}
\quranayah[25][15]
\end{Arabic}}
\flushleft{\begin{hindi}
(ऐ रसूल) तुम पूछो तो कि जहन्नुम बेहतर है या हमेशा रहने का बाग़ (बेहश्त) जिसका परहेज़गारों से वायदा किया गया है कि वह उन (के आमाल) का सिला होगा और आख़िरी ठिकाना
\end{hindi}}
\flushright{\begin{Arabic}
\quranayah[25][16]
\end{Arabic}}
\flushleft{\begin{hindi}
जिस चीज़ की ख्वाहिश करेंगें उनके लिए वहाँ मौजूद होगी (और) वह हमेशा (उसी हाल में) रहेंगें ये तुम्हारे परवरदिगार पर एक लाज़िमी और माँगा हुआ वायदा है
\end{hindi}}
\flushright{\begin{Arabic}
\quranayah[25][17]
\end{Arabic}}
\flushleft{\begin{hindi}
और जिस दिन ख़ुदा उन लोगों को और जिनकी ये लोग ख़ुदा को छोड़कर परसतिश किया करते हैं (उनको) जमा करेगा और पूछेगा क्या तुम ही ने हमारे उन बन्दों को गुमराह कर दिया था या ये लोग खुद राह रास्ते से भटक गए थे
\end{hindi}}
\flushright{\begin{Arabic}
\quranayah[25][18]
\end{Arabic}}
\flushleft{\begin{hindi}
(उनके माबूद) अर्ज़ करेंगें- सुबहान अल्लाह (हम तो ख़ुद तेरे बन्दे थे) हमें ये किसी तरह ज़ेबा न था कि हम तुझे छोड़कर दूसरे को अपना सरपरस्त बनाते (फिर अपने को क्यों कर माबूद बनाते) मगर बात तो ये है कि तू ही ने इनको बाप दादाओं को चैन दिया-यहाँ तक कि इन लोगों ने (तेरी) याद भुला दी और ये ख़ुद हलाक होने वाले लोग थे
\end{hindi}}
\flushright{\begin{Arabic}
\quranayah[25][19]
\end{Arabic}}
\flushleft{\begin{hindi}
तब (काफ़िरों से कहा जाएगा कि) तुम जो कुछ कह रहे हो उसमें तो तुम्हारे माबूदों ने तुम्हें झूठला दिया तो अब तुम न (हमारे अज़ाब के) टाल देने की सकत रखते हो न किसी से मदद ले सकते हो और (याद रखो) तुममें से जो ज़ुल्म करेगा हम उसको बड़े (सख्त) अज़ाब का (मज़ा) चखाएगें
\end{hindi}}
\flushright{\begin{Arabic}
\quranayah[25][20]
\end{Arabic}}
\flushleft{\begin{hindi}
और (ऐ रसूल) हमने तुम से पहले जितने पैग़म्बर भेजे वह सब के सब यक़ीनन बिला शक खाना खाते थे और बाज़ारों में चलते फिरते थे और हमने तुम में से एक को एक का (ज़रिया) आज़माइश बना दिया (मुसलमानों) क्या तुम अब भी सब्र करते हो (या नहीं) और तुम्हारा परवरदिगार (सब की) देख भाल कर रहा है
\end{hindi}}
\flushright{\begin{Arabic}
\quranayah[25][21]
\end{Arabic}}
\flushleft{\begin{hindi}
और जो लोग (क़यामत में) हमारी हुज़ूरी की उम्मीद नहीं रखते कहा करते हैं कि आख़िर फरिश्ते हमारे पास क्यों नहीं नाज़िल किए गए या हम अपने परवरदिगार को (क्यों नहीं) देखते उन लोगों ने अपने जी में अपने को (बहुत) बड़ा समझ लिया है और बड़ी सरकशी की
\end{hindi}}
\flushright{\begin{Arabic}
\quranayah[25][22]
\end{Arabic}}
\flushleft{\begin{hindi}
जिस दिन ये लोग फरिश्तों को देखेंगे उस दिन गुनाह गारों को कुछ खुशी न होगी और फरिश्तों को देखकर कहेंगे दूर दफान
\end{hindi}}
\flushright{\begin{Arabic}
\quranayah[25][23]
\end{Arabic}}
\flushleft{\begin{hindi}
और उन लोगों ने (दुनिया में) जो कुछ नेक काम किए हैं हम उसकी तरफ तवज्जों करेंगें तो हम उसको (गोया) उड़ती हुई ख़ाक बनाकर (बरबाद कर) देगें
\end{hindi}}
\flushright{\begin{Arabic}
\quranayah[25][24]
\end{Arabic}}
\flushleft{\begin{hindi}
उस दिन जन्नत वालों का ठिकाना भी बेहतर है बेहतर होगा और आरमगाह भी अच्छी से अच्छी
\end{hindi}}
\flushright{\begin{Arabic}
\quranayah[25][25]
\end{Arabic}}
\flushleft{\begin{hindi}
और जिस दिन आसमान बदली के सबब से फट जाएगा और फरिश्ते कसरत से (जूक दर ज़ूक) नाज़िल किए जाएँगे
\end{hindi}}
\flushright{\begin{Arabic}
\quranayah[25][26]
\end{Arabic}}
\flushleft{\begin{hindi}
उसे दिन की सल्तनल ख़ास ख़ुदा ही के लिए होगी और वह दिन काफिरों पर बड़ा सख्त होगा
\end{hindi}}
\flushright{\begin{Arabic}
\quranayah[25][27]
\end{Arabic}}
\flushleft{\begin{hindi}
और जिस दिन जुल्म करने वाला अपने हाथ (मारे अफ़सोस के) काटने लगेगा और कहेगा काश रसूल के साथ मैं भी (दीन का सीधा) रास्ता पकड़ता
\end{hindi}}
\flushright{\begin{Arabic}
\quranayah[25][28]
\end{Arabic}}
\flushleft{\begin{hindi}
हाए अफसोस काश मै फला शख्स को अपना दोस्त न बनाता
\end{hindi}}
\flushright{\begin{Arabic}
\quranayah[25][29]
\end{Arabic}}
\flushleft{\begin{hindi}
बेशक यक़ीनन उसने हमारे पास नसीहत आने के बाद मुझे बहकाया और शैतान तो आदमी को रुसवा करने वाला ही है
\end{hindi}}
\flushright{\begin{Arabic}
\quranayah[25][30]
\end{Arabic}}
\flushleft{\begin{hindi}
और (उस वक्त) रसूल (बारगाहे ख़ुदा वन्दी में) अर्ज़ करेगें कि ऐ मेरे परवरदिगार मेरी क़ौम ने तो इस क़ुरान को बेकार बना दिया
\end{hindi}}
\flushright{\begin{Arabic}
\quranayah[25][31]
\end{Arabic}}
\flushleft{\begin{hindi}
और हमने (गोया ख़ुद) गुनाहगारों में से हर नबी के दुश्मन बना दिए हैं और तुम्हारा परवरदिगार हिदायत और मददगारी के लिए काफी है
\end{hindi}}
\flushright{\begin{Arabic}
\quranayah[25][32]
\end{Arabic}}
\flushleft{\begin{hindi}
और कुफ्फार कहने लगे कि उनके ऊपर (आख़िर) क़ुरान का कुल (एक ही दफा) क्यों नहीं नाज़िल किया गया (हमने) इस तरह इसलिए (नाज़िल किया) ताकि तुम्हारे दिल को तस्कीन देते रहें और हमने इसको ठहर ठहर कर नाज़िल किया
\end{hindi}}
\flushright{\begin{Arabic}
\quranayah[25][33]
\end{Arabic}}
\flushleft{\begin{hindi}
और (ये कुफ्फार) चाहे कैसी ही (अनोखी) मसल बयान करेंगे मगर हम तुम्हारे पास (उनका) बिल्कुल ठीक और निहायत उम्दा (जवाब) बयान कर देगें
\end{hindi}}
\flushright{\begin{Arabic}
\quranayah[25][34]
\end{Arabic}}
\flushleft{\begin{hindi}
जो लोग (क़यामत के दिन) अपने अपने मोहसिनों के बल जहन्नुम में हकाए जाएगें वही लोग बदतर जगह में होगें और सब से ज्यादा राह रास्त से भटकने वाले
\end{hindi}}
\flushright{\begin{Arabic}
\quranayah[25][35]
\end{Arabic}}
\flushleft{\begin{hindi}
और अलबत्ता हमने मूसा को किताब (तौरैत) अता की और उनके साथ उनके भाई हारुन को (उनका) वज़ीर बनाया
\end{hindi}}
\flushright{\begin{Arabic}
\quranayah[25][36]
\end{Arabic}}
\flushleft{\begin{hindi}
तो हमने कहा तुम दोनों उन लोगों के पास जा जो हमारी (कुदरत की) निशानियों को झुठलाते हैं जाओ (और समझाओ जब न माने) तो हमने उन्हें खूब बरबाद कर डाला
\end{hindi}}
\flushright{\begin{Arabic}
\quranayah[25][37]
\end{Arabic}}
\flushleft{\begin{hindi}
और नूह की क़ौम को जब उन लोगों ने (हमारे) पैग़म्बरों को झुठलाया तो हमने उन्हें डुबो दिया और हमने उनको लोगों (के हैरत) की निशानी बनाया और हमने ज़ालिमों के वास्ते दर्दनाक अज़ाब तैयार कर रखा है
\end{hindi}}
\flushright{\begin{Arabic}
\quranayah[25][38]
\end{Arabic}}
\flushleft{\begin{hindi}
और (इसी तरह) आद और समूद और नहर वालों और उनके दरमियान में बहुत सी जमाअतों को (हमने हलाक कर डाला)
\end{hindi}}
\flushright{\begin{Arabic}
\quranayah[25][39]
\end{Arabic}}
\flushleft{\begin{hindi}
और हमने हर एक से मिसालें बयान कर दी थीं और (खूब समझाया) मगर न माना
\end{hindi}}
\flushright{\begin{Arabic}
\quranayah[25][40]
\end{Arabic}}
\flushleft{\begin{hindi}
हमने उनको ख़ूब सत्यानास कर छोड़ा और ये लोग (कुफ्फ़ारे मक्का) उस बस्ती पर (हो) आए हैं जिस पर (पत्थरों की) बुरी बारिश बरसाई गयी तो क्या उन लोगों ने इसको देखा न होगा मगर (बात ये है कि) ये लोग मरने के बाद जी उठने की उम्मीद नहीं रखते (फिर क्यों ईमान लाएँ)
\end{hindi}}
\flushright{\begin{Arabic}
\quranayah[25][41]
\end{Arabic}}
\flushleft{\begin{hindi}
और (ऐ रसूल) ये लोग तुम्हें जब देखते हैं तो तुम से मसख़रा पन ही करने लगते हैं कि क्या यही वह (हज़रत) हैं जिन्हें अल्लाह ने रसूल बनाकर भेजा है (माज़ अल्लाह)
\end{hindi}}
\flushright{\begin{Arabic}
\quranayah[25][42]
\end{Arabic}}
\flushleft{\begin{hindi}
अगर बुतों की परसतिश पर साबित क़दम न रहते तो इस शख्स ने हमको हमारे माबूदों से बहका दिया था और बहुत जल्द (क़यामत में) जब ये लोग अज़ाब को देखेंगें तो उन्हें मालूम हो जाएगा कि राहे रास्त से कौन ज्यादा भटका हुआ था
\end{hindi}}
\flushright{\begin{Arabic}
\quranayah[25][43]
\end{Arabic}}
\flushleft{\begin{hindi}
क्या तुमने उस शख्स को भी देखा है जिसने अपनी नफ़सियानी ख्वाहिश को अपना माबूद बना रखा है तो क्या तुम उसके ज़िम्मेदार हो सकते हो (कि वह गुमराह न हों)
\end{hindi}}
\flushright{\begin{Arabic}
\quranayah[25][44]
\end{Arabic}}
\flushleft{\begin{hindi}
क्या ये तुम्हारा ख्याल है कि इन (कुफ्फ़ारों) में अक्सर (बात) सुनते या समझते है (नहीं) ये तो बस बिल्कुल मिसल जानवरों के हैं बल्कि उन से भी ज्यादा राह (रास्त) से भटके हुए
\end{hindi}}
\flushright{\begin{Arabic}
\quranayah[25][45]
\end{Arabic}}
\flushleft{\begin{hindi}
(ऐ रसूल) क्या तुमने अपने परवरदिगार की कुदरत की तरफ नज़र नहीं की कि उसने क्योंकर साये को फैला दिया अगर वह चहता तो उसे (एक ही जगह) ठहरा हुआ कर देता फिर हमने आफताब को (उसकी शिनाख्त के वास्ते) उसका रहनुमा बना दिया
\end{hindi}}
\flushright{\begin{Arabic}
\quranayah[25][46]
\end{Arabic}}
\flushleft{\begin{hindi}
फिर हमने उसको थोड़ा थोड़ा करके अपनी तरफ खीच लिया
\end{hindi}}
\flushright{\begin{Arabic}
\quranayah[25][47]
\end{Arabic}}
\flushleft{\begin{hindi}
और वही तो वह (ख़ुदा) है जिसने तुम्हारे वास्ते रात को पर्दा बनाया और नींद को राहत और दिन को (कारोबार के लिए) उठ खड़ा होने का वक्त बनाया
\end{hindi}}
\flushright{\begin{Arabic}
\quranayah[25][48]
\end{Arabic}}
\flushleft{\begin{hindi}
और वही तो वह (ख़ुदा) है जिसने अपनी रहमत (बारिश) के आगे आगे हवाओं को खुश ख़बरी देने के लिए (पेश ख़ेमा बना के) भेजा और हम ही ने आसमान से बहुत पाक और सुथरा हुआ पानी बरसाया
\end{hindi}}
\flushright{\begin{Arabic}
\quranayah[25][49]
\end{Arabic}}
\flushleft{\begin{hindi}
ताकि हम उसके ज़रिए से मुर्दा (वीरान) शहर को ज़िन्दा (आबाद) कर दें और अपनी मख़लूकात में से चौपायों और बहुत से आदमियों को उससे सेराब करें
\end{hindi}}
\flushright{\begin{Arabic}
\quranayah[25][50]
\end{Arabic}}
\flushleft{\begin{hindi}
और हमने पानी को उनके दरमियान (तरह तरह से) तक़सीम किया ताकि लोग नसीहत हासिल करें मगर अक्सर लोगों ने नाशुक्री के सिवा कुछ न माना
\end{hindi}}
\flushright{\begin{Arabic}
\quranayah[25][51]
\end{Arabic}}
\flushleft{\begin{hindi}
और अगर हम चाहते तो हर बस्ती में ज़रुर एक (अज़ाबे ख़ुदा से) डराने वाला पैग़म्बर भेजते
\end{hindi}}
\flushright{\begin{Arabic}
\quranayah[25][52]
\end{Arabic}}
\flushleft{\begin{hindi}
(तो ऐ रसूल) तुम काफिरों की इताअत न करना और उनसे कुरान के (दलाएल) से खूब लड़ों
\end{hindi}}
\flushright{\begin{Arabic}
\quranayah[25][53]
\end{Arabic}}
\flushleft{\begin{hindi}
और वही तो वह (ख़ुदा) है जिसने दरयाओं को आपस में मिला दिया (और बावजूद कि) ये खालिस मज़ेदार मीठा है और ये बिल्कुल खारी कड़वा (मगर दोनों को मिलाया) और दोनों के दरमियान एक आड़ और मज़बूत ओट बना दी है (कि गड़बड़ न हो)
\end{hindi}}
\flushright{\begin{Arabic}
\quranayah[25][54]
\end{Arabic}}
\flushleft{\begin{hindi}
और वही तो वह (ख़ुदा) है जिसने पानी (मनी) से आदमी को पैदा किया फिर उसको ख़ानदान और सुसराल वाला बनाया और (ऐ रसूल) तुम्हारा परवरदिगार हर चीज़ पर क़ादिर है
\end{hindi}}
\flushright{\begin{Arabic}
\quranayah[25][55]
\end{Arabic}}
\flushleft{\begin{hindi}
और लोग (कुफ्फ़ारे मक्का) ख़ुदा को छोड़कर उस चीज़ की परसतिश करते हैं जो न उन्हें नफा ही दे सकती है और न नुक़सान ही पहुँचा सकती है और काफिर (अबूजहल) तो हर वक्त अपने परवरदिगार की मुख़ालेफत पर ज़ोर लगाए हुए है
\end{hindi}}
\flushright{\begin{Arabic}
\quranayah[25][56]
\end{Arabic}}
\flushleft{\begin{hindi}
और (ऐ रसूल) हमने तो तुमको बस (नेकी को जन्नत की) खुशख़बरी देने वाला और (बुरों को अज़ाब से) डराने वाला बनाकर भेजा है
\end{hindi}}
\flushright{\begin{Arabic}
\quranayah[25][57]
\end{Arabic}}
\flushleft{\begin{hindi}
और उन लोगों से तुम कह दो कि मै इस (तबलीगे रिसालत) पर तुमसे कुछ मज़दूरी तो माँगता नहीं हूँ मगर तमन्ना ये है कि जो चाहे अपने परवरदिगार तक पहुँचने की राह पकडे
\end{hindi}}
\flushright{\begin{Arabic}
\quranayah[25][58]
\end{Arabic}}
\flushleft{\begin{hindi}
और (ऐ रसूल) तुम उस (ख़ुदा) पर भरोसा रखो जो ऐसा ज़िन्दा है कि कभी नहीं मरेगा और उसकी हम्द व सना की तस्बीह पढ़ो और वह अपने बन्दों के गुनाहों की वाक़िफ कारी में काफी है (वह ख़ुद समझ लेगा)
\end{hindi}}
\flushright{\begin{Arabic}
\quranayah[25][59]
\end{Arabic}}
\flushleft{\begin{hindi}
जिसने सारे आसमान व ज़मीन और जो कुछ उन दोनों में है छह: दिन में पैदा किया फिर अर्श (के बनाने) पर आमादा हुआ और वह बड़ा मेहरबान है तो तुम उसका हाल किसी बाख़बर ही से पूछना
\end{hindi}}
\flushright{\begin{Arabic}
\quranayah[25][60]
\end{Arabic}}
\flushleft{\begin{hindi}
और जब उन कुफ्फारों से कहा जाता है कि रहमान (ख़ुदा) को सजदा करो तो कहते हैं कि रहमान क्या चीज़ है तुम जिसके लिए कहते हो हम उस का सजदा करने लगें और (इससे) उनकी नफरत और बढ़ जाती है (60) सजदा
\end{hindi}}
\flushright{\begin{Arabic}
\quranayah[25][61]
\end{Arabic}}
\flushleft{\begin{hindi}
बहुत बाबरकत है वह ख़ुदा जिसने आसमान में बुर्ज बनाए और उन बुर्जों में (आफ़ताब का) चिराग़ और जगमगाता चाँद बनाया
\end{hindi}}
\flushright{\begin{Arabic}
\quranayah[25][62]
\end{Arabic}}
\flushleft{\begin{hindi}
और वही तो वह (ख़ुदा) है जिसने रात और दिन (एक) को (एक का) जानशीन बनाया (ये) उस के (समझने के) लिए है जो नसीहत हासिल करना चाहे या शुक्र गुज़ारी का इरादा करें
\end{hindi}}
\flushright{\begin{Arabic}
\quranayah[25][63]
\end{Arabic}}
\flushleft{\begin{hindi}
और (ख़ुदाए) रहमान के ख़ास बन्दे तो वह हैं जो ज़मीन पर फिरौतनी के साथ चलते हैं और जब जाहिल उनसे (जिहालत) की बात करते हैं तो कहते हैं कि सलाम (तुम सलामत रहो)
\end{hindi}}
\flushright{\begin{Arabic}
\quranayah[25][64]
\end{Arabic}}
\flushleft{\begin{hindi}
और वह लोग जो अपने परवरदिगार के वास्ते सज़दे और क़याम में रात काट देते हैं
\end{hindi}}
\flushright{\begin{Arabic}
\quranayah[25][65]
\end{Arabic}}
\flushleft{\begin{hindi}
और वह लोग जो दुआ करते हैं कि परवरदिगारा हम से जहन्नुम का अज़ाब फेरे रहना क्योंकि उसका अज़ाब बहुत (सख्त और पाएदार होगा)
\end{hindi}}
\flushright{\begin{Arabic}
\quranayah[25][66]
\end{Arabic}}
\flushleft{\begin{hindi}
बेशक वह बहुत बुरा ठिकाना और बुरा मक़ाम है
\end{hindi}}
\flushright{\begin{Arabic}
\quranayah[25][67]
\end{Arabic}}
\flushleft{\begin{hindi}
और वह लोग कि जब खर्च करते हैं तो न फुज़ूल ख़र्ची करते हैं और न तंगी करते हैं और उनका ख़र्च उसके दरमेयान औसत दर्जे का रहता है
\end{hindi}}
\flushright{\begin{Arabic}
\quranayah[25][68]
\end{Arabic}}
\flushleft{\begin{hindi}
और वह लोग जो ख़ुदा के साथ दूसरे माबूदों की परसतिश नही करते और जिस जान के मारने को ख़ुदा ने हराम कर दिया है उसे नाहक़ क़त्ल नहीं करते और न ज़िना करते हैं और जो शख्स ऐसा करेगा वह आप अपने गुनाह की सज़ा भुगतेगा
\end{hindi}}
\flushright{\begin{Arabic}
\quranayah[25][69]
\end{Arabic}}
\flushleft{\begin{hindi}
कि क़यामत के दिन उसके लिए अज़ाब दूना कर दिया जाएगा और उसमें हमेशा ज़लील व ख़वार रहेगा
\end{hindi}}
\flushright{\begin{Arabic}
\quranayah[25][70]
\end{Arabic}}
\flushleft{\begin{hindi}
मगर (हाँ) जिस शख्स ने तौबा की और ईमान क़ुबूल किया और अच्छे अच्छे काम किए तो (अलबत्ता) उन लोगों की बुराइयों को ख़ुदा नेकियों से बदल देगा और ख़ुदा तो बड़ा बख्शने वाला मेहरबान है
\end{hindi}}
\flushright{\begin{Arabic}
\quranayah[25][71]
\end{Arabic}}
\flushleft{\begin{hindi}
और जिस शख्स ने तौबा कर ली और अच्छे अच्छे काम किए तो बेशक उसने ख़ुदा की तरफ (सच्चे दिल से) हक़ीकक़तन रुजु की
\end{hindi}}
\flushright{\begin{Arabic}
\quranayah[25][72]
\end{Arabic}}
\flushleft{\begin{hindi}
और वह लोग जो फरेब के पास ही नही खड़े होते और वह लोग जब किसी बेहूदा काम के पास से गुज़रते हैं तो बुर्ज़ुगाना अन्दाज़ से गुज़र जाते हैं
\end{hindi}}
\flushright{\begin{Arabic}
\quranayah[25][73]
\end{Arabic}}
\flushleft{\begin{hindi}
और वह लोग कि जब उन्हें उनके परवरदिगार की आयतें याद दिलाई जाती हैं तो बहरे अन्धें होकर गिर नहीं पड़ते बल्कि जी लगाकर सुनते हैं
\end{hindi}}
\flushright{\begin{Arabic}
\quranayah[25][74]
\end{Arabic}}
\flushleft{\begin{hindi}
और वह लोग जो (हमसे) अर्ज़ करते हैं कि परवरदिगार हमें हमारी बीबियों और औलादों की तरफ से ऑंखों की ठन्डक अता फरमा और हमको परहेज़गारों का पेशवा बना
\end{hindi}}
\flushright{\begin{Arabic}
\quranayah[25][75]
\end{Arabic}}
\flushleft{\begin{hindi}
ये वह लोग हैं जिन्हें उनकी जज़ा में (बेहश्त के) बाला ख़ाने अता किए जाएँगें और वहाँ उन्हें ताज़ीम व सलाम (का बदला) पेश किया जाएगा
\end{hindi}}
\flushright{\begin{Arabic}
\quranayah[25][76]
\end{Arabic}}
\flushleft{\begin{hindi}
ये लोग उसी में हमेशा रहेंगें और वह रहने और ठहरने की अच्छी जगह है
\end{hindi}}
\flushright{\begin{Arabic}
\quranayah[25][77]
\end{Arabic}}
\flushleft{\begin{hindi}
(ऐ रसूल) तुम कह दो कि अगर दुआ नही किया करते तो मेरा परवरदिगार भी तुम्हारी कुछ परवाह नही करता तुमने तो (उसके रसूल को) झुठलाया तो अन क़रीब ही (उसका वबाल) तुम्हारे सर पडेग़ा
\end{hindi}}
\chapter{Ash-Shu'ara' (The Poets)}
\begin{Arabic}
\Huge{\centerline{\basmalah}}\end{Arabic}
\flushright{\begin{Arabic}
\quranayah[26][1]
\end{Arabic}}
\flushleft{\begin{hindi}
ता सीन मीम
\end{hindi}}
\flushright{\begin{Arabic}
\quranayah[26][2]
\end{Arabic}}
\flushleft{\begin{hindi}
ये वाज़ेए व रौशन किताब की आयतें है
\end{hindi}}
\flushright{\begin{Arabic}
\quranayah[26][3]
\end{Arabic}}
\flushleft{\begin{hindi}
(ऐ रसूल) शायद तुम (इस फिक्र में) अपनी जान हलाक कर डालोगे कि ये (कुफ्फार) मोमिन क्यो नहीं हो जाते
\end{hindi}}
\flushright{\begin{Arabic}
\quranayah[26][4]
\end{Arabic}}
\flushleft{\begin{hindi}
अगर हम चाहें तो उन लोगों पर आसमान से कोई ऐसा मौजिज़ा नाज़िल करें कि उन लोगों की गर्दनें उसके सामने झुक जाएँ
\end{hindi}}
\flushright{\begin{Arabic}
\quranayah[26][5]
\end{Arabic}}
\flushleft{\begin{hindi}
और (लोगों का क़ायदा है कि) जब उनके पास कोई कोई नसीहत की बात ख़ुदा की तरफ से आयी तो ये लोग उससे मुँह फेरे बगैर नहीं रहे
\end{hindi}}
\flushright{\begin{Arabic}
\quranayah[26][6]
\end{Arabic}}
\flushleft{\begin{hindi}
उन लोगों ने झुठलाया ज़रुर तो अनक़रीब ही (उन्हें) इस (अज़ाब) की हक़ीकत मालूम हो जाएगी जिसकी ये लोग हँसी उड़ाया करते थे
\end{hindi}}
\flushright{\begin{Arabic}
\quranayah[26][7]
\end{Arabic}}
\flushleft{\begin{hindi}
क्या इन लोगों ने ज़मीन की तरफ भी (ग़ौर से) नहीं देखा कि हमने हर रंग की उम्दा उम्दा चीजें उसमें किस कसरत से उगायी हैं
\end{hindi}}
\flushright{\begin{Arabic}
\quranayah[26][8]
\end{Arabic}}
\flushleft{\begin{hindi}
यक़ीनन इसमें (भी क़ुदरत) ख़ुदा की एक बड़ी निशानी है मगर उनमें से अक्सर ईमान लाने वाले ही नहीं
\end{hindi}}
\flushright{\begin{Arabic}
\quranayah[26][9]
\end{Arabic}}
\flushleft{\begin{hindi}
और इसमें शक नहीं कि तेरा परवरदिगार यक़ीनन (हर चीज़ पर) ग़ालिब (और) मेहरबान है
\end{hindi}}
\flushright{\begin{Arabic}
\quranayah[26][10]
\end{Arabic}}
\flushleft{\begin{hindi}
(ऐ रसूल वह वक्त याद करो) जब तुम्हारे परवरदिगार ने मूसा को आवाज़ दी कि (इन) ज़ालिमों फिरऔन की क़ौम के पास जाओ (हिदायत करो)
\end{hindi}}
\flushright{\begin{Arabic}
\quranayah[26][11]
\end{Arabic}}
\flushleft{\begin{hindi}
क्या ये लोग (मेरे ग़ज़ब से) डरते नहीं है
\end{hindi}}
\flushright{\begin{Arabic}
\quranayah[26][12]
\end{Arabic}}
\flushleft{\begin{hindi}
मूसा ने अर्ज़ कि परवरदिगार मैं डरता हूँ कि (मुबादा) वह लोग मुझे झुठला दे
\end{hindi}}
\flushright{\begin{Arabic}
\quranayah[26][13]
\end{Arabic}}
\flushleft{\begin{hindi}
और (उनके झुठलाने से) मेरा दम रुक जाए और मेरी ज़बान (अच्छी तरह) न चले तो हारुन के पास पैग़ाम भेज दे (कि मेरा साथ दे)
\end{hindi}}
\flushright{\begin{Arabic}
\quranayah[26][14]
\end{Arabic}}
\flushleft{\begin{hindi}
(और इसके अलावा) उनका मेरे सर एक जुर्म भी है (कि मैने एक शख्स को मार डाला था)
\end{hindi}}
\flushright{\begin{Arabic}
\quranayah[26][15]
\end{Arabic}}
\flushleft{\begin{hindi}
तो मैं डरता हूँ कि (शायद) मुझे ये लाग मार डालें ख़ुदा ने कहा हरगिज़ नहीं अच्छा तुम दोनों हमारी निशानियाँ लेकर जाओ हम तुम्हारे साथ हैं
\end{hindi}}
\flushright{\begin{Arabic}
\quranayah[26][16]
\end{Arabic}}
\flushleft{\begin{hindi}
और (सारी गुफ्तगू) अच्छी तरह सुनते हैं ग़रज़ तुम दोनों फिरऔन के पास जाओ और कह दो कि हम सारे जहाँन के परवरदिगार के रसूल हैं (और पैग़ाम लाएँ हैं)
\end{hindi}}
\flushright{\begin{Arabic}
\quranayah[26][17]
\end{Arabic}}
\flushleft{\begin{hindi}
कि आप बनी इसराइल को हमारे साथ भेज दीजिए
\end{hindi}}
\flushright{\begin{Arabic}
\quranayah[26][18]
\end{Arabic}}
\flushleft{\begin{hindi}
(चुनान्चे मूसा गए और कहा) फिरऔन बोला (मूसा) क्या हमने तुम्हें यहाँ रख कर बचपने में तुम्हारी परवरिश नहीं की और तुम अपनी उम्र से बरसों हम मे रह सह चुके हो
\end{hindi}}
\flushright{\begin{Arabic}
\quranayah[26][19]
\end{Arabic}}
\flushleft{\begin{hindi}
और तुम अपना वह काम (ख़ून क़िब्ती) जो कर गए और तुम (बड़े) नाशुक्रे हो
\end{hindi}}
\flushright{\begin{Arabic}
\quranayah[26][20]
\end{Arabic}}
\flushleft{\begin{hindi}
मूसा ने कहा (हाँ) मैने उस वक्त उस काम को किया जब मै हालते ग़फलत में था
\end{hindi}}
\flushright{\begin{Arabic}
\quranayah[26][21]
\end{Arabic}}
\flushleft{\begin{hindi}
फिर जब मै आप लोगों से डरा तो भाग खड़ा हुआ फिर (कुछ अरसे के बाद) मेरे परवरदिगार ने मुझे नुबूवत अता फरमायी और मुझे भी एक पैग़म्बर बनाया
\end{hindi}}
\flushright{\begin{Arabic}
\quranayah[26][22]
\end{Arabic}}
\flushleft{\begin{hindi}
और ये भी कोई एहसान हे जिसे आप मुझ पर जता रहे है कि आप ने बनी इसराईल को ग़ुलाम बना रखा है
\end{hindi}}
\flushright{\begin{Arabic}
\quranayah[26][23]
\end{Arabic}}
\flushleft{\begin{hindi}
फिरऔन ने पूछा (अच्छा ये तो बताओ) रब्बुल आलमीन क्या चीज़ है
\end{hindi}}
\flushright{\begin{Arabic}
\quranayah[26][24]
\end{Arabic}}
\flushleft{\begin{hindi}
मूसा ने कहाँ सारे आसमान व ज़मीन का और जो कुछ इन दोनों के दरमियान है (सबका) मालिक अगर आप लोग यक़ीन कीजिए (तो काफी है)
\end{hindi}}
\flushright{\begin{Arabic}
\quranayah[26][25]
\end{Arabic}}
\flushleft{\begin{hindi}
फिरऔन ने उन लोगो से जो उसके इर्द गिर्द (बैठे) थे कहा क्या तुम लोग नहीं सुनते हो
\end{hindi}}
\flushright{\begin{Arabic}
\quranayah[26][26]
\end{Arabic}}
\flushleft{\begin{hindi}
मूसा ने कहा (वही ख़ुदा जो कि) तुम्हारा परवरदिगार और तुम्हारे बाप दादाओं का परवरदिगार है
\end{hindi}}
\flushright{\begin{Arabic}
\quranayah[26][27]
\end{Arabic}}
\flushleft{\begin{hindi}
फिरऔन ने कहा (लोगों) ये रसूल जो तुम्हारे पास भेजा गया है हो न हो दीवाना है
\end{hindi}}
\flushright{\begin{Arabic}
\quranayah[26][28]
\end{Arabic}}
\flushleft{\begin{hindi}
मूसा ने कहा (वह ख़ुदा जो) पूरब पश्चिम और जो कुछ इन दोनों के दरमियान (सबका) मालिक है अगर तुम समझते हो (तो यही काफी है)
\end{hindi}}
\flushright{\begin{Arabic}
\quranayah[26][29]
\end{Arabic}}
\flushleft{\begin{hindi}
फिरऔन ने कहा अगर तुम मेरे सिवा किसी और को (अपना) ख़ुदा बनाया है तो मै ज़रुर तुम्हे कैदी बनाऊँगा
\end{hindi}}
\flushright{\begin{Arabic}
\quranayah[26][30]
\end{Arabic}}
\flushleft{\begin{hindi}
मूसा ने कहा अगरचे मैं आपको एक वाजेए व रौशन मौजिज़ा भी दिखाऊ (तो भी)
\end{hindi}}
\flushright{\begin{Arabic}
\quranayah[26][31]
\end{Arabic}}
\flushleft{\begin{hindi}
फिरऔन ने कहा (अच्छा) तो तुम अगर (अपने दावे में) सच्चे हो तो ला दिखाओ
\end{hindi}}
\flushright{\begin{Arabic}
\quranayah[26][32]
\end{Arabic}}
\flushleft{\begin{hindi}
बस (ये सुनते ही) मूसा ने अपनी छड़ी (ज़मीन पर) डाल दी फिर तो यकायक वह एक सरीही अज़दहा बन गया
\end{hindi}}
\flushright{\begin{Arabic}
\quranayah[26][33]
\end{Arabic}}
\flushleft{\begin{hindi}
और (जेब से) अपना हाथ बाहर निकाला तो यकायक देखने वालों के वास्ते बहुत सफेद चमकदार था
\end{hindi}}
\flushright{\begin{Arabic}
\quranayah[26][34]
\end{Arabic}}
\flushleft{\begin{hindi}
(इस पर) फिरऔन अपने दरबारियों से जो उसके गिर्द (बैठे) थे कहने लगा
\end{hindi}}
\flushright{\begin{Arabic}
\quranayah[26][35]
\end{Arabic}}
\flushleft{\begin{hindi}
कि ये तो यक़ीनी बड़ा खिलाड़ी जादूगर है ये तो चाहता है कि अपने जादू के ज़ोर से तुम्हें तुम्हारे मुल्क से बाहर निकाल दे तो तुम लोग क्या हुक्म लगाते हो
\end{hindi}}
\flushright{\begin{Arabic}
\quranayah[26][36]
\end{Arabic}}
\flushleft{\begin{hindi}
दरबारियों ने कहा अभी इसको और इसके भाई को (चन्द) मोहलत दीजिए
\end{hindi}}
\flushright{\begin{Arabic}
\quranayah[26][37]
\end{Arabic}}
\flushleft{\begin{hindi}
और तमाम शहरों में जादूगरों के जमा करने को हरकारे रवाना कीजिए कि वह लोग तमाम बड़े बड़े खिलाड़ी जादूगरों की आपके सामने ला हाज़िर करें
\end{hindi}}
\flushright{\begin{Arabic}
\quranayah[26][38]
\end{Arabic}}
\flushleft{\begin{hindi}
ग़रज़ वक्ते मुकर्रर हुआ सब जादूगर उस मुक़र्रर के वायदे पर जमा किए गए
\end{hindi}}
\flushright{\begin{Arabic}
\quranayah[26][39]
\end{Arabic}}
\flushleft{\begin{hindi}
और लोगों में मुनादी करा दी गयी कि तुम लोग अब भी जमा होगे
\end{hindi}}
\flushright{\begin{Arabic}
\quranayah[26][40]
\end{Arabic}}
\flushleft{\begin{hindi}
या नहीं ताकि अगर जादूगर ग़ालिब और वर है तो हम लोग उनकी पैरवी करें
\end{hindi}}
\flushright{\begin{Arabic}
\quranayah[26][41]
\end{Arabic}}
\flushleft{\begin{hindi}
अलग़रज जब सब जादूगर आ गये तो जादूगरों ने फिरऔन से कहा कि अगर हम ग़ालिब आ गए तो हमको यक़ीनन कुछ इनाम (सरकार से) मिलेगा
\end{hindi}}
\flushright{\begin{Arabic}
\quranayah[26][42]
\end{Arabic}}
\flushleft{\begin{hindi}
फिरऔन ने कहा हा (ज़रुर मिलेगा) और (इनाम क्या चीज़ है) तुम उस वक्त (मेरे) मुकररेबीन (बारगाह) से हो गए
\end{hindi}}
\flushright{\begin{Arabic}
\quranayah[26][43]
\end{Arabic}}
\flushleft{\begin{hindi}
मूसा ने जादूगरों से कहा (मंत्र व तंत्र) जो कुछ तुम्हें फेंकना हो फेंको
\end{hindi}}
\flushright{\begin{Arabic}
\quranayah[26][44]
\end{Arabic}}
\flushleft{\begin{hindi}
इस पर जादूगरों ने अपनी रस्सियाँ और अपनी छड़ियाँ (मैदान में) डाल दी और कहने लगे फिरऔन के जलाल की क़सम हम ही ज़रुर ग़ालिब रहेंगे
\end{hindi}}
\flushright{\begin{Arabic}
\quranayah[26][45]
\end{Arabic}}
\flushleft{\begin{hindi}
तब मूसा ने अपनी छड़ी डाली तो जादूगरों ने जो कुछ (शोबदे) बनाए थे उसको वह निगलने लगी
\end{hindi}}
\flushright{\begin{Arabic}
\quranayah[26][46]
\end{Arabic}}
\flushleft{\begin{hindi}
ये देखते ही जादूगर लोग सजदे में (मूसा के सामने) गिर पडे
\end{hindi}}
\flushright{\begin{Arabic}
\quranayah[26][47]
\end{Arabic}}
\flushleft{\begin{hindi}
और कहने लगे हम सारे जहाँ के परवरदिगार पर ईमान लाए
\end{hindi}}
\flushright{\begin{Arabic}
\quranayah[26][48]
\end{Arabic}}
\flushleft{\begin{hindi}
जो मूसा और हारुन का परवरदिगार है
\end{hindi}}
\flushright{\begin{Arabic}
\quranayah[26][49]
\end{Arabic}}
\flushleft{\begin{hindi}
फिरऔन ने कहा (हाए) क़ब्ल इसके कि मै तुम्हें इजाज़त दूँ तुम इस पर ईमान ले आए बेशक ये तुम्हारा बड़ा (गुरु है जिसने तुम सबको जादू सिखाया है तो ख़ैर) अभी तुम लोगों को (इसका नतीजा) मालूम हो जाएगा कि हम यक़ीनन तुम्हारे एक तरफ के हाथ और दूसरी तरफ के पाँव काट डालेगें और तुम सब के सब को सूली देगें
\end{hindi}}
\flushright{\begin{Arabic}
\quranayah[26][50]
\end{Arabic}}
\flushleft{\begin{hindi}
वह बोले कुछ परवाह नही हमको तो बहरहाल अपने परवरदिगार की तरफ लौट कर जाना है
\end{hindi}}
\flushright{\begin{Arabic}
\quranayah[26][51]
\end{Arabic}}
\flushleft{\begin{hindi}
हम चँकि सबसे पहले ईमान लाए है इसलिए ये उम्मीद रखते हैं कि हमारा परवरदिगार हमारी ख़ताएँ माफ कर देगा
\end{hindi}}
\flushright{\begin{Arabic}
\quranayah[26][52]
\end{Arabic}}
\flushleft{\begin{hindi}
और हमने मूसा के पास वही भेजी कि तुम मेरे बन्दों को लेकर रातों रात निकल जाओ क्योंकि तुम्हारा पीछा किया जाएगा
\end{hindi}}
\flushright{\begin{Arabic}
\quranayah[26][53]
\end{Arabic}}
\flushleft{\begin{hindi}
तब फिरऔन ने (लश्कर जमा करने के ख्याल से) तमाम शहरों में (धड़ा धड़) हरकारे रवाना किए
\end{hindi}}
\flushright{\begin{Arabic}
\quranayah[26][54]
\end{Arabic}}
\flushleft{\begin{hindi}
(और कहा) कि ये लोग मूसा के साथ बनी इसराइल थोड़ी सी (मुट्ठी भर की) जमाअत हैं
\end{hindi}}
\flushright{\begin{Arabic}
\quranayah[26][55]
\end{Arabic}}
\flushleft{\begin{hindi}
और उन लोगों ने हमें सख्त गुस्सा दिलाया है
\end{hindi}}
\flushright{\begin{Arabic}
\quranayah[26][56]
\end{Arabic}}
\flushleft{\begin{hindi}
और हम सबके सब बा साज़ों सामान हैं
\end{hindi}}
\flushright{\begin{Arabic}
\quranayah[26][57]
\end{Arabic}}
\flushleft{\begin{hindi}
(तुम भी आ जाओ कि सब मिलकर ताअककुब (पीछा) करें)
\end{hindi}}
\flushright{\begin{Arabic}
\quranayah[26][58]
\end{Arabic}}
\flushleft{\begin{hindi}
ग़रज़ हमने इन लोगों को (मिस्र के) बाग़ों और चश्मों और खज़ानों और इज्ज़त की जगह से (यूँ) निकाल बाहर किया
\end{hindi}}
\flushright{\begin{Arabic}
\quranayah[26][59]
\end{Arabic}}
\flushleft{\begin{hindi}
(और जो नाफरमानी करे) इसी तरह सज़ा होगी और आख़िर हमने उन्हीं चीज़ों का मालिक बनी इसराइल को बनाया
\end{hindi}}
\flushright{\begin{Arabic}
\quranayah[26][60]
\end{Arabic}}
\flushleft{\begin{hindi}
ग़रज़ (मूसा) तो रात ही को चले गए
\end{hindi}}
\flushright{\begin{Arabic}
\quranayah[26][61]
\end{Arabic}}
\flushleft{\begin{hindi}
और उन लोगों ने सूरज निकलते उनका पीछा किया तो जब दोनों जमाअतें (इतनी करीब हुयीं कि) एक दूसरे को देखने लगी तो मूसा के साथी (हैरान होकर) कहने लगे
\end{hindi}}
\flushright{\begin{Arabic}
\quranayah[26][62]
\end{Arabic}}
\flushleft{\begin{hindi}
कि अब तो पकड़े गए मूसा ने कहा हरगिज़ नहीं क्योंकि मेरे साथ मेरा परवरदिगार है
\end{hindi}}
\flushright{\begin{Arabic}
\quranayah[26][63]
\end{Arabic}}
\flushleft{\begin{hindi}
वह फौरन मुझे कोई (मुखलिसी का) रास्ता बता देगा तो हमने मूसा के पास वही भेजी कि अपनी छड़ी दरिया पर मारो (मारना था कि) फौरन दरिया फुट के टुकड़े टुकड़े हो गया तो गोया हर टुकड़ा एक बड़ा ऊँचा पहाड़ था
\end{hindi}}
\flushright{\begin{Arabic}
\quranayah[26][64]
\end{Arabic}}
\flushleft{\begin{hindi}
और हमने उसी जगह दूसरे फरीक (फिरऔन के साथी) को क़रीब कर दिया
\end{hindi}}
\flushright{\begin{Arabic}
\quranayah[26][65]
\end{Arabic}}
\flushleft{\begin{hindi}
और मूसा और उसके साथियों को हमने (डूबने से) बचा लिया
\end{hindi}}
\flushright{\begin{Arabic}
\quranayah[26][66]
\end{Arabic}}
\flushleft{\begin{hindi}
फिर दूसरे फरीक़ (फिरऔन और उसके साथियों) को डुबोकर हलाक़ कर दिया
\end{hindi}}
\flushright{\begin{Arabic}
\quranayah[26][67]
\end{Arabic}}
\flushleft{\begin{hindi}
बेशक इसमें यक़ीनन एक बड़ी इबरत है और उनमें अक्सर ईमान लाने वाले ही न थे
\end{hindi}}
\flushright{\begin{Arabic}
\quranayah[26][68]
\end{Arabic}}
\flushleft{\begin{hindi}
और इसमें तो शक ही न था कि तुम्हारा परवरदिगार यक़ीनन (सब पर) ग़ालिब और बड़ा मेहरबान है
\end{hindi}}
\flushright{\begin{Arabic}
\quranayah[26][69]
\end{Arabic}}
\flushleft{\begin{hindi}
और (ऐ रसूल) उन लोगों के सामने इबराहीम का किस्सा बयान करों
\end{hindi}}
\flushright{\begin{Arabic}
\quranayah[26][70]
\end{Arabic}}
\flushleft{\begin{hindi}
जब उन्होंने अपने (मुँह बोले) बाप और अपनी क़ौम से कहा
\end{hindi}}
\flushright{\begin{Arabic}
\quranayah[26][71]
\end{Arabic}}
\flushleft{\begin{hindi}
कि तुम लोग किसकी इबादत करते हो तो वह बोले हम बुतों की इबादत करते हैं और उन्हीं के मुजाविर बन जाते हैं
\end{hindi}}
\flushright{\begin{Arabic}
\quranayah[26][72]
\end{Arabic}}
\flushleft{\begin{hindi}
इबराहीम ने कहा भला जब तुम लोग उन्हें पुकारते हो तो वह तुम्हारी कुछ सुनते हैं
\end{hindi}}
\flushright{\begin{Arabic}
\quranayah[26][73]
\end{Arabic}}
\flushleft{\begin{hindi}
या तम्हें कुछ नफा या नुक़सान पहुँचा सकते हैं
\end{hindi}}
\flushright{\begin{Arabic}
\quranayah[26][74]
\end{Arabic}}
\flushleft{\begin{hindi}
कहने लगे (कि ये सब तो कुछ नहीं) बल्कि हमने अपने बाप दादाओं को ऐसा ही करते पाया है
\end{hindi}}
\flushright{\begin{Arabic}
\quranayah[26][75]
\end{Arabic}}
\flushleft{\begin{hindi}
इबराहीम ने कहा क्या तुमने देखा भी कि जिन चीज़ों कीे तुम परसतिश करते हो
\end{hindi}}
\flushright{\begin{Arabic}
\quranayah[26][76]
\end{Arabic}}
\flushleft{\begin{hindi}
या तुम्हारे अगले बाप दादा (करते थे) ये सब मेरे यक़ीनी दुश्मन हैं
\end{hindi}}
\flushright{\begin{Arabic}
\quranayah[26][77]
\end{Arabic}}
\flushleft{\begin{hindi}
मगर सारे जहाँ का पालने वाला जिसने मुझे पैदा किया (वही मेरा दोस्त है)
\end{hindi}}
\flushright{\begin{Arabic}
\quranayah[26][78]
\end{Arabic}}
\flushleft{\begin{hindi}
फिर वही मेरी हिदायत करता है
\end{hindi}}
\flushright{\begin{Arabic}
\quranayah[26][79]
\end{Arabic}}
\flushleft{\begin{hindi}
और वह शख्स जो मुझे (खाना) खिलाता है और मुझे (पानी) पिलाता है
\end{hindi}}
\flushright{\begin{Arabic}
\quranayah[26][80]
\end{Arabic}}
\flushleft{\begin{hindi}
और जब बीमार पड़ता हूँ तो वही मुझे शिफा इनायत फरमाता है
\end{hindi}}
\flushright{\begin{Arabic}
\quranayah[26][81]
\end{Arabic}}
\flushleft{\begin{hindi}
और वह वही हेै जो मुझे मार डालेगा और उसके बाद (फिर) मुझे ज़िन्दा करेगा
\end{hindi}}
\flushright{\begin{Arabic}
\quranayah[26][82]
\end{Arabic}}
\flushleft{\begin{hindi}
और वह वही है जिससे मै उम्मीद रखता हूँ कि क़यामत के दिन मेरी ख़ताओं को बख्श देगा
\end{hindi}}
\flushright{\begin{Arabic}
\quranayah[26][83]
\end{Arabic}}
\flushleft{\begin{hindi}
परवरदिगार मुझे इल्म व फहम अता फरमा और मुझे नेकों के साथ शामिल कर
\end{hindi}}
\flushright{\begin{Arabic}
\quranayah[26][84]
\end{Arabic}}
\flushleft{\begin{hindi}
और आइन्दा आने वाली नस्लों में मेरा ज़िक्रे ख़ैर क़ायम रख
\end{hindi}}
\flushright{\begin{Arabic}
\quranayah[26][85]
\end{Arabic}}
\flushleft{\begin{hindi}
और मुझे भी नेअमत के बाग़ (बेहश्त) के वारिसों में से बना
\end{hindi}}
\flushright{\begin{Arabic}
\quranayah[26][86]
\end{Arabic}}
\flushleft{\begin{hindi}
और मेरे (मुँह बोले) बाप (चचा आज़र) को बख्श दे क्योंकि वह गुमराहों में से है
\end{hindi}}
\flushright{\begin{Arabic}
\quranayah[26][87]
\end{Arabic}}
\flushleft{\begin{hindi}
और जिस दिन लोग क़ब्रों से उठाए जाएँगें मुझे रुसवा न करना
\end{hindi}}
\flushright{\begin{Arabic}
\quranayah[26][88]
\end{Arabic}}
\flushleft{\begin{hindi}
जिस दिन न तो माल ही कुछ काम आएगा और न लड़के बाले
\end{hindi}}
\flushright{\begin{Arabic}
\quranayah[26][89]
\end{Arabic}}
\flushleft{\begin{hindi}
मगर जो शख्स ख़ुदा के सामने (गुनाहों से) पाक दिल लिए हुए हाज़िर होगा (वह फायदे में रहेगा)
\end{hindi}}
\flushright{\begin{Arabic}
\quranayah[26][90]
\end{Arabic}}
\flushleft{\begin{hindi}
और बेहश्त परहेज़ गारों के क़रीब कर दी जाएगी
\end{hindi}}
\flushright{\begin{Arabic}
\quranayah[26][91]
\end{Arabic}}
\flushleft{\begin{hindi}
और दोज़ख़ गुमराहों के सामने ज़ाहिर कर दी जाएगी
\end{hindi}}
\flushright{\begin{Arabic}
\quranayah[26][92]
\end{Arabic}}
\flushleft{\begin{hindi}
और उन लोगों (अहले जहन्नुम) से पूछा जाएगा कि ख़ुदा को छोड़कर जिनकी तुम परसतिश करते थे (आज) वह कहाँ हैं
\end{hindi}}
\flushright{\begin{Arabic}
\quranayah[26][93]
\end{Arabic}}
\flushleft{\begin{hindi}
क्या वह तुम्हारी कुछ मदद कर सकते हैं या वह ख़ुद अपनी आप बाहम मदद कर सकते हैं
\end{hindi}}
\flushright{\begin{Arabic}
\quranayah[26][94]
\end{Arabic}}
\flushleft{\begin{hindi}
फिर वह (माबूद) और गुमराह लोग और शैतान का लशकर
\end{hindi}}
\flushright{\begin{Arabic}
\quranayah[26][95]
\end{Arabic}}
\flushleft{\begin{hindi}
(ग़रज़ सबके सब) जहन्नुम में औधें मुँह ढकेल दिए जाएँगे
\end{hindi}}
\flushright{\begin{Arabic}
\quranayah[26][96]
\end{Arabic}}
\flushleft{\begin{hindi}
और ये लोग जहन्नुम में बाहम झगड़ा करेंगे और अपने माबूद से कहेंगे
\end{hindi}}
\flushright{\begin{Arabic}
\quranayah[26][97]
\end{Arabic}}
\flushleft{\begin{hindi}
ख़ुदा की क़सम हम लोग तो यक़ीनन सरीही गुमराही में थे
\end{hindi}}
\flushright{\begin{Arabic}
\quranayah[26][98]
\end{Arabic}}
\flushleft{\begin{hindi}
कि हम तुम को सारे जहाँन के पालने वाले (ख़ुदा) के बराबर समझते रहे
\end{hindi}}
\flushright{\begin{Arabic}
\quranayah[26][99]
\end{Arabic}}
\flushleft{\begin{hindi}
और हमको बस (उन) गुनाहगारों ने (जो मुझसे पहले हुए) गुमराह किया
\end{hindi}}
\flushright{\begin{Arabic}
\quranayah[26][100]
\end{Arabic}}
\flushleft{\begin{hindi}
तो अब तो न कोई (साहब) मेरी सिफारिश करने वाले हैं
\end{hindi}}
\flushright{\begin{Arabic}
\quranayah[26][101]
\end{Arabic}}
\flushleft{\begin{hindi}
और न कोई दिलबन्द दोस्त हैं
\end{hindi}}
\flushright{\begin{Arabic}
\quranayah[26][102]
\end{Arabic}}
\flushleft{\begin{hindi}
तो काश हमें अब दुनिया में दोबारा जाने का मौक़ा मिलता तो हम (ज़रुर) ईमान वालों से होते
\end{hindi}}
\flushright{\begin{Arabic}
\quranayah[26][103]
\end{Arabic}}
\flushleft{\begin{hindi}
इबराहीम के इस किस्से में भी यक़ीनन एक बड़ी इबरत है और इनमें से अक्सर ईमान लाने वाले थे भी नहीं
\end{hindi}}
\flushright{\begin{Arabic}
\quranayah[26][104]
\end{Arabic}}
\flushleft{\begin{hindi}
और इसमे तो शक ही नहीं कि तुम्हारा परवरदिगार (सब पर) ग़ालिब और बड़ा मेहरबान है
\end{hindi}}
\flushright{\begin{Arabic}
\quranayah[26][105]
\end{Arabic}}
\flushleft{\begin{hindi}
(यूँ ही) नूह की क़ौम ने पैग़म्बरो को झुठलाया
\end{hindi}}
\flushright{\begin{Arabic}
\quranayah[26][106]
\end{Arabic}}
\flushleft{\begin{hindi}
कि जब उनसे उन के भाई नूह ने कहा कि तुम लोग (ख़ुदा से) क्यों नहीं डरते मै तो तुम्हारा यक़ीनी अमानत दार पैग़म्बर हूँ
\end{hindi}}
\flushright{\begin{Arabic}
\quranayah[26][107]
\end{Arabic}}
\flushleft{\begin{hindi}
तुम खुदा से डरो और मेरी इताअत करो
\end{hindi}}
\flushright{\begin{Arabic}
\quranayah[26][108]
\end{Arabic}}
\flushleft{\begin{hindi}
और मैं इस (तबलीग़े रिसालत) पर कुछ उजरत तो माँगता नहीं
\end{hindi}}
\flushright{\begin{Arabic}
\quranayah[26][109]
\end{Arabic}}
\flushleft{\begin{hindi}
मेरी उजरत तो बस सारे जहाँ के पालने वाले ख़ुदा पर है
\end{hindi}}
\flushright{\begin{Arabic}
\quranayah[26][110]
\end{Arabic}}
\flushleft{\begin{hindi}
तो ख़ुदा से डरो और मेरी इताअत करो वह लोग बोले जब कमीनो मज़दूरों वग़ैरह ने (लालच से) तुम्हारी पैरवी कर ली है
\end{hindi}}
\flushright{\begin{Arabic}
\quranayah[26][111]
\end{Arabic}}
\flushleft{\begin{hindi}
तो हम तुम पर क्या ईमान लाएं
\end{hindi}}
\flushright{\begin{Arabic}
\quranayah[26][112]
\end{Arabic}}
\flushleft{\begin{hindi}
नूह ने कहा ये लोग जो कुछ करते थे मुझे क्या ख़बर (और क्या ग़रज़)
\end{hindi}}
\flushright{\begin{Arabic}
\quranayah[26][113]
\end{Arabic}}
\flushleft{\begin{hindi}
इन लोगों का हिसाब तो मेरे परवरदिगार के ज़िम्मे है
\end{hindi}}
\flushright{\begin{Arabic}
\quranayah[26][114]
\end{Arabic}}
\flushleft{\begin{hindi}
काश तुम (इतनी) समझ रखते और मै तो ईमानदारों को अपने पास से निकालने वाला नहीं
\end{hindi}}
\flushright{\begin{Arabic}
\quranayah[26][115]
\end{Arabic}}
\flushleft{\begin{hindi}
मै तो सिर्फ (अज़ाबे ख़ुदा से) साफ साफ डराने वाला हूँ
\end{hindi}}
\flushright{\begin{Arabic}
\quranayah[26][116]
\end{Arabic}}
\flushleft{\begin{hindi}
वह लोग कहने लगे ऐ नूह अगर तुम अपनी हरकत से बाज़ न आओगे तो ज़रुर संगसार कर दिए जाओगे
\end{hindi}}
\flushright{\begin{Arabic}
\quranayah[26][117]
\end{Arabic}}
\flushleft{\begin{hindi}
नूह ने अर्ज की परवरदिगार मेरी क़ौम ने यक़ीनन मुझे झुठलाया
\end{hindi}}
\flushright{\begin{Arabic}
\quranayah[26][118]
\end{Arabic}}
\flushleft{\begin{hindi}
तो अब तू मेरे और इन लोगों के दरमियान एक क़तई फैसला कर दे और मुझे और जो मोमिनीन मेरे साथ हें उनको नजात दे
\end{hindi}}
\flushright{\begin{Arabic}
\quranayah[26][119]
\end{Arabic}}
\flushleft{\begin{hindi}
ग़रज़ हमने नूह और उनके साथियों को जो भरी हुई कश्ती में थे नजात दी
\end{hindi}}
\flushright{\begin{Arabic}
\quranayah[26][120]
\end{Arabic}}
\flushleft{\begin{hindi}
फिर उसके बाद हमने बाक़ी लोगों को ग़रक कर दिया
\end{hindi}}
\flushright{\begin{Arabic}
\quranayah[26][121]
\end{Arabic}}
\flushleft{\begin{hindi}
बेशक इसमे भी यक़ीनन बड़ी इबरत है और उनमें से बहुतेरे ईमान लाने वाले ही न थे
\end{hindi}}
\flushright{\begin{Arabic}
\quranayah[26][122]
\end{Arabic}}
\flushleft{\begin{hindi}
और इसमें तो शक ही नहीं कि तुम्हारा परवरदिगार (सब पर) ग़ालिब मेहरबान है
\end{hindi}}
\flushright{\begin{Arabic}
\quranayah[26][123]
\end{Arabic}}
\flushleft{\begin{hindi}
(इसी तरह क़ौम) आद ने पैग़म्बरों को झुठलाया
\end{hindi}}
\flushright{\begin{Arabic}
\quranayah[26][124]
\end{Arabic}}
\flushleft{\begin{hindi}
जब उनके भाई हूद ने उनसे कहा कि तुम ख़ुदा से क्यों नही डरते
\end{hindi}}
\flushright{\begin{Arabic}
\quranayah[26][125]
\end{Arabic}}
\flushleft{\begin{hindi}
मैं तो यक़ीनन तुम्हारा अमानतदार पैग़म्बर हूँ
\end{hindi}}
\flushright{\begin{Arabic}
\quranayah[26][126]
\end{Arabic}}
\flushleft{\begin{hindi}
तो ख़ुदा से डरो और मेरी इताअत करो
\end{hindi}}
\flushright{\begin{Arabic}
\quranayah[26][127]
\end{Arabic}}
\flushleft{\begin{hindi}
मै तो तुम से इस (तबलीग़े रिसालत) पर कुछ मज़दूरी भी नहीं माँगता मेरी उजरत तो बस सारी ख़ुदायी के पालने वाले (ख़ुदा) पर है
\end{hindi}}
\flushright{\begin{Arabic}
\quranayah[26][128]
\end{Arabic}}
\flushleft{\begin{hindi}
तो क्या तुम ऊँची जगह पर बेकार यादगारे बनाते फिरते हो
\end{hindi}}
\flushright{\begin{Arabic}
\quranayah[26][129]
\end{Arabic}}
\flushleft{\begin{hindi}
और बड़े बड़े महल तामीर करते हो गोया तुम हमेशा (यहीं) रहोगे
\end{hindi}}
\flushright{\begin{Arabic}
\quranayah[26][130]
\end{Arabic}}
\flushleft{\begin{hindi}
और जब तुम (किसी पर) हाथ डालते हो तो सरकशी से हाथ डालते हो
\end{hindi}}
\flushright{\begin{Arabic}
\quranayah[26][131]
\end{Arabic}}
\flushleft{\begin{hindi}
तो तुम ख़ुदा से डरो और मेरी इताअत करो
\end{hindi}}
\flushright{\begin{Arabic}
\quranayah[26][132]
\end{Arabic}}
\flushleft{\begin{hindi}
और उस शख्स से डरो जिसने तुम्हारी उन चीज़ों से मदद की जिन्हें तुम खूब जानते हो
\end{hindi}}
\flushright{\begin{Arabic}
\quranayah[26][133]
\end{Arabic}}
\flushleft{\begin{hindi}
अच्छा सुनो उसने तुम्हारे चार पायों और लड़के बालों वग़ैरह और चश्मों से मदद की
\end{hindi}}
\flushright{\begin{Arabic}
\quranayah[26][134]
\end{Arabic}}
\flushleft{\begin{hindi}
मै तो यक़ीनन तुम पर
\end{hindi}}
\flushright{\begin{Arabic}
\quranayah[26][135]
\end{Arabic}}
\flushleft{\begin{hindi}
एक बड़े (सख्त) रोज़ के अज़ाब से डरता हूँ
\end{hindi}}
\flushright{\begin{Arabic}
\quranayah[26][136]
\end{Arabic}}
\flushleft{\begin{hindi}
वह लोग कहने लगे ख्वाह तुम नसीहत करो या न नसीहत करो हमारे वास्ते (सब) बराबर है
\end{hindi}}
\flushright{\begin{Arabic}
\quranayah[26][137]
\end{Arabic}}
\flushleft{\begin{hindi}
ये (डरावा) तो बस अगले लोगों की आदत है
\end{hindi}}
\flushright{\begin{Arabic}
\quranayah[26][138]
\end{Arabic}}
\flushleft{\begin{hindi}
हालाँकि हम पर अज़ाब (वग़ैरह अब) किया नहीं जाएगा
\end{hindi}}
\flushright{\begin{Arabic}
\quranayah[26][139]
\end{Arabic}}
\flushleft{\begin{hindi}
ग़रज़ उन लोगों ने हूद को झुठला दिया तो हमने भी उनको हलाक कर डाला बेशक इस वाक़िये में यक़ीनी एक बड़ी इबरत है आर उनमें से बहुतेरे ईमान लाने वाले भी न थे
\end{hindi}}
\flushright{\begin{Arabic}
\quranayah[26][140]
\end{Arabic}}
\flushleft{\begin{hindi}
और इसमें शक नहीं कि तुम्हारा परवरदिगार यक़ीनन (सब पर) ग़ालिब (और) बड़ा मेहरबान है
\end{hindi}}
\flushright{\begin{Arabic}
\quranayah[26][141]
\end{Arabic}}
\flushleft{\begin{hindi}
(इसी तरह क़ौम) समूद ने पैग़म्बरों को झुठलाया
\end{hindi}}
\flushright{\begin{Arabic}
\quranayah[26][142]
\end{Arabic}}
\flushleft{\begin{hindi}
जब उनके भाई सालेह ने उनसे कहा कि तुम (ख़ुदा से) क्यो नहीं डरते
\end{hindi}}
\flushright{\begin{Arabic}
\quranayah[26][143]
\end{Arabic}}
\flushleft{\begin{hindi}
मैं तो यक़ीनन तुम्हारा अमानतदार पैग़म्बर हूँ
\end{hindi}}
\flushright{\begin{Arabic}
\quranayah[26][144]
\end{Arabic}}
\flushleft{\begin{hindi}
तो खुदा से डरो और मेरी इताअत करो
\end{hindi}}
\flushright{\begin{Arabic}
\quranayah[26][145]
\end{Arabic}}
\flushleft{\begin{hindi}
और मै तो तुमसे इस (तबलीगे रिसालत) पर कुछ मज़दूरी भी नहीं माँगता- मेरी मज़दूरी तो बस सारी ख़ुदाई के पालने वाले (ख़ुदा पर है)
\end{hindi}}
\flushright{\begin{Arabic}
\quranayah[26][146]
\end{Arabic}}
\flushleft{\begin{hindi}
क्या जो चीजें यहाँ (दुनिया में) मौजूद है
\end{hindi}}
\flushright{\begin{Arabic}
\quranayah[26][147]
\end{Arabic}}
\flushleft{\begin{hindi}
बाग़ और चश्में और खेतिया और छुहारे जिनकी कलियाँ लतीफ़ व नाज़ुक होती है
\end{hindi}}
\flushright{\begin{Arabic}
\quranayah[26][148]
\end{Arabic}}
\flushleft{\begin{hindi}
उन्हीं मे तुम लोग इतमिनान से (हमेशा के लिए) छोड़ दिए जाओगे
\end{hindi}}
\flushright{\begin{Arabic}
\quranayah[26][149]
\end{Arabic}}
\flushleft{\begin{hindi}
और (इस वजह से) पूरी महारत और तकलीफ के साथ पहाड़ों को काट काट कर घर बनाते हो
\end{hindi}}
\flushright{\begin{Arabic}
\quranayah[26][150]
\end{Arabic}}
\flushleft{\begin{hindi}
तो ख़ुदा से डरो और मेरी इताअत करो
\end{hindi}}
\flushright{\begin{Arabic}
\quranayah[26][151]
\end{Arabic}}
\flushleft{\begin{hindi}
और ज्यादती करने वालों का कहा न मानों
\end{hindi}}
\flushright{\begin{Arabic}
\quranayah[26][152]
\end{Arabic}}
\flushleft{\begin{hindi}
जो रुए ज़मीन पर फ़साद फैलाया करते हैं और (ख़राबियों की) इसलाह नहीं करते
\end{hindi}}
\flushright{\begin{Arabic}
\quranayah[26][153]
\end{Arabic}}
\flushleft{\begin{hindi}
वह लोग बोले कि तुम पर तो बस जादू कर दिया गया है (कि ऐसी बातें करते हो)
\end{hindi}}
\flushright{\begin{Arabic}
\quranayah[26][154]
\end{Arabic}}
\flushleft{\begin{hindi}
तुम भी तो आख़िर हमारे ही ऐसे आदमी हो पस अगर तुम सच्चे हो तो कोई मौजिज़ा हमारे पास ला (दिखाओ)
\end{hindi}}
\flushright{\begin{Arabic}
\quranayah[26][155]
\end{Arabic}}
\flushleft{\begin{hindi}
सालेह ने कहा- यही ऊँटनी (मौजिज़ा) है एक बारी इसके पानी पीने की है और एक मुक़र्रर दिन तुम्हारे पीने का
\end{hindi}}
\flushright{\begin{Arabic}
\quranayah[26][156]
\end{Arabic}}
\flushleft{\begin{hindi}
और इसको कोई तकलीफ़ न पहुँचाना वरना एक बड़े (सख्त) ज़ोर का अज़ाब तुम्हे ले डालेगा
\end{hindi}}
\flushright{\begin{Arabic}
\quranayah[26][157]
\end{Arabic}}
\flushleft{\begin{hindi}
इस पर भी उन लोगों ने उसके पाँव काट डाले और (उसको मार डाला) फिर ख़़ुद पशेमान हुए
\end{hindi}}
\flushright{\begin{Arabic}
\quranayah[26][158]
\end{Arabic}}
\flushleft{\begin{hindi}
फिर उन्हें अज़ाब ने ले डाला-बेशक इसमें यक़ीनन एक बड़ी इबरत है और इनमें के बहुतेरे ईमान लाने वाले भी न थे
\end{hindi}}
\flushright{\begin{Arabic}
\quranayah[26][159]
\end{Arabic}}
\flushleft{\begin{hindi}
और इसमें शक ही नहीं कि तुम्हारा परवरदिगार (सब पर) ग़ालिब और मेहरबान है
\end{hindi}}
\flushright{\begin{Arabic}
\quranayah[26][160]
\end{Arabic}}
\flushleft{\begin{hindi}
इसी तरह लूत की क़ौम ने पैग़म्बरों को झुठलाया
\end{hindi}}
\flushright{\begin{Arabic}
\quranayah[26][161]
\end{Arabic}}
\flushleft{\begin{hindi}
जब उनके भाई लूत ने उनसे कहा कि तुम (ख़ुदा से) क्यों नहीं डरते
\end{hindi}}
\flushright{\begin{Arabic}
\quranayah[26][162]
\end{Arabic}}
\flushleft{\begin{hindi}
मै तो यक़ीनन तुम्हारा अमानतदार पैग़म्बर हूँ तो ख़ुदा से डरो
\end{hindi}}
\flushright{\begin{Arabic}
\quranayah[26][163]
\end{Arabic}}
\flushleft{\begin{hindi}
और मेरी इताअत करो
\end{hindi}}
\flushright{\begin{Arabic}
\quranayah[26][164]
\end{Arabic}}
\flushleft{\begin{hindi}
और मै तो तुमसे इस (तबलीगे रिसालत) पर कुछ मज़दूरी भी नहीं माँगता मेरी मज़दूरी तो बस सारी ख़ुदायी के पालने वाले (ख़ुदा) पर है
\end{hindi}}
\flushright{\begin{Arabic}
\quranayah[26][165]
\end{Arabic}}
\flushleft{\begin{hindi}
क्या तुम लोग (शहवत परस्ती के लिए) सारे जहाँ के लोगों में मर्दों ही के पास जाते हो
\end{hindi}}
\flushright{\begin{Arabic}
\quranayah[26][166]
\end{Arabic}}
\flushleft{\begin{hindi}
और तुम्हारे वास्ते जो बीवियाँ तुम्हारे परवरदिगार ने पैदा की है उन्हें छोड़ देते हो (ये कुछ नहीं) बल्कि तुम लोग हद से गुज़र जाने वाले आदमी हो
\end{hindi}}
\flushright{\begin{Arabic}
\quranayah[26][167]
\end{Arabic}}
\flushleft{\begin{hindi}
उन लोगों ने कहा ऐ लूत अगर तुम बाज़ न आओगे तो तुम ज़रुर निकल बाहर कर दिए जाओगे
\end{hindi}}
\flushright{\begin{Arabic}
\quranayah[26][168]
\end{Arabic}}
\flushleft{\begin{hindi}
लूत ने कहा मै यक़ीनन तुम्हारी (नाशाइसता) हरकत से बेज़ार हूँ
\end{hindi}}
\flushright{\begin{Arabic}
\quranayah[26][169]
\end{Arabic}}
\flushleft{\begin{hindi}
(और दुआ की) परवरदिगार जो कुछ ये लोग करते है उससे मुझे और मेरे लड़कों को नजात दे
\end{hindi}}
\flushright{\begin{Arabic}
\quranayah[26][170]
\end{Arabic}}
\flushleft{\begin{hindi}
तो हमने उनको और उनके सब लड़कों को नजात दी
\end{hindi}}
\flushright{\begin{Arabic}
\quranayah[26][171]
\end{Arabic}}
\flushleft{\begin{hindi}
मगर (लूत की) बूढ़ी औरत कि वह पीछे रह गयी
\end{hindi}}
\flushright{\begin{Arabic}
\quranayah[26][172]
\end{Arabic}}
\flushleft{\begin{hindi}
(और हलाक हो गयी) फिर हमने उन लोगों को हलाक कर डाला
\end{hindi}}
\flushright{\begin{Arabic}
\quranayah[26][173]
\end{Arabic}}
\flushleft{\begin{hindi}
और उन पर हमने (पत्थरों का) मेंह बरसाया तो जिन लोगों को (अज़ाबे ख़ुदा से) डराया गया था
\end{hindi}}
\flushright{\begin{Arabic}
\quranayah[26][174]
\end{Arabic}}
\flushleft{\begin{hindi}
उन पर क्या बड़ी बारिश हुई इस वाक़िये में भी एक बड़ी इबरत है और इनमें से बहुतेरे ईमान लाने वाले ही न थे
\end{hindi}}
\flushright{\begin{Arabic}
\quranayah[26][175]
\end{Arabic}}
\flushleft{\begin{hindi}
और इसमे तो शक ही नहीं कि तुम्हारा परवरदिगार यक़ीनन सब पर ग़ालिब (और) बड़ा मेहरबान है
\end{hindi}}
\flushright{\begin{Arabic}
\quranayah[26][176]
\end{Arabic}}
\flushleft{\begin{hindi}
इसी तरह जंगल के रहने वालों ने (मेरे) पैग़म्बरों को झुठलाया
\end{hindi}}
\flushright{\begin{Arabic}
\quranayah[26][177]
\end{Arabic}}
\flushleft{\begin{hindi}
जब शुएब ने उनसे कहा कि तुम (ख़ुदा से) क्यों नहीं डरते
\end{hindi}}
\flushright{\begin{Arabic}
\quranayah[26][178]
\end{Arabic}}
\flushleft{\begin{hindi}
मै तो बिला शुबाह तुम्हारा अमानदार हूँ
\end{hindi}}
\flushright{\begin{Arabic}
\quranayah[26][179]
\end{Arabic}}
\flushleft{\begin{hindi}
तो ख़ुदा से डरो और मेरी इताअत करो
\end{hindi}}
\flushright{\begin{Arabic}
\quranayah[26][180]
\end{Arabic}}
\flushleft{\begin{hindi}
मै तो तुमसे इस (तबलीग़े रिसालत) पर कुछ मज़दूरी भी नहीं माँगता मेरी मज़दूरी तो बस सारी ख़ुदाई के पालने वाले (ख़ुदा) के ज़िम्मे है
\end{hindi}}
\flushright{\begin{Arabic}
\quranayah[26][181]
\end{Arabic}}
\flushleft{\begin{hindi}
तुम (जब कोई चीज़ नाप कर दो तो) पूरा पैमाना दिया करो और नुक़सान (कम देने वाले) न बनो
\end{hindi}}
\flushright{\begin{Arabic}
\quranayah[26][182]
\end{Arabic}}
\flushleft{\begin{hindi}
और तुम (जब तौलो तो) ठीक तराज़ू से डन्डी सीधी रखकर तौलो
\end{hindi}}
\flushright{\begin{Arabic}
\quranayah[26][183]
\end{Arabic}}
\flushleft{\begin{hindi}
और लोगों को उनकी चीज़े (जो ख़रीदें) कम न ज्यादा करो और ज़मीन से फसाद न फैलाते फिरो
\end{hindi}}
\flushright{\begin{Arabic}
\quranayah[26][184]
\end{Arabic}}
\flushleft{\begin{hindi}
और उस (ख़ुदा) से डरो जिसने तुम्हे और अगली ख़िलकत को पैदा किया
\end{hindi}}
\flushright{\begin{Arabic}
\quranayah[26][185]
\end{Arabic}}
\flushleft{\begin{hindi}
वह लोग कहने लगे तुम पर तो बस जादू कर दिया गया है (कि ऐसी बातें करते हों)
\end{hindi}}
\flushright{\begin{Arabic}
\quranayah[26][186]
\end{Arabic}}
\flushleft{\begin{hindi}
और तुम तो हमारे ही ऐसे एक आदमी हो और हम लोग तो तुमको झूठा ही समझते हैं
\end{hindi}}
\flushright{\begin{Arabic}
\quranayah[26][187]
\end{Arabic}}
\flushleft{\begin{hindi}
तो अगर तुम सच्चे हो तो हम पर आसमान का एक टुकड़ा गिरा दो
\end{hindi}}
\flushright{\begin{Arabic}
\quranayah[26][188]
\end{Arabic}}
\flushleft{\begin{hindi}
और शुएब ने कहा जो तुम लोग करते हो मेरा परवरदिगार ख़ूब जानता है
\end{hindi}}
\flushright{\begin{Arabic}
\quranayah[26][189]
\end{Arabic}}
\flushleft{\begin{hindi}
ग़रज़ उन लोगों ने शुएब को झुठलाया तो उन्हें साएबान (अब्र) के अज़ाब ने ले डाला- इसमे शक नहीं कि ये भी एक बड़े (सख्त) दिन का अज़ाब था
\end{hindi}}
\flushright{\begin{Arabic}
\quranayah[26][190]
\end{Arabic}}
\flushleft{\begin{hindi}
इसमे भी शक नहीं कि इसमें (समझदारों के लिए) एक बड़ी इबरत है और उनमें के बहुतेरे ईमान लाने वाले ही न थे
\end{hindi}}
\flushright{\begin{Arabic}
\quranayah[26][191]
\end{Arabic}}
\flushleft{\begin{hindi}
और बेशक तुम्हारा परवरदिगार यक़ीनन (सब पर) ग़ालिब (और) बड़ा मेहरबान है
\end{hindi}}
\flushright{\begin{Arabic}
\quranayah[26][192]
\end{Arabic}}
\flushleft{\begin{hindi}
और (ऐ रसूल) बेशक ये (क़ुरान) सारी ख़ुदायी के पालने वाले (ख़ुदा) का उतारा हुआ है
\end{hindi}}
\flushright{\begin{Arabic}
\quranayah[26][193]
\end{Arabic}}
\flushleft{\begin{hindi}
जिसे रुहुल अमीन (जिबरील) साफ़ अरबी ज़बान में लेकर तुम्हारे दिल पर नाज़िल हुए है
\end{hindi}}
\flushright{\begin{Arabic}
\quranayah[26][194]
\end{Arabic}}
\flushleft{\begin{hindi}
ताकि तुम भी और पैग़म्बरों की तरह
\end{hindi}}
\flushright{\begin{Arabic}
\quranayah[26][195]
\end{Arabic}}
\flushleft{\begin{hindi}
लोगों को अज़ाबे ख़ुदा से डराओ
\end{hindi}}
\flushright{\begin{Arabic}
\quranayah[26][196]
\end{Arabic}}
\flushleft{\begin{hindi}
और बेशक इसकी ख़बर अगले पैग़म्बरों की किताबों मे (भी मौजूद) है
\end{hindi}}
\flushright{\begin{Arabic}
\quranayah[26][197]
\end{Arabic}}
\flushleft{\begin{hindi}
क्या उनके लिए ये कोई (काफ़ी) निशानी नहीं है कि इसको उलेमा बनी इसराइल जानते हैं
\end{hindi}}
\flushright{\begin{Arabic}
\quranayah[26][198]
\end{Arabic}}
\flushleft{\begin{hindi}
और अगर हम इस क़ुरान को किसी दूसरी ज़बान वाले पर नाज़िल करते
\end{hindi}}
\flushright{\begin{Arabic}
\quranayah[26][199]
\end{Arabic}}
\flushleft{\begin{hindi}
और वह उन अरबो के सामने उसको पढ़ता तो भी ये लोग उस पर ईमान लाने वाले न थे
\end{hindi}}
\flushright{\begin{Arabic}
\quranayah[26][200]
\end{Arabic}}
\flushleft{\begin{hindi}
इसी तरह हमने (गोया ख़ुद) इस इन्कार को गुनाहगारों के दिलों में राह दी
\end{hindi}}
\flushright{\begin{Arabic}
\quranayah[26][201]
\end{Arabic}}
\flushleft{\begin{hindi}
ये लोग जब तक दर्दनाक अज़ाब को न देख लेगें उस पर ईमान न लाएँगे
\end{hindi}}
\flushright{\begin{Arabic}
\quranayah[26][202]
\end{Arabic}}
\flushleft{\begin{hindi}
कि वह यकायक इस हालत में उन पर आ पडेग़ा कि उन्हें ख़बर भी न होगी
\end{hindi}}
\flushright{\begin{Arabic}
\quranayah[26][203]
\end{Arabic}}
\flushleft{\begin{hindi}
(मगर जब अज़ाब नाज़िल होगा) तो वह लोग कहेंगे कि क्या हमें (इस वक्त क़ुछ) मोहलत मिल सकती है
\end{hindi}}
\flushright{\begin{Arabic}
\quranayah[26][204]
\end{Arabic}}
\flushleft{\begin{hindi}
तो क्या ये लोग हमारे अज़ाब की जल्दी कर रहे हैं
\end{hindi}}
\flushright{\begin{Arabic}
\quranayah[26][205]
\end{Arabic}}
\flushleft{\begin{hindi}
तो क्या तुमने ग़ौर किया कि अगर हम उनको सालो साल चैन करने दे
\end{hindi}}
\flushright{\begin{Arabic}
\quranayah[26][206]
\end{Arabic}}
\flushleft{\begin{hindi}
उसके बाद जिस (अज़ाब) का उनसे वायदा किया जाता है उनके पास आ पहुँचे
\end{hindi}}
\flushright{\begin{Arabic}
\quranayah[26][207]
\end{Arabic}}
\flushleft{\begin{hindi}
तो जिन चीज़ों से ये लोग चैन किया करते थे कुछ भी काम न आएँगी
\end{hindi}}
\flushright{\begin{Arabic}
\quranayah[26][208]
\end{Arabic}}
\flushleft{\begin{hindi}
और हमने किसी बस्ती को बग़ैर उसके हलाक़ नहीं किया कि उसके समझाने को (पहले से) डराने वाले (पैग़म्बर भेज दिए) थे
\end{hindi}}
\flushright{\begin{Arabic}
\quranayah[26][209]
\end{Arabic}}
\flushleft{\begin{hindi}
और हम ज़ालिम नहीं है
\end{hindi}}
\flushright{\begin{Arabic}
\quranayah[26][210]
\end{Arabic}}
\flushleft{\begin{hindi}
और इस क़ुरान को शयातीन लेकर नाज़िल नही हुए
\end{hindi}}
\flushright{\begin{Arabic}
\quranayah[26][211]
\end{Arabic}}
\flushleft{\begin{hindi}
और ये काम न तो उनके लिए मुनासिब था और न वह कर सकते थे
\end{hindi}}
\flushright{\begin{Arabic}
\quranayah[26][212]
\end{Arabic}}
\flushleft{\begin{hindi}
बल्कि वह तो (वही के) सुनने से महरुम हैं
\end{hindi}}
\flushright{\begin{Arabic}
\quranayah[26][213]
\end{Arabic}}
\flushleft{\begin{hindi}
(ऐ रसूल) तुम ख़ुदा के साथ किसी दूसरे माबूद की इबादत न करो वरना तुम भी मुबतिलाए अज़ाब किए जाओगे
\end{hindi}}
\flushright{\begin{Arabic}
\quranayah[26][214]
\end{Arabic}}
\flushleft{\begin{hindi}
और (ऐ रसूल) तुम अपने क़रीबी रिश्तेदारों को (अज़ाबे ख़ुदा से) डराओ
\end{hindi}}
\flushright{\begin{Arabic}
\quranayah[26][215]
\end{Arabic}}
\flushleft{\begin{hindi}
और जो मोमिनीन तुम्हारे पैरो हो गए हैं उनके सामने अपना बाजू झुकाओ
\end{hindi}}
\flushright{\begin{Arabic}
\quranayah[26][216]
\end{Arabic}}
\flushleft{\begin{hindi}
(तो वाज़ेए करो) पस अगर लोग तुम्हारी नाफ़रमानी करें तो तुम (साफ साफ) कह दो कि मैं तुम्हारे करतूतों से बरी उज़ ज़िम्मा हूँ
\end{hindi}}
\flushright{\begin{Arabic}
\quranayah[26][217]
\end{Arabic}}
\flushleft{\begin{hindi}
और तुम उस (ख़ुदा) पर जो सबसे (ग़ालिब और) मेहरबान है
\end{hindi}}
\flushright{\begin{Arabic}
\quranayah[26][218]
\end{Arabic}}
\flushleft{\begin{hindi}
भरोसा रखो कि जब तुम (नमाजे तहज्जुद में) खड़े होते हो
\end{hindi}}
\flushright{\begin{Arabic}
\quranayah[26][219]
\end{Arabic}}
\flushleft{\begin{hindi}
और सजदा
\end{hindi}}
\flushright{\begin{Arabic}
\quranayah[26][220]
\end{Arabic}}
\flushleft{\begin{hindi}
करने वालों (की जमाअत) में तुम्हारा फिरना (उठना बैठना सजदा रुकूउ वगैरह सब) देखता है
\end{hindi}}
\flushright{\begin{Arabic}
\quranayah[26][221]
\end{Arabic}}
\flushleft{\begin{hindi}
बेशक वह बड़ा सुनने वाला वाक़िफ़कार है क्या मै तुम्हें बता दूँ कि शयातीन किन लोगों पर नाज़िल हुआ करते हैं
\end{hindi}}
\flushright{\begin{Arabic}
\quranayah[26][222]
\end{Arabic}}
\flushleft{\begin{hindi}
(लो सुनो) ये लोग झूठे बद किरदार पर नाज़िल हुआ करते हैं
\end{hindi}}
\flushright{\begin{Arabic}
\quranayah[26][223]
\end{Arabic}}
\flushleft{\begin{hindi}
जो (फ़रिश्तों की बातों पर कान लगाए रहते हैं) कि कुछ सुन पाएँ
\end{hindi}}
\flushright{\begin{Arabic}
\quranayah[26][224]
\end{Arabic}}
\flushleft{\begin{hindi}
हालाँकि उनमें के अक्सर तो (बिल्कुल) झूठे हैं और शायरों की पैरवी तो गुमराह लोग किया करते हैं
\end{hindi}}
\flushright{\begin{Arabic}
\quranayah[26][225]
\end{Arabic}}
\flushleft{\begin{hindi}
क्या तुम नहीं देखते कि ये लोग जंगल जंगल सरगिरदॉ मारे मारे फिरते हैं
\end{hindi}}
\flushright{\begin{Arabic}
\quranayah[26][226]
\end{Arabic}}
\flushleft{\begin{hindi}
और ये लोग ऐसी बाते कहते हैं जो कभी करते नहीं
\end{hindi}}
\flushright{\begin{Arabic}
\quranayah[26][227]
\end{Arabic}}
\flushleft{\begin{hindi}
मगर (हाँ) जिन लोगों ने ईमान क़ुबूल किया और अच्छे अच्छे काम किए और क़सरत से ख़ुदा का ज़िक्र किया करते हैं और जब उन पर ज़ुल्म किया जा चुका उसके बाद उन्होंनें बदला लिया और जिन लोगों ने ज़ुल्म किया है उन्हें अनक़रीब ही मालूम हो जाएगा कि वह किस जगह लौटाए जाएँगें
\end{hindi}}
\chapter{An-Naml (The Naml)}
\begin{Arabic}
\Huge{\centerline{\basmalah}}\end{Arabic}
\flushright{\begin{Arabic}
\quranayah[27][1]
\end{Arabic}}
\flushleft{\begin{hindi}
ता सीन ये क़ुरान वाजेए व रौशन किताब की आयतें है
\end{hindi}}
\flushright{\begin{Arabic}
\quranayah[27][2]
\end{Arabic}}
\flushleft{\begin{hindi}
(ये) उन ईमानदारों के लिए (अज़सरतापा) हिदायत और (जन्नत की) ख़ुशखबरी है
\end{hindi}}
\flushright{\begin{Arabic}
\quranayah[27][3]
\end{Arabic}}
\flushleft{\begin{hindi}
जो नमाज़ को पाबन्दी से अदा करते हैं और ज़कात दिया करते हैं और यही लोग आख़िरत (क़यामत) का भी यक़ीन रखते हैं
\end{hindi}}
\flushright{\begin{Arabic}
\quranayah[27][4]
\end{Arabic}}
\flushleft{\begin{hindi}
इसमें शक नहीं कि जो लोग आखिरत पर ईमान नहीं रखते (गोया) हमने ख़ुद (उनकी कारस्तानियों को उनकी नज़र में) अच्छा कर दिखाया है
\end{hindi}}
\flushright{\begin{Arabic}
\quranayah[27][5]
\end{Arabic}}
\flushleft{\begin{hindi}
तो ये लोग भटकते फिरते हैं- यही वह लोग हैं जिनके लिए (क़यामत में) बड़ा अज़ाब है और यही लोग आख़िरत में सबसे ज्यादा घाटा उठाने वाले हैं
\end{hindi}}
\flushright{\begin{Arabic}
\quranayah[27][6]
\end{Arabic}}
\flushleft{\begin{hindi}
और (ऐ रसूल) तुमको तो क़ुरान एक बडे वाक़िफकार हकीम की बारगाह से अता किया जाता है
\end{hindi}}
\flushright{\begin{Arabic}
\quranayah[27][7]
\end{Arabic}}
\flushleft{\begin{hindi}
(वह वाक़िया याद दिलाओ) जब मूसा ने अपने लड़के बालों से कहा कि मैने (अपनी बायीं तरफ) आग देखी है (एक ज़रा ठहरो तो) मै वहाँ से कुछ (राह की) ख़बर लाँऊ या तुम्हें एक सुलगता हुआ आग का अंगारा ला दूँ ताकि तुम तापो
\end{hindi}}
\flushright{\begin{Arabic}
\quranayah[27][8]
\end{Arabic}}
\flushleft{\begin{hindi}
ग़रज़ जब मूसा इस आग के पास आए तो उनको आवाज़ आयी कि मुबारक है वह जो आग में (तजल्ली दिखाना) है और जो उसके गिर्द है और वह ख़ुदा सारे जहाँ का पालने वाला है
\end{hindi}}
\flushright{\begin{Arabic}
\quranayah[27][9]
\end{Arabic}}
\flushleft{\begin{hindi}
(हर ऐब से) पाक व पाकीज़ा है- ऐ मूसा इसमें शक नहीं कि मै ज़बरदस्त हिकमत वाला हूँ
\end{hindi}}
\flushright{\begin{Arabic}
\quranayah[27][10]
\end{Arabic}}
\flushleft{\begin{hindi}
और (हाँ) अपनी छड़ी तो (ज़मीन पर) डाल दो तो जब मूसा ने उसको देखा कि वह इस तरह लहरा रही है गोया वह जिन्दा अज़दहा है तो पिछले पावँ भाग चले और पीछे मुड़कर भी न देखा (तो हमने कहा) ऐ मूसा डरो नहीं हमारे पास पैग़म्बर लोग डरा नहीं करते हैं
\end{hindi}}
\flushright{\begin{Arabic}
\quranayah[27][11]
\end{Arabic}}
\flushleft{\begin{hindi}
(मुतमइन हो जाते है) मगर जो शख्स गुनाह करे फिर गुनाह के बाद उसे नेकी (तौबा) से बदल दे तो अलबत्ता बड़ा बख्शने वाला मेहरबान हूँ
\end{hindi}}
\flushright{\begin{Arabic}
\quranayah[27][12]
\end{Arabic}}
\flushleft{\begin{hindi}
(वहाँ) और अपना हाथ अपने गरेबॉ में तो डालो कि वह सफेद बुर्राक़ होकर बेऐब निकल आएगा (ये वह मौजिज़े) मिन जुमला नौ मोजिज़ात के हैं जो तुमको मिलेगें तुम फिरऔन और उसकी क़ौम के पास (जाओ) क्योंकि वह बदकिरदार लोग हैं
\end{hindi}}
\flushright{\begin{Arabic}
\quranayah[27][13]
\end{Arabic}}
\flushleft{\begin{hindi}
तो जब उनके पास हमारे ऑंखें खोल देने वाले मैजिज़े आए तो कहने लगे ये तो खुला हुआ जादू है
\end{hindi}}
\flushright{\begin{Arabic}
\quranayah[27][14]
\end{Arabic}}
\flushleft{\begin{hindi}
और बावजूद के उनके दिल को उन मौजिज़ात का यक़ीन था मगर फिर भी उन लोगों ने सरकशी और तकब्बुर से उनको न माना तो (ऐ रसूल) देखो कि (आखिर) मुफसिदों का अन्जाम क्या होगा
\end{hindi}}
\flushright{\begin{Arabic}
\quranayah[27][15]
\end{Arabic}}
\flushleft{\begin{hindi}
और इसमें शक नहीं कि हमने दाऊद और सुलेमान को इल्म अता किया और दोनों ने (ख़ुश होकर) कहा ख़ुदा का शुक्र जिसने हमको अपने बहुतेरे ईमानदार बन्दों पर फज़ीलत दी
\end{hindi}}
\flushright{\begin{Arabic}
\quranayah[27][16]
\end{Arabic}}
\flushleft{\begin{hindi}
और (इल्म हिकमत जाएदाद (मनकूला) गैर मनकूला सब में) सुलेमान दाऊद के वारिस हुए और कहा कि लोग हम को (ख़ुदा के फज़ल से) परिन्दों की बोली भी सिखायी गयी है और हमें (दुनिया की) हर चीज़ अता की गयी है इसमें शक नहीं कि ये यक़ीनी (ख़ुदा का) सरीही फज़ल व करम है
\end{hindi}}
\flushright{\begin{Arabic}
\quranayah[27][17]
\end{Arabic}}
\flushleft{\begin{hindi}
और सुलेमान के सामने उनके लशकर जिन्नात और आदमी और परिन्दे सब जमा किए जाते थे
\end{hindi}}
\flushright{\begin{Arabic}
\quranayah[27][18]
\end{Arabic}}
\flushleft{\begin{hindi}
तो वह सबके सब (मसल मसल) खडे क़िए जाते थे (ग़रज़ इस तरह लशकर चलता) यहाँ तक कि जब (एक दिन) चीटीयों के मैदान में आ निकले तो एक चीटीं बोली ऐ चीटीयों अपने अपने बिल में घुस जाओ- ऐसा न हो कि सुलेमान और उनका लश्कर तुम्हे रौन्द डाले और उन्हें उसकी ख़बर भी न हो
\end{hindi}}
\flushright{\begin{Arabic}
\quranayah[27][19]
\end{Arabic}}
\flushleft{\begin{hindi}
तो सुलेमान इस बात से मुस्कुरा के हँस पड़ें और अर्ज क़ी परवरदिगार मुझे तौफीक़ अता फरमा कि जैसी जैसी नेअमतें तूने मुझ पर और मेरे वालदैन पर नाज़िल फरमाई हैं मै (उनका) शुक्रिया अदा करुँ और मैं ऐसे नेक काम करुँ जिसे तू पसन्द फरमाए और तू अपनी ख़ास मेहरबानी से मुझे (अपने) नेकोकार बन्दों में दाखिल कर
\end{hindi}}
\flushright{\begin{Arabic}
\quranayah[27][20]
\end{Arabic}}
\flushleft{\begin{hindi}
और सुलेमान ने परिन्दों (के लश्कर) की हाज़िरी ली तो कहने लगे कि क्या बात है कि मै हुदहुद को (उसकी जगह पर) नहीं देखता क्या (वाक़ई में) वह कही ग़ायब है
\end{hindi}}
\flushright{\begin{Arabic}
\quranayah[27][21]
\end{Arabic}}
\flushleft{\begin{hindi}
(अगर ऐसा है तो) मै उसे सख्त से सख्त सज़ा दूँगा या (नहीं तो ) उसे ज़बाह ही कर डालूँगा या वह (अपनी बेगुनाही की) कोई साफ दलील मेरे पास पेश करे
\end{hindi}}
\flushright{\begin{Arabic}
\quranayah[27][22]
\end{Arabic}}
\flushleft{\begin{hindi}
ग़रज़ सुलेमान ने थोड़ी ही देर (तवक्कुफ़ किया था कि (हुदहुद) आ गया) तो उसने अर्ज़ की मुझे यह बात मालूम हुई है जो अब तक हुज़ूर को भी मालूम नहीं है और आप के पास शहरे सबा से एक तहक़ीकी ख़बर लेकर आया हूँ
\end{hindi}}
\flushright{\begin{Arabic}
\quranayah[27][23]
\end{Arabic}}
\flushleft{\begin{hindi}
मैने एक औरत को देखा जो वहाँ के लोगों पर सलतनत करती है और उसे (दुनिया की) हर चीज़ अता की गयी है और उसका एक बड़ा तख्त है
\end{hindi}}
\flushright{\begin{Arabic}
\quranayah[27][24]
\end{Arabic}}
\flushleft{\begin{hindi}
मैने खुद मलका को देखा और उसकी क़ौम को देखा कि वह लोग ख़ुदा को छोड़कर आफताब को सजदा करते हैं शैतान ने उनकी करतूतों को (उनकी नज़र में) अच्छा कर दिखाया है और उनको राहे रास्त से रोक रखा है
\end{hindi}}
\flushright{\begin{Arabic}
\quranayah[27][25]
\end{Arabic}}
\flushleft{\begin{hindi}
तो उन्हें (इतनी सी बात भी नहीं सूझती) कि वह लोग ख़ुदा ही का सजदा क्यों नहीं करते जो आसमान और ज़मीन की पोशीदा बातों को ज़ाहिर कर देता है और तुम लोग जो कुछ छिपाकर या ज़ाहिर करके करते हो सब जानता है
\end{hindi}}
\flushright{\begin{Arabic}
\quranayah[27][26]
\end{Arabic}}
\flushleft{\begin{hindi}
अल्लाह वह है जिससे सिवा कोई माबूद नहीं वही (इतने) बड़े अर्श का मालिक है (सजदा)
\end{hindi}}
\flushright{\begin{Arabic}
\quranayah[27][27]
\end{Arabic}}
\flushleft{\begin{hindi}
(ग़रज़) सुलेमान ने कहा हम अभी देखते हैं कि तूने सच सच कहा या तू झूठा है
\end{hindi}}
\flushright{\begin{Arabic}
\quranayah[27][28]
\end{Arabic}}
\flushleft{\begin{hindi}
(अच्छा) हमारा ये ख़त लेकर जा और उसको उन लोगों के सामने डाल दे फिर उन के पास से जाना फिर देखते रहना कि वह लोग अख़िर क्या जवाब देते हैं
\end{hindi}}
\flushright{\begin{Arabic}
\quranayah[27][29]
\end{Arabic}}
\flushleft{\begin{hindi}
(ग़रज़) हुद हुद ने मलका के पास ख़त पहुँचा दिया तो मलका बोली ऐ (मेरे दरबार के) सरदारों ये एक वाजिबुल एहतराम ख़त मेरे पास डाल दिया गया है
\end{hindi}}
\flushright{\begin{Arabic}
\quranayah[27][30]
\end{Arabic}}
\flushleft{\begin{hindi}
सुलेमान की तरफ से है (ये उसका सरनामा) है बिस्मिल्लाहिररहमानिरहीम
\end{hindi}}
\flushright{\begin{Arabic}
\quranayah[27][31]
\end{Arabic}}
\flushleft{\begin{hindi}
(और मज़मून) यह है कि मुझ से सरकशी न करो और मेरे सामने फरमाबरदार बन कर हाज़िर हो
\end{hindi}}
\flushright{\begin{Arabic}
\quranayah[27][32]
\end{Arabic}}
\flushleft{\begin{hindi}
तब मलका (विलक़ीस) बोली ऐ मेरे दरबार के सरदारों तुम मेरे इस मामले में मुझे राय दो (क्योंकि मेरा तो ये क़ायदा है कि) जब तक तुम लोग मेरे सामने मौजूद न हो (मशवरा न दे दो) मैं किसी अम्र में क़तई फैसला न किया करती
\end{hindi}}
\flushright{\begin{Arabic}
\quranayah[27][33]
\end{Arabic}}
\flushleft{\begin{hindi}
उन लोगों ने अर्ज़ की हम बड़े ज़ोरावर बडे लड़ने वाले हैं और (आइन्दा) हर अम्र का आप को एख्तियार है तो जो हुक्म दे आप (खुद अच्छी) तरह इसके अन्जाम पर ग़ौर कर ले
\end{hindi}}
\flushright{\begin{Arabic}
\quranayah[27][34]
\end{Arabic}}
\flushleft{\begin{hindi}
मलका ने कहा बादशाहों का क़ायदा है कि जब किसी बस्ती में (बज़ोरे फ़तेह) दाख़िल हो जाते हैं तो उसको उजाड़ देते हैं और वहाँ के मुअज़िज़ लोगों को ज़लील व रुसवा कर देते हैं और ये लोग भी ऐसा ही करेंगे
\end{hindi}}
\flushright{\begin{Arabic}
\quranayah[27][35]
\end{Arabic}}
\flushleft{\begin{hindi}
और मैं उनके पास (एलचियों की माअरफ़त कुछ तोहफा भेजकर देखती हूँ कि एलची लोग क्या जवाब लाते हैं) ग़रज़ जब बिलक़ीस का एलची (तोहफा लेकर) सुलेमान के पास आया
\end{hindi}}
\flushright{\begin{Arabic}
\quranayah[27][36]
\end{Arabic}}
\flushleft{\begin{hindi}
तो सुलेमान ने कहा क्या तुम लोग मुझे माल की मदद देते हो तो ख़ुदा ने जो (माल दुनिया) मुझे अता किया है वह (माल) उससे जो तुम्हें बख्शा है कहीं बेहतर है (मैं तो नही) बल्कि तुम्ही लोग अपने तोहफे तहायफ़ से ख़ुश हुआ करो
\end{hindi}}
\flushright{\begin{Arabic}
\quranayah[27][37]
\end{Arabic}}
\flushleft{\begin{hindi}
(फिर तोहफा लाने वाले ने कहा) तो उन्हीं लोगों के पास जा हम यक़ीनन ऐसे लश्कर से उन पर चढ़ाई करेंगे जिसका उससे मुक़ाबला न हो सकेगा और हम ज़रुर उन्हें वहाँ से ज़लील व रुसवा करके निकाल बाहर करेंगे
\end{hindi}}
\flushright{\begin{Arabic}
\quranayah[27][38]
\end{Arabic}}
\flushleft{\begin{hindi}
(जब वह जा चुका) तो सुलेमान ने अपने अहले दरबार से कहा ऐ मेरे दरबार के सरदारो तुममें से कौन ऐसा है कि क़ब्ल इसके वह लोग मेरे सामने फरमाबरदार बनकर आयें
\end{hindi}}
\flushright{\begin{Arabic}
\quranayah[27][39]
\end{Arabic}}
\flushleft{\begin{hindi}
मलिका का तख्त मेरे पास ले आए (इस पर) जिनों में से एक दियो बोल उठा कि क़ब्ल इसके कि हुज़ूर (दरबार बरख़ास्त करके) अपनी जगह से उठे मै तख्त आपके पास ले आऊँगा और यक़ीनन उस पर क़ाबू रखता हूँ (और) ज़िम्मेदार हूँ
\end{hindi}}
\flushright{\begin{Arabic}
\quranayah[27][40]
\end{Arabic}}
\flushleft{\begin{hindi}
इस पर अभी सुलेमान कुछ कहने न पाए थे कि वह शख्स (आसिफ़ बिन बरख़िया) जिसके पास किताबे (ख़ुदा) का किस कदर इल्म था बोला कि मै आप की पलक झपकने से पहले तख्त को आप के पास हाज़िर किए देता हूँ (बस इतने ही में आ गया) तो जब सुलेमान ने उसे अपने पास मौजूद पाया तो कहने लगे ये महज़ मेरे परवरदिगार का फज़ल व करम है ताकि वह मेरा इम्तेहान ले कि मै उसका शुक्र करता हूँ या नाशुक्री करता हूँ और जो कोई शुक्र करता है वह अपनी ही भलाई के लिए शुक्र करता है और जो शख्स ना शुक्री करता है तो (याद रखिए) मेरा परवरदिगार यक़ीनन बेपरवा और सख़ी है
\end{hindi}}
\flushright{\begin{Arabic}
\quranayah[27][41]
\end{Arabic}}
\flushleft{\begin{hindi}
(उसके बाद) सुलेमान ने कहा कि उसके तख्त में (उसकी अक्ल के इम्तिहान के लिए) तग़य्युर तबददुल कर दो ताकि हम देखें कि फिर भी वह समझ रखती है या उन लोगों में है जो कुछ समझ नहीं रखते
\end{hindi}}
\flushright{\begin{Arabic}
\quranayah[27][42]
\end{Arabic}}
\flushleft{\begin{hindi}
(चुनान्चे ऐसा ही किया गया) फिर जब बिलक़ीस (सुलेमान के पास) आयी तो पूछा गया कि तुम्हारा तख्त भी ऐसा ही है वह बोली गोया ये वही है (फिर कहने लगी) हमको तो उससे पहले ही (आपकी नुबूवत) मालूम हो गयी थी और हम तो आपके फ़रमाबरदार थे ही
\end{hindi}}
\flushright{\begin{Arabic}
\quranayah[27][43]
\end{Arabic}}
\flushleft{\begin{hindi}
और ख़ुदा के सिवा जिसे वह पूजती थी सुलेमान ने उससे उसे रोक दिया क्योंकि वह काफिर क़ौम की थी (और आफताब को पूजती थी)
\end{hindi}}
\flushright{\begin{Arabic}
\quranayah[27][44]
\end{Arabic}}
\flushleft{\begin{hindi}
फिर उससे कहा गया कि आप अब महल मे चलिए तो जब उसने महल (में शीशे के फर्श) को देखा तो उसको गहरा पानी समझी (और गुज़रने के लिए इस तरह अपने पाएचे उठा लिए कि) अपनी दोनों पिन्डलियाँ खोल दी सुलेमान ने कहा (तुम डरो नहीं) ये (पानी नहीं है) महल है जो शीशे से मढ़ा हुआ है (उस वक्त तम्बीह हुई और) अर्ज़ की परवरदिगार मैने (आफताब को पूजा कर) यक़ीनन अपने ऊपर ज़ुल्म किया
\end{hindi}}
\flushright{\begin{Arabic}
\quranayah[27][45]
\end{Arabic}}
\flushleft{\begin{hindi}
और अब मैं सुलेमान के साथ सारे जहाँ के पालने वाले खुदा पर ईमान लाती हूँ और हम ही ने क़ौम समूद के पास उनके भाई सालेह को पैग़म्बर बनाकर भेजा कि तुम लोग ख़ुदा की इबादत करो तो वह सालेह के आते ही (मोमिन व काफिर) दो फरीक़ बनकर बाहम झगड़ने लगे
\end{hindi}}
\flushright{\begin{Arabic}
\quranayah[27][46]
\end{Arabic}}
\flushleft{\begin{hindi}
सालेह ने कहा ऐ मेरी क़ौम (आख़िर) तुम लोग भलाई से पहल बुराई के वास्ते जल्दी क्यों कर रहे हो तुम लोग ख़ुदा की बारगाह में तौबा व अस्तग़फार क्यों नही करते ताकि तुम पर रहम किया जाए
\end{hindi}}
\flushright{\begin{Arabic}
\quranayah[27][47]
\end{Arabic}}
\flushleft{\begin{hindi}
वह लोग बोले हमने तो तुम से और तुम्हारे साथियों से बुरा शगुन पाया सालेह ने कहा तुम्हारी बदकिस्मती ख़ुदा के पास है (ये सब कुछ नहीं) बल्कि तुम लोगों की आज़माइश की जा रही है
\end{hindi}}
\flushright{\begin{Arabic}
\quranayah[27][48]
\end{Arabic}}
\flushleft{\begin{hindi}
और शहर में नौ आदमी थे जो मुल्क के बानीये फसाद थे और इसलाह की फिक्र न करते थे-उन लोगों ने (आपस में) कहा कि बाहम ख़ुदा की क़सम खाते जाओ
\end{hindi}}
\flushright{\begin{Arabic}
\quranayah[27][49]
\end{Arabic}}
\flushleft{\begin{hindi}
कि हम लोग सालेह और उसके लड़के बालो पर शब खून करे उसके बाद उसके वाली वारिस से कह देगें कि हम लोग उनके घर वालों को हलाक़ होते वक्त मौजूद ही न थे और हम लोग तो यक़ीनन सच्चे हैं
\end{hindi}}
\flushright{\begin{Arabic}
\quranayah[27][50]
\end{Arabic}}
\flushleft{\begin{hindi}
और उन लोगों ने एक तदबीर की और हमने भी एक तदबीर की और (हमारी तदबीर की) उनको ख़बर भी न हुई
\end{hindi}}
\flushright{\begin{Arabic}
\quranayah[27][51]
\end{Arabic}}
\flushleft{\begin{hindi}
तो (ऐ रसूल) तुम देखो उनकी तदबीर का क्या (बुरा) अन्जाम हुआ कि हमने उनको और सारी क़ौम को हलाक कर डाला
\end{hindi}}
\flushright{\begin{Arabic}
\quranayah[27][52]
\end{Arabic}}
\flushleft{\begin{hindi}
ये बस उनके घर हैं कि उनकी नाफरमानियों की वज़ह से ख़ाली वीरान पड़े हैं इसमे शक नही कि उस वाक़िये में वाक़िफ कार लोगों के लिए बड़ी इबरत है
\end{hindi}}
\flushright{\begin{Arabic}
\quranayah[27][53]
\end{Arabic}}
\flushleft{\begin{hindi}
और हमने उन लोगों को जो ईमान लाए थे और परहेज़गार थे बचा लिया
\end{hindi}}
\flushright{\begin{Arabic}
\quranayah[27][54]
\end{Arabic}}
\flushleft{\begin{hindi}
और (ऐ रसूल) लूत को (याद करो) जब उन्होंने अपनी क़ौम से कहा कि क्या तुम देखभाल कर (समझ बूझ कर) ऐसी बेहयाई करते हो
\end{hindi}}
\flushright{\begin{Arabic}
\quranayah[27][55]
\end{Arabic}}
\flushleft{\begin{hindi}
क्या तुम औरतों को छोड़कर शहवत से मर्दों के आते हो (ये तुम अच्छा नहीं करते) बल्कि तुम लोग बड़ी जाहिल क़ौम हो तो लूत की क़ौम का इसके सिवा कुछ जवाब न था
\end{hindi}}
\flushright{\begin{Arabic}
\quranayah[27][56]
\end{Arabic}}
\flushleft{\begin{hindi}
कि वह लोग बोल उठे कि लूत के खानदान को अपनी बस्ती (सदूम) से निकाल बाहर करो ये लोग बड़े पाक साफ बनना चाहते हैं
\end{hindi}}
\flushright{\begin{Arabic}
\quranayah[27][57]
\end{Arabic}}
\flushleft{\begin{hindi}
ग़रज हमने लूत को और उनके ख़ानदान को बचा लिया मगर उनकी बीवी कि हमने उसकी तक़दीर में पीछे रह जाने वालों में लिख दिया था
\end{hindi}}
\flushright{\begin{Arabic}
\quranayah[27][58]
\end{Arabic}}
\flushleft{\begin{hindi}
और (फिर तो) हमने उन लोगों पर (पत्थर का) मेंह बरसाया तो जो लोग डराए जा चुके थे उन पर क्या बुरा मेंह बरसा
\end{hindi}}
\flushright{\begin{Arabic}
\quranayah[27][59]
\end{Arabic}}
\flushleft{\begin{hindi}
(ऐ रसूल) तुम कह दो (उनके हलाक़ होने पर) खुदा का शुक्र और उसके बरगुज़ीदा बन्दों पर सलाम भला ख़ुदा बेहतर है या वह चीज़ जिसे ये लोग शरीके ख़ुदा कहते हैं
\end{hindi}}
\flushright{\begin{Arabic}
\quranayah[27][60]
\end{Arabic}}
\flushleft{\begin{hindi}
भला वह कौन है जिसने आसमान और ज़मीन को पैदा किया और तुम्हारे वास्ते आसमान से पानी बरसाया फिर हम ही ने पानी से दिल चस्प (ख़ुशनुमा) बाग़ उठाए तुम्हारे तो ये बस की बात न थी कि तुम उनके दरख्तों को उगा सकते तो क्या ख़ुदा के साथ कोई और माबूद भी है (हरगिज़ नहीं) बल्कि ये लोग खुद अपने जी से गढ़ के बुतो को उसके बराबर बनाते हैं
\end{hindi}}
\flushright{\begin{Arabic}
\quranayah[27][61]
\end{Arabic}}
\flushleft{\begin{hindi}
भला वह कौन है जिसने ज़मीन को (लोगों के) ठहरने की जगह बनाया और उसके दरमियान जा बजा नहरें दौड़ायी और उसकी मज़बूती के वास्ते पहाड़ बनाए और (मीठे खारी) दरियाओं के दरमियान हदे फासिल बनाया तो क्या ख़ुदा के साथ कोई और माबूद भी है (हरगिज़ नहीं) बल्कि उनमें के अकसर कुछ जानते ही नहीं
\end{hindi}}
\flushright{\begin{Arabic}
\quranayah[27][62]
\end{Arabic}}
\flushleft{\begin{hindi}
भला वह कौन है कि जब मुज़तर उसे पुकारे तो दुआ क़ुबूल करता है और मुसीबत को दूर करता है और तुम लोगों को ज़मीन में (अपना) नायब बनाता है तो क्या ख़ुदा के साथ कोई और माबूद है (हरगिज़ नहीं) उस पर भी तुम लोग बहुत कम नसीहत व इबरत हासिल करते हो
\end{hindi}}
\flushright{\begin{Arabic}
\quranayah[27][63]
\end{Arabic}}
\flushleft{\begin{hindi}
भला वह कौन है जो तुम लोगों की ख़़ुश्की और तरी की तारिक़ियों में राह दिखाता है और कौन उसकी बाराने रहमत के आगे आगे (बारिश की) ख़ुशखबरी लेकर हवाओं को भेजता है-क्या ख़ुदा के साथ कोई और माबूद भी है (हरगिज़ नहीं) ये लोग जिन चीज़ों को ख़ुदा का शरीक ठहराते हैं ख़ुदा उससे बालातर है
\end{hindi}}
\flushright{\begin{Arabic}
\quranayah[27][64]
\end{Arabic}}
\flushleft{\begin{hindi}
भला वह कौन हैं जो ख़िलकत को नए सिरे से पैदा करता है फिर उसे दोबारा (मरने के बाद) पैदा करेगा और कौन है जो तुम लोगों को आसमान व ज़मीन से रिज़क़ देता है- तो क्या ख़ुदा के साथ कोई और माबूद भी है (हरग़िज़ नहीं) (ऐ रसूल) तुम (इन मुशरेकीन से) कहा दो कि अगर तुम सच्चे हो तो अपनी दलील पेश करो
\end{hindi}}
\flushright{\begin{Arabic}
\quranayah[27][65]
\end{Arabic}}
\flushleft{\begin{hindi}
(ऐ रसूल इन से) कह दो कि जितने लोग आसमान व ज़मीन में हैं उनमे से कोई भी गैब की बात के सिवा नहीं जानता और वह भी तो नहीं समझते कि क़ब्र से दोबारा कब ज़िन्दा उठ खडे क़िए जाएँगें
\end{hindi}}
\flushright{\begin{Arabic}
\quranayah[27][66]
\end{Arabic}}
\flushleft{\begin{hindi}
बल्कि (असल ये है कि) आख़िरत के बारे में उनके इल्म का ख़ात्मा हो गया है बल्कि उसकी तरफ से शक में पड़ें हैं बल्कि (सच ये है कि) इससे ये लोग अंधे बने हुए हैं
\end{hindi}}
\flushright{\begin{Arabic}
\quranayah[27][67]
\end{Arabic}}
\flushleft{\begin{hindi}
और कुफ्फार कहने लगे कि क्या जब हम और हमारे बाप दादा (सड़ गल कर) मिट्टी हो जाएँगें तो क्या हम फिर निकाले जाएँगें
\end{hindi}}
\flushright{\begin{Arabic}
\quranayah[27][68]
\end{Arabic}}
\flushleft{\begin{hindi}
उसका तो पहले भी हम से और हमारे बाप दादाओं से वायदा किया गया था (कहाँ का उठना और कैसी क़यामत) ये तो हो न हो अगले लोगों के ढकोसले हैं
\end{hindi}}
\flushright{\begin{Arabic}
\quranayah[27][69]
\end{Arabic}}
\flushleft{\begin{hindi}
(ऐ रसूल) लोगों से कह दो कि रुए ज़मीन पर ज़रा चल फिर कर देखो तो गुनाहगारों का अन्जाम क्या हुआ
\end{hindi}}
\flushright{\begin{Arabic}
\quranayah[27][70]
\end{Arabic}}
\flushleft{\begin{hindi}
(ऐ रसूल) तुम उनके हाल पर कुछ अफ़सोस न करो और जो चालें ये लोग (तुम्हारे ख़िलाफ) चल रहे हैं उससे तंग दिल न हो
\end{hindi}}
\flushright{\begin{Arabic}
\quranayah[27][71]
\end{Arabic}}
\flushleft{\begin{hindi}
और ये (कुफ्फ़ार मुसलमानों से) पूछते हैं कि अगर तुम सच्चे हो तो (आख़िर) ये (क़यामत या अज़ाब का) वायदा कब पूरा होगा
\end{hindi}}
\flushright{\begin{Arabic}
\quranayah[27][72]
\end{Arabic}}
\flushleft{\begin{hindi}
(ऐ रसूल) तुम कह दो कि जिस (अज़ाब) की तुम लोग जल्दी मचा रहे हो क्या अजब है इसमे से कुछ करीब आ गया हो
\end{hindi}}
\flushright{\begin{Arabic}
\quranayah[27][73]
\end{Arabic}}
\flushleft{\begin{hindi}
और इसमें तो शक ही नहीं कि तुम्हारा परवरदिगार लोगों पर बड़ा फज़ल व करम करने वाला है मगर बहुतेरे लोग (उसका) शुक्र नहीं करते
\end{hindi}}
\flushright{\begin{Arabic}
\quranayah[27][74]
\end{Arabic}}
\flushleft{\begin{hindi}
और इसमें तो शक नहीं जो बातें उनके दिलों में पोशीदा हैं और जो कुछ ये एलानिया करते हैं तुम्हारा परवरदिगार यक़ीनी जानता है
\end{hindi}}
\flushright{\begin{Arabic}
\quranayah[27][75]
\end{Arabic}}
\flushleft{\begin{hindi}
और आसमान व ज़मीन में कोई ऐसी बात पोशीदा नहीं जो वाज़ेए व रौशन किताब (लौहे महफूज़) में (लिखी) मौजूद न हो
\end{hindi}}
\flushright{\begin{Arabic}
\quranayah[27][76]
\end{Arabic}}
\flushleft{\begin{hindi}
इसमें भी शक नहीं कि ये क़ुरान बनी इसराइल पर उनकी अक्सर बातों को जिन में ये इख्तेलाफ़ करते हैं ज़ाहिर कर देता है
\end{hindi}}
\flushright{\begin{Arabic}
\quranayah[27][77]
\end{Arabic}}
\flushleft{\begin{hindi}
और इसमें भी शक नहीं कि ये कुरान ईमानदारों के वास्ते अज़सरतापा हिदायत व रहमत है
\end{hindi}}
\flushright{\begin{Arabic}
\quranayah[27][78]
\end{Arabic}}
\flushleft{\begin{hindi}
(ऐ रसूल) बेशक तुम्हारा परवरदिगार अपने हुक्म से उनके आपस (के झगड़ों) का फैसला कर देगा और वह (सब पर) ग़ालिब और वाक़िफकार है
\end{hindi}}
\flushright{\begin{Arabic}
\quranayah[27][79]
\end{Arabic}}
\flushleft{\begin{hindi}
तो (ऐ रसूल) तुम खुदा पर भरोसा रखो बेशक तुम यक़ीनी सरीही हक़ पर हो
\end{hindi}}
\flushright{\begin{Arabic}
\quranayah[27][80]
\end{Arabic}}
\flushleft{\begin{hindi}
बेशक न तो तुम मुर्दों को (अपनी बात) सुना सकते हो और न बहरों को अपनी आवाज़ सुना सकते हो (ख़ासकर) जब वह पीठ फेर कर भाग ख़डें हो
\end{hindi}}
\flushright{\begin{Arabic}
\quranayah[27][81]
\end{Arabic}}
\flushleft{\begin{hindi}
और न तुम अंधें को उनकी गुमराही से राह पर ला सकते हो तुम तो बस उन्हीं लोगों को (अपनी बात) सुना सकते हो जो हमारी आयतों पर ईमान रखते हैं
\end{hindi}}
\flushright{\begin{Arabic}
\quranayah[27][82]
\end{Arabic}}
\flushleft{\begin{hindi}
फिर वही लोग तो मानने वाले भी हैं जब उन लोगों पर (क़यामत का) वायदा पूरा होगा तो हम उनके वास्ते ज़मीन से एक चलने वाला निकाल खड़ा करेंगे जो उनसे ये बाते करेंगा कि (फलॉ फला) लोग हमारी आयतो का यक़ीन नहीं रखते थे
\end{hindi}}
\flushright{\begin{Arabic}
\quranayah[27][83]
\end{Arabic}}
\flushleft{\begin{hindi}
और (उस दिन को याद करो) जिस दिन हम हर उम्मत से एक ऐसे गिरोह को जो हमारी आयतों को झुठलाया करते थे (ज़िन्दा करके) जमा करेंगे फिर उन की टोलियाँ अलहदा अलहदा करेंगे
\end{hindi}}
\flushright{\begin{Arabic}
\quranayah[27][84]
\end{Arabic}}
\flushleft{\begin{hindi}
यहाँ तक कि जब वह सब (ख़ुदा के सामने) आएँगें और ख़ुदा उनसे कहेगा क्या तुम ने हमारी आयतों को बगैर अच्छी तरह समझे बूझे झुठलाया-भला तुम क्या क्या करते थे और चूँकि ये लोग ज़ुल्म किया करते थे
\end{hindi}}
\flushright{\begin{Arabic}
\quranayah[27][85]
\end{Arabic}}
\flushleft{\begin{hindi}
इन पर (अज़ाब का) वायदा पूरा हो गया फिर ये लोग कुछ बोल भी तो न सकेंगें
\end{hindi}}
\flushright{\begin{Arabic}
\quranayah[27][86]
\end{Arabic}}
\flushleft{\begin{hindi}
क्या इन लोगों ने ये भी न देखा कि हमने रात को इसलिए बनाया कि ये लोग इसमे चैन करें और दिन को रौशन (ताकि देखभाल करे) बेशक इसमें ईमान लाने वालों के लिए (कुदरते ख़ुदा की) बहुत सी निशानियाँ हैं
\end{hindi}}
\flushright{\begin{Arabic}
\quranayah[27][87]
\end{Arabic}}
\flushleft{\begin{hindi}
और (उस दिन याद करो) जिस दिन सूर फूँका जाएगा तो जितने लोग आसमानों मे हैं और जितने लोग ज़मीन में हैं (ग़रज़ सब के सब) दहल जाएंगें मगर जिस शख्स को ख़ुदा चाहे (वो अलबत्ता मुतमइन रहेगा) और सब लोग उसकी बारगाह में ज़िल्लत व आजिज़ी की हालत में हाज़िर होगें
\end{hindi}}
\flushright{\begin{Arabic}
\quranayah[27][88]
\end{Arabic}}
\flushleft{\begin{hindi}
और तुम पहाड़ों को देखकर उन्हें मज़बूर जमे हुए समझतें हो हालाकि ये (क़यामत के दिन) बादल की तरह उड़े उडे फ़िरेगें (ये भी) ख़ुदा की कारीगरी है कि जिसने हर चीज़ को ख़ूब मज़बूत बनाया है बेशक जो कुछ तुम लोग करते हो उससे वह ख़ूब वाक़िफ़ है
\end{hindi}}
\flushright{\begin{Arabic}
\quranayah[27][89]
\end{Arabic}}
\flushleft{\begin{hindi}
जो शख्स नेक काम करेगा उसके लिए उसकी जज़ा उससे कहीं बेहतर है ओर ये लोग उस दिन ख़ौफ व ख़तरे से महफूज़ रहेंगे
\end{hindi}}
\flushright{\begin{Arabic}
\quranayah[27][90]
\end{Arabic}}
\flushleft{\begin{hindi}
और जो लोग बुरा काम करेंगे वह मुँह के बल जहन्नुम में झोक दिए जाएँगे (और उनसे कहा जाएगा कि) जो कुछ तुम (दुनिया में) करते थे बस उसी का जज़ा तुम्हें दी जाएगी
\end{hindi}}
\flushright{\begin{Arabic}
\quranayah[27][91]
\end{Arabic}}
\flushleft{\begin{hindi}
(ऐ रसूल उनसे कह दो कि) मुझे तो बस यही हुक्म दिया गया है कि मै इस शहर (मक्का) के मालिक की इबादत करुँ जिसने उसे इज्ज़त व हुरमत दी है और हर चीज़ उसकी है और मुझे ये हुक्म दिया गया कि मै (उसके) फरमाबरदार बन्दों में से हूँ
\end{hindi}}
\flushright{\begin{Arabic}
\quranayah[27][92]
\end{Arabic}}
\flushleft{\begin{hindi}
और ये कि मै क़ुरान पढ़ा करुँ फिर जो शख्स राह पर आया तो अपनी ज़ात के नफे क़े वास्ते राह पर आया और जो गुमराह हुआ तो तुम कह दो कि मै भी एक एक डराने वाला हूँ
\end{hindi}}
\flushright{\begin{Arabic}
\quranayah[27][93]
\end{Arabic}}
\flushleft{\begin{hindi}
और तुम कह दो कि अल्हमदोलिल्लाह वह अनक़रीब तुम्हें (अपनी क़ुदरत की) निशानियाँ दिखा देगा तो तुम उन्हें पहचान लोगे और जो कुछ तुम करते हो तुम्हारा परवरदिगार उससे ग़ाफिल नहीं है
\end{hindi}}
\chapter{Al-Qasas (The Narrative)}
\begin{Arabic}
\Huge{\centerline{\basmalah}}\end{Arabic}
\flushright{\begin{Arabic}
\quranayah[28][1]
\end{Arabic}}
\flushleft{\begin{hindi}
ता सीन मीम
\end{hindi}}
\flushright{\begin{Arabic}
\quranayah[28][2]
\end{Arabic}}
\flushleft{\begin{hindi}
(ऐ रसूल) ये वाज़ेए व रौशन किताब की आयतें हैं
\end{hindi}}
\flushright{\begin{Arabic}
\quranayah[28][3]
\end{Arabic}}
\flushleft{\begin{hindi}
(जिसमें) हम तुम्हारें सामने मूसा और फिरऔन का वाक़िया ईमानदार लोगों के नफ़े के वास्ते ठीक ठीक बयान करते हैं
\end{hindi}}
\flushright{\begin{Arabic}
\quranayah[28][4]
\end{Arabic}}
\flushleft{\begin{hindi}
बेशक फिरऔन ने (मिस्र की) सरज़मीन में बहुत सर उठाया था और उसने वहाँ के रहने वालों को कई गिरोह कर दिया था उनमें से एक गिरोह (बनी इसराइल) को आजिज़ कर रखा थ कि उनके बेटों को ज़बाह करवा देता था और उनकी औरतों (बेटियों) को ज़िन्दा छोड़ देता था बेशक वह भी मुफ़सिदों में था
\end{hindi}}
\flushright{\begin{Arabic}
\quranayah[28][5]
\end{Arabic}}
\flushleft{\begin{hindi}
और हम तो ये चाहते हैं कि जो लोग रुए ज़मीन में कमज़ोर कर दिए गए हैं उनपर एहसान करे और उन्हींको (लोगों का) पेशवा बनाएँ और उन्हीं को इस (सरज़मीन) का मालिक बनाएँ
\end{hindi}}
\flushright{\begin{Arabic}
\quranayah[28][6]
\end{Arabic}}
\flushleft{\begin{hindi}
और उन्हीं को रुए ज़मीन पर पूरी क़ुदरत अता करे और फिरऔन और हामान और उन दोनों के लश्करो को उन्हीं कमज़ोरों के हाथ से वह चीज़ें दिखायें जिससे ये लोग डरते थे
\end{hindi}}
\flushright{\begin{Arabic}
\quranayah[28][7]
\end{Arabic}}
\flushleft{\begin{hindi}
और हमने मूसा की माँ के पास ये वही भेजी कि तुम उसको दूध पिला लो फिर जब उसकी निस्बत तुमको कोई ख़ौफ हो तो इसको (एक सन्दूक़ में रखकर) दरिया में डाल दो और (उस पर) तुम कुछ न डरना और न कुढ़ना (तुम इतमेनान रखो) हम उसको फिर तुम्हारे पास पहुँचा देगें और उसको (अपना) रसूल बनाएँगें
\end{hindi}}
\flushright{\begin{Arabic}
\quranayah[28][8]
\end{Arabic}}
\flushleft{\begin{hindi}
(ग़रज़ मूसा की माँ ने दरिया में डाल दिया) वह सन्दूक़ बहते बहते फिरऔन के महल के पास आ लगा तो फिरऔन के लोगों ने उसे उठा लिया ताकि (एक दिन यही) उनका दुश्मन और उनके राज का बायस बने इसमें शक नहीं कि फिरऔन और हामान उन दोनों के लश्कर ग़लती पर थे
\end{hindi}}
\flushright{\begin{Arabic}
\quranayah[28][9]
\end{Arabic}}
\flushleft{\begin{hindi}
और (जब मूसा महल में लाए गए तो) फिरऔन की बीबी (आसिया अपने शौहर से) बोली कि ये मेरी और तुम्हारी (दोनों की) ऑंखों की ठन्डक है तो तुम लोग इसको क़त्ल न करो क्या अजब है कि ये हमको नफ़ा पहुँचाए या हम उसे ले पालक ही बना लें और उन्हें (उसी के हाथ से बर्बाद होने की) ख़बर न थी
\end{hindi}}
\flushright{\begin{Arabic}
\quranayah[28][10]
\end{Arabic}}
\flushleft{\begin{hindi}
इधर तो ये हो रहा था और (उधर) मूसा की माँ का दिल ऐसा बेचैन हो गया कि अगर हम उसके दिल को मज़बूत कर देते तो क़रीब था कि मूसा का हाल ज़ाहिर कर देती (और हमने इसीलिए ढारस दी) ताकि वह (हमारे वायदे का) यक़ीन रखे
\end{hindi}}
\flushright{\begin{Arabic}
\quranayah[28][11]
\end{Arabic}}
\flushleft{\begin{hindi}
और मूसा की माँ ने (दरिया में डालते वक्त) उनकी बहन (कुलसूम) से कहा कि तुम इसके पीछे पीछे (अलग) चली जाओ तो वह मूसा को दूर से देखती रही और उन लोगो को उसकी ख़बर भी न हुई
\end{hindi}}
\flushright{\begin{Arabic}
\quranayah[28][12]
\end{Arabic}}
\flushleft{\begin{hindi}
और हमने मूसा पर पहले ही से और दाईयों (के दूध) को हराम कर दिया था (कि किसी की छाती से मुँह न लगाया) तब मूसा की बहन बोली भला मै तुम्हें एक घराने का पता बताऊ कि वह तुम्हारी ख़ातिर इस बच्चे की परवरिश कर देंगे और वह यक़ीनन इसके खैरख्वाह होगे
\end{hindi}}
\flushright{\begin{Arabic}
\quranayah[28][13]
\end{Arabic}}
\flushleft{\begin{hindi}
ग़रज़ (इस तरकीब से) हमने मूसा को उसकी माँ तक फिर पहुँचा दिया ताकि उसकी ऑंख ठन्डी हो जाए और रंज न करे और ताकि समझ ले ख़ुदा का वायदा बिल्कुल ठीक है मगर उनमें के अक्सर नहीं जानते हैं
\end{hindi}}
\flushright{\begin{Arabic}
\quranayah[28][14]
\end{Arabic}}
\flushleft{\begin{hindi}
और जब मूसा अपनी जवानी को पहुँचे और (हाथ पाँव निकाल के) दुरुस्त हो गए तो हमने उनको हिकमत और इल्म अता किया और नेकी करने वालों को हम यूँ जज़ाए खैर देते हैं
\end{hindi}}
\flushright{\begin{Arabic}
\quranayah[28][15]
\end{Arabic}}
\flushleft{\begin{hindi}
और एक दिन इत्तिफाक़न मूसा शहर में ऐसे वक्त अाए कि वहाँ के लोग (नींद की) ग़फलत में पडे हुए थे तो देखा कि वहाँ दो आदमी आपस में लड़े मरते हैं ये (एक) तो उनकी क़ौम (बनी इसराइल) में का है और वह (दूसरा) उनके दुश्मन की क़ौम (क़िब्ती) का है तो जो शख्स उनकी क़ौम का था उसने उस शख्स से जो उनके दुश्मनों में था (ग़लबा हासिल करने के लिए) मूसा से मदद माँगी ये सुनते ही मूसा ने उसे एक घूसा मारा था कि उसका काम तमाम हो गया फिर (ख्याल करके) कहने लगे ये शैतान का काम था इसमें शक नहीं कि वह दुश्मन और खुल्लम खुल्ला गुमराह करने वाला है
\end{hindi}}
\flushright{\begin{Arabic}
\quranayah[28][16]
\end{Arabic}}
\flushleft{\begin{hindi}
(फिर बारगाहे ख़ुदा में) अर्ज़ की परवरदिगार बेशक मैने अपने ऊपर आप ज़ुल्म किया (कि इस शहर में आया) तो तू मुझे (दुश्मनों से) पोशीदा रख-ग़रज़ ख़ुदा ने उन्हें पोशीदा रखा (इसमें तो शक नहीं कि वह बड़ा पोशीदा रखने वाला मेहरबान है)
\end{hindi}}
\flushright{\begin{Arabic}
\quranayah[28][17]
\end{Arabic}}
\flushleft{\begin{hindi}
मूसा ने अर्ज क़ी परवरदिगार चूँकि तूने मुझ पर एहसान किया है मै भी आइन्दा गुनाहगारों का हरगिज़ मदद गार न बनूगाँ
\end{hindi}}
\flushright{\begin{Arabic}
\quranayah[28][18]
\end{Arabic}}
\flushleft{\begin{hindi}
ग़रज़ (रात तो जो त्यों गुज़री) सुबह को उम्मीदो बीम की हालत में मूसा शहर में गए तो क्या देखते हैं कि वही शख्स जिसने कल उनसे मदद माँगी थी उनसे (फिर) फरियाद कर रहा है-मूसा ने उससे कहा बेशक तू यक़ीनी खुल्लम खुल्ला गुमराह है
\end{hindi}}
\flushright{\begin{Arabic}
\quranayah[28][19]
\end{Arabic}}
\flushleft{\begin{hindi}
ग़रज़ जब मूसा ने चाहा कि उस शख्स पर जो दोनों का दुश्मन था (छुड़ाने के लिए) हाथ बढ़ाएँ तो क़िब्ती कहने लगा कि ऐ मूसा जिस तरह तुमने कल एक आदमी को मार डाला (उसी तरह) मुझे भी मार डालना चाहते हो तो तुम बस ये चाहते हो कि रुए ज़मीन में सरकश बन कर रहो और मसलह (क़ौम) बनकर रहना नहीं चाहते
\end{hindi}}
\flushright{\begin{Arabic}
\quranayah[28][20]
\end{Arabic}}
\flushleft{\begin{hindi}
और एक शख्स शहर के उस किनारे से डराता हुआ आया और (मूसा से) कहने लगा मूसा (तुम ये यक़ीन जानो कि शहर के) बड़े बड़े आदमी तुम्हारे आदमी तुम्हारे बारे में मशवरा कर रहे हैं कि तुमको कत्ल कर डालें तो तुम (शहर से) निकल भागो
\end{hindi}}
\flushright{\begin{Arabic}
\quranayah[28][21]
\end{Arabic}}
\flushleft{\begin{hindi}
मै तुमसे ख़ैरख्वाहाना (भलाइ के लिए) कहता हूँ ग़रज़ मूसा वहाँ से उम्मीद व बीम की हालत में निकल खडे हुए और (बारगाहे ख़ुदा में) अर्ज़ की परवरदिगार मुझे ज़ालिम लोगों (के हाथ) से नजात दे
\end{hindi}}
\flushright{\begin{Arabic}
\quranayah[28][22]
\end{Arabic}}
\flushleft{\begin{hindi}
और जब मदियन की तरफ रुख़ किया (और रास्ता मालूम न था) तो आप ही आप बोले मुझे उम्मीद है कि मेरा परवरदिगार मुझे सीधे रास्ता दिखा दे
\end{hindi}}
\flushright{\begin{Arabic}
\quranayah[28][23]
\end{Arabic}}
\flushleft{\begin{hindi}
और (आठ दिन फाक़ा करते चले) जब शहर मदियन के कुओं पर (जो शहर के बाहर था) पहुँचें तो कुओं पर लोगों की भीड़ देखी कि वह (अपने जानवरों को) पानी पिला रहे हैं और उन सबके पीछे दो औरतो (हज़रत शुएब की बेटियों) को देखा कि वह (अपनी बकरियों को) रोके खड़ी है मूसा ने पूछा कि तुम्हारा क्या मतलब है वह बोली जब तक सब चरवाहे (अपने जानवरों को) ख़ूब छक के पानी पिला कर फिर न जाएँ हम नहीं पिला सकते और हमारे वालिद बहुत बूढे हैं
\end{hindi}}
\flushright{\begin{Arabic}
\quranayah[28][24]
\end{Arabic}}
\flushleft{\begin{hindi}
तब मूसा ने उन की (बकरियों) के लिए (पानी खीच कर) पिला दिया फिर वहाँ से हट कर छांव में जा बैठे तो (चूँकि बहुत भूक थी) अर्ज क़ी परवरदिगार (उस वक्त) ज़ो नेअमत तू मेरे पास भेज दे मै उसका सख्त हाजत मन्द हूँ
\end{hindi}}
\flushright{\begin{Arabic}
\quranayah[28][25]
\end{Arabic}}
\flushleft{\begin{hindi}
इतने में उन्हीं दो मे से एक औरत शर्मीली चाल से आयी (और मूसा से) कहने लगी-मेरे वालिद तुम को बुलाते हैं ताकि तुमने जो (हमारी बकरियों को) पानी पिला दिया है तुम्हें उसकी मज़दूरी दे ग़रज़ जब मूसा उनके पास आए और उनसे अपने किस्से बयान किए तो उन्होंने कहा अब कुछ अन्देशा न करो तुमने ज़ालिम लोगों के हाथ से नजात पायी
\end{hindi}}
\flushright{\begin{Arabic}
\quranayah[28][26]
\end{Arabic}}
\flushleft{\begin{hindi}
(इसी असना में) उन दोनों में से एक लड़की ने कहा ऐ अब्बा इन को नौकर रख लीजिए क्योंकि आप जिसको भी नौकर रखें सब में बेहतर वह है जो मज़बूत और अमानतदार हो
\end{hindi}}
\flushright{\begin{Arabic}
\quranayah[28][27]
\end{Arabic}}
\flushleft{\begin{hindi}
(और इनमें दोनों बातें पायी जाती हैं तब) शुएब ने कहा मै चाहता हूँ कि अपनी दोनों लड़कियों में से एक के साथ तुम्हारा इस (महर) पर निकाह कर दूँ कि तुम आठ बरस तक मेरी नौकरी करो और अगर तुम दस बरस पूरे कर दो तो तुम्हारा एहसान और मै तुम पर मेहनत मशक्क़त भी डालना नही चाहता और तुम मुझे इन्शा अल्लाह नेको कार आदमी पाओगे
\end{hindi}}
\flushright{\begin{Arabic}
\quranayah[28][28]
\end{Arabic}}
\flushleft{\begin{hindi}
मूसा ने कहा ये मेरे और आप के दरमियान (मुहाएदा) है दोनों मुद्दतों मे से मै जो भी पूरी कर दूँ (मुझे एख्तियार है) फिर मुझ पर जब्र और ज्यादती (देने का आपको हक़) नहीं और हम आप जो कुछ कर रहे हैं (उसका) ख़ुदा गवाह है
\end{hindi}}
\flushright{\begin{Arabic}
\quranayah[28][29]
\end{Arabic}}
\flushleft{\begin{hindi}
ग़रज़ मूसा का छोटी लड़की से निकाह हो गया और रहने लगे फिर जब मूसा ने अपनी (दस बरस की) मुद्दत पूरी की और बीवी को लेकर चले तो अंधेरीरात जाड़ों के दिन राह भूल गए और बीबी सफ़ूरा को दर्द ज़ेह शुरु हुआ (इतने में) कोहेतूर की तरफ आग दिखायी दी तो अपने लड़के बालों से कहा तुम लोग ठहरो मैने यक़ीनन आग देखी है (मै वहाँ जाता हूँ) क्या अजब है वहाँ से (रास्ते की) कुछ ख़बर लाऊँ या आग की कोई चिंगारी (लेता आऊँ) ताकि तुम लोग तापो
\end{hindi}}
\flushright{\begin{Arabic}
\quranayah[28][30]
\end{Arabic}}
\flushleft{\begin{hindi}
ग़रज़ जब मूसा आग के पास आए तो मैदान के दाहिने किनारे से इस मुबारक जगह में एक दरख्त से उन्हें आवाज़ आयी कि ऐ मूसा इसमें शक नहीं कि मै ही अल्लाह सारे जहाँ का पालने वाला हूँ
\end{hindi}}
\flushright{\begin{Arabic}
\quranayah[28][31]
\end{Arabic}}
\flushleft{\begin{hindi}
और यह (भी आवाज़ आयी) कि तुम आपनी छड़ी (ज़मीन पर) डाल दो फिर जब (डाल दिया तो) देखा कि वह इस तरह बल खा रही है कि गोया वह (ज़िन्दा) अजदहा है तो पीठ फेरके भागे और पीछे मुड़कर भी न देखा (तो हमने फरमाया) ऐ मूसा आगे आओ और डरो नहीं तुम पर हर तरह अमन व अमान में हो
\end{hindi}}
\flushright{\begin{Arabic}
\quranayah[28][32]
\end{Arabic}}
\flushleft{\begin{hindi}
(अच्छा और लो) अपना हाथ गरेबान में डालो (और निकाल लो) तो सफेद बुर्राक़ होकर बेऐब निकल आया और ख़ौफ की (वजह) से अपने बाजू अपनी तरफ समेट लो (ताकि ख़ौफ जाता रहे) ग़रज़ ये दोनों (असा व यदे बैज़ा) तुम्हारे परवरदिगार की तरफ से (तुम्हारी नुबूवत की) दो दलीलें फिरऔन और उसके दरबार के सरदारों के वास्ते हैं और इसमें शक नहीं कि वह बदकार लोग थे
\end{hindi}}
\flushright{\begin{Arabic}
\quranayah[28][33]
\end{Arabic}}
\flushleft{\begin{hindi}
मूसा ने अर्ज़ की परवरदिगार मैने उनमें से एक शख्स को मार डाला था तो मै डरता हूँ कहीं (उसके बदले) मुझे न मार डालें
\end{hindi}}
\flushright{\begin{Arabic}
\quranayah[28][34]
\end{Arabic}}
\flushleft{\begin{hindi}
और मेरा भाई हारुन वह मुझसे (ज़बान में ज्यादा) फ़सीह है तो तू उसे मेरे साथ मेरा मददगार बनाकर भेज कि वह मेरी तसदीक करे क्योंकि यक़ीनन मै इस बात से डरता हूँ कि मुझे वह लोग झुठला देंगे (तो उनके जवाब के लिए गोयाइ की ज़रुरत है)
\end{hindi}}
\flushright{\begin{Arabic}
\quranayah[28][35]
\end{Arabic}}
\flushleft{\begin{hindi}
फ़रमाया अच्छा हम अनक़रीब तुम्हारे भाई की वजह से तुम्हारे बाज़ू क़वी कर देगें और तुम दोनों को ऐसा ग़लबा अता करेंगें कि फिरऔनी लोग तुम दोनों तक हमारे मौजिज़े की वजह से पहुँच भी न सकेंगे लो जाओ तुम दोनो और तुम्हारे पैरवी करने वाले गालिब रहेंगे
\end{hindi}}
\flushright{\begin{Arabic}
\quranayah[28][36]
\end{Arabic}}
\flushleft{\begin{hindi}
ग़रज़ जब मूसा हमारे वाजेए व रौशन मौजिज़े लेकर उनके पास आए तो वह लोग कहने लगे कि ये तो बस अपने दिल का गढ़ा हुआ जादू है और हमने तो अपने अगले बाप दादाओं (के ज़माने) में ऐसी बात सुनी भी नहीें
\end{hindi}}
\flushright{\begin{Arabic}
\quranayah[28][37]
\end{Arabic}}
\flushleft{\begin{hindi}
और मूसा ने कहा मेरा परवरदिगार उस शख्स से ख़ूब वाक़िफ़ है जो उसकी बारगाह से हिदायत लेकर आया है और उस शख्स से भी जिसके लिए आख़िरत का घर है इसमें तो शक ही नहीं कि ज़ालिम लोग कामयाब नहीं होते
\end{hindi}}
\flushright{\begin{Arabic}
\quranayah[28][38]
\end{Arabic}}
\flushleft{\begin{hindi}
और (ये सुनकर) फिरऔन ने कहा ऐ मेरे दरबार के सरदारों मुझ को तो अपने सिवा तुम्हारा कोई परवरदिगार मालूम नही होता (और मूसा दूसरे को ख़ुदा बताता है) तो ऐ हामान (वज़ीर फिरऔन) तुम मेरे वास्ते मिट्टी (की ईटों) का पजावा सुलगाओ फिर मेरे वास्ते एक पुख्ता महल तैयार कराओ ताकि मै (उस पर चढ़ कर) मूसा के ख़ुदा को देंखू और मै तो यक़ीनन मूसा को झूठा समझता हूँ
\end{hindi}}
\flushright{\begin{Arabic}
\quranayah[28][39]
\end{Arabic}}
\flushleft{\begin{hindi}
और फिरऔन और उसके लश्कर ने रुए ज़मीन में नाहक़ सर उठाया था और उन लोगों ने समझ लिया था कि हमारी बारगाह मे वह कभी पलट कर नही आएँगे
\end{hindi}}
\flushright{\begin{Arabic}
\quranayah[28][40]
\end{Arabic}}
\flushleft{\begin{hindi}
तो हमने उसको और उसके लश्कर को ले डाला फिर उन सबको दरिया में डाल दिया तो (ऐ रसूल) ज़रा देखों तो कि ज़ालिमों का कैसा बुरा अन्जाम हुआ
\end{hindi}}
\flushright{\begin{Arabic}
\quranayah[28][41]
\end{Arabic}}
\flushleft{\begin{hindi}
और हमने उनको (गुमराहों का) पेशवा बनाया कि (लोगों को) जहन्नुम की तरफ बुलाते है और क़यामत के दिन (ऐसे बेकस होगें कि) उनको किसी तरह की मदद न दी जाएगी
\end{hindi}}
\flushright{\begin{Arabic}
\quranayah[28][42]
\end{Arabic}}
\flushleft{\begin{hindi}
और हमने दुनिया में भी तो लानत उन के पीछे लगा दी है और क़यामत के दिन उनके चेहरे बिगाड़ दिए जायेंगे
\end{hindi}}
\flushright{\begin{Arabic}
\quranayah[28][43]
\end{Arabic}}
\flushleft{\begin{hindi}
और हमने बहुतेरी अगली उम्मतों को हलाक कर डाला उसके बाद मूसा को किताब (तौरैत) अता की जो लोगों के लिए अजसरतापा बसीरत और हिदायत और रहमत थी ताकि वह लोग इबरत व नसीहत हासिल करें
\end{hindi}}
\flushright{\begin{Arabic}
\quranayah[28][44]
\end{Arabic}}
\flushleft{\begin{hindi}
और (ऐ रसूल) जिस वक्त हमने मूसा के पास अपना हुक्म भेजा था तो तुम (तूर के) मग़रिबी जानिब मौजूद न थे और न तुम उन वाक्यात को चश्मदीद देखने वालों में से थे
\end{hindi}}
\flushright{\begin{Arabic}
\quranayah[28][45]
\end{Arabic}}
\flushleft{\begin{hindi}
मगर हमने (मूसा के बाद) बहुतेरी उम्मतें पैदा की फिर उन पर एक ज़माना दराज़ गुज़र गया और न तुम मदैन के लोगों में रहे थे कि उनके सामने हमारी आयते पढ़ते (और न तुम को उन के हालात मालूम होते) मगर हम तो (तुमको) पैग़म्बर बनाकर भेजने वाले थे
\end{hindi}}
\flushright{\begin{Arabic}
\quranayah[28][46]
\end{Arabic}}
\flushleft{\begin{hindi}
और न तुम तूर की किसी जानिब उस वक्त मौजूद थे जब हमने (मूसा को) आवाज़ दी थी (ताकि तुम देखते) मगर ये तुम्हारे परवरदिगार की मेहरबानी है ताकि तुम उन लोगों को जिनके पास तुमसे पहले कोई डराने वाला आया ही नहीं डराओ ताकि ये लोग नसीहत हासिल करें
\end{hindi}}
\flushright{\begin{Arabic}
\quranayah[28][47]
\end{Arabic}}
\flushleft{\begin{hindi}
और अगर ये नही होता कि जब उन पर उनकी अगली करतूतों की बदौलत कोई मुसीबत पड़ती तो बेसाख्ता कह बैठते कि परवरदिगार तूने हमारे पास कोई पैग़म्बर क्यों न भेजा कि हम तेरे हुक्मों पर चलते और ईमानदारों में होते (तो हम तुमको न भेजते )
\end{hindi}}
\flushright{\begin{Arabic}
\quranayah[28][48]
\end{Arabic}}
\flushleft{\begin{hindi}
मगर फिर जब हमारी बारगाह से (दीन) हक़ उनके पास पहुँचा तो कहने लगे जैसे (मौजिज़े) मूसा को अता हुए थे वैसे ही इस रसूल (मोहम्मद) को क्यों नही दिए गए क्या जो मौजिज़े इससे पहले मूसा को अता हुए थे उनसे इन लोगों ने इन्कार न किया था कुफ्फ़ार तो ये भी कह गुज़रे कि ये दोनों के दोनों (तौरैत व कुरान) जादू हैं कि बाहम एक दूसरे के मददगार हो गए हैं
\end{hindi}}
\flushright{\begin{Arabic}
\quranayah[28][49]
\end{Arabic}}
\flushleft{\begin{hindi}
और ये भी कह चुके कि हम एब के मुन्किर हैं (ऐ रसूल) तुम (इन लोगों से) कह दो कि अगर सच्चे हो तो ख़ुदा की तरफ से एक ऐसी किताब जो इन दोनों से हिदायत में बेहतर हो ले आओ
\end{hindi}}
\flushright{\begin{Arabic}
\quranayah[28][50]
\end{Arabic}}
\flushleft{\begin{hindi}
कि मै भी उस पर चलँ फिर अगर ये लोग (इस पर भी) न मानें तो समझ लो कि ये लोग बस अपनी हवा व हवस की पैरवी करते है और जो शख्स ख़ुदा की हिदायत को छोड़ कर अपनी हवा व हवस की पैरवी करते है उससे ज्यादा गुमराह कौन होगा बेशक ख़ुदा सरकश लोगों को मंज़िले मक़सूद तक नहीं पहुँचाया करता
\end{hindi}}
\flushright{\begin{Arabic}
\quranayah[28][51]
\end{Arabic}}
\flushleft{\begin{hindi}
और हम यक़ीनन लगातार (अपने एहकाम भेजकर) उनकी नसीहत करते रहे ताकि वह लोग नसीहत हासिल करें
\end{hindi}}
\flushright{\begin{Arabic}
\quranayah[28][52]
\end{Arabic}}
\flushleft{\begin{hindi}
जिन लोगों को हमने इससे पहले किताब अता की है वह उस (क़ुरान) पर ईमान लाते हैं
\end{hindi}}
\flushright{\begin{Arabic}
\quranayah[28][53]
\end{Arabic}}
\flushleft{\begin{hindi}
और जब उनके सामने ये पढ़ा जाता है तो बोल उठते हैं कि हम तो इस पर ईमान ला चुके बेशक ये ठीक है (और) हमारे परवरदिगार की तरफ से है हम तो इसको पहले ही मानते थे
\end{hindi}}
\flushright{\begin{Arabic}
\quranayah[28][54]
\end{Arabic}}
\flushleft{\begin{hindi}
यही वह लोग हैं जिन्हें (इनके आमाले ख़ैर की) दोहरी जज़ा दी जाएगी-चूँकि उन लोगों ने सब्र किया और बदी को नेकी से दफ़ा करते हैं और जो कुछ हमने उन्हें अता किया है उसमें से (हमारी राह में) ख़र्च करते हैं
\end{hindi}}
\flushright{\begin{Arabic}
\quranayah[28][55]
\end{Arabic}}
\flushleft{\begin{hindi}
और जब किसी से कोई बुरी बात सुनी तो उससे किनारा कश रहे और साफ कह दिया कि हमारे वास्ते हमारी कारगुज़ारियाँ हैं और तुम्हारे वास्ते तुम्हारी कारस्तानियाँ (बस दूर ही से) तुम्हें सलाम है हम जाहिलो (की सोहबत) के ख्वाहॉ नहीं
\end{hindi}}
\flushright{\begin{Arabic}
\quranayah[28][56]
\end{Arabic}}
\flushleft{\begin{hindi}
(ऐ रसूल) बेशक तुम जिसे चाहो मंज़िले मक़सूद तक नहीं पहुँचा सकते मगर हाँ जिसे खुदा चाहे मंज़िल मक़सूद तक पहुचाए और वही हिदायत याफ़ता लोगों से ख़ूब वाक़िफ़ है
\end{hindi}}
\flushright{\begin{Arabic}
\quranayah[28][57]
\end{Arabic}}
\flushleft{\begin{hindi}
(ऐ रसूल) कुफ्फ़ार (मक्का) तुमसे कहते हैं कि अगर हम तुम्हारे साथ दीन हक़ की पैरवी करें तो हम अपने मुल्क़ से उचक लिए जाएँ (ये क्या बकते है) क्या हमने उन्हें हरम (मक्का) में जहाँ हर तरह का अमन है जगह नहीं दी वहाँ हर किस्म के फल रोज़ी के वास्ते हमारी बारगाह से खिंचे चले जाते हैं मगर बहुतेरे लोग नहीं जाते
\end{hindi}}
\flushright{\begin{Arabic}
\quranayah[28][58]
\end{Arabic}}
\flushleft{\begin{hindi}
और हमने तो बहुतेरी बस्तियाँ बरबाद कर दी जो अपनी मइशत (रोजी) में बहुत इतराहट से (ज़िन्दगी) बसर किया करती थीं-(तो देखो) ये उन ही के (उजड़े हुए) घर हैं जो उनके बाद फिर आबाद नहीं हुए मगर बहुत कम और (आख़िर) हम ही उनके (माल व असबाब के) वारिस हुए
\end{hindi}}
\flushright{\begin{Arabic}
\quranayah[28][59]
\end{Arabic}}
\flushleft{\begin{hindi}
और तुम्हारा परवरदिगार जब तक उन गाँव के सदर मक़ाम पर अपना पैग़म्बर न भेज ले और वह उनके सामने हमारी आयतें न पढ़ दे (उस वक्त तक) बस्तियों को बरबाद नहीं कर दिया करता-और हम तो बस्तियों को बरबाद करते ही नहीं जब तक वहाँ के लोग ज़ालिम न हों
\end{hindi}}
\flushright{\begin{Arabic}
\quranayah[28][60]
\end{Arabic}}
\flushleft{\begin{hindi}
और तुम लोगों को जो कुछ अता हुआ है तो दुनिया की (ज़रा सी) ज़िन्दगी का फ़ायदा और उसकी आराइश है और जो कुछ ख़ुदा के पास है वह उससे कही बेहतर और पाएदार है तो क्या तुम इतना भी नहीं समझते
\end{hindi}}
\flushright{\begin{Arabic}
\quranayah[28][61]
\end{Arabic}}
\flushleft{\begin{hindi}
तो क्या वह शख्स जिससे हमने (बेहश्त का) अच्छा वायदा किया है और वह उसे पाकर रहेगा उस शख्स के बराबर हो सकता है जिसे हमने दुनियावी ज़िन्दगी के (चन्द रोज़ा) फायदे अता किए हैं और फिर क़यामत के दिन (जवाब देही के वास्ते हमारे सामने) हाज़िर किए जाएँगें
\end{hindi}}
\flushright{\begin{Arabic}
\quranayah[28][62]
\end{Arabic}}
\flushleft{\begin{hindi}
और जिस दिन ख़ुदा उन कुफ्फ़ार को पुकारेगा और पूछेगा कि जिनको तुम हमारा शरीक ख्याल करते थे वह (आज) कहाँ हैं (ग़रज़ वह शरीक भी बुलाँए जाएँगे)
\end{hindi}}
\flushright{\begin{Arabic}
\quranayah[28][63]
\end{Arabic}}
\flushleft{\begin{hindi}
वह लोग जो हमारे अज़ाब के मुस्ताजिब हो चुके हैं कह देगे कि परवरदिगार यही वह लोग हैं जिन्हें हमने गुमराह किया था जिस तरह हम ख़़ुद गुमराह हुए उसी तरह हमने इनको गुमराह किया-अब हम तेरी बारगाह में (उनसे) दस्तबरदार होते है-ये लोग हमारी इबादत नहीं करते थे
\end{hindi}}
\flushright{\begin{Arabic}
\quranayah[28][64]
\end{Arabic}}
\flushleft{\begin{hindi}
और कहा जाएगा कि भला अपने उन शरीको को (जिन्हें तुम ख़ुदा समझते थे) बुलाओ तो ग़रज़ वह लोग उन्हें बुलाएँगे तो वह उन्हें जवाब तक नही देगें और (अपनी ऑंखों से) अज़ाब को देखेंगें काश ये लोग (दुनिया में) राह पर आए होते
\end{hindi}}
\flushright{\begin{Arabic}
\quranayah[28][65]
\end{Arabic}}
\flushleft{\begin{hindi}
और (वह दिन याद करो) जिस दिन ख़ुदा लोगों को पुकार कर पूछेगा कि तुम लोगों ने पैग़म्बरों को (उनके समझाने पर) क्या जवाब दिया
\end{hindi}}
\flushright{\begin{Arabic}
\quranayah[28][66]
\end{Arabic}}
\flushleft{\begin{hindi}
तब उस दिन उन्हें बातें न सूझ पडेग़ी (और) फिर बाहम एक दूसरे से पूछ भी न सकेगें
\end{hindi}}
\flushright{\begin{Arabic}
\quranayah[28][67]
\end{Arabic}}
\flushleft{\begin{hindi}
मगर हाँ जिस शख्स ने तौबा कर ली और ईमान लाया और अच्छे अच्छे काम किए तो क़रीब है कि ये लोग अपनी मुरादें पाने वालों से होंगे
\end{hindi}}
\flushright{\begin{Arabic}
\quranayah[28][68]
\end{Arabic}}
\flushleft{\begin{hindi}
और तुम्हारा परवरदिगार जो चाहता है पैदा करता है और (जिसे चाहता है) मुन्तख़िब करता है और ये इन्तिख़ाब लोगों के एख्तियार में नहीं है और जिस चीज़ को ये लोग ख़ुदा का शरीक बनाते हैं उससे ख़ुदा पाक और (कहीं) बरतर है
\end{hindi}}
\flushright{\begin{Arabic}
\quranayah[28][69]
\end{Arabic}}
\flushleft{\begin{hindi}
और (ऐ रसूल) ये लोग जो बातें अपने दिलों में छिपाते हैं और जो कुछ ज़ाहिर करते हैं तुम्हारा परवरदिगार खूब जानता है
\end{hindi}}
\flushright{\begin{Arabic}
\quranayah[28][70]
\end{Arabic}}
\flushleft{\begin{hindi}
और वही ख़ुदा है उसके सिवा कोई क़ाबिले परसतिश नहीं दुनिया और आख़िरत में उस की तारीफ़ है और उसकी हुकूमत है और तुम लोग (मरने के बाद) उसकी तरफ लौटाए जाओगे
\end{hindi}}
\flushright{\begin{Arabic}
\quranayah[28][71]
\end{Arabic}}
\flushleft{\begin{hindi}
(ऐ रसूल इन लोगों से) कहो कि भला तुमने देखा कि अगर ख़ुदा हमेशा के लिए क़यामत तक तुम्हारे सरों पर रात को छाए रहता तो अल्लाह के सिवा कौन ख़ुदा है जो तुम्हारे पास रौशनी ले आता तो क्या तुम सुनते नहीं हो
\end{hindi}}
\flushright{\begin{Arabic}
\quranayah[28][72]
\end{Arabic}}
\flushleft{\begin{hindi}
(ऐ रसूल उन से) कह दो कि भला तुमने देखा कि अगर ख़ुदा क़यामत तक बराबर तुम्हारे सरों पर दिन किए रहता तो अल्लाह के सिवा कौन ख़ुदा है जो तुम्हारे लिए रात को ले आता कि तुम लोग इसमें रात को आराम करो तो क्या तुम लोग (इतना भी) नहीं देखते
\end{hindi}}
\flushright{\begin{Arabic}
\quranayah[28][73]
\end{Arabic}}
\flushleft{\begin{hindi}
और उसने अपनी मेहरबानी से तुम्हारे वास्ते रात और दिन को बनाया ताकि तुम रात में आराम करो और दिन में उसके फज़ल व करम (रोज़ी) की तलाश करो और ताकि तुम लोग शुक्र करो
\end{hindi}}
\flushright{\begin{Arabic}
\quranayah[28][74]
\end{Arabic}}
\flushleft{\begin{hindi}
और (उस दिन को याद करो) जिस दिन वह उन्हें पुकार कर पूछेगा जिनको तुम लोग मेरा शरीक ख्याल करते थे वह (आज) कहाँ हैं
\end{hindi}}
\flushright{\begin{Arabic}
\quranayah[28][75]
\end{Arabic}}
\flushleft{\begin{hindi}
और हम हर एक उम्मत से एक गवाह (पैग़म्बर) निकाले (सामने बुलाएँगे) फिर (उस दिन मुशरेकीन से) कहेंगे कि अपनी (बराअत की) दलील पेश करो तब उन्हें मालूम हो जाएगा कि हक़ ख़ुदा ही की तरफ़ है और जो इफ़तेरा परवाज़ियाँ ये लोग किया करते थे सब उनसे ग़ायब हो जाएँगी
\end{hindi}}
\flushright{\begin{Arabic}
\quranayah[28][76]
\end{Arabic}}
\flushleft{\begin{hindi}
(नाशुक्री का एक क़िस्सा सुनो) मूसा की क़ौम से एक शख्स कारुन (नामी) था तो उसने उन पर सरकशी शुरु की और हमने उसको इस क़दर ख़ज़ाने अता किए थे कि उनकी कुन्जियाँ एक सकतदार जमाअत (की जामअत) को उठाना दूभर हो जाता था जब (एक बार) उसकी क़ौम ने उससे कहा कि (अपनी दौलत पर) इतरा मत क्योंकि ख़ुदा इतराने वालों को दोस्त नहीं रखता
\end{hindi}}
\flushright{\begin{Arabic}
\quranayah[28][77]
\end{Arabic}}
\flushleft{\begin{hindi}
और जो कुछ ख़ुदा ने तूझे दे रखा है उसमें आख़िरत के घर की भी जुस्तजू कर और दुनिया से जिस क़दर तेरा हिस्सा है मत भूल जा और जिस तरह ख़ुदा ने तेरे साथ एहसान किया है तू भी औरों के साथ एहसान कर और रुए ज़मीन में फसाद का ख्वाहा न हो-इसमें शक नहीं कि ख़ुदा फ़साद करने वालों को दोस्त नहीं रखता
\end{hindi}}
\flushright{\begin{Arabic}
\quranayah[28][78]
\end{Arabic}}
\flushleft{\begin{hindi}
तो क़ारुन कहने लगा कि ये (माल व दौलत) तो मुझे अपने इल्म (कीमिया) की वजह से हासिल होता है क्या क़ारुन ने ये भी न ख्याल किया कि अल्लाह उसके पहले उन लोगों को हलाक़ कर चुका है जो उससे क़ूवत और हैसियत में कहीं बढ़ बढ़ के थे और गुनाहगारों से (उनकी सज़ा के वक्त) उनके गुनाहों की पूछताछ नहीं हुआ करती
\end{hindi}}
\flushright{\begin{Arabic}
\quranayah[28][79]
\end{Arabic}}
\flushleft{\begin{hindi}
ग़रज़ (एक दिन क़ारुन) अपनी क़ौम के सामने बड़ी आराइश और ठाठ के साथ निकला तो जो लोग दुनिया को (चन्द रोज़ा) ज़िन्दगी के तालिब थे (इस शान से देख कर) कहने लगे जो माल व दौलत क़ारुन को अता हुई है काश मेरे लिए भी होती इसमें शक नहीं कि क़ारुन बड़ा नसीब वर था
\end{hindi}}
\flushright{\begin{Arabic}
\quranayah[28][80]
\end{Arabic}}
\flushleft{\begin{hindi}
और जिन लोगों को (हमारी बारगाह में) इल्म अता हुआ था कहनें लगे तुम्हारा नास हो जाए (अरे) जो शख्स ईमान लाए और अच्छे काम करे उसके लिए तो ख़ुदा का सवाब इससे कही बेहतर है और वह तो अब सब्र करने वालों के सिवा दूसरे नहीं पा सकते
\end{hindi}}
\flushright{\begin{Arabic}
\quranayah[28][81]
\end{Arabic}}
\flushleft{\begin{hindi}
और हमने क़ारुन और उसके घर बार को ज़मीन में धंसा दिया फिर ख़ुदा के सिवा कोई जमाअत ऐसी न थी कि उसकी मदद करती और न खुद आप अपनी मदद आप कर सका
\end{hindi}}
\flushright{\begin{Arabic}
\quranayah[28][82]
\end{Arabic}}
\flushleft{\begin{hindi}
और जिन लोगों ने कल उसके जाह व मरतबे की तमन्ना की थी वह (आज ये तमाशा देखकर) कहने लगे अरे माज़अल्लाह ये तो ख़ुदा ही अपने बन्दों से जिसकी रोज़ी चाहता है कुशादा कर देता है और जिसकी रोज़ी चाहता है तंग कर देता है और अगर (कहीं) ख़ुदा हम पर मेहरबानी न करता (और इतना माल दे देता) तो उसकी तरह हमको भी ज़रुर धॅसा देता-और माज़अल्लाह (सच है) हरगिज़ कुफ्फार अपनी मुरादें न पाएँगें
\end{hindi}}
\flushright{\begin{Arabic}
\quranayah[28][83]
\end{Arabic}}
\flushleft{\begin{hindi}
ये आख़िरत का घर तो हम उन्हीं लोगों के लिए ख़ास कर देगें जो रुए ज़मीन पर न सरकशी करना चाहते हैं और न फसाद-और (सच भी यूँ ही है कि) फिर अन्जाम तो परहेज़गारों ही का है
\end{hindi}}
\flushright{\begin{Arabic}
\quranayah[28][84]
\end{Arabic}}
\flushleft{\begin{hindi}
जो शख्स नेकी करेगा तो उसके लिए उसे कहीं बेहतर बदला है औ जो बुरे काम करेगा तो वह याद रखे कि जिन लोगों ने बुराइयाँ की हैं उनका वही बदला हे जो दुनिया में करते रहे हैं
\end{hindi}}
\flushright{\begin{Arabic}
\quranayah[28][85]
\end{Arabic}}
\flushleft{\begin{hindi}
(ऐ रसूल) ख़ुदा जिसने तुम पर क़ुरान नाज़िल किया ज़रुर ठिकाने तक पहुँचा देगा (ऐ रसूल) तुम कह दो कि कौन राह पर आया और कौन सरीही गुमराही में पड़ा रहा
\end{hindi}}
\flushright{\begin{Arabic}
\quranayah[28][86]
\end{Arabic}}
\flushleft{\begin{hindi}
इससे मेरा परवरदिगार ख़ूब वाक़िफ है और तुमको तो ये उम्मीद न थी कि तुम्हारे पास ख़ुदा की तरफ से किताब नाज़िल की जाएगी मगर तुम्हारे परवरदिगार की मेहरबानी से नाज़िल हुई तो तुम हरग़िज़ काफिरों के पुश्त पनाह न बनना
\end{hindi}}
\flushright{\begin{Arabic}
\quranayah[28][87]
\end{Arabic}}
\flushleft{\begin{hindi}
कहीं ऐसा न हो एहकामे ख़ुदा वन्दी नाज़िल होने के बाद तुमको ये लोग उनकी तबलीग़ से रोक दें और तुम अपने परवरदिगार की तरफ (लोगों को) बुलाते जाओ और ख़बरदार मुशरेकीन से हरगिज़ न होना
\end{hindi}}
\flushright{\begin{Arabic}
\quranayah[28][88]
\end{Arabic}}
\flushleft{\begin{hindi}
और ख़ुदा के सिवा किसी और माबूद की परसतिश न करना उसके सिवा कोई क़ाबिले परसतिश नहीं उसकी ज़ात के सिवा हर चीज़ फना होने वाली है उसकी हुकूमत है और तुम लोग उसकी तरफ़ (मरने के बाद) लौटाये जाओगे
\end{hindi}}
\chapter{Al-'Ankabut (The Spider)}
\begin{Arabic}
\Huge{\centerline{\basmalah}}\end{Arabic}
\flushright{\begin{Arabic}
\quranayah[29][1]
\end{Arabic}}
\flushleft{\begin{hindi}
अलिफ़ लाम मीम
\end{hindi}}
\flushright{\begin{Arabic}
\quranayah[29][2]
\end{Arabic}}
\flushleft{\begin{hindi}
क्या लोगों ने ये समझ लिया है कि (सिर्फ) इतना कह देने से कि हम ईमान लाए छोड़ दिए जाएँगे और उनका इम्तेहान न लिया जाएगा
\end{hindi}}
\flushright{\begin{Arabic}
\quranayah[29][3]
\end{Arabic}}
\flushleft{\begin{hindi}
और हमने तो उन लोगों का भी इम्तिहान लिया जो उनसे पहले गुज़र गए ग़रज़ ख़ुदा उन लोगों को जो सच्चे (दिल से ईमान लाए) हैं यक़ीनन अलहाएदा देखेगा और झूठों को भी (अलहाएदा) ज़रुर देखेगा
\end{hindi}}
\flushright{\begin{Arabic}
\quranayah[29][4]
\end{Arabic}}
\flushleft{\begin{hindi}
क्या जो लोग बुरे बुरे काम करते हैं उन्होंने ये समझ लिया है कि वह हमसे (बचकर) निकल जाएँगे (अगर ऐसा है तो) ये लोग क्या ही बुरे हुक्म लगाते हैं
\end{hindi}}
\flushright{\begin{Arabic}
\quranayah[29][5]
\end{Arabic}}
\flushleft{\begin{hindi}
जो शख्स ख़ुदा से मिलने (क़यामत के आने) की उम्मीद रखता है तो (समझ रखे कि) ख़ुदा की (मुक़र्रर की हुई) मीयाद ज़रुर आने वाली है और वह (सबकी) सुनता (और) जानता है
\end{hindi}}
\flushright{\begin{Arabic}
\quranayah[29][6]
\end{Arabic}}
\flushleft{\begin{hindi}
और जो शख्स (इबादत में) कोशिश करता है तो बस अपने ही वास्ते कोशिश करता है (क्योंकि) इसमें तो शक ही नहीं कि ख़ुदा सारे जहाँन (की इबादत) से बेनियाज़ है
\end{hindi}}
\flushright{\begin{Arabic}
\quranayah[29][7]
\end{Arabic}}
\flushleft{\begin{hindi}
और जिन लोगों ने ईमान क़ुबूल किया और अच्छे अच्छे काम किए हम यक़ीनन उनके गुनाहों की तरफ से कफ्फारा क़रार देगें और ये (दुनिया में) जो आमाल करते थे हम उनके आमाल की उन्हें अच्छी से अच्छी जज़ा अता करेंगे
\end{hindi}}
\flushright{\begin{Arabic}
\quranayah[29][8]
\end{Arabic}}
\flushleft{\begin{hindi}
और हमने इन्सान को अपने माँ बाप से अच्छा बरताव करने का हुक्म दिया है और (ये भी कि) अगर तुझे तेरे माँ बाप इस बात पर मजबूर करें कि ऐसी चीज़ को मेरा शरीक बना जिन (के शरीक होने) का मुझे इल्म तक नहीं तो उनका कहना न मानना तुम सबको (आख़िर एक दिन) मेरी तरफ लौट कर आना है मै जो कुछ तुम लोग (दुनिया में) करते थे बता दूँगा
\end{hindi}}
\flushright{\begin{Arabic}
\quranayah[29][9]
\end{Arabic}}
\flushleft{\begin{hindi}
और जिन लोगों ने ईमान क़ुबूल किया और अच्छे अच्छे काम किए हम उन्हें (क़यामत के दिन) ज़रुर नेको कारों में दाख़िल करेंगे
\end{hindi}}
\flushright{\begin{Arabic}
\quranayah[29][10]
\end{Arabic}}
\flushleft{\begin{hindi}
और कुछ लोग ऐसे भी हैं जो (ज़बान से तो) कह देते हैं कि हम ख़ुदा पर ईमान लाए फिर जब उनको ख़ुदा के बारे में कुछ तकलीफ़ पहुँची तो वह लोगों की तकलीफ़ देही को अज़ाब के बराबर ठहराते हैं और (ऐ रसूल) अगर तुम्हारे पास तुम्हारे परवरदिगार की मदद आ पहुँची और तुम्हें फतेह हुई तो यही लोग कहने लगते हैं कि हम भी तो तुम्हारे साथ ही साथ थे भला जो कुछ सारे जहाँन के दिलों में है क्या ख़ुदा बख़ूबी वाक़िफ नहीं (ज़रुर है)
\end{hindi}}
\flushright{\begin{Arabic}
\quranayah[29][11]
\end{Arabic}}
\flushleft{\begin{hindi}
और जिन लोगों ने ईमान क़ुबूल किया ख़ुदा उनको यक़ीनन जानता है और मुनाफेक़ीन को भी ज़रुर जानता है
\end{hindi}}
\flushright{\begin{Arabic}
\quranayah[29][12]
\end{Arabic}}
\flushleft{\begin{hindi}
और कुफ्फार ईमान वालों से कहने लगे कि हमारे तरीक़े पर चलो और (क़यामत में) तुम्हारे गुनाहों (के बोझ) को हम (अपने सर) ले लेंगे हालॉकि ये लोग ज़रा भी तो उनके गुनाह उठाने वाले नहीं ये लोग यक़ीनी झूठे हैं
\end{hindi}}
\flushright{\begin{Arabic}
\quranayah[29][13]
\end{Arabic}}
\flushleft{\begin{hindi}
और (हाँ) ये लोग अपने (गुनाह के) बोझे तो यक़ीनी उठाएँगें ही और अपने बोझो के साथ जिन्हें गुमराह किया उनके बोझे भी उठाएँगे और जो इफ़ितेरा परदाज़िया ये लोग करते रहे हैं क़यामत के दिन उन से ज़रुर उसकी बाज़पुर्स होगी
\end{hindi}}
\flushright{\begin{Arabic}
\quranayah[29][14]
\end{Arabic}}
\flushleft{\begin{hindi}
और हमने नूह को उनकी क़ौम के पास (पैग़म्बर बनाकर) भेजा तो वह उनमें पचास कम हज़ार बरस रहे (और हिदायत किया किए और जब न माना) तो आख़िर तूफान ने उन्हें ले डाला और वह उस वक्त भी सरकश ही थे
\end{hindi}}
\flushright{\begin{Arabic}
\quranayah[29][15]
\end{Arabic}}
\flushleft{\begin{hindi}
फिर हमने नूह और कश्ती में रहने वालों को बचा लिया और हमने इस वाक़िये को सारी ख़ुदाई के वास्ते (अपनी क़ुदरत की) निशानी क़रार दी
\end{hindi}}
\flushright{\begin{Arabic}
\quranayah[29][16]
\end{Arabic}}
\flushleft{\begin{hindi}
और इबराहीम को (याद करो) जब उन्होंने कहा कि (भाईयों) ख़ुदा की इबादत करो और उससे डरो अगर तुम समझते बूझते हो तो यही तुम्हारे हक़ में बेहतर है
\end{hindi}}
\flushright{\begin{Arabic}
\quranayah[29][17]
\end{Arabic}}
\flushleft{\begin{hindi}
(मगर) तुम लोग तो ख़ुदा को छोड़कर सिर्फ बुतों की परसतिश करते हैं और झूठी बातें (अपने दिल से) गढ़ते हो इसमें तो शक ही नहीं कि ख़ुदा को छोड़कर जिन लोगों की तुम परसतिश करते हो वह तुम्हारी रोज़ी का एख्तेयार नही रखते-बस ख़ुदा ही से रोज़ी भी माँगों और उसकी इबादत भी करो उसका शुक्र करो (क्योंकि) तुम लोग (एक दिन) उसी की तरफ लौटाए जाओगे
\end{hindi}}
\flushright{\begin{Arabic}
\quranayah[29][18]
\end{Arabic}}
\flushleft{\begin{hindi}
और (ऐ अहले मक्का) अगर तुमने (मेरे रसूल को) झुठलाया तो (कुछ परवाह नहीं) तुमसे पहले भी तो बहुतेरी उम्मते (अपने पैग़म्बरों को) झुठला चुकी हैं और रसूल के ज़िम्मे तो सिर्फ (एहक़ाम का) पहुँचा देना है
\end{hindi}}
\flushright{\begin{Arabic}
\quranayah[29][19]
\end{Arabic}}
\flushleft{\begin{hindi}
बस क्या उन लोगों ने इस पर ग़ौर नहीं किया कि ख़ुदा किस तरह मख़लूकात को पहले पहल पैदा करता है और फिर उसको दोबारा पैदा करेगा ये तो ख़ुदा के नज़दीक बहुत आसान बात है
\end{hindi}}
\flushright{\begin{Arabic}
\quranayah[29][20]
\end{Arabic}}
\flushleft{\begin{hindi}
(ऐ रसूल इन लोगों से) तुम कह दो कि ज़रा रुए ज़मीन पर चलफिर कर देखो तो कि ख़ुदा ने किस तरह पहले पहल मख़लूक को पैदा किया फिर (उसी तरह वही) ख़ुदा (क़यामत के दिन) आखिरी पैदाइश पैदा करेगा- बेशक ख़ुदा हर चीज़ पर क़ादिर है
\end{hindi}}
\flushright{\begin{Arabic}
\quranayah[29][21]
\end{Arabic}}
\flushleft{\begin{hindi}
जिस पर चाहे अज़ाब करे और जिस पर चाहे रहम करे और तुम लोग (सब के सब) उसी की तरफ लौटाए जाओगे
\end{hindi}}
\flushright{\begin{Arabic}
\quranayah[29][22]
\end{Arabic}}
\flushleft{\begin{hindi}
और न तो तुम ज़मीन ही में ख़ुदा को ज़ेर कर सकते हो और न आसमान में और ख़ुदा के सिवा न तो तुम्हारा कोई सरपरस्त है और न मददगार
\end{hindi}}
\flushright{\begin{Arabic}
\quranayah[29][23]
\end{Arabic}}
\flushleft{\begin{hindi}
और जिन लोगों ने ख़ुदा की आयतों और (क़यामत के दिन) उसके सामने हाज़िर होने से इन्कार किया मेरी रहमत से मायूस हो गए हैं और उन्हीं लोगों के वास्ते दर्दनाक अज़ाब है
\end{hindi}}
\flushright{\begin{Arabic}
\quranayah[29][24]
\end{Arabic}}
\flushleft{\begin{hindi}
ग़रज़ इबराहीम की क़ौम के पास (इन बातों का) इसके सिवा कोई जवाब न था कि बाहम कहने लगे इसको मार डालो या जला (कर ख़ाक) कर डालो (आख़िर वह कर गुज़रे) तो ख़ुदा ने उनको आग से बचा लिया इसमें शक नहीं कि दुनियादार लोगों के वास्ते इस वाकिये में (कुदरते ख़ुदा की) बहुत सी निशानियाँ हैं
\end{hindi}}
\flushright{\begin{Arabic}
\quranayah[29][25]
\end{Arabic}}
\flushleft{\begin{hindi}
और इबराहीम ने (अपनी क़ौम से) कहा कि तुम लोगों ने ख़ुदा को छोड़कर बुतो को सिर्फ दुनिया की ज़िन्दगी में बाहम मोहब्त करने की वजह से (ख़ुदा) बना रखा है फिर क़यामत के दिन तुम में से एक का एक इनकार करेगा और एक दूसरे पर लानत करेगा और (आख़िर) तुम लोगों का ठिकाना जहन्नुम है और (उस वक्त तुम्हारा कोई भी मददगार न होगा)
\end{hindi}}
\flushright{\begin{Arabic}
\quranayah[29][26]
\end{Arabic}}
\flushleft{\begin{hindi}
तब सिर्फ लूत इबराहीम पर ईमान लाए और इबराहीम ने कहा मै तो देस को छोड़कर अपने परवरदिगार की तरफ (जहाँ उसको मंज़ूर हो ) निकल जाऊँगा
\end{hindi}}
\flushright{\begin{Arabic}
\quranayah[29][27]
\end{Arabic}}
\flushleft{\begin{hindi}
इसमे शक नहीं कि वह ग़ालिब (और) हिकमत वाला है और हमने इबराहीम को इसहाक़ (सा बेटा) और याक़ूब (सा पोता) अता किया और उनकी नस्ल में पैग़म्बरी और किताब क़रार दी और हम न इबराहीम को दुनिया में भी अच्छा बदला अता किया और वह तो आख़ेरत में भी यक़ीनी नेको कारों से हैं
\end{hindi}}
\flushright{\begin{Arabic}
\quranayah[29][28]
\end{Arabic}}
\flushleft{\begin{hindi}
(और ऐ रसूल) लूत को (याद करो) जब उन्होंने अपनी क़ौम से कहा कि तुम लोग अजब बेहयाई का काम करते हो कि तुमसे पहले सारी खुदायी के लोगों में से किसी ने नहीं किया
\end{hindi}}
\flushright{\begin{Arabic}
\quranayah[29][29]
\end{Arabic}}
\flushleft{\begin{hindi}
तुम लोग (औरतों को छोड़कर कज़ाए शहवत के लिए) मर्दों की तरफ गिरते हो और (मुसाफिरों की) रहजनी करते हो और तुम लोग अपनी महफिलों में बुरी बुरी हरकते करते हो तो (इन सब बातों का) लूत की क़ौम के पास इसके सिवा कोई जवाब न था कि वह लोग कहने लगे कि भला अगर तुम सच्चे हो तो हम पर ख़ुदा का अज़ाब तो ले आओ
\end{hindi}}
\flushright{\begin{Arabic}
\quranayah[29][30]
\end{Arabic}}
\flushleft{\begin{hindi}
तब लूत ने दुआ की कि परवरदिगार इन मुफ़सिद लोगों के मुक़ाबले में मेरी मदद कर
\end{hindi}}
\flushright{\begin{Arabic}
\quranayah[29][31]
\end{Arabic}}
\flushleft{\begin{hindi}
(उस वक्त अज़ाब की तैयारी हुई) और जब हमारे भेजे हुए फ़रिश्ते इबराहीम के पास (बुढ़ापे में बेटे की) खुशखबरी लेकर आए तो (इबराहीम से) बोले हम लोग अनक़रीब इस गाँव के रहने वालों को हलाक करने वाले हैं (क्योंकि) इस बस्ती के रहने वाले यक़ीनी (बड़े) सरकश है
\end{hindi}}
\flushright{\begin{Arabic}
\quranayah[29][32]
\end{Arabic}}
\flushleft{\begin{hindi}
(ये सुन कर) इबराहीम ने कहा कि इस बस्ती में तो लूत भी है वह फरिश्ते बोले जो लोग इस बस्ती में हैं हम लोग उनसे खूब वाक़िफ़ हैं हम तो उनको और उनके लड़के बालों को यक़ीनी बचा लेंगे मगर उनकी बीबी को वह (अलबता) पीछे रह जाने वालों में होगीं
\end{hindi}}
\flushright{\begin{Arabic}
\quranayah[29][33]
\end{Arabic}}
\flushleft{\begin{hindi}
और जब हमारे भेजे हुए फरिश्ते लूत के पास आए लूत उनके आने से ग़मग़ीन हुए और उन (की मेहमानी) से दिल तंग हुए (क्योंकि वह नौजवान खूबसूरत मर्दों की सरूत में आए थे) फरिश्तों ने कहा आप ख़ौफ न करें और कुढ़े नही हम आपको और आपके लड़के बालों को बचा लेगें मगर आपकी बीबी (क्योंकि वह पीछे रह जाने वालो से होगी)
\end{hindi}}
\flushright{\begin{Arabic}
\quranayah[29][34]
\end{Arabic}}
\flushleft{\begin{hindi}
हम यक़ीनन इसी बस्ती के रहने वालों पर चूँकि ये लोग बदकारियाँ करते रहे एक आसमानी अज़ाब नाज़िल करने वाले हैं
\end{hindi}}
\flushright{\begin{Arabic}
\quranayah[29][35]
\end{Arabic}}
\flushleft{\begin{hindi}
और हमने यक़ीनी उस (उलटी हुई बस्ती) में से समझदार लोगों के वास्ते (इबरत की) एक वाज़ेए व रौशन निशानी बाक़ी रखी है
\end{hindi}}
\flushright{\begin{Arabic}
\quranayah[29][36]
\end{Arabic}}
\flushleft{\begin{hindi}
और (हमने) मदियन के रहने वालों के पास उनके भाई शुएब को पैग़म्बर बनाकर भेजा उन्होंने (अपनी क़ौम से) कहा ऐ मेरी क़ौम ख़ुदा की इबादत करो और रोज़े आखेरत की उम्मीद रखो और रुए ज़मीन में फ़साद न फैलाते फिरो
\end{hindi}}
\flushright{\begin{Arabic}
\quranayah[29][37]
\end{Arabic}}
\flushleft{\begin{hindi}
तो उन लोगों ने शुऐब को झुठलाया पस ज़लज़ले (भूचाल) ने उन्हें ले डाला- तो वह लोग अपने घरों में औंधे ज़ानू के बल पड़े रह गए
\end{hindi}}
\flushright{\begin{Arabic}
\quranayah[29][38]
\end{Arabic}}
\flushleft{\begin{hindi}
और क़ौम आद और समूद को (भी हलाक कर डाला) और (ऐ अहले मक्का) तुम को तो उनके (उजड़े हुए) घर भी (रास्ता आते जाते) मालूम हो चुके और शैतान ने उनकी नज़र में उनके कामों को अच्छा कर दिखाया था और उन्हें (सीधी) राह (चलने) से रोक दिया था हालॉकि वह बड़े होशियार थे
\end{hindi}}
\flushright{\begin{Arabic}
\quranayah[29][39]
\end{Arabic}}
\flushleft{\begin{hindi}
और (हम ही ने) क़ारुन व फिरऔन व हामान को भी (हलाक कर डाला) हालॉकि उन लोगों के पास मूसा वाजेए व रौशन मौजिज़े लेकर आए फिर भी ये लोग रुए ज़मीन में सरकशी करते फिरे और हमसे (निकल कर) कहीं आगे न बढ़ सके
\end{hindi}}
\flushright{\begin{Arabic}
\quranayah[29][40]
\end{Arabic}}
\flushleft{\begin{hindi}
तो हमने सबको उनके गुनाह की सज़ा में ले डाला चुनांन्चे उनमे से बाज़ तो वह थे जिन पर हमने पत्थर वाली ऑंधी भेजी और बाज़ उनमें से वह थे जिन को एक सख्त चिंघाड़ ने ले डाला और बाज़ उनमें से वह थे जिनको हमने ज़मीन मे धॅसा दिया और बाज़ उनमें से वह थे जिन्हें हमने डुबो मारा और ये बात नहीं कि ख़ुदा ने उन पर ज़ुल्म किया हो बल्कि (सच युं है कि) ये लोग ख़ुद (ख़ुदा की नाफ़रमानी करके) आप अपने ऊपर ज़ुल्म करते रहे
\end{hindi}}
\flushright{\begin{Arabic}
\quranayah[29][41]
\end{Arabic}}
\flushleft{\begin{hindi}
जिन लोगों ने ख़ुदा के सिवा दूसरे कारसाज़ बना रखे हैं उनकी मसल उस मकड़ी की सी है जिसने (अपने ख्याल नाक़िस में) एक घर बनाया और उसमें तो शक ही नहीं कि तमाम घरों से बोदा घर मकड़ी का होता है मगर ये लोग (इतना भी) जानते हो
\end{hindi}}
\flushright{\begin{Arabic}
\quranayah[29][42]
\end{Arabic}}
\flushleft{\begin{hindi}
ख़ुदा को छोड़कर ये लोग जिस चीज़ को पुकारते हैं उससे ख़ुदा यक़ीनी वाक़िफ है और वह तो (सब पर) ग़ालिब (और) हिकमत वाला है
\end{hindi}}
\flushright{\begin{Arabic}
\quranayah[29][43]
\end{Arabic}}
\flushleft{\begin{hindi}
और हम ये मिसाले लोगों के (समझाने) के वास्ते बयान करते हैं और उन को तो बस उलमा ही समझते हैं
\end{hindi}}
\flushright{\begin{Arabic}
\quranayah[29][44]
\end{Arabic}}
\flushleft{\begin{hindi}
ख़ुदा ने सारे आसमान और ज़मीन को बिल्कुल ठीक पैदा किया इसमें शक नहीं कि उसमें ईमानदारों के वास्ते (कुदरते ख़ुदा की) यक़ीनी बड़ी निशानी है
\end{hindi}}
\flushright{\begin{Arabic}
\quranayah[29][45]
\end{Arabic}}
\flushleft{\begin{hindi}
(ऐ रसूल) जो किताब तुम्हारे पास नाज़िल की गयी है उसकी तिलावत करो और पाबन्दी से नमाज़ पढ़ो बेशक नमाज़ बेहयाई और बुरे कामों से बाज़ रखती है और ख़ुदा की याद यक़ीनी बड़ा मरतबा रखती है और तुम लोग जो कुछ करते हो ख़ुदा उससे वाक़िफ है
\end{hindi}}
\flushright{\begin{Arabic}
\quranayah[29][46]
\end{Arabic}}
\flushleft{\begin{hindi}
और (ऐ ईमानदारों) अहले किताब से मनाज़िरा न किया करो मगर उमदा और शाएस्ता अलफाज़ व उनवान से लेकिन उनमें से जिन लोगों ने तुम पर ज़ुल्म किया (उनके साथ रिआयत न करो) और साफ साफ कह दो कि जो किताब हम पर नाज़िल हुई और जो किताब तुम पर नाज़िल हुई है हम तो सब पर ईमान ला चुके और हमारा माबूद और तुम्हारा माबूद एक ही है और हम उसी के फरमाबरदार है
\end{hindi}}
\flushright{\begin{Arabic}
\quranayah[29][47]
\end{Arabic}}
\flushleft{\begin{hindi}
और (ऐ रसूल जिस तरह अगले पैग़म्बरों पर किताबें उतारी) उसी तरह हमने तुम्हारे पास किताब नाज़िल की तो जिन लोगों को हमने (पहले) किताब अता की है वह उस पर भी ईमान रखते हैं और (अरबो) में से बाज़ वह हैं जो उस पर ईमान रखते हैं और हमारी आयतों के तो बस पक्के कट्टर काफिर ही मुनकिर है
\end{hindi}}
\flushright{\begin{Arabic}
\quranayah[29][48]
\end{Arabic}}
\flushleft{\begin{hindi}
और (ऐ रसूल) क़ुरान से पहले न तो तुम कोई किताब ही पढ़ते थे और न अपने हाथ से तुम लिखा करते थे ऐसा होता तो ये झूठे ज़रुर (तुम्हारी नबुवत में) शक करते
\end{hindi}}
\flushright{\begin{Arabic}
\quranayah[29][49]
\end{Arabic}}
\flushleft{\begin{hindi}
मगर जिन लोगों को (ख़ुदा की तरफ से) इल्म अता हुआ है उनके दिल में ये (क़ुरान) वाजेए व रौशन आयतें हैं और सरकशी के सिवा हमारी आयतो से कोई इन्कार नहीं करता
\end{hindi}}
\flushright{\begin{Arabic}
\quranayah[29][50]
\end{Arabic}}
\flushleft{\begin{hindi}
और (कुफ्फ़ार अरब) कहते हैं कि इस (रसूल) पर उसके परवरदिगार की तरफ से मौजिज़े क्यों नही नाज़िल होते (ऐ रसूल उनसे) कह दो कि मौजिज़े तो बस ख़ुदा ही के पास हैं और मै तो सिर्फ साफ साफ (अज़ाबे ख़ुदा से) डराने वाला हूँ
\end{hindi}}
\flushright{\begin{Arabic}
\quranayah[29][51]
\end{Arabic}}
\flushleft{\begin{hindi}
क्या उनके लिए ये काफी नहीं कि हमने तुम पर क़ुरान नाज़िल किया जो उनके सामने पढ़ा जाता है इसमें शक नहीं कि ईमानदार लोगों के लिए इसमें (ख़ुदा की बड़ी) मेहरबानी और (अच्छी ख़ासी) नसीहत है
\end{hindi}}
\flushright{\begin{Arabic}
\quranayah[29][52]
\end{Arabic}}
\flushleft{\begin{hindi}
तुम कह दो कि मेरे और तुम्हारे दरमियान गवाही के वास्ते ख़ुदा ही काफी है जो सारे आसमान व ज़मीन की चीज़ों को जानता है-और जिन लोगों ने बातिल को माना और ख़ुदा से इन्कार किया वही लोग बड़े घाटे में रहेंगे
\end{hindi}}
\flushright{\begin{Arabic}
\quranayah[29][53]
\end{Arabic}}
\flushleft{\begin{hindi}
और (ऐ रसूल) तुमसे लोग अज़ाब के नाज़िल होने की जल्दी करते हैं और अगर (अज़ाब का) वक्त मुअय्यन न होता तो यक़ीनन उनके पास अब तक अज़ाब आ जाता और (आख़िर एक दिन) उन पर अचानक ज़रुर आ पड़ेगा और उनको ख़बर भी न होगी
\end{hindi}}
\flushright{\begin{Arabic}
\quranayah[29][54]
\end{Arabic}}
\flushleft{\begin{hindi}
ये लोग तुमसे अज़ाब की जल्दी करते हैं और ये यक़ीनी बात है कि दोज़ख़ काफिरों को (इस तरह) घेर कर रहेगी (कि रुक न सकेंगे)
\end{hindi}}
\flushright{\begin{Arabic}
\quranayah[29][55]
\end{Arabic}}
\flushleft{\begin{hindi}
जिस दिन अज़ाब उनके सर के ऊपर से और उनके पॉव के नीचे से उनको ढॉके होगा और ख़ुदा (उनसे) फरमाएगा कि जो जो कारस्तानियॉ तुम (दुनिया में) करते थे अब उनका मज़ा चखो
\end{hindi}}
\flushright{\begin{Arabic}
\quranayah[29][56]
\end{Arabic}}
\flushleft{\begin{hindi}
ऐ मेरे ईमानदार बन्दों मेरी ज़मीन तो यक़ीनन कुशादा है तो तुम मेरी ही इबादत करो
\end{hindi}}
\flushright{\begin{Arabic}
\quranayah[29][57]
\end{Arabic}}
\flushleft{\begin{hindi}
हर शख्स (एक न एक दिन) मौत का मज़ा चखने वाला है फिर तुम सब आख़िर हमारी ही तरफ लौटए जाओंगे
\end{hindi}}
\flushright{\begin{Arabic}
\quranayah[29][58]
\end{Arabic}}
\flushleft{\begin{hindi}
और जिन लोगों ने ईमान क़ुबूल किया और अच्छे अच्छे काम किए उनको हम बेहश्त के झरोखों में जगह देगें जिनके नीचे नहरें जारी हैं जिनमें वह हमेशा रहेंगे (अच्छे चलन वालो की भी क्या ख़ूब ख़री मज़दूरी है)
\end{hindi}}
\flushright{\begin{Arabic}
\quranayah[29][59]
\end{Arabic}}
\flushleft{\begin{hindi}
जिन्होंने (दुनिया में मुसिबतों पर) सब्र किया और अपने परवरदिगार पर भरोसा रखते हैं
\end{hindi}}
\flushright{\begin{Arabic}
\quranayah[29][60]
\end{Arabic}}
\flushleft{\begin{hindi}
और ज़मीन पर चलने वालों में बहुतेरे ऐसे हैं जो अपनी रोज़ी अपने ऊपर लादे नहीं फिरते ख़ुदा ही उनको भी रोज़ी देता है और तुम को भी और वह बड़ा सुनने वाला वाक़िफकार है
\end{hindi}}
\flushright{\begin{Arabic}
\quranayah[29][61]
\end{Arabic}}
\flushleft{\begin{hindi}
(ऐ रसूल) अगर तुम उनसे पूछो कि (भला) किसने सारे आसमान व ज़मीन को पैदा किया और चाँद और सूरज को काम में लगाया तो वह ज़रुर यही कहेंगे कि अल्लाह ने फिर वह कहाँ बहके चले जाते हैं
\end{hindi}}
\flushright{\begin{Arabic}
\quranayah[29][62]
\end{Arabic}}
\flushleft{\begin{hindi}
ख़ुदा ही अपने बन्दों में से जिसकी रोज़ी चाहता है कुशादा कर देता है और जिसके लिए चाहता है तंग कर देता है इसमें शक नहीं कि ख़ुदा ही हर चीज़ से वाक़िफ है
\end{hindi}}
\flushright{\begin{Arabic}
\quranayah[29][63]
\end{Arabic}}
\flushleft{\begin{hindi}
और (ऐ रसूल) अगर तुम उससे पूछो कि किसने आसमान से पानी बरसाया फिर उसके ज़रिये से ज़मीन को इसके मरने (परती होने) के बाद ज़िन्दा (आबाद) किया तो वह ज़रुर यही कहेंगे कि अल्लाह ने (ऐ रसूल) तुम कह दो अल्हम दो लिल्लाह-मगर उनमे से बहुतेरे (इतना भी) नहीं समझते
\end{hindi}}
\flushright{\begin{Arabic}
\quranayah[29][64]
\end{Arabic}}
\flushleft{\begin{hindi}
और ये दुनिया की ज़िन्दगी तो खेल तमाशे के सिवा कुछ नहीं और मगर ये लोग समझें बूझें तो इसमे शक नहीं कि अबदी ज़िन्दगी (की जगह) तो बस आख़ेरत का घर है (बाक़ी लग़ो)
\end{hindi}}
\flushright{\begin{Arabic}
\quranayah[29][65]
\end{Arabic}}
\flushleft{\begin{hindi}
फिर जब ये लोग कश्ती में सवार होते हैं तो निहायत ख़ुलूस से उसकी इबादत करने वाले बन कर ख़ुदा से दुआ करते हैं फिर जब उन्हें ख़ुश्की में (पहुँचा कर) नजात देता है तो फौरन शिर्क करने लगते हैं
\end{hindi}}
\flushright{\begin{Arabic}
\quranayah[29][66]
\end{Arabic}}
\flushleft{\begin{hindi}
ताकि जो (नेअमतें) हमने उन्हें अता की हैं उनका इन्कार कर बैठें और ताकि (दुनिया में) ख़ूब चैन कर लें तो अनक़रीब ही (इसका नतीजा) उन्हें मालूम हो जाएगा
\end{hindi}}
\flushright{\begin{Arabic}
\quranayah[29][67]
\end{Arabic}}
\flushleft{\begin{hindi}
क्या उन लोगों ने इस पर ग़ौर नहीं किया कि हमने हरम (मक्का) को अमन व इत्मेनान की जगह बनाया हालॉकि उनके गिर्द व नवाह से लोग उचक ले जाते हैं तो क्या ये लोग झूठे माबूदों पर ईमान लाते हैं और ख़ुदा की नेअमत की नाशुक्री करते हैं
\end{hindi}}
\flushright{\begin{Arabic}
\quranayah[29][68]
\end{Arabic}}
\flushleft{\begin{hindi}
और जो शख्स ख़ुदा पर झूठ बोहतान बॉधे या जब उसके पास कोई सच्ची बात आए तो झुठला दे इससे बढ़कर ज़ालिम कौन होगा क्या (इन) काफिरों का ठिकाना जहन्नुम में नहीं है (ज़रुर है)
\end{hindi}}
\flushright{\begin{Arabic}
\quranayah[29][69]
\end{Arabic}}
\flushleft{\begin{hindi}
और जिन लोगों ने हमारी राह में जिहाद किया उन्हें हम ज़रुर अपनी राह की हिदायत करेंगे और इसमें शक नही कि ख़ुदा नेकोकारों का साथी है
\end{hindi}}
\chapter{Ar-Rum (The Romans)}
\begin{Arabic}
\Huge{\centerline{\basmalah}}\end{Arabic}
\flushright{\begin{Arabic}
\quranayah[30][1]
\end{Arabic}}
\flushleft{\begin{hindi}
अलिफ़ लाम मीम
\end{hindi}}
\flushright{\begin{Arabic}
\quranayah[30][2]
\end{Arabic}}
\flushleft{\begin{hindi}
(यहाँ से) बहुत क़रीब के मुल्क में रोमी (नसारा अहले फ़ारस आतिश परस्तों से) हार गए
\end{hindi}}
\flushright{\begin{Arabic}
\quranayah[30][3]
\end{Arabic}}
\flushleft{\begin{hindi}
मगर ये लोग अनक़रीब ही अपने हार जाने के बाद चन्द सालों में फिर (अहले फ़ारस पर) ग़ालिब आ जाएँगे
\end{hindi}}
\flushright{\begin{Arabic}
\quranayah[30][4]
\end{Arabic}}
\flushleft{\begin{hindi}
क्योंकि (इससे) पहले और बाद (ग़रज़ हर ज़माने में) हर अम्र का एख्तेयार ख़ुदा ही को है और उस दिन ईमानदार लोग ख़ुदा की मदद से खुश हो जाएँगे
\end{hindi}}
\flushright{\begin{Arabic}
\quranayah[30][5]
\end{Arabic}}
\flushleft{\begin{hindi}
वह जिसकी चाहता है मदद करता है और वह (सब पर) ग़ालिब रहम करने वाला है
\end{hindi}}
\flushright{\begin{Arabic}
\quranayah[30][6]
\end{Arabic}}
\flushleft{\begin{hindi}
(ये) ख़ुदा का वायदा है) ख़ुदा अपने वायदे के ख़िलाफ नहीं किया करता मगर अकसर लोग नहीं जानते हैं
\end{hindi}}
\flushright{\begin{Arabic}
\quranayah[30][7]
\end{Arabic}}
\flushleft{\begin{hindi}
ये लोग बस दुनियावी ज़िन्दगी की ज़ाहिरी हालत को जानते हैं और ये लोग आखेरत से बिल्कुल ग़ाफिल हैं
\end{hindi}}
\flushright{\begin{Arabic}
\quranayah[30][8]
\end{Arabic}}
\flushleft{\begin{hindi}
क्या उन लोगों ने अपने दिल में (इतना भी) ग़ौर नहीं किया कि ख़ुदा ने सारे आसमान और ज़मीन को और जो चीजे उन दोनों के दरमेयान में हैं बस बिल्कुल ठीक और एक मुक़र्रर मियाद के वास्ते पैदा किया है और कुछ शक नहीं कि बहुतेरे लोग तो अपने परवरदिगार की (बारगाह) के हुज़ूर में (क़यामत) ही को किसी तरह नहीं मानते
\end{hindi}}
\flushright{\begin{Arabic}
\quranayah[30][9]
\end{Arabic}}
\flushleft{\begin{hindi}
क्या ये लोग रुए ज़मीन पर चले फिरे नहीं कि देखते कि जो लोग इनसे पहले गुज़र गए उनका अन्जाम कैसा (बुरा) हुआ हालॉकि जो लोग उनसे पहले क़ूवत में भी कहीं ज्यादा थे और जिस क़दर ज़मीन उन लोगों ने आबाद की है उससे कहीं ज्यादा (ज़मीन की) उन लोगों ने काश्त भी की थी और उसको आबाद भी किया था और उनके पास भी उनके पैग़म्बर वाज़ेए व रौशन मौजिज़े लेकर आ चुके थे (मगर उन लोगों ने न माना) तो ख़ुदा ने उन पर कोई ज़ुल्म नहीं किया मगर वह लोग (कुफ्र व सरकशी से) आप अपने ऊपर ज़ुल्म करते रहे
\end{hindi}}
\flushright{\begin{Arabic}
\quranayah[30][10]
\end{Arabic}}
\flushleft{\begin{hindi}
फिर जिन लोगों ने बुराई की थी उनका अन्जाम बुरा ही हुआ क्योंकि उन लोगों ने ख़ुदा की आयतों को झुठलाया था और उनके साथ मसखरा पन किया किए
\end{hindi}}
\flushright{\begin{Arabic}
\quranayah[30][11]
\end{Arabic}}
\flushleft{\begin{hindi}
ख़ुदा ही ने मख़लूकात को पहली बार पैदा किया फिर वही दुबारा (पैदा करेगा) फिर तुम सब लोग उसी की तरफ लौटाए जाओगे
\end{hindi}}
\flushright{\begin{Arabic}
\quranayah[30][12]
\end{Arabic}}
\flushleft{\begin{hindi}
और जिस दिन क़यामत बरपा होगी (उस दिन) गुनेहगार लोग ना उम्मीद होकर रह जाएँगे
\end{hindi}}
\flushright{\begin{Arabic}
\quranayah[30][13]
\end{Arabic}}
\flushleft{\begin{hindi}
और उनके (बनाए हुए ख़ुदा के) शरीकों में से कोई उनका सिफारिशी न होगा और ये लोग ख़़ुद भी अपने शरीकों से इन्कार कर जाएँगे
\end{hindi}}
\flushright{\begin{Arabic}
\quranayah[30][14]
\end{Arabic}}
\flushleft{\begin{hindi}
और जिस दिन क़यामत बरपा होगी उस दिन (मोमिनों से) कुफ्फ़ार जुदा हो जाएँगें
\end{hindi}}
\flushright{\begin{Arabic}
\quranayah[30][15]
\end{Arabic}}
\flushleft{\begin{hindi}
फिर जिन लोगों ने ईमान क़ुबूल किया और अच्छे अच्छे काम किए तो वह बाग़े बेहश्त में निहाल कर दिए जाएँगे
\end{hindi}}
\flushright{\begin{Arabic}
\quranayah[30][16]
\end{Arabic}}
\flushleft{\begin{hindi}
मगर जिन लोगों के कुफ्र एख्तेयार किया और हमारी आयतों और आखेरत की हुज़ूरी को झुठलाया तो ये लोग अज़ाब में गिरफ्तार किए जाएँगे
\end{hindi}}
\flushright{\begin{Arabic}
\quranayah[30][17]
\end{Arabic}}
\flushleft{\begin{hindi}
फिर जिस वक्त तुम लोगों की शाम हो और जिस वक्त तुम्हारी सुबह हो ख़ुदा की पाकीज़गी ज़ाहिर करो
\end{hindi}}
\flushright{\begin{Arabic}
\quranayah[30][18]
\end{Arabic}}
\flushleft{\begin{hindi}
और सारे आसमान व ज़मीन में तीसरे पहर को और जिस वक्त तुम लोगों की दोपहर हो जाए वही क़ाबिले तारीफ़ है
\end{hindi}}
\flushright{\begin{Arabic}
\quranayah[30][19]
\end{Arabic}}
\flushleft{\begin{hindi}
वही ज़िन्दा को मुर्दे से निकालता है और वही मुर्दे को जिन्दा से पैदा करता है और ज़मीन को मरने (परती होने) के बाद ज़िन्दा (आबाद) करता है और इसी तरह तुम लोग भी (मरने के बाद निकाले जाओगे)
\end{hindi}}
\flushright{\begin{Arabic}
\quranayah[30][20]
\end{Arabic}}
\flushleft{\begin{hindi}
और उस (की कुदरत) की निशानियों में ये भी है कि उसने तुमको मिट्टी से पैदा किया फिर यकायक तुम आदमी बनकर (ज़मीन पर) चलने फिरने लगे
\end{hindi}}
\flushright{\begin{Arabic}
\quranayah[30][21]
\end{Arabic}}
\flushleft{\begin{hindi}
और उसी की (क़ुदरत) की निशानियों में से एक ये (भी) है कि उसने तुम्हारे वास्ते तुम्हारी ही जिन्स की बीवियाँ पैदा की ताकि तुम उनके साथ रहकर चैन करो और तुम लोगों के दरमेयान प्यार और उलफ़त पैदा कर दी इसमें शक नहीं कि इसमें ग़ौर करने वालों के वास्ते (क़ुदरते ख़ुदा की) यक़ीनी बहुत सी निशानियाँ हैं
\end{hindi}}
\flushright{\begin{Arabic}
\quranayah[30][22]
\end{Arabic}}
\flushleft{\begin{hindi}
और उस (की कुदरत) की निशानियों में आसमानो और ज़मीन का पैदा करना और तुम्हारी ज़बानो और रंगतो का एख़तेलाफ भी है यकीनन इसमें वाक़िफकारों के लिए बहुत सी निशानियाँ हैं
\end{hindi}}
\flushright{\begin{Arabic}
\quranayah[30][23]
\end{Arabic}}
\flushleft{\begin{hindi}
और रात और दिन को तुम्हारा सोना और उसके फज़ल व करम (रोज़ी) की तलाश करना भी उसकी (क़ुदरत की) निशानियों से है बेशक जो लोग सुनते हैं उनके लिए इसमें (क़ुदरते ख़ुदा की) बहुत सी निशानियाँ हैं
\end{hindi}}
\flushright{\begin{Arabic}
\quranayah[30][24]
\end{Arabic}}
\flushleft{\begin{hindi}
और उसी की (क़ुदरत की) निशानियों में से एक ये भी है कि वह तुमको डराने वाला उम्मीद लाने के वास्ते बिजली दिखाता है और आसमान से पानी बरसाता है और उसके ज़रिए से ज़मीन को उसके परती होने के बाद आबाद करता है बेशक अक्लमंदों के वास्ते इसमें (क़ुदरते ख़ुदा की) बहुत सी दलीलें हैं
\end{hindi}}
\flushright{\begin{Arabic}
\quranayah[30][25]
\end{Arabic}}
\flushleft{\begin{hindi}
और उसी की (क़ुदरत की) निशानियों में से एक ये भी है कि आसमान और ज़मीन उसके हुक्म से क़ायम हैं फिर (मरने के बाद) जिस वक्त तुमको एक बार बुलाएगा तो तुम सबके सब ज़मीन से (ज़िन्दा हो होकर) निकल पड़ोगे
\end{hindi}}
\flushright{\begin{Arabic}
\quranayah[30][26]
\end{Arabic}}
\flushleft{\begin{hindi}
और जो लोग आसमानों में है सब उसी के है और सब उसी के ताबेए फरमान हैं
\end{hindi}}
\flushright{\begin{Arabic}
\quranayah[30][27]
\end{Arabic}}
\flushleft{\begin{hindi}
और वह ऐसा (क़ादिरे मुत्तालिक़ है जो मख़लूकात को पहली बार पैदा करता है फिर दोबारा (क़यामत के दिन) पैदा करेगा और ये उस पर बहुत आसान है और सारे आसमान व जमीन सबसे बालातर उसी की शान है और वही (सब पर) ग़ालिब हिकमत वाला है
\end{hindi}}
\flushright{\begin{Arabic}
\quranayah[30][28]
\end{Arabic}}
\flushleft{\begin{hindi}
और हमने (तुम्हारे समझाने के वास्ते) तुम्हारी ही एक मिसाल बयान की है हमने जो कुछ तुम्हे अता किया है क्या उसमें तुम्हारी लौन्डी गुलामों में से कोई (भी) तुम्हारा शरीक है कि (वह और) तुम उसमें बराबर हो जाओ (और क्या) तुम उनसे ऐसा ही ख़ौफ रखते हो जितना तुम्हें अपने लोगों का (हक़ हिस्सा न देने में) ख़ौफ होता है फिर बन्दों को खुदा का शरीक क्यों बनाते हो) अक्ल मन्दों के वास्ते हम यूँ अपनी आयतों को तफसीलदार बयान करते हैं
\end{hindi}}
\flushright{\begin{Arabic}
\quranayah[30][29]
\end{Arabic}}
\flushleft{\begin{hindi}
मगर सरकशों ने तो बगैर समझे बूझे अपनी नफसियानी ख्वाहिशों की पैरवी कर ली (और ख़ुदा का शरीक ठहरा दिया) ग़रज़ ख़ुदा जिसे गुमराही में छोड़ दे (फिर) उसे कौन राहे रास्त पर ला सकता है और उनका कोई मददगार (भी) नहीं
\end{hindi}}
\flushright{\begin{Arabic}
\quranayah[30][30]
\end{Arabic}}
\flushleft{\begin{hindi}
तो (ऐ रसूल) तुम बातिल से कतरा के अपना रुख़ दीन की तरफ किए रहो यही ख़ुदा की बनावट है जिस पर उसने लोगों को पैदा किया है ख़ुदा की (दुरुस्त की हुई) बनावट में तग़य्युर तबद्दुल (उलट फेर) नहीं हो सकता यही मज़बूत और (बिल्कुल सीधा) दीन है मगर बहुत से लोग नहीं जानते हैं
\end{hindi}}
\flushright{\begin{Arabic}
\quranayah[30][31]
\end{Arabic}}
\flushleft{\begin{hindi}
उसी की तरफ रुजू होकर (ख़ुदा की इबादत करो) और उसी से डरते रहो और पाबन्दी से नमाज़ पढ़ो और मुशरेकीन से न हो जाना
\end{hindi}}
\flushright{\begin{Arabic}
\quranayah[30][32]
\end{Arabic}}
\flushleft{\begin{hindi}
जिन्होंने अपने (असली) दीन में तफरेक़ा परवाज़ी की और मुख्तलिफ़ फिरके क़े बन गए जो (दीन) जिस फिरके क़े पास है उसी में निहाल है
\end{hindi}}
\flushright{\begin{Arabic}
\quranayah[30][33]
\end{Arabic}}
\flushleft{\begin{hindi}
और जब लोगों को कोई मुसीबत छू भी गयी तो उसी की तरफ रुजू होकर अपने परवरदिगार को पुकारने लगते हैं फिर जब वह अपनी रहमत की लज्ज़त चखा देता है तो उन्हीं में से कुछ लोग अपने परवरदिगार के साथ शिर्क करने लगते हैं
\end{hindi}}
\flushright{\begin{Arabic}
\quranayah[30][34]
\end{Arabic}}
\flushleft{\begin{hindi}
ताकि जो (नेअमत) हमने उन्हें दी है उसकी नाशुक्री करें ख़ैर (दुनिया में चन्दरोज़ चैन कर लो) फिर तो बहुत जल्द (अपने किए का मज़ा) तुम्हे मालूम ही होगा
\end{hindi}}
\flushright{\begin{Arabic}
\quranayah[30][35]
\end{Arabic}}
\flushleft{\begin{hindi}
क्या हमने उन लोगों पर कोई दलील नाज़िल की है जो उस (के हक़ होने) को बयान करती है जिसे ये लोग ख़ुदा का शरीक ठहराते हैं (हरग़िज नहीं)
\end{hindi}}
\flushright{\begin{Arabic}
\quranayah[30][36]
\end{Arabic}}
\flushleft{\begin{hindi}
और जब हमने लोगों को (अपनी रहमत की लज्ज़त) चखा दी तो वह उससे खुश हो गए और जब उन्हें अपने हाथों की अगली कारसतानियो की बदौलत कोई मुसीबत पहुँची तो यकबारगी मायूस होकर बैठे रहते हैं
\end{hindi}}
\flushright{\begin{Arabic}
\quranayah[30][37]
\end{Arabic}}
\flushleft{\begin{hindi}
क्या उन लोगों ने (इतना भी) ग़ौर नहीं किया कि खुदा ही जिसकी रोज़ी चाहता है कुशादा कर देता है और (जिसकी चाहता है) तंग करता है-कुछ शक नहीं कि इसमें ईमानरदार लोगों के वास्ते (कुदरत ख़ुदा की) बहुत सी निशानियाँ हैं
\end{hindi}}
\flushright{\begin{Arabic}
\quranayah[30][38]
\end{Arabic}}
\flushleft{\begin{hindi}
(तो ऐ रसूल अपनी) क़राबतदार (फातिमा ज़हरा) का हक़ फिदक दे दो और मोहताज व परदेसियों का (भी) जो लोग ख़ुदा की ख़ुशनूदी के ख्वाहॉ हैं उन के हक़ में सब से बेहतर यही है और ऐसे ही लोग आखेरत में दिली मुरादे पाएँगें
\end{hindi}}
\flushright{\begin{Arabic}
\quranayah[30][39]
\end{Arabic}}
\flushleft{\begin{hindi}
और तुम लोग जो सूद देते हो ताकि लोगों के माल (दौलत) में तरक्क़ी हो तो (याद रहे कि ऐसा माल) ख़ुदा के यहॉ फूलता फलता नही और तुम लोग जो ख़ुदा की ख़ुशनूदी के इरादे से ज़कात देते हो तो ऐसे ही लोग (ख़ुदा की बारगाह से) दूना दून लेने वाले हैं
\end{hindi}}
\flushright{\begin{Arabic}
\quranayah[30][40]
\end{Arabic}}
\flushleft{\begin{hindi}
ख़ुदा वह (क़ादिर तवाना है) जिसने तुमको पैदा किया फिर उसी ने रोज़ी दी फिर वही तुमको मार डालेगा फिर वही तुमको (दोबारा) ज़िन्दा करेगा भला तुम्हारे (बनाए हुए ख़ुदा के) शरीकों में से कोई भी ऐसा है जो इन कामों में से कुछ भी कर सके जिसे ये लोग (उसका) शरीक बनाते हैं
\end{hindi}}
\flushright{\begin{Arabic}
\quranayah[30][41]
\end{Arabic}}
\flushleft{\begin{hindi}
वह उससे पाक व पाकीज़ा और बरतर है ख़़ुद लोगों ही के अपने हाथों की कारस्तानियों की बदौलत ख़ुश्क व तर में फसाद फैल गया ताकि जो कुछ ये लोग कर चुके हैं ख़ुदा उन को उनमें से बाज़ करतूतों का मज़ा चखा दे ताकि ये लोग अब भी बाज़ आएँ
\end{hindi}}
\flushright{\begin{Arabic}
\quranayah[30][42]
\end{Arabic}}
\flushleft{\begin{hindi}
(ऐ रसूल) तुम कह दो कि ज़रा रुए ज़मीन पर चल फिरकर देखो तो कि जो लोग उसके क़ब्ल गुज़र गए उनके (अफ़आल) का अंजाम क्या हुआ उनमें से बहुतेरे तो मुशरिक ही हैं
\end{hindi}}
\flushright{\begin{Arabic}
\quranayah[30][43]
\end{Arabic}}
\flushleft{\begin{hindi}
तो (ऐ रसूल) तुम उस दिन के आने से पहले जो खुदा की तरफ से आकर रहेगा (और) कोई उसे रोक नहीं सकता अपना रुख़ मज़बूत (और सीधे दीन की तरफ किए रहो उस दिन लोग (परेशान होकर) अलग अलग हो जाएँगें
\end{hindi}}
\flushright{\begin{Arabic}
\quranayah[30][44]
\end{Arabic}}
\flushleft{\begin{hindi}
जो काफ़िर बन बैठा उस पर उस के कुफ्र का वबाल है और जिन्होने अच्छे काम किए वह अपने ही आसाइश का सामान कर रहें है
\end{hindi}}
\flushright{\begin{Arabic}
\quranayah[30][45]
\end{Arabic}}
\flushleft{\begin{hindi}
ताकि जो लोग ईमान लाए और अच्छे अच्छे काम किए उनको ख़ुदा अपने फज़ल व (करम) से अच्छी जज़ा अता करेगा वह यक़ीनन कुफ्फ़ार से उलफ़त नहीं रखता
\end{hindi}}
\flushright{\begin{Arabic}
\quranayah[30][46]
\end{Arabic}}
\flushleft{\begin{hindi}
उसी की (क़ुदरत) की निशानियों में से एक ये भी है कि वह हवाओं को (बारिश) की ख़ुशख़बरी के वास्ते (क़ब्ल से) भेज दिया करता है और ताकि तुम्हें अपनी रहमत की लज्ज़त चखाए और इसलिए भी कि (इसकी बदौलत) कश्तियां उसके हुक्म से चल खड़ी हो और ताकि तुम उसके फज़ल व करम से (अपनी रोज़ी) की तलाश करो और इसलिए भी ताकि तुम शुक्र करो
\end{hindi}}
\flushright{\begin{Arabic}
\quranayah[30][47]
\end{Arabic}}
\flushleft{\begin{hindi}
औ (ऐ रसूल) हमने तुमसे पहले और भी बहुत से पैग़म्बर उनकी क़ौमों के पास भेजे तो वह पैग़म्बर वाज़ेए व रौशन लेकर आए (मगर उन लोगों ने न माना) तो उन मुजरिमों से हमने (खूब) बदला लिया और हम पर तो मोमिनीन की मदद करना लाज़िम था ही
\end{hindi}}
\flushright{\begin{Arabic}
\quranayah[30][48]
\end{Arabic}}
\flushleft{\begin{hindi}
ख़ुदा ही (क़ादिर तवाना) है जो हवाओं को भेजता है तो वह बादलों को उड़ाए उड़ाए फिरती हैं फिर वही ख़ुदा बादल को जिस तरह चाहता है आसमान में फैला देता है और (कभी) उसको टुकड़े (टुकड़े) कर देता है फिर तुम देखते हो कि बूँदियां उसके दरमियान से निकल पड़ती हैं फिर जब ख़ुदा उन्हें अपने बन्दों में से जिस पर चहता है बरसा देता है तो वह लोग खुशियाँ माानने लगते हैं
\end{hindi}}
\flushright{\begin{Arabic}
\quranayah[30][49]
\end{Arabic}}
\flushleft{\begin{hindi}
अगरचे ये लोग उन पर (बाराने रहमत) नाज़िल होने से पहले (बारिश से) शुरु ही से बिल्कुल मायूस (और मज़बूर) थे
\end{hindi}}
\flushright{\begin{Arabic}
\quranayah[30][50]
\end{Arabic}}
\flushleft{\begin{hindi}
ग़रज़ ख़ुदा की रहमत के आसार की तरफ देखो तो कि वह क्योंकर ज़मीन को उसकी परती होने के बाद आबाद करता है बेशक यक़ीनी वही मुर्दो को ज़िन्दा करने वाला और वही हर चीज़ पर क़ादिर है
\end{hindi}}
\flushright{\begin{Arabic}
\quranayah[30][51]
\end{Arabic}}
\flushleft{\begin{hindi}
और अगर हम (खेती की नुकसान देह) हवा भेजें फिर लोग खेती को (उसी हवा की वजह से) ज़र्द (परस मुर्दा) देखें तो वह लोग इसके बाद (फौरन) नाशुक्री करने लगें
\end{hindi}}
\flushright{\begin{Arabic}
\quranayah[30][52]
\end{Arabic}}
\flushleft{\begin{hindi}
(ऐ रसूल) तुम तो (अपनी) आवाज़ न मुर्दो ही को सुना सकते हो और न बहरों को सुना सकते हो (ख़ुसूसन) जब वह पीठ फेरकर चले जाएँ
\end{hindi}}
\flushright{\begin{Arabic}
\quranayah[30][53]
\end{Arabic}}
\flushleft{\begin{hindi}
और न तुम अंधों को उनकी गुमराही से (फेरकर) राह पर ला सकते हो तो तुम तो बस उन्हीं लोगों को सुना (समझा) सकते हो जो हमारी आयतों को दिल से मानें फिर यही लोग इस्लाम लाने वाले हैं
\end{hindi}}
\flushright{\begin{Arabic}
\quranayah[30][54]
\end{Arabic}}
\flushleft{\begin{hindi}
खुदा ही तो है जिसने तुम्हें (एक निहायत) कमज़ोर चीज़ (नुत्फे) से पैदा किया फिर उसी ने (तुम में) बचपने की कमज़ोरी के बाद (शबाब की) क़ूवत अता की फिर उसी ने (तुममें जवानी की) क़ूवत के बाद कमज़ोरी और बुढ़ापा पैदा कर दिया वह जो चाहता पैदा करता है-और वही बड़ा वाकिफकार और (हर चीज़ पर) क़ाबू रखता है
\end{hindi}}
\flushright{\begin{Arabic}
\quranayah[30][55]
\end{Arabic}}
\flushleft{\begin{hindi}
और जिस दिन क़यामत बरपा होगी तो गुनाहगार लोग कसमें खाएँगें कि वह (दुनिया में) घड़ी भर से ज्यादा नहीं ठहरे यूँ ही लोग (दुनिया में भी) इफ़तेरा परदाज़ियाँ करते रहे
\end{hindi}}
\flushright{\begin{Arabic}
\quranayah[30][56]
\end{Arabic}}
\flushleft{\begin{hindi}
और जिन लोगों को (ख़ुदा की बारगाह से) इल्म और ईमान दिया गया है जवाब देगें कि (हाए) तुम तो ख़ुदा की किताब के मुताबिक़ रोज़े क़यामत तक (बराबर) ठहरे रहे फिर ये तो क़यामत का ही दिन है मगर तुम लोग तो उसका यक़ीन ही न रखते थे
\end{hindi}}
\flushright{\begin{Arabic}
\quranayah[30][57]
\end{Arabic}}
\flushleft{\begin{hindi}
तो उस दिन सरकश लोगों को न उनकी उज्र माअज़ेरत कुछ काम आएगी और न उनकी सुनवाई होगी
\end{hindi}}
\flushright{\begin{Arabic}
\quranayah[30][58]
\end{Arabic}}
\flushleft{\begin{hindi}
और हमने तो इस कुरान में (लोगों के समझाने को) हर तरह की मिसल बयान कर दी और अगर तुम उनके पास कोई सा मौजिज़ा ले आओ
\end{hindi}}
\flushright{\begin{Arabic}
\quranayah[30][59]
\end{Arabic}}
\flushleft{\begin{hindi}
तो भी यक़ीनन कुफ्फ़ार यही बोल उठेंगे कि तुम लोग निरे दग़ाबाज़ हो जो लोग समझ (और इल्म) नहीं रखते उनके दिलों पर नज़र करके ख़ुदा यू तसदीक़ करता है (कि ये ईमान न लाएँगें)
\end{hindi}}
\flushright{\begin{Arabic}
\quranayah[30][60]
\end{Arabic}}
\flushleft{\begin{hindi}
तो (ऐ रसूल) तुम सब्र करो बेशक ख़ुदा का वायदा सच्चा है और (कहीं) ऐसा न हो कि जो (तुम्हारी) तसदीक़ नहीं करते तुम्हें (बहका कर) ख़फ़ीफ़ करे दें
\end{hindi}}
\chapter{Luqman (Luqman)}
\begin{Arabic}
\Huge{\centerline{\basmalah}}\end{Arabic}
\flushright{\begin{Arabic}
\quranayah[31][1]
\end{Arabic}}
\flushleft{\begin{hindi}
अलिफ़ लाम मीम
\end{hindi}}
\flushright{\begin{Arabic}
\quranayah[31][2]
\end{Arabic}}
\flushleft{\begin{hindi}
ये सूरा हिकमत से भरी हुई किताब की आयतें है
\end{hindi}}
\flushright{\begin{Arabic}
\quranayah[31][3]
\end{Arabic}}
\flushleft{\begin{hindi}
जो (अज़सरतापा) उन लोगों के लिए हिदायत व रहमत है
\end{hindi}}
\flushright{\begin{Arabic}
\quranayah[31][4]
\end{Arabic}}
\flushleft{\begin{hindi}
जो पाबन्दी से नमाज़ अदा करते हैं और ज़कात देते हैं और वही लोग आख़िरत का भी यक़ीन रखते हैं
\end{hindi}}
\flushright{\begin{Arabic}
\quranayah[31][5]
\end{Arabic}}
\flushleft{\begin{hindi}
यही लोग अपने परवरदिगार की हिदायत पर आमिल हैं और यही लोग (क़यामत में) अपनी दिली मुरादें पाएँगे
\end{hindi}}
\flushright{\begin{Arabic}
\quranayah[31][6]
\end{Arabic}}
\flushleft{\begin{hindi}
और लोगों में बाज़ (नज़र बिन हारिस) ऐसा है जो बेहूदा क़िस्से (कहानियाँ) ख़रीदता है ताकि बग़ैर समझे बूझे (लोगों को) ख़ुदा की (सीधी) राह से भड़का दे और आयातें ख़ुदा से मसख़रापन करे ऐसे ही लोगों के लिए बड़ा रुसवा करने वाला अज़ाब है
\end{hindi}}
\flushright{\begin{Arabic}
\quranayah[31][7]
\end{Arabic}}
\flushleft{\begin{hindi}
और जब उसके सामने हमारी आयतें पढ़ी जाती हैं तो शेख़ी के मारे मुँह फेरकर (इस तरह) चल देता है गोया उसने इन आयतों को सुना ही नहीं जैसे उसके दोनो कानों में ठेठी है तो (ऐ रसूल) तुम उसको दर्दनाक अज़ाब की (अभी से) खुशख़बरी दे दे
\end{hindi}}
\flushright{\begin{Arabic}
\quranayah[31][8]
\end{Arabic}}
\flushleft{\begin{hindi}
बेशक जो लोग ईमान लाए और उन्होंने अच्छे काम किए उनके लिए नेअमत के (हरे भरे बेहश्ती) बाग़ हैं कि यो उनमें हमेशा रहेंगे
\end{hindi}}
\flushright{\begin{Arabic}
\quranayah[31][9]
\end{Arabic}}
\flushleft{\begin{hindi}
ये ख़ुदा का पक्का वायदा है और वह तो (सब पर) ग़ालिब हिकमत वाला है
\end{hindi}}
\flushright{\begin{Arabic}
\quranayah[31][10]
\end{Arabic}}
\flushleft{\begin{hindi}
तुम उन्हें देख रहे हो कि उसी ने बग़ैर सुतून के आसमानों को बना डाला और उसी ने ज़मीन पर (भारी भारी) पहाड़ों के लंगर डाल दिए कि (मुबादा) तुम्हें लेकर किसी तरफ जुम्बिश करे और उसी ने हर तरह चल फिर करने वाले (जानवर) ज़मीन में फैलाए और हमने आसमान से पानी बरसाया और (उसके ज़रिए से) ज़मीन में हर रंग के नफ़ीस जोड़े पैदा किए
\end{hindi}}
\flushright{\begin{Arabic}
\quranayah[31][11]
\end{Arabic}}
\flushleft{\begin{hindi}
(ऐ रसूल उनसे कह दो कि) ये तो खुदा की ख़िलक़त है कि (भला) तुम लोग मुझे दिखाओं तो कि जो (जो माबूद) ख़ुदा के सिवा तुमने बना रखे है उन्होंने क्या पैदा किया बल्कि सरकश लोग (कुफ्फ़ार) सरीही गुमराही में (पडे) हैं
\end{hindi}}
\flushright{\begin{Arabic}
\quranayah[31][12]
\end{Arabic}}
\flushleft{\begin{hindi}
और यक़ीनन हम ने लुक़मान को हिकमत अता की (और हुक्म दिया था कि) तुम ख़ुदा का शुक्र करो और जो ख़ुदा का शुक्र करेगा-वह अपने ही फायदे के लिए शुक्र करता है और जिसने नाशुक्री की तो (अपना बिगाड़ा) क्योंकी ख़ुदा तो (बहरहाल) बे परवाह (और) क़ाबिल हमदो सना है
\end{hindi}}
\flushright{\begin{Arabic}
\quranayah[31][13]
\end{Arabic}}
\flushleft{\begin{hindi}
और (वह वक्त याद करो) जब लुक़मान ने अपने बेटे से उसकी नसीहत करते हुए कहा ऐ बेटा (ख़बरदार कभी किसी को) ख़ुदा का शरीक न बनाना (क्योंकि) शिर्क यक़ीनी बड़ा सख्त गुनाह है
\end{hindi}}
\flushright{\begin{Arabic}
\quranayah[31][14]
\end{Arabic}}
\flushleft{\begin{hindi}
(जिस की बख़्शिस नहीं) और हमने इन्सान को जिसे उसकी माँ ने दुख पर दुख सह के पेट में रखा (इसके अलावा) दो बरस में (जाके) उसकी दूध बढ़ाई की (अपने और) उसके माँ बाप के बारे में ताक़ीद की कि मेरा भी शुक्रिया अदा करो और अपने वालदैन का (भी) और आख़िर सबको मेरी तरफ लौट कर जाना है
\end{hindi}}
\flushright{\begin{Arabic}
\quranayah[31][15]
\end{Arabic}}
\flushleft{\begin{hindi}
और अगर तेरे माँ बाप तुझे इस बात पर मजबूर करें कि तू मेरा शरीक ऐसी चीज़ को क़रार दे जिसका तुझे इल्म भी नहीं तो तू (इसमें) उनकी इताअत न करो (मगर तकलीफ़ न पहुँचाना) और दुनिया (के कामों) में उनका अच्छी तरह साथ दे और उन लोगों के तरीक़े पर चल जो (हर बात में) मेरी (ही) तरफ रुजू करे फिर (तो आख़िर) तुम सबकी रुजू मेरी ही तरफ है तब (दुनिया में) जो कुछ तुम करते थे
\end{hindi}}
\flushright{\begin{Arabic}
\quranayah[31][16]
\end{Arabic}}
\flushleft{\begin{hindi}
(उस वक्त उसका अन्जाम) बता दूँगा ऐ बेटा इसमें शक नहीं कि वह अमल (अच्छा हो या बुरा) अगर राई के बराबर भी हो और फिर वह किसी सख्त पत्थर के अन्दर या आसमान में या ज़मीन मे (छुपा हुआ) हो तो भी ख़ुदा उसे (क़यामत के दिन) हाज़िर कर देगा बेशक ख़ुदा बड़ा बारीकबीन वाक़िफकार है
\end{hindi}}
\flushright{\begin{Arabic}
\quranayah[31][17]
\end{Arabic}}
\flushleft{\begin{hindi}
ऐ बेटा नमाज़ पाबन्दी से पढ़ा कर और (लोगों से) अच्छा काम करने को कहो और बुरे काम से रोको और जो मुसीबत तुम पर पडे उस पर सब्र करो (क्योंकि) बेशक ये बड़ी हिम्मत का काम है
\end{hindi}}
\flushright{\begin{Arabic}
\quranayah[31][18]
\end{Arabic}}
\flushleft{\begin{hindi}
और लोगों के सामने (गुरुर से) अपना मुँह न फुलाना और ज़मीन पर अकड़कर न चलना क्योंकि ख़ुदा किसी अकड़ने वाले और इतराने वाले को दोस्त नहीं रखता और अपनी चाल ढाल में मियाना रवी एख्तेयार करो
\end{hindi}}
\flushright{\begin{Arabic}
\quranayah[31][19]
\end{Arabic}}
\flushleft{\begin{hindi}
और दूसरो से बोलने में अपनी आवाज़ धीमी रखो क्योंकि आवाज़ों में तो सब से बुरी आवाज़ (चीख़ने की वजह से) गधों की है
\end{hindi}}
\flushright{\begin{Arabic}
\quranayah[31][20]
\end{Arabic}}
\flushleft{\begin{hindi}
क्या तुम लोगों ने इस पर ग़ौर नहीं किया कि जो कुछ आसमानों में है और जो कुछ ज़मीन में है (ग़रज़ सब कुछ) ख़ुदा ही ने यक़ीनी तुम्हारा ताबेए कर दिया है और तुम पर अपनी ज़ाहिरी और बातिनी नेअमतें पूरी कर दीं और बाज़ लोग (नुसर बिन हारिस वगैरह) ऐसे भी हैं जो (ख्वाह मा ख्वाह) ख़ुदा के बारे में झगड़ते हैं (हालॉकि उनके पास) न इल्म है और न हिदायत है और न कोई रौशन किताब है
\end{hindi}}
\flushright{\begin{Arabic}
\quranayah[31][21]
\end{Arabic}}
\flushleft{\begin{hindi}
और जब उनसे कहा जाता है कि जो (किताब) ख़ुदा ने नाज़िल की है उसकी पैरवी करो तो (छूटते ही) कहते हैं कि नहीं हम तो उसी (तरीक़े से चलेंगे) जिस पर हमने अपने बाप दादाओं को पाया भला अगरचे शैतान उनके बाप दादाओं को जहन्नुम के अज़ाब की तरफ बुलाता रहा हो (तो भी उन्ही की पैरवी करेंगे)
\end{hindi}}
\flushright{\begin{Arabic}
\quranayah[31][22]
\end{Arabic}}
\flushleft{\begin{hindi}
और जो शख्स ख़ुदा के आगे अपना सर (तस्लीम) ख़म करे और वह नेकोकार (भी) हो तो बेशक उसने (ईमान की) मज़बूत रस्सी पकड़ ली और (आख़िर तो) सब कामों का अन्जाम ख़ुदा ही की तरफ है
\end{hindi}}
\flushright{\begin{Arabic}
\quranayah[31][23]
\end{Arabic}}
\flushleft{\begin{hindi}
और (ऐ रसूल) जो काफिर बन बैठे तो तुम उसके कुफ्र से कुढ़ों नही उन सबको तो हमारी तरफ लौट कर आना है तो जो कुछ उन लोगों ने किया है (उसका नतीजा) हम बता देगें बेशक ख़ुदा दिलों के राज़ से (भी) खूब वाक़िफ है
\end{hindi}}
\flushright{\begin{Arabic}
\quranayah[31][24]
\end{Arabic}}
\flushleft{\begin{hindi}
हम उन्हें चन्द रोज़ों तक चैन करने देगें फिर उन्हें मजबूर करके सख्त अज़ाब की तरफ खीच लाएँगें
\end{hindi}}
\flushright{\begin{Arabic}
\quranayah[31][25]
\end{Arabic}}
\flushleft{\begin{hindi}
और (ऐ रसूल) तुम अगर उनसे पूछो कि सारे आसमान और ज़मीन को किसने पैदा किया तो ज़रुर कह देगे कि अल्लाह ने (ऐ रसूल) इस पर तुम कह दो अल्हमदोलिल्लाह मगर उनमें से अक्सर (इतना भी) नहीं जानते हैं
\end{hindi}}
\flushright{\begin{Arabic}
\quranayah[31][26]
\end{Arabic}}
\flushleft{\begin{hindi}
जो कुछ सारे आसमान और ज़मीन में है (सब) ख़ुदा ही का है बेशक ख़ुदा तो (हर चीज़ से) बेपरवा (और बहरहाल) क़ाबिले हम्दो सना है
\end{hindi}}
\flushright{\begin{Arabic}
\quranayah[31][27]
\end{Arabic}}
\flushleft{\begin{hindi}
और जितने दरख्त ज़मीन में हैं सब के सब क़लम बन जाएँ और समन्दर उसकी सियाही बनें और उसके (ख़त्म होने के) बाद और सात समन्दर (सियाही हो जाएँ और ख़ुदा का इल्म और उसकी बातें लिखी जाएँ) तो भी ख़ुदा की बातें ख़त्म न होगीं बेशक ख़ुदा सब पर ग़ालिब (और) दाना (बीना) है
\end{hindi}}
\flushright{\begin{Arabic}
\quranayah[31][28]
\end{Arabic}}
\flushleft{\begin{hindi}
तुम सबका पैदा करना और फिर (मरने के बाद) जिला उठाना एक शख्स के (पैदा करने और जिला उठाने के) बराबर है बेशक ख़ुदा (तुम सब की) सुनता और सब कुछ देख रहा है
\end{hindi}}
\flushright{\begin{Arabic}
\quranayah[31][29]
\end{Arabic}}
\flushleft{\begin{hindi}
क्या तूने ये भी ख्याल न किया कि ख़ुदा ही रात को (बढ़ा के) दिन में दाख़िल कर देता है (तो रात बढ़ जाती है) और दिन को (बढ़ा के) रात में दाख़िल कर देता है (तो दिन बढ़ जाता है) उसी ने आफताब व माहताब को (गोया) तुम्हारा ताबेए बना दिया है कि एक मुक़र्रर मीयाद तक (यूँ ही) चलता रहेगा और (क्या तूने ये भी ख्याल न किया कि) जो कुछ तुम करते हो ख़ुदा उससे ख़ूब वाकिफकार है
\end{hindi}}
\flushright{\begin{Arabic}
\quranayah[31][30]
\end{Arabic}}
\flushleft{\begin{hindi}
ये (सब बातें) इस सबब से हैं कि ख़ुदा ही यक़ीनी बरहक़ (माबूद) है और उस के सिवा जिसको लोग पुकारते हैं यक़ीनी बिल्कुल बातिल और इसमें शक नहीं कि ख़ुदा ही आलीशान और बड़ा रुतबे वाला है
\end{hindi}}
\flushright{\begin{Arabic}
\quranayah[31][31]
\end{Arabic}}
\flushleft{\begin{hindi}
क्या तूने इस पर भी ग़ौर नहीं किया कि ख़ुदा ही के फज़ल से कश्ती दरिया में बहती चलती रहती है ताकि (लकड़ी में ये क़ूवत देकर) तुम लोगों को अपनी (कुदरत की) बाज़ निशानियाँ दिखा दे बेशक उस में भी तमाम सब्र व शुक्र करने वाले (बन्दों) के लिए (कुदरत ख़ुदा की) बहुत सी निशानियाँ दिखा दे बेशक इसमें भी तमाम सब्र व शुक्र करने वाले (बन्दों) के लिए (क़ुदरते ख़ुदा की) बहुत सी निशानियाँ हैं
\end{hindi}}
\flushright{\begin{Arabic}
\quranayah[31][32]
\end{Arabic}}
\flushleft{\begin{hindi}
और जब उन्हें मौज (ऊँची होकर) साएबानों की तरह (ऊपर से) ढॉक लेती है तो निरा खुरा उसी का अक़ीदा रखकर ख़ुदा को पुकारने लगते हैं फिर जब ख़ुदा उनको नजात देकर खुश्की तक पहुँचा देता है तो उनमें से बाज़ तो कुछ देर एतदाल पर रहते हैं (और बाज़ पक्के काफिर) और हमारी (क़ुदरत की) निशानियों से इन्कार तो बस बदएहद और नाशुक्रे ही लोग करते हैं
\end{hindi}}
\flushright{\begin{Arabic}
\quranayah[31][33]
\end{Arabic}}
\flushleft{\begin{hindi}
लोगों अपने परवरदिगार से डरो और उस दिन का ख़ौफ रखो जब न कोई बाप अपने बेटे के काम आएगा और न कोई बेटा अपने बाप के कुछ काम आ सकेगा ख़ुदा का (क़यामत का) वायदा बिल्कुल पक्का है तो (कहीं) तुम लोगों को दुनिया की (चन्द रोज़ा) ज़िन्दगी धोखे में न डाले और न कहीं तुम्हें फरेब देने वाला (शैतान) कुछ फ़रेब दे
\end{hindi}}
\flushright{\begin{Arabic}
\quranayah[31][34]
\end{Arabic}}
\flushleft{\begin{hindi}
बेशक ख़ुदा ही के पास क़यामत (के आने) का इल्म है और वही (जब मौक़ा मुनासिब देखता है) पानी बरसाता है और जो कुछ औरतों के पेट में (नर मादा) है जानता है और कोई शख्स (इतना भी तो) नहीं जानता कि वह ख़़ुद कल क्या करेगा और कोई शख्स ये (भी) नहीं जानता है कि वह किस सर ज़मीन पर मरे (गड़े) गा बेशक ख़ुदा (सब बातों से) आगाह ख़बरदार है
\end{hindi}}
\chapter{As-Sajdah (The Adoration)}
\begin{Arabic}
\Huge{\centerline{\basmalah}}\end{Arabic}
\flushright{\begin{Arabic}
\quranayah[32][1]
\end{Arabic}}
\flushleft{\begin{hindi}
अलिफ़ लाम मीम
\end{hindi}}
\flushright{\begin{Arabic}
\quranayah[32][2]
\end{Arabic}}
\flushleft{\begin{hindi}
इसमे कुछ शक नहीं कि किताब क़ुरान का नाज़िल करना सारे जहाँ के परवरदिगार की तरफ से है
\end{hindi}}
\flushright{\begin{Arabic}
\quranayah[32][3]
\end{Arabic}}
\flushleft{\begin{hindi}
क्या ये लोग (ये कहते हैं कि इसको इस शख्स (रसूल) ने अपनी जी से गढ़ लिया है नहीं ये बिल्कुल तुम्हारे परवरदिगार की तरफ से बरहक़ है ताकि तुम उन लोगों को (ख़ुदा के अज़ाब से) डराओ जिनके पास तुमसे पहले कोई डराने वाला आया ही नहीं ताकि ये लोग राह पर आएँ
\end{hindi}}
\flushright{\begin{Arabic}
\quranayah[32][4]
\end{Arabic}}
\flushleft{\begin{hindi}
ख़ुदा ही तो है जिसने सारे आसमान और ज़मीन और जितनी चीज़े इन दोनो के दरमियान हैं छह: दिन में पैदा की फिर अर्श (के बनाने) पर आमादा हुआ उसके सिवा न कोई तुम्हारा सरपरस्त है न कोई सिफारिशी तो क्या तुम (इससे भी) नसीहत व इबरत हासिल नहीं करते
\end{hindi}}
\flushright{\begin{Arabic}
\quranayah[32][5]
\end{Arabic}}
\flushleft{\begin{hindi}
आसमान से ज़मीन तक के हर अम्र का वही मुद्ब्बिर (व मुन्तज़िम) है फिर ये बन्दोबस्त उस दिन जिस की मिक़दार तुम्हारे शुमार से हज़ार बरस से होगी उसी की बारगाह में पेश होगा
\end{hindi}}
\flushright{\begin{Arabic}
\quranayah[32][6]
\end{Arabic}}
\flushleft{\begin{hindi}
वही (मुदब्बिर) पोशीदा और ज़ाहिर का जानने वाला (सब पर) ग़ालिब मेहरबान है
\end{hindi}}
\flushright{\begin{Arabic}
\quranayah[32][7]
\end{Arabic}}
\flushleft{\begin{hindi}
वह (क़ादिर) जिसने जो चीज़ बनाई (निख सुख से) ख़ूब (दुरुस्त) बनाई और इन्सान की इबतेदाई ख़िलक़त मिट्टी से की
\end{hindi}}
\flushright{\begin{Arabic}
\quranayah[32][8]
\end{Arabic}}
\flushleft{\begin{hindi}
उसकी नस्ल (इन्सानी जिस्म के) खुलासा यानी (नुत्फे के से) ज़लील पानी से बनाई
\end{hindi}}
\flushright{\begin{Arabic}
\quranayah[32][9]
\end{Arabic}}
\flushleft{\begin{hindi}
फिर उस (के पुतले) को दुरुस्त किया और उसमें अपनी तरफ से रुह फूँकी और तुम लोगों के (सुनने के) लिए कान और (देखने के लिए) ऑंखें और (समझने के लिए) दिल बनाएँ (इस पर भी) तुम लोग बहुत कम शुक्र करते हो
\end{hindi}}
\flushright{\begin{Arabic}
\quranayah[32][10]
\end{Arabic}}
\flushleft{\begin{hindi}
और ये लोग कहते हैं कि जब हम ज़मीन में नापैद हो जाएँगे तो क्या हम फिर नया जन्म लेगे (क़यामत से नही) बल्कि ये लोग अपने परवरदिगार के (सामने हुज़ूरी ही) से इन्कार रखते हैं
\end{hindi}}
\flushright{\begin{Arabic}
\quranayah[32][11]
\end{Arabic}}
\flushleft{\begin{hindi}
(ऐ रसूल) तुम कह दो कि मल्कुलमौत जो तुम्हारे ऊपर तैनात है वही तुम्हारी रुहें क़ब्ज़ करेगा उसके बाद तुम सबके सब अपने परवरदिगार की तरफ लौटाए जाओगे
\end{hindi}}
\flushright{\begin{Arabic}
\quranayah[32][12]
\end{Arabic}}
\flushleft{\begin{hindi}
और (ऐ रसूल) तुम को बहुत अफसोस होगा अगर तुम मुजरिमों को देखोगे कि वह (हिसाब के वक्त) अपने परवरदिगार की बारगाह में अपने सर झुकाए खड़े हैं और(अर्ज़ कर रहे हैं) परवरदिगार हमने (अच्छी तरह देखा और सुन लिया तू हमें दुनिया में एक दफा फिर लौटा दे कि हम नेक काम करें
\end{hindi}}
\flushright{\begin{Arabic}
\quranayah[32][13]
\end{Arabic}}
\flushleft{\begin{hindi}
और अब तो हमको (क़यामत का)े पूरा पूरा यक़ीन है और (ख़ुदा फरमाएगा कि) अगर हम चाहते तो दुनिया ही में हर शख़्श को (मजबूर करके) राहे रास्त पर ले आते मगर मेरी तरफ से (रोजे अज़ा) ये बात क़रार पा चुकी है कि मै जहन्नुम को जिन्नात और आदमियों से भर दूँगा
\end{hindi}}
\flushright{\begin{Arabic}
\quranayah[32][14]
\end{Arabic}}
\flushleft{\begin{hindi}
तो चूँकि तुम आज के दिन हुज़ूरी को भूले बैठे थे तो अब उसका मज़ा चखो हमने तुमको क़सदन भुला दिया और जैसी जैसी तुम्हारी करतूतें थीं (उनके बदले) अब हमेशा के अज़ाब के मज़े चखो
\end{hindi}}
\flushright{\begin{Arabic}
\quranayah[32][15]
\end{Arabic}}
\flushleft{\begin{hindi}
हमारी आयतों पर ईमान बस वही लोग लाते हैं कि जिस वक्त उन्हें वह (आयते) याद दिलायी गयीं तो फौरन सजदे में गिर पड़ने और अपने परवरदिगार की हम्दो सना की तस्बीह पढ़ने लगे और ये लोग तकब्बुर नही करते (15) (सजदा)
\end{hindi}}
\flushright{\begin{Arabic}
\quranayah[32][16]
\end{Arabic}}
\flushleft{\begin{hindi}
(रात) के वक्त उनके पहलू बिस्तरों से आशना नहीं होते और (अज़ाब के) ख़ौफ और (रहमत की) उम्मीद पर अपने परवरदिगार की इबादत करते हैं और हमने जो कुछ उन्हें अता किया है उसमें से (ख़ुदा की) राह में ख़र्च करते हैं
\end{hindi}}
\flushright{\begin{Arabic}
\quranayah[32][17]
\end{Arabic}}
\flushleft{\begin{hindi}
उन लोगों की कारगुज़ारियों के बदले में कैसी कैसी ऑंखों की ठन्डक उनके लिए ढकी छिपी रखी है उसको कोई शख़्श जानता ही नहीं
\end{hindi}}
\flushright{\begin{Arabic}
\quranayah[32][18]
\end{Arabic}}
\flushleft{\begin{hindi}
तो क्या जो शख़्श ईमानदार है उस शख़्श के बराबर हो जाएगा जो बदकार है (हरगिज़ नहीं) ये दोनों बराबर नही हो सकते
\end{hindi}}
\flushright{\begin{Arabic}
\quranayah[32][19]
\end{Arabic}}
\flushleft{\begin{hindi}
लेकिन जो लोग ईमान लाए और उन्होंने अच्छे अच्छे काम किए उनके लिए तो रहने सहने के लिए (बेहश्त के) बाग़ात हैं ये सामाने ज़ियाफ़त उन कारगुज़ारियों का बदला है जो वह (दुनिया में) कर चुके थे
\end{hindi}}
\flushright{\begin{Arabic}
\quranayah[32][20]
\end{Arabic}}
\flushleft{\begin{hindi}
और जिन लोगों ने बदकारी की उनका ठिकाना तो (बस) जहन्नुम है वह जब उसमें से निकल जाने का इरादा करेंगे तो उसी में फिर ढकेल दिए जाएँगे और उन से कहा जाएगा कि दोज़ख़ के जिस अज़ाब को तुम झुठलाते थे अब उसके मज़े चखो
\end{hindi}}
\flushright{\begin{Arabic}
\quranayah[32][21]
\end{Arabic}}
\flushleft{\begin{hindi}
और हम यक़ीनी (क़यामत के) बड़े अज़ाब से पहले दुनिया के (मामूली) अज़ाब का मज़ा चखाएँगें जो अनक़रीब होगा ताकि ये लोग अब भी (मेरी तरफ) रुज़ू करें
\end{hindi}}
\flushright{\begin{Arabic}
\quranayah[32][22]
\end{Arabic}}
\flushleft{\begin{hindi}
और जिस शख़्श को उसके परवरदिगार की आयतें याद दिलायी जाएँ और वह उनसे मुँह फेर उससे बढ़कर और ज़ालिम कौन होगा हम गुनाहगारों से इन्तक़ाम लेगें और ज़रुर लेंगे
\end{hindi}}
\flushright{\begin{Arabic}
\quranayah[32][23]
\end{Arabic}}
\flushleft{\begin{hindi}
और (ऐ रसूल) हमने तो मूसा को भी (आसमानी किताब) तौरेत अता की थी तुम भी इस किताब (कुरान) के (अल्लाह की तरफ से) मिलने में शक में न पड़े रहो और हमने इस (तौरेत) तो तुम को भी बनी इसराईल के लिए रहनुमा क़रार दिया था
\end{hindi}}
\flushright{\begin{Arabic}
\quranayah[32][24]
\end{Arabic}}
\flushleft{\begin{hindi}
और उन्ही (बनी इसराईल) में से हमने कुछ लोगों को चूंकि उन्होंने (मुसीबतों पर) सब्र किया था पेशवा बनाया जो हमारे हुक्म से (लोगो की) हिदायत करते थे और (इसके अलावा) हमारी आयतो का दिल से यक़ीन रखते थे
\end{hindi}}
\flushright{\begin{Arabic}
\quranayah[32][25]
\end{Arabic}}
\flushleft{\begin{hindi}
(ऐ रसूल) हसमें शक़ नहीं कि जिन बातों में लोग (दुनिया में) बाहम झगड़ते रहते हैं क़यामत के दिन तुम्हारा परवरदिगार क़तई फैसला कर देगा
\end{hindi}}
\flushright{\begin{Arabic}
\quranayah[32][26]
\end{Arabic}}
\flushleft{\begin{hindi}
क्या उन लोगों को ये मालूम नहीं कि हमने उनसे पहले कितनी उम्मतों को हलाक कर डाला जिन के घरों में ये लोग चल फिर रहें हैं बेशक उसमे (कुदरते ख़ुदा की) बहुत सी निशानियाँ हैं तो क्या ये लोग सुनते नहीं हैं
\end{hindi}}
\flushright{\begin{Arabic}
\quranayah[32][27]
\end{Arabic}}
\flushleft{\begin{hindi}
क्या इन लोगों ने इस पर भी ग़ौर नहीं किया कि हम चटियल मैदान (इफ़तादा) ज़मीन की तरफ पानी को जारी करते हैं फिर उसके ज़रिए से हम घास पात लगाते हैं जिसे उनके जानवर और ये ख़ुद भी खाते हैं तो क्या ये लोग इतना भी नहीं देखते
\end{hindi}}
\flushright{\begin{Arabic}
\quranayah[32][28]
\end{Arabic}}
\flushleft{\begin{hindi}
और ये लोग कहते है कि अगर तुम लोग सच्चे हो (कि क़यामत आएगी) तो (आख़िर) ये फैसला कब होगा
\end{hindi}}
\flushright{\begin{Arabic}
\quranayah[32][29]
\end{Arabic}}
\flushleft{\begin{hindi}
(ऐ रसूल) तुम कह दो कि फैसले के दिन कुफ्फ़ार को उनका ईमान लाना कुछ काम न आएगा और न उनको (इसकी) मोहलत दी जाएगी
\end{hindi}}
\flushright{\begin{Arabic}
\quranayah[32][30]
\end{Arabic}}
\flushleft{\begin{hindi}
ग़रज़ तुम उनकी बातों का ख्याल छोड़ दो और तुम मुन्तज़िर रहो (आख़िर) वह लोग भी तो इन्तज़ार कर रहे हैं
\end{hindi}}
\chapter{Al-Ahzab (The Allies)}
\begin{Arabic}
\Huge{\centerline{\basmalah}}\end{Arabic}
\flushright{\begin{Arabic}
\quranayah[33][1]
\end{Arabic}}
\flushleft{\begin{hindi}
ऐ नबी खुदा ही से डरते रहो और काफिरों और मुनाफिक़ों की बात न मानो इसमें शक नहीं कि खुदा बड़ा वाक़िफकार हकीम है।
\end{hindi}}
\flushright{\begin{Arabic}
\quranayah[33][2]
\end{Arabic}}
\flushleft{\begin{hindi}
और तुम्हारे परवरदिगार की तरफ से तुम्हारे पास जो ''वही'' की जाती है (बस) उसी की पैरवी करो तुम लोग जो कुछ कर रहे हो खुदा उससे यक़ीनी अच्छा तरह आगाह है।
\end{hindi}}
\flushright{\begin{Arabic}
\quranayah[33][3]
\end{Arabic}}
\flushleft{\begin{hindi}
और खुदा ही पर भरोसा रखो और खुदा ही कारसाजी के लिए काफी है
\end{hindi}}
\flushright{\begin{Arabic}
\quranayah[33][4]
\end{Arabic}}
\flushleft{\begin{hindi}
ख़ुदा ने किसी आदमी के सीने में दो दिल नहीं पैदा किये कि (एक ही वक्त दो इरादे कर सके) और न उसने तुम्हारी बीवियों को जिन से तुम जेहार करते हो तुम्हारी माएँ बना दी और न उसने तुम्हारे लिये पालकों को तुम्हारे बेटे बना दिये। ये तो फ़क़त तुम्हारी मुँह बोली बात (और ज़ुबानी जमा खर्च) है और (चाहे किसी को बुरी लगे या अच्छी) खुदा तो सच्ची कहता है और सीधी राह दिखाता है।
\end{hindi}}
\flushright{\begin{Arabic}
\quranayah[33][5]
\end{Arabic}}
\flushleft{\begin{hindi}
लिये पालकों का उनके (असली) बापों के नाम से पुकारा करो यही खुदा के नज़दीक बहुत ठीक है हाँ अगर तुम लोग उनके असली बापों को न जानते हो तो तुम्हारे दीनी भाई और दोस्त हैं (उन्हें भाई या दोस्त कहकर पुकारा करो) और हाँ इसमें भूल चूक जाओ तो अलबत्ता उसका तुम पर कोई इल्ज़ाम नहीं है मगर जब तुम दिल से जानबूझ कर करो (तो ज़रूर गुनाह है) और खुदा तो बड़ा बख्शने वाला मेहरबान है।
\end{hindi}}
\flushright{\begin{Arabic}
\quranayah[33][6]
\end{Arabic}}
\flushleft{\begin{hindi}
नबी तो मोमिनीन से खुद उनकी जानों से भी बढ़कर हक़ रखते हैं (क्योंकि वह गोया उम्मत के मेहरबान बाप हैं) और उनकी बीवियाँ (गोया) उनकी माएँ हैं और मोमिनीन व मुहाजिरीन में से (जो लोग बाहम) क़राबतदार हैं। किताबें खुदा की रूह से (ग़ैरों की निस्बत) एक दूसरे के (तर्के के) ज्यादा हक़दार हैं मगर (जब) तुम अपने दोस्तों के साथ सुलूक करना चाहो (तो दूसरी बात है) ये तो किताबे (खुदा) में लिखा हुआ (मौजूद) है
\end{hindi}}
\flushright{\begin{Arabic}
\quranayah[33][7]
\end{Arabic}}
\flushleft{\begin{hindi}
और (ऐ रसूल वह वक्त याद करो) जब हमने और पैग़म्बरों से और ख़ास तुमसे और नूह और इबराहीम और मूसा और मरियम के बेटे ईसा से एहदो पैमाने लिया और उन लोगों से हमने सख्त एहद लिया था
\end{hindi}}
\flushright{\begin{Arabic}
\quranayah[33][8]
\end{Arabic}}
\flushleft{\begin{hindi}
ताकि (क़यामत के दिन) सच्चों (पैग़म्बरों) से उनकी सच्चाई तबलीग़े रिसालत का हाल दरियाफ्त करें और काफिरों के वास्ते तो उसने दर्दनाक अज़ाब तैयार ही कर रखा है।
\end{hindi}}
\flushright{\begin{Arabic}
\quranayah[33][9]
\end{Arabic}}
\flushleft{\begin{hindi}
(ऐ ईमानदारों खुदा की) उन नेअमतों को याद करो जो उसने तुम पर नाज़िल की हैं (जंगे खन्दक में) जब तुम पर (काफिरों का) लशकर (उमड़ के) आ पड़ा तो (हमने तुम्हारी मदद की) उन पर ऑंधी भेजी और (इसके अलावा फरिश्तों का ऐसा लश्कर भेजा) जिसको तुमने देखा तक नहीं और तुम जो कुछ कर रहे थे खुदा उसे खूब देख रहा था
\end{hindi}}
\flushright{\begin{Arabic}
\quranayah[33][10]
\end{Arabic}}
\flushleft{\begin{hindi}
जिस वक्त वह लोग तुम पर तुम्हारे ऊपर से आ पड़े और तुम्हारे नीचे की तरफ से भी पिल गए और जिस वक्त (उनकी कसरत से) तुम्हारी ऑंखें ख़ैरा हो गयीं थी और (ख़ौफ से) कलेजे मुँह को आ गए थे और ख़ुदा पर तरह-तरह के (बुरे) ख्याल करने लगे थे।
\end{hindi}}
\flushright{\begin{Arabic}
\quranayah[33][11]
\end{Arabic}}
\flushleft{\begin{hindi}
यहाँ पर मोमिनों का इम्तिहान लिया गया था और ख़ूब अच्छी तरह झिंझोड़े गए थे।
\end{hindi}}
\flushright{\begin{Arabic}
\quranayah[33][12]
\end{Arabic}}
\flushleft{\begin{hindi}
और जिस वक्त मुनाफेक़ीन और वह लोग जिनके दिलों में (कुफ्र का) मरज़ था कहने लगे थे कि खुदा ने और उसके रसूल ने जो हमसे वायदे किए थे वह बस बिल्कुल धोखे की टट्टी था।
\end{hindi}}
\flushright{\begin{Arabic}
\quranayah[33][13]
\end{Arabic}}
\flushleft{\begin{hindi}
और अब उनमें का एक गिरोह कहने लगा था कि ऐ मदीने वालों अब (दुश्मन के मुक़ाबलें में ) तुम्हारे कहीं ठिकाना नहीं तो (बेहतर है कि अब भी) पलट चलो और उनमें से कुछ लोग रसूल से (घर लौट जाने की) इजाज़त माँगने लगे थे कि हमारे घर (मर्दों से) बिल्कुल ख़ाली (गैर महफूज़) पड़े हुए हैं - हालाँकि वह ख़ाली (ग़ैर महफूज़) न थे (बल्कि) वह लोग तो (इसी बहाने से) बस भागना चाहते हैं
\end{hindi}}
\flushright{\begin{Arabic}
\quranayah[33][14]
\end{Arabic}}
\flushleft{\begin{hindi}
और अगर ऐसा ही लश्कर उन लोगों पर मदीने के एतराफ से आ पड़े और उन से फसाद (ख़ाना जंगी) करने की दरख्वास्त की जाए तो ये लोग उसके लिए (फौरन) आ मौजूद हों
\end{hindi}}
\flushright{\begin{Arabic}
\quranayah[33][15]
\end{Arabic}}
\flushleft{\begin{hindi}
और (उस वक्त) अपने घरों में भी बहुत कम तवक्क़ुफ़ करेंगे (मगर ये तो जिहाद है) हालाँकि उन लोगों ने पहले ही खुदा से एहद किया था कि हम दुश्मन के मुक़ाबले में (अपनी) पीठ न फेरेंगे और खुदा के एहद की पूछगछ तो (एक न एक दिन) होकर रहेगी
\end{hindi}}
\flushright{\begin{Arabic}
\quranayah[33][16]
\end{Arabic}}
\flushleft{\begin{hindi}
(ऐ रसूल उनसे) कह दो कि अगर तुम मौत का क़त्ल (के ख़ौफ) से भागे भी तो (यह) भागना तुम्हें हरगिज़ कुछ भी मुफ़ीद न होगा और अगर तुम भागकर बच भी गए तो बस यही न की दुनिया में चन्द रोज़ा और चैनकर लो
\end{hindi}}
\flushright{\begin{Arabic}
\quranayah[33][17]
\end{Arabic}}
\flushleft{\begin{hindi}
(ऐ रसूल) तुम उनसे कह दो कि अगर खुदा तुम्हारे साथ बुराई का इरादा कर बैठे तो तुम्हें उसके (अज़ाब) से कौन ऐसा है जो बचाए या भलाई ही करना चाहे (तो कौन रोक सकता है) और ये लोग खुदा के सिवा न तो किसी को अपना सरपरस्त पाएँगे और न मद्दगार
\end{hindi}}
\flushright{\begin{Arabic}
\quranayah[33][18]
\end{Arabic}}
\flushleft{\begin{hindi}
तुममें से जो लोग (दूसरों को जिहाद से) रोकते हैं खुदा उनको खूब जानता है और (उनको भी खूब जानता है) जो अपने भाई बन्दों से कहते हैं कि हमारे पास चले भी आओ और खुद भी (फक़त पीछा छुड़ाने को लड़ाई के खेत) में बस एक ज़रा सा आकर तुमसे अपनी जान चुराई
\end{hindi}}
\flushright{\begin{Arabic}
\quranayah[33][19]
\end{Arabic}}
\flushleft{\begin{hindi}
और चल दिए और जब (उन पर) कोई ख़ौफ (का मौक़ा) आ पड़ा तो देखते हो कि (आस से) तुम्हारी तरफ देखते हैं (और) उनकी ऑंखें इस तरह घूमती हैं जैसे किसी शख्स पर मौत की बेहोशी छा जाए फिर वह ख़ौफ (का मौक़ा) जाता रहा और ईमानदारों की फतेह हुई तो माले (ग़नीमत) पर गिरते पड़ते फौरन तुम पर अपनी तेज़ ज़बानों से ताना कसने लगे ये लोग (शुरू) से ईमान ही नहीं लाए (फक़त ज़बानी जमा ख़र्च थी) तो खुदा ने भी इनका किया कराया सब अकारत कर दिया और ये तो खुदा के वास्ते एक (निहायत) आसान बात थी
\end{hindi}}
\flushright{\begin{Arabic}
\quranayah[33][20]
\end{Arabic}}
\flushleft{\begin{hindi}
(मदीने का मुहासेरा करने वाले चल भी दिए मगर) ये लोग अभी यही समझ रहे हैं कि (काफ़िरों के) लश्कर अभी नहीं गए और अगर कहीं (कुफ्फार का) लश्कर फिर आ पहुँचे तो ये लोग चाहेंगे कि काश वह जंगलों में गँवारों में जा बसते और (वहीं से बैठे बैठे) तुम्हारे हालात दरयाफ्त करते रहते और अगर उनको तुम लोगों में रहना पड़ता तो फ़क़त (पीछा छुड़ाने को) ज़रा ज़हूर (कहीं) लड़ते
\end{hindi}}
\flushright{\begin{Arabic}
\quranayah[33][21]
\end{Arabic}}
\flushleft{\begin{hindi}
(मुसलमानों) तुम्हारे वास्ते तो खुद रसूल अल्लाह का (ख़न्दक़ में बैठना) एक अच्छा नमूना था (मगर हाँ यह) उस शख्स के वास्ते है जो खुदा और रोजे आखेरत की उम्मीद रखता हो और खुदा की याद बाकसरत करता हो
\end{hindi}}
\flushright{\begin{Arabic}
\quranayah[33][22]
\end{Arabic}}
\flushleft{\begin{hindi}
और जब सच्चे ईमानदारों ने (कुफ्फार के) जमघटों को देखा तो (बेतकल्लुफ़) कहने लगे कि ये वही चीज़ तो है जिसका हम से खुदा ने और उसके रसूल ने वायदा किया था (इसकी परवाह क्या है) और खुदा ने और उसके रसूल ने बिल्कुल ठीक कहा था और (इसके देखने से) उनका ईमानदार और उनकी इताअत और भी ज़िन्दा हो गयी
\end{hindi}}
\flushright{\begin{Arabic}
\quranayah[33][23]
\end{Arabic}}
\flushleft{\begin{hindi}
ईमानदारों में से कुछ लोग ऐसे भी हैं कि खुदा से उन्होंने (जॉनिसारी का) जो एहद किया था उसे पूरा कर दिखाया ग़रज़ उनमें से बाज़ वह हैं जो (मर कर) अपना वक्त पूरा कर गए और उनमें से बाज़ (हुक्मे खुदा के) मुन्तज़िर बैठे हैं और उन लोगों ने (अपनी बात) ज़रा भी नहीं बदली
\end{hindi}}
\flushright{\begin{Arabic}
\quranayah[33][24]
\end{Arabic}}
\flushleft{\begin{hindi}
ये इम्तेहान इसलिए था ताकि खुद सच्चे (ईमानदारों) को उनकी सच्चाई की जज़ाए ख़ैर दे और अगर चाहे तो मुनाफेक़ीन की सज़ा करे या (अगर वह लागे तौबा करें तो) खुदा उनकी तौबा कुबूल फरमाए इसमें शक नहीं कि खुदा बड़ा बख्शने वाला मेहरबान है
\end{hindi}}
\flushright{\begin{Arabic}
\quranayah[33][25]
\end{Arabic}}
\flushleft{\begin{hindi}
और खुदा ने (अपनी कुदरत से) क़ाफिरों को मदीने से फेर दिया (और वह लोग) अपनी झुंझलाहट में (फिर गए) और इन्हें कुछ फायदे भी न हुआ और खुदा ने (अपनी मेहरबानी से) मोमिनीन को लड़ने की नौबत न आने दी और खुदा तो (बड़ा) ज़बरदस्त (और) ग़ालिब हैं
\end{hindi}}
\flushright{\begin{Arabic}
\quranayah[33][26]
\end{Arabic}}
\flushleft{\begin{hindi}
और अहले किताब में से जिन लोगों (बनी कुरैज़ा) ने उन (कुफ्फार) की मदद की थी खुदा उनको उनके क़िलों से (बेदख़ल करके) नीचे उतार लाया और उनके दिलों में (तुम्हारा) ऐसा रोब बैठा दिया कि तुम उनके कुछ लोगों को क़त्ल करने लगे
\end{hindi}}
\flushright{\begin{Arabic}
\quranayah[33][27]
\end{Arabic}}
\flushleft{\begin{hindi}
और कुछ को क़ैदी (और गुलाम) बनाने और तुम ही लोगों को उनकी ज़मीन और उनके घर और उनके माल और उस ज़मीन (खैबर) का खुदा ने मालिक बना दिया जिसमें तुमने क़दम तक नहीं रखा था और खुदा तो हर चीज़ पर क़ादिर वतवाना है
\end{hindi}}
\flushright{\begin{Arabic}
\quranayah[33][28]
\end{Arabic}}
\flushleft{\begin{hindi}
ऐ रसूल अपनी बीवियों से कह दो कि अगर तुम (फक़त) दुनियावी ज़िन्दगी और उसकी आराइश व ज़ीनत की ख्वाहॉ हो तो उधर आओ मैं तुम लोगों को कुछ साज़ो सामान दे दूँ और उनवाने शाइस्ता से रूख़सत कर दूँ
\end{hindi}}
\flushright{\begin{Arabic}
\quranayah[33][29]
\end{Arabic}}
\flushleft{\begin{hindi}
और अगर तुम लोग खुदा और उसके रसूल और आखेरत के घर की ख्वाहॉ हो तो (अच्छी तरह ख्याल रखो कि) तुम लोगों में से नेकोकार औरतों के लिए खुदा ने यक़ीनन् बड़ा (बड़ा) अज्र व (सवाब) मुहय्या कर रखा है
\end{hindi}}
\flushright{\begin{Arabic}
\quranayah[33][30]
\end{Arabic}}
\flushleft{\begin{hindi}
ऐ पैग़म्बर की बीबियों तुममें से जो कोई किसी सरीही ना शाइस्ता हरकत की का मुरतिब हुई तो (याद रहे कि) उसका अज़ाब भी दुगना बढ़ा दिया जाएगा और खुदा के वास्ते (निहायत) आसान है
\end{hindi}}
\flushright{\begin{Arabic}
\quranayah[33][31]
\end{Arabic}}
\flushleft{\begin{hindi}
और तुममें से जो (बीवी) खुदा और उसके रसूल की ताबेदारी अच्छे (अच्छे) काम करेगी उसको हम उसका सवाब भी दोहरा अता करेगें और हमने उसके लिए (जन्नत में) इज्ज़त की रोज़ी तैयार कर रखी है
\end{hindi}}
\flushright{\begin{Arabic}
\quranayah[33][32]
\end{Arabic}}
\flushleft{\begin{hindi}
ऐ नबी की बीवियों तुम और मामूली औरतों की सी तो हो वही (बस) अगर तुम को परहेज़गारी मंजूर रहे तो (अजनबी आदमी से) बात करने में नरम नरम (लगी लिपटी) बात न करो ताकि जिसके दिल में (शहवते ज़िना का) मर्ज़ है वह (कुछ और) इरादा (न) करे
\end{hindi}}
\flushright{\begin{Arabic}
\quranayah[33][33]
\end{Arabic}}
\flushleft{\begin{hindi}
और (साफ-साफ) उनवाने शाइस्ता से बात किया करो और अपने घरों में निचली बैठी रहो और अगले ज़माने जाहिलियत की तरह अपना बनाव सिंगार न दिखाती फिरो और पाबन्दी से नमाज़ पढ़ा करो और (बराबर) ज़कात दिया करो और खुदा और उसके रसूल की इताअत करो ऐ (पैग़म्बर के) अहले बैत खुदा तो बस ये चाहता है कि तुमको (हर तरह की) बुराई से दूर रखे और जो पाक व पाकीज़ा दिखने का हक़ है वैसा पाक व पाकीज़ा रखे
\end{hindi}}
\flushright{\begin{Arabic}
\quranayah[33][34]
\end{Arabic}}
\flushleft{\begin{hindi}
और (ऐ नबी की बीबियों) तुम्हारे घरों में जो खुदा की आयतें और (अक़ल व हिकमत की बातें) पढ़ी जाती हैं उनको याद रखो कि बेशक ख़ुदा बड़ा बारीक है वाक़िफकार है
\end{hindi}}
\flushright{\begin{Arabic}
\quranayah[33][35]
\end{Arabic}}
\flushleft{\begin{hindi}
(दिल लगा के सुनो) मुसलमान मर्द और मुसलमान औरतें और ईमानदार मर्द और ईमानदार औरतें और फरमाबरदार मर्द और फरमाबरदार औरतें और रास्तबाज़ मर्द और रास्तबाज़ औरतें और सब्र करने वाले मर्द और सब्र करने वाली औरतें और फिरौतनी करने वाले मर्द और फिरौतनी करने वाली औरतें और Âैरात करने वाले मर्द और Âैरात करने वाली औरतें और रोज़ादार मर्द और रोज़ादार औरतें और अपनी शर्मगाहों की हिफाज़त करने वाले मर्द और हिफाज़त करने वाली औरतें और खुदा की बकसरत याद करने वाले मर्द और याद करने वाली औरतें बेशक इन सब लोगों के वास्ते खुदा ने मग़फिरत और (बड़ा) सवाब मुहैय्या कर रखा है
\end{hindi}}
\flushright{\begin{Arabic}
\quranayah[33][36]
\end{Arabic}}
\flushleft{\begin{hindi}
और न किसी ईमानदार मर्द को ये मुनासिब है और न किसी ईमानदार औरत को जब खुदा और उसके रसूल किसी काम का हुक्म दें तो उनको अपने उस काम (के करने न करने) अख़तेयार हो और (याद रहे कि) जिस शख्स ने खुदा और उसके रसूल की नाफरमानी की वह यक़ीनन खुल्लम खुल्ला गुमराही में मुब्तिला हो चुका
\end{hindi}}
\flushright{\begin{Arabic}
\quranayah[33][37]
\end{Arabic}}
\flushleft{\begin{hindi}
और (ऐ रसूल वह वक्त याद करो) जब तुम उस शख्स (ज़ैद) से कह रहे थे जिस पर खुदा ने एहसान (अलग) किया था और तुमने उस पर (अलग) एहसान किया था कि अपनी बीबी (ज़ैनब) को अपनी ज़ौज़ियत में रहने दे और खुदा से डेर खुद तुम इस बात को अपने दिल में छिपाते थे जिसको (आख़िरकार) खुदा ज़ाहिर करने वाला था और तुम लोगों से डरते थे हालॉकि खुदा इसका ज्यादा हक़दार है कि तुम उस से डरो ग़रज़ जब ज़ैद अपनी हाजत पूरी कर चुका (तलाक़ दे दी) तो हमने (हुक्म देकर) उस औरत (ज़ैनब) का निकाह तुमसे कर दिया ताकि आम मोमिनीन को अपने ले पालक लड़कों की बीवियों (से निकाह करने) में जब वह अपना मतलब उन औरतों से पूरा कर चुकें (तलाक़ दे दें) किसी तरह की तंगी न रहे और खुदा का हुक्म तो किया कराया हुआ (क़तई) होता है
\end{hindi}}
\flushright{\begin{Arabic}
\quranayah[33][38]
\end{Arabic}}
\flushleft{\begin{hindi}
जो हुक्म खुदा ने पैग़म्बर पर फर्ज क़र दिया (उसके करने) में उस पर कोई मुज़ाएका नहीं जो लोग (उनसे) पहले गुज़र चुके हैं उनके बारे में भी खुदा का (यही) दस्तूर (जारी) रहा है (कि निकाह में तंगी न की) और खुदा का हुक्म तो (ठीक अन्दाज़े से) मुक़र्रर हुआ होता है
\end{hindi}}
\flushright{\begin{Arabic}
\quranayah[33][39]
\end{Arabic}}
\flushleft{\begin{hindi}
वह लोग जो खुदा के पैग़ामों को (लोगों तक जूँ का तूँ) पहुँचाते थे और उससे डरते थे और खुदा के सिवा किसी से नहीं डरते थे (फिर तुम क्यों डरते हो) और हिसाब लेने के वास्ते तो खुद काफ़ी है
\end{hindi}}
\flushright{\begin{Arabic}
\quranayah[33][40]
\end{Arabic}}
\flushleft{\begin{hindi}
(लोगों) मोहम्मद तुम्हारे मर्दों में से (हक़ीक़तन) किसी के बाप नहीं हैं (फिर जैद की बीवी क्यों हराम होने लगी) बल्कि अल्लाह के रसूल और नबियों की मोहर (यानी ख़त्म करने वाले) हैं और खुदा तो हर चीज़ से खूब वाक़िफ है
\end{hindi}}
\flushright{\begin{Arabic}
\quranayah[33][41]
\end{Arabic}}
\flushleft{\begin{hindi}
ऐ ईमानवालों बाकसरत खुदा की याद किया करो और
\end{hindi}}
\flushright{\begin{Arabic}
\quranayah[33][42]
\end{Arabic}}
\flushleft{\begin{hindi}
सुबह व शाम उसकी तसबीह करते रहो
\end{hindi}}
\flushright{\begin{Arabic}
\quranayah[33][43]
\end{Arabic}}
\flushleft{\begin{hindi}
वह वही तो है जो खुद तुमपर दूरूद (दर्दों रहमत) भेजता है और उसके फ़रिश्ते ताकि तुमको (कुफ़्र की) तारीक़ियों से निकालकर (ईमान की) रौशनी में ले जाए और खुदा ईमानवालों पर बड़ा मेहरबान है
\end{hindi}}
\flushright{\begin{Arabic}
\quranayah[33][44]
\end{Arabic}}
\flushleft{\begin{hindi}
जिस दिन उसकी बारगाह में हाज़िर होंगे (उस दिन) उनकी मुरादात (उसकी तरफ से हर क़िस्म की) सलामती होगी और खुदा ने तो उनके वास्ते बहुत अच्छा बदला (बेहश्त) तैयार रखा है
\end{hindi}}
\flushright{\begin{Arabic}
\quranayah[33][45]
\end{Arabic}}
\flushleft{\begin{hindi}
ऐ नबी हमने तुमको (लोगों का) गवाह और (नेकों को बेहश्त की) खुशख़बरी देने वाला और बदों को अज़ाब से डराने वाला
\end{hindi}}
\flushright{\begin{Arabic}
\quranayah[33][46]
\end{Arabic}}
\flushleft{\begin{hindi}
और खुदा की तरफ उसी के हुक्म से बुलाने वाला और (ईमान व हिदायत का) रौशन चिराग़ बनाकर भेजा
\end{hindi}}
\flushright{\begin{Arabic}
\quranayah[33][47]
\end{Arabic}}
\flushleft{\begin{hindi}
और तुम मोमिनीन को खुशख़बरी दे दो कि उनके लिए खुदा की तरफ से बहुत बड़ी (मेहरबानी और) बख्शिश है
\end{hindi}}
\flushright{\begin{Arabic}
\quranayah[33][48]
\end{Arabic}}
\flushleft{\begin{hindi}
और (ऐ रसूल) तुम (कहीं) काफिरों और मुनाफिक़ों की इताअत न करना और उनकी ईज़ारसानी का ख्याल छोड़ दो और खुदा पर भरोसा रखो और कारसाज़ी में खुदा काफ़ी है
\end{hindi}}
\flushright{\begin{Arabic}
\quranayah[33][49]
\end{Arabic}}
\flushleft{\begin{hindi}
ऐ ईमानवालों जब तुम मोमिना औरतों से (बग़ैर मेहर मुक़र्रर किये) निकाह करो उसके बाद उन्हें अपने हाथ लगाने से पहले ही तलाक़ दे दो तो फिर तुमको उनपर कोई हक़ नहीं कि (उनसे) इद्दा पूरा कराओ उनको तो कुछ (कपड़े रूपये वग़ैरह) देकर उनवाने शाइस्ता से रूख़सत कर दो
\end{hindi}}
\flushright{\begin{Arabic}
\quranayah[33][50]
\end{Arabic}}
\flushleft{\begin{hindi}
ऐ नबी हमने तुम्हारे वास्ते तुम्हारी उन बीवियों को हलाल कर दिया है जिनको तुम मेहर दे चुके हो और तुम्हारी उन लौंडियों को (भी) जो खुदा ने तुमको (बग़ैर लड़े-भिड़े) माले ग़नीमत में अता की है और तुम्हारे चचा की बेटियाँ और तुम्हारी फूफियों की बेटियाँ और तुम्हारे मामू की बेटियाँ और तुम्हारी ख़ालाओं की बेटियाँ जो तुम्हारे साथ हिजरत करके आयी हैं (हलाल कर दी और हर ईमानवाली औरत (भी हलाल कर दी) अगर वह अपने को (बग़ैर मेहर) नबी को दे दें और नबी भी उससे निकाह करना चाहते हों मगर (ऐ रसूल) ये हुक्म सिर्फ तुम्हारे वास्ते ख़ास है और मोमिनीन के लिए नहीं और हमने जो कुछ (मेहर या क़ीमत) आम मोमिनीन पर उनकी बीवियों और उनकी लौंडियों के बारे में मुक़र्रर कर दिया है हम खूब जानते हैं और (तुम्हारी रिआयत इसलिए है) ताकि तुमको (बीवियों की तरफ से) कोई दिक्क़त न हो और खुदा तो बड़ा बख़शने वाला मेहरबान है
\end{hindi}}
\flushright{\begin{Arabic}
\quranayah[33][51]
\end{Arabic}}
\flushleft{\begin{hindi}
इनमें से जिसको (जब) चाहो अलग कर दो और जिसको (जब तक) चाहो अपने पास रखो और जिन औरतों को तुमने अलग कर दिया था अगर फिर तुम उनके ख्वाहॉ हो तो भी तुम पर कोई मज़ाएक़ा नहीं है ये (अख़तेयार जो तुमको दिया गया है) ज़रूर इस क़ाबिल है कि तुम्हारी बीवियों की ऑंखें ठन्डी रहे और आर्जूदा ख़ातिर न हो और वो कुछ तुम उन्हें दे दो सबकी सब उस पर राज़ी रहें और जो कुछ तुम्हारे दिलों में है खुदा उसको ख़ुब जानता है और खुदा तो बड़ा वाक़िफकार बुर्दबार है
\end{hindi}}
\flushright{\begin{Arabic}
\quranayah[33][52]
\end{Arabic}}
\flushleft{\begin{hindi}
(ऐ रसूल) अब उन (नौ) के बाद (और) औरतें तुम्हारे वास्ते हलाल नहीं और न ये जायज़ है कि उनके बदले उनमें से किसी को छोड़कर और बीबियाँ कर लो अगर चे तुमको उनका हुस्न कैसा ही भला (क्यों न) मालूम हो मगर तुम्हारी लौंडियाँ (इस के बाद भी जायज़ हैं) और खुदा तो हर चीज़ का निगरॉ है
\end{hindi}}
\flushright{\begin{Arabic}
\quranayah[33][53]
\end{Arabic}}
\flushleft{\begin{hindi}
ऐ ईमानदारों तुम लोग पैग़म्बर के घरों में न जाया करो मगर जब तुमको खाने के वास्ते (अन्दर आने की) इजाज़त दी जाए (लेकिन) उसके पकने का इन्तेज़ार (नबी के घर बैठकर) न करो मगर जब तुमको बुलाया जाए तो (ठीक वक्त पर) जाओ और फिर जब खा चुको तो (फौरन अपनी अपनी जगह) चले जाया करो और बातों में न लग जाया करो क्योंकि इससे पैग़म्बर को अज़ीयत होती है तो वह तुम्हारा लैहाज़ करते हैं और खुदा तो ठीक (ठीक कहने) से झेंपता नहीं और जब पैग़म्बर की बीवियों से कुछ माँगना हो तो पर्दे के बाहर से माँगा करो यही तुम्हारे दिलों और उनके दिलों के वास्ते बहुत सफाई की बात है और तुम्हारे वास्ते ये जायज़ नहीं कि रसूले खुदा को (किसी तरह) अज़ीयत दो और न ये जायज़ है कि तुम उसके बाद कभी उनकी बीवियों से निकाह करो बेशक ये ख़ुदा के नज़दीक बड़ा (गुनाह) है
\end{hindi}}
\flushright{\begin{Arabic}
\quranayah[33][54]
\end{Arabic}}
\flushleft{\begin{hindi}
चाहे किसी चीज़ को तुम ज़ाहिर करो या उसे छिपाओ खुदा तो (बहरहाल) हर चीज़ से यक़ीनी खूब आगाह है
\end{hindi}}
\flushright{\begin{Arabic}
\quranayah[33][55]
\end{Arabic}}
\flushleft{\begin{hindi}
औरतों पर न अपने बाप दादाओं (के सामने होने) में कुछ गुनाह है और न अपने बेटों के और न अपने भाईयों के और न अपने भतीजों के और अपने भांजों के और न अपनी (क़िस्म कि) औरतों के और न अपनी लौंडियों के सामने होने में कुछ गुनाह है (ऐ पैग़म्बर की बीबियों) तुम लोग खुदा से डरती रहो इसमें कोई शक ही नहीं की खुदा (तुम्हारे आमाल में) हर चीज़ से वाक़िफ़ है
\end{hindi}}
\flushright{\begin{Arabic}
\quranayah[33][56]
\end{Arabic}}
\flushleft{\begin{hindi}
इसमें भी शक नहीं कि खुदा और उसके फरिश्ते पैग़म्बर (और उनकी आल) पर दुरूद भेजते हैं तो ऐ ईमानदारों तुम भी दुरूद भेजते रहो और बराबर सलाम करते रहो
\end{hindi}}
\flushright{\begin{Arabic}
\quranayah[33][57]
\end{Arabic}}
\flushleft{\begin{hindi}
बेशक जो लोग खुदा को और उसके रसूल को अज़ीयत देते हैं उन पर खुदा ने दुनिया और आखेरत (दोनों) में लानत की है और उनके लिए रूसवाई का अज़ाब तैयार कर रखा है
\end{hindi}}
\flushright{\begin{Arabic}
\quranayah[33][58]
\end{Arabic}}
\flushleft{\begin{hindi}
और जो लोग ईमानदार मर्द और ईमानदार औरतों को बगैर कुछ किए द्दरे (तोहमत देकर) अज़ीयत देते हैं तो वह एक बोहतान और सरीह गुनाह का बोझ (अपनी गर्दन पर) उठाते हैं
\end{hindi}}
\flushright{\begin{Arabic}
\quranayah[33][59]
\end{Arabic}}
\flushleft{\begin{hindi}
ऐ नबी अपनी बीवियों और अपनी लड़कियों और मोमिनीन की औरतों से कह दो कि (बाहर निकलते वक्त) अपने (चेहरों और गर्दनों) पर अपनी चादरों का घूंघट लटका लिया करें ये उनकी (शराफ़त की) पहचान के वास्ते बहुत मुनासिब है तो उन्हें कोई छेड़ेगा नहीं और खुदा तो बड़ा बख्शने वाला मेहरबान है
\end{hindi}}
\flushright{\begin{Arabic}
\quranayah[33][60]
\end{Arabic}}
\flushleft{\begin{hindi}
ऐ रसूल मुनाफेक़ीन और वह लोग जिनके दिलों में (कुफ़्र का) मर्ज़ है और जो लोग मदीने में बुरी ख़बरें उड़ाया करते हैं- अगर ये लोग (अपनी शरारतों से) बाज़ न आएंगें तो हम तुम ही को (एक न एक दिन) उन पर मुसल्लत कर देगें फिर वह तुम्हारे पड़ोस में चन्द रोज़ों के सिवा ठहरने (ही) न पाएँगे
\end{hindi}}
\flushright{\begin{Arabic}
\quranayah[33][61]
\end{Arabic}}
\flushleft{\begin{hindi}
लानत के मारे जहाँ कहीं हत्थे चढ़े पकड़े गए और फिर बुरी तरह मार डाले गए
\end{hindi}}
\flushright{\begin{Arabic}
\quranayah[33][62]
\end{Arabic}}
\flushleft{\begin{hindi}
जो लोग पहले गुज़र गए उनके बारे में (भी) खुदा की (यही) आदत (जारी) रही और तुम खुदा की आदत में हरगिज़ तग़य्युर तबद्दुल न पाओगे
\end{hindi}}
\flushright{\begin{Arabic}
\quranayah[33][63]
\end{Arabic}}
\flushleft{\begin{hindi}
(ऐ रसूल) लोग तुमसे क़यामत के बारे में पूछा करते हैं (तुम उनसे) कह दो कि उसका इल्म तो बस खुदा को है और तुम क्या जानो शायद क़यामत क़रीब ही हो
\end{hindi}}
\flushright{\begin{Arabic}
\quranayah[33][64]
\end{Arabic}}
\flushleft{\begin{hindi}
ख़ुदा ने क़ाफिरों पर यक़ीनन लानत की है और उनके लिए जहन्नुम को तैयार कर रखा है
\end{hindi}}
\flushright{\begin{Arabic}
\quranayah[33][65]
\end{Arabic}}
\flushleft{\begin{hindi}
जिसमें वह हमेशा अबदल आबाद रहेंगे न किसी को अपना सरपरस्त पाएँगे न मद्दगार
\end{hindi}}
\flushright{\begin{Arabic}
\quranayah[33][66]
\end{Arabic}}
\flushleft{\begin{hindi}
जिस दिन उनके मुँह जहन्नुम की तरफ फेर दिए जाएँगें तो उस दिन अफ़सोसनाक लहजे में कहेंगे ऐ काश हमने खुदा की इताअत की होती और रसूल का कहना माना होता
\end{hindi}}
\flushright{\begin{Arabic}
\quranayah[33][67]
\end{Arabic}}
\flushleft{\begin{hindi}
और कहेंगे कि परवरदिगारहमने अपने सरदारों अपने बड़ों का कहना माना तो उन्हों ही ने हमें गुमराह कर दिया
\end{hindi}}
\flushright{\begin{Arabic}
\quranayah[33][68]
\end{Arabic}}
\flushleft{\begin{hindi}
परवरदिगारा (हम पर तो अज़ाब सही है मगर) उन लोगों पर दोहरा अज़ाब नाज़िल कर और उन पर बड़ी से बड़ी लानत कर
\end{hindi}}
\flushright{\begin{Arabic}
\quranayah[33][69]
\end{Arabic}}
\flushleft{\begin{hindi}
ऐ ईमानवालों (ख़बरदार कहीं) तुम लोग भी उनके से न हो जाना जिन्होंने मूसा को तकलीफ दी तो खुदा ने उनकी तोहमतों से मूसा को बरी कर दिया और मूसा खुदा के नज़दीक एक रवादार (इज्ज़त करने वाले) (पैग़म्बर) थे
\end{hindi}}
\flushright{\begin{Arabic}
\quranayah[33][70]
\end{Arabic}}
\flushleft{\begin{hindi}
ऐ ईमानवालों खुदा से डरते रहो और (जब कहो तो) दुरूस्त बात कहा करो
\end{hindi}}
\flushright{\begin{Arabic}
\quranayah[33][71]
\end{Arabic}}
\flushleft{\begin{hindi}
तो खुदा तुम्हारी कारगुज़ारियों को दुरूस्त कर देगा और तुम्हारे गुनाह बख्श देगा और जिस शख्स ने खुदा और उसके रसूल की इताअत की वह तो अपनी मुराद को खूब अच्छी तरह पहुँच गया
\end{hindi}}
\flushright{\begin{Arabic}
\quranayah[33][72]
\end{Arabic}}
\flushleft{\begin{hindi}
बेशक हमने (रोज़े अज़ल) अपनी अमानत (इताअत इबादत) को सारे आसमान और ज़मीन पहाड़ों के सामने पेश किया तो उन्होंने उसके (बार) उठाने से इन्कार किया और उससे डर गए और आदमी ने उसे (बे ताम्मुल) उठा लिया बेशक इन्सान (अपने हक़ में) बड़ा ज़ालिम (और) नादान है
\end{hindi}}
\flushright{\begin{Arabic}
\quranayah[33][73]
\end{Arabic}}
\flushleft{\begin{hindi}
इसका नतीजा यह हुआ कि खुदा मुनाफिक़ मर्दों और मुनाफिक़ औरतों और मुशरिक मर्दों और मुशरिक औरतों को (उनके किए की) सज़ा देगा और ईमानदार मर्दों और ईमानदार औरतों की (तक़सीर अमानत की) तौबा क़ुबूल फरमाएगा और खुदा तो बड़ा बख़शने वाला मेहरबान है
\end{hindi}}
\chapter{Al-Saba' (The Saba')}
\begin{Arabic}
\Huge{\centerline{\basmalah}}\end{Arabic}
\flushright{\begin{Arabic}
\quranayah[34][1]
\end{Arabic}}
\flushleft{\begin{hindi}
हर क़िस्म की तारीफ उसी खुदा के लिए (दुनिया में भी) सज़ावार है कि जो कुछ आसमानों में है और जो कुछ ज़मीन में है (ग़रज़ सब कुछ) उसी का है और आख़ेरत में (भी हर तरफ) उसी की तारीफ है और वही वाक़िफकार हकीम है
\end{hindi}}
\flushright{\begin{Arabic}
\quranayah[34][2]
\end{Arabic}}
\flushleft{\begin{hindi}
(जो) चीज़ें (बीज वग़ैरह) ज़मीन में दाख़िल हुई है और जो चीज़ (दरख्त वग़ैरह) इसमें से निकलती है और जो चीज़ (पानी वग़ैरह) आसामन से नाज़िल होती है और जो चीज़ (नज़ारात फरिश्ते वग़ैरह) उस पर चढ़ती है (सब) को जानता है और वही बड़ा बख्शने वाला है
\end{hindi}}
\flushright{\begin{Arabic}
\quranayah[34][3]
\end{Arabic}}
\flushleft{\begin{hindi}
और कुफ्फार कहने लगे कि हम पर तो क़यामत आएगी ही नहीं (ऐ रसूल) तुम कह दो हॉ (हॉ) मुझ को अपने उस आलेमुल ग़ैब परवरदिगार की क़सम है जिससे ज़र्रा बराबर (कोई चीज़) न आसमान में छिपी हुई है और न ज़मीन में कि क़यामत ज़रूर आएगी और ज़र्रे से छोटी चीज़ और ज़र्रे से बडी (ग़रज़ जितनी चीज़े हैं सब) वाजेए व रौशन किताब लौहे महफूज़ में महफूज़ हैं
\end{hindi}}
\flushright{\begin{Arabic}
\quranayah[34][4]
\end{Arabic}}
\flushleft{\begin{hindi}
ताकि जिन लोगों ने ईमान क़ुबूल किया और (अच्छे) काम किए उनको खुदा जज़ाए खैर दे यही वह लोग हैं जिनके लिए (गुनाहों की) मग़फेरत और (बहुत ही) इज्ज़त की रोज़ी है
\end{hindi}}
\flushright{\begin{Arabic}
\quranayah[34][5]
\end{Arabic}}
\flushleft{\begin{hindi}
और जिन लोगों ने हमारी आयतों (के तोड़) में मुक़ाबिले की दौड़-धूप की उन ही के लिए दर्दनाक अज़ाब की सज़ा होगी
\end{hindi}}
\flushright{\begin{Arabic}
\quranayah[34][6]
\end{Arabic}}
\flushleft{\begin{hindi}
और (ऐ रसूल) जिन लोगों को (हमारी बारगाह से) इल्म अता किया गया है वह जानते हैं कि जो (क़ुरान) तुम्हारे परवरदिगार की तरफ से तुम पर नाज़िल हुआ है बिल्कुल ठीक है और सज़ावार हम्द (व सना) ग़ालिब (खुदा) की राह दिखाता है
\end{hindi}}
\flushright{\begin{Arabic}
\quranayah[34][7]
\end{Arabic}}
\flushleft{\begin{hindi}
और कुफ्फ़ार (मसख़रेपन से बाहम) कहते हैं कि कहो तो हम तुम्हें ऐसा आदमी (मोहम्मद) बता दें जो तुम से बयान करेगा कि जब तुम (मर कर सड़ ग़ल जाओगे और) बिल्कुल रेज़ा रेज़ा हो जाओगे तो तुम यक़ीनन एक नए जिस्म में आओगे
\end{hindi}}
\flushright{\begin{Arabic}
\quranayah[34][8]
\end{Arabic}}
\flushleft{\begin{hindi}
क्या उस शख्स (मोहम्मद) ने खुदा पर झूठ तूफान बाँधा है या उसे जुनून (हो गया) है (न मोहम्मद झूठा है न उसे जुनून है) बल्कि खुद वह लोग जो आख़ेरत पर ईमान नहीं रखते अज़ाब और पहले दरजे की गुमराही में पड़े हुए हैं
\end{hindi}}
\flushright{\begin{Arabic}
\quranayah[34][9]
\end{Arabic}}
\flushleft{\begin{hindi}
तो क्या उन लोगों ने आसमान और ज़मीन की तरफ भी जो उनके आगे और उनके पीछे (सब तरफ से घेरे) हैं ग़ौर नहीं किया कि अगर हम चाहे तो उन लोगों को ज़मीन में धँसा दें या उन पर आसमान का कोई टुकड़ा ही गिरा दें इसमें शक नहीं कि इसमें हर रूजू करने वाले बन्दे के लिए यक़ीनी बड़ी इबरत है
\end{hindi}}
\flushright{\begin{Arabic}
\quranayah[34][10]
\end{Arabic}}
\flushleft{\begin{hindi}
और हमने यक़ीनन दाऊद को अपनी बारगाह से बुर्जुग़ी इनायत की थी (और पहाड़ों को हुक्म दिया) कि ऐ पहाड़ों तसबीह करने में उनका साथ दो और परिन्द को (ताबेए कर दिया) और उनके वास्ते लोहे को (मोम की तरह) नरम कर दिया था
\end{hindi}}
\flushright{\begin{Arabic}
\quranayah[34][11]
\end{Arabic}}
\flushleft{\begin{hindi}
कि फँराख़ व कुशादा जिरह बनाओ और (कड़ियों के) जोड़ने में अन्दाज़े का ख्याल रखो और तुम सब के सब अच्छे (अच्छे) काम करो वो कुछ तुम लोग करते हो मैं यक़ीनन देख रहा हूँ
\end{hindi}}
\flushright{\begin{Arabic}
\quranayah[34][12]
\end{Arabic}}
\flushleft{\begin{hindi}
और हवा को सुलेमान का (ताबेइदार बना दिया था) कि उसकी सुबह की रफ्तार एक महीने (मुसाफ़त) की थी और इसी तरह उसकी शाम की रफ्तार एक महीने (के मुसाफत) की थी और हमने उनके लिए तांबे (को पिघलाकर) उसका चश्मा जारी कर दिया था और जिन्नात (को उनका ताबेदार कर दिया था कि उन) में कुछ लोग उनके परवरदिगार के हुक्म से उनके सामने काम काज करते थे और उनमें से जिसने हमारे हुक्म से इनहराफ़ किया है उसे हम (क़यामत में) जहन्नुम के अज़ाब का मज़ा चख़ाँएगे
\end{hindi}}
\flushright{\begin{Arabic}
\quranayah[34][13]
\end{Arabic}}
\flushleft{\begin{hindi}
ग़रज़ सुलेमान को जो बनवाना मंज़ूर होता ये जिन्नात उनके लिए बनाते थे (जैसे) मस्जिदें, महल, क़िले और (फरिश्ते अम्बिया की) तस्वीरें और हौज़ों के बराबर प्याले और (एक जगह) गड़ी हुई (बड़ी बड़ी) देग़ें (कि एक हज़ार आदमी का खाना पक सके) ऐ दाऊद की औलाद शुक्र करते रहो और मेरे बन्दों में से शुक्र करने वाले (बन्दे) थोड़े से हैं
\end{hindi}}
\flushright{\begin{Arabic}
\quranayah[34][14]
\end{Arabic}}
\flushleft{\begin{hindi}
फिर जब हमने सुलेमान पर मौत का हुक्म जारी किया तो (मर गए) मगर लकड़ी के सहारे खड़े थे और जिन्नात को किसी ने उनके मरने का पता न बताया मगर ज़मीन की दीमक ने कि वह सुलेमान के असा को खा रही थी फिर (जब खोखला होकर टूट गया और) सुलेमान (की लाश) गिरी तो जिन्नात ने जाना कि अगर वह लोग ग़ैब वॉ (ग़ैब के जानने वाले) होते तो (इस) ज़लील करने वाली (काम करने की) मुसीबत में न मुब्तिला रहते
\end{hindi}}
\flushright{\begin{Arabic}
\quranayah[34][15]
\end{Arabic}}
\flushleft{\begin{hindi}
और (क़ौम) सबा के लिए तो यक़ीनन ख़ुद उन्हीं के घरों में (कुदरते खुदा की) एक बड़ी निशानी थी कि उनके शहर के दोनों तरफ दाहिने बाएं (हरे-भरे) बाग़ात थे (और उनको हुक्म था) कि अपने परवरदिगार की दी हुई रोज़ी Âाओ (पियो) और उसका शुक्र अदा करो (दुनिया में) ऐसा पाकीज़ा शहर और (आख़ेरत में) परवरदिगार सा बख्शने वाला
\end{hindi}}
\flushright{\begin{Arabic}
\quranayah[34][16]
\end{Arabic}}
\flushleft{\begin{hindi}
इस पर भी उन लोगों ने मुँह फेर लिया (और पैग़म्बरों का कहा न माना) तो हमने (एक ही बन्द तोड़कर) उन पर बड़े ज़ोरों का सैलाब भेज दिया और (उनको तबाह करके) उनके दोनों बाग़ों के बदले ऐसे दो बाग़ दिए जिनके फल बदमज़ा थे और उनमें झाऊ था और कुछ थोड़ी सी बेरियाँ थी
\end{hindi}}
\flushright{\begin{Arabic}
\quranayah[34][17]
\end{Arabic}}
\flushleft{\begin{hindi}
ये हमने उनकी नाशुक्री की सज़ा दी और हम तो बड़े नाशुक्रों ही की सज़ा किया करते हैं
\end{hindi}}
\flushright{\begin{Arabic}
\quranayah[34][18]
\end{Arabic}}
\flushleft{\begin{hindi}
और हम अहले सबा और (शाम) की उन बस्तियों के दरमियान जिनमें हमने बरकत अता की थी और चन्द बस्तियाँ (सरे राह) आबाद की थी जो बाहम नुमाया थीं और हमने उनमें आमद व रफ्त की राह मुक़र्रर की थी कि उनमें रातों को दिनों को (जब जी चाहे) बेखटके चलो फिरो
\end{hindi}}
\flushright{\begin{Arabic}
\quranayah[34][19]
\end{Arabic}}
\flushleft{\begin{hindi}
तो वह लोग ख़ुद कहने लगे परवरदिगार (क़रीब के सफर में लुत्फ नहीं) तो हमारे सफ़रों में दूरी पैदा कर दे और उन लोगों ने खुद अपने ऊपर ज़ुल्म किया तो हमने भी उनको (तबाह करके उनके) अफसाने बना दिए - और उनकी धज्जियाँ उड़ा के उनको तितिर बितिर कर दिया बेशक उनमें हर सब्र व शुक्र करने वालोंके वास्ते बड़ी इबरते हैं
\end{hindi}}
\flushright{\begin{Arabic}
\quranayah[34][20]
\end{Arabic}}
\flushleft{\begin{hindi}
और शैतान ने अपने ख्याल को (जो उनके बारे में किया था) सच कर दिखाया तो उन लोगों ने उसकी पैरवी की मगर ईमानवालों का एक गिरोह (न भटका)
\end{hindi}}
\flushright{\begin{Arabic}
\quranayah[34][21]
\end{Arabic}}
\flushleft{\begin{hindi}
और शैतान का उन लोगों पर कुछ क़ाबू तो था नहीं मगर ये (मतलब था) कि हम उन लोगों को जो आख़ेरत का यक़ीन रखते हैं उन लोगों से अलग देख लें जो उसके बारे में शक में (पड़े) हैं और तुम्हारा परवरदिगार तो हर चीज़ का निगरॉ है
\end{hindi}}
\flushright{\begin{Arabic}
\quranayah[34][22]
\end{Arabic}}
\flushleft{\begin{hindi}
(ऐ रसूल इनसे) कह दो कि जिन लोगों को तुम खुद ख़ुदा के सिवा (माबूद) समझते हो पुकारो (तो मालूम हो जाएगा कि) वह लोग ज़र्रा बराबर न आसमानों में कुछ इख़तेयार रखते हैं और न ज़मीन में और न उनकी उन दोनों में शिरकत है और न उनमें से कोई खुदा का (किसी चीज़ में) मद्दगार है
\end{hindi}}
\flushright{\begin{Arabic}
\quranayah[34][23]
\end{Arabic}}
\flushleft{\begin{hindi}
जिसके लिए वह खुद इजाज़त अता फ़रमाए उसके सिवा कोई सिफारिश उसकी बारगाह में काम न आएगी (उसके दरबार की हैबत) यहाँ तक (है) कि जब (शिफ़ाअत का) हुक्म होता है तो शिफ़ाअत करने वाले बेहोश हो जाते हैं फिर तब उनके दिलों की घबराहट दूर कर दी जाती है तो पूछते हैं कि तुम्हारे परवरदिगार ने क्या हुक्म दिया
\end{hindi}}
\flushright{\begin{Arabic}
\quranayah[34][24]
\end{Arabic}}
\flushleft{\begin{hindi}
तो मुक़र्रिब फरिश्ते कहते हैं कि जो वाजिबी था (ऐ रसूल) तुम (इनसे) पूछो तो कि भला तुमको सारे आसमान और ज़मीन से कौन रोज़ी देता है (वह क्या कहेंगे) तुम खुद कह दो कि खुदा और मैं या तुम (दोनों में से एक तो) ज़रूर राहे रास्त पर है (और दूसरा गुमराह) या वह सरीही गुमराही में पड़ा है (और दूसरा राहे रास्त पर)
\end{hindi}}
\flushright{\begin{Arabic}
\quranayah[34][25]
\end{Arabic}}
\flushleft{\begin{hindi}
(ऐ रसूल) तुम (उनसे) कह दो न हमारे गुनाहों की तुमसे पूछ गछ होगी और न तुम्हारी कारस्तानियों की हम से बाज़ पुर्स
\end{hindi}}
\flushright{\begin{Arabic}
\quranayah[34][26]
\end{Arabic}}
\flushleft{\begin{hindi}
(ऐ रसूल) तुम (उनसे) कह दो कि हमारा परवरदिगार (क़यामत में) हम सबको इकट्ठा करेगा फिर हमारे दरमियान (ठीक) फैसला कर देगा और वह तो ठीक-ठीक फैसला करने वाला वाक़िफकार है
\end{hindi}}
\flushright{\begin{Arabic}
\quranayah[34][27]
\end{Arabic}}
\flushleft{\begin{hindi}
(ऐ रसूल तुम कह दो कि जिनको तुम ने खुदा का शरीक बनाकर) खुदा के साथ मिलाया है ज़रा उन्हें मुझे भी तो दिखा दो हरगिज़ (कोई शरीक नहीं) बल्कि खुदा ग़ालिब हिकमत वाला है
\end{hindi}}
\flushright{\begin{Arabic}
\quranayah[34][28]
\end{Arabic}}
\flushleft{\begin{hindi}
(ऐ रसूल) हमने तुमको तमाम (दुनिया के) लोगों के लिए (नेकों को बेहश्त की) खुशखबरी देने वाला और (बन्दों को अज़ाब से) डराने वाला (पैग़म्बर) बनाकर भेजा मगर बहुतेरे लोग (इतना भी) नहीं जानते
\end{hindi}}
\flushright{\begin{Arabic}
\quranayah[34][29]
\end{Arabic}}
\flushleft{\begin{hindi}
और (उलटे) कहते हैं कि अगर तुम (अपने दावे में) सच्चे हो तो (आख़िर) ये क़यामत का वायदा कब पूरा होगा
\end{hindi}}
\flushright{\begin{Arabic}
\quranayah[34][30]
\end{Arabic}}
\flushleft{\begin{hindi}
(ऐ रसूल) तुम उनसे कह दो कि तुम लोगों के वास्ते एक ख़ास दिन की मीयाद मुक़र्रर है कि न तुम उससे एक घड़ी पीछे रह सकते हो और न आगे ही बड़ सकते हो
\end{hindi}}
\flushright{\begin{Arabic}
\quranayah[34][31]
\end{Arabic}}
\flushleft{\begin{hindi}
और जो लोग काफिर हों बैठे कहते हैं कि हम तो न इस क़ुरान पर हरगिज़ ईमान लाएँगे और न उस (किताब) पर जो इससे पहले नाज़िल हो चुकी और (ऐ रसूल तुमको बहुत ताज्जुब हो) अगर तुम देखो कि जब ये ज़ालिम क़यामत के दिन अपने परवरदिगार के सामने खड़े किए जायेंगे (और) उनमें का एक दूसरे की तरफ (अपनी) बात को फेरता होगा कि कमज़ोर अदना (दरजे के) लोग बड़े (सरकश) लोगों से कहते होगें कि अगर तुम (हमें) न (बहकाए) होते तो हम ज़रूर ईमानवाले होते (इस मुसीबत में न पड़ते)
\end{hindi}}
\flushright{\begin{Arabic}
\quranayah[34][32]
\end{Arabic}}
\flushleft{\begin{hindi}
तो सरकश लोग कमज़ोरों से (मुख़ातिब होकर) कहेंगे कि जब तुम्हारे पास (खुदा की तरफ़ से) हिदायत आयी तो थी तो क्या उसके आने के बाद हमने तुमको (ज़बरदस्ती अम्ल करने से) रोका था (हरगिज़ नहीं) बल्कि तुम तो खुद मुजरिम थे
\end{hindi}}
\flushright{\begin{Arabic}
\quranayah[34][33]
\end{Arabic}}
\flushleft{\begin{hindi}
और कमज़ोर लोग बड़े लोगों से कहेंगे (कि ज़बरदस्ती तो नहीं की मगर हम खुद भी गुमराह नहीं हुए) बल्कि (तुम्हारी) रात-दिन की फरेबदेही ने (गुमराह किया कि) तुम लोग हमको खुदा न मानने और उसका शरीक ठहराने का बराबर हुक्म देते रहे (तो हम क्या करते) और जब ये लोग अज़ाब को (अपनी ऑंखों से) देख लेंगे तो दिल ही दिल में पछताएँगे और जो लोग काफिर हो बैठे हम उनकी गर्दनों में तौक़ डाल देंगे जो कारस्तानियां ये लोग (दुनिया में) करते थे उसी के मुवाफिक़ तो सज़ा दी जाएगी
\end{hindi}}
\flushright{\begin{Arabic}
\quranayah[34][34]
\end{Arabic}}
\flushleft{\begin{hindi}
और हमने किसी बस्ती में कोई डराने वाला पैग़म्बर नहीं भेजा मगर वहाँ के लोग ये ज़रूर बोल उठेंगे कि जो एहकाम देकर तुम भेजे गए हो हम उनको नहीं मानते
\end{hindi}}
\flushright{\begin{Arabic}
\quranayah[34][35]
\end{Arabic}}
\flushleft{\begin{hindi}
और ये भी कहने लगे कि हम तो (ईमानदारों से) माल और औलाद में कहीं ज्यादा है और हम पर आख़ेरत में (अज़ाब) भी नहीं किया जाएगा
\end{hindi}}
\flushright{\begin{Arabic}
\quranayah[34][36]
\end{Arabic}}
\flushleft{\begin{hindi}
(ऐ रसूल) तुम कह दो कि मेरा परवरदिगार जिसके लिए चाहता है रोज़ी कुशादा कर देता है और (जिसके लिऐ चाहता है) तंग करता है मगर बहुतेरे लोग नहीं जानते हैं
\end{hindi}}
\flushright{\begin{Arabic}
\quranayah[34][37]
\end{Arabic}}
\flushleft{\begin{hindi}
और (याद रखो) तुम्हारे माल और तुम्हारी औलाद की ये हस्ती नहीं कि तुम को हमारी बारगाह में मुक़र्रिब बना दें मगर (हाँ) जिसने ईमान कुबूल किया और अच्छे (अच्छे) काम किए उन लोगों के लिए तो उनकी कारगुज़ारियों की दोहरी जज़ा है और वह लोग (बेहश्त के) झरोखों में इत्मेनान से रहेंगे
\end{hindi}}
\flushright{\begin{Arabic}
\quranayah[34][38]
\end{Arabic}}
\flushleft{\begin{hindi}
और जो लोग हमारी आयतों (की तोड़) में मुक़ाबले की नीयत से दौड़ द्दूप करते हैं वही लोग (जहन्नुम के) अज़ाब में झोक दिए जाएॅगे
\end{hindi}}
\flushright{\begin{Arabic}
\quranayah[34][39]
\end{Arabic}}
\flushleft{\begin{hindi}
(ऐ रसूल) तुम कह दो कि मेरा परवरदिगार अपने बन्दों में से जिसके लिए चाहता है रोज़ी कुशादा कर देता है और (जिसके लिए चाहता है) तंग कर देता है और जो कुछ भी तुम लोग (उसकी राह में) ख़र्च करते हो वह उसका ऐवज देगा और वह तो सबसे बेहतर रोज़ी देनेवाला है
\end{hindi}}
\flushright{\begin{Arabic}
\quranayah[34][40]
\end{Arabic}}
\flushleft{\begin{hindi}
और (वह दिन याद करो) जिस दिन सब लोगों को इकट्ठा करेगा फिर फरिश्तों से पूछेगा कि क्या ये लोग तुम्हारी परसतिश करते थे फरिश्ते अर्ज़ करेंगे (बारे इलाहा) तू (हर ऐब से) पाक व पाकीज़ा है
\end{hindi}}
\flushright{\begin{Arabic}
\quranayah[34][41]
\end{Arabic}}
\flushleft{\begin{hindi}
तू ही हमारा मालिक है न ये लोग (ये लोग हमारी नहीं) बल्कि जिन्नात (खबाएस भूत-परेत) की परसतिश करते थे कि उनमें के अक्सर लोग उन्हीं पर ईमान रखते थे
\end{hindi}}
\flushright{\begin{Arabic}
\quranayah[34][42]
\end{Arabic}}
\flushleft{\begin{hindi}
तब (खुदा फरमाएगा) आज तो तुममें से कोई न दूसरे के फायदे ही पहुँचाने का इख्तेयार रखता है और न ज़रर का और हम सरकशों से कहेंगे कि (आज) उस अज़ाब के मज़े चखो जिसे तुम (दुनिया में) झुठलाया करते थे
\end{hindi}}
\flushright{\begin{Arabic}
\quranayah[34][43]
\end{Arabic}}
\flushleft{\begin{hindi}
और जब उनके सामने हमारी वाज़ेए व रौशन आयतें पढ़ी जाती थीं तो बाहम कहते थे कि ये (रसूल) भी तो बस (हमारा ही जैसा) आदमी है ये चाहता है कि जिन चीज़ों को तुम्हारे बाप-दादा पूजते थे (उनकी परसतिश) से तुम को रोक दें और कहने लगे कि ये (क़ुरान) तो बस निरा झूठ है और अपने जी का गढ़ा हुआ है और जो लोग काफ़िर हो बैठो जब उनके पास हक़ बात आयी तो उसके बारे में कहने लगे कि ये तो बस खुला हुआ जादू है
\end{hindi}}
\flushright{\begin{Arabic}
\quranayah[34][44]
\end{Arabic}}
\flushleft{\begin{hindi}
और (ऐ रसूल) हमने तो उन लोगों को न (आसमानी) किताबें अता की तुम्हें जिन्हें ये लोग पढ़ते और न तुमसे पहले इन लोगों के पास कोई डरानेवाला (पैग़म्बर) भेजा (उस पर भी उन्होंने क़द्र न की)
\end{hindi}}
\flushright{\begin{Arabic}
\quranayah[34][45]
\end{Arabic}}
\flushleft{\begin{hindi}
और जो लोग उनसे पहले गुज़र गए उन्होंने भी (पैग़म्बरों को) झुठलाया था हालॉकि हमने जितना उन लोगों को दिया था ये लोग (अभी) उसके दसवें हिस्सा को (भी) नहीं पहुँचे उस पर उन लोगों न मेरे (पैग़म्बरों को) झुठलाया था तो तुमने देखा कि मेरा (अज़ाब उन पर) कैसा सख्त हुआ
\end{hindi}}
\flushright{\begin{Arabic}
\quranayah[34][46]
\end{Arabic}}
\flushleft{\begin{hindi}
(ऐ रसूल) तुम कह दो कि मैं तुमसे नसीहत की बस एक बात कहता हूँ (वह) ये (है) कि तुम लोग बाज़ खुदा के वास्ते एक-एक और दो-दो उठ खड़े हो और अच्छी तरह ग़ौर करो तो (देख लोगे कि) तुम्हारे रफीक़ (मोहम्मद स0) को किसी तरह का जुनून नहीं वह तो बस तुम्हें एक सख्त अज़ाब (क़यामत) के सामने (आने) से डराने वाला है
\end{hindi}}
\flushright{\begin{Arabic}
\quranayah[34][47]
\end{Arabic}}
\flushleft{\begin{hindi}
(ऐ रसूल) तुम (ये भी) कह दो कि (तबलीख़े रिसालत की) मैंने तुमसे कुछ उजरत माँगी हो तो वह तुम्हीं को (मुबारक) हो मेरी उजरत तो बस खुदा पर है और वही (तुम्हारे आमाल अफआल) हर चीज़ से खूब वाक़िफ है
\end{hindi}}
\flushright{\begin{Arabic}
\quranayah[34][48]
\end{Arabic}}
\flushleft{\begin{hindi}
(ऐ रसूल) तुम उनसे कह दो कि मेरा बड़ा गैबवाँ परवरदिगार (मेरे दिल में) दीन हक़ को बराबर ऊपर से उतारता है
\end{hindi}}
\flushright{\begin{Arabic}
\quranayah[34][49]
\end{Arabic}}
\flushleft{\begin{hindi}
(अब उनसे) कह दो दीने हक़ आ गया और इतना तो भी (समझो की) बातिल (माबूद) शुरू-शुरू कुछ पैदा करता है न (मरने के बाद) दोबारा ज़िन्दा कर सकता है
\end{hindi}}
\flushright{\begin{Arabic}
\quranayah[34][50]
\end{Arabic}}
\flushleft{\begin{hindi}
(ऐ रसूल) तुम ये भी कह दो कि अगर मैं गुमराह हो गया हूँ तो अपनी ही जान पर मेरी गुमराही (का वबाल) है और अगर मैं राहे रास्त पर हूँ तो इस ''वही'' के तुफ़ैल से जो मेरा परवरदिगार मेरी तरफ़ भेजता है बेशक वह सुनने वाला (और बहुत) क़रीब है
\end{hindi}}
\flushright{\begin{Arabic}
\quranayah[34][51]
\end{Arabic}}
\flushleft{\begin{hindi}
और (ऐ रसूल) काश तुम देखते (तो सख्त ताज्जुब करते) जब ये कुफ्फार (मैदाने हशर में) घबराए-घबराए फिरते होंगे तो भी छुटकारा न होगा
\end{hindi}}
\flushright{\begin{Arabic}
\quranayah[34][52]
\end{Arabic}}
\flushleft{\begin{hindi}
और आस ही पास से (बाआसानी) गिरफ्तार कर लिए जाएँगे और (उस वक्त बेबसी में) कहेंगे कि अब हम रसूलों पर ईमान लाए और इतनी दूर दराज़ जगह से (ईमान पर) उनका दसतरस (पहुँचना) कहाँ मुमकिन है
\end{hindi}}
\flushright{\begin{Arabic}
\quranayah[34][53]
\end{Arabic}}
\flushleft{\begin{hindi}
हालॉकि ये लोग उससे पहले ही जब उनका दसतरस था इन्कार कर चुके और (दुनिया में तमाम उम्र) बे देखे भाले (अटकल के) तके बड़ी-बड़ी दूर से चलाते रहे
\end{hindi}}
\flushright{\begin{Arabic}
\quranayah[34][54]
\end{Arabic}}
\flushleft{\begin{hindi}
और अब तो उनके और उनकी तमन्नाओं के दरमियान (उसी तरह) पर्दा डाल दिया गया है जिस तरह उनसे पहले उनके हमरंग लोगों के साथ (यही बरताव) किया जा चुका इसमें शक नहीं कि वह लोग बड़े बेचैन करने वाले शक में पड़े हुए थे
\end{hindi}}
\chapter{Al-Fatir (The Originator)}
\begin{Arabic}
\Huge{\centerline{\basmalah}}\end{Arabic}
\flushright{\begin{Arabic}
\quranayah[35][1]
\end{Arabic}}
\flushleft{\begin{hindi}
हर तरह की तारीफ खुदा ही के लिए (मख़सूस) है जो सारे आसमान और ज़मीन का पैदा करने वाला फरिश्तों का (अपना) क़ासिद बनाने वाला है जिनके दो-दो और तीन-तीन और चार-चार पर होते हैं (मख़लूक़ात की) पैदाइश में जो (मुनासिब) चाहता है बढ़ा देता है बेशक खुदा हर चीज़ पर क़ादिर (व तवाना है)
\end{hindi}}
\flushright{\begin{Arabic}
\quranayah[35][2]
\end{Arabic}}
\flushleft{\begin{hindi}
लोगों के वास्ते जब (अपनी) रहमत (के दरवाजे) ख़ोल दे तो कोई उसे जारी नहीं कर सकता और जिस चीज़ को रोक ले उसके बाद उसे कोई रोक नहीं सकता और वही हर चीज़ पर ग़ालिब और दाना व बीना हकीम है
\end{hindi}}
\flushright{\begin{Arabic}
\quranayah[35][3]
\end{Arabic}}
\flushleft{\begin{hindi}
लोगों खुदा के एहसानात को जो उसने तुम पर किए हैं याद करो क्या खुदा के सिवा कोई और ख़ालिक है जो आसमान और ज़मीन से तुम्हारी रोज़ी पहुँचाता है उसके सिवा कोई माबूद क़ाबिले परसतिश नहीं फिर तुम किधर बहके चले जा रहे हो
\end{hindi}}
\flushright{\begin{Arabic}
\quranayah[35][4]
\end{Arabic}}
\flushleft{\begin{hindi}
और (ऐ रसूल) अगर ये लोग तुम्हें झुठलाएँ तो (कुढ़ो नहीं) तुमसे पहले बहुतेरे पैग़म्बर (लोगों के हाथों) झुठलाए जा चुके हैं और (आख़िर) कुल उमूर की रूजू तो खुदा ही की तरफ है
\end{hindi}}
\flushright{\begin{Arabic}
\quranayah[35][5]
\end{Arabic}}
\flushleft{\begin{hindi}
लोगों खुदा का (क़यामत का) वायदा यक़ीनी बिल्कुल सच्चा है तो (कहीं) तुम्हें दुनिया की (चन्द रोज़ा) ज़िन्दगी फ़रेब में न लाए (ऐसा न हो कि शैतान) तुम्हें खुदा के बारे में धोखा दे
\end{hindi}}
\flushright{\begin{Arabic}
\quranayah[35][6]
\end{Arabic}}
\flushleft{\begin{hindi}
बेशक शैतान तुम्हारा दुश्मन है तो तुम भी उसे अपना दुशमन बनाए रहो वह तो अपने गिरोह को बस इसलिए बुलाता है कि वह लोग (सब के सब) जहन्नुमी बन जाएँ
\end{hindi}}
\flushright{\begin{Arabic}
\quranayah[35][7]
\end{Arabic}}
\flushleft{\begin{hindi}
जिन लोगों ने (दुनिया में) कुफ्र इख़तेयार किया उनके लिए (आखेरत में) सख्त अज़ाब है और जिन लोगों ने ईमान क़ुबूल किया और अच्छे-अच्छे काम किए उनके लिए (गुनाहों की) मग़फेरत और निहायत अच्छा बदला (बेहश्त) है
\end{hindi}}
\flushright{\begin{Arabic}
\quranayah[35][8]
\end{Arabic}}
\flushleft{\begin{hindi}
तो भला वह शख्स जिसे उस का बुरा काम शैतानी (अग़वॉ से) अच्छा कर दिखाया गया है औ वह उसे अच्छा समझने लगा है (कभी मोमिन नेकोकार के बराबर हो सकता है हरगिज़ नहीं) तो यक़ीनी (बात) ये है कि ख़ुदा जिसे चाहता है गुमराही में छोड़ देता है और जिसे चाहता है राहे रास्त पर आने (की तौफ़ीक़) देता है तो (ऐ रसूल कहीं) उन (बदबख्तों) पर अफसोस कर करके तुम्हारे दम न निकल जाए जो कुछ ये लोग करते हैं ख़ुदा उससे खूब वाक़िफ़ है
\end{hindi}}
\flushright{\begin{Arabic}
\quranayah[35][9]
\end{Arabic}}
\flushleft{\begin{hindi}
और ख़ुदा ही वह (क़ादिर व तवाना) है जो हवाओं को भेजता है तो हवाएँ बादलों को उड़ाए-उड़ाए फिरती है फिर हम उस बादल को मुर्दा (उफ़तादा) शहर की तरफ हका देते हैं फिर हम उसके ज़रिए से ज़मीन को उसके मर जाने के बाद शादाब कर देते हैं यूँ ही (मुर्दों को क़यामत में जी उठना होगा)
\end{hindi}}
\flushright{\begin{Arabic}
\quranayah[35][10]
\end{Arabic}}
\flushleft{\begin{hindi}
जो शख्स इज्ज़त का ख्वाहाँ हो तो ख़ुदा से माँगे क्योंकि सारी इज्ज़त खुदा ही की है उसकी बारगाह तक अच्छी बातें (बुलन्द होकर) पहुँचतीं हैं और अच्छे काम को वह खुद बुलन्द फरमाता है और जो लोग (तुम्हारे ख़िलाफ) बुरी-बुरी तदबीरें करते रहते हैं उनके लिए क़यामत में सख्त अज़ाब है और (आख़िर) उन लोगों की तदबीर मटियामेट हो जाएगी
\end{hindi}}
\flushright{\begin{Arabic}
\quranayah[35][11]
\end{Arabic}}
\flushleft{\begin{hindi}
और खुदा ही ने तुम लोगों को (पहले पहल) मिट्टी से पैदा किया फिर नतफ़े से फिर तुमको जोड़ा (नर मादा) बनाया और बग़ैर उसके इल्म (इजाज़त) के न कोई औरत हमेला होती है और न जनती है और न किसी शख्स की उम्र में ज्यादती होती है और न किसी की उम्र से कमी की जाती है मगर वह किताब (लौहे महफूज़) में (ज़रूर) है बेशक ये बात खुदा पर बहुत ही आसान है
\end{hindi}}
\flushright{\begin{Arabic}
\quranayah[35][12]
\end{Arabic}}
\flushleft{\begin{hindi}
(उसकी कुदरत देखो) दो समन्दर बावजूद मिल जाने के यकसाँ नहीं हो जाते ये (एक तो) मीठा खुश ज़ाएका कि उसका पीना सुवारत (ख़ुश्गवार) है और ये (दूसरा) खारी कड़ुवा है और (इस इख़तेलाफ पर भी) तुम लोग दोनों से (मछली का) तरो ताज़ा गोश्त (यकसाँ) खाते हो और (अपने लिए ज़ेवरात (मोती वग़ैरह) निकालते हो जिन्हें तुम पहनते हो और तुम देखते हो कि कश्तियां दरिया में (पानी को) फाड़ती चली जाती हैं ताकि उसके फज्ल (व करम तिजारत) की तलाश करो और ताकि तुम लोग शुक्र करो
\end{hindi}}
\flushright{\begin{Arabic}
\quranayah[35][13]
\end{Arabic}}
\flushleft{\begin{hindi}
वही रात को (बढ़ा के) दिन में दाख़िल करता है (तो रात बढ़ जाती है) और वही दिन को (बढ़ा के) रात में दाख़िल करता है (तो दिन बढ़ जाता है और) उसी ने सूरज और चाँद को अपना मुतीइ बना रखा है कि हर एक अपने (अपने) मुअय्यन (तय) वक्त पर चला करता है वही खुदा तुम्हारा परवरदिगार है उसी की सलतनत है और उसे छोड़कर जिन माबूदों को तुम पुकारते हो वह छुवारों की गुठली की झिल्ली के बराबर भी तो इख़तेयार नहीं रखते
\end{hindi}}
\flushright{\begin{Arabic}
\quranayah[35][14]
\end{Arabic}}
\flushleft{\begin{hindi}
अगर तुम उनको पुकारो तो वह तुम्हारी पुकार को सुनते नहीं अगर (बिफ़रज़े मुहाल) सुनों भी तो तुम्हारी दुआएँ नहीं कुबूल कर सकते और क़यामत के दिन तुम्हारे शिर्क से इन्कार कर बैठेंगें और वाक़िफकार (शख्स की तरह कोई दूसरा उनकी पूरी हालत) तुम्हें बता नहीं सकता
\end{hindi}}
\flushright{\begin{Arabic}
\quranayah[35][15]
\end{Arabic}}
\flushleft{\begin{hindi}
लोगों तुम सब के सब खुदा के (हर वक्त) मोहताज हो और (सिर्फ) खुदा ही (सबसे) बेपरवा सज़ावारे हम्द (व सना) है
\end{hindi}}
\flushright{\begin{Arabic}
\quranayah[35][16]
\end{Arabic}}
\flushleft{\begin{hindi}
अगर वह चाहे तो तुम लोगों को (अदम के पर्दे में) ले जाए और एक नयी ख़िलक़त ला बसाए
\end{hindi}}
\flushright{\begin{Arabic}
\quranayah[35][17]
\end{Arabic}}
\flushleft{\begin{hindi}
और ये कुछ खुदा के वास्ते दुशवार नहीं
\end{hindi}}
\flushright{\begin{Arabic}
\quranayah[35][18]
\end{Arabic}}
\flushleft{\begin{hindi}
और याद रहे कि कोई शख्स किसी दूसरे (के गुनाह) का बोझ नहीं उठाएगा और अगर कोई (अपने गुनाहों का) भारी बोझ उठाने वाला अपना बोझ उठाने के वास्ते (किसी को) बुलाएगा तो उसके बारे में से कुछ भी उठाया न जाएगा अगरचे (कोई किसी का) कराबतदार ही (क्यों न) हो (ऐ रसूल) तुम तो बस उन्हीं लोगों को डरा सकते हो जो बे देखे भाले अपने परवरदिगार का ख़ौफ रखते हैं और पाबन्दी से नमाज़ पढ़ते हैं और (याद रखो कि) जो शख्स पाक साफ रहता है वह अपने ही फ़ायदे के वास्ते पाक साफ रहता है और (आख़िरकार सबको हिरफिर के) खुदा ही की तरफ जाना है
\end{hindi}}
\flushright{\begin{Arabic}
\quranayah[35][19]
\end{Arabic}}
\flushleft{\begin{hindi}
और अन्धा (क़ाफिर) और ऑंखों वाला (मोमिन किसी तरह) बराबर नहीं हो सकते
\end{hindi}}
\flushright{\begin{Arabic}
\quranayah[35][20]
\end{Arabic}}
\flushleft{\begin{hindi}
और न अंधेरा (कुफ्र) और उजाला (ईमान) बराबर है
\end{hindi}}
\flushright{\begin{Arabic}
\quranayah[35][21]
\end{Arabic}}
\flushleft{\begin{hindi}
और न छाँव (बेहिश्त) और धूप (दोज़ख़ बराबर है)
\end{hindi}}
\flushright{\begin{Arabic}
\quranayah[35][22]
\end{Arabic}}
\flushleft{\begin{hindi}
और न ज़िन्दे (मोमिनीन) और न मुर्दें (क़ाफिर) बराबर हो सकते हैं और खुदा जिसे चाहता है अच्छी तरह सुना (समझा) देता है और (ऐ रसूल) जो (कुफ्फ़ार मुर्दों की तरह) क़ब्रों में हैं उन्हें तुम अपनी (बातें) नहीं समझा सकते हो
\end{hindi}}
\flushright{\begin{Arabic}
\quranayah[35][23]
\end{Arabic}}
\flushleft{\begin{hindi}
तुम तो बस (एक खुदा से) डराने वाले हो
\end{hindi}}
\flushright{\begin{Arabic}
\quranayah[35][24]
\end{Arabic}}
\flushleft{\begin{hindi}
हम ही ने तुमको यक़ीनन कुरान के साथ खुशख़बरी देने वाला और डराने वाला ( पैग़म्बर) बनाकर भेजा और कोई उम्मत (दुनिया में)
\end{hindi}}
\flushright{\begin{Arabic}
\quranayah[35][25]
\end{Arabic}}
\flushleft{\begin{hindi}
ऐसी नहीं गुज़री कि उसके पास (हमारा) डराने वाला पैग़म्बर न आया हो और अगर ये लोग तुम्हें झुठलाएँ तो कुछ परवाह नहीं करो क्योंकि इनके अगलों ने भी (अपने पैग़म्बरों को) झुठलाया है (हालाँकि) उनके पास उनके पैग़म्बर वाज़ेए व रौशन मौजिजे अौर सहीफ़े और रौशन किताब लेकर आए थे
\end{hindi}}
\flushright{\begin{Arabic}
\quranayah[35][26]
\end{Arabic}}
\flushleft{\begin{hindi}
फिर हमने उन लोगों को जो काफ़िर हो बैठे ले डाला तो (तुमने देखाकि) मेरा अज़ाब (उन पर) कैसा (सख्त हुआ
\end{hindi}}
\flushright{\begin{Arabic}
\quranayah[35][27]
\end{Arabic}}
\flushleft{\begin{hindi}
अब क्या तुमने इस पर भी ग़ौर नहीं किया कि यक़ीनन खुदा ही ने आसमान से पानी बरसाया फिर हम (खुदा) ने उसके ज़रिए से तरह-तरह की रंगतों के फल पैदा किए और पहाड़ों में क़तआत (टुकड़े रास्ते) हैं जिनके रंग मुख़तलिफ है कुछ तो सफेद (बुर्राक़) और कुछ लाल (लाल) और कुछ बिल्कुल काले सियाह
\end{hindi}}
\flushright{\begin{Arabic}
\quranayah[35][28]
\end{Arabic}}
\flushleft{\begin{hindi}
और इसी तरह आदमियों और जानवरों और चारपायों की भी रंगते तरह-तरह की हैं उसके बन्दों में ख़ुदा का ख़ौफ करने वाले तो बस उलेमा हैं बेशक खुदा (सबसे) ग़ालिब और बख्शने वाला है
\end{hindi}}
\flushright{\begin{Arabic}
\quranayah[35][29]
\end{Arabic}}
\flushleft{\begin{hindi}
बेशक जो लोग Âुदा की किताब पढ़ा करते हैं और पाबन्दी से नमाज़ पढ़ते हैं और जो कुछ हमने उन्हें अता किया है उसमें से छिपा के और दिखा के (खुदा की राह में) देते हैं वह यक़ीनन ऐसे व्यापार का आसरा रखते हैं
\end{hindi}}
\flushright{\begin{Arabic}
\quranayah[35][30]
\end{Arabic}}
\flushleft{\begin{hindi}
जिसका कभी टाट न उलटेगा ताकि खुदा उन्हें उनकी मज़दूरियाँ भरपूर अता करे बल्कि अपने फज़ल (व करम) से उसे कुछ और बढ़ा देगा बेशक वह बड़ा बख्शने वाला है (और) बड़ा क़द्रदान है
\end{hindi}}
\flushright{\begin{Arabic}
\quranayah[35][31]
\end{Arabic}}
\flushleft{\begin{hindi}
और हमने जो किताब तुम्हारे पास ''वही'' के ज़रिए से भेजी वह बिल्कुल ठीक है और जो (किताबें इससे पहले की) उसके सामने (मौजूद) हैं उनकी तसदीक़ भी करती हैं - बेशक खुदा अपने बन्दों (के हालात) से खूब वाक़िफ है (और) देख रहा है
\end{hindi}}
\flushright{\begin{Arabic}
\quranayah[35][32]
\end{Arabic}}
\flushleft{\begin{hindi}
फिर हमने अपने बन्दगान में से ख़ास उनको कुरान का वारिस बनाया जिन्हें (अहल समझकर) मुन्तख़िब किया क्योंकि बन्दों में से कुछ तो (नाफरमानी करके) अपनी जान पर सितम ढाते हैं और कुछ उनमें से (नेकी बदी के) दरमियान हैं और उनमें से कुछ लोग खुदा के इख़तेयार से नेकों में (औरों से) गोया सबकत ले गए हैं (इन्तेख़ाब व सबक़त) तो खुदा का बड़ा फज़ल है
\end{hindi}}
\flushright{\begin{Arabic}
\quranayah[35][33]
\end{Arabic}}
\flushleft{\begin{hindi}
(और उसका सिला बेहिश्त के) सदा बहार बाग़ात हैं जिनमें ये लोग दाख़िल होंगे और उन्हें वहाँ सोने के कंगन और मोती पहनाए जाएँगे और वहाँ उनकी (मामूली) पोशाक ख़ालिस रेशमी होगी
\end{hindi}}
\flushright{\begin{Arabic}
\quranayah[35][34]
\end{Arabic}}
\flushleft{\begin{hindi}
और ये लोग (खुशी के लहजे में) कहेंगे खुदा का शुक्र जिसने हम से (हर क़िस्म का) रंज व ग़म दूर कर दिया बेशक हमारा परवरदिगार बड़ा बख्शने वाला (और) क़दरदान है
\end{hindi}}
\flushright{\begin{Arabic}
\quranayah[35][35]
\end{Arabic}}
\flushleft{\begin{hindi}
जिसने हमको अपने फज़ल (व करम) से हमेशगी के घर (बेहिश्त) में उतारा (मेहमान किया) जहाँ हमें कोई तकलीफ छुयेगी भी तो नहीं और न कोई थकान ही पहुँचेगी
\end{hindi}}
\flushright{\begin{Arabic}
\quranayah[35][36]
\end{Arabic}}
\flushleft{\begin{hindi}
और जो लोग काफिर हो बैठे उनके लिए जहन्नुम की आग है न उनकी कज़ा ही आएगी कि वह मर जाए और तकलीफ से नजात मिले और न उनसे उनके अज़ाब ही में तख़फीफ की जाएगी हम हर नाशुक्रे की सज़ा यूँ ही किया करते हैं
\end{hindi}}
\flushright{\begin{Arabic}
\quranayah[35][37]
\end{Arabic}}
\flushleft{\begin{hindi}
और ये लोग दोजख़ में (पड़े) चिल्लाया करेगें कि परवरदिगार अब हमको (यहाँ से) निकाल दे तो जो कुछ हम करते थे उसे छोड़कर नेक काम करेंगे (तो ख़ुदा जवाब देगा कि) क्या हमने तुम्हें इतनी उम्रें न दी थी कि जिनमें जिसको जो कुछ सोंचना समझना (मंज़ूर) हो खूब सोच समझ ले और (उसके अलावा) तुम्हारे पास (हमारा) डराने वाला (पैग़म्बर) भी पहुँच गया था तो (अपने किए का मज़ा) चखो क्योंकि सरकश लोगों का कोई मद्दगार नहीं
\end{hindi}}
\flushright{\begin{Arabic}
\quranayah[35][38]
\end{Arabic}}
\flushleft{\begin{hindi}
बेशक खुदा सारे आसमान व ज़मीन की पोशीदा बातों से ख़ूब वाक़िफ है वह यक़ीनी दिलों के पोशीदा राज़ से बाख़बर है
\end{hindi}}
\flushright{\begin{Arabic}
\quranayah[35][39]
\end{Arabic}}
\flushleft{\begin{hindi}
वह वही खुदा है जिसने रूए ज़मीन में तुम लोगों को (अगलों का) जानशीन बनाया फिर जो शख्स काफ़िर होगा तो उसके कुफ़्र का वबाल उसी पर पड़ेगा और काफ़िरों को उनका कुफ्र उनके परवरदिगार की बारगाह में ग़ज़ब के सिवा कुछ बढ़ाएगा नहीं और कुफ्फ़ार को उनका कुफ़्र घाटे के सिवा कुछ नफ़ा न देगा
\end{hindi}}
\flushright{\begin{Arabic}
\quranayah[35][40]
\end{Arabic}}
\flushleft{\begin{hindi}
(ऐ रसूल) तुम (उनसे) पूछो तो कि खुदा के सिवा अपने जिन शरीकों की तुम क़यादत करते थे क्या तुमने उन्हें (कुछ) देखा भी मुझे भी ज़रा दिखाओ तो कि उन्होंने ज़मीन (की चीज़ों) से कौन सी चीज़ पैदा की या आसमानों में कुछ उनका आधा साझा है या हमने खुद उन्हें कोई किताब दी है कि वह उसकी दलील रखते हैं (ये सब तो कुछ नहीं) बल्कि ये ज़ालिम एक दूसरे से (धोखे और) फरेब ही का वायदा करते हैं
\end{hindi}}
\flushright{\begin{Arabic}
\quranayah[35][41]
\end{Arabic}}
\flushleft{\begin{hindi}
बेशक खुदा ही सारे आसमान और ज़मीन अपनी जगह से हट जाने से रोके हुए है और अगर (फर्ज़ करे कि) ये अपनी जगह से हट जाए तो फिर तो उसके सिवा उन्हें कोई रोक नहीं सकता बेशक वह बड़ा बुर्दबर (और) बड़ा बख्शने वाला है
\end{hindi}}
\flushright{\begin{Arabic}
\quranayah[35][42]
\end{Arabic}}
\flushleft{\begin{hindi}
और ये लोग तो खुदा की बड़ी-बड़ी सख्त क़समें खा (कर कहते) थे कि बेशक अगर उनके पास कोई डराने वाला (पैग़म्बर) आएगा तो वह ज़रूर हर एक उम्मत से ज्यादा रूबसह होंगे फिर जब उनके पास डराने वाला (रसूल) आ पहुँचा तो (उन लोगों को) रूए ज़मीन में सरकशी और बुरी-बुरी तद्बीरें करने की वजह से
\end{hindi}}
\flushright{\begin{Arabic}
\quranayah[35][43]
\end{Arabic}}
\flushleft{\begin{hindi}
(उसके आने से) उनकी नफरत को तरक्की ही होती गयी और बुदी तद्बीर (की बुराई) तो बुरी तद्बीर करने वाले ही पर पड़ती है तो (हो न हो) ये लोग बस अगले ही लोगों के बरताव के मुन्तज़िर हैं तो (बेहतर) तुम खुदा के दसतूर में कभी तब्दीली न पाओगे और खुदा की आदत में हरगिज़ कोई तग़य्युर न पाओगे
\end{hindi}}
\flushright{\begin{Arabic}
\quranayah[35][44]
\end{Arabic}}
\flushleft{\begin{hindi}
तो क्या उन लोगों ने रूए ज़मीन पर चल फिर कर नहीं देखा कि जो लोग उनके पहले थे और उनसे ज़ोर व कूवत में भी कहीं बढ़-चढ़ के थे फिर उनका (नाफ़रमानी की वजह से) क्या (ख़राब) अन्जाम हुआ और खुदा ऐसा (गया गुज़रा) नहीं है कि उसे कोई चीज़ आजिज़ कर सके (न इतने) आसमानों में और न ज़मीन में बेशक वह बड़ा ख़बरदार (और) बड़ी (क़ाबू) कुदरत वाला है
\end{hindi}}
\flushright{\begin{Arabic}
\quranayah[35][45]
\end{Arabic}}
\flushleft{\begin{hindi}
और अगर (कहीं) खुदा लोगों की करतूतों की गिरफ्त करता तो (जैसी उनकी करनी है) रूए ज़मीन पर किसी जानवर को बाक़ी न छोड़ता मगर वह तो एक मुक़र्रर मियाद तक लोगों को मोहलत देता है (कि जो करना हो कर लो) फिर जब उनका (वह) वक्त अा जाएगा तो खुदा यक़ीनी तौर पर अपने बन्दों (के हाल) को देख रहा है (जो जैसा करेगा वैसा पाएगा)
\end{hindi}}
\chapter{Ya Sin (Ya Sin)}
\begin{Arabic}
\Huge{\centerline{\basmalah}}\end{Arabic}
\flushright{\begin{Arabic}
\quranayah[36][1]
\end{Arabic}}
\flushleft{\begin{hindi}
यासीन
\end{hindi}}
\flushright{\begin{Arabic}
\quranayah[36][2]
\end{Arabic}}
\flushleft{\begin{hindi}
इस पुरअज़ हिकमत कुरान की क़सम
\end{hindi}}
\flushright{\begin{Arabic}
\quranayah[36][3]
\end{Arabic}}
\flushleft{\begin{hindi}
(ऐ रसूल) तुम बिलाशक यक़ीनी पैग़म्बरों में से हो
\end{hindi}}
\flushright{\begin{Arabic}
\quranayah[36][4]
\end{Arabic}}
\flushleft{\begin{hindi}
(और दीन के बिल्कुल) सीधे रास्ते पर (साबित क़दम) हो
\end{hindi}}
\flushright{\begin{Arabic}
\quranayah[36][5]
\end{Arabic}}
\flushleft{\begin{hindi}
जो बड़े मेहरबान (और) ग़ालिब (खुदा) का नाज़िल किया हुआ (है)
\end{hindi}}
\flushright{\begin{Arabic}
\quranayah[36][6]
\end{Arabic}}
\flushleft{\begin{hindi}
ताकि तुम उन लोगों को (अज़ाबे खुदा से) डराओ जिनके बाप दादा (तुमसे पहले किसी पैग़म्बर से) डराए नहीं गए
\end{hindi}}
\flushright{\begin{Arabic}
\quranayah[36][7]
\end{Arabic}}
\flushleft{\begin{hindi}
तो वह दीन से बिल्कुल बेख़बर हैं उन में अक्सर तो (अज़ाब की) बातें यक़ीनन बिल्कुल ठीक पूरी उतरे ये लोग तो ईमान लाएँगे नहीं
\end{hindi}}
\flushright{\begin{Arabic}
\quranayah[36][8]
\end{Arabic}}
\flushleft{\begin{hindi}
हमने उनकी गर्दनों में (भारी-भारी लोहे के) तौक़ डाल दिए हैं और ठुड्डियों तक पहुँचे हुए हैं कि वह गर्दनें उठाए हुए हैं (सर झुका नहीं सकते)
\end{hindi}}
\flushright{\begin{Arabic}
\quranayah[36][9]
\end{Arabic}}
\flushleft{\begin{hindi}
हमने एक दीवार उनके आगे बना दी है और एक दीवार उनके पीछे फिर ऊपर से उनको ढाँक दिया है तो वह कुछ देख नहीं सकते
\end{hindi}}
\flushright{\begin{Arabic}
\quranayah[36][10]
\end{Arabic}}
\flushleft{\begin{hindi}
और (ऐ रसूल) उनके लिए बराबर है ख्वाह तुम उन्हें डराओ या न डराओ ये (कभी) ईमान लाने वाले नहीं हैं
\end{hindi}}
\flushright{\begin{Arabic}
\quranayah[36][11]
\end{Arabic}}
\flushleft{\begin{hindi}
तुम तो बस उसी शख्स को डरा सकते हो जो नसीहत माने और बेदेखे भाले खुदा का ख़ौफ़ रखे तो तुम उसको (गुनाहों की) माफी और एक बाइज्ज़त (व आबरू) अज्र की खुशख़बरी दे दो
\end{hindi}}
\flushright{\begin{Arabic}
\quranayah[36][12]
\end{Arabic}}
\flushleft{\begin{hindi}
हम ही यक़ीनन मुर्दों को ज़िन्दा करते हैं और जो कुछ लोग पहले कर चुके हैं (उनको) और उनकी (अच्छी या बुरी बाक़ी माँदा) निशानियों को लिखते जाते हैं और हमने हर चीज़ का एक सरीह व रौशन पेशवा में घेर दिया है
\end{hindi}}
\flushright{\begin{Arabic}
\quranayah[36][13]
\end{Arabic}}
\flushleft{\begin{hindi}
और (ऐ रसूल) तुम (इनसे) मिसाल के तौर पर एक गाँव (अता किया) वालों का क़िस्सा बयान करो जब वहाँ (हमारे) पैग़म्बर आए
\end{hindi}}
\flushright{\begin{Arabic}
\quranayah[36][14]
\end{Arabic}}
\flushleft{\begin{hindi}
इस तरह कि जब हमने उनके पास दो (पैग़म्बर योहना और यूनुस) भेजे तो उन लोगों ने दोनों को झुठलाया जब हमने एक तीसरे (पैग़म्बर शमऊन) से (उन दोनों को) मद्द दी तो इन तीनों ने कहा कि हम तुम्हारे पास खुदा के भेजे हुए (आए) हैं
\end{hindi}}
\flushright{\begin{Arabic}
\quranayah[36][15]
\end{Arabic}}
\flushleft{\begin{hindi}
वह लोग कहने लगे कि तुम लोग भी तो बस हमारे ही जैसे आदमी हो और खुदा ने कुछ नाज़िल (वाज़िल) नहीं किया है तुम सब के सब बस बिल्कुल झूठे हो
\end{hindi}}
\flushright{\begin{Arabic}
\quranayah[36][16]
\end{Arabic}}
\flushleft{\begin{hindi}
तब उन पैग़म्बरों ने कहा हमारा परवरदिगार जानता है कि हम यक़ीनन उसी के भेजे हुए (आए) हैं और (तुम मानो या न मानो)
\end{hindi}}
\flushright{\begin{Arabic}
\quranayah[36][17]
\end{Arabic}}
\flushleft{\begin{hindi}
हम पर तो बस खुल्लम खुल्ला एहकामे खुदा का पहुँचा देना फर्ज़ है
\end{hindi}}
\flushright{\begin{Arabic}
\quranayah[36][18]
\end{Arabic}}
\flushleft{\begin{hindi}
वह बोले हमने तुम लोगों को बहुत नहस क़दम पाया कि (तुम्हारे आते ही क़हत में मुबतेला हुए) तो अगर तुम (अपनी बातों से) बाज़ न आओगे तो हम लोग तुम्हें ज़रूर संगसार कर देगें और तुमको यक़ीनी हमारा दर्दनाक अज़ाब पहुँचेगा
\end{hindi}}
\flushright{\begin{Arabic}
\quranayah[36][19]
\end{Arabic}}
\flushleft{\begin{hindi}
पैग़म्बरों ने कहा कि तुम्हारी बद शुगूनी (तुम्हारी करनी से) तुम्हारे साथ है क्या जब नसीहत की जाती है (तो तुम उसे बदफ़ाली कहते हो नहीं) बल्कि तुम खुद (अपनी) हद से बढ़ गए हो
\end{hindi}}
\flushright{\begin{Arabic}
\quranayah[36][20]
\end{Arabic}}
\flushleft{\begin{hindi}
और (इतने में) शहर के उस सिरे से एक शख्स (हबीब नज्जार) दौड़ता हुआ आया और कहने लगा कि ऐ मेरी क़ौम (इन) पैग़म्बरों का कहना मानो
\end{hindi}}
\flushright{\begin{Arabic}
\quranayah[36][21]
\end{Arabic}}
\flushleft{\begin{hindi}
ऐसे लोगों का (ज़रूर) कहना मानो जो तुमसे (तबलीख़े रिसालत की) कुछ मज़दूरी नहीं माँगते और वह लोग हिदायत याफ्ता भी हैं
\end{hindi}}
\flushright{\begin{Arabic}
\quranayah[36][22]
\end{Arabic}}
\flushleft{\begin{hindi}
और मुझे क्या (ख़ब्त) हुआ है कि जिसने मुझे पैदा किया है उसकी इबादत न करूँ हालाँकि तुम सब के बस (आख़िर) उसी की तरफ लौटकर जाओगे
\end{hindi}}
\flushright{\begin{Arabic}
\quranayah[36][23]
\end{Arabic}}
\flushleft{\begin{hindi}
क्या मैं उसे छोड़कर दूसरों को माबूद बना लूँ अगर खुदा मुझे कोई तकलीफ पहुँचाना चाहे तो न उनकी सिफारिश ही मेरे कुछ काम आएगी और न ये लोग मुझे (इस मुसीबत से) छुड़ा ही सकेंगें
\end{hindi}}
\flushright{\begin{Arabic}
\quranayah[36][24]
\end{Arabic}}
\flushleft{\begin{hindi}
(अगर ऐसा करूँ) तो उस वक्त मैं यक़ीनी सरीही गुमराही में हूँ
\end{hindi}}
\flushright{\begin{Arabic}
\quranayah[36][25]
\end{Arabic}}
\flushleft{\begin{hindi}
मैं तो तुम्हारे परवरदिगार पर ईमान ला चुका हूँ मेरी बात सुनो और मानो; मगर उन लोगों ने उसे संगसार कर डाला
\end{hindi}}
\flushright{\begin{Arabic}
\quranayah[36][26]
\end{Arabic}}
\flushleft{\begin{hindi}
तब उसे खुदा का हुक्म हुआ कि बेहिश्त में जा (उस वक्त भी उसको क़ौम का ख्याल आया तो कहा)
\end{hindi}}
\flushright{\begin{Arabic}
\quranayah[36][27]
\end{Arabic}}
\flushleft{\begin{hindi}
मेरे परवरदिगार ने जो मुझे बख्श दिया और मुझे बुर्ज़ुग लोगों में शामिल कर दिया काश इसको मेरी क़ौम के लोग जान लेते और ईमान लाते
\end{hindi}}
\flushright{\begin{Arabic}
\quranayah[36][28]
\end{Arabic}}
\flushleft{\begin{hindi}
और हमने उसके मरने के बाद उसकी क़ौम पर उनकी तबाही के लिए न तो आसमान से कोई लशकर उतारा और न हम कभी इतनी सी बात के वास्ते लशकर उतारने वाले थे
\end{hindi}}
\flushright{\begin{Arabic}
\quranayah[36][29]
\end{Arabic}}
\flushleft{\begin{hindi}
वह तो सिर्फ एक चिंघाड थी (जो कर दी गयी बस) फिर तो वह फौरन चिराग़े सहरी की तरह बुझ के रह गए
\end{hindi}}
\flushright{\begin{Arabic}
\quranayah[36][30]
\end{Arabic}}
\flushleft{\begin{hindi}
हाए अफसोस बन्दों के हाल पर कि कभी उनके पास कोई रसूल नहीं आया मगर उन लोगों ने उसके साथ मसख़रापन ज़रूर किया
\end{hindi}}
\flushright{\begin{Arabic}
\quranayah[36][31]
\end{Arabic}}
\flushleft{\begin{hindi}
क्या उन लोगों ने इतना भी ग़ौर नहीं किया कि हमने उनसे पहले कितनी उम्मतों को हलाक कर डाला और वह लोग उनके पास हरगिज़ पलट कर नहीं आ सकते
\end{hindi}}
\flushright{\begin{Arabic}
\quranayah[36][32]
\end{Arabic}}
\flushleft{\begin{hindi}
(हाँ) अलबत्ता सब के सब इकट्ठा हो कर हमारी बारगाह में हाज़िर किए जाएँगे
\end{hindi}}
\flushright{\begin{Arabic}
\quranayah[36][33]
\end{Arabic}}
\flushleft{\begin{hindi}
और उनके (समझने) के लिए मेरी कुदरत की एक निशानी मुर्दा (परती) ज़मीन है कि हमने उसको (पानी से) ज़िन्दा कर दिया और हम ही ने उससे दाना निकाला तो उसे ये लोग खाया करते हैं
\end{hindi}}
\flushright{\begin{Arabic}
\quranayah[36][34]
\end{Arabic}}
\flushleft{\begin{hindi}
और हम ही ने ज़मीन में छुहारों और अंगूरों के बाग़ लगाए और हमही ने उसमें पानी के चशमें जारी किए
\end{hindi}}
\flushright{\begin{Arabic}
\quranayah[36][35]
\end{Arabic}}
\flushleft{\begin{hindi}
ताकि लोग उनके फल खाएँ और कुछ उनके हाथों ने उसे नहीं बनाया (बल्कि खुदा ने) तो क्या ये लोग (इस पर भी) शुक्र नहीं करते
\end{hindi}}
\flushright{\begin{Arabic}
\quranayah[36][36]
\end{Arabic}}
\flushleft{\begin{hindi}
वह (हर ऐब से) पाक साफ है जिसने ज़मीन से उगने वाली चीज़ों और खुद उन लोगों के और उन चीज़ों के जिनकी उन्हें ख़बर नहीं सबके जोड़े पैदा किए
\end{hindi}}
\flushright{\begin{Arabic}
\quranayah[36][37]
\end{Arabic}}
\flushleft{\begin{hindi}
और मेरी क़ुदरत की एक निशानी रात है जिससे हम दिन को खींच कर निकाल लेते (जाएल कर देते) हैं तो उस वक्त ये लोग अंधेरे में रह जाते हैं
\end{hindi}}
\flushright{\begin{Arabic}
\quranayah[36][38]
\end{Arabic}}
\flushleft{\begin{hindi}
और (एक निशानी) आफताब है जो अपने एक ठिकाने पर चल रहा है ये (सबसे) ग़ालिब वाक़िफ (खुदा) का (बाँद्दा हुआ) अन्दाज़ा है
\end{hindi}}
\flushright{\begin{Arabic}
\quranayah[36][39]
\end{Arabic}}
\flushleft{\begin{hindi}
और हमने चाँद के लिए मंज़िलें मुक़र्रर कर दीं हैं यहाँ तक कि हिर फिर के (आख़िर माह में) खजूर की पुरानी टहनी का सा (पतला टेढ़ा) हो जाता है
\end{hindi}}
\flushright{\begin{Arabic}
\quranayah[36][40]
\end{Arabic}}
\flushleft{\begin{hindi}
न तो आफताब ही से ये बन पड़ता है कि वह माहताब को जा ले और न रात ही दिन से आगे बढ़ सकती है (चाँद, सूरज, सितारे) हर एक अपने-अपने आसमान (मदार) में चक्कर लगा रहें हैं
\end{hindi}}
\flushright{\begin{Arabic}
\quranayah[36][41]
\end{Arabic}}
\flushleft{\begin{hindi}
और उनके लिए (मेरी कुदरत) की एक निशानी ये है कि उनके बुर्ज़ुगों को (नूह की) भरी हुई कश्ती में सवार किया
\end{hindi}}
\flushright{\begin{Arabic}
\quranayah[36][42]
\end{Arabic}}
\flushleft{\begin{hindi}
और उस कशती के मिसल उन लोगों के वास्ते भी वह चीज़े (कशतियाँ) जहाज़ पैदा कर दी
\end{hindi}}
\flushright{\begin{Arabic}
\quranayah[36][43]
\end{Arabic}}
\flushleft{\begin{hindi}
जिन पर ये लोग सवार हुआ करते हैं और अगर हम चाहें तो उन सब लोगों को डुबा मारें फिर न कोई उन का फरियाद रस होगा और न वह लोग छुटकारा ही पा सकते हैं
\end{hindi}}
\flushright{\begin{Arabic}
\quranayah[36][44]
\end{Arabic}}
\flushleft{\begin{hindi}
मगर हमारी मेहरबानी से और चूँकि एक (ख़ास) वक्त तक (उनको) चैन करने देना (मंज़ूर) है
\end{hindi}}
\flushright{\begin{Arabic}
\quranayah[36][45]
\end{Arabic}}
\flushleft{\begin{hindi}
और जब उन कुफ्फ़ार से कहा जाता है कि इस (अज़ाब से) बचो (हर वक्त तुम्हारे साथ-साथ) तुम्हारे सामने और तुम्हारे पीछे (मौजूद) है ताकि तुम पर रहम किया जाए
\end{hindi}}
\flushright{\begin{Arabic}
\quranayah[36][46]
\end{Arabic}}
\flushleft{\begin{hindi}
(तो परवाह नहीं करते) और उनकी हालत ये है कि जब उनके परवरदिगार की निशानियों में से कोई निशानी उनके पास आयी तो ये लोग मुँह मोड़े बग़ैर कभी नहीं रहे
\end{hindi}}
\flushright{\begin{Arabic}
\quranayah[36][47]
\end{Arabic}}
\flushleft{\begin{hindi}
और जब उन (कुफ्फ़ार) से कहा जाता है कि (माले दुनिया से) जो खुदा ने तुम्हें दिया है उसमें से कुछ (खुदा की राह में भी) ख़र्च करो तो (ये) कुफ्फ़ार ईमानवालों से कहते हैं कि भला हम उस शख्स को खिलाएँ जिसे (तुम्हारे ख्याल के मुवाफ़िक़) खुदा चाहता तो उसको खुद खिलाता कि तुम लोग बस सरीही गुमराही में (पड़े हुए) हो
\end{hindi}}
\flushright{\begin{Arabic}
\quranayah[36][48]
\end{Arabic}}
\flushleft{\begin{hindi}
और कहते हैं कि (भला) अगर तुम लोग (अपने दावे में सच्चे हो) तो आख़िर ये (क़यामत का) वायदा कब पूरा होगा
\end{hindi}}
\flushright{\begin{Arabic}
\quranayah[36][49]
\end{Arabic}}
\flushleft{\begin{hindi}
(ऐ रसूल) ये लोग एक सख्त चिंघाड़ (सूर) के मुनतज़िर हैं जो उन्हें (उस वक्त) ले डालेगी
\end{hindi}}
\flushright{\begin{Arabic}
\quranayah[36][50]
\end{Arabic}}
\flushleft{\begin{hindi}
जब ये लोग बाहम झगड़ रहे होगें फिर न तो ये लोग वसीयत ही करने पायेंगे और न अपने लड़के बालों ही की तरफ लौट कर जा सकेगें
\end{hindi}}
\flushright{\begin{Arabic}
\quranayah[36][51]
\end{Arabic}}
\flushleft{\begin{hindi}
और फिर (जब दोबारा) सूर फूँका जाएगा तो उसी दम ये सब लोग (अपनी-अपनी) क़ब्रों से (निकल-निकल के) अपने परवरदिगार की बारगाह की तरफ चल खड़े होगे
\end{hindi}}
\flushright{\begin{Arabic}
\quranayah[36][52]
\end{Arabic}}
\flushleft{\begin{hindi}
और (हैरान होकर) कहेगें हाए अफसोस हम तो पहले सो रहे थे हमें ख्वाबगाह से किसने उठाया (जवाब आएगा) कि ये वही (क़यामत का) दिन है जिसका खुदा ने (भी) वायदा किया था
\end{hindi}}
\flushright{\begin{Arabic}
\quranayah[36][53]
\end{Arabic}}
\flushleft{\begin{hindi}
और पैग़म्बरों ने भी सच कहा था (क़यामत तो) बस एक सख्त चिंघाड़ होगी फिर एका एकी ये लोग सब के सब हमारे हुजूर में हाज़िर किए जाएँगे
\end{hindi}}
\flushright{\begin{Arabic}
\quranayah[36][54]
\end{Arabic}}
\flushleft{\begin{hindi}
फिर आज (क़यामत के दिन) किसी शख्स पर कुछ भी ज़ुल्म न होगा और तुम लोगों को तो उसी का बदला दिया जाएगा जो तुम लोग (दुनिया में) किया करते थे
\end{hindi}}
\flushright{\begin{Arabic}
\quranayah[36][55]
\end{Arabic}}
\flushleft{\begin{hindi}
बेहश्त के रहने वाले आज (रोजे क़यामत) एक न एक मशग़ले में जी बहला रहे हैं
\end{hindi}}
\flushright{\begin{Arabic}
\quranayah[36][56]
\end{Arabic}}
\flushleft{\begin{hindi}
वह अपनी बीवियों के साथ (ठन्डी) छाँव में तकिया लगाए तख्तों पर (चैन से) बैठे हुए हैं
\end{hindi}}
\flushright{\begin{Arabic}
\quranayah[36][57]
\end{Arabic}}
\flushleft{\begin{hindi}
बेिहश्त में उनके लिए (ताज़ा) मेवे (तैयार) हैं और जो वह चाहें उनके लिए (हाज़िर) है
\end{hindi}}
\flushright{\begin{Arabic}
\quranayah[36][58]
\end{Arabic}}
\flushleft{\begin{hindi}
मेहरबान परवरदिगार की तरफ से सलाम का पैग़ाम आएगा
\end{hindi}}
\flushright{\begin{Arabic}
\quranayah[36][59]
\end{Arabic}}
\flushleft{\begin{hindi}
और (एक आवाज़ आएगी कि) ऐ गुनाहगारों तुम लोग (इनसे) अलग हो जाओ
\end{hindi}}
\flushright{\begin{Arabic}
\quranayah[36][60]
\end{Arabic}}
\flushleft{\begin{hindi}
ऐ आदम की औलाद क्या मैंने तुम्हारे पास ये हुक्म नहीं भेजा था कि (ख़बरदार) शैतान की परसतिश न करना वह यक़ीनी तुम्हारा खुल्लम खुल्ला दुश्मन है
\end{hindi}}
\flushright{\begin{Arabic}
\quranayah[36][61]
\end{Arabic}}
\flushleft{\begin{hindi}
और ये कि (देखो) सिर्फ मेरी इबादत करना यही (नजात की) सीधी राह है
\end{hindi}}
\flushright{\begin{Arabic}
\quranayah[36][62]
\end{Arabic}}
\flushleft{\begin{hindi}
और (बावजूद इसके) उसने तुममें से बहुतेरों को गुमराह कर छोड़ा तो क्या तुम (इतना भी) नहीं समझते थे
\end{hindi}}
\flushright{\begin{Arabic}
\quranayah[36][63]
\end{Arabic}}
\flushleft{\begin{hindi}
ये वही जहन्नुम है जिसका तुमसे वायदा किया गया था
\end{hindi}}
\flushright{\begin{Arabic}
\quranayah[36][64]
\end{Arabic}}
\flushleft{\begin{hindi}
तो अब चूँकि तुम कुफ्र करते थे इस वजह से आज इसमें (चुपके से) चले जाओ
\end{hindi}}
\flushright{\begin{Arabic}
\quranayah[36][65]
\end{Arabic}}
\flushleft{\begin{hindi}
आज हम उनके मुँह पर मुहर लगा देगें और (जो) कारसतानियाँ ये लोग दुनिया में कर रहे थे खुद उनके हाथ हमको बता देगें और उनके पाँव गवाही देगें
\end{hindi}}
\flushright{\begin{Arabic}
\quranayah[36][66]
\end{Arabic}}
\flushleft{\begin{hindi}
और अगर हम चाहें तो उनकी ऑंखों पर झाडू फेर दें तो ये लोग राह को पड़े चक्कर लगाते ढूँढते फिरें मगर कहाँ देख पाँएगे
\end{hindi}}
\flushright{\begin{Arabic}
\quranayah[36][67]
\end{Arabic}}
\flushleft{\begin{hindi}
और अगर हम चाहे तो जहाँ ये हैं (वहीं) उनकी सूरतें बदल (करके) (पत्थर मिट्टी बना) दें फिर न तो उनमें आगे जाने का क़ाबू रहे और न (घर) लौट सकें
\end{hindi}}
\flushright{\begin{Arabic}
\quranayah[36][68]
\end{Arabic}}
\flushleft{\begin{hindi}
और हम जिस शख्स को (बहुत) ज्यादा उम्र देते हैं तो उसे ख़िलक़त में उलट (कर बच्चों की तरह मजबूर कर) देते हैं तो क्या वह लोग समझते नहीं
\end{hindi}}
\flushright{\begin{Arabic}
\quranayah[36][69]
\end{Arabic}}
\flushleft{\begin{hindi}
और हमने न उस (पैग़म्बर) को शेर की तालीम दी है और न शायरी उसकी शान के लायक़ है ये (किताब) तो बस (निरी) नसीहत और साफ-साफ कुरान है
\end{hindi}}
\flushright{\begin{Arabic}
\quranayah[36][70]
\end{Arabic}}
\flushleft{\begin{hindi}
ताकि जो ज़िन्दा (दिल आक़िल) हों उसे (अज़ाब से) डराए और काफ़िरों पर (अज़ाब का) क़ौल साबित हो जाए (और हुज्जत बाक़ी न रहे)
\end{hindi}}
\flushright{\begin{Arabic}
\quranayah[36][71]
\end{Arabic}}
\flushleft{\begin{hindi}
क्या उन लोगों ने इस पर भी ग़ौर नहीं किया कि हमने उनके फायदे के लिए चारपाए उस चीज़ से पैदा किए जिसे हमारी ही क़ुदरत ने बनाया तो ये लोग (ख्वाहमाख्वाह) उनके मालिक बन गए
\end{hindi}}
\flushright{\begin{Arabic}
\quranayah[36][72]
\end{Arabic}}
\flushleft{\begin{hindi}
और हम ही ने चार पायों को उनका मुतीय बना दिया तो बाज़ उनकी सवारियां हैं और बाज़ को खाते हैं
\end{hindi}}
\flushright{\begin{Arabic}
\quranayah[36][73]
\end{Arabic}}
\flushleft{\begin{hindi}
और चार पायों में उनके (और) बहुत से फायदे हैं और पीने की चीज़ (दूध) तो क्या ये लोग (इस पर भी) शुक्र नहीं करते
\end{hindi}}
\flushright{\begin{Arabic}
\quranayah[36][74]
\end{Arabic}}
\flushleft{\begin{hindi}
और लोगों ने ख़ुदा को छोड़कर (फर्ज़ी माबूद बनाए हैं ताकि उन्हें उनसे कुछ मद्द मिले हालाँकि वह लोग उनकी किसी तरह मद्द कर ही नहीं सकते
\end{hindi}}
\flushright{\begin{Arabic}
\quranayah[36][75]
\end{Arabic}}
\flushleft{\begin{hindi}
और ये कुफ्फ़ार उन माबूदों के लशकर हैं (और क़यामत में) उन सबकी हाज़िरी ली जाएगी
\end{hindi}}
\flushright{\begin{Arabic}
\quranayah[36][76]
\end{Arabic}}
\flushleft{\begin{hindi}
तो (ऐ रसूल) तुम इनकी बातों से आज़ुरदा ख़ातिर (पेरशान) न हो जो कुछ ये लोग छिपा कर करते हैं और जो कुछ खुल्लम खुल्ला करते हैं-हम सबको यक़ीनी जानते हैं
\end{hindi}}
\flushright{\begin{Arabic}
\quranayah[36][77]
\end{Arabic}}
\flushleft{\begin{hindi}
क्या आदमी ने इस पर भी ग़ौर नहीं किया कि हम ही ने इसको एक ज़लील नुत्फे से पैदा किया फिर वह यकायक (हमारा ही) खुल्लम खुल्ला मुक़ाबिल (बना) है
\end{hindi}}
\flushright{\begin{Arabic}
\quranayah[36][78]
\end{Arabic}}
\flushleft{\begin{hindi}
और हमारी निसबत बातें बनाने लगा और अपनी ख़िलक़त (की हालत) भूल गया और कहने लगा कि भला जब ये हड्डियां (सड़गल कर) ख़ाक हो जाएँगी तो (फिर) कौन (दोबारा) ज़िन्दा कर सकता है
\end{hindi}}
\flushright{\begin{Arabic}
\quranayah[36][79]
\end{Arabic}}
\flushleft{\begin{hindi}
(ऐ रसूल) तुम कह दो कि उसको वही ज़िन्दा करेगा जिसने उनको (जब ये कुछ न थे) पहली बार ज़िन्दा कर (रखा)
\end{hindi}}
\flushright{\begin{Arabic}
\quranayah[36][80]
\end{Arabic}}
\flushleft{\begin{hindi}
और वह हर तरह की पैदाइश से वाक़िफ है जिसने तुम्हारे वास्ते (मिर्ख़ और अफ़ार के) हरे दरख्त से आग पैदा कर दी फिर तुम उससे (और) आग सुलगा लेते हो
\end{hindi}}
\flushright{\begin{Arabic}
\quranayah[36][81]
\end{Arabic}}
\flushleft{\begin{hindi}
(भला) जिस (खुदा) ने सारे आसमान और ज़मीन पैदा किए क्या वह इस पर क़ाबू नहीं रखता कि उनके मिस्ल (दोबारा) पैदा कर दे हाँ (ज़रूर क़ाबू रखता है) और वह तो पैदा करने वाला वाक़िफ़कार है
\end{hindi}}
\flushright{\begin{Arabic}
\quranayah[36][82]
\end{Arabic}}
\flushleft{\begin{hindi}
उसकी शान तो ये है कि जब किसी चीज़ को (पैदा करना) चाहता है तो वह कह देता है कि ''हो जा'' तो (फौरन) हो जाती है
\end{hindi}}
\flushright{\begin{Arabic}
\quranayah[36][83]
\end{Arabic}}
\flushleft{\begin{hindi}
तो वह ख़ुद (हर नफ्स से) पाक साफ़ है जिसके क़ब्ज़े कुदरत में हर चीज़ की हिकमत है और तुम लोग उसी की तरफ लौट कर जाओगे
\end{hindi}}
\chapter{As-Saffat (Those Ranging in Ranks)}
\begin{Arabic}
\Huge{\centerline{\basmalah}}\end{Arabic}
\flushright{\begin{Arabic}
\quranayah[37][1]
\end{Arabic}}
\flushleft{\begin{hindi}
(इबादत या जिहाद में) पर बाँधने वालों की (क़सम)
\end{hindi}}
\flushright{\begin{Arabic}
\quranayah[37][2]
\end{Arabic}}
\flushleft{\begin{hindi}
फिर (बदों को बुराई से) झिड़क कर डाँटने वाले की (क़सम)
\end{hindi}}
\flushright{\begin{Arabic}
\quranayah[37][3]
\end{Arabic}}
\flushleft{\begin{hindi}
फिर कुरान पढ़ने वालों की क़सम है
\end{hindi}}
\flushright{\begin{Arabic}
\quranayah[37][4]
\end{Arabic}}
\flushleft{\begin{hindi}
तुम्हारा माबूद (यक़ीनी) एक ही है
\end{hindi}}
\flushright{\begin{Arabic}
\quranayah[37][5]
\end{Arabic}}
\flushleft{\begin{hindi}
जो सारे आसमान ज़मीन का और जो कुछ इन दोनों के दरमियान है (सबका) परवरदिगार है
\end{hindi}}
\flushright{\begin{Arabic}
\quranayah[37][6]
\end{Arabic}}
\flushleft{\begin{hindi}
और (चाँद सूरज तारे के) तुलूउ व (गुरूब) के मक़ामात का भी मालिक है हम ही ने नीचे वाले आसमान को तारों की आरइश (जगमगाहट) से आरास्ता किया
\end{hindi}}
\flushright{\begin{Arabic}
\quranayah[37][7]
\end{Arabic}}
\flushleft{\begin{hindi}
और (तारों को) हर सरकश शैतान से हिफ़ाज़त के वास्ते (भी पैदा किया)
\end{hindi}}
\flushright{\begin{Arabic}
\quranayah[37][8]
\end{Arabic}}
\flushleft{\begin{hindi}
कि अब शैतान आलमे बाला की तरफ़ कान भी नहीं लगा सकते और (जहाँ सुन गुन लेना चाहा तो) हर तरफ़ से खदेड़ने के लिए शहाब फेके जाते हैं
\end{hindi}}
\flushright{\begin{Arabic}
\quranayah[37][9]
\end{Arabic}}
\flushleft{\begin{hindi}
और उनके लिए पाएदार अज़ाब है
\end{hindi}}
\flushright{\begin{Arabic}
\quranayah[37][10]
\end{Arabic}}
\flushleft{\begin{hindi}
मगर जो (शैतान शाज़ व नादिर फरिश्तों की) कोई बात उचक ले भागता है तो आग का दहकता हुआ तीर उसका पीछा करता है
\end{hindi}}
\flushright{\begin{Arabic}
\quranayah[37][11]
\end{Arabic}}
\flushleft{\begin{hindi}
तो (ऐ रसूल) तुम उनसे पूछो तो कि उनका पैदा करना ज्यादा दुश्वार है या उन (मज़कूरा) चीज़ों का जिनको हमने पैदा किया हमने तो उन लोगों को लसदार मिट्टी से पैदा किया
\end{hindi}}
\flushright{\begin{Arabic}
\quranayah[37][12]
\end{Arabic}}
\flushleft{\begin{hindi}
बल्कि तुम (उन कुफ्फ़ार के इन्कार पर) ताज्जुब करते हो और वह लोग (तुमसे) मसख़रापन करते हैं
\end{hindi}}
\flushright{\begin{Arabic}
\quranayah[37][13]
\end{Arabic}}
\flushleft{\begin{hindi}
और जब उन्हें समझाया जाता है तो समझते नहीं हैं
\end{hindi}}
\flushright{\begin{Arabic}
\quranayah[37][14]
\end{Arabic}}
\flushleft{\begin{hindi}
और जब किसी मौजिजे क़ो देखते हैं तो (उससे) मसख़रापन करते हैं
\end{hindi}}
\flushright{\begin{Arabic}
\quranayah[37][15]
\end{Arabic}}
\flushleft{\begin{hindi}
और कहते हैं कि ये तो बस खुला हुआ जादू है
\end{hindi}}
\flushright{\begin{Arabic}
\quranayah[37][16]
\end{Arabic}}
\flushleft{\begin{hindi}
भला जब हम मर जाएँगे और ख़ाक और हड्डियाँ रह जाएँगे
\end{hindi}}
\flushright{\begin{Arabic}
\quranayah[37][17]
\end{Arabic}}
\flushleft{\begin{hindi}
तो क्या हम या हमारे अगले बाप दादा फिर दोबारा क़ब्रों से उठा खड़े किए जाँएगे
\end{hindi}}
\flushright{\begin{Arabic}
\quranayah[37][18]
\end{Arabic}}
\flushleft{\begin{hindi}
(ऐ रसूल) तुम कह दो कि हाँ (ज़रूर उठाए जाओगे)
\end{hindi}}
\flushright{\begin{Arabic}
\quranayah[37][19]
\end{Arabic}}
\flushleft{\begin{hindi}
और तुम ज़लील होगे और वह (क़यामत) तो एक ललकार होगी फिर तो वह लोग फ़ौरन ही (ऑंखे फाड़-फाड़ के) देखने लगेंगे
\end{hindi}}
\flushright{\begin{Arabic}
\quranayah[37][20]
\end{Arabic}}
\flushleft{\begin{hindi}
और कहेंगे हाए अफसोस ये तो क़यामत का दिन है
\end{hindi}}
\flushright{\begin{Arabic}
\quranayah[37][21]
\end{Arabic}}
\flushleft{\begin{hindi}
(जवाब आएगा) ये वही फैसले का दिन है जिसको तुम लोग (दुनिया में) झूठ समझते थे
\end{hindi}}
\flushright{\begin{Arabic}
\quranayah[37][22]
\end{Arabic}}
\flushleft{\begin{hindi}
(और फ़रिश्तों को हुक्म होगा कि) जो लोग (दुनिया में) सरकशी करते थे उनको और उनके साथियों को और खुदा को छोड़कर जिनकी परसतिश करते हैं
\end{hindi}}
\flushright{\begin{Arabic}
\quranayah[37][23]
\end{Arabic}}
\flushleft{\begin{hindi}
उनको (सबको) इकट्ठा करो फिर उन्हें जहन्नुम की राह दिखाओ
\end{hindi}}
\flushright{\begin{Arabic}
\quranayah[37][24]
\end{Arabic}}
\flushleft{\begin{hindi}
और (हाँ ज़रा) उन्हें ठहराओ तो उनसे कुछ पूछना है
\end{hindi}}
\flushright{\begin{Arabic}
\quranayah[37][25]
\end{Arabic}}
\flushleft{\begin{hindi}
(अरे कमबख्तों) अब तुम्हें क्या होगा कि एक दूसरे की मदद नहीं करते
\end{hindi}}
\flushright{\begin{Arabic}
\quranayah[37][26]
\end{Arabic}}
\flushleft{\begin{hindi}
(जवाब क्या देंगे) बल्कि वह तो आज गर्दन झुकाए हुए हैं
\end{hindi}}
\flushright{\begin{Arabic}
\quranayah[37][27]
\end{Arabic}}
\flushleft{\begin{hindi}
और एक दूसरे की तरफ मुतावज्जे होकर बाहम पूछताछ करेंगे
\end{hindi}}
\flushright{\begin{Arabic}
\quranayah[37][28]
\end{Arabic}}
\flushleft{\begin{hindi}
(और इन्सान शयातीन से) कहेंगे कि तुम ही तो हमारी दाहिनी तरफ से (हमें बहकाने को) चढ़ आते थे
\end{hindi}}
\flushright{\begin{Arabic}
\quranayah[37][29]
\end{Arabic}}
\flushleft{\begin{hindi}
वह जवाब देगें (हम क्या जानें) तुम तो खुद ईमान लाने वाले न थे
\end{hindi}}
\flushright{\begin{Arabic}
\quranayah[37][30]
\end{Arabic}}
\flushleft{\begin{hindi}
और (साफ़ तो ये है कि) हमारी तुम पर कुछ हुकूमत तो थी नहीं बल्कि तुम खुद सरकश लोग थे
\end{hindi}}
\flushright{\begin{Arabic}
\quranayah[37][31]
\end{Arabic}}
\flushleft{\begin{hindi}
फिर अब तो लोगों पर हमारे परवरदिगार का (अज़ाब का) क़ौल पूरा हो गया कि अब हम सब यक़ीनन अज़ाब का मज़ा चखेंगे
\end{hindi}}
\flushright{\begin{Arabic}
\quranayah[37][32]
\end{Arabic}}
\flushleft{\begin{hindi}
हम खुद गुमराह थे तो तुम को भी गुमराह किया
\end{hindi}}
\flushright{\begin{Arabic}
\quranayah[37][33]
\end{Arabic}}
\flushleft{\begin{hindi}
ग़रज़ ये लोग सब के सब उस दिन अज़ाब में शरीक होगें
\end{hindi}}
\flushright{\begin{Arabic}
\quranayah[37][34]
\end{Arabic}}
\flushleft{\begin{hindi}
और हम तो गुनाहगारों के साथ यूँ ही किया करते हैं ये लोग ऐसे (शरीर) थे
\end{hindi}}
\flushright{\begin{Arabic}
\quranayah[37][35]
\end{Arabic}}
\flushleft{\begin{hindi}
कि जब उनसे कहा जाता था कि खुदा के सिवा कोई माबूद नहीं तो अकड़ा करते थे
\end{hindi}}
\flushright{\begin{Arabic}
\quranayah[37][36]
\end{Arabic}}
\flushleft{\begin{hindi}
और ये लोग कहते थे कि क्या एक पागल शायर के लिए हम अपने माबूदों को छोड़ बैठें (अरे कम्बख्तों ये शायर या पागल नहीं)
\end{hindi}}
\flushright{\begin{Arabic}
\quranayah[37][37]
\end{Arabic}}
\flushleft{\begin{hindi}
बल्कि ये तो हक़ बात लेकर आया है और (अगले) पैग़म्बरों की तसदीक़ करता है
\end{hindi}}
\flushright{\begin{Arabic}
\quranayah[37][38]
\end{Arabic}}
\flushleft{\begin{hindi}
तुम लोग (अगर न मानोगे) तो ज़रूर दर्दनाक अज़ाब का मज़ा चखोगे
\end{hindi}}
\flushright{\begin{Arabic}
\quranayah[37][39]
\end{Arabic}}
\flushleft{\begin{hindi}
और तुम्हें तो उसके किये का बदला दिया जाएगा जो (जो दुनिया में) करते रहे
\end{hindi}}
\flushright{\begin{Arabic}
\quranayah[37][40]
\end{Arabic}}
\flushleft{\begin{hindi}
मगर खुदा के बरगुजीदा बन्दे
\end{hindi}}
\flushright{\begin{Arabic}
\quranayah[37][41]
\end{Arabic}}
\flushleft{\begin{hindi}
उनके वास्ते (बेहिश्त में) एक मुक़र्रर रोज़ी होगी
\end{hindi}}
\flushright{\begin{Arabic}
\quranayah[37][42]
\end{Arabic}}
\flushleft{\begin{hindi}
(और वह भी ऐसी वैसी नहीं) हर क़िस्म के मेवे
\end{hindi}}
\flushright{\begin{Arabic}
\quranayah[37][43]
\end{Arabic}}
\flushleft{\begin{hindi}
और वह लोग बड़ी इज्ज़त से नेअमत के (लदे हुए)
\end{hindi}}
\flushright{\begin{Arabic}
\quranayah[37][44]
\end{Arabic}}
\flushleft{\begin{hindi}
बाग़ों में तख्तों पर (चैन से) आमने सामने बैठे होगे
\end{hindi}}
\flushright{\begin{Arabic}
\quranayah[37][45]
\end{Arabic}}
\flushleft{\begin{hindi}
उनमें साफ सफेद बुर्राक़ शराब के जाम का दौर चल रहा होगा
\end{hindi}}
\flushright{\begin{Arabic}
\quranayah[37][46]
\end{Arabic}}
\flushleft{\begin{hindi}
जो पीने वालों को बड़ा मज़ा देगी
\end{hindi}}
\flushright{\begin{Arabic}
\quranayah[37][47]
\end{Arabic}}
\flushleft{\begin{hindi}
(और फिर) न उस शराब में ख़ुमार की वजह से) दर्द सर होगा और न वह उस (के पीने) से मतवाले होंगे
\end{hindi}}
\flushright{\begin{Arabic}
\quranayah[37][48]
\end{Arabic}}
\flushleft{\begin{hindi}
और उनके पहलू में (शर्म से) नीची निगाहें करने वाली बड़ी बड़ी ऑंखों वाली परियाँ होगी
\end{hindi}}
\flushright{\begin{Arabic}
\quranayah[37][49]
\end{Arabic}}
\flushleft{\begin{hindi}
(उनकी) गोरी-गोरी रंगतों में हल्की सी सुर्ख़ी ऐसी झलकती होगी
\end{hindi}}
\flushright{\begin{Arabic}
\quranayah[37][50]
\end{Arabic}}
\flushleft{\begin{hindi}
गोया वह अन्डे हैं जो छिपाए हुए रखे हो
\end{hindi}}
\flushright{\begin{Arabic}
\quranayah[37][51]
\end{Arabic}}
\flushleft{\begin{hindi}
फिर एक दूसरे की तरफ मुतावज्जे पाकर बाहम बातचीत करते करते उनमें से एक कहने वाला बोल उठेगा कि (दुनिया में) मेरा एक दोस्त था
\end{hindi}}
\flushright{\begin{Arabic}
\quranayah[37][52]
\end{Arabic}}
\flushleft{\begin{hindi}
और (मुझसे) कहा करता था कि क्या तुम भी क़यामत की तसदीक़ करने वालों में हो
\end{hindi}}
\flushright{\begin{Arabic}
\quranayah[37][53]
\end{Arabic}}
\flushleft{\begin{hindi}
(भला जब हम मर जाएँगे) और (सड़ गल कर) मिट्टी और हव्ी (होकर) रह जाएँगे तो क्या हमको दोबारा ज़िन्दा करके हमारे (आमाल का) बदला दिया जाएगा
\end{hindi}}
\flushright{\begin{Arabic}
\quranayah[37][54]
\end{Arabic}}
\flushleft{\begin{hindi}
(फिर अपने बेहश्त के साथियों से कहेगा)
\end{hindi}}
\flushright{\begin{Arabic}
\quranayah[37][55]
\end{Arabic}}
\flushleft{\begin{hindi}
तो क्या तुम लोग भी (मेरे साथ उसे झांक कर देखोगे) ग़रज़ झाँका तो उसे बीच जहन्नुम में (पड़ा हुआ) देखा
\end{hindi}}
\flushright{\begin{Arabic}
\quranayah[37][56]
\end{Arabic}}
\flushleft{\begin{hindi}
(ये देख कर बेसाख्ता) बोल उठेगा कि खुदा की क़सम तुम तो मुझे भी तबाह करने ही को थे
\end{hindi}}
\flushright{\begin{Arabic}
\quranayah[37][57]
\end{Arabic}}
\flushleft{\begin{hindi}
और अगर मेरे परवरदिगार का एहसान न होता तो मैं भी (इस वक्त) तेरी तरह जहन्नुम में गिरफ्तार किया गया होता
\end{hindi}}
\flushright{\begin{Arabic}
\quranayah[37][58]
\end{Arabic}}
\flushleft{\begin{hindi}
(अब बताओ) क्या (मैं तुम से न कहता था) कि हम को इस पहली मौत के सिवा फिर मरना नहीं है
\end{hindi}}
\flushright{\begin{Arabic}
\quranayah[37][59]
\end{Arabic}}
\flushleft{\begin{hindi}
और न हम पर (आख़ेरत) में अज़ाब होगा
\end{hindi}}
\flushright{\begin{Arabic}
\quranayah[37][60]
\end{Arabic}}
\flushleft{\begin{hindi}
(तो तुम्हें यक़ीन न होता था) ये यक़ीनी बहुत बड़ी कामयाबी है
\end{hindi}}
\flushright{\begin{Arabic}
\quranayah[37][61]
\end{Arabic}}
\flushleft{\begin{hindi}
ऐसी (ही कामयाबी) के वास्ते काम करने वालों को कारगुज़ारी करनी चाहिए
\end{hindi}}
\flushright{\begin{Arabic}
\quranayah[37][62]
\end{Arabic}}
\flushleft{\begin{hindi}
भला मेहमानी के वास्ते ये (सामान) बेहतर है या थोहड़ का दरख्त (जो जहन्नुमियों के वास्ते होगा)
\end{hindi}}
\flushright{\begin{Arabic}
\quranayah[37][63]
\end{Arabic}}
\flushleft{\begin{hindi}
जिसे हमने यक़ीनन ज़ालिमों की आज़माइश के लिए बनाया है
\end{hindi}}
\flushright{\begin{Arabic}
\quranayah[37][64]
\end{Arabic}}
\flushleft{\begin{hindi}
ये वह दरख्त हैं जो जहन्नुम की तह में उगता है
\end{hindi}}
\flushright{\begin{Arabic}
\quranayah[37][65]
\end{Arabic}}
\flushleft{\begin{hindi}
उसके फल ऐसे (बदनुमा) हैं गोया (हू बहू) साँप के फन जिसे छूते दिल डरे
\end{hindi}}
\flushright{\begin{Arabic}
\quranayah[37][66]
\end{Arabic}}
\flushleft{\begin{hindi}
फिर ये (जहन्नुमी लोग) यक़ीनन उसमें से खाएँगे फिर उसी से अपने पेट भरेंगे
\end{hindi}}
\flushright{\begin{Arabic}
\quranayah[37][67]
\end{Arabic}}
\flushleft{\begin{hindi}
फिर उसके ऊपर से उन को खूब खौलता हुआ पानी (पीप वग़ैरह में) मिला मिलाकर पीने को दिया जाएगा
\end{hindi}}
\flushright{\begin{Arabic}
\quranayah[37][68]
\end{Arabic}}
\flushleft{\begin{hindi}
फिर (खा पीकर) उनको जहन्नुम की तरफ यक़ीनन लौट जाना होगा
\end{hindi}}
\flushright{\begin{Arabic}
\quranayah[37][69]
\end{Arabic}}
\flushleft{\begin{hindi}
उन लोगों ने अपन बाप दादा को गुमराह पाया था
\end{hindi}}
\flushright{\begin{Arabic}
\quranayah[37][70]
\end{Arabic}}
\flushleft{\begin{hindi}
ये लोग भी उनके पीछे दौड़े चले जा रहे हैं
\end{hindi}}
\flushright{\begin{Arabic}
\quranayah[37][71]
\end{Arabic}}
\flushleft{\begin{hindi}
और उनके क़ब्ल अगलों में से बहुतेरे गुमराह हो चुके
\end{hindi}}
\flushright{\begin{Arabic}
\quranayah[37][72]
\end{Arabic}}
\flushleft{\begin{hindi}
उन लोगों के डराने वाले (पैग़म्बरों) को भेजा था
\end{hindi}}
\flushright{\begin{Arabic}
\quranayah[37][73]
\end{Arabic}}
\flushleft{\begin{hindi}
ज़रा देखो तो कि जो लोग डराए जा चुके थे उनका क्या बुरा अन्जाम हुआ
\end{hindi}}
\flushright{\begin{Arabic}
\quranayah[37][74]
\end{Arabic}}
\flushleft{\begin{hindi}
मगर (हाँ) खुदा के निरे खरे बन्दे (महफूज़ रहे)
\end{hindi}}
\flushright{\begin{Arabic}
\quranayah[37][75]
\end{Arabic}}
\flushleft{\begin{hindi}
और नूह ने (अपनी कौम से मायूस होकर) हमें ज़रूर पुकारा था (देखो हम) क्या खूब जवाब देने वाले थे
\end{hindi}}
\flushright{\begin{Arabic}
\quranayah[37][76]
\end{Arabic}}
\flushleft{\begin{hindi}
और हमने उनको और उनके लड़के वालों को बड़ी (सख्त) मुसीबत से नजात दी
\end{hindi}}
\flushright{\begin{Arabic}
\quranayah[37][77]
\end{Arabic}}
\flushleft{\begin{hindi}
और हमने (उनमें वह बरकत दी कि) उनकी औलाद को (दुनिया में) बरक़रार रखा
\end{hindi}}
\flushright{\begin{Arabic}
\quranayah[37][78]
\end{Arabic}}
\flushleft{\begin{hindi}
और बाद को आने वाले लोगों में उनका अच्छा चर्चा बाक़ी रखा
\end{hindi}}
\flushright{\begin{Arabic}
\quranayah[37][79]
\end{Arabic}}
\flushleft{\begin{hindi}
कि सारी खुदायी में (हर तरफ से) नूह पर सलाम है
\end{hindi}}
\flushright{\begin{Arabic}
\quranayah[37][80]
\end{Arabic}}
\flushleft{\begin{hindi}
हम नेकी करने वालों को यूँ जज़ाए ख़ैर अता फरमाते हैं
\end{hindi}}
\flushright{\begin{Arabic}
\quranayah[37][81]
\end{Arabic}}
\flushleft{\begin{hindi}
इसमें शक नहीं कि नूह हमारे (ख़ास) ईमानदार बन्दों से थे
\end{hindi}}
\flushright{\begin{Arabic}
\quranayah[37][82]
\end{Arabic}}
\flushleft{\begin{hindi}
फिर हमने बाक़ी लोगों को डुबो दिया
\end{hindi}}
\flushright{\begin{Arabic}
\quranayah[37][83]
\end{Arabic}}
\flushleft{\begin{hindi}
और यक़ीनन उन्हीं के तरीक़ो पर चलने वालों में इबराहीम (भी) ज़रूर थे
\end{hindi}}
\flushright{\begin{Arabic}
\quranayah[37][84]
\end{Arabic}}
\flushleft{\begin{hindi}
जब वह अपने परवरदिगार (कि इबादत) की तरफ (पहलू में) ऐसा दिल लिए हुए बढ़े जो (हर ऐब से पाक था
\end{hindi}}
\flushright{\begin{Arabic}
\quranayah[37][85]
\end{Arabic}}
\flushleft{\begin{hindi}
जब उन्होंने अपने (मुँह बोले) बाप और अपनी क़ौम से कहा कि तुम लोग किस चीज़ की परसतिश करते हो
\end{hindi}}
\flushright{\begin{Arabic}
\quranayah[37][86]
\end{Arabic}}
\flushleft{\begin{hindi}
क्या खुदा को छोड़कर दिल से गढ़े हुए माबूदों की तमन्ना रखते हो
\end{hindi}}
\flushright{\begin{Arabic}
\quranayah[37][87]
\end{Arabic}}
\flushleft{\begin{hindi}
फिर सारी खुदाई के पालने वाले के साथ तुम्हारा क्या ख्याल है
\end{hindi}}
\flushright{\begin{Arabic}
\quranayah[37][88]
\end{Arabic}}
\flushleft{\begin{hindi}
फिर (एक ईद में उन लोगों ने चलने को कहा) तो इबराहीम ने सितारों की तरफ़ एक नज़र देखा
\end{hindi}}
\flushright{\begin{Arabic}
\quranayah[37][89]
\end{Arabic}}
\flushleft{\begin{hindi}
और कहा कि मैं (अनक़रीब) बीमार पड़ने वाला हूँ
\end{hindi}}
\flushright{\begin{Arabic}
\quranayah[37][90]
\end{Arabic}}
\flushleft{\begin{hindi}
तो वह लोग इबराहीम के पास से पीठ फेर फेर कर हट गए
\end{hindi}}
\flushright{\begin{Arabic}
\quranayah[37][91]
\end{Arabic}}
\flushleft{\begin{hindi}
(बस) फिर तो इबराहीम चुपके से उनके बुतों की तरफ मुतावज्जे हुए और (तान से) कहा तुम्हारे सामने इतने चढ़ाव रखते हैं
\end{hindi}}
\flushright{\begin{Arabic}
\quranayah[37][92]
\end{Arabic}}
\flushleft{\begin{hindi}
आख़िर तुम खाते क्यों नहीं (अरे तुम्हें क्या हो गया है)
\end{hindi}}
\flushright{\begin{Arabic}
\quranayah[37][93]
\end{Arabic}}
\flushleft{\begin{hindi}
कि तुम बोलते तक नहीं
\end{hindi}}
\flushright{\begin{Arabic}
\quranayah[37][94]
\end{Arabic}}
\flushleft{\begin{hindi}
फिर तो इबराहीम दाहिने हाथ से मारते हुए उन पर पिल पड़े (और तोड़-फोड़ कर एक बड़े बुत के गले में कुल्हाड़ी डाल दी)
\end{hindi}}
\flushright{\begin{Arabic}
\quranayah[37][95]
\end{Arabic}}
\flushleft{\begin{hindi}
जब उन लोगों को ख़बर हुई तो इबराहीम के पास दौड़ते हुए पहुँचे
\end{hindi}}
\flushright{\begin{Arabic}
\quranayah[37][96]
\end{Arabic}}
\flushleft{\begin{hindi}
इबराहीम ने कहा (अफ़सोस) तुम लोग उसकी परसतिश करते हो जिसे तुम लोग खुद तराश कर बनाते हो
\end{hindi}}
\flushright{\begin{Arabic}
\quranayah[37][97]
\end{Arabic}}
\flushleft{\begin{hindi}
हालाँकि तुमको और जिसको तुम लोग बनाते हो (सबको) खुदा ही ने पैदा किया है (ये सुनकर) वह लोग (आपस में कहने लगे) इसके लिए (भट्टी की सी) एक इमारत बनाओ
\end{hindi}}
\flushright{\begin{Arabic}
\quranayah[37][98]
\end{Arabic}}
\flushleft{\begin{hindi}
और (उसमें आग सुलगा कर उसी दहकती हुई आग में इसको डाल दो) फिर उन लोगों ने इबराहीम के साथ मक्कारी करनी चाही
\end{hindi}}
\flushright{\begin{Arabic}
\quranayah[37][99]
\end{Arabic}}
\flushleft{\begin{hindi}
तो हमने (आग सर्द गुलज़ार करके) उन्हें नीचा दिखाया और जब (आज़र ने) इबराहीम को निकाल दिया तो बोले मैं अपने परवरदिगार की तरफ जाता हूँ
\end{hindi}}
\flushright{\begin{Arabic}
\quranayah[37][100]
\end{Arabic}}
\flushleft{\begin{hindi}
वह अनक़रीब ही मुझे रूबरा कर देगा (फिर ग़रज की) परवरदिगार मुझे एक नेको कार (फरज़न्द) इनायत फरमा
\end{hindi}}
\flushright{\begin{Arabic}
\quranayah[37][101]
\end{Arabic}}
\flushleft{\begin{hindi}
तो हमने उनको एक बड़े नरम दिले लड़के (के पैदा होने की) खुशख़बरी दी
\end{hindi}}
\flushright{\begin{Arabic}
\quranayah[37][102]
\end{Arabic}}
\flushleft{\begin{hindi}
फिर जब इस्माईल अपने बाप के साथ दौड़ धूप करने लगा तो (एक दफा) इबराहीम ने कहा बेटा खूब मैं (वही के ज़रिये क्या) देखता हूँ कि मैं तो खुद तुम्हें ज़िबाह कर रहा हूँ तो तुम भी ग़ौर करो तुम्हारी इसमें क्या राय है इसमाईल ने कहा अब्बा जान जो आपको हुक्म हुआ है उसको (बे तअम्मुल) कीजिए अगर खुदा ने चाहा तो मुझे आप सब्र करने वालों में से पाएगे
\end{hindi}}
\flushright{\begin{Arabic}
\quranayah[37][103]
\end{Arabic}}
\flushleft{\begin{hindi}
फिर जब दोनों ने ये ठान ली और बाप ने बेटे को (ज़िबाह करने के लिए) माथे के बल लिटाया
\end{hindi}}
\flushright{\begin{Arabic}
\quranayah[37][104]
\end{Arabic}}
\flushleft{\begin{hindi}
और हमने (आमादा देखकर) आवाज़ दी ऐ इबराहीम
\end{hindi}}
\flushright{\begin{Arabic}
\quranayah[37][105]
\end{Arabic}}
\flushleft{\begin{hindi}
तुमने अपने ख्वाब को सच कर दिखाया अब तुम दोनों को बड़े मरतबे मिलेगें हम नेकी करने वालों को यूँ जज़ाए ख़ैर देते हैं
\end{hindi}}
\flushright{\begin{Arabic}
\quranayah[37][106]
\end{Arabic}}
\flushleft{\begin{hindi}
इसमें शक नहीं कि ये यक़ीनी बड़ा सख्त और सरीही इम्तिहान था
\end{hindi}}
\flushright{\begin{Arabic}
\quranayah[37][107]
\end{Arabic}}
\flushleft{\begin{hindi}
और हमने इस्माईल का फ़िदया एक ज़िबाहे अज़ीम (बड़ी कुर्बानी) क़रार दिया
\end{hindi}}
\flushright{\begin{Arabic}
\quranayah[37][108]
\end{Arabic}}
\flushleft{\begin{hindi}
और हमने उनका अच्छा चर्चा बाद को आने वालों में बाक़ी रखा है
\end{hindi}}
\flushright{\begin{Arabic}
\quranayah[37][109]
\end{Arabic}}
\flushleft{\begin{hindi}
कि (सारी खुदायी में) इबराहीम पर सलाम (ही सलाम) हैं
\end{hindi}}
\flushright{\begin{Arabic}
\quranayah[37][110]
\end{Arabic}}
\flushleft{\begin{hindi}
हम यूँ नेकी करने वालों को जज़ाए ख़ैर देते हैं
\end{hindi}}
\flushright{\begin{Arabic}
\quranayah[37][111]
\end{Arabic}}
\flushleft{\begin{hindi}
बेशक इबराहीम हमारे (ख़ास) ईमानदार बन्दों में थे
\end{hindi}}
\flushright{\begin{Arabic}
\quranayah[37][112]
\end{Arabic}}
\flushleft{\begin{hindi}
और हमने इबराहीम को इसहाक़ (के पैदा होने की) खुशख़बरी दी थी
\end{hindi}}
\flushright{\begin{Arabic}
\quranayah[37][113]
\end{Arabic}}
\flushleft{\begin{hindi}
जो एक नेकोसार नबी थे और हमने खुद इबराहीम पर और इसहाक़ पर अपनी बरकत नाज़िल की और इन दोनों की नस्ल में बाज़ तो नेकोकार और बाज़ (नाफरमानी करके) अपनी जान पर सरीही सितम ढ़ाने वाला
\end{hindi}}
\flushright{\begin{Arabic}
\quranayah[37][114]
\end{Arabic}}
\flushleft{\begin{hindi}
और हमने मूसा और हारून पर बहुत से एहसानात किए हैं
\end{hindi}}
\flushright{\begin{Arabic}
\quranayah[37][115]
\end{Arabic}}
\flushleft{\begin{hindi}
और खुद दोनों को और इनकी क़ौम को बड़ी (सख्त) मुसीबत से नजात दी
\end{hindi}}
\flushright{\begin{Arabic}
\quranayah[37][116]
\end{Arabic}}
\flushleft{\begin{hindi}
और (फिरऔन के मुक़ाबले में) हमने उनकी मदद की तो (आख़िर) यही लोग ग़ालिब रहे
\end{hindi}}
\flushright{\begin{Arabic}
\quranayah[37][117]
\end{Arabic}}
\flushleft{\begin{hindi}
और हमने उन दोनों को एक वाज़ेए उलम तालिब किताब (तौरेत) अता की
\end{hindi}}
\flushright{\begin{Arabic}
\quranayah[37][118]
\end{Arabic}}
\flushleft{\begin{hindi}
और दोनों को सीधी राह की हिदायत फ़रमाई
\end{hindi}}
\flushright{\begin{Arabic}
\quranayah[37][119]
\end{Arabic}}
\flushleft{\begin{hindi}
और बाद को आने वालों में उनका ज़िक्रे ख़ैर बाक़ी रखा
\end{hindi}}
\flushright{\begin{Arabic}
\quranayah[37][120]
\end{Arabic}}
\flushleft{\begin{hindi}
कि (हर जगह) मूसा और हारून पर सलाम (ही सलाम) है
\end{hindi}}
\flushright{\begin{Arabic}
\quranayah[37][121]
\end{Arabic}}
\flushleft{\begin{hindi}
हम नेकी करने वालों को यूँ जज़ाए ख़ैर अता फरमाते हैं
\end{hindi}}
\flushright{\begin{Arabic}
\quranayah[37][122]
\end{Arabic}}
\flushleft{\begin{hindi}
बेशक ये दोनों हमारे (ख़ालिस ईमानदार बन्दों में से थे)
\end{hindi}}
\flushright{\begin{Arabic}
\quranayah[37][123]
\end{Arabic}}
\flushleft{\begin{hindi}
और इसमें शक नहीं कि इलियास यक़ीनन पैग़म्बरों में से थे
\end{hindi}}
\flushright{\begin{Arabic}
\quranayah[37][124]
\end{Arabic}}
\flushleft{\begin{hindi}
जब उन्होंने अपनी क़ौम से कहा कि तुम लोग (ख़ुदा से) क्यों नहीं डरते
\end{hindi}}
\flushright{\begin{Arabic}
\quranayah[37][125]
\end{Arabic}}
\flushleft{\begin{hindi}
क्या तुम लोग बाल (बुत) की परसतिश करते हो और खुदा को छोड़े बैठे हो जो सबसे बेहतर पैदा करने वाला है
\end{hindi}}
\flushright{\begin{Arabic}
\quranayah[37][126]
\end{Arabic}}
\flushleft{\begin{hindi}
और (जो) तुम्हारा परवरदिगार और तुम्हारे अगले बाप दादाओं का (भी) परवरदिगार है
\end{hindi}}
\flushright{\begin{Arabic}
\quranayah[37][127]
\end{Arabic}}
\flushleft{\begin{hindi}
तो उसे लोगों ने झुठला दिया तो ये लोग यक़ीनन (जहन्नुम) में गिरफ्तार किए जाएँगे
\end{hindi}}
\flushright{\begin{Arabic}
\quranayah[37][128]
\end{Arabic}}
\flushleft{\begin{hindi}
मगर खुदा के निरे खरे बन्दे महफूज़ रहेंगे
\end{hindi}}
\flushright{\begin{Arabic}
\quranayah[37][129]
\end{Arabic}}
\flushleft{\begin{hindi}
और हमने उनका ज़िक्र ख़ैर बाद को आने वालों में बाक़ी रखा
\end{hindi}}
\flushright{\begin{Arabic}
\quranayah[37][130]
\end{Arabic}}
\flushleft{\begin{hindi}
कि (हर तरफ से) आले यासीन पर सलाम (ही सलाम) है
\end{hindi}}
\flushright{\begin{Arabic}
\quranayah[37][131]
\end{Arabic}}
\flushleft{\begin{hindi}
हम यक़ीनन नेकी करने वालों को ऐसा ही बदला दिया करते हैं
\end{hindi}}
\flushright{\begin{Arabic}
\quranayah[37][132]
\end{Arabic}}
\flushleft{\begin{hindi}
बेशक वह हमारे (ख़ालिस) ईमानदार बन्दों में थे
\end{hindi}}
\flushright{\begin{Arabic}
\quranayah[37][133]
\end{Arabic}}
\flushleft{\begin{hindi}
और इसमें भी शक नहीं कि लूत यक़ीनी पैग़म्बरों में से थे
\end{hindi}}
\flushright{\begin{Arabic}
\quranayah[37][134]
\end{Arabic}}
\flushleft{\begin{hindi}
जब हमने उनको और उनके लड़के वालों सब को नजात दी
\end{hindi}}
\flushright{\begin{Arabic}
\quranayah[37][135]
\end{Arabic}}
\flushleft{\begin{hindi}
मगर एक (उनकी) बूढ़ी बीबी जो पीछे रह जाने वालों ही में थीं
\end{hindi}}
\flushright{\begin{Arabic}
\quranayah[37][136]
\end{Arabic}}
\flushleft{\begin{hindi}
फिर हमने बाक़ी लोगों को तबाह व बर्बाद कर दिया
\end{hindi}}
\flushright{\begin{Arabic}
\quranayah[37][137]
\end{Arabic}}
\flushleft{\begin{hindi}
और ऐ अहले मक्का तुम लोग भी उन पर से (कभी) सुबह को और (कभी) शाम को (आते जाते गुज़रते हो)
\end{hindi}}
\flushright{\begin{Arabic}
\quranayah[37][138]
\end{Arabic}}
\flushleft{\begin{hindi}
तो क्या तुम (इतना भी) नहीं समझते
\end{hindi}}
\flushright{\begin{Arabic}
\quranayah[37][139]
\end{Arabic}}
\flushleft{\begin{hindi}
और इसमें शक नहीं कि यूनुस (भी) पैग़म्बरों में से थे
\end{hindi}}
\flushright{\begin{Arabic}
\quranayah[37][140]
\end{Arabic}}
\flushleft{\begin{hindi}
(वह वक्त याद करो) जब यूनुस भाग कर एक भरी हुई कश्ती के पास पहुँचे
\end{hindi}}
\flushright{\begin{Arabic}
\quranayah[37][141]
\end{Arabic}}
\flushleft{\begin{hindi}
तो (अहले कश्ती ने) कुरआ डाला तो (उनका ही नाम निकला) यूनुस ने ज़क उठायी (और दरिया में गिर पड़े)
\end{hindi}}
\flushright{\begin{Arabic}
\quranayah[37][142]
\end{Arabic}}
\flushleft{\begin{hindi}
तो उनको एक मछली निगल गयी और यूनुस खुद (अपनी) मलामत कर रहे थे
\end{hindi}}
\flushright{\begin{Arabic}
\quranayah[37][143]
\end{Arabic}}
\flushleft{\begin{hindi}
फिर अगर यूनुस (खुदा की) तसबीह (व ज़िक्र) न करते
\end{hindi}}
\flushright{\begin{Arabic}
\quranayah[37][144]
\end{Arabic}}
\flushleft{\begin{hindi}
तो रोज़े क़यामत तक मछली के पेट में रहते
\end{hindi}}
\flushright{\begin{Arabic}
\quranayah[37][145]
\end{Arabic}}
\flushleft{\begin{hindi}
फिर हमने उनको (मछली के पेट से निकाल कर) एक खुले मैदान में डाल दिया
\end{hindi}}
\flushright{\begin{Arabic}
\quranayah[37][146]
\end{Arabic}}
\flushleft{\begin{hindi}
और (वह थोड़ी देर में) बीमार निढाल हो गए थे और हमने उन पर साये के लिए एक कद्दू का दरख्त उगा दिया
\end{hindi}}
\flushright{\begin{Arabic}
\quranayah[37][147]
\end{Arabic}}
\flushleft{\begin{hindi}
और (इसके बाद) हमने एक लाख बल्कि (एक हिसाब से) ज्यादा आदमियों की तरफ (पैग़म्बर बना कर भेजा)
\end{hindi}}
\flushright{\begin{Arabic}
\quranayah[37][148]
\end{Arabic}}
\flushleft{\begin{hindi}
तो वह लोग (उन पर) ईमान लाए फिर हमने (भी) एक ख़ास वक्त तक उनको चैन से रखा
\end{hindi}}
\flushright{\begin{Arabic}
\quranayah[37][149]
\end{Arabic}}
\flushleft{\begin{hindi}
तो (ऐ रसूल) उन कुफ्फ़ार से पूछो कि क्या तुम्हारे परवरदिगार के लिए बेटियाँ हैं और उनके लिए बेटे
\end{hindi}}
\flushright{\begin{Arabic}
\quranayah[37][150]
\end{Arabic}}
\flushleft{\begin{hindi}
(क्या वाक़ई) हमने फरिश्तों की औरतें बनाया है और ये लोग (उस वक्त) मौजूद थे
\end{hindi}}
\flushright{\begin{Arabic}
\quranayah[37][151]
\end{Arabic}}
\flushleft{\begin{hindi}
ख़बरदार (याद रखो कि) ये लोग यक़ीनन अपने दिल से गढ़-गढ़ के कहते हैं कि खुदा औलाद वाला है
\end{hindi}}
\flushright{\begin{Arabic}
\quranayah[37][152]
\end{Arabic}}
\flushleft{\begin{hindi}
और ये लोग यक़ीनी झूठे हैं
\end{hindi}}
\flushright{\begin{Arabic}
\quranayah[37][153]
\end{Arabic}}
\flushleft{\begin{hindi}
क्या खुदा ने (अपने लिए) बेटियों को बेटों पर तरजीह दी है
\end{hindi}}
\flushright{\begin{Arabic}
\quranayah[37][154]
\end{Arabic}}
\flushleft{\begin{hindi}
(अरे कम्बख्तों) तुम्हें क्या जुनून हो गया है तुम लोग (बैठे-बैठे) कैसा फैसला करते हो
\end{hindi}}
\flushright{\begin{Arabic}
\quranayah[37][155]
\end{Arabic}}
\flushleft{\begin{hindi}
तो क्या तुम (इतना भी) ग़ौर नहीं करते
\end{hindi}}
\flushright{\begin{Arabic}
\quranayah[37][156]
\end{Arabic}}
\flushleft{\begin{hindi}
या तुम्हारे पास (इसकी) कोई वाज़ेए व रौशन दलील है
\end{hindi}}
\flushright{\begin{Arabic}
\quranayah[37][157]
\end{Arabic}}
\flushleft{\begin{hindi}
तो अगर तुम (अपने दावे में) सच्चे हो तो अपनी किताब पेश करो
\end{hindi}}
\flushright{\begin{Arabic}
\quranayah[37][158]
\end{Arabic}}
\flushleft{\begin{hindi}
और उन लोगों ने खुदा और जिन्नात के दरमियान रिश्ता नाता मुक़र्रर किया है हालाँकि जिन्नात बखूबी जानते हैं कि वह लोग यक़ीनी (क़यामत में बन्दों की तरह) हाज़िर किए जाएँगे
\end{hindi}}
\flushright{\begin{Arabic}
\quranayah[37][159]
\end{Arabic}}
\flushleft{\begin{hindi}
ये लोग जो बातें बनाया करते हैं इनसे खुदा पाक साफ़ है
\end{hindi}}
\flushright{\begin{Arabic}
\quranayah[37][160]
\end{Arabic}}
\flushleft{\begin{hindi}
मगर खुदा के निरे खरे बन्दे (ऐसा नहीं कहते)
\end{hindi}}
\flushright{\begin{Arabic}
\quranayah[37][161]
\end{Arabic}}
\flushleft{\begin{hindi}
ग़रज़ तुम लोग खुद और तुम्हारे माबूद
\end{hindi}}
\flushright{\begin{Arabic}
\quranayah[37][162]
\end{Arabic}}
\flushleft{\begin{hindi}
उसके ख़िलाफ (किसी को) बहका नहीं सकते
\end{hindi}}
\flushright{\begin{Arabic}
\quranayah[37][163]
\end{Arabic}}
\flushleft{\begin{hindi}
मगर उसको जो जहन्नुम में झोंका जाने वाला है
\end{hindi}}
\flushright{\begin{Arabic}
\quranayah[37][164]
\end{Arabic}}
\flushleft{\begin{hindi}
और फरिश्ते या आइम्मा तो ये कहते हैं कि मैं हर एक का एक दरजा मुक़र्रर है
\end{hindi}}
\flushright{\begin{Arabic}
\quranayah[37][165]
\end{Arabic}}
\flushleft{\begin{hindi}
और हम तो यक़ीनन (उसकी इबादत के लिए) सफ बाँधे खड़े रहते हैं
\end{hindi}}
\flushright{\begin{Arabic}
\quranayah[37][166]
\end{Arabic}}
\flushleft{\begin{hindi}
और हम तो यक़ीनी (उसकी) तस्बीह पढ़ा करते हैं
\end{hindi}}
\flushright{\begin{Arabic}
\quranayah[37][167]
\end{Arabic}}
\flushleft{\begin{hindi}
अगरचे ये कुफ्फार (इस्लाम के क़ब्ल) कहा करते थे
\end{hindi}}
\flushright{\begin{Arabic}
\quranayah[37][168]
\end{Arabic}}
\flushleft{\begin{hindi}
कि अगर हमारे पास भी अगले लोगों का तज़किरा (किसी किताबे खुदा में) होता
\end{hindi}}
\flushright{\begin{Arabic}
\quranayah[37][169]
\end{Arabic}}
\flushleft{\begin{hindi}
तो हम भी खुदा के निरे खरे बन्दे ज़रूर हो जाते
\end{hindi}}
\flushright{\begin{Arabic}
\quranayah[37][170]
\end{Arabic}}
\flushleft{\begin{hindi}
(मगर जब किताब आयी) तो उन लोगों ने उससे इन्कार किया ख़ैर अनक़रीब (उसका नतीजा) उन्हें मालूम हो जाएगा
\end{hindi}}
\flushright{\begin{Arabic}
\quranayah[37][171]
\end{Arabic}}
\flushleft{\begin{hindi}
और अपने ख़ास बन्दों पैग़म्बरों से हमारी बात पक्की हो चुकी है
\end{hindi}}
\flushright{\begin{Arabic}
\quranayah[37][172]
\end{Arabic}}
\flushleft{\begin{hindi}
कि इन लोगों की (हमारी बारगाह से) यक़ीनी मदद की जाएगी
\end{hindi}}
\flushright{\begin{Arabic}
\quranayah[37][173]
\end{Arabic}}
\flushleft{\begin{hindi}
और हमारा लश्कर तो यक़ीनन ग़ालिब रहेगा
\end{hindi}}
\flushright{\begin{Arabic}
\quranayah[37][174]
\end{Arabic}}
\flushleft{\begin{hindi}
तो (ऐ रसूल) तुम उनसे एक ख़ास वक्त तक मुँह फेरे रहो
\end{hindi}}
\flushright{\begin{Arabic}
\quranayah[37][175]
\end{Arabic}}
\flushleft{\begin{hindi}
और इनको देखते रहो तो ये लोग अनक़रीब ही (अपना नतीजा) देख लेगे
\end{hindi}}
\flushright{\begin{Arabic}
\quranayah[37][176]
\end{Arabic}}
\flushleft{\begin{hindi}
तो क्या ये लोग हमारे अज़ाब की जल्दी कर रहे हैं
\end{hindi}}
\flushright{\begin{Arabic}
\quranayah[37][177]
\end{Arabic}}
\flushleft{\begin{hindi}
फिर जब (अज़ाब) उनकी अंगनाई में उतर पडेग़ा तो जो लोग डराए जा चुके हैं उनकी भी क्या बुरी सुबह होगी
\end{hindi}}
\flushright{\begin{Arabic}
\quranayah[37][178]
\end{Arabic}}
\flushleft{\begin{hindi}
और उन लोगों से एक ख़ास वक्त तक मुँह फेरे रहो
\end{hindi}}
\flushright{\begin{Arabic}
\quranayah[37][179]
\end{Arabic}}
\flushleft{\begin{hindi}
और देखते रहो ये लोग तो खुद अनक़रीब ही अपना अन्जाम देख लेगें
\end{hindi}}
\flushright{\begin{Arabic}
\quranayah[37][180]
\end{Arabic}}
\flushleft{\begin{hindi}
ये लोग जो बातें (खुदा के बारे में) बनाया करते हैं उनसे तुम्हारा परवरदिगार इज्ज़त का मालिक पाक साफ है
\end{hindi}}
\flushright{\begin{Arabic}
\quranayah[37][181]
\end{Arabic}}
\flushleft{\begin{hindi}
और पैग़म्बरों पर (दुरूद) सलाम हो
\end{hindi}}
\flushright{\begin{Arabic}
\quranayah[37][182]
\end{Arabic}}
\flushleft{\begin{hindi}
और कुल तारीफ खुदा ही के लिए सज़ावार हैं जो सारे जहाँन का पालने वाला है
\end{hindi}}
\chapter{Sad (Sad)}
\begin{Arabic}
\Huge{\centerline{\basmalah}}\end{Arabic}
\flushright{\begin{Arabic}
\quranayah[38][1]
\end{Arabic}}
\flushleft{\begin{hindi}
सआद नसीहत करने वाले कुरान की क़सम (तुम बरहक़ नबी हो)
\end{hindi}}
\flushright{\begin{Arabic}
\quranayah[38][2]
\end{Arabic}}
\flushleft{\begin{hindi}
मगर ये कुफ्फ़ार (ख्वाहमख्वाह) तकब्बुर और अदावत में (पड़े अंधे हो रहें हैं)
\end{hindi}}
\flushright{\begin{Arabic}
\quranayah[38][3]
\end{Arabic}}
\flushleft{\begin{hindi}
हमने उन से पहले कितने गिरोह हलाक कर डाले तो (अज़ाब के वक्त) ये लोग चीख़ उठे मगर छुटकारे का वक्त ही न रहा था
\end{hindi}}
\flushright{\begin{Arabic}
\quranayah[38][4]
\end{Arabic}}
\flushleft{\begin{hindi}
और उन लोगों ने इस बात से ताज्जुब किया कि उन्हीं में का (अज़ाबे खुदा से) एक डरानेवाला (पैग़म्बर) उनके पास आया और काफिर लोग कहने लगे कि ये तो बड़ा (खिलाड़ी) जादूगर और पक्का झूठा है
\end{hindi}}
\flushright{\begin{Arabic}
\quranayah[38][5]
\end{Arabic}}
\flushleft{\begin{hindi}
भला (देखो तो) उसने तमाम माबूदों को (मटियामेट करके बस) एक ही माबूद क़ायम रखा ये तो यक़ीनी बड़ी ताज्जुब खेज़ बात है
\end{hindi}}
\flushright{\begin{Arabic}
\quranayah[38][6]
\end{Arabic}}
\flushleft{\begin{hindi}
और उनमें से चन्द रवादार लोग (मजलिस व अज़ा से) ये (कह कर) चल खड़े हुए कि (यहाँ से) चल दो और अपने माबूदों की इबादत पर जमे रहो यक़ीनन इसमें (उसकी) कुछ ज़ाती ग़रज़ है
\end{hindi}}
\flushright{\begin{Arabic}
\quranayah[38][7]
\end{Arabic}}
\flushleft{\begin{hindi}
हम लोगों ने तो ये बात पिछले दीन में कभी सुनी भी नहीं हो न हो ये उसकी मन गढ़ंत है
\end{hindi}}
\flushright{\begin{Arabic}
\quranayah[38][8]
\end{Arabic}}
\flushleft{\begin{hindi}
क्या हम सब लोगों में बस (मोहम्मद ही क़ाबिल था कि) उस पर कुरान नाज़िल हुआ, नहीं बात ये है कि इनके (सिरे से) मेरे कलाम ही में शक है कि मेरा है या नहीं बल्कि असल ये है कि इन लोगों ने अभी तक अज़ाब के मज़े नहीं चखे
\end{hindi}}
\flushright{\begin{Arabic}
\quranayah[38][9]
\end{Arabic}}
\flushleft{\begin{hindi}
(इस वजह से ये शरारत है) (ऐ रसूल) तुम्हारे ज़बरदस्त फ़य्याज़ परवरदिगार के रहमत के ख़ज़ाने इनके पास हैं
\end{hindi}}
\flushright{\begin{Arabic}
\quranayah[38][10]
\end{Arabic}}
\flushleft{\begin{hindi}
या सारे आसमान व ज़मीन और उन दोनों के दरमियान की सलतनत इन्हीं की ख़ास है तब इनको चाहिए कि रास्ते या सीढियाँ लगाकर (आसमान पर) चढ़ जाएँ और इन्तेज़ाम करें
\end{hindi}}
\flushright{\begin{Arabic}
\quranayah[38][11]
\end{Arabic}}
\flushleft{\begin{hindi}
(ऐ रसूल उन पैग़म्बरों के साथ झगड़ने वाले) गिरोहों में से यहाँ तुम्हारे मुक़ाबले में भी एक लशकर है जो शिकस्त खाएगा
\end{hindi}}
\flushright{\begin{Arabic}
\quranayah[38][12]
\end{Arabic}}
\flushleft{\begin{hindi}
उनसे पहले नूह की क़ौम और आद और फिरऔन मेंख़ों वाला
\end{hindi}}
\flushright{\begin{Arabic}
\quranayah[38][13]
\end{Arabic}}
\flushleft{\begin{hindi}
और समूद और लूत की क़ौम और जंगल के रहने वाले (क़ौम शुऐब ये सब पैग़म्बरों को) झुठला चुकी हैं यही वह गिरोह है
\end{hindi}}
\flushright{\begin{Arabic}
\quranayah[38][14]
\end{Arabic}}
\flushleft{\begin{hindi}
(जो शिकस्त खा चुके) सब ही ने तो पैग़म्बरों को झुठलाया तो हमारा अज़ाब ठीक आ नाज़िल हुआ
\end{hindi}}
\flushright{\begin{Arabic}
\quranayah[38][15]
\end{Arabic}}
\flushleft{\begin{hindi}
और ये (काफिर) लोग बस एक चिंघाड़ (सूर के मुन्तज़िर हैं जो फिर उन्हें) चश्में ज़दन की मोहलत न देगी
\end{hindi}}
\flushright{\begin{Arabic}
\quranayah[38][16]
\end{Arabic}}
\flushleft{\begin{hindi}
और ये लोग (मज़ाक से) कहते हैं कि परवरदिगार हिसाब के दिन (क़यामत के) क़ब्ल ही (जो) हमारी क़िस्मत को लिखा (हो) हमें जल्दी दे दे
\end{hindi}}
\flushright{\begin{Arabic}
\quranayah[38][17]
\end{Arabic}}
\flushleft{\begin{hindi}
(ऐ रसूल) जैसी जैसी बातें ये लोग करते हैं उन पर सब्र करो और हमारे बन्दे दाऊद को याद करो जो बड़े कूवत वाले थे
\end{hindi}}
\flushright{\begin{Arabic}
\quranayah[38][18]
\end{Arabic}}
\flushleft{\begin{hindi}
बेशक वह हमारी बारगाह में बड़े रूजू करने वाले थे हमने पहाड़ों को भी ताबेदार बना दिया था कि उनके साथ सुबह और शाम (खुदा की) तस्बीह करते थे
\end{hindi}}
\flushright{\begin{Arabic}
\quranayah[38][19]
\end{Arabic}}
\flushleft{\begin{hindi}
और परिन्दे भी (यादे खुदा के वक्त सिमट) आते और उनके फरमाबरदार थे
\end{hindi}}
\flushright{\begin{Arabic}
\quranayah[38][20]
\end{Arabic}}
\flushleft{\begin{hindi}
और हमने उनकी सल्तनत को मज़बूत कर दिया और हमने उनको हिकमत और बहस के फैसले की कूवत अता फरमायी थी
\end{hindi}}
\flushright{\begin{Arabic}
\quranayah[38][21]
\end{Arabic}}
\flushleft{\begin{hindi}
(ऐ रसूल) क्या तुम तक उन दावेदारों की भी ख़बर पहुँची है कि जब वह हुजरे (इबादत) की दीवार फाँद पडे
\end{hindi}}
\flushright{\begin{Arabic}
\quranayah[38][22]
\end{Arabic}}
\flushleft{\begin{hindi}
(और) जब दाऊद के पास आ खड़े हुए तो वह उनसे डर गए उन लोगों ने कहा कि आप डरें नहीं (हम दोनों) एक मुक़द्दमें के फ़रीकैन हैं कि हम में से एक ने दूसरे पर ज्यादती की है तो आप हमारे दरमियान ठीक-ठीक फैसला कर दीजिए और इन्साफ से ने गुज़रिये और हमें सीधी राह दिखा दीजिए
\end{hindi}}
\flushright{\begin{Arabic}
\quranayah[38][23]
\end{Arabic}}
\flushleft{\begin{hindi}
(मुराद ये हैं कि) ये (शख्स) मेरा भाई है और उसके पास निनान्नवे दुम्बियाँ हैं और मेरे पास सिर्फ एक दुम्बी है उस पर भी ये मुझसे कहता है कि ये दुम्बी भी मुझी को दे दें और बातचीत में मुझ पर सख्ती करता है
\end{hindi}}
\flushright{\begin{Arabic}
\quranayah[38][24]
\end{Arabic}}
\flushleft{\begin{hindi}
दाऊद ने (बग़ैर इसके कि मुदा आलैह से कुछ पूछें) कह दिया कि ये जो तेरी दुम्बी माँग कर अपनी दुम्बियों में मिलाना चाहता है तो ये तुझ पर ज़ुल्म करता है और अक्सर शुरका (की) यकीनन (ये हालत है कि) एक दूसरे पर जुल्म किया करते हैं मगर जिन लोगों ने (सच्चे दिल से) ईमान कुबूल किया और अच्छे (अच्छे) काम किए (वह ऐसा नहीं करते) और ऐसे लोग बहुत ही कम हैं (ये सुनकर दोनों चल दिए) और अब दाऊद ने समझा कि हमने उनका इमितेहान लिया (और वह ना कामयाब रहे) फिर तो अपने परवरदिगार से बख्शिश की दुआ माँगने लगे और सजदे में गिर पड़े और (मेरी) तरफ रूजू की (24) (सजदा)
\end{hindi}}
\flushright{\begin{Arabic}
\quranayah[38][25]
\end{Arabic}}
\flushleft{\begin{hindi}
तो हमने उनकी वह ग़लती माफ कर दी और इसमें शक नहीं कि हमारी बारगाह में उनका तक़र्रुब और अन्जाम अच्छा हुआ
\end{hindi}}
\flushright{\begin{Arabic}
\quranayah[38][26]
\end{Arabic}}
\flushleft{\begin{hindi}
(हमने फरमाया) ऐ दाऊद हमने तुमको ज़मीन में (अपना) नाएब क़रार दिया तो तुम लोगों के दरमियान बिल्कुल ठीक फैसला किया करो और नफ़सियानी ख्वाहिश की पैरवी न करो बसा ये पीरों तुम्हें ख़ुदा की राह से बहका देगी इसमें शक नहीं कि जो लोग खुदा की राह में भटकते हैं उनकी बड़ी सख्त सज़ा होगी क्योंकि उन लोगों ने हिसाब के दिन (क़यामत) को भुला दिया
\end{hindi}}
\flushright{\begin{Arabic}
\quranayah[38][27]
\end{Arabic}}
\flushleft{\begin{hindi}
और हमने आसमान और ज़मीन और जो चीज़ें उन दोनों के दरमियान हैं बेकार नहीं पैदा किया ये उन लोगों का ख्याल है जो काफ़िर हो बैठे तो जो लोग दोज़ख़ के मुनकिर हैं उन पर अफ़सोस है
\end{hindi}}
\flushright{\begin{Arabic}
\quranayah[38][28]
\end{Arabic}}
\flushleft{\begin{hindi}
क्या जिन लोगों ने ईमान कुबूल किया और अच्छे-अच्छे काम किए उनको हम (उन लोगों के बराबर) कर दें जो रूए ज़मीन में फसाद फैलाया करते हैं या हम परहेज़गारों को मिसल बदकारों के बना दें
\end{hindi}}
\flushright{\begin{Arabic}
\quranayah[38][29]
\end{Arabic}}
\flushleft{\begin{hindi}
(ऐ रसूल) किताब (कुरान) जो हमने तुम्हारे पास नाज़िल की है (बड़ी) बरकत वाली है ताकि लोग इसकी आयतों में ग़ौर करें और ताकि अक्ल वाले नसीहत हासिल करें
\end{hindi}}
\flushright{\begin{Arabic}
\quranayah[38][30]
\end{Arabic}}
\flushleft{\begin{hindi}
और हमने दाऊद को सुलेमान (सा बेटा) अता किया (सुलेमान भी) क्या अच्छे बन्दे थे
\end{hindi}}
\flushright{\begin{Arabic}
\quranayah[38][31]
\end{Arabic}}
\flushleft{\begin{hindi}
बेशक वह हमारी तरफ रूजू करने वाले थे इत्तोफाक़न एक दफ़ा तीसरे पहर को ख़ासे के असील घोड़े उनके सामने पेश किए गए
\end{hindi}}
\flushright{\begin{Arabic}
\quranayah[38][32]
\end{Arabic}}
\flushleft{\begin{hindi}
तो देखने में उलझे के नवाफिल में देर हो गयी जब याद आया तो बोले कि मैंने अपने परवरदिगार की याद पर माल की उलफ़त को तरजीह दी यहाँ तक कि आफ़ताब (मग़रिब के) पर्दे में छुप गया
\end{hindi}}
\flushright{\begin{Arabic}
\quranayah[38][33]
\end{Arabic}}
\flushleft{\begin{hindi}
(तो बोले अच्छा) इन घोड़ों को मेरे पास वापस लाओ (जब आए) तो (देर के कफ्फ़ारा में) घोड़ों की टाँगों और गर्दनों पर हाथ फेर (काट) ने लगे
\end{hindi}}
\flushright{\begin{Arabic}
\quranayah[38][34]
\end{Arabic}}
\flushleft{\begin{hindi}
और हमने सुलेमान का इम्तेहान लिया और उनके तख्त पर एक बेजान धड़ लाकर गिरा दिया
\end{hindi}}
\flushright{\begin{Arabic}
\quranayah[38][35]
\end{Arabic}}
\flushleft{\begin{hindi}
फिर (सुलेमान ने मेरी तरफ) रूजू की (और) कहा परवरदिगार मुझे बख्श दे और मुझे वह मुल्क अता फरमा जो मेरे बाद किसी के वास्ते शायाँह न हो इसमें तो शक नहीं कि तू बड़ा बख्शने वाला है
\end{hindi}}
\flushright{\begin{Arabic}
\quranayah[38][36]
\end{Arabic}}
\flushleft{\begin{hindi}
तो हमने हवा को उनका ताबेए कर दिया कि जहाँ वह पहुँचना चाहते थे उनके हुक्म के मुताबिक़ धीमी चाल चलती थी
\end{hindi}}
\flushright{\begin{Arabic}
\quranayah[38][37]
\end{Arabic}}
\flushleft{\begin{hindi}
और (इसी तरह) जितने शयातीन (देव) इमारत बनाने वाले और ग़ोता लगाने वाले थे
\end{hindi}}
\flushright{\begin{Arabic}
\quranayah[38][38]
\end{Arabic}}
\flushleft{\begin{hindi}
सबको (ताबेए कर दिया और इसके अलावा) दूसरे देवों को भी जो ज़ंज़ीरों में जकड़े हुए थे
\end{hindi}}
\flushright{\begin{Arabic}
\quranayah[38][39]
\end{Arabic}}
\flushleft{\begin{hindi}
ऐ सुलेमान ये हमारी बेहिसाब अता है पस (उसे लोगों को देकर) एहसान करो या (सब) अपने ही पास रखो
\end{hindi}}
\flushright{\begin{Arabic}
\quranayah[38][40]
\end{Arabic}}
\flushleft{\begin{hindi}
और इसमें शक नहीं कि सुलेमान की हमारी बारगाह में कुर्ब व मज़ेलत और उमदा जगह है
\end{hindi}}
\flushright{\begin{Arabic}
\quranayah[38][41]
\end{Arabic}}
\flushleft{\begin{hindi}
और (ऐ रसूल) हमारे (ख़ास) बन्दे अय्यूब को याद करो जब उन्होंने अपने परवरगिार से फरियाद की कि मुझको शैतान ने बहुत अज़ीयत और तकलीफ पहुँचा रखी है
\end{hindi}}
\flushright{\begin{Arabic}
\quranayah[38][42]
\end{Arabic}}
\flushleft{\begin{hindi}
तो हमने कहा कि अपने पाँव से (ज़मीन को) ठुकरा दो और चश्मा निकाला तो हमने कहा (ऐ अय्यूब) तुम्हारे नहाने और पीने के वास्ते ये ठन्डा पानी (हाज़िर) है
\end{hindi}}
\flushright{\begin{Arabic}
\quranayah[38][43]
\end{Arabic}}
\flushleft{\begin{hindi}
और हमने उनको और उनके लड़के वाले और उनके साथ उतने ही और अपनी ख़ास मेहरबानी से अता किए
\end{hindi}}
\flushright{\begin{Arabic}
\quranayah[38][44]
\end{Arabic}}
\flushleft{\begin{hindi}
और अक्लमंदों के लिए इबरत व नसीहत (क़रार दी) और हमने कहा ऐ अय्यूब तुम अपने हाथ से सींको का मट्ठा लो (और उससे अपनी बीवी को) मारो अपनी क़सम में झूठे न बनो हमने कहा अय्यूब को यक़ीनन साबिर पाया वह क्या अच्छे बन्दे थे
\end{hindi}}
\flushright{\begin{Arabic}
\quranayah[38][45]
\end{Arabic}}
\flushleft{\begin{hindi}
बेशक वह हमारी बारगाह में बड़े झुकने वाले थे और (ऐ रसूल) हमारे बन्दों में इब्राहीम और इसहाक़ और बेशक वह (हमारी बारगाह में) बड़े झुकने वाले थे और (ऐ रसूल) हमारे बन्दों में इबराहीम और इसहाक़ और याकूब को याद करो जो कुवत और बसीरत वाले थे
\end{hindi}}
\flushright{\begin{Arabic}
\quranayah[38][46]
\end{Arabic}}
\flushleft{\begin{hindi}
हमने उन लोगों को एक ख़ास सिफत आख़ेरत की याद से मुमताज़ किया था
\end{hindi}}
\flushright{\begin{Arabic}
\quranayah[38][47]
\end{Arabic}}
\flushleft{\begin{hindi}
और इसमें शक नहीं कि ये लोग हमारी बारगाह में बरगुज़ीदा और नेक लोगों में हैं
\end{hindi}}
\flushright{\begin{Arabic}
\quranayah[38][48]
\end{Arabic}}
\flushleft{\begin{hindi}
और (ऐ रसूल) इस्माईल और अलयसा और जुलकिफ़ल को (भी) याद करो और (ये) सब नेक बन्दों में हैं
\end{hindi}}
\flushright{\begin{Arabic}
\quranayah[38][49]
\end{Arabic}}
\flushleft{\begin{hindi}
ये एक नसीहत है और इसमें शक नहीं कि परहेज़गारों के लिए (आख़ेरत में) यक़ीनी अच्छी आरामगाह है
\end{hindi}}
\flushright{\begin{Arabic}
\quranayah[38][50]
\end{Arabic}}
\flushleft{\begin{hindi}
(यानि) हमेशा रहने के (बेहिश्त के) सदाबहार बाग़ात जिनके दरवाज़े उनके लिए (बराबर) खुले होगें
\end{hindi}}
\flushright{\begin{Arabic}
\quranayah[38][51]
\end{Arabic}}
\flushleft{\begin{hindi}
और ये लोग वहाँ तकिये लगाए हुए (चैन से बैठे) होगें वहाँ (खुद्दामे बेहिश्त से) कसरत से मेवे और शराब मँगवाएँगे
\end{hindi}}
\flushright{\begin{Arabic}
\quranayah[38][52]
\end{Arabic}}
\flushleft{\begin{hindi}
और उनके पहलू में नीची नज़रों वाली (शरमीली) कमसिन बीवियाँ होगी
\end{hindi}}
\flushright{\begin{Arabic}
\quranayah[38][53]
\end{Arabic}}
\flushleft{\begin{hindi}
(मोमिनों) ये वह चीज़ हैं जिनका हिसाब के दिन (क़यामत) के लिए तुमसे वायदा किया जाता है
\end{hindi}}
\flushright{\begin{Arabic}
\quranayah[38][54]
\end{Arabic}}
\flushleft{\begin{hindi}
बेशक ये हमारी (दी हुई) रोज़ी है जो कभी तमाम न होगी
\end{hindi}}
\flushright{\begin{Arabic}
\quranayah[38][55]
\end{Arabic}}
\flushleft{\begin{hindi}
ये परहेज़गारों का (अन्जाम) है और सरकशों का तो यक़ीनी बुरा ठिकाना है
\end{hindi}}
\flushright{\begin{Arabic}
\quranayah[38][56]
\end{Arabic}}
\flushleft{\begin{hindi}
जहन्नुम जिसमें उनको जाना पड़ेगा तो वह क्या बुरा ठिकाना है
\end{hindi}}
\flushright{\begin{Arabic}
\quranayah[38][57]
\end{Arabic}}
\flushleft{\begin{hindi}
ये खौलता हुआ पानी और पीप और इस तरह अनवा अक़साम की दूसरी चीज़े हैं
\end{hindi}}
\flushright{\begin{Arabic}
\quranayah[38][58]
\end{Arabic}}
\flushleft{\begin{hindi}
तो ये लोग उन्हीं पड़े चखा करें (कुछ लोगों के बारे में) बड़ों से कहा जाएगा
\end{hindi}}
\flushright{\begin{Arabic}
\quranayah[38][59]
\end{Arabic}}
\flushleft{\begin{hindi}
ये (तुम्हारी चेलों की) फौज भी तुम्हारे साथ ही ढूँसी जाएगी उनका भला न हो ये सब भी दोज़ख़ को जाने वाले हैं
\end{hindi}}
\flushright{\begin{Arabic}
\quranayah[38][60]
\end{Arabic}}
\flushleft{\begin{hindi}
तो चेले कहेंगें (हम क्यों) बल्कि तुम (जहन्नुमी हो) तुम्हारा ही भला न हो तो तुम ही लोगों ने तो इस (बला) से हमारा सामना करा दिया तो जहन्नुम भी क्या बुरी जगह है
\end{hindi}}
\flushright{\begin{Arabic}
\quranayah[38][61]
\end{Arabic}}
\flushleft{\begin{hindi}
(फिर वह) अर्ज़ करेगें परवरदिगार जिस शख्स ने हमारा इस (बला) से सामना करा दिया तो तू उस पर हमसे बढ़कर जहन्नुम में दो गुना अज़ाब कर
\end{hindi}}
\flushright{\begin{Arabic}
\quranayah[38][62]
\end{Arabic}}
\flushleft{\begin{hindi}
और (फिर) खुद भी कहेगें हमें क्या हो गया है कि हम जिन लोगों को (दुनिया में) शरीर शुमार करते थे हम उनको यहाँ (दोज़ख़) में नहीं देखते
\end{hindi}}
\flushright{\begin{Arabic}
\quranayah[38][63]
\end{Arabic}}
\flushleft{\begin{hindi}
क्या हम उनसे (नाहक़) मसखरापन करते थे या उनकी तरफ से (हमारी) ऑंखे पलट गयी हैं
\end{hindi}}
\flushright{\begin{Arabic}
\quranayah[38][64]
\end{Arabic}}
\flushleft{\begin{hindi}
इसमें शक नहीं कि जहन्नुमियों का बाहम झगड़ना ये बिल्कुल यक़ीनी ठीक है
\end{hindi}}
\flushright{\begin{Arabic}
\quranayah[38][65]
\end{Arabic}}
\flushleft{\begin{hindi}
(ऐ रसूल) तुम कह दो कि मैं तो बस (अज़ाबे खुदा से) डराने वाला हूँ और यकता क़हार खुदा के सिवा कोई माबूद क़ाबिले परसतिश नहीं
\end{hindi}}
\flushright{\begin{Arabic}
\quranayah[38][66]
\end{Arabic}}
\flushleft{\begin{hindi}
सारे आसमान और ज़मीन का और जो चीज़े उन दोनों के दरमियान हैं (सबका) परवरदिगार ग़ालिब बड़ा बख्शने वाला है
\end{hindi}}
\flushright{\begin{Arabic}
\quranayah[38][67]
\end{Arabic}}
\flushleft{\begin{hindi}
(ऐ रसूल) तुम कह दो कि ये (क़यामत) एक बहुत बड़ा वाक़िया है
\end{hindi}}
\flushright{\begin{Arabic}
\quranayah[38][68]
\end{Arabic}}
\flushleft{\begin{hindi}
जिससे तुम लोग (ख्वाहमाख्वाह) मुँह फेरते हो
\end{hindi}}
\flushright{\begin{Arabic}
\quranayah[38][69]
\end{Arabic}}
\flushleft{\begin{hindi}
आलम बाला के रहने वाले (फरिश्ते) जब वाहम बहस करते थे उसकी मुझे भी ख़बर न थी
\end{hindi}}
\flushright{\begin{Arabic}
\quranayah[38][70]
\end{Arabic}}
\flushleft{\begin{hindi}
मेरे पास तो बस वही की गयी है कि मैं (खुदा के अज़ाब से) साफ-साफ डराने वाला हूँ
\end{hindi}}
\flushright{\begin{Arabic}
\quranayah[38][71]
\end{Arabic}}
\flushleft{\begin{hindi}
(वह बहस ये थी कि) जब तुम्हारे परवरदिगार ने फरिश्तों से कहा कि मैं गीली मिट्टी से एक आदमी बनाने वाला हूँ
\end{hindi}}
\flushright{\begin{Arabic}
\quranayah[38][72]
\end{Arabic}}
\flushleft{\begin{hindi}
तो जब मैं उसको दुरूस्त कर लूँ और इसमें अपनी (पैदा) की हुई रूह फूँक दो तो तुम सब के सब उसके सामने सजदे में गिर पड़ना
\end{hindi}}
\flushright{\begin{Arabic}
\quranayah[38][73]
\end{Arabic}}
\flushleft{\begin{hindi}
तो सब के सब कुल फरिश्तों ने सजदा किया
\end{hindi}}
\flushright{\begin{Arabic}
\quranayah[38][74]
\end{Arabic}}
\flushleft{\begin{hindi}
मगर (एक) इबलीस ने कि वह शेख़ी में आ गया और काफिरों में हो गया
\end{hindi}}
\flushright{\begin{Arabic}
\quranayah[38][75]
\end{Arabic}}
\flushleft{\begin{hindi}
ख़ुदा ने (इबलीस से) फरमाया कि ऐ इबलीस जिस चीज़ को मैंने अपनी ख़ास कुदरत से पैदा किया (भला) उसको सजदा करने से तुझे किसी ने रोका क्या तूने तक़ब्बुर किया या वाकई तू बड़े दरजे वालें में है
\end{hindi}}
\flushright{\begin{Arabic}
\quranayah[38][76]
\end{Arabic}}
\flushleft{\begin{hindi}
इबलीस बोल उठा कि मैं उससे बेहतर हूँ तूने मुझे आग से पैदा किया और इसको तूने गीली मिट्टी से पैदा किया
\end{hindi}}
\flushright{\begin{Arabic}
\quranayah[38][77]
\end{Arabic}}
\flushleft{\begin{hindi}
(कहाँ आग कहाँ मिट्टी) खुदा ने फरमाया कि तू यहाँ से निकल (दूर हो) तू यक़ीनी मरदूद है
\end{hindi}}
\flushright{\begin{Arabic}
\quranayah[38][78]
\end{Arabic}}
\flushleft{\begin{hindi}
और तुझ पर रोज़ जज़ा (क़यामत) तक मेरी फिटकार पड़ा करेगी
\end{hindi}}
\flushright{\begin{Arabic}
\quranayah[38][79]
\end{Arabic}}
\flushleft{\begin{hindi}
शैतान ने अर्ज़ की परवरदिगार तू मुझे उस दिन तक की मोहलत अता कर जिसमें सब लोग (दोबारा) उठा खड़े किए जायेंगे
\end{hindi}}
\flushright{\begin{Arabic}
\quranayah[38][80]
\end{Arabic}}
\flushleft{\begin{hindi}
फरमाया तुझे एक वक्त मुअय्यन के दिन तक की मोहलत दी गयी
\end{hindi}}
\flushright{\begin{Arabic}
\quranayah[38][81]
\end{Arabic}}
\flushleft{\begin{hindi}
वह बोला तेरी ही इज्ज़त व जलाल की क़सम
\end{hindi}}
\flushright{\begin{Arabic}
\quranayah[38][82]
\end{Arabic}}
\flushleft{\begin{hindi}
उनमें से तेरे ख़ालिस बन्दों के सिवा सब के सब को ज़रूर गुमराह करूँगा
\end{hindi}}
\flushright{\begin{Arabic}
\quranayah[38][83]
\end{Arabic}}
\flushleft{\begin{hindi}
खुदा ने फरमाया तो (हम भी) हक़ बात (कहे देते हैं)
\end{hindi}}
\flushright{\begin{Arabic}
\quranayah[38][84]
\end{Arabic}}
\flushleft{\begin{hindi}
और मैं तो हक़ ही कहा करता हूँ
\end{hindi}}
\flushright{\begin{Arabic}
\quranayah[38][85]
\end{Arabic}}
\flushleft{\begin{hindi}
कि मैं तुझसे और जो लोग तेरी ताबेदारी करेंगे उन सब से जहन्नुम को ज़रूर भरूँगा
\end{hindi}}
\flushright{\begin{Arabic}
\quranayah[38][86]
\end{Arabic}}
\flushleft{\begin{hindi}
(ऐ रसूल) तुम कह दो कि मैं तो तुमसे न इस (तबलीग़े रिसालत) की मज़दूरी माँगता हूँ और न मैं (झूठ मूठ) बनावट करने वाला हूँ
\end{hindi}}
\flushright{\begin{Arabic}
\quranayah[38][87]
\end{Arabic}}
\flushleft{\begin{hindi}
ये (क़ुरान) तो बस सारे जहाँन के लिए नसीहत है
\end{hindi}}
\flushright{\begin{Arabic}
\quranayah[38][88]
\end{Arabic}}
\flushleft{\begin{hindi}
और कुछ दिनों बाद तुमको इसकी हक़ीकत मालूम हो जाएगी
\end{hindi}}
\chapter{Az-Zumar (The Companies)}
\begin{Arabic}
\Huge{\centerline{\basmalah}}\end{Arabic}
\flushright{\begin{Arabic}
\quranayah[39][1]
\end{Arabic}}
\flushleft{\begin{hindi}
(इस) किताब (क़ुरान) का नाज़िल करना उस खुदा की बारगाह से है जो (सब पर) ग़ालिब हिकमत वाला है
\end{hindi}}
\flushright{\begin{Arabic}
\quranayah[39][2]
\end{Arabic}}
\flushleft{\begin{hindi}
(ऐ रसूल) हमने किताब (कुरान) को बिल्कुल ठीक नाज़िल किया है तो तुम इबादत को उसी के लिए निरा खुरा करके खुदा की बन्दगी किया करो
\end{hindi}}
\flushright{\begin{Arabic}
\quranayah[39][3]
\end{Arabic}}
\flushleft{\begin{hindi}
आगाह रहो कि इबादत तो ख़ास खुदा ही के लिए है और जिन लोगों ने खुदा के सिवा (औरों को अपना) सरपरस्त बना रखा है और कहते हैं कि हम तो उनकी परसतिश सिर्फ़ इसलिए करते हैं कि ये लोग खुदा की बारगाह में हमारा तक़र्रब बढ़ा देगें इसमें शक नहीं कि जिस बात में ये लोग झगड़ते हैं (क़यामत के दिन) खुदा उनके दरमियान इसमें फैसला कर देगा बेशक खुदा झूठे नाशुक्रे को मंज़िले मक़सूद तक नहीं पहुँचाया करता
\end{hindi}}
\flushright{\begin{Arabic}
\quranayah[39][4]
\end{Arabic}}
\flushleft{\begin{hindi}
अगर खुदा किसी को (अपना) बेटा बनाना चाहता तो अपने मख़लूक़ात में से जिसे चाहता मुन्तखिब कर लेता (मगर) वह तो उससे पाक व पाकीज़ा है वह तो यकता बड़ा ज़बरदस्त अल्लाह है
\end{hindi}}
\flushright{\begin{Arabic}
\quranayah[39][5]
\end{Arabic}}
\flushleft{\begin{hindi}
उसी ने सारे आसमान और ज़मीन को बजा (दुरूस्त) पैदा किया वही रात को दिन पर ऊपर तले लपेटता है और वही दिन को रात पर तह ब तह लपेटता है और उसी ने आफताब और महताब को अपने बस में कर लिया है कि ये सबके सब अपने (अपने) मुक़रर्र वक्त चलते रहेगें आगाह रहो कि वही ग़ालिब बड़ा बख्शने वाला है
\end{hindi}}
\flushright{\begin{Arabic}
\quranayah[39][6]
\end{Arabic}}
\flushleft{\begin{hindi}
उसी ने तुम सबको एक ही शख्स से पैदा किया फिर उस (की बाक़ी मिट्टी) से उसकी बीबी (हौव्वा) को पैदा किया और उसी ने तुम्हारे लिए आठ क़िस्म के चारपाए पैदा किए वही तुमको तुम्हारी माँओं के पेट में एक क़िस्म की पैदाइश के बाद दूसरी क़िस्म (नुत्फे जमा हुआ खून लोथड़ा) की पैदाइश से तेहरे तेहरे अंधेरों (पेट) रहम और झिल्ली में पैदा करता है वही अल्लाह तुम्हारा परवरदिगार है उसी की बादशाही है उसके सिवा माबूद नहीं तो तुम लोग कहाँ फिरे जाते हो
\end{hindi}}
\flushright{\begin{Arabic}
\quranayah[39][7]
\end{Arabic}}
\flushleft{\begin{hindi}
अगर तुमने उसकी नाशुक्री की तो (याद रखो कि) खुदा तुमसे बिल्कुल बेपरवाह है और अपने बन्दों से कुफ्र और नाशुक्री को पसन्द नहीं करता और अगर तुम शुक्र करोगे तो वह उसको तुम्हारे वास्ते पसन्द करता है और (क़यामत में) कोई किसी (के गुनाह) का बोझ (अपनी गर्दन पर) नहीं उठाएगा फिर तुमको अपने परवरदिगार की तरफ लौटना है तो (दुनिया में) जो कुछ (भला बुरा) करते थे वह तुम्हें बता देगा वह यक़ीनन दिलों के राज़ (तक) से खूब वाक़िफ है
\end{hindi}}
\flushright{\begin{Arabic}
\quranayah[39][8]
\end{Arabic}}
\flushleft{\begin{hindi}
और आदमी (की हालत तो ये है कि) जब उसको कोई तकलीफ पहुँचती है तो उसी की तरफ रूजू करके अपने परवरदिगार से दुआ करता है (मगर) फिर जब खुदा अपनी तरफ से उसे नेअमत अता फ़रमा देता है तो जिस काम के लिए पहले उससे दुआ करता था उसे भुला देता है और बल्कि खुदा का शरीक बनाने लगता है ताकि (उस ज़रिए से और लोगों को भी) उसकी राह से गुमराह कर दे (ऐ रसूल ऐसे शख्स से) कह दो कि थोड़े दिनों और अपने कुफ्र (की हालत) में चैन कर लो
\end{hindi}}
\flushright{\begin{Arabic}
\quranayah[39][9]
\end{Arabic}}
\flushleft{\begin{hindi}
(आख़िर) तू यक़ीनी जहन्नुमियों में होगा क्या जो शख्स रात के अवक़ात में सजदा करके और खड़े-खड़े (खुदा की) इबादत करता हो और आख़ेरत से डरता हो अपने परवरदिगार की रहमत का उम्मीदवार हो (नाशुक्रे) काफिर के बराबर हो सकता है (ऐ रसूल) तुम पूछो तो कि भला कहीं जानने वाले और न जाननेवाले लोग बराबर हो सकते हैं (मगर) नसीहत इबरतें तो बस अक्लमन्द ही लोग मानते हैं
\end{hindi}}
\flushright{\begin{Arabic}
\quranayah[39][10]
\end{Arabic}}
\flushleft{\begin{hindi}
(ऐ रसूल) तुम कह दो कि ऐ मेरे ईमानदार बन्दों अपने परवरदिगार (ही) से डरते रहो (क्योंकि) जिन लोगों ने इस दुनिया में नेकी की उन्हीं के लिए (आख़ेरत में) भलाई है और खुदा की ज़मीन तो कुशादा है (जहाँ इबादत न कर सको उसे छोड़ दो) सब्र करने वालों ही की तो उनका भरपूर बेहिसाब बदला दिया जाएगा
\end{hindi}}
\flushright{\begin{Arabic}
\quranayah[39][11]
\end{Arabic}}
\flushleft{\begin{hindi}
(ऐ रसूल) तुम कह दो कि मुझे तो ये हुक्म दिया गया है कि मैं इबादत को उसके लिए ख़ास करके खुदा ही की बन्दगी करो
\end{hindi}}
\flushright{\begin{Arabic}
\quranayah[39][12]
\end{Arabic}}
\flushleft{\begin{hindi}
और मुझे तो ये हुक्म दिया गया है कि मैं सबसे पहल मुसलमान हूँ
\end{hindi}}
\flushright{\begin{Arabic}
\quranayah[39][13]
\end{Arabic}}
\flushleft{\begin{hindi}
(ऐ रसूल) तुम कह दो कि अगर मैं अपने परवरदिगार की नाफरमानी करूँ तो मैं एक बड़ी (सख्त) दिन (क़यामत) के अज़ाब से डरता हूँ
\end{hindi}}
\flushright{\begin{Arabic}
\quranayah[39][14]
\end{Arabic}}
\flushleft{\begin{hindi}
(ऐ रसूल) तुम कह दो कि मैं अपनी इबादत को उसी के वास्ते ख़ालिस करके खुदा ही की बन्दगी करता हूँ (अब रहे तुम) तो उसके सिवा जिसको चाहो पूजो
\end{hindi}}
\flushright{\begin{Arabic}
\quranayah[39][15]
\end{Arabic}}
\flushleft{\begin{hindi}
(ऐ रसूल) तुम कह दो कि फिल हक़ीक़त घाटे में वही लोग हैं जिन्होंने अपना और अपने लड़के वालों का क़यामत के दिन घाटा किया आगाह रहो कि सरीही (खुल्लम खुल्ला) घाटा यही है कि उनके लिए उनके ऊपर से आग ही के ओढ़ने होगें
\end{hindi}}
\flushright{\begin{Arabic}
\quranayah[39][16]
\end{Arabic}}
\flushleft{\begin{hindi}
और उनके नीचे भी (आग ही के) बिछौने ये वह अज़ाब है जिससे खुदा अपने बन्दों को डराता है तो ऐ मेरे बन्दों मुझी से डरते रहो
\end{hindi}}
\flushright{\begin{Arabic}
\quranayah[39][17]
\end{Arabic}}
\flushleft{\begin{hindi}
और जो लोग बुतों से उनके पूजने से बचे रहे और ख़ुदा ही की तरफ रूजु की उनके लिए (जन्नत की) ख़ुशख़बरी है
\end{hindi}}
\flushright{\begin{Arabic}
\quranayah[39][18]
\end{Arabic}}
\flushleft{\begin{hindi}
तो (ऐ रसूल) तुम मेरे (ख़ास) बन्दों को खुशख़बरी दे दो जो बात को जी लगाकर सुनते हैं और फिर उसमें से अच्छी बात पर अमल करते हैं यही वह लोग हैं जिनकी खुदा ने हिदायत की और यही लोग अक्लमन्द हैं
\end{hindi}}
\flushright{\begin{Arabic}
\quranayah[39][19]
\end{Arabic}}
\flushleft{\begin{hindi}
तो (ऐ रसूल) भला जिस शख्स पर अज़ाब का वायदा पूरा हो चुका हो तो क्या तुम उस शख्स की ख़लासी दे सकते हो
\end{hindi}}
\flushright{\begin{Arabic}
\quranayah[39][20]
\end{Arabic}}
\flushleft{\begin{hindi}
जो आग में (पड़ा) हो मगर जो लोग अपने परवरदिगार से डरते रहे उनके ऊँचे-ऊँचे महल हैं (और) बाला ख़ानों पर बालाख़ाने बने हुए हैं जिनके नीचे नहरें जारी हैं ये खुदा का वायदा है (और) वायदा ख़िलाफी नहीं किया करता
\end{hindi}}
\flushright{\begin{Arabic}
\quranayah[39][21]
\end{Arabic}}
\flushleft{\begin{hindi}
क्या तुमने इस पर ग़ौर नहीं किया कि खुदा ही ने आसमान से पानी बरसाया फिर उसको ज़मीन में चश्में बनाकर जारी किया फिर उसके ज़रिए से रंग बिरंग (के गल्ले) की खेती उगाता है फिर (पकने के बाद) सूख जाती है तो तुम को वह ज़र्द दिखायी देती है फिर खुदा उसे चूर-चूर भूसा कर देता है बेशक इसमें अक्लमन्दों के लिए (बड़ी) इबरत व नसीहत है
\end{hindi}}
\flushright{\begin{Arabic}
\quranayah[39][22]
\end{Arabic}}
\flushleft{\begin{hindi}
तो क्या वह शख्स जिस के सीने को खुदा ने (क़ुबूल) इस्लाम के लिए कुशादा कर दिया है तो वह अपने परवरदिगार (की हिदायत) की रौशनी पर (चलता) है मगर गुमराहों के बराबर हो सकता है अफसोस तो उन लोगों पर है जिनके दिल खुदा की याद से (ग़ाफ़िल होकर) सख्त हो गए हैं
\end{hindi}}
\flushright{\begin{Arabic}
\quranayah[39][23]
\end{Arabic}}
\flushleft{\begin{hindi}
ये लोग सरीही गुमराही में (पड़े) हैं ख़ुदा ने बहुत ही अच्छा कलाम (यावी ये) किताब नाज़िल फरमाई (जिसकी आयतें) एक दूसरे से मिलती जुलती हैं और (एक बात कई-कई बार) दोहराई गयी है उसके सुनने से उन लोगों के रोंगटे खड़े हो जाते हैं जो अपने परवरदिगार से डरते हैं फिर उनके जिस्म नरम हो जाते हैं और उनके दिल खुदा की याद की तरफ बा इतमेनान मुतावज्जे हो जाते हैं ये खुदा की हिदायत है इसी से जिसकी चाहता है हिदायत करता है और खुदा जिसको गुमराही में छोड़ दे तो उसको कोई राह पर लाने वाला नहीं
\end{hindi}}
\flushright{\begin{Arabic}
\quranayah[39][24]
\end{Arabic}}
\flushleft{\begin{hindi}
तो क्या जो शख्स क़यामत के दिन अपने मुँह को बड़े अज़ाब की सिपर बनाएगा (नाज़ी के बराबर हो सकता है) और ज़ालिमों से कहा जाएगा कि तुम (दुनिया में) जैसा कुछ करते थे अब उसके मज़े चखो
\end{hindi}}
\flushright{\begin{Arabic}
\quranayah[39][25]
\end{Arabic}}
\flushleft{\begin{hindi}
जो लोग उनसे पहले गुज़र गए उन्होंने भी (पैग़म्बरों को) झुठलाया तो उन पर अज़ाब इस तरह आ पहुँचा कि उन्हें ख़बर भी न हुई
\end{hindi}}
\flushright{\begin{Arabic}
\quranayah[39][26]
\end{Arabic}}
\flushleft{\begin{hindi}
तो खुदा ने उन्हें (इसी) दुनिया की ज़िन्दगी में रूसवाई की लज्ज़त चखा दी और आख़ेरत का अज़ाब तो यक़ीनी उससे कहीं बढ़कर है काश ये लोग ये बात जानते
\end{hindi}}
\flushright{\begin{Arabic}
\quranayah[39][27]
\end{Arabic}}
\flushleft{\begin{hindi}
और हमने तो इस क़ुरान में लोगों के (समझाने के) वास्ते हर तरह की मिसाल बयान कर दी है ताकि ये लोग नसीहत हासिल करें
\end{hindi}}
\flushright{\begin{Arabic}
\quranayah[39][28]
\end{Arabic}}
\flushleft{\begin{hindi}
(हम ने तो साफ और सलीस) एक अरबी कुरान (नाज़िल किया) जिसमें ज़रा भी कजी (पेचीदगी) नहीं
\end{hindi}}
\flushright{\begin{Arabic}
\quranayah[39][29]
\end{Arabic}}
\flushleft{\begin{hindi}
ताकि ये लोग (समझकर) खुदा से डरे ख़ुदा ने एक मिसाल बयान की है कि एक शख्स (ग़ुलाम) है जिसमें कई झगड़ालू साझी हैं और एक ज़ालिम है कि पूरा एक शख्स का है उन दोनों की हालत यकसाँ हो सकती हैं (हरगिज़ नहीं) अल्हमदोलिल्लाह मगर उनमें अक्सर इतना भी नहीं जानते
\end{hindi}}
\flushright{\begin{Arabic}
\quranayah[39][30]
\end{Arabic}}
\flushleft{\begin{hindi}
(ऐ रसूल) बेशक तुम भी मरने वाले हो
\end{hindi}}
\flushright{\begin{Arabic}
\quranayah[39][31]
\end{Arabic}}
\flushleft{\begin{hindi}
और ये लोग भी यक़ीनन मरने वाले हैं फिर तुम लोग क़यामत के दिन अपने परवरदिगार की बारगाह में बाहम झगड़ोगे
\end{hindi}}
\flushright{\begin{Arabic}
\quranayah[39][32]
\end{Arabic}}
\flushleft{\begin{hindi}
तो इससे बढ़कर ज़ालिम कौन होगा जो ख़ुदा पर झूठ (तूफान) बाँधे और जब उसके पास सच्ची बात आए तो उसको झुठला दे क्या जहन्नुम में कााफिरों का ठिकाना नहीं है
\end{hindi}}
\flushright{\begin{Arabic}
\quranayah[39][33]
\end{Arabic}}
\flushleft{\begin{hindi}
(ज़रूर है) और याद रखो कि जो शख्स (रसूल) सच्ची बात लेकर आया वह और जिसने उसकी तसदीक़ की यही लोग तो परहेज़गार हैं
\end{hindi}}
\flushright{\begin{Arabic}
\quranayah[39][34]
\end{Arabic}}
\flushleft{\begin{hindi}
ये लोग जो चाहेंगे उनके लिए परवर दिगार के पास (मौजूद) है, ये नेकी करने वालों की जज़ाए ख़ैर है
\end{hindi}}
\flushright{\begin{Arabic}
\quranayah[39][35]
\end{Arabic}}
\flushleft{\begin{hindi}
ताकि ख़ुदा उन लोगों की बुराइयों को जो उन्होने की हैं दूर कर दे और उनके अच्छे कामों के एवज़ जो वह कर चुके थे उसका अज्र (सवाब) अता फरमाए
\end{hindi}}
\flushright{\begin{Arabic}
\quranayah[39][36]
\end{Arabic}}
\flushleft{\begin{hindi}
क्या ख़ुदा अपने बन्दों (की मदद) के लिए काफ़ी नहीं है (ज़रूर है) और (ऐ रसूल) तुमको लोग ख़ुदा के सिवा (दूसरे माबूदों) से डराते हैं और ख़ुदा जिसे गुमराही में छोड़ दे तो उसका कोई राह पर लाने वाला नहीं है
\end{hindi}}
\flushright{\begin{Arabic}
\quranayah[39][37]
\end{Arabic}}
\flushleft{\begin{hindi}
और जिस शख्स की हिदायत करे तो उसका कोई गुमराह करने वाला नहीं। क्या ख़दा ज़बरदस्त और बदला लेने वाला नहीं है (ज़रूर है)
\end{hindi}}
\flushright{\begin{Arabic}
\quranayah[39][38]
\end{Arabic}}
\flushleft{\begin{hindi}
और (ऐ रसूल) अगर तुम इनसे पूछो कि सारे आसमान व ज़मीन को किसने पैदा किया तो ये लोग यक़ीनन कहेंगे कि ख़ुदा ने, तुम कह दो कि तो क्या तुमने ग़ौर किया है कि ख़ुदा को छोड़ कर जिन लोगों की तुम इबादत करते हो अगर ख़ुदा मुझे कोई तक़लीफ पहुँचाना चाहे तो क्या वह लोग उसके नुक़सान को (मुझसे) रोक सकते हैं या अगर ख़ुदा मुझ पर मेहरबानी करना चाहे तो क्या वह लोग उसकी मेहरबानी रोक सकते हैं (ऐ रसूल) तुम कहो कि ख़ुदा मेरे लिए काफ़ी है उसी पर भरोसा करने वाले भरोसा करते हैं
\end{hindi}}
\flushright{\begin{Arabic}
\quranayah[39][39]
\end{Arabic}}
\flushleft{\begin{hindi}
(ऐ रसूल) तुम कह दो कि ऐ मेरी क़ौम तुम अपनी जगह (जो चाहो) अमल किए जाओ मै
\end{hindi}}
\flushright{\begin{Arabic}
\quranayah[39][40]
\end{Arabic}}
\flushleft{\begin{hindi}
भी (अपनी जगह) कुछ कर रहा हूँ, फिर अनक़रीब ही तुम्हें मालूम हो जाएगा कि किस पर वह आफत आती है जो उसको (दुनिया में) रूसवा कर देगी और (आख़िर में) उस पर दायमी अज़ाब भी नाज़िल होगा
\end{hindi}}
\flushright{\begin{Arabic}
\quranayah[39][41]
\end{Arabic}}
\flushleft{\begin{hindi}
(ऐ रसूल) हमने तुम्हारे पास (ये) किताब (क़ुरान) सच्चाई के साथ लोगों (की हिदायत) के वास्ते नाज़िल की है, पस जो राह पर आया तो अपने ही (भले के) लिए और जो गुमराह हुआ तो उसकी गुमराही का वबाल भी उसी पर है और फिर तुम कुछ उनके ज़िम्मेदार तो हो नहीं
\end{hindi}}
\flushright{\begin{Arabic}
\quranayah[39][42]
\end{Arabic}}
\flushleft{\begin{hindi}
ख़ुदा ही लोगों के मरने के वक्त उनकी रूहें (अपनी तरफ़) खींच बुलाता है और जो लोग नहीं मरे (उनकी रूहें) उनकी नींद में (खींच ली जाती हैं) बस जिन के बारे में ख़ुदा मौत का हुक्म दे चुका है उनकी रूहों को रोक रखता है और बाक़ी (सोने वालों की रूहों) को फिर एक मुक़र्रर वक्त तक के वास्ते भेज देता है जो लोग (ग़ौर) और फिक्र करते हैं उनके लिए (क़ुदरते ख़ुदा की) यक़ीनी बहुत सी निशानियाँ हैं
\end{hindi}}
\flushright{\begin{Arabic}
\quranayah[39][43]
\end{Arabic}}
\flushleft{\begin{hindi}
क्या उन लोगों ने ख़ुदा के सिवा (दूसरे) सिफारिशी बना रखे है (ऐ रसूल) तुम कह दो कि अगरचे वह लोग न कुछ एख़तेयार रखते हों न कुछ समझते हों
\end{hindi}}
\flushright{\begin{Arabic}
\quranayah[39][44]
\end{Arabic}}
\flushleft{\begin{hindi}
(तो भी सिफारिशी बनाओगे) तुम कह दो कि सारी सिफारिश तो ख़ुदा के लिए ख़ास है- सारे आसमान व ज़मीन की हुकूमत उसी के लिए ख़ास है, फिर तुम लोगों को उसकी तरफ लौट कर जाना है
\end{hindi}}
\flushright{\begin{Arabic}
\quranayah[39][45]
\end{Arabic}}
\flushleft{\begin{hindi}
और जब सिर्फ अल्लाह का ज़िक्र किया जाता है तो जो लोग आख़ेरत पर ईमान नहीं रखते उनके दिल मुतनफ़िफ़र हो जाते हैं और जब ख़ुदा के सिवा और (माबूदों) का ज़िक्र किया जाता है तो बस फौरन उनकी बाछें खिल जाती हैं
\end{hindi}}
\flushright{\begin{Arabic}
\quranayah[39][46]
\end{Arabic}}
\flushleft{\begin{hindi}
(ऐ रसूल) तुम कह दो कि ऐ ख़ुदा (ऐ) सारे आसमान और ज़मीन पैदा करने वाले, ज़ाहिर व बातिन के जानने वाले हक़ बातों में तेरे बन्दे आपस में झगड़ रहे हैं तू ही उनके दरमियान फैसला कर देगा
\end{hindi}}
\flushright{\begin{Arabic}
\quranayah[39][47]
\end{Arabic}}
\flushleft{\begin{hindi}
और अगर नाफरमानों के पास रूए ज़मीन की सारी काएनात मिल जाएग बल्कि उनके साथ उतनी ही और भी हो तो क़यामत के दिन ये लोग यक़ीनन सख्त अज़ाब का फ़िदया दे निकलें (और अपना छुटकारा कराना चाहें) और (उस वक्त) उनके सामने ख़ुदा की तरफ से वह बात पेश आएगी जिसका उन्हें वहम व गुमान भी न था
\end{hindi}}
\flushright{\begin{Arabic}
\quranayah[39][48]
\end{Arabic}}
\flushleft{\begin{hindi}
और जो बदकिरदारियाँ उन लोगों ने की थीं (वह सब) उनके सामने खुल जाएँगीं और जिस (अज़ाब) पर यह लोग क़हक़हे लगाते थे वह उन्हें घेरेगा
\end{hindi}}
\flushright{\begin{Arabic}
\quranayah[39][49]
\end{Arabic}}
\flushleft{\begin{hindi}
इन्सान को तो जब कोई बुराई छू गयी बस वह लगा हमसे दुआएँ माँगने, फिर जब हम उसे अपनी तरफ़ से कोई नेअमत अता करते हैं तो कहने लगता है कि ये तो सिर्फ (मेरे) इल्म के ज़ोर से मुझे दिया गया है (ये ग़लती है) बल्कि ये तो एक आज़माइश है मगर उन में के अक्सर नहीं जानते हैं
\end{hindi}}
\flushright{\begin{Arabic}
\quranayah[39][50]
\end{Arabic}}
\flushleft{\begin{hindi}
जो लोग उनसे पहले थे वह भी ऐसी बातें बका करते थे फिर (जब हमारा अज़ाब आया) तो उनकी कारस्तानियाँ उनके कुछ भी काम न आई
\end{hindi}}
\flushright{\begin{Arabic}
\quranayah[39][51]
\end{Arabic}}
\flushleft{\begin{hindi}
ग़रज़ उनके आमाल के बुरे नतीजे उन्हें भुगतने पड़े और उन (कुफ्फ़ारे मक्का) में से जिन लोगों ने नाफरमानियाँ की हैं उन्हें भी अपने अपने आमाल की सज़ाएँ भुगतनी पड़ेंगी और ये लोग (ख़ुदा को)े आजिज़ नहीं कर सकते
\end{hindi}}
\flushright{\begin{Arabic}
\quranayah[39][52]
\end{Arabic}}
\flushleft{\begin{hindi}
क्या उन लोगों को इतनी बात भी मालूम नहीं कि ख़ुदा ही जिसके लिए चाहता है रोज़ी फराख़ करता है और (जिसके लिए चाहता है) तंग करता है इसमें शक नहीं कि क्या इसमें ईमानदार लोगों के (कुदरत की) बहुत सी निशानियाँ हैं
\end{hindi}}
\flushright{\begin{Arabic}
\quranayah[39][53]
\end{Arabic}}
\flushleft{\begin{hindi}
(ऐ रसूल) तुम कह दो कि ऐ मेरे (ईमानदार) बन्दों जिन्होने (गुनाह करके) अपनी जानों पर ज्यादतियाँ की हैं तुम लोग ख़ुदा की रहमत से नाउम्मीद न होना बेशक ख़ुदा (तुम्हारे) कुल गुनाहों को बख्श देगा वह बेशक बड़ा बख्शने वाला मेहरबान है
\end{hindi}}
\flushright{\begin{Arabic}
\quranayah[39][54]
\end{Arabic}}
\flushleft{\begin{hindi}
और अपने परवरदिगार की तरफ रूजू करो और उसी के फरमाबरदार बन जाओ (मगर) उस वक्त क़े क़ब्ल ही कि तुम पर जब अज़ाब आ नाज़िल हो (और) फिर तुम्हारी मदद न की जा सके
\end{hindi}}
\flushright{\begin{Arabic}
\quranayah[39][55]
\end{Arabic}}
\flushleft{\begin{hindi}
और जो जो अच्छी बातें तुम्हारे परवरदिगार की तरफ से तुम पर नाज़िल हई हैं उन पर चलो (मगर) उसके क़ब्ल कि तुम पर एक बारगी अज़ाब नाज़िल हो और तुमको उसकी ख़बर भी न हो
\end{hindi}}
\flushright{\begin{Arabic}
\quranayah[39][56]
\end{Arabic}}
\flushleft{\begin{hindi}
(कहीं ऐसा न हो कि) (तुममें से) कोई शख्स कहने लगे कि हाए अफ़सोस मेरी इस कोताही पर जो मैने ख़ुदा (की बारगाह) का तक़र्रुब हासिल करने में की और मैं तो बस उन बातों पर हँसता ही रहा
\end{hindi}}
\flushright{\begin{Arabic}
\quranayah[39][57]
\end{Arabic}}
\flushleft{\begin{hindi}
या ये कहने लगे कि अगर ख़ुदा मेरी हिदायत करता तो मैं ज़रूर परहेज़गारों में से होता
\end{hindi}}
\flushright{\begin{Arabic}
\quranayah[39][58]
\end{Arabic}}
\flushleft{\begin{hindi}
या जब अज़ाब को (आते) देखें तो कहने लगे कि काश मुझे (दुनिया में) फिर दोबारा जाना मिले तो मैं नेकी कारों में हो जाऊँ
\end{hindi}}
\flushright{\begin{Arabic}
\quranayah[39][59]
\end{Arabic}}
\flushleft{\begin{hindi}
उस वक्त ख़ुदा कहेगा ( हाँ ) हाँ तेरे पास मेरी आयतें पहुँची तो तूने उन्हें झुठलाया और शेख़ी कर बैठा और तू भी काफिरों में से था (अब तेरी एक न सुनी जाएगी)
\end{hindi}}
\flushright{\begin{Arabic}
\quranayah[39][60]
\end{Arabic}}
\flushleft{\begin{hindi}
और जिन लोगों ने ख़ुदा पर झूठे बोहतान बाँधे - तुम क़यामत के दिन देखोगे उनके चेहरे सियाह होंगे क्या गुरूर करने वालों का ठिकाना जहन्नुम में नहीं है (ज़रूर है)
\end{hindi}}
\flushright{\begin{Arabic}
\quranayah[39][61]
\end{Arabic}}
\flushleft{\begin{hindi}
और जो लोग परहेज़गार हैं ख़ुदा उन्हें उनकी कामयाबी (और सआदत) के सबब निजात देगा कि उन्हें तकलीफ छुएगी भी नहीं और न यह लोग (किसी तरह) रंजीदा दिल होंगे
\end{hindi}}
\flushright{\begin{Arabic}
\quranayah[39][62]
\end{Arabic}}
\flushleft{\begin{hindi}
ख़ुदा ही हर चीज़ का जानने वाला है और वही हर चीज़ का निगेहबान है
\end{hindi}}
\flushright{\begin{Arabic}
\quranayah[39][63]
\end{Arabic}}
\flushleft{\begin{hindi}
सारे आसमान व ज़मीन की कुन्जियाँ उसके पास है और जो लोग उसकी आयतों से इन्कार कर बैठें वही घाटे में रहेगें
\end{hindi}}
\flushright{\begin{Arabic}
\quranayah[39][64]
\end{Arabic}}
\flushleft{\begin{hindi}
(ऐ रसूल) तुम कह दो कि नादानों भला तुम मुझसे ये कहते हो कि मैं ख़ुदा के सिवा किसी दूसरे की इबादत करूँ
\end{hindi}}
\flushright{\begin{Arabic}
\quranayah[39][65]
\end{Arabic}}
\flushleft{\begin{hindi}
और (ऐ रसूल) तुम्हारी तरफ और उन (पैग़म्बरों) की तरफ जो तुमसे पहले हो चुके हैं यक़ीनन ये वही भेजी जा की है कि अगर (कहीं) शिर्क किया तो यक़ीनन तुम्हारे सारे अमल अकारत हो जाएँगे और तुम तो ज़रूर घाटे में आ जाओगे
\end{hindi}}
\flushright{\begin{Arabic}
\quranayah[39][66]
\end{Arabic}}
\flushleft{\begin{hindi}
बल्कि तुम ख़ुदा ही कि इबादत करो और शुक्र गुज़ारों में हो
\end{hindi}}
\flushright{\begin{Arabic}
\quranayah[39][67]
\end{Arabic}}
\flushleft{\begin{hindi}
और उन लोगों ने ख़ुदा की जैसी क़द्र दानी करनी चाहिए थी उसकी ( कुछ भी ) कद्र न की हालाँकि ( वह ऐसा क़ादिर है कि) क़यामत के दिन सारी ज़मीन (गोया) उसकी मुट्ठी में होगी और सारे आसमान (गोया) उसके दाहिने हाथ में लिपटे हुए हैं जिसे ये लोग उसका शरीक बनाते हैं वह उससे पाकीज़ा और बरतर है
\end{hindi}}
\flushright{\begin{Arabic}
\quranayah[39][68]
\end{Arabic}}
\flushleft{\begin{hindi}
और जब (पहली बार) सूर फँका जाएगा तो जो लोग आसमानों में हैं और जो लोग ज़मीन में हैं (मौत से) बेहोश होकर गिर पड़ेंगें) मगर (हाँ) जिस को ख़ुदा चाहे वह अलबत्ता बच जाएगा) फिर जब दोबारा सूर फूँका जाएगा तो फौरन सब के सब खड़े हो कर देखने लगेंगें
\end{hindi}}
\flushright{\begin{Arabic}
\quranayah[39][69]
\end{Arabic}}
\flushleft{\begin{hindi}
और ज़मीन अपने परवरदिगार के नूर से जगमगा उठेगी और (आमाल की) किताब (लोगों के सामने) रख दी जाएगी और पैग़म्बर और गवाह ला हाज़िर किए जाएँगे और उनमें इन्साफ के साथ फैसला कर दिया जाएगा और उन पर ( ज़र्रा बराबर ) ज़ुल्म नहीं किया जाएगा
\end{hindi}}
\flushright{\begin{Arabic}
\quranayah[39][70]
\end{Arabic}}
\flushleft{\begin{hindi}
और जिस शख्स ने जैसा किया हो उसे उसका पूरा पूरा बदला मिल जाएगा, और जो कुछ ये लोग करते हैं वह उससे ख़ूब वाक़िफ है
\end{hindi}}
\flushright{\begin{Arabic}
\quranayah[39][71]
\end{Arabic}}
\flushleft{\begin{hindi}
और जो लोग काफिर थे उनके ग़ोल के ग़ोल जहन्नुम की तरफ हॅकाए जाएँगे और यहाँ तक की जब जहन्नुम के पास पहुँचेगें तो उसके दरवाज़े खोल दिए जाएगें और उसके दरोग़ा उनसे पूछेंगे कि क्या तुम लोगों में के पैग़म्बर तुम्हारे पास नहीं आए थे जो तुमको तुम्हारे परवरदिगार की आयतें पढ़कर सुनाते और तुमको इस रोज़ (बद) के पेश आने से डराते वह लोग जवाब देगें कि हाँ (आए तो थे) मगर (हमने न माना) और अज़ाब का हुक्म काफिरों के बारे में पूरा हो कर रहेगा
\end{hindi}}
\flushright{\begin{Arabic}
\quranayah[39][72]
\end{Arabic}}
\flushleft{\begin{hindi}
(तब उनसे) कहा जाएगा कि जहन्नुम के दरवाज़ों में धँसो और हमेशा इसी में रहो ग़रज़ तकब्बुर करने वाले का (भी) क्या बुरा ठिकाना है
\end{hindi}}
\flushright{\begin{Arabic}
\quranayah[39][73]
\end{Arabic}}
\flushleft{\begin{hindi}
और जो लोग अपने परवरदिगार से डरते थे वह गिर्दो गिर्दा (गिरोह गिरोह) बेहिश्त की तरफ़ (एजाज़ व इकराम से) बुलाए जाएगें यहाँ तक कि जब उसके पास पहुँचेगें और बेहिश्त के दरवाज़े खोल दिये जाएँगें और उसके निगेहबान उन से कहेंगें सलाम अलैकुम तुम अच्छे रहे, तुम बेहिश्त में हमेशा के लिए दाख़िल हो जाओ
\end{hindi}}
\flushright{\begin{Arabic}
\quranayah[39][74]
\end{Arabic}}
\flushleft{\begin{hindi}
और ये लोग कहेंगें ख़ुदा का शुक्र जिसने अपना वायदा हमसे सच्चा कर दिखाया और हमें (बेहिश्त की) सरज़मीन का मालिक बनाया कि हम बेहिश्त में जहाँ चाहें रहें तो नेक चलन वालों की भी क्या ख़ूब (खरी) मज़दूरी है
\end{hindi}}
\flushright{\begin{Arabic}
\quranayah[39][75]
\end{Arabic}}
\flushleft{\begin{hindi}
और (उस दिन) फरिश्तों को देखोगे कि अर्श के गिर्दा गिर्द घेरे हुए डटे होंगे और अपने परवरदिगार की तारीफ की (तसबीह) कर रहे होंगे और लोगों के दरमियान ठीक फैसला कर दिया जाएगा और (हर तरफ से यही) सदा बुलन्द होगी अल्हमदो लिल्लाहे रब्बिल आलेमीन
\end{hindi}}
\chapter{Al-Mu'min (The Believer)}
\begin{Arabic}
\Huge{\centerline{\basmalah}}\end{Arabic}
\flushright{\begin{Arabic}
\quranayah[40][1]
\end{Arabic}}
\flushleft{\begin{hindi}
हा मीम
\end{hindi}}
\flushright{\begin{Arabic}
\quranayah[40][2]
\end{Arabic}}
\flushleft{\begin{hindi}
(इस) किताब (कुरान) का नाज़िल करना (ख़ास बारगाहे) ख़ुदा से है जो (सबसे) ग़ालिब बड़ा वाक़िफ़कार है
\end{hindi}}
\flushright{\begin{Arabic}
\quranayah[40][3]
\end{Arabic}}
\flushleft{\begin{hindi}
गुनाहों का बख्शने वाला और तौबा का क़ुबूल करने वाला सख्त अज़ाब देने वाला साहिबे फज़ल व करम है उसके सिवा कोई माबूद नहीं उसी की तरफ (सबको) लौट कर जाना है
\end{hindi}}
\flushright{\begin{Arabic}
\quranayah[40][4]
\end{Arabic}}
\flushleft{\begin{hindi}
ख़ुदा की आयतों में बस वही लोग झगड़े पैदा करते हैं जो काफिर हैं तो (ऐ रसूल) उन लोगों का शहरों (शहरों) घूमना फिरना और माल हासिल करना
\end{hindi}}
\flushright{\begin{Arabic}
\quranayah[40][5]
\end{Arabic}}
\flushleft{\begin{hindi}
तुम्हें इस धोखे में न डाले (कि उन पर आज़ाब न होगा) इन के पहले नूह की क़ौम ने और उन के बाद और उम्मतों ने (अपने पैग़म्बरों को) झुठलाया और हर उम्मत ने अपने पैग़म्बरों के बारे में यही ठान लिया कि उन्हें गिरफ्तार कर (के क़त्ल कर डालें) और बेहूदा बातों की आड़ पकड़ कर लड़ने लगें - ताकि उसके ज़रिए से हक़ बात को उखाड़ फेंकें तो मैंने, उन्हें गिरफ्तार कर लिया फिर देखा कि उन पर (मेरा अज़ाब कैसा (सख्त हुआ)
\end{hindi}}
\flushright{\begin{Arabic}
\quranayah[40][6]
\end{Arabic}}
\flushleft{\begin{hindi}
और इसी तरह तुम्हारे परवरदिगार का अज़ाब का हुक्म (उन) काफ़िरों पर पूरा हो चुका है कि यह लोग यक़ीनी जहन्नुमी हैं
\end{hindi}}
\flushright{\begin{Arabic}
\quranayah[40][7]
\end{Arabic}}
\flushleft{\begin{hindi}
जो (फ़रिश्ते) अर्श को उठाए हुए हैं और जो उस के गिर्दा गिर्द (तैनात) हैं (सब) अपने परवरदिगार की तारीफ़ के साथ तसबीह करते हैं और उस पर ईमान रखते हैं और मोमिनों के लिए बख़शिश की दुआएं माँगा करते हैं कि परवरदिगार तेरी रहमत और तेरा इल्म हर चीज़ पर अहाता किए हुए हैं, तो जिन लोगों ने (सच्चे) दिल से तौबा कर ली और तेरे रास्ते पर चले उनको बख्श दे और उनको जहन्नुम के अज़ाब से बचा ले
\end{hindi}}
\flushright{\begin{Arabic}
\quranayah[40][8]
\end{Arabic}}
\flushleft{\begin{hindi}
ऐ हमारे पालने वाले इन को सदाबहार बाग़ों में जिनका तूने उन से वायदा किया है दाख़िल कर और उनके बाप दादाओं और उनकी बीवीयों और उनकी औलाद में से जो लोग नेक हो उनको (भी बख्श दें) बेशक तू ही ज़बरदस्त (और) हिकमत वाला है
\end{hindi}}
\flushright{\begin{Arabic}
\quranayah[40][9]
\end{Arabic}}
\flushleft{\begin{hindi}
और उनको हर किस्म की बुराइयों से महफूज़ रख और जिसको तूने उस दिन ( कयामत ) के अज़ाबों से बचा लिया उस पर तूने बड़ा रहम किया और यही तो बड़ी कामयाबी है
\end{hindi}}
\flushright{\begin{Arabic}
\quranayah[40][10]
\end{Arabic}}
\flushleft{\begin{hindi}
(हाँ) जिन लोगों ने कुफ्र एख्तेयार किया उनसे पुकार कर कह दिया जाएगा कि जितना तुम (आज) अपनी जान से बेज़ार हो उससे बढ़कर ख़ुदा तुमसे बेज़ार था जब तुम ईमान की तरफ बुलाए जाते थे तो कुफ्र करते थे
\end{hindi}}
\flushright{\begin{Arabic}
\quranayah[40][11]
\end{Arabic}}
\flushleft{\begin{hindi}
वह लोग कहेंगे कि ऐ हमारे परवरदिगार तू हमको दो बार मार चुका और दो बार ज़िन्दा कर चुका तो अब हम अपने गुनाहों का एक़रार करते हैं तो क्या (यहाँ से) निकलने की भी कोई सबील है
\end{hindi}}
\flushright{\begin{Arabic}
\quranayah[40][12]
\end{Arabic}}
\flushleft{\begin{hindi}
ये इसलिए कि जब तन्हा ख़ुदा पुकारा जाता था तो तुम ईन्कार करते थे और अगर उसके साथ शिर्क किया जाता था तो तुम मान लेते थे तो (आज) ख़ुदा की हुकूमत है जो आलीशान (और) बुर्ज़ुग है
\end{hindi}}
\flushright{\begin{Arabic}
\quranayah[40][13]
\end{Arabic}}
\flushleft{\begin{hindi}
वही तो है जो तुमको (अपनी कुदरत की) निशानियाँ दिखाता है और तुम्हारे लिए आसमान से रोज़ी नाज़िल करता है और नसीहत तो बस वही हासिल करता है जो (उसकी तरफ) रूज़ू करता है
\end{hindi}}
\flushright{\begin{Arabic}
\quranayah[40][14]
\end{Arabic}}
\flushleft{\begin{hindi}
पस तुम लोग ख़ुदा की इबादत को ख़ालिस करके उसी को पुकारो अगरचे कुफ्फ़ार बुरा मानें
\end{hindi}}
\flushright{\begin{Arabic}
\quranayah[40][15]
\end{Arabic}}
\flushleft{\begin{hindi}
ख़ुदा तो बड़ा आली मरतबा अर्श का मालिक है, वह अपने बन्दों में से जिस पर चाहता है अपने हुक्म से 'वही' नाज़िल करता है ताकि (लोगों को) मुलाकात (क़यामत) के दिन से डराएं
\end{hindi}}
\flushright{\begin{Arabic}
\quranayah[40][16]
\end{Arabic}}
\flushleft{\begin{hindi}
जिस दिन वह लोग (क़ब्रों) से निकल पड़ेंगे (और) उनको कोई चीज़ ख़ुदा से पोशीदा नहीं रहेगी (और निदा आएगी) आज किसकी बादशाहत है ( फिर ख़ुदा ख़़ुद कहेगा ) ख़ास ख़ुदा की जो अकेला (और) ग़ालिब है
\end{hindi}}
\flushright{\begin{Arabic}
\quranayah[40][17]
\end{Arabic}}
\flushleft{\begin{hindi}
आज हर शख्स को उसके किए का बदला दिया जाएगा, आज किसी पर कुछ भी ज़ुल्म न किया जाएगा बेशक ख़ुदा बहुत जल्द हिसाब लेने वाला है
\end{hindi}}
\flushright{\begin{Arabic}
\quranayah[40][18]
\end{Arabic}}
\flushleft{\begin{hindi}
(ऐ रसूल) तुम उन लोगों को उस दिन से डराओ जो अनक़रीब आने वाला है जब लोगों के कलेजे घुट घुट के (मारे डर के) मुँह को आ जाएंगें (उस वक्त) न तो सरकशों का कोई सच्चा दोस्त होगा और न कोई ऐसा सिफारिशी जिसकी बात मान ली जाए
\end{hindi}}
\flushright{\begin{Arabic}
\quranayah[40][19]
\end{Arabic}}
\flushleft{\begin{hindi}
ख़ुदा तो ऑंखों की दुज़दीदा (ख़यानत की) निगाह को भी जानता है और उन बातों को भी जो (लोगों के) सीनों में पोशीदा है
\end{hindi}}
\flushright{\begin{Arabic}
\quranayah[40][20]
\end{Arabic}}
\flushleft{\begin{hindi}
और ख़ुदा ठीक ठीक हुक्म देता है, और उसके सिवा जिनकी ये लोग इबादत करते हैं वह तो कुछ भी हुक्म नहीं दे सकते, इसमें शक नहीं कि ख़ुदा सुनने वाला देखने वाला है
\end{hindi}}
\flushright{\begin{Arabic}
\quranayah[40][21]
\end{Arabic}}
\flushleft{\begin{hindi}
क्या उन लोगों ने रूए ज़मीन पर चल फिर कर नहीं देखा कि जो लोग उनसे पहले थे उनका अन्जाम क्या हुआ (हालाँकि) वह लोग कुवत (शान और उम्र सब) में और ज़मीन पर अपनी निशानियाँ (यादगारें इमारतें) वग़ैरह छोड़ जाने में भी उनसे कहीं बढ़ चढ़ के थे तो ख़ुदा ने उनके गुनाहों की वजह से उनकी ले दे की, और ख़ुदा (के ग़ज़ब से) उनका कोई बचाने वाला भी न था
\end{hindi}}
\flushright{\begin{Arabic}
\quranayah[40][22]
\end{Arabic}}
\flushleft{\begin{hindi}
ये इस सबब से कि उनके पैग़म्बरान उनके पास वाज़ेए व रौशन मौजिज़े ले कर आए इस पर भी उन लोगों ने न माना तो ख़ुदा ने उन्हें ले डाला इसमें तो शक ही नहीं कि वह क़वी (और) सख्त अज़ाब वाला है
\end{hindi}}
\flushright{\begin{Arabic}
\quranayah[40][23]
\end{Arabic}}
\flushleft{\begin{hindi}
और हमने मूसा को अपनी निशानियाँ और रौशन दलीलें देकर
\end{hindi}}
\flushright{\begin{Arabic}
\quranayah[40][24]
\end{Arabic}}
\flushleft{\begin{hindi}
फिरऔन और हामान और क़ारून के पास भेजा तो वह लोग कहने लगे कि (ये तो) एक बड़ा झूठा (और) जादूगर है
\end{hindi}}
\flushright{\begin{Arabic}
\quranayah[40][25]
\end{Arabic}}
\flushleft{\begin{hindi}
ग़रज़ जब मूसा उन लोगों के पास हमारी तरफ से सच्चा दीन ले कर आये तो वह बोले कि जो लोग उनके साथ ईमान लाए हैं उनके बेटों को तो मार डालों और उनकी औरतों को (लौन्डिया बनाने के लिए) ज़िन्दा रहने दो और काफ़िरों की तद्बीरें तो बे ठिकाना होती हैं
\end{hindi}}
\flushright{\begin{Arabic}
\quranayah[40][26]
\end{Arabic}}
\flushleft{\begin{hindi}
और फिरऔन कहने लगा मुझे छोड़ दो कि मैं मूसा को तो क़त्ल कर डालूँ, और ( मैं देखूँ ) अपने परवरदिगार को तो अपनी मदद के लिए बुलालें (भाईयों) मुझे अन्देशा है कि (मुबादा) तुम्हारे दीन को उलट पुलट कर डाले या मुल्क में फसाद पैदा कर दें
\end{hindi}}
\flushright{\begin{Arabic}
\quranayah[40][27]
\end{Arabic}}
\flushleft{\begin{hindi}
और मूसा ने कहा कि मैं तो हर मुताकब्बिर से जो हिसाब के दिन (क़यामत पर ईमान नहीं लाता) अपने और तुम्हारे परवरदिगार की पनाह ले चुका हूं
\end{hindi}}
\flushright{\begin{Arabic}
\quranayah[40][28]
\end{Arabic}}
\flushleft{\begin{hindi}
और फिरऔन के लोगों में एक ईमानदार शख्स (हिज़कील) ने जो अपने ईमान को छिपाए रहता था (लोगों से कहा) कि क्या तुम लोग ऐसे शख्स के क़त्ल के दरपै हो जो (सिर्फ) ये कहता है कि मेरा परवरदिगार अल्लाह है) हालाँकि वह तुम्हारे परवरदिगार की तरफ से तुम्हारे पास मौजिज़े लेकर आया और अगर (बिल ग़रज़) ये शख्स झूठा है तो इसके झूठ का बवाल इसी पर पड़ेगा और अगर यह कहीं सच्चा हुआ तो जिस (अज़ाब की) तुम्हें ये धमकी देता है उसमें से कुछ तो तुम लोगों पर ज़रूर वाकेए होकर रहेगा बेशक ख़ुदा उस शख्स की हिदायत नहीं करता जो हद से गुज़रने वाला (और) झूठा हो
\end{hindi}}
\flushright{\begin{Arabic}
\quranayah[40][29]
\end{Arabic}}
\flushleft{\begin{hindi}
ऐ मेरी क़ौम आज तो (बेशक) तुम्हारी बादशाहत है (और) मुल्क में तुम्हारा ही बोल बाला है लेकिन (कल) अगर ख़ुदा का अज़ाब हम पर आ जाए तो हमारी कौन मदद करेगा फिरऔन ने कहा मैं तो वही बात समझाता हूँ जो मैं ख़़ुद समझता हूँ और वही राह दिखाता हूँ जिसमें भलाई है
\end{hindi}}
\flushright{\begin{Arabic}
\quranayah[40][30]
\end{Arabic}}
\flushleft{\begin{hindi}
तो जो शख्स (दर पर्दा) ईमान ला चुका था कहने लगा, भाईयों मुझे तो तुम्हारी निस्बत भी और उम्मतों की तरह रोज़ (बद) का अन्देशा है
\end{hindi}}
\flushright{\begin{Arabic}
\quranayah[40][31]
\end{Arabic}}
\flushleft{\begin{hindi}
(कहीं तुम्हारा भी वही हाल न हो) जैसा कि नूह की क़ौम और आद समूद और उनके बाद वाले लोगों का हाल हुआ और ख़ुदा तो बन्दों पर ज़ुल्म करना चाहता ही नहीं
\end{hindi}}
\flushright{\begin{Arabic}
\quranayah[40][32]
\end{Arabic}}
\flushleft{\begin{hindi}
और ऐ हमारी क़ौम मुझे तो तुम्हारी निस्बत कयामत के दिन का अन्देशा है
\end{hindi}}
\flushright{\begin{Arabic}
\quranayah[40][33]
\end{Arabic}}
\flushleft{\begin{hindi}
जिस दिन तुम पीठ फेर कर (जहन्नुम) की तरफ चल खड़े होगे तो ख़ुदा के (अज़ाब) से तुम्हारा बचाने वाला न होगा, और जिसको ख़ुदा गुमराही में छोड़ दे तो उसका कोई रूबराह करने वाला नहीं
\end{hindi}}
\flushright{\begin{Arabic}
\quranayah[40][34]
\end{Arabic}}
\flushleft{\begin{hindi}
और (इससे) पहले यूसुफ़ भी तुम्हारे पास मौजिज़े लेकर आए थे तो जो जो लाए थे तुम लोग उसमें बराबर शक ही करते रहे यहाँ तक कि जब उन्होने वफात पायी तो तुम कहने लगे कि अब उनके बाद ख़ुदा हरगिज़ कोई रसूल नहीं भेजेगा जो हद से गुज़रने वाला और शक करने वाला है ख़ुदा उसे यू हीं गुमराही में छोड़ देता है
\end{hindi}}
\flushright{\begin{Arabic}
\quranayah[40][35]
\end{Arabic}}
\flushleft{\begin{hindi}
जो लोग बगैर इसके कि उनके पास कोई दलील आई हो ख़ुदा की आयतों में (ख्वाह मा ख्वाह) झगड़े किया करते हैं वह ख़ुदा के नज़दीक और ईमानदारों के नज़दीक सख्त नफरत खेज़ हैं, यूँ ख़ुदा हर मुतकब्बिर सरकश के दिल पर अलामत मुक़र्रर कर देता है
\end{hindi}}
\flushright{\begin{Arabic}
\quranayah[40][36]
\end{Arabic}}
\flushleft{\begin{hindi}
और फिरऔन ने कहा ऐ हामान हमारे लिए एक महल बनवा दे ताकि (उस पर चढ़ कर) रसतों पर पहुँच जाऊ
\end{hindi}}
\flushright{\begin{Arabic}
\quranayah[40][37]
\end{Arabic}}
\flushleft{\begin{hindi}
(यानि) आसमानों के रसतों पर फिर मूसा के ख़ुदा को झांक कर (देख) लूँ और मैं तो उसे यक़ीनी झूठा समझता हूँ, और इसी तरह फिरऔन की बदकिरदारयाँ उसको भली करके दिखा दी गयीं थीं और वह राहे रास्ता से रोक दिया गया था, और फिरऔन की तद्बीर तो बस बिल्कुल ग़ारत गुल ही थीं
\end{hindi}}
\flushright{\begin{Arabic}
\quranayah[40][38]
\end{Arabic}}
\flushleft{\begin{hindi}
और जो शख्स (दर पर्दा) ईमानदार था कहने लगा भाईयों मेरा कहना मानों मैं तुम्हें हिदायत के रास्ते दिख दूंगा
\end{hindi}}
\flushright{\begin{Arabic}
\quranayah[40][39]
\end{Arabic}}
\flushleft{\begin{hindi}
भाईयों ये दुनियावी ज़िन्दगी तो बस (चन्द रोज़ा) फ़ायदा है और आखेरत ही हमेशा रहने का घर है
\end{hindi}}
\flushright{\begin{Arabic}
\quranayah[40][40]
\end{Arabic}}
\flushleft{\begin{hindi}
जो बुरा काम करेगा तो उसे बदला भी वैसा ही मिलेगा, और जो नेक काम करेगा मर्द हो या औरत मगर ईमानदार हो तो ऐसे लोग बेहिश्त में दाख़िल होंगे वहाँ उन्हें बेहिसाब रोज़ी मिलेगी
\end{hindi}}
\flushright{\begin{Arabic}
\quranayah[40][41]
\end{Arabic}}
\flushleft{\begin{hindi}
और ऐ मेरी क़ौम मुझे क्या हुआ है कि मैं तुमको नजाद की तरफ बुलाता हूँ और तुम मुझे दोज़ख़ की तरफ बुलाते हो
\end{hindi}}
\flushright{\begin{Arabic}
\quranayah[40][42]
\end{Arabic}}
\flushleft{\begin{hindi}
तुम मुझे बुलाते हो कि मै ख़ुदा के साथ कुफ्र करूं और उस चीज़ को उसका शरीक बनाऊ जिसका मुझे इल्म में भी नहीं, और मैं तुम्हें ग़ालिब (और) बड़े बख्शने वाले ख़ुदा की तरफ बुलाता हूँ
\end{hindi}}
\flushright{\begin{Arabic}
\quranayah[40][43]
\end{Arabic}}
\flushleft{\begin{hindi}
बेशक तुम जिस चीज़ की तरफ़ मुझे बुलाते हो वह न तो दुनिया ही में पुकारे जाने के क़ाबिल है और न आख़िरत में और आख़िर में हम सबको ख़ुदा ही की तरफ लौट कर जाना है और इसमें तो शक ही नहीं कि हद से बढ़ जाने वाले जहन्नुमी हैं
\end{hindi}}
\flushright{\begin{Arabic}
\quranayah[40][44]
\end{Arabic}}
\flushleft{\begin{hindi}
तो जो मैं तुमसे कहता हूँ अनक़रीब ही उसे याद करोगे और मैं तो अपना काम ख़ुदा ही को सौंपे देता हूँ कुछ शक नहीं की ख़ुदा बन्दों ( के हाल ) को ख़ूब देख रहा है
\end{hindi}}
\flushright{\begin{Arabic}
\quranayah[40][45]
\end{Arabic}}
\flushleft{\begin{hindi}
तो ख़ुदा ने उसे उनकी तद्बीरों की बुराई से महफूज़ रखा और फिरऔनियों को बड़े अज़ाब ने ( हर तरफ ) से घेर लिया
\end{hindi}}
\flushright{\begin{Arabic}
\quranayah[40][46]
\end{Arabic}}
\flushleft{\begin{hindi}
और अब तो कब्र में दोज़ख़ की आग है कि वह लोग (हर) सुबह व शाम उसके सामने ला खड़े किए जाते हैं और जिस दिन क़यामत बरपा होगी (हुक्म होगा) फिरऔन के लोगों को सख्त से सख्त अज़ाब में झोंक दो
\end{hindi}}
\flushright{\begin{Arabic}
\quranayah[40][47]
\end{Arabic}}
\flushleft{\begin{hindi}
और ये लोग जिस वक्त ज़हन्नुम में बाहम झगड़ेंगें तो कम हैसियत लोग बड़े आदमियों से कहेंगे कि हम तुम्हारे ताबे थे तो क्या तुम इस वक्त (दोज़ख़ की) आग का कुछ हिस्सा हमसे हटा सकते हो
\end{hindi}}
\flushright{\begin{Arabic}
\quranayah[40][48]
\end{Arabic}}
\flushleft{\begin{hindi}
तो बड़े लोग कहेंगें (अब तो) हम (तुम) सबके सब आग में पड़े हैं ख़ुदा (को) तो बन्दों के बारे में (जो कुछ) फैसला (करना था) कर चुका
\end{hindi}}
\flushright{\begin{Arabic}
\quranayah[40][49]
\end{Arabic}}
\flushleft{\begin{hindi}
और जो लोग आग में (जल रहे) होंगे जहन्नुम के दरोग़ाओं से दरख्वास्त करेंगे कि अपने परवरदिगार से अर्ज़ करो कि एक दिन तो हमारे अज़ाब में तख़फ़ीफ़ कर दें
\end{hindi}}
\flushright{\begin{Arabic}
\quranayah[40][50]
\end{Arabic}}
\flushleft{\begin{hindi}
वह जवाब देंगे कि क्या तुम्हारे पास तुम्हारे पैग़म्बर साफ व रौशन मौजिज़े लेकर नहीं आए थे वह कहेंगे (हाँ) आए तो थे, तब फरिश्ते तो कहेंगे फिर तुम ख़़ुद (क्यों) न दुआ करो, हालाँकि काफ़िरों की दुआ तो बस बेकार ही है
\end{hindi}}
\flushright{\begin{Arabic}
\quranayah[40][51]
\end{Arabic}}
\flushleft{\begin{hindi}
हम अपने पैग़म्बरों की और ईमान वालों की दुनिया की ज़िन्दगी में भी ज़रूर मदद करेंगे और जिस दिन गवाह (पैग़म्बर फरिश्ते गवाही को) उठ खड़े होंगे
\end{hindi}}
\flushright{\begin{Arabic}
\quranayah[40][52]
\end{Arabic}}
\flushleft{\begin{hindi}
(उस दिन भी) जिस दिन ज़ालिमों को उनकी माज़ेरत कुछ भी फायदे न देगी और उन पर फिटकार (बरसती) होगी और उनके लिए बहुत बुरा घर (जहन्नुम) है
\end{hindi}}
\flushright{\begin{Arabic}
\quranayah[40][53]
\end{Arabic}}
\flushleft{\begin{hindi}
और हम ही ने मूसा को हिदायत (की किताब तौरेत) दी और बनी इसराईल को (उस) किताब का वारिस बनाया
\end{hindi}}
\flushright{\begin{Arabic}
\quranayah[40][54]
\end{Arabic}}
\flushleft{\begin{hindi}
जो अक्लमन्दों के लिए (सरतापा) हिदायत व नसीहत है
\end{hindi}}
\flushright{\begin{Arabic}
\quranayah[40][55]
\end{Arabic}}
\flushleft{\begin{hindi}
(ऐ रसूल) तुम (उनकी शरारत) पर सब्र करो बेशक ख़ुदा का वायदा सच्चा है, और अपने (उम्मत की) गुनाहों की माफी माँगो और सुबह व शाम अपने परवरदिगार की हम्द व सना के साथ तसबीह करते रहो
\end{hindi}}
\flushright{\begin{Arabic}
\quranayah[40][56]
\end{Arabic}}
\flushleft{\begin{hindi}
जिन लोगों के पास (ख़ुदा की तरफ से) कोई दलील तो आयी नहीं और (फिर) वह ख़ुदा की आयतों में (ख्वाह मा ख्वाह) झगड़े निकालते हैं, उनके दिल में बुराई (की बेजां हवस) के सिवा कुछ नहीं हालाँकि वह लोग उस तक कभी पहुँचने वाले नहीं तो तुम बस ख़ुदा की पनाह माँगते रहो बेशक वह बड़ा सुनने वाला (और) देखने वाला है
\end{hindi}}
\flushright{\begin{Arabic}
\quranayah[40][57]
\end{Arabic}}
\flushleft{\begin{hindi}
सारे आसमान और ज़मीन का पैदा करना लोगों के पैदा करने की ये निस्बत यक़ीनी बड़ा (काम) है मगर अक्सर लोग (इतना भी) नहीं जानते
\end{hindi}}
\flushright{\begin{Arabic}
\quranayah[40][58]
\end{Arabic}}
\flushleft{\begin{hindi}
और अंधा और ऑंख वाला (दोनों) बराबर नहीं हो सकते और न मोमिनीन जिन्होने अच्छे काम किए और न बदकार (ही) बराबर हो सकते हैं बात ये है कि तुम लोग बहुत कम ग़ौर करते हो, कयामत तो ज़रूर आने वाली है
\end{hindi}}
\flushright{\begin{Arabic}
\quranayah[40][59]
\end{Arabic}}
\flushleft{\begin{hindi}
इसमें किसी तरह का शक नहीं मगर अक्सर लोग (इस पर भी) ईमान नहीं रखते
\end{hindi}}
\flushright{\begin{Arabic}
\quranayah[40][60]
\end{Arabic}}
\flushleft{\begin{hindi}
और तुम्हारा परवरदिगार इरशाद फ़रमाता है कि तुम मुझसे दुआएं माँगों मैं तुम्हारी (दुआ) क़ुबूल करूँगा जो लोग हमारी इबादत से अकड़ते हैं वह अनक़रीब ही ज़लील व ख्वार हो कर यक़ीनन जहन्नुम वासिल होंगे
\end{hindi}}
\flushright{\begin{Arabic}
\quranayah[40][61]
\end{Arabic}}
\flushleft{\begin{hindi}
ख़ुदा ही तो है जिसने तुम्हारे वास्ते रात बनाई ताकि तुम उसमें आराम करो और दिन को रौशन (बनाया) तकि काम करो बेशक ख़ुदा लोगों परा बड़ा फज़ल व करम वाला है, मगर अक्सर लोग उसका शुक्र नहीं अदा करते
\end{hindi}}
\flushright{\begin{Arabic}
\quranayah[40][62]
\end{Arabic}}
\flushleft{\begin{hindi}
यही ख़ुदा तुम्हारा परवरदिगार है जो हर चीज़ का ख़ालिक़ है, और उसके सिवा कोई माबूद नहीं, फिर तुम कहाँ बहके जा रहे हो
\end{hindi}}
\flushright{\begin{Arabic}
\quranayah[40][63]
\end{Arabic}}
\flushleft{\begin{hindi}
जो लोग ख़ुदा की आयतों से इन्कार रखते थे वह इसी तरह भटक रहे थे
\end{hindi}}
\flushright{\begin{Arabic}
\quranayah[40][64]
\end{Arabic}}
\flushleft{\begin{hindi}
अल्लाह ही तो है जिसने तुम्हारे वास्ते ज़मीन को ठहरने की जगह और आसमान को छत बनाया और उसी ने तुम्हारी सूरतें बनायीं तो अच्छी सूरतें बनायीं और उसी ने तुम्हें साफ सुथरी चीज़ें खाने को दीं यही अल्लाह तो तुम्हारा परवरदिगार है तो ख़ुदा बहुत ही मुतबर्रिक है जो सारे जहाँन का पालने वाला है
\end{hindi}}
\flushright{\begin{Arabic}
\quranayah[40][65]
\end{Arabic}}
\flushleft{\begin{hindi}
वही (हमेशा) ज़िन्दा है और उसके सिवा कोई माबूद नहीं तो निरी खरी उसी की इबादत करके उसी से ये दुआ माँगो, सब तारीफ ख़ुदा ही को सज़ावार है और जो सारे जहाँन का पालने वाला है
\end{hindi}}
\flushright{\begin{Arabic}
\quranayah[40][66]
\end{Arabic}}
\flushleft{\begin{hindi}
(ऐ रसूल) तुम कह दो कि जब मेरे पास मेरे परवरदिगार की बारगाह से खुले हुए मौजिज़े आ चुके तो मुझे इस बात की मनाही कर दी गयी है कि ख़ुदा को छोड़ कर जिनको तुम पूजते हो मैं उनकी परसतिश करूँ और मुझे तो यह हुक्म हो चुका है कि मैं सारे जहाँन के पालने वाले का फरमाबरदार बनु
\end{hindi}}
\flushright{\begin{Arabic}
\quranayah[40][67]
\end{Arabic}}
\flushleft{\begin{hindi}
वही वह ख़ुदा है जिसने तुमको पहले (पहल) मिटटी से पैदा किया फिर नुत्फे से, फिर जमे हुए ख़ून फिर तुमको बच्चा बनाकर (माँ के पेट) से निकलता है (ताकि बढ़ों) फिर (ज़िन्दा रखता है) ताकि तुम अपनी जवानी को पहुँचो फिर (और ज़िन्दा रखता है ताकि तुम बूढ़े हो जाओ और तुममें से कोई ऐसा भी है जो (इससे) पहले मर जाता है ग़रज़ (तुमको उस वक्त तक ज़िन्दा रखता है) की तुम (मौत के) मुकर्रर वक्त तक पहुँच जाओ
\end{hindi}}
\flushright{\begin{Arabic}
\quranayah[40][68]
\end{Arabic}}
\flushleft{\begin{hindi}
और ताकि तुम (उसकी क़ुदरत को समझो) वह वही (ख़ुदा) है जो जिलाता और मारता है, फिर जब वह किसी काम का करना ठान लेता है तो बस उससे कह देता है कि 'हो जा' तो वह फ़ौरन हो जाता है
\end{hindi}}
\flushright{\begin{Arabic}
\quranayah[40][69]
\end{Arabic}}
\flushleft{\begin{hindi}
(ऐ रसूल) क्या तुमने उन लोगों (की हालत) पर ग़ौर नहीं किया जो ख़ुदा की आयतों में झगड़े निकाला करते हैं
\end{hindi}}
\flushright{\begin{Arabic}
\quranayah[40][70]
\end{Arabic}}
\flushleft{\begin{hindi}
ये कहाँ भटके चले जा रहे हैं, जिन लोगों ने किताबे (ख़ुदा) और उन बातों को जो हमने पैग़म्बरों को देकर भेजा था झुठलाया तो बहुत जल्द उसका नतीजा उन्हें मालूम हो जाएगा
\end{hindi}}
\flushright{\begin{Arabic}
\quranayah[40][71]
\end{Arabic}}
\flushleft{\begin{hindi}
जब (भारी भारी) तौक़ और ज़ंजीरें उनकी गर्दनों में होंगी (और पहले) खौलते हुए पानी में घसीटे जाएँगे
\end{hindi}}
\flushright{\begin{Arabic}
\quranayah[40][72]
\end{Arabic}}
\flushleft{\begin{hindi}
फिर (जहन्नुम की) आग में झोंक दिए जाएँगे
\end{hindi}}
\flushright{\begin{Arabic}
\quranayah[40][73]
\end{Arabic}}
\flushleft{\begin{hindi}
फिर उनसे पूछा जाएगा कि ख़ुदा के सिवा जिनको (उसका) शरीक बनाते थे
\end{hindi}}
\flushright{\begin{Arabic}
\quranayah[40][74]
\end{Arabic}}
\flushleft{\begin{hindi}
(इस वक्त) क़हाँ हैं वह कहेंगे अब तो वह हमसे जाते रहे बल्कि (सच यूँ है कि) हम तो पहले ही से (ख़ुदा के सिवा) किसी चीज़ की परसतिश न करते थे यूँ ख़ुदा काफिरों को बौखला देगा
\end{hindi}}
\flushright{\begin{Arabic}
\quranayah[40][75]
\end{Arabic}}
\flushleft{\begin{hindi}
(कि कुछ समझ में न आएगा) ये उसकी सज़ा है कि तुम दुनिया में नाहक (बात पर) निहाल थे और इसकी सज़ा है कि तुम इतराया करते थे
\end{hindi}}
\flushright{\begin{Arabic}
\quranayah[40][76]
\end{Arabic}}
\flushleft{\begin{hindi}
अब जहन्नुम के दरवाज़े में दाख़िल हो जाओ (और) हमेशा उसी में (पड़े) रहो, ग़रज़ तकब्बुर करने वालों का भी (क्या) बुरा ठिकाना है
\end{hindi}}
\flushright{\begin{Arabic}
\quranayah[40][77]
\end{Arabic}}
\flushleft{\begin{hindi}
तो (ऐ रसूल) तुम सब्र करो ख़ुदा का वायदा यक़ीनी सच्चा है तो जिस (अज़ाब) की हम उन्हें धमकी देते हैं अगर हम तुमको उसमें कुछ दिखा दें या तुम ही को (इसके क़ब्ल) दुनिया से उठा लें तो (आख़िर फिर) उनको हमारी तरफ लौट कर आना है,
\end{hindi}}
\flushright{\begin{Arabic}
\quranayah[40][78]
\end{Arabic}}
\flushleft{\begin{hindi}
और तुमसे पहले भी हमने बहुत से पैग़म्बर भेजे उनमें से कुछ तो ऐसे हैं जिनके हालात हमने तुमसे बयान कर दिए, और कुछ ऐसे हैं जिनके हालात तुमसे नहीं दोहराए और किसी पैग़म्बर की ये मजाल न थी कि ख़ुदा के ऐख्तेयार दिए बग़ैर कोई मौजिज़ा दिखा सकें फिर जब ख़ुदा का हुक्म (अज़ाब) आ पहुँचा तो ठीक ठीक फैसला कर दिया गया और अहले बातिल ही इस घाटे में रहे,
\end{hindi}}
\flushright{\begin{Arabic}
\quranayah[40][79]
\end{Arabic}}
\flushleft{\begin{hindi}
ख़ुदा ही तो वह है जिसने तुम्हारे लिए चारपाए पैदा किए ताकि तुम उनमें से किसी पर सवार होते हो और किसी को खाते हो
\end{hindi}}
\flushright{\begin{Arabic}
\quranayah[40][80]
\end{Arabic}}
\flushleft{\begin{hindi}
और तुम्हारे लिए उनमें (और भी) फायदे हैं और ताकि तुम उन पर (चढ़ कर) अपनी दिली मक़सद तक पहुँचो और उन पर और (नीज़) कश्तियों पर सवार फिरते हो
\end{hindi}}
\flushright{\begin{Arabic}
\quranayah[40][81]
\end{Arabic}}
\flushleft{\begin{hindi}
और वह तुमको अपनी (कुदरत की) निशानियाँ दिखाता है तो तुम ख़ुदा की किन किन निशानियों को न मानोगे
\end{hindi}}
\flushright{\begin{Arabic}
\quranayah[40][82]
\end{Arabic}}
\flushleft{\begin{hindi}
तो क्या ये लोग रूए ज़मीन पर चले फिरे नहीं, तो देखते कि जो लोग इनसे पहले थे उनका क्या अंजाम हुआ, जो उनसे (तादाद में) कहीं ज्यादा थे और क़ूवत और ज़मीन पर (अपनी) निशानियाँ (यादगारें) छोड़ने में भी कहीं बढ़ चढ़ कर थे तो जो कुछ उन लोगों ने किया कराया था उनके कुछ भी काम न आया
\end{hindi}}
\flushright{\begin{Arabic}
\quranayah[40][83]
\end{Arabic}}
\flushleft{\begin{hindi}
फिर जब उनके पैग़म्बर उनके पास वाज़ेए व रौशन मौजिज़े ले कर आए तो जो इल्म (अपने ख्याल में) उनके पास था उस पर नाज़िल हुए और जिस (अज़ाब) की ये लोग हँसी उड़ाते थे उसी ने उनको चारों तरफ से घेर लिया
\end{hindi}}
\flushright{\begin{Arabic}
\quranayah[40][84]
\end{Arabic}}
\flushleft{\begin{hindi}
तो जब इन लोगों ने हमारे अज़ाब को देख लिया तो कहने लगे, हम यकता ख़ुदा पर ईमान लाए और जिस चीज़ को हम उसका शरीक बनाते थे हम उनको नहीं मानते
\end{hindi}}
\flushright{\begin{Arabic}
\quranayah[40][85]
\end{Arabic}}
\flushleft{\begin{hindi}
तो जब उन लोगों ने हमारा (अज़ाब) आते देख लिया तो अब उनका ईमान लाना कुछ भी फायदेमन्द नहीं हो सकता (ये) ख़ुदा की आदत (है) जो अपने बन्दों के बारे में (सदा से) चली आती है और काफ़िर लोग इस वक्त घाटे मे रहे
\end{hindi}}
\chapter{Ha Mim (Ha Mim)}
\begin{Arabic}
\Huge{\centerline{\basmalah}}\end{Arabic}
\flushright{\begin{Arabic}
\quranayah[41][1]
\end{Arabic}}
\flushleft{\begin{hindi}
हा मीम
\end{hindi}}
\flushright{\begin{Arabic}
\quranayah[41][2]
\end{Arabic}}
\flushleft{\begin{hindi}
(ये किताब) रहमान व रहीम ख़ुदा की तरफ से नाज़िल हुई है ये (वह) किताब अरबी क़ुरान है
\end{hindi}}
\flushright{\begin{Arabic}
\quranayah[41][3]
\end{Arabic}}
\flushleft{\begin{hindi}
जिसकी आयतें समझदार लोगें के वास्ते तफ़सील से बयान कर दी गयीं हैं
\end{hindi}}
\flushright{\begin{Arabic}
\quranayah[41][4]
\end{Arabic}}
\flushleft{\begin{hindi}
ं(नेको कारों को) ख़़ुशख़बरी देने वाली और (बदकारों को) डराने वाली है इस पर भी उनमें से अक्सर ने मुँह फेर लिया और वह सुनते ही नहीं
\end{hindi}}
\flushright{\begin{Arabic}
\quranayah[41][5]
\end{Arabic}}
\flushleft{\begin{hindi}
और कहने लगे जिस चीज़ की तरफ तुम हमें बुलाते हो उससे तो हमारे दिल पर्दों में हैं (कि दिल को नहीं लगती) और हमारे कानों में गिर्दानी (बहरापन है) कि कुछ सुनायी नहीं देता और हमारे तुम्हारे दरमियान एक पर्दा (हायल) है तो तुम (अपना) काम करो हम (अपना) काम करते हैं
\end{hindi}}
\flushright{\begin{Arabic}
\quranayah[41][6]
\end{Arabic}}
\flushleft{\begin{hindi}
(ऐ रसूल) कह दो कि मैं भी बस तुम्हारा ही सा आदमी हूँ (मगर फ़र्क ये है कि) मुझ पर 'वही' आती है कि तुम्हारा माबूद बस (वही) यकता ख़ुदा है तो सीधे उसकी तरफ मुतावज्जे रहो और उसी से बख़शिश की दुआ माँगो, और मुशरेकों पर अफसोस है
\end{hindi}}
\flushright{\begin{Arabic}
\quranayah[41][7]
\end{Arabic}}
\flushleft{\begin{hindi}
जो ज़कात नहीं देते और आखेरत के भी क़ायल नहीं
\end{hindi}}
\flushright{\begin{Arabic}
\quranayah[41][8]
\end{Arabic}}
\flushleft{\begin{hindi}
बेशक जो लोग ईमान लाए और अच्छे अच्छे काम करते रहे और उनके लिए वह सवाब है जो कभी ख़त्म होने वाला नहीं
\end{hindi}}
\flushright{\begin{Arabic}
\quranayah[41][9]
\end{Arabic}}
\flushleft{\begin{hindi}
(ऐ रसूल) तुम कह दो कि क्या तुम उस (ख़ुदा) से इन्कार करते हो जिसने ज़मीन को दो दिन में पैदा किया और तुम (औरों को) उसका हमसर बनाते हो, यही तो सारे जहाँ का सरपरस्त है
\end{hindi}}
\flushright{\begin{Arabic}
\quranayah[41][10]
\end{Arabic}}
\flushleft{\begin{hindi}
और उसी ने ज़मीन में उसके ऊपर से पहाड़ पैदा किए और उसी ने इसमें बरकत अता की और उसी ने एक मुनासिब अन्दाज़ पर इसमें सामाने माईशत का बन्दोबस्त किया (ये सब कुछ) चार दिन में और तमाम तलबगारों के लिए बराबर है
\end{hindi}}
\flushright{\begin{Arabic}
\quranayah[41][11]
\end{Arabic}}
\flushleft{\begin{hindi}
फिर आसमान की तरफ मुतावज्जे हुआ और (उस वक्त) धुएँ (का सा) था उसने उससे और ज़मीन से फरमाया कि तुम दोनों आओ ख़़ुशी से ख्वाह कराहत से, दोनों ने अर्ज़ की हम ख़़ुशी ख़़ुशी हाज़िर हैं
\end{hindi}}
\flushright{\begin{Arabic}
\quranayah[41][12]
\end{Arabic}}
\flushleft{\begin{hindi}
(और हुक्म के पाबन्द हैं) फिर उसने दोनों में उस (धुएँ) के सात आसमान बनाए और हर आसमान में उसके (इन्तेज़ाम) का हुक्म (कार कुनान कज़ा व क़दर के पास) भेज दिया और हमने नीचे वाले आसमान को (सितारों के) चिराग़ों से मज़य्यन किया और (शैतानों से महफूज़) रखा ये वाक़िफ़कार ग़ालिब ख़ुदा के (मुक़र्रर किए हुए) अन्दाज़ हैं
\end{hindi}}
\flushright{\begin{Arabic}
\quranayah[41][13]
\end{Arabic}}
\flushleft{\begin{hindi}
फिर अगर हम पर भी ये कुफ्फार मुँह फेरें तो कह दो कि मैं तुम को ऐसी बिजली गिरने (के अज़ाब से) डराता हूँ जैसी क़ौम आद व समूद की बिजली की कड़क
\end{hindi}}
\flushright{\begin{Arabic}
\quranayah[41][14]
\end{Arabic}}
\flushleft{\begin{hindi}
जब उनके पास उनके आगे से और पीछे से पैग़म्बर (ये ख़बर लेकर) आए कि ख़ुदा के सिवा किसी की इबादत न करो तो कहने लगे कि अगर हमारा परवरदिगार चाहता तो फ़रिश्ते नाज़िल करता और जो (बातें) देकर तुम लोग भेजे गए हो हम तो उसे नहीं मानते
\end{hindi}}
\flushright{\begin{Arabic}
\quranayah[41][15]
\end{Arabic}}
\flushleft{\begin{hindi}
तो आद नाहक़ रूए ज़मीन में ग़ुरूर करने लगे और कहने लगे कि हम से बढ़ के क़ूवत में कौन है, क्या उन लोगों ने इतना भी ग़ौर न किया कि ख़ुदा जिसने उनको पैदा किया है वह उनसे क़ूवत में कहीं बढ़ के है, ग़रज़ वह लोग हमारी आयतों से इन्कार ही करते रहे
\end{hindi}}
\flushright{\begin{Arabic}
\quranayah[41][16]
\end{Arabic}}
\flushleft{\begin{hindi}
तो हमने भी (तो उनके) नहूसत के दिनों में उन पर बड़ी ज़ोरों की ऑंधी चलाई ताकि दुनिया की ज़िन्दगी में भी उनको रूसवाई के अज़ाब का मज़ा चखा दें और आखेरत का अज़ाब तो और ज्यादा रूसवा करने वाला ही होगा और (फिर) उनको कहीं से मदद भी न मिलेगी
\end{hindi}}
\flushright{\begin{Arabic}
\quranayah[41][17]
\end{Arabic}}
\flushleft{\begin{hindi}
और रहे समूद तो हमने उनको सीधा रास्ता दिखाया, मगर उन लोगों ने हिदायत के मुक़ाबले में गुमराही को पसन्द किया तो उन की करतूतों की बदौलत ज़िल्लत के अज़ाब की बिजली ने उनको ले डाला
\end{hindi}}
\flushright{\begin{Arabic}
\quranayah[41][18]
\end{Arabic}}
\flushleft{\begin{hindi}
और जो लोग ईमान लाए और परहेज़गारी करते थे उनको हमने (इस) मुसीबत से बचा लिया
\end{hindi}}
\flushright{\begin{Arabic}
\quranayah[41][19]
\end{Arabic}}
\flushleft{\begin{hindi}
और जिस दिन ख़ुदा के दुशमन दोज़ख़ की तरफ हकाए जाएँगे तो ये लोग तरतीब वार खड़े किए जाएँगे
\end{hindi}}
\flushright{\begin{Arabic}
\quranayah[41][20]
\end{Arabic}}
\flushleft{\begin{hindi}
यहाँ तक की जब सब के सब जहन्नुम के पास जाएँगे तो उनके कान और उनकी ऑंखें और उनके (गोश्त पोस्त) उनके ख़िलाफ उनके मुक़ाबले में उनकी कारस्तानियों की गवाही देगें
\end{hindi}}
\flushright{\begin{Arabic}
\quranayah[41][21]
\end{Arabic}}
\flushleft{\begin{hindi}
और ये लोग अपने आज़ा से कहेंगे कि तुमने हमारे ख़िलाफ क्यों गवाही दी तो वह जवाब देंगे कि जिस ख़ुदा ने हर चीज़ को गोया किया उसने हमको भी (अपनी क़ुदरत से) गोया किया और उसी ने तुमको पहली बार पैदा किया था और (आख़िर) उसी की तरफ लौट कर जाओगे
\end{hindi}}
\flushright{\begin{Arabic}
\quranayah[41][22]
\end{Arabic}}
\flushleft{\begin{hindi}
और (तुम्हारी तो ये हालत थी कि) तुम लोग इस ख्याल से (अपने गुनाहों की) पर्दा दारी भी तो नहीं करते थे कि तुम्हारे कान और तुम्हारी ऑंखे और तुम्हारे आज़ा तुम्हारे बरख़िलाफ गवाही देंगे बल्कि तुम इस ख्याल मे (भूले हुए) थे कि ख़ुदा को तुम्हारे बहुत से कामों की ख़बर ही नहीं
\end{hindi}}
\flushright{\begin{Arabic}
\quranayah[41][23]
\end{Arabic}}
\flushleft{\begin{hindi}
और तुम्हारी इस बदख्याली ने जो तुम अपने परवरदिगार के बारे में रखते थे तुम्हें तबाह कर छोड़ा आख़िर तुम घाटे में रहे
\end{hindi}}
\flushright{\begin{Arabic}
\quranayah[41][24]
\end{Arabic}}
\flushleft{\begin{hindi}
फिर अगर ये लोग सब्र भी करें तो भी इनका ठिकाना दोज़ख़ ही है और अगर तौबा करें तो भी इनकी तौबा क़ुबूल न की जाएगी
\end{hindi}}
\flushright{\begin{Arabic}
\quranayah[41][25]
\end{Arabic}}
\flushleft{\begin{hindi}
और हमने (गोया ख़ुद शैतान को) उनका हमनशीन मुक़र्रर कर दिया था तो उन्होने उनके अगले पिछले तमाम उमूर उनकी नज़रों में भले कर दिखाए तो जिन्नात और इन्सानो की उम्मतें जो उनसे पहले गुज़र चुकी थीं उनके शुमूल (साथ) में (अज़ाब का) वायदा उनके हक़ में भी पूरा हो कर रहा बेशक ये लोग अपने घाटे के दरपै थे
\end{hindi}}
\flushright{\begin{Arabic}
\quranayah[41][26]
\end{Arabic}}
\flushleft{\begin{hindi}
और कुफ्फ़ार कहने लगे कि इस क़ुरान को सुनो ही नहीं और जब पढ़ें (तो) इसके (बीच) में ग़ुल मचा दिया करो ताकि (इस तरकीब से) तुम ग़ालिब आ जाओ
\end{hindi}}
\flushright{\begin{Arabic}
\quranayah[41][27]
\end{Arabic}}
\flushleft{\begin{hindi}
तो हम भी काफ़िरों को सख्त अज़ाब के मज़े चखाएँगे और इनकी कारस्तानियों की बहुत बड़ी सज़ा ये दोज़ख़ है
\end{hindi}}
\flushright{\begin{Arabic}
\quranayah[41][28]
\end{Arabic}}
\flushleft{\begin{hindi}
ख़ुदा के दुशमनों का बदला है कि वह जो हमरी आयतों से इन्कार करते थे उसकी सज़ा में उनके लिए उसमें हमेशा (रहने) का घर है,
\end{hindi}}
\flushright{\begin{Arabic}
\quranayah[41][29]
\end{Arabic}}
\flushleft{\begin{hindi}
और (क़यामत के दिन) कुफ्फ़ार कहेंगे कि ऐ हमारे परवरदिगार जिनों और इन्सानों में से जिन लोगों ने हमको गुमराह किया था (एक नज़र) उनको हमें दिखा दे कि हम उनको पाँव तले (रौन्द) डालें ताकि वह ख़ूब ज़लील हों
\end{hindi}}
\flushright{\begin{Arabic}
\quranayah[41][30]
\end{Arabic}}
\flushleft{\begin{hindi}
और जिन लोगों ने (सच्चे दिल से) कहा कि हमारा परवरदिगार तो (बस) ख़ुदा है, फिर वह उसी पर भी क़ायम भी रहे उन पर मौत के वक्त (रहमत के) फ़रिश्ते नाज़िल होंगे (और कहेंगे) कि कुछ ख़ौफ न करो और न ग़म खाओ और जिस बेहिश्त का तुमसे वायदा किया गया था उसकी ख़ुशियां मनाओ
\end{hindi}}
\flushright{\begin{Arabic}
\quranayah[41][31]
\end{Arabic}}
\flushleft{\begin{hindi}
हम दुनिया की ज़िन्दगी में तुम्हारे दोस्त थे और आखेरत में भी तुम्हारे (रफ़ीक़) हैं और जिस चीज़ का भी तुम्हार जी चाहे बेहिश्त में तुम्हारे वास्ते मौजूद है और जो चीज़ तलब करोगे वहाँ तुम्हारे लिए (हाज़िर) होगी
\end{hindi}}
\flushright{\begin{Arabic}
\quranayah[41][32]
\end{Arabic}}
\flushleft{\begin{hindi}
(ये) बख्शने वाले मेहरबान (ख़ुदा) की तरफ़ से (तुम्हारी मेहमानी है)
\end{hindi}}
\flushright{\begin{Arabic}
\quranayah[41][33]
\end{Arabic}}
\flushleft{\begin{hindi}
और इस से बेहतर किसकी बात हो सकती है जो (लोगों को) ख़ुदा की तरफ बुलाए और अच्छे अच्छे काम करे और कहे कि मैं भी यक़ीनन (ख़ुदा के) फरमाबरदार बन्दों में हूं
\end{hindi}}
\flushright{\begin{Arabic}
\quranayah[41][34]
\end{Arabic}}
\flushleft{\begin{hindi}
और भलाई बुराई (कभी) बराबर नहीं हो सकती तो (सख्त कलामी का) ऐसे तरीके से जवाब दो जो निहायत अच्छा हो (ऐसा करोगे) तो (तुम देखोगे) जिस में और तुममें दुशमनी थी गोया वह तुम्हारा दिल सोज़ दोस्त है
\end{hindi}}
\flushright{\begin{Arabic}
\quranayah[41][35]
\end{Arabic}}
\flushleft{\begin{hindi}
ये बात बस उन्हीं लोगों को हासिल हुईहै जो सब्र करने वाले हैं और उन्हीं लोगों को हासिल होती है जो बड़े नसीबवर हैं
\end{hindi}}
\flushright{\begin{Arabic}
\quranayah[41][36]
\end{Arabic}}
\flushleft{\begin{hindi}
और अगर तुम्हें शैतान की तरफ से वसवसा पैदा हो तो ख़ुदा की पनाह माँग लिया करो बेशक वह (सबकी) सुनता जानता है
\end{hindi}}
\flushright{\begin{Arabic}
\quranayah[41][37]
\end{Arabic}}
\flushleft{\begin{hindi}
और उसकी (कुदरत की) निशानियों में से रात और दिन और सूरज और चाँद हैं तो तुम लोग न सूरज को सजदा करो और न चाँद को, और अगर तुम ख़ुदा ही की इबादत करनी मंज़ूर रहे तो बस उसी को सजदा करो जिसने इन चीज़ों को पैदा किया है
\end{hindi}}
\flushright{\begin{Arabic}
\quranayah[41][38]
\end{Arabic}}
\flushleft{\begin{hindi}
पस अगर ये लोग सरकशी करें तो (ख़ुदा को भी उनकी परवाह नहीं) वो लोग (फ़रिश्ते) तुम्हारे परवरदिगार की बारगाह में हैं वह रात दिन उसकी तसबीह करते रहते हैं और वह लोग उकताते भी नहीं
\end{hindi}}
\flushright{\begin{Arabic}
\quranayah[41][39]
\end{Arabic}}
\flushleft{\begin{hindi}
उसकी क़ुदरत की निशानियों में से एक ये भी है कि तुम ज़मीन को ख़़ुश्क और बेआब ओ गयाह देखते हो फिर जब हम उस पर पानी बरसा देते हैं तो लहलहाने लगती है और फूल जाती है जिस ख़ुदा ने (मुर्दा) ज़मीन को ज़िन्दा किया वह यक़ीनन मुर्दों को भी जिलाएगा बेशक वह हर चीज़ पर क़ादिर है
\end{hindi}}
\flushright{\begin{Arabic}
\quranayah[41][40]
\end{Arabic}}
\flushleft{\begin{hindi}
जो लोग हमारी आयतों में हेर फेर पैदा करते हैं वह हरगिज़ हमसे पोशीदा नहीं हैं भला जो शख्स दोज़ख़ में डाला जाएगा वह बेहतर है या वह शख्स जो क़यामत के दिन बेख़ौफ व ख़तर आएगा (ख़ैर) जो चाहो सो करो (मगर) जो कुछ तुम करते हो वह (ख़ुदा) उसको देख रहा है
\end{hindi}}
\flushright{\begin{Arabic}
\quranayah[41][41]
\end{Arabic}}
\flushleft{\begin{hindi}
जिन लोगों ने नसीहत को जब वह उनके पास आयी न माना (वह अपना नतीजा देख लेंगे) और ये क़ुरान तो यक़ीनी एक आली मरतबा किताब है
\end{hindi}}
\flushright{\begin{Arabic}
\quranayah[41][42]
\end{Arabic}}
\flushleft{\begin{hindi}
कि झूठ न तो उसके आगे फटक सकता है और न उसके पीछे से और खूबियों वाले दाना (ख़ुदा) की बारगाह से नाज़िल हुई है
\end{hindi}}
\flushright{\begin{Arabic}
\quranayah[41][43]
\end{Arabic}}
\flushleft{\begin{hindi}
(ऐ रसूल) तुमसे से भी बस वही बातें कहीं जाती हैं जो तुमसे और रसूलों से कही जा चुकी हैं बेशक तुम्हारा परवरदिगार बख्शने वाला भी है और दर्दनाक अज़ाब वाला भी है
\end{hindi}}
\flushright{\begin{Arabic}
\quranayah[41][44]
\end{Arabic}}
\flushleft{\begin{hindi}
और अगर हम इस क़ुरान को अरबी ज़बान के सिवा दूसरी ज़बान में नाज़िल करते तो ये लोग ज़रूर कह न बैठते कि इसकी आयतें (हमारी) ज़बान में क्यों तफ़सीलदार बयान नहीं की गयी क्या (खूब क़ुरान तो) अजमी और (मुख़ातिब) अरबी (ऐ रसूल) तुम कह दो कि ईमानदारों के लिए तो ये (कुरान अज़सरतापा) हिदायत और (हर मर्ज़ की) शिफ़ा है और जो लोग ईमान नहीं रखते उनके कानों (के हक़) में गिरानी (बहरापन) है और वह (कुरान) उनके हक़ में नाबीनाई (का सबब) है तो गिरानी की वजह से गोया वह लोग बड़ी दूर की जगह से पुकारे जाते है
\end{hindi}}
\flushright{\begin{Arabic}
\quranayah[41][45]
\end{Arabic}}
\flushleft{\begin{hindi}
(और नहीं सुनते) और हम ही ने मूसा को भी किताब (तौरैत) अता की थी तो उसमें भी इसमें एख्तेलाफ किया गया और अगर तुम्हारे परवरदिगार की तरफ से एक बात पहले न हो चुकी होती तो उनमें कब का फैसला कर दिया गया होता, और ये लोग ऐसे शक़ में पड़े हुए हैं जिसने उन्हें बेचैन कर दिया है
\end{hindi}}
\flushright{\begin{Arabic}
\quranayah[41][46]
\end{Arabic}}
\flushleft{\begin{hindi}
जिसने अच्छे अच्छे काम किये तो अपने नफे क़े लिए और जो बुरा काम करे उसका वबाल भी उसी पर है और तुम्हारा परवरदिगार तो बन्दों पर (कभी) ज़ुल्म करने वाला नहीं
\end{hindi}}
\flushright{\begin{Arabic}
\quranayah[41][47]
\end{Arabic}}
\flushleft{\begin{hindi}
क़यामत के इल्म का हवाला उसी की तरफ है (यानि वही जानता है) और बगैर उसके इल्म व (इरादे) के न तो फल अपने पौरों से निकलते हैं और न किसी औरत को हमल रखता है और न वह बच्चा जनती है और जिस दिन (ख़ुदा) उन (मुशरेकीन) को पुकारेगा और पूछेगा कि मेरे शरीक कहाँ हैं- वह कहेंगे हम तो तुझ से अर्ज़ कर चूके हैं कि हम में से कोई (उनसे) वाकिफ़ ही नहीं
\end{hindi}}
\flushright{\begin{Arabic}
\quranayah[41][48]
\end{Arabic}}
\flushleft{\begin{hindi}
और इससे पहले जिन माबूदों की परसतिश करते थे वह ग़ायब हो गये और ये लोग समझ जाएगें कि उनके लिए अब मुख़लिसी नहीं
\end{hindi}}
\flushright{\begin{Arabic}
\quranayah[41][49]
\end{Arabic}}
\flushleft{\begin{hindi}
इन्सान भलाई की दुआए मांगने से तो कभी उकताता नहीं और अगर उसको कोई तकलीफ पहुँच जाए तो (फौरन) न उम्मीद और बेआस हो जाता है
\end{hindi}}
\flushright{\begin{Arabic}
\quranayah[41][50]
\end{Arabic}}
\flushleft{\begin{hindi}
और अगर उसको कोई तकलीफ पहुँच जाने के बाद हम उसको अपनी रहमत का मज़ा चखाएँ तो यक़ीनी कहने लगता है कि ये तो मेरे लिए ही है और मैं नहीं ख़याल करता कि कभी क़यामत बरपा होगी और अगर (क़यामत हो भी और) मैं अपने परवरदिगार की तरफ़ लौटाया भी जाऊँ तो भी मेरे लिए यक़ीनन उसके यहाँ भलाई ही तो है जो आमाल करते रहे हम उनको (क़यामत में) ज़रूर बता देंगें और उनको सख्त अज़ाब का मज़ा चख़ाएगें
\end{hindi}}
\flushright{\begin{Arabic}
\quranayah[41][51]
\end{Arabic}}
\flushleft{\begin{hindi}
(वह अलग) और जब हम इन्सान पर एहसान करते हैं तो (हमारी तरफ से) मुँह फेर लेता है और मुँह बदलकर चल देता है और जब उसे तकलीफ़ पहुँचती है तो लम्बी चौड़ी दुआएँ करने लगता है
\end{hindi}}
\flushright{\begin{Arabic}
\quranayah[41][52]
\end{Arabic}}
\flushleft{\begin{hindi}
(ऐ रसूल) तुम कहो कि भला देखो तो सही कि अगर ये (क़ुरान) ख़ुदा की बारगाह से (आया) हो और फिर तुम उससे इन्कार करो तो जो (ऐसे) परले दर्जे की मुख़ालेफत में (पड़ा) हो उससे बढ़कर और कौन गुमराह हो सकता है
\end{hindi}}
\flushright{\begin{Arabic}
\quranayah[41][53]
\end{Arabic}}
\flushleft{\begin{hindi}
हम अनक़रीब ही अपनी (क़ुदरत) की निशानियाँ अतराफ (आलम) में और ख़़ुद उनमें भी दिखा देगें यहाँ तक कि उन पर ज़ाहिर हो जाएगा कि वही यक़ीनन हक़ है क्या तुम्हारा परवरदिगार इसके लिए काफी नहीं कि वह हर चीज़ पर क़ाबू रखता है
\end{hindi}}
\flushright{\begin{Arabic}
\quranayah[41][54]
\end{Arabic}}
\flushleft{\begin{hindi}
देखो ये लोग अपने परवरदिगार के रूबरू हाज़िर होने से शक़ में (पड़े) हैं सुन रखो वह हर चीज़ पर हावी है
\end{hindi}}
\chapter{Ash-Shura (Counsel)}
\begin{Arabic}
\Huge{\centerline{\basmalah}}\end{Arabic}
\flushright{\begin{Arabic}
\quranayah[42][1]
\end{Arabic}}
\flushleft{\begin{hindi}
हा मीम
\end{hindi}}
\flushright{\begin{Arabic}
\quranayah[42][2]
\end{Arabic}}
\flushleft{\begin{hindi}
ऐन सीन काफ़
\end{hindi}}
\flushright{\begin{Arabic}
\quranayah[42][3]
\end{Arabic}}
\flushleft{\begin{hindi}
(ऐ रसूल) ग़ालिब व दाना ख़ुदा तुम्हारी तरफ़ और जो (पैग़म्बर) तुमसे पहले गुज़रे उनकी तरफ यूँ ही वही भेजता रहता है जो कुछ आसमानों में है और जो कुछ ज़मीन में है ग़रज़ सब कुछ उसी का है
\end{hindi}}
\flushright{\begin{Arabic}
\quranayah[42][4]
\end{Arabic}}
\flushleft{\begin{hindi}
और वह तो (बड़ा) आलीशान (और) बुर्ज़ुग है
\end{hindi}}
\flushright{\begin{Arabic}
\quranayah[42][5]
\end{Arabic}}
\flushleft{\begin{hindi}
(उनकी बातों से) क़रीब है कि सारे आसमान (उसकी हैबत के मारे) अपने ऊपर वार से फट पड़े और फ़रिश्ते तो अपने परवरदिगार की तारीफ़ के साथ तसबीह करते हैं और जो लोग ज़मीन में हैं उनके लिए (गुनाहों की) माफी माँगा करते हैं सुन रखो कि ख़ुदा ही यक़ीनन बड़ा बख्शने वाला मेहरबान है
\end{hindi}}
\flushright{\begin{Arabic}
\quranayah[42][6]
\end{Arabic}}
\flushleft{\begin{hindi}
और जिन लोगों ने ख़ुदा को छोड़ कर (और) अपने सरपरस्त बना रखे हैं ख़ुदा उनकी निगरानी कर रहा है (ऐ रसूल) तुम उनके निगेहबान नहीं हो
\end{hindi}}
\flushright{\begin{Arabic}
\quranayah[42][7]
\end{Arabic}}
\flushleft{\begin{hindi}
और हमने तुम्हारे पास अरबी क़ुरान यूँ भेजा ताकि तुम मक्का वालों को और जो लोग इसके इर्द गिर्द रहते हैं उनको डराओ और (उनको) क़यामत के दिन से भी डराओ जिस (के आने) में कुछ भी शक़ नहीं (उस दिन) एक फरीक़ (मानने वाला) जन्नत में होगा और फरीक़ (सानी) दोज़ख़ में
\end{hindi}}
\flushright{\begin{Arabic}
\quranayah[42][8]
\end{Arabic}}
\flushleft{\begin{hindi}
और अगर ख़ुदा चाहता तो इन सबको एक ही गिरोह बना देता मगर वह तो जिसको चाहता है (हिदायत करके) अपनी रहमत में दाख़िल कर लेता है और ज़ालिमों का तो (उस दिन) न कोई यार है और न मददगार
\end{hindi}}
\flushright{\begin{Arabic}
\quranayah[42][9]
\end{Arabic}}
\flushleft{\begin{hindi}
क्या उन लोगों ने ख़ुदा के सिवा (दूसरे) कारसाज़ बनाए हैं तो कारसाज़ बस ख़ुदा ही है और वही मुर्दों को ज़िन्दा करेगा और वही हर चीज़ पर क़ुदरत रखता है
\end{hindi}}
\flushright{\begin{Arabic}
\quranayah[42][10]
\end{Arabic}}
\flushleft{\begin{hindi}
और तुम लोग जिस चीज़ में बाहम एख्तेलाफ़ात रखते हो उसका फैसला ख़ुदा ही के हवाले है वही ख़ुदा तो मेरा परवरदिगार है मैं उसी पर भरोसा रखता हूँ और उसी की तरफ रूजू करता हूँ
\end{hindi}}
\flushright{\begin{Arabic}
\quranayah[42][11]
\end{Arabic}}
\flushleft{\begin{hindi}
सारे आसमान व ज़मीन का पैदा करने वाला (वही) है उसी ने तुम्हारे लिए तुम्हारी ही जिन्स के जोड़े बनाए और चारपायों के जोड़े भी (उसी ने बनाए) उस (तरफ) में तुमको फैलाता रहता है कोई चीज़ उसकी मिसल नहीं और वह हर चीज़ को सुनता देखता है
\end{hindi}}
\flushright{\begin{Arabic}
\quranayah[42][12]
\end{Arabic}}
\flushleft{\begin{hindi}
सारे आसमान व ज़मीन की कुन्जियाँ उसके पास हैं जिसके लिए चाहता है रोज़ी को फराख़ कर देता है (जिसके लिए) चाहता है तंग कर देता है बेशक वह हर चीज़ से ख़ूब वाक़िफ़ है
\end{hindi}}
\flushright{\begin{Arabic}
\quranayah[42][13]
\end{Arabic}}
\flushleft{\begin{hindi}
उसने तुम्हारे लिए दीन का वही रास्ता मुक़र्रर किया जिस (पर चलने का) नूह को हुक्म दिया था और (ऐ रसूल) उसी की हमने तुम्हारे पास वही भेजी है और उसी का इब्राहीम और मूसा और ईसा को भी हुक्म दिया था (वह) ये (है कि) दीन को क़ायम रखना और उसमें तफ़रक़ा न डालना जिस दीन की तरफ तुम मुशरेकीन को बुलाते हो वह उन पर बहुत शाक़ ग़ुज़रता है ख़ुदा जिसको चाहता है अपनी बारगाह का बरगुज़ीदा कर लेता है और जो उसकी तरफ रूजू करे (अपनी तरफ़ (पहुँचने) का रास्ता दिखा देता है
\end{hindi}}
\flushright{\begin{Arabic}
\quranayah[42][14]
\end{Arabic}}
\flushleft{\begin{hindi}
और ये लोग मुतफ़र्रिक़ हुए भी तो इल्म (हक़) आ चुकने के बाद और (वह भी) महज़ आपस की ज़िद से और अगर तुम्हारे परवरदिगार की तरफ़ से एक वक्ते मुक़र्रर तक के लिए (क़यामत का) वायदा न हो चुका होता तो उनमें कबका फैसला हो चुका होता और जो लोग उनके बाद (ख़ुदा की) किताब के वारिस हुए वह उसकी तरफ से बहुत सख्त शुबहे में (पड़े हुए) हैं
\end{hindi}}
\flushright{\begin{Arabic}
\quranayah[42][15]
\end{Arabic}}
\flushleft{\begin{hindi}
तो (ऐ रसूल) तुम (लोगों को) उसी (दीन) की तरफ बुलाते रहे जो और जैसा तुमको हुक्म हुआ है (उसी पर क़ायम रहो और उनकी नफ़सियानी ख्वाहिशों की पैरवी न करो और साफ़ साफ़ कह दो कि जो किताब ख़ुदा ने नाज़िल की है उस पर मैं ईमान रखता हूँ और मुझे हुक्म हुआ है कि मैं तुम्हारे एख्तेलाफात के (दरमेयान) इन्साफ़ (से फ़ैसला) करूँ ख़ुदा ही हमारा भी परवरदिगार है और वही तुम्हारा भी परवरदिगार है हमारी कारगुज़ारियाँ हमारे ही लिए हैं और तुम्हारी कारस्तानियाँ तुम्हारे वास्ते हममें और तुममें तो कुछ हुज्जत (व तक़रार की ज़रूरत) नहीं ख़ुदा ही हम (क़यामत में) सबको इकट्ठा करेगा
\end{hindi}}
\flushright{\begin{Arabic}
\quranayah[42][16]
\end{Arabic}}
\flushleft{\begin{hindi}
और उसी की तरफ लौट कर जाना है और जो लोग उसके मान लिए जाने के बाद ख़ुदा के बारे में (ख्वाहमख्वाह) झगड़ा करते हैं उनके परवरदिगार के नज़दीक उनकी दलील लग़ो बातिल है और उन पर (ख़ुदा का) ग़ज़ब और उनके लिए सख्त अज़ाब है
\end{hindi}}
\flushright{\begin{Arabic}
\quranayah[42][17]
\end{Arabic}}
\flushleft{\begin{hindi}
ख़ुदा ही तो है जिसने सच्चाई के साथ किताब नाज़िल की और अदल (व इन्साफ़ भी नाज़िल किया) और तुमको क्या मालूम यायद क़यामत क़रीब ही हो
\end{hindi}}
\flushright{\begin{Arabic}
\quranayah[42][18]
\end{Arabic}}
\flushleft{\begin{hindi}
(फिर ये ग़फ़लत कैसी) जो लोग इस पर ईमान नहीं रखते वह तो इसके लिए जल्दी कर रहे हैं और जो मोमिन हैं वह उससे डरते हैं और जानते हैं कि क़यामत यक़ीनी बरहक़ है आगाह रहो कि जो लोग क़यामत के बारे में शक़ किया करते हैं वह बड़े परले दर्जे की गुमराही में हैं
\end{hindi}}
\flushright{\begin{Arabic}
\quranayah[42][19]
\end{Arabic}}
\flushleft{\begin{hindi}
और ख़ुदा अपने बन्दों (के हाल) पर बड़ा मेहरबान है जिसको (जितनी) रोज़ी चाहता है देता है वह ज़ोर वाला ज़बरदस्त है
\end{hindi}}
\flushright{\begin{Arabic}
\quranayah[42][20]
\end{Arabic}}
\flushleft{\begin{hindi}
जो शख़्श आखेरत की खेती का तालिब हो हम उसके लिए उसकी खेती में अफ़ज़ाइश करेंगे और दुनिया की खेती का ख़ास्तगार हो तो हम उसको उसी में से देंगे मगर आखेरत में फिर उसका कुछ हिस्सा न होगा
\end{hindi}}
\flushright{\begin{Arabic}
\quranayah[42][21]
\end{Arabic}}
\flushleft{\begin{hindi}
क्या उन लोगों के (बनाए हुए) ऐसे शरीक हैं जिन्होंने उनके लिए ऐसा दीन मुक़र्रर किया है जिसकी ख़ुदा ने इजाज़त नहीं दी और अगर फ़ैसले (के दिन) का वायदा न होता तो उनमें यक़ीनी अब तक फैसला हो चुका होता और ज़ालिमों के वास्ते ज़रूर दर्दनाक अज़ाब है
\end{hindi}}
\flushright{\begin{Arabic}
\quranayah[42][22]
\end{Arabic}}
\flushleft{\begin{hindi}
(क़यामत के दिन) देखोगे कि ज़ालिम लोग अपने आमाल (के वबाल) से डर रहे होंगे और वह उन पर पड़ कर रहेगा और जिन्होने ईमान क़ुबूल किया और अच्छे काम किए वह बेहिश्त के बाग़ों में होंगे वह जो कुछ चाहेंगे उनके लिए उनके परवरदिगार की बारगाह में (मौजूद) है यही तो (ख़ुदा का) बड़ा फज़ल है
\end{hindi}}
\flushright{\begin{Arabic}
\quranayah[42][23]
\end{Arabic}}
\flushleft{\begin{hindi}
यही (ईनाम) है जिसकी ख़ुदा अपने उन बन्दों को ख़ुशख़बरी देता है जो ईमान लाए और नेक काम करते रहे (ऐ रसूल) तुम कह दो कि मैं इस (तबलीग़े रिसालत) का अपने क़रातबदारों (अहले बैत) की मोहब्बत के सिवा तुमसे कोई सिला नहीं मांगता और जो शख़्श नेकी हासिल करेगा हम उसके लिए उसकी ख़ूबी में इज़ाफा कर देंगे बेशक वह बड़ा बख्शने वाला क़दरदान है
\end{hindi}}
\flushright{\begin{Arabic}
\quranayah[42][24]
\end{Arabic}}
\flushleft{\begin{hindi}
क्या ये लोग (तुम्हारी निस्बत कहते हैं कि इस (रसूल) ने ख़ुदा पर झूठा बोहतान बाँधा है तो अगर (ऐसा) होता तो) ख़ुदा चाहता तो तुम्हारे दिल पर मोहर लगा देता (कि तुम बात ही न कर सकते) और ख़ुदा तो झूठ को नेस्तनाबूद और अपनी बातों से हक़ को साबित करता है वह यक़ीनी दिलों के राज़ से ख़ूब वाक़िफ है
\end{hindi}}
\flushright{\begin{Arabic}
\quranayah[42][25]
\end{Arabic}}
\flushleft{\begin{hindi}
और वही तो है जो अपने बन्दों की तौबा क़ुबूल करता है और गुनाहों को माफ़ करता है और तुम लोग जो कुछ भी करते हो वह जानता है
\end{hindi}}
\flushright{\begin{Arabic}
\quranayah[42][26]
\end{Arabic}}
\flushleft{\begin{hindi}
और जो लोग ईमान लाए और अच्छे अच्छे काम करते रहे उनकी (दुआ) क़ुबूल करता है फज़ल व क़रम से उनको बढ़ कर देता है और काफिरों के लिए सख्त अज़ाब है
\end{hindi}}
\flushright{\begin{Arabic}
\quranayah[42][27]
\end{Arabic}}
\flushleft{\begin{hindi}
और अगर ख़ुदा ने अपने बन्दों की रोज़ी में फराख़ी कर दे तो वह लोग ज़रूर (रूए) ज़मीन से सरकशी करने लगें मगर वह तो बाक़दरे मुनासिब जिसकी रोज़ी (जितनी) चाहता है नाज़िल करता है वह बेशक अपने बन्दों से ख़बरदार (और उनको) देखता है
\end{hindi}}
\flushright{\begin{Arabic}
\quranayah[42][28]
\end{Arabic}}
\flushleft{\begin{hindi}
और वही तो है जो लोगों के नाउम्मीद हो जाने के बाद मेंह बरसाता है और अपनी रहमत (बारिश की बरकतों) को फैला देता है और वही कारसाज़ (और) हम्द व सना के लायक़ है
\end{hindi}}
\flushright{\begin{Arabic}
\quranayah[42][29]
\end{Arabic}}
\flushleft{\begin{hindi}
और उसी की (क़ुदरत की) निशानियों में से सारे आसमान व ज़मीन का पैदा करना और उन जानदारों का भी जो उसने आसमान व ज़मीन में फैला रखे हैं और जब चाहे उनके जमा कर लेने पर (भी) क़ादिर है
\end{hindi}}
\flushright{\begin{Arabic}
\quranayah[42][30]
\end{Arabic}}
\flushleft{\begin{hindi}
और जो मुसीबत तुम पर पड़ती है वह तुम्हारे अपने ही हाथों की करतूत से और (उस पर भी) वह बहुत कुछ माफ कर देता है
\end{hindi}}
\flushright{\begin{Arabic}
\quranayah[42][31]
\end{Arabic}}
\flushleft{\begin{hindi}
और तुम लोग ज़मीन में (रह कर) तो ख़ुदा को किसी तरह हरा नहीं सकते और ख़ुदा के सिवा तुम्हारा न कोई दोस्त है और न मददगार
\end{hindi}}
\flushright{\begin{Arabic}
\quranayah[42][32]
\end{Arabic}}
\flushleft{\begin{hindi}
और उसी की (क़ुदरत) की निशानियों में से समन्दर में (चलने वाले) (बादबानी जहाज़) है जो गोया पहाड़ हैं
\end{hindi}}
\flushright{\begin{Arabic}
\quranayah[42][33]
\end{Arabic}}
\flushleft{\begin{hindi}
अगर ख़ुदा चाहे तो हवा को ठहरा दे तो जहाज़ भी समन्दर की सतह पर (खड़े के खड़े) रह जाएँ बेशक तमाम सब्र और शुक्र करने वालों के वास्ते इन बातों में (ख़ुदा की क़ुदरत की) बहुत सी निशानियाँ हैं
\end{hindi}}
\flushright{\begin{Arabic}
\quranayah[42][34]
\end{Arabic}}
\flushleft{\begin{hindi}
(या वह चाहे तो) उनको उनके आमाल (बद) के सबब तबाह कर दे
\end{hindi}}
\flushright{\begin{Arabic}
\quranayah[42][35]
\end{Arabic}}
\flushleft{\begin{hindi}
और वह बहुत कुछ माफ़ करता है और जो लोग हमारी निशानियों में (ख्वाहमाख्वाह) झगड़ा करते हैं वह अच्छी तरह समझ लें कि उनको किसी तरह (अज़ाब से) छुटकारा नहीं
\end{hindi}}
\flushright{\begin{Arabic}
\quranayah[42][36]
\end{Arabic}}
\flushleft{\begin{hindi}
(लोगों) तुमको जो कुछ (माल) दिया गया है वह दुनिया की ज़िन्दगी का (चन्द रोज़) साज़ोसामान है और जो कुछ ख़ुदा के यहाँ है वह कहीं बेहतर और पायदार है (मगर ये) ख़ास उन ही लोगों के लिए है जो ईमान लाए और अपने परवरदिगार पर भरोसा रखते हैं
\end{hindi}}
\flushright{\begin{Arabic}
\quranayah[42][37]
\end{Arabic}}
\flushleft{\begin{hindi}
और जो लोग बड़े बड़े गुनाहों और बेहयाई की बातों से बचे रहते हैं और ग़ुस्सा आ जाता है तो माफ कर देते हैं
\end{hindi}}
\flushright{\begin{Arabic}
\quranayah[42][38]
\end{Arabic}}
\flushleft{\begin{hindi}
और जो अपने परवरदिगार का हुक्म मानते हैं और नमाज़ पढ़ते हैं और उनके कुल काम आपस के मशवरे से होते हैं और जो कुछ हमने उन्हें अता किया है उसमें से (राहे ख़ुदा में) ख़र्च करते हैं
\end{hindi}}
\flushright{\begin{Arabic}
\quranayah[42][39]
\end{Arabic}}
\flushleft{\begin{hindi}
और (वह ऐसे हैं) कि जब उन पर किसी किस्म की ज्यादती की जाती है तो बस वाजिबी बदला ले लेते हैं
\end{hindi}}
\flushright{\begin{Arabic}
\quranayah[42][40]
\end{Arabic}}
\flushleft{\begin{hindi}
और बुराई का बदला तो वैसी ही बुराई है उस पर भी जो शख्स माफ कर दे और (मामले की) इसलाह कर दें तो इसका सवाब ख़ुदा के ज़िम्मे है बेशक वह ज़ुल्म करने वालों को पसन्द नहीं करता
\end{hindi}}
\flushright{\begin{Arabic}
\quranayah[42][41]
\end{Arabic}}
\flushleft{\begin{hindi}
और जिस पर ज़ुल्म हुआ हो अगर वह उसके बाद इन्तेक़ाम ले तो ऐसे लोगों पर कोई इल्ज़ाम नहीं
\end{hindi}}
\flushright{\begin{Arabic}
\quranayah[42][42]
\end{Arabic}}
\flushleft{\begin{hindi}
इल्ज़ाम तो बस उन्हीं लोगों पर होगा जो लोगों पर ज़ुल्म करते हैं और रूए ज़मीन में नाहक़ ज्यादतियाँ करते फिरते हैं उन्हीं लोगों के लिए दर्दनाक अज़ाब है
\end{hindi}}
\flushright{\begin{Arabic}
\quranayah[42][43]
\end{Arabic}}
\flushleft{\begin{hindi}
और जो सब्र करे और कुसूर माफ़ कर दे तो बेशक ये बड़े हौसले के काम हैं
\end{hindi}}
\flushright{\begin{Arabic}
\quranayah[42][44]
\end{Arabic}}
\flushleft{\begin{hindi}
और जिसको ख़ुदा गुमराही में छोड़ दे तो उसके बाद उसका कोई सरपरस्त नहीं और तुम ज़ालिमों को देखोगे कि जब (दोज़ख़) का अज़ाब देखेंगे तो कहेंगे कि भला (दुनिया में) फिर लौट कर जाने की कोई सबील है
\end{hindi}}
\flushright{\begin{Arabic}
\quranayah[42][45]
\end{Arabic}}
\flushleft{\begin{hindi}
और तुम उनको देखोगे कि दोज़ख़ के सामने लाए गये हैं (और) ज़िल्लत के मारे कटे जाते हैं (और) कनक्खियों से देखे जाते हैं और मोमिनीन कहेंगे कि हकीक़त में वही बड़े घाटे में हैं जिन्होने क़यामत के दिन अपने आप को और अपने घर वालों को ख़सारे में डाला देखो ज़ुल्म करने वाले दाएमी अज़ाब में रहेंगे
\end{hindi}}
\flushright{\begin{Arabic}
\quranayah[42][46]
\end{Arabic}}
\flushleft{\begin{hindi}
और ख़ुदा के सिवा न उनके सरपरस्त ही होंगे जो उनकी मदद को आएँ और जिसको ख़ुदा गुमराही में छोड़ दे तो उसके लिए (हिदायत की) कोई राह नहीं
\end{hindi}}
\flushright{\begin{Arabic}
\quranayah[42][47]
\end{Arabic}}
\flushleft{\begin{hindi}
(लोगों) उस दिन के पहले जो ख़ुदा की तरफ से आयेगा और किसी तरह (टाले न टलेगा) अपने परवरदिगार का हुक्म मान लो (क्यों कि) उस दिन न तो तुमको कहीं पनाह की जगह मिलेगी और न तुमसे (गुनाह का) इन्कार ही बन पड़ेगा
\end{hindi}}
\flushright{\begin{Arabic}
\quranayah[42][48]
\end{Arabic}}
\flushleft{\begin{hindi}
फिर अगर मुँह फेर लें तो (ऐ रसूल) हमने तुमको उनका निगेहबान बनाकर नहीं भेजा तुम्हारा काम तो सिर्फ (एहकाम का) पहुँचा देना है और जब हम इन्सान को अपनी रहमत का मज़ा चखाते हैं तो वह उससे ख़ुश हो जाता है और अगर उनको उन्हीं के हाथों की पहली करतूतों की बदौलत कोई तकलीफ पहुँचती (सब एहसान भूल गए) बेशक इन्सान बड़ा नाशुक्रा है
\end{hindi}}
\flushright{\begin{Arabic}
\quranayah[42][49]
\end{Arabic}}
\flushleft{\begin{hindi}
सारे आसमान व ज़मीन की हुकूमत ख़ास ख़ुदा ही की है जो चाहता है पैदा करता है (और) जिसे चाहता है (फ़क़त) बेटियाँ देता है और जिसे चाहता है (महज़) बेटा अता करता है
\end{hindi}}
\flushright{\begin{Arabic}
\quranayah[42][50]
\end{Arabic}}
\flushleft{\begin{hindi}
या उनको बेटे बेटियाँ (औलाद की) दोनों किस्में इनायत करता है और जिसको चाहता है बांझ बना देता है बेशक वह बड़ा वाकिफ़कार क़ादिर है
\end{hindi}}
\flushright{\begin{Arabic}
\quranayah[42][51]
\end{Arabic}}
\flushleft{\begin{hindi}
और किसी आदमी के लिए ये मुमकिन नहीं कि ख़ुदा उससे बात करे मगर वही के ज़रिए से (जैसे) (दाऊद) परदे के पीछे से जैसे (मूसा) या कोई फ़रिश्ता भेज दे (जैसे मोहम्मद) ग़रज़ वह अपने एख्तेयार से जो चाहता है पैग़ाम भेज देता है बेशक वह आलीशान हिकमत वाला है
\end{hindi}}
\flushright{\begin{Arabic}
\quranayah[42][52]
\end{Arabic}}
\flushleft{\begin{hindi}
और इसी तरह हमने अपने हुक्म को रूह (क़ुरान) तुम्हारी तरफ 'वही' के ज़रिए से भेजे तो तुम न किताब ही को जानते थे कि क्या है और न ईमान को मगर इस (क़ुरान) को एक नूर बनाया है कि इससे हम अपने बन्दों में से जिसकी चाहते हैं हिदायत करते हैं और इसमें शक़ नहीं कि तुम (ऐ रसूल) सीधा ही रास्ता दिखाते हो
\end{hindi}}
\flushright{\begin{Arabic}
\quranayah[42][53]
\end{Arabic}}
\flushleft{\begin{hindi}
(यानि) उसका रास्ता कि जो आसमानों में है और जो कुछ ज़मीन में है (ग़रज़ सब कुछ) उसी का है सुन रखो सब काम ख़ुदा ही की तरफ रूजू होंगे और वही फैसला करेगा
\end{hindi}}
\chapter{Az-Zukhruf (Gold)}
\begin{Arabic}
\Huge{\centerline{\basmalah}}\end{Arabic}
\flushright{\begin{Arabic}
\quranayah[43][1]
\end{Arabic}}
\flushleft{\begin{hindi}
हा मीम
\end{hindi}}
\flushright{\begin{Arabic}
\quranayah[43][2]
\end{Arabic}}
\flushleft{\begin{hindi}
रौशन किताब (क़ुरान) की क़सम
\end{hindi}}
\flushright{\begin{Arabic}
\quranayah[43][3]
\end{Arabic}}
\flushleft{\begin{hindi}
हमने इस किताब को अरबी ज़बान कुरान ज़रूर बनाया है ताकि तुम समझो
\end{hindi}}
\flushright{\begin{Arabic}
\quranayah[43][4]
\end{Arabic}}
\flushleft{\begin{hindi}
और बेशक ये (क़ुरान) असली किताब (लौह महफूज़) में (भी जो) मेरे पास है लिखी हुई है (और) यक़ीनन बड़े रूतबे की (और) पुरअज़ हिकमत है
\end{hindi}}
\flushright{\begin{Arabic}
\quranayah[43][5]
\end{Arabic}}
\flushleft{\begin{hindi}
भला इस वजह से कि तुम ज्यादती करने वाले लोग हो हम तुमको नसीहत करने से मुँह मोड़ेंगे (हरगिज़ नहीं)
\end{hindi}}
\flushright{\begin{Arabic}
\quranayah[43][6]
\end{Arabic}}
\flushleft{\begin{hindi}
और हमने अगले लोगों को बहुत से पैग़म्बर भेजे थे
\end{hindi}}
\flushright{\begin{Arabic}
\quranayah[43][7]
\end{Arabic}}
\flushleft{\begin{hindi}
और कोई पैग़म्बर उनके पास ऐसा नहीं आया जिससे इन लोगों ने ठट्ठे नहीं किए हो
\end{hindi}}
\flushright{\begin{Arabic}
\quranayah[43][8]
\end{Arabic}}
\flushleft{\begin{hindi}
तो उनमें से जो ज्यादा ज़ोरावर थे तो उनको हमने हलाक कर मारा और (दुनिया में) अगलों के अफ़साने जारी हो गए
\end{hindi}}
\flushright{\begin{Arabic}
\quranayah[43][9]
\end{Arabic}}
\flushleft{\begin{hindi}
और (ऐ रसूल) अगर तुम उनसे पूछो कि सारे आसमान व ज़मीन को किसने पैदा किया तो वह ज़रूर कह देंगे कि उनको बड़े वाक़िफ़कार ज़बरदस्त (ख़ुदा ने) पैदा किया है
\end{hindi}}
\flushright{\begin{Arabic}
\quranayah[43][10]
\end{Arabic}}
\flushleft{\begin{hindi}
जिसने तुम लोगों के वास्ते ज़मीन का बिछौना बनाया और (फिर) उसमें तुम्हारे नफ़े के लिए रास्ते बनाए ताकि तुम राह मालूम करो
\end{hindi}}
\flushright{\begin{Arabic}
\quranayah[43][11]
\end{Arabic}}
\flushleft{\begin{hindi}
और जिसने एक (मुनासिब) अन्दाजे क़े साथ आसमान से पानी बरसाया फिर हम ही ने उसके (ज़रिए) से मुर्दा (परती) शहर को ज़िन्दा (आबाद) किया उसी तरह तुम भी (क़यामत के दिन क़ब्रों से) निकाले जाओगे
\end{hindi}}
\flushright{\begin{Arabic}
\quranayah[43][12]
\end{Arabic}}
\flushleft{\begin{hindi}
और जिसने हर किस्म की चीज़े पैदा कीं और तुम्हारे लिए कश्तियां बनायीं और चारपाए (पैदा किए) जिन पर तुम सवार होते हो
\end{hindi}}
\flushright{\begin{Arabic}
\quranayah[43][13]
\end{Arabic}}
\flushleft{\begin{hindi}
ताकि तुम उसकी पीठ पर चढ़ो और जब उस पर (अच्छी तरह) सीधे हो बैठो तो अपने परवरदिगार का एहसान माना करो और कहो कि वह (ख़ुदा हर ऐब से) पाक है जिसने इसको हमारा ताबेदार बनाया हालॉकि हम तो ऐसे (ताक़तवर) न थे कि उस पर क़ाबू पाते
\end{hindi}}
\flushright{\begin{Arabic}
\quranayah[43][14]
\end{Arabic}}
\flushleft{\begin{hindi}
और हमको तो यक़ीनन अपने परवरदिगार की तरफ लौट कर जाना है
\end{hindi}}
\flushright{\begin{Arabic}
\quranayah[43][15]
\end{Arabic}}
\flushleft{\begin{hindi}
और उन लोगों ने उसके बन्दों में से उसके लिए औलाद क़रार दी है इसमें शक़ नहीं कि इन्सान खुल्लम खुल्ला बड़ा ही नाशक्रा है
\end{hindi}}
\flushright{\begin{Arabic}
\quranayah[43][16]
\end{Arabic}}
\flushleft{\begin{hindi}
क्या उसने अपनी मख़लूक़ात में से ख़ुद तो बेटियाँ ली हैं और तुमको चुनकर बेटे दिए हैं
\end{hindi}}
\flushright{\begin{Arabic}
\quranayah[43][17]
\end{Arabic}}
\flushleft{\begin{hindi}
हालॉकि जब उनमें किसी शख़्श को उस चीज़ (बेटी) की ख़ुशख़बरी दी जाती है जिसकी मिसल उसने ख़ुदा के लिए बयान की है तो वह (ग़ुस्से के मारे) सियाह हो जाता है और ताव पेंच खाने लगता है
\end{hindi}}
\flushright{\begin{Arabic}
\quranayah[43][18]
\end{Arabic}}
\flushleft{\begin{hindi}
क्या वह (औरत) जो ज़ेवरों में पाली पोसी जाए और झगड़े में (अच्छी तरह) बात तक न कर सकें (ख़ुदा की बेटी हो सकती है)
\end{hindi}}
\flushright{\begin{Arabic}
\quranayah[43][19]
\end{Arabic}}
\flushleft{\begin{hindi}
और उन लोगों ने फ़रिश्तों को कि वह भी ख़ुदा के बन्दे हैं (ख़ुदा की) बेटियाँ बनायी हैं लोग फरिश्तों की पैदाइश क्यों खड़े देख रहे थे अभी उनकी शहादत क़लम बन्द कर ली जाती है
\end{hindi}}
\flushright{\begin{Arabic}
\quranayah[43][20]
\end{Arabic}}
\flushleft{\begin{hindi}
और (क़यामत) में उनसे बाज़पुर्स की जाएगी और कहते हैं कि अगर ख़ुदा चाहता तो हम उनकी परसतिश न करते उनको उसकी कुछ ख़बर ही नहीं ये लोग तो बस अटकल पच्चू बातें किया करते हैं
\end{hindi}}
\flushright{\begin{Arabic}
\quranayah[43][21]
\end{Arabic}}
\flushleft{\begin{hindi}
या हमने उनको उससे पहले कोई किताब दी थी कि ये लोग उसे मज़बूत थामें हुए हैं
\end{hindi}}
\flushright{\begin{Arabic}
\quranayah[43][22]
\end{Arabic}}
\flushleft{\begin{hindi}
बल्कि ये लोग तो ये कहते हैं कि हमने अपने बाप दादाओं को एक तरीके पर पाया और हम उनको क़दम ब क़दम ठीक रास्ते पर चले जा रहें हैं
\end{hindi}}
\flushright{\begin{Arabic}
\quranayah[43][23]
\end{Arabic}}
\flushleft{\begin{hindi}
और (ऐ रसूल) इसी तरह हमने तुमसे पहले किसी बस्ती में कोई डराने वाला (पैग़म्बर) नहीं भेजा मगर वहाँ के ख़ुशहाल लोगों ने यही कहा कि हमने अपने बाप दादाओं को एक तरीके पर पाया, और हम यक़ीनी उनके क़दम ब क़दम चले जा रहे हैं
\end{hindi}}
\flushright{\begin{Arabic}
\quranayah[43][24]
\end{Arabic}}
\flushleft{\begin{hindi}
(इस पर) उनके पैग़म्बर ने कहा भी जिस तरीक़े पर तुमने अपने बाप दादाओं को पाया अगरचे मैं तुम्हारे पास इससे बेहतर राहे रास्त पर लाने वाला दीन लेकर आया हूँ (तो भी न मानोगे) वह बोले (कुछ हो मगर) हम तो उस दीन को जो तुम देकर भेजे गए हो मानने वाले नहीं
\end{hindi}}
\flushright{\begin{Arabic}
\quranayah[43][25]
\end{Arabic}}
\flushleft{\begin{hindi}
तो हमने उनसे बदला लिया (तो ज़रा) देखो तो कि झुठलाने वालों का क्या अन्जाम हुआ
\end{hindi}}
\flushright{\begin{Arabic}
\quranayah[43][26]
\end{Arabic}}
\flushleft{\begin{hindi}
(और वह वख्त याद करो) जब इब्राहीम ने अपने (मुँह बोले) बाप (आज़र) और अपनी क़ौम से कहा कि जिन चीज़ों को तुम लोग पूजते हो मैं यक़ीनन उससे बेज़ार हूँ
\end{hindi}}
\flushright{\begin{Arabic}
\quranayah[43][27]
\end{Arabic}}
\flushleft{\begin{hindi}
मगर उसकी इबादत करता हूँ, जिसने मुझे पैदा किया तो वही बहुत जल्द मेरी हिदायत करेगा
\end{hindi}}
\flushright{\begin{Arabic}
\quranayah[43][28]
\end{Arabic}}
\flushleft{\begin{hindi}
और उसी (ईमान) को इब्राहीम ने अपनी औलाद में हमेशा बाक़ी रहने वाली बात छोड़ गए ताकि वह (ख़ुदा की तरफ रूजू) करें
\end{hindi}}
\flushright{\begin{Arabic}
\quranayah[43][29]
\end{Arabic}}
\flushleft{\begin{hindi}
बल्कि मैं उनको और उनके बाप दादाओं को फायदा पहुँचाता रहा यहाँ तक कि उनके पास (दीने) हक़ और साफ़ साफ़ बयान करने वाला रसूल आ पहुँचा
\end{hindi}}
\flushright{\begin{Arabic}
\quranayah[43][30]
\end{Arabic}}
\flushleft{\begin{hindi}
और जब उनके पास (दीन) हक़ आ गया तो कहने लगे ये तो जादू है और हम तो हरगिज़ इसके मानने वाले नहीं
\end{hindi}}
\flushright{\begin{Arabic}
\quranayah[43][31]
\end{Arabic}}
\flushleft{\begin{hindi}
और कहने लगे कि ये क़ुरान इन दो बस्तियों (मक्के ताएफ) में से किसी बड़े आदमी पर क्यों नहीं नाज़िल किया गया
\end{hindi}}
\flushright{\begin{Arabic}
\quranayah[43][32]
\end{Arabic}}
\flushleft{\begin{hindi}
ये लोग तुम्हारे परवरदिगार की रहमत को (अपने तौर पर) बाँटते हैं हमने तो इनके दरमियान उनकी रोज़ी दुनयावी ज़िन्दगी में बाँट ही दी है और एक के दूसरे पर दर्जे बुलन्द किए हैं ताकि इनमें का एक दूसरे से ख़िदमत ले और जो माल (मतआ) ये लोग जमा करते फिरते हैं ख़ुदा की रहमत (पैग़म्बर) इससे कहीं बेहतर है
\end{hindi}}
\flushright{\begin{Arabic}
\quranayah[43][33]
\end{Arabic}}
\flushleft{\begin{hindi}
और अगर ये बात न होती कि (आख़िर) सब लोग एक ही तरीक़े के हो जाएँगे तो हम उनके लिए जो ख़ुदा से इन्कार करते हैं उनके घरों की छतें और वही सीढ़ियाँ जिन पर वह चढ़ते हैं (उतरते हैं)
\end{hindi}}
\flushright{\begin{Arabic}
\quranayah[43][34]
\end{Arabic}}
\flushleft{\begin{hindi}
और उनके घरों के दरवाज़े और वह तख्त जिन पर तकिये लगाते हैं चाँदी और सोने के बना देते
\end{hindi}}
\flushright{\begin{Arabic}
\quranayah[43][35]
\end{Arabic}}
\flushleft{\begin{hindi}
ये सब साज़ो सामान, तो बस दुनियावी ज़िन्दगी के (चन्द रोज़ा) साज़ो सामान हैं (जो मिट जाएँगे) और आख़ेरत (का सामान) तो तुम्हारे परवरदिगार के यहॉ ख़ास परहेज़गारों के लिए है
\end{hindi}}
\flushright{\begin{Arabic}
\quranayah[43][36]
\end{Arabic}}
\flushleft{\begin{hindi}
और जो शख़्श ख़ुदा की चाह से अन्धा बनता है हम (गोया ख़ुद) उसके वास्ते शैतान मुक़र्रर कर देते हैं तो वही उसका (हर दम का) साथी है
\end{hindi}}
\flushright{\begin{Arabic}
\quranayah[43][37]
\end{Arabic}}
\flushleft{\begin{hindi}
और वह (शयातीन) उन लोगों को (ख़ुदा की) राह से रोकते रहते हैं बावजूद इसके वह उसी ख्याल में हैं कि वह यक़ीनी राहे रास्त पर हैं
\end{hindi}}
\flushright{\begin{Arabic}
\quranayah[43][38]
\end{Arabic}}
\flushleft{\begin{hindi}
यहाँ तक कि जब (क़यामत में) हमारे पास आएगा तो (अपने साथी शैतान से) कहेगा काश मुझमें और तुममें पूरब पश्चिम का फ़ासला होता ग़रज़ (शैतान भी) क्या ही बुरा रफीक़ है
\end{hindi}}
\flushright{\begin{Arabic}
\quranayah[43][39]
\end{Arabic}}
\flushleft{\begin{hindi}
और जब तुम नाफरमानियाँ कर चुके तो (शयातीन के साथ) तुम्हारा अज़ाब में शरीक होना भी आज तुमको (अज़ाब की कमी में) कोई फायदा नहीं पहुँचा सकता
\end{hindi}}
\flushright{\begin{Arabic}
\quranayah[43][40]
\end{Arabic}}
\flushleft{\begin{hindi}
तो (ऐ रसूल) क्या तुम बहरों को सुना सकते हो या अन्धे को और उस शख़्श को जो सरीही गुमराही में पड़ा हो रास्ता दिखा सकते हो (हरगिज़ नहीं)
\end{hindi}}
\flushright{\begin{Arabic}
\quranayah[43][41]
\end{Arabic}}
\flushleft{\begin{hindi}
तो अगर हम तुमको (दुनिया से) ले भी जाएँ तो भी हमको उनसे बदला लेना ज़रूरी है
\end{hindi}}
\flushright{\begin{Arabic}
\quranayah[43][42]
\end{Arabic}}
\flushleft{\begin{hindi}
या (तुम्हारी ज़िन्दगी ही में) जिस अज़ाब का हमने उनसे वायदा किया है तुमको दिखा दें तो उन पर हर तरह क़ाबू रखते हैं
\end{hindi}}
\flushright{\begin{Arabic}
\quranayah[43][43]
\end{Arabic}}
\flushleft{\begin{hindi}
तो तुम्हारे पास जो वही भेजी गयी है तुम उसे मज़बूत पकड़े रहो इसमें शक़ नहीं कि तुम सीधी राह पर हो
\end{hindi}}
\flushright{\begin{Arabic}
\quranayah[43][44]
\end{Arabic}}
\flushleft{\begin{hindi}
और ये (क़ुरान) तुम्हारे लिए और तुम्हारी क़ौम के लिए नसीहत है और अनक़रीब ही तुम लोगों से इसकी बाज़पुर्स की जाएगी
\end{hindi}}
\flushright{\begin{Arabic}
\quranayah[43][45]
\end{Arabic}}
\flushleft{\begin{hindi}
और हमने तुमसे पहले अपने जितने पैग़म्बर भेजे हैं उन सब से दरियाफ्त कर देखो क्या हमने ख़ुदा कि सिवा और माबूद बनाएा थे कि उनकी इबादत की जाए
\end{hindi}}
\flushright{\begin{Arabic}
\quranayah[43][46]
\end{Arabic}}
\flushleft{\begin{hindi}
और हम ही ने यक़ीनन मूसा को अपनी निशानियाँ देकर फिरऔन और उसके दरबारियों के पास (पैग़म्बर बनाकर) भेजा था तो मूसा ने कहा कि मैं सारे जहॉन के पालने वाले (ख़ुदा) का रसूल हूँ
\end{hindi}}
\flushright{\begin{Arabic}
\quranayah[43][47]
\end{Arabic}}
\flushleft{\begin{hindi}
तो जब मूसा उन लोगों के पास हमारे मौजिज़े लेकर आए तो वह लोग उन मौजिज़ों की हँसी उड़ाने लगे
\end{hindi}}
\flushright{\begin{Arabic}
\quranayah[43][48]
\end{Arabic}}
\flushleft{\begin{hindi}
और हम जो मौजिज़ा उन को दिखाते थे वह दूसरे से बढ़ कर होता था और आख़िर हमने उनको अज़ाब में गिरफ्तार किया ताकि ये लोग बाज़ आएँ
\end{hindi}}
\flushright{\begin{Arabic}
\quranayah[43][49]
\end{Arabic}}
\flushleft{\begin{hindi}
और (जब) अज़ाब में गिरफ्तार हुए तो (मूसा से) कहने लगे ऐ जादूगर इस एहद के मुताबिक़ जो तुम्हारे परवरदिगार ने तुमसे किया है हमारे वास्ते दुआ कर
\end{hindi}}
\flushright{\begin{Arabic}
\quranayah[43][50]
\end{Arabic}}
\flushleft{\begin{hindi}
(अगर अब की छूटे) तो हम ज़रूर ऊपर आ जाएँगे फिर जब हमने उनसे अज़ाब को हटा दिया तो वह फौरन (अपना) अहद तोड़ बैठे
\end{hindi}}
\flushright{\begin{Arabic}
\quranayah[43][51]
\end{Arabic}}
\flushleft{\begin{hindi}
और फिरऔन ने अपने लोगों में पुकार कर कहा ऐ मेरी क़ौम क्या (ये) मुल्क मिस्र हमारा नहीं और (क्या) ये नहरें जो हमारे (शाही महल के) नीचे बह रही हैं (हमारी नहीं) तो क्या तुमको इतना भी नहीं सूझता
\end{hindi}}
\flushright{\begin{Arabic}
\quranayah[43][52]
\end{Arabic}}
\flushleft{\begin{hindi}
या (सूझता है कि) मैं इस शख़्श (मूसा) से जो एक ज़लील आदमी है और (हकले पन की वजह से) साफ़ गुफ्तगू भी नहीं कर सकता
\end{hindi}}
\flushright{\begin{Arabic}
\quranayah[43][53]
\end{Arabic}}
\flushleft{\begin{hindi}
कहीं बहुत बेहतर हूँ (अगर ये बेहतर है तो इसके लिए सोने के कंगन) (ख़ुदा के हॉ से) क्यों नहीं उतारे गये या उसके साथ फ़रिश्ते जमा होकर आते
\end{hindi}}
\flushright{\begin{Arabic}
\quranayah[43][54]
\end{Arabic}}
\flushleft{\begin{hindi}
ग़रज़ फिरऔन ने (बातें बनाकर) अपनी क़ौम की अक़ल मार दी और वह लोग उसके ताबेदार बन गये बेशक वह लोग बदकार थे ही
\end{hindi}}
\flushright{\begin{Arabic}
\quranayah[43][55]
\end{Arabic}}
\flushleft{\begin{hindi}
ग़रज़ जब उन लोगों ने हमको झुझंला दिया तो हमने भी उनसे बदला लिया तो हमने उन सब (के सब) को डुबो दिया
\end{hindi}}
\flushright{\begin{Arabic}
\quranayah[43][56]
\end{Arabic}}
\flushleft{\begin{hindi}
फिर हमने उनको गया गुज़रा और पिछलों के वास्ते इबरत बना दिया
\end{hindi}}
\flushright{\begin{Arabic}
\quranayah[43][57]
\end{Arabic}}
\flushleft{\begin{hindi}
और(ऐ रसूल) जब मरियम के बेटे (ईसा) की मिसाल बयान की गयी तो उससे तुम्हारी क़ौम के लोग खिलखिला कर हंसने लगे
\end{hindi}}
\flushright{\begin{Arabic}
\quranayah[43][58]
\end{Arabic}}
\flushleft{\begin{hindi}
और बोल उठे कि भला हमारे माबूद अच्छे हैं या वह (ईसा) उन लोगों ने जो ईसा की मिसाल तुमसे बयान की है तो सिर्फ झगड़ने को
\end{hindi}}
\flushright{\begin{Arabic}
\quranayah[43][59]
\end{Arabic}}
\flushleft{\begin{hindi}
बल्कि (हक़ तो यह है कि) ये लोग हैं झगड़ालू ईसा तो बस हमारे एक बन्दे थे जिन पर हमने एहसान किया (नबी बनाया और मौजिज़े दिये) और उनको हमने बनी इसराईल के लिए (अपनी कुदरत का) नमूना बनाया
\end{hindi}}
\flushright{\begin{Arabic}
\quranayah[43][60]
\end{Arabic}}
\flushleft{\begin{hindi}
और अगर हम चाहते तो तुम ही लोगों में से (किसी को) फ़रिश्ते बना देते जो तुम्हारी जगह ज़मीन में रहते
\end{hindi}}
\flushright{\begin{Arabic}
\quranayah[43][61]
\end{Arabic}}
\flushleft{\begin{hindi}
और वह तो यक़ीनन क़यामत की एक रौशन दलील है तुम लोग इसमें हरगिज़ यक़ न करो और मेरी पैरवी करो यही सीधा रास्ता है
\end{hindi}}
\flushright{\begin{Arabic}
\quranayah[43][62]
\end{Arabic}}
\flushleft{\begin{hindi}
और (कहीं) शैतान तुम लोगों को (इससे) रोक न दे वही यक़ीनन तुम्हारा खुल्लम खुल्ला दुश्मन है
\end{hindi}}
\flushright{\begin{Arabic}
\quranayah[43][63]
\end{Arabic}}
\flushleft{\begin{hindi}
और जब ईसा वाज़ेए व रौशन मौजिज़े लेकर आये तो (लोगों से) कहा मैं तुम्हारे पास दानाई (की किताब) लेकर आया हूँ ताकि बाज़ बातें जिन में तुम लोग एख्तेलाफ करते थे तुमको साफ-साफ बता दूँ तो तुम लोग ख़ुदा से डरो और मेरा कहा मानो
\end{hindi}}
\flushright{\begin{Arabic}
\quranayah[43][64]
\end{Arabic}}
\flushleft{\begin{hindi}
बेशक ख़ुदा ही मेरा और तुम्हार परवरदिगार है तो उसी की इबादत करो यही सीधा रास्ता है
\end{hindi}}
\flushright{\begin{Arabic}
\quranayah[43][65]
\end{Arabic}}
\flushleft{\begin{hindi}
तो इनमें से कई फिरक़े उनसे एख्तेलाफ करने लगे तो जिन लोगों ने ज़ुल्म किया उन पर दर्दनांक दिन के अज़ब से अफ़सोस है
\end{hindi}}
\flushright{\begin{Arabic}
\quranayah[43][66]
\end{Arabic}}
\flushleft{\begin{hindi}
क्या ये लोग बस क़यामत के ही मुन्ज़िर बैठे हैं कि अचानक ही उन पर आ जाए और उन को ख़बर तक न हो
\end{hindi}}
\flushright{\begin{Arabic}
\quranayah[43][67]
\end{Arabic}}
\flushleft{\begin{hindi}
(दिली) दोस्त इस दिन (बाहम) एक दूसरे के दुशमन होगें मगर परहेज़गार कि वह दोस्त ही रहेगें
\end{hindi}}
\flushright{\begin{Arabic}
\quranayah[43][68]
\end{Arabic}}
\flushleft{\begin{hindi}
और ख़ुदा उनसे कहेगा ऐ मेरे बन्दों आज न तो तुमको कोई ख़ौफ है और न तुम ग़मग़ीन होगे
\end{hindi}}
\flushright{\begin{Arabic}
\quranayah[43][69]
\end{Arabic}}
\flushleft{\begin{hindi}
(यह) वह लोग हैं जो हमारी आयतों पर ईमान लाए और (हमारे) फ़रमाबरदार थे
\end{hindi}}
\flushright{\begin{Arabic}
\quranayah[43][70]
\end{Arabic}}
\flushleft{\begin{hindi}
तो तुम अपनी बीवियों समैत एजाज़ व इकराम से बेहिश्त में दाखिल हो जाओ
\end{hindi}}
\flushright{\begin{Arabic}
\quranayah[43][71]
\end{Arabic}}
\flushleft{\begin{hindi}
उन पर सोने की एक रिक़ाबियों और प्यालियों का दौर चलेगा और वहाँ जिस चीज़ को जी चाहे और जिससे ऑंखें लज्ज़त उठाएं (सब मौजूद हैं) और तुम उसमें हमेशा रहोगे
\end{hindi}}
\flushright{\begin{Arabic}
\quranayah[43][72]
\end{Arabic}}
\flushleft{\begin{hindi}
और ये जन्नत जिसके तुम वारिस (हिस्सेदार) कर दिये गये हो तुम्हारी क़ारगुज़ारियों का सिला है
\end{hindi}}
\flushright{\begin{Arabic}
\quranayah[43][73]
\end{Arabic}}
\flushleft{\begin{hindi}
वहाँ तुम्हारे वास्ते बहुत से मेवे हैं जिनको तुम खाओगे
\end{hindi}}
\flushright{\begin{Arabic}
\quranayah[43][74]
\end{Arabic}}
\flushleft{\begin{hindi}
(गुनाहगार कुफ्फ़ार) तो यक़ीकन जहन्नुम के अज़ाब में हमेशा रहेगें
\end{hindi}}
\flushright{\begin{Arabic}
\quranayah[43][75]
\end{Arabic}}
\flushleft{\begin{hindi}
जो उनसे कभी नाग़ा न किया जाएगा और वह इसी अज़ाब में नाउम्मीद होकर रहेंगें
\end{hindi}}
\flushright{\begin{Arabic}
\quranayah[43][76]
\end{Arabic}}
\flushleft{\begin{hindi}
और हमने उन पर कोई ज़ुल्म नहीं किया बल्कि वह लोग ख़ुद अपने ऊपर ज़ुल्म कर रहे हैं
\end{hindi}}
\flushright{\begin{Arabic}
\quranayah[43][77]
\end{Arabic}}
\flushleft{\begin{hindi}
और (जहन्नुमी) पुकारेगें कि ऐ मालिक (दरोग़ा ए जहन्नुम कोई तरकीब करो) तुम्हारा परवरदिगार हमें मौत ही दे दे वह जवाब देगा कि तुमको इसी हाल में रहना है
\end{hindi}}
\flushright{\begin{Arabic}
\quranayah[43][78]
\end{Arabic}}
\flushleft{\begin{hindi}
(ऐ कुफ्फ़ार मक्का) हम तो तुम्हारे पास हक़ लेकर आयें हैं तुम मे से बहुत से हक़ (बात से चिढ़ते) हैं
\end{hindi}}
\flushright{\begin{Arabic}
\quranayah[43][79]
\end{Arabic}}
\flushleft{\begin{hindi}
क्या उन लोगों ने कोई बात ठान ली है हमने भी (कुछ ठान लिया है)
\end{hindi}}
\flushright{\begin{Arabic}
\quranayah[43][80]
\end{Arabic}}
\flushleft{\begin{hindi}
क्या ये लोग कुछ समझते हैं कि हम उनके भेद और उनकी सरग़ोशियों को नहीं सुनते हॉ (ज़रूर सुनते हैं) और हमारे फ़रिश्ते उनके पास हैं और उनकी सब बातें लिखते जाते हैं
\end{hindi}}
\flushright{\begin{Arabic}
\quranayah[43][81]
\end{Arabic}}
\flushleft{\begin{hindi}
(ऐ रसूल) तुम कह दो कि अगर ख़ुदा की कोई औलाद होती तो मैं सबसे पहले उसकी इबादत को तैयार हूँ
\end{hindi}}
\flushright{\begin{Arabic}
\quranayah[43][82]
\end{Arabic}}
\flushleft{\begin{hindi}
ये लोग जो कुछ बयान करते हैं सारे आसमान व ज़मीन का मालिक अर्श का मालिक (ख़ुदा) उससे पाक व पाक़ीज़ा है
\end{hindi}}
\flushright{\begin{Arabic}
\quranayah[43][83]
\end{Arabic}}
\flushleft{\begin{hindi}
तो तुम उन्हें छोड़ दो कि पड़े बक बक करते और खेलते रहते हैं यहाँ तक कि जिस दिन का उनसे वायदा किया जाता है
\end{hindi}}
\flushright{\begin{Arabic}
\quranayah[43][84]
\end{Arabic}}
\flushleft{\begin{hindi}
उनके सामने आ मौजूद हो और आसमान में भी (उसी की इबादत की जाती है और वही ज़मीन में भी माबूद है और वही वाकिफ़कार हिकमत वाला है
\end{hindi}}
\flushright{\begin{Arabic}
\quranayah[43][85]
\end{Arabic}}
\flushleft{\begin{hindi}
और वही बहुत बाबरकत है जिसके लिए सारे आसमान व ज़मीन और दोनों के दरमियान की हुक़ुमत है और क़यामत की ख़बर भी उसी को है और तुम लोग उसकी तरफ लौटाए जाओगे
\end{hindi}}
\flushright{\begin{Arabic}
\quranayah[43][86]
\end{Arabic}}
\flushleft{\begin{hindi}
और ख़ुदा के सिवा जिनकी ये लोग इबादत करतें हैं वह तो सिफारिश का भी एख्तेयार नहीं रख़ते मगर (हॉ) जो लोग समझ बूझ कर हक़ बात (तौहीद) की गवाही दें (तो खैर)
\end{hindi}}
\flushright{\begin{Arabic}
\quranayah[43][87]
\end{Arabic}}
\flushleft{\begin{hindi}
और अगर तुम उनसे पूछोगे कि उनको किसने पैदा किया तो ज़रूर कह देगें कि अल्लाह ने फिर (बावजूद इसके) ये कहाँ बहके जा रहे हैं
\end{hindi}}
\flushright{\begin{Arabic}
\quranayah[43][88]
\end{Arabic}}
\flushleft{\begin{hindi}
और (उसी को) रसूल के उस क़ौल का भी इल्म है कि परवरदिगार ये लोग हरगिज़ ईमान न लाएँगे
\end{hindi}}
\flushright{\begin{Arabic}
\quranayah[43][89]
\end{Arabic}}
\flushleft{\begin{hindi}
तो तुम उनसे मुँह फेर लो और कह दो कि तुम को सलाम तो उन्हें अनक़रीब ही (शरारत का नतीजा) मालूम हो जाएगा
\end{hindi}}
\chapter{Ad-Dukhan (The Drought)}
\begin{Arabic}
\Huge{\centerline{\basmalah}}\end{Arabic}
\flushright{\begin{Arabic}
\quranayah[44][1]
\end{Arabic}}
\flushleft{\begin{hindi}
हा मीम
\end{hindi}}
\flushright{\begin{Arabic}
\quranayah[44][2]
\end{Arabic}}
\flushleft{\begin{hindi}
वाज़ेए व रौशन किताब (कुरान) की क़सम
\end{hindi}}
\flushright{\begin{Arabic}
\quranayah[44][3]
\end{Arabic}}
\flushleft{\begin{hindi}
हमने इसको मुबारक रात (शबे क़द्र) में नाज़िल किया बेशक हम (अज़ाब से) डराने वाले थे
\end{hindi}}
\flushright{\begin{Arabic}
\quranayah[44][4]
\end{Arabic}}
\flushleft{\begin{hindi}
इसी रात को तमाम दुनिया के हिक़मत व मसलेहत के (साल भर के) काम फ़ैसले किये जाते हैं
\end{hindi}}
\flushright{\begin{Arabic}
\quranayah[44][5]
\end{Arabic}}
\flushleft{\begin{hindi}
यानि हमारे यहाँ से हुक्म होकर (बेशक) हम ही (पैग़म्बरों के) भेजने वाले हैं
\end{hindi}}
\flushright{\begin{Arabic}
\quranayah[44][6]
\end{Arabic}}
\flushleft{\begin{hindi}
ये तुम्हारे परवरदिगार की मेहरबानी है, वह बेशक बड़ा सुनने वाला वाक़िफ़कार है
\end{hindi}}
\flushright{\begin{Arabic}
\quranayah[44][7]
\end{Arabic}}
\flushleft{\begin{hindi}
सारे आसमान व ज़मीन और जो कुछ इन दोनों के दरमियान है सबका मालिक
\end{hindi}}
\flushright{\begin{Arabic}
\quranayah[44][8]
\end{Arabic}}
\flushleft{\begin{hindi}
अगर तुममें यक़ीन करने की सलाहियत है (तो करो) उसके सिवा कोई माबूद नहीं - वही जिलाता है वही मारता है तुम्हारा मालिक और तुम्हारे (अगले) बाप दादाओं का भी मालिक है
\end{hindi}}
\flushright{\begin{Arabic}
\quranayah[44][9]
\end{Arabic}}
\flushleft{\begin{hindi}
लेकिन ये लोग तो शक़ में पड़े खेल रहे हैं
\end{hindi}}
\flushright{\begin{Arabic}
\quranayah[44][10]
\end{Arabic}}
\flushleft{\begin{hindi}
तो तुम उस दिन का इन्तेज़ार करो कि आसमान से ज़ाहिर ब ज़ाहिर धुऑं निकलेगा
\end{hindi}}
\flushright{\begin{Arabic}
\quranayah[44][11]
\end{Arabic}}
\flushleft{\begin{hindi}
(और) लोगों को ढाँक लेगा ये दर्दनाक अज़ाब है
\end{hindi}}
\flushright{\begin{Arabic}
\quranayah[44][12]
\end{Arabic}}
\flushleft{\begin{hindi}
कुफ्फ़ार भी घबराकर कहेंगे कि परवरदिगार हमसे अज़ाब को दूर दफ़ा कर दे हम भी ईमान लाते हैं
\end{hindi}}
\flushright{\begin{Arabic}
\quranayah[44][13]
\end{Arabic}}
\flushleft{\begin{hindi}
(उस वक्त) भला क्या उनको नसीहत होगी जब उनके पास पैग़म्बर आ चुके जो साफ़ साफ़ बयान कर देते थे
\end{hindi}}
\flushright{\begin{Arabic}
\quranayah[44][14]
\end{Arabic}}
\flushleft{\begin{hindi}
इस पर भी उन लोगों ने उससे मुँह फेरा और कहने लगे ये तो (सिखाया) पढ़ाया हुआ दीवाना है
\end{hindi}}
\flushright{\begin{Arabic}
\quranayah[44][15]
\end{Arabic}}
\flushleft{\begin{hindi}
(अच्छा ख़ैर) हम थोड़े दिन के लिए अज़ाब को टाल देते हैं मगर हम जानते हैं तुम ज़रूर फिर कुफ्र करोगे
\end{hindi}}
\flushright{\begin{Arabic}
\quranayah[44][16]
\end{Arabic}}
\flushleft{\begin{hindi}
हम बेशक (उनसे) पूरा बदला तो बस उस दिन लेगें जिस दिन सख्त पकड़ पकड़ेंगे
\end{hindi}}
\flushright{\begin{Arabic}
\quranayah[44][17]
\end{Arabic}}
\flushleft{\begin{hindi}
और उनसे पहले हमने क़ौमे फिरऔन की आज़माइश की और उनके पास एक आली क़दर पैग़म्बर (मूसा) आए
\end{hindi}}
\flushright{\begin{Arabic}
\quranayah[44][18]
\end{Arabic}}
\flushleft{\begin{hindi}
(और कहा) कि ख़ुदा के बन्दों (बनी इसराईल) को मेरे हवाले कर दो मैं (ख़ुदा की तरफ से) तुम्हारा एक अमानतदार पैग़म्बर हूँ
\end{hindi}}
\flushright{\begin{Arabic}
\quranayah[44][19]
\end{Arabic}}
\flushleft{\begin{hindi}
और ख़ुदा के सामने सरकशी न करो मैं तुम्हारे पास वाज़ेए व रौशन दलीलें ले कर आया हूँ
\end{hindi}}
\flushright{\begin{Arabic}
\quranayah[44][20]
\end{Arabic}}
\flushleft{\begin{hindi}
और इस बात से कि तुम मुझे संगसार करो मैं अपने और तुम्हारे परवरदिगार (ख़ुदा) की पनाह मांगता हूँ
\end{hindi}}
\flushright{\begin{Arabic}
\quranayah[44][21]
\end{Arabic}}
\flushleft{\begin{hindi}
और अगर तुम मुझ पर ईमान नहीं लाए तो तुम मुझसे अलग हो जाओ
\end{hindi}}
\flushright{\begin{Arabic}
\quranayah[44][22]
\end{Arabic}}
\flushleft{\begin{hindi}
(मगर वह सुनाने लगे) तब मूसा ने अपने परवरदिगार से दुआ की कि ये बड़े शरीर लोग हैं
\end{hindi}}
\flushright{\begin{Arabic}
\quranayah[44][23]
\end{Arabic}}
\flushleft{\begin{hindi}
तो ख़ुदा ने हुक्म दिया कि तुम मेरे बन्दों (बनी इसराईल) को रातों रात लेकर चले जाओ और तुम्हारा पीछा भी ज़रूर किया जाएगा
\end{hindi}}
\flushright{\begin{Arabic}
\quranayah[44][24]
\end{Arabic}}
\flushleft{\begin{hindi}
और दरिया को अपनी हालत पर ठहरा हुआ छोड़ कर (पार हो) जाओ (तुम्हारे बाद) उनका सारा लशकर डुबो दिया जाएगा
\end{hindi}}
\flushright{\begin{Arabic}
\quranayah[44][25]
\end{Arabic}}
\flushleft{\begin{hindi}
वह लोग (ख़ुदा जाने) कितने बाग़ और चश्में और खेतियाँ
\end{hindi}}
\flushright{\begin{Arabic}
\quranayah[44][26]
\end{Arabic}}
\flushleft{\begin{hindi}
और नफीस मकानात और आराम की चीज़ें
\end{hindi}}
\flushright{\begin{Arabic}
\quranayah[44][27]
\end{Arabic}}
\flushleft{\begin{hindi}
जिनमें वह ऐश और चैन किया करते थे छोड़ गये यूँ ही हुआ
\end{hindi}}
\flushright{\begin{Arabic}
\quranayah[44][28]
\end{Arabic}}
\flushleft{\begin{hindi}
और उन तमाम चीज़ों का दूसरे लोगों को मालिक बना दिया
\end{hindi}}
\flushright{\begin{Arabic}
\quranayah[44][29]
\end{Arabic}}
\flushleft{\begin{hindi}
तो उन लोगों पर आसमान व ज़मीन को भी रोना न आया और न उन्हें मोहलत ही दी गयी
\end{hindi}}
\flushright{\begin{Arabic}
\quranayah[44][30]
\end{Arabic}}
\flushleft{\begin{hindi}
और हमने बनी इसराईल को ज़िल्लत के अज़ाब से फिरऔन (के पन्जे) से नजात दी
\end{hindi}}
\flushright{\begin{Arabic}
\quranayah[44][31]
\end{Arabic}}
\flushleft{\begin{hindi}
वह बेशक सरकश और हद से बाहर निकल गया था
\end{hindi}}
\flushright{\begin{Arabic}
\quranayah[44][32]
\end{Arabic}}
\flushleft{\begin{hindi}
और हमने बनी इसराईल को समझ बूझ कर सारे जहॉन से बरगुज़ीदा किया था
\end{hindi}}
\flushright{\begin{Arabic}
\quranayah[44][33]
\end{Arabic}}
\flushleft{\begin{hindi}
और हमने उनको ऐसी निशानियाँ दी थीं जिनमें (उनकी) सरीही आज़माइश थी
\end{hindi}}
\flushright{\begin{Arabic}
\quranayah[44][34]
\end{Arabic}}
\flushleft{\begin{hindi}
ये (कुफ्फ़ारे मक्का) (मुसलमानों से) कहते हैं
\end{hindi}}
\flushright{\begin{Arabic}
\quranayah[44][35]
\end{Arabic}}
\flushleft{\begin{hindi}
कि हमें तो सिर्फ एक बार मरना है और फिर हम दोबारा (ज़िन्दा करके) उठाए न जाएँगे
\end{hindi}}
\flushright{\begin{Arabic}
\quranayah[44][36]
\end{Arabic}}
\flushleft{\begin{hindi}
तो अगर तुम सच्चे हो तो हमारे बाप दादाओं को (ज़िन्दा करके) ले आओ
\end{hindi}}
\flushright{\begin{Arabic}
\quranayah[44][37]
\end{Arabic}}
\flushleft{\begin{hindi}
भला ये लोग (क़ूवत में) अच्छे हैं या तुब्बा की क़ौम और वह लोग जो उनसे पहले हो चुके हमने उन सबको हलाक कर दिया (क्योंकि) वह ज़रूर गुनाहगार थे
\end{hindi}}
\flushright{\begin{Arabic}
\quranayah[44][38]
\end{Arabic}}
\flushleft{\begin{hindi}
और हमने सारे आसमान व ज़मीन और जो चीज़े उन दोनों के दरमियान में हैं उनको खेलते हुए नहीं बनाया
\end{hindi}}
\flushright{\begin{Arabic}
\quranayah[44][39]
\end{Arabic}}
\flushleft{\begin{hindi}
इन दोनों को हमने बस ठीक (मसलहत से) पैदा किया मगर उनमें के बहुतेरे लोग नहीं जानते
\end{hindi}}
\flushright{\begin{Arabic}
\quranayah[44][40]
\end{Arabic}}
\flushleft{\begin{hindi}
बेशक फ़ैसला (क़यामत) का दिन उन सब (के दोबार ज़िन्दा होने) का मुक़र्रर वक्त है
\end{hindi}}
\flushright{\begin{Arabic}
\quranayah[44][41]
\end{Arabic}}
\flushleft{\begin{hindi}
जिस दिन कोई दोस्त किसी दोस्त के कुछ काम न आएगा और न उन की मदद की जाएगी
\end{hindi}}
\flushright{\begin{Arabic}
\quranayah[44][42]
\end{Arabic}}
\flushleft{\begin{hindi}
मगर जिन पर ख़ुदा रहम फरमाए बेशक वह (ख़ुदा) सब पर ग़ालिब बड़ा रहम करने वाला है
\end{hindi}}
\flushright{\begin{Arabic}
\quranayah[44][43]
\end{Arabic}}
\flushleft{\begin{hindi}
(आख़ेरत में) थोहड़ का दरख्त
\end{hindi}}
\flushright{\begin{Arabic}
\quranayah[44][44]
\end{Arabic}}
\flushleft{\begin{hindi}
ज़रूर गुनेहगार का खाना होगा
\end{hindi}}
\flushright{\begin{Arabic}
\quranayah[44][45]
\end{Arabic}}
\flushleft{\begin{hindi}
जैसे पिघला हुआ तांबा वह पेटों में इस तरह उबाल खाएगा
\end{hindi}}
\flushright{\begin{Arabic}
\quranayah[44][46]
\end{Arabic}}
\flushleft{\begin{hindi}
जैसे खौलता हुआ पानी उबाल खाता है
\end{hindi}}
\flushright{\begin{Arabic}
\quranayah[44][47]
\end{Arabic}}
\flushleft{\begin{hindi}
(फरिश्तों को हुक्म होगा) इसको पकड़ो और घसीटते हुए दोज़ख़ के बीचों बीच में ले जाओ
\end{hindi}}
\flushright{\begin{Arabic}
\quranayah[44][48]
\end{Arabic}}
\flushleft{\begin{hindi}
फिर उसके सर पर खौलते हुए पानी का अज़ाब डालो फिर उससे ताआनन कहा जाएगा अब मज़ा चखो
\end{hindi}}
\flushright{\begin{Arabic}
\quranayah[44][49]
\end{Arabic}}
\flushleft{\begin{hindi}
बेशक तू तो बड़ा इज्ज़त वाला सरदार है
\end{hindi}}
\flushright{\begin{Arabic}
\quranayah[44][50]
\end{Arabic}}
\flushleft{\begin{hindi}
ये वही दोज़ख़ तो है जिसमें तुम लोग शक़ किया करते थे
\end{hindi}}
\flushright{\begin{Arabic}
\quranayah[44][51]
\end{Arabic}}
\flushleft{\begin{hindi}
बेशक परहेज़गार लोग अमन की जगह
\end{hindi}}
\flushright{\begin{Arabic}
\quranayah[44][52]
\end{Arabic}}
\flushleft{\begin{hindi}
(यानि) बाग़ों और चश्मों में होंगे
\end{hindi}}
\flushright{\begin{Arabic}
\quranayah[44][53]
\end{Arabic}}
\flushleft{\begin{hindi}
रेशम की कभी बारीक़ और कभी दबीज़ पोशाकें पहने हुए एक दूसरे के आमने सामने बैठे होंगे
\end{hindi}}
\flushright{\begin{Arabic}
\quranayah[44][54]
\end{Arabic}}
\flushleft{\begin{hindi}
ऐसा ही होगा और हम बड़ी बड़ी ऑंखों वाली हूरों से उनके जोड़े लगा देंगे
\end{hindi}}
\flushright{\begin{Arabic}
\quranayah[44][55]
\end{Arabic}}
\flushleft{\begin{hindi}
वहाँ इत्मेनान से हर किस्म के मेवे मंगवा कर खायेंगे
\end{hindi}}
\flushright{\begin{Arabic}
\quranayah[44][56]
\end{Arabic}}
\flushleft{\begin{hindi}
वहाँ पहली दफ़ा की मौत के सिवा उनको मौत की तलख़ी चख़नी ही न पड़ेगी और ख़ुदा उनको दोज़ख़ के अज़ाब से महफूज़ रखेगा
\end{hindi}}
\flushright{\begin{Arabic}
\quranayah[44][57]
\end{Arabic}}
\flushleft{\begin{hindi}
(ये) तुम्हारे परवरदिगार का फज़ल है यही तो बड़ी कामयाबी है
\end{hindi}}
\flushright{\begin{Arabic}
\quranayah[44][58]
\end{Arabic}}
\flushleft{\begin{hindi}
तो हमने इस क़ुरान को तुम्हारी ज़बान में (इसलिए) आसान कर दिया है ताकि ये लोग नसीहत पकड़ें तो
\end{hindi}}
\flushright{\begin{Arabic}
\quranayah[44][59]
\end{Arabic}}
\flushleft{\begin{hindi}
(नतीजे के) तुम भी मुन्तज़िर रहो ये लोग भी मुन्तज़िर हैं
\end{hindi}}
\chapter{Al-Jathiyah (The Kneeling)}
\begin{Arabic}
\Huge{\centerline{\basmalah}}\end{Arabic}
\flushright{\begin{Arabic}
\quranayah[45][1]
\end{Arabic}}
\flushleft{\begin{hindi}
हा मीम
\end{hindi}}
\flushright{\begin{Arabic}
\quranayah[45][2]
\end{Arabic}}
\flushleft{\begin{hindi}
ये किताब (क़ुरान) ख़ुदा की तरफ से नाज़िल हुईहै जो ग़ालिब और दाना है
\end{hindi}}
\flushright{\begin{Arabic}
\quranayah[45][3]
\end{Arabic}}
\flushleft{\begin{hindi}
बेशक आसमान और ज़मीन में ईमान वालों के लिए (क़ुदरते ख़ुदा की) बहुत सी निशानियाँ हैं
\end{hindi}}
\flushright{\begin{Arabic}
\quranayah[45][4]
\end{Arabic}}
\flushleft{\begin{hindi}
और तुम्हारी पैदाइश में (भी) और जिन जानवरों को वह (ज़मीन पर) फैलाता रहता है (उनमें भी) यक़ीन करने वालों के वास्ते बहुत सी निशानियाँ हैं
\end{hindi}}
\flushright{\begin{Arabic}
\quranayah[45][5]
\end{Arabic}}
\flushleft{\begin{hindi}
और रात दिन के आने जाने में और ख़ुदा ने आसमान से जो (ज़रिया) रिज़क (पानी) नाज़िल फरमाया फिर उससे ज़मीन को उसके मर जाने के बाद ज़िन्दा किया (उसमें) और हवाओं फेर बदल में अक्लमन्द लोगों के लिए बहुत सी निशानियाँ हैं
\end{hindi}}
\flushright{\begin{Arabic}
\quranayah[45][6]
\end{Arabic}}
\flushleft{\begin{hindi}
ये ख़ुदा की आयतें हैं जिनको हम ठीक (ठीक) तुम्हारे सामने पढ़ते हैं तो ख़ुदा और उसकी आयतों के बाद कौन सी बात होगी
\end{hindi}}
\flushright{\begin{Arabic}
\quranayah[45][7]
\end{Arabic}}
\flushleft{\begin{hindi}
जिस पर ये लोग ईमान लाएंगे हर झूठे गुनाहगार पर अफसोस है
\end{hindi}}
\flushright{\begin{Arabic}
\quranayah[45][8]
\end{Arabic}}
\flushleft{\begin{hindi}
कि ख़ुदा की आयतें उसके सामने पढ़ी जाती हैं और वह सुनता भी है फिर ग़ुरूर से (कुफ़्र पर) अड़ा रहता है गोया उसने उन आयतों को सुना ही नहीं तो (ऐ रसूल) तुम उसे दर्दनाक अज़ाब की ख़ुशख़बरी दे दो
\end{hindi}}
\flushright{\begin{Arabic}
\quranayah[45][9]
\end{Arabic}}
\flushleft{\begin{hindi}
और जब हमारी आयतों में से किसी आयत पर वाक़िफ़ हो जाता है तो उसकी हँसी उड़ाता है ऐसे ही लोगों के वास्ते ज़लील करने वाला अज़ाब है
\end{hindi}}
\flushright{\begin{Arabic}
\quranayah[45][10]
\end{Arabic}}
\flushleft{\begin{hindi}
जहन्नुम तो उनके पीछे ही (पीछे) है और जो कुछ वह आमाल करते रहे न तो वही उनके कुछ काम आएँगे और न जिनको उन्होंने ख़ुदा को छोड़कर (अपने) सरपरस्त बनाए थे और उनके लिए बड़ा (सख्त) अज़ाब है
\end{hindi}}
\flushright{\begin{Arabic}
\quranayah[45][11]
\end{Arabic}}
\flushleft{\begin{hindi}
ये (क़ुरान) है और जिन लोगों ने अपने परवरदिगार की आयतों से इन्कार किया उनके लिए सख्त किस्म का दर्दनाक अज़ाब होगा
\end{hindi}}
\flushright{\begin{Arabic}
\quranayah[45][12]
\end{Arabic}}
\flushleft{\begin{hindi}
ख़ुदा ही तो है जिसने दरिया को तुम्हारे क़ाबू में कर दिया ताकि उसके हुक्म से उसमें कश्तियां चलें और ताकि उसके फज़ल (व करम) से (मआश की) तलाश करो और ताकि तुम शुक्र करो
\end{hindi}}
\flushright{\begin{Arabic}
\quranayah[45][13]
\end{Arabic}}
\flushleft{\begin{hindi}
और जो कुछ आसमानों में है और जो कुछ ज़मीन में है सबको अपने (हुक्म) से तुम्हारे काम में लगा दिया है जो लोग ग़ौर करते हैं उनके लिए इसमें (क़ुदरते ख़ुदा की) बहुत सी निशानियाँ हैं
\end{hindi}}
\flushright{\begin{Arabic}
\quranayah[45][14]
\end{Arabic}}
\flushleft{\begin{hindi}
(ऐ रसूल) मोमिनों से कह दो कि जो लोग ख़ुदा के दिनों की (जो जज़ा के लिए मुक़र्रर हैं) तवक्क़ो नहीं रखते उनसे दरगुज़र करें ताकि वह लोगों के आमाल का बदला दे
\end{hindi}}
\flushright{\begin{Arabic}
\quranayah[45][15]
\end{Arabic}}
\flushleft{\begin{hindi}
जो शख़्श नेक काम करता है तो ख़ास अपने लिए और बुरा काम करेगा तो उस का वबाल उसी पर होगा फिर (आख़िर) तुम अपने परवरदिगार की तरफ लौटाए जाओगे
\end{hindi}}
\flushright{\begin{Arabic}
\quranayah[45][16]
\end{Arabic}}
\flushleft{\begin{hindi}
और हमने बनी इसराईल को किताब (तौरेत) और हुकूमत और नबूवत अता की और उन्हें उम्दा उम्दा चीज़ें खाने को दीं और उनको सारे जहॉन पर फ़ज़ीलत दी
\end{hindi}}
\flushright{\begin{Arabic}
\quranayah[45][17]
\end{Arabic}}
\flushleft{\begin{hindi}
और उनको दीन की खुली हुई दलीलें इनायत की तो उन लोगों ने इल्म आ चुकने के बाद बस आपस की ज़िद में एक दूसरे से एख्तेलाफ़ किया कि ये लोग जिन बातों से एख्तेलाफ़ कर रहें हैं क़यामत के दिन तुम्हारा परवरदिगार उनमें फैसला कर देगा
\end{hindi}}
\flushright{\begin{Arabic}
\quranayah[45][18]
\end{Arabic}}
\flushleft{\begin{hindi}
फिर (ऐ रसूल) हमने तुमको दीन के खुले रास्ते पर क़ायम किया है तो इसी (रास्ते) पर चले जाओ और नादानों की ख्वाहिशो की पैरवी न करो
\end{hindi}}
\flushright{\begin{Arabic}
\quranayah[45][19]
\end{Arabic}}
\flushleft{\begin{hindi}
ये लोग ख़ुदा के सामने तुम्हारे कुछ भी काम न आएँगे और ज़ालिम लोग एक दूसरे के मददगार हैं और ख़ुदा तो परहेज़गारों का मददगार है
\end{hindi}}
\flushright{\begin{Arabic}
\quranayah[45][20]
\end{Arabic}}
\flushleft{\begin{hindi}
ये (क़ुरान) लोगों (की) हिदायत के लिए दलीलो का मजमूआ है और बातें करने वाले लोगों के लिए (अज़सरतापा) हिदायत व रहमत है
\end{hindi}}
\flushright{\begin{Arabic}
\quranayah[45][21]
\end{Arabic}}
\flushleft{\begin{hindi}
जो लोग बुरा काम किया करते हैं क्या वह ये समझते हैं कि हम उनको उन लोगों के बराबर कर देंगे जो ईमान लाए और अच्छे अच्छे काम भी करते रहे और उन सब का जीना मरना एक सा होगा ये लोग (क्या) बुरे हुक्म लगाते हैं
\end{hindi}}
\flushright{\begin{Arabic}
\quranayah[45][22]
\end{Arabic}}
\flushleft{\begin{hindi}
और ख़ुदा ने सारे आसमान व ज़मीन को हिकमत व मसलेहत से पैदा किया और ताकि हर शख़्श को उसके किये का बदला दिया जाए और उन पर (किसी तरह का) ज़ुल्म नहीं किया जाएगा
\end{hindi}}
\flushright{\begin{Arabic}
\quranayah[45][23]
\end{Arabic}}
\flushleft{\begin{hindi}
भला तुमने उस शख़्श को भी देखा है जिसने अपनी नफसियानी ख़वाहिशों को माबूद बना रखा है और (उसकी हालत) समझ बूझ कर ख़ुदा ने उसे गुमराही में छोड़ दिया है और उसके कान और दिल पर अलामत मुक़र्रर कर दी है (कि ये ईमान न लाएगा) और उसकी ऑंख पर पर्दा डाल दिया है फिर ख़ुदा के बाद उसकी हिदायत कौन कर सकता है तो क्या तुम लोग (इतना भी) ग़ौर नहीं करते
\end{hindi}}
\flushright{\begin{Arabic}
\quranayah[45][24]
\end{Arabic}}
\flushleft{\begin{hindi}
और वह लोग कहते हैं कि हमारी ज़िन्दगी तो बस दुनिया ही की है (यहीं) मरते हैं और (यहीं) जीते हैं और हमको बस ज़माना ही (जिलाता) मारता है और उनको इसकी कुछ ख़बर तो है नहीं ये लोग तो बस अटकल की बातें करते हैं
\end{hindi}}
\flushright{\begin{Arabic}
\quranayah[45][25]
\end{Arabic}}
\flushleft{\begin{hindi}
और जब उनके सामने हमारी खुली खुली आयतें पढ़ी जाती हैं तो उनकी कट हुज्जती बस यही होती है कि वह कहते हैं कि अगर तुम सच्चे हो तो हमारे बाप दादाओं को (जिला कर) ले तो आओ
\end{hindi}}
\flushright{\begin{Arabic}
\quranayah[45][26]
\end{Arabic}}
\flushleft{\begin{hindi}
(ऐ रसूल) तुम कह दो कि ख़ुदा ही तुमको ज़िन्दा (पैदा) करता है और वही तुमको मारता है फिर वही तुमको क़यामत के दिन जिस (के होने) में किसी तरह का शक़ नहीं जमा करेगा मगर अक्सर लोग नहीं जानते
\end{hindi}}
\flushright{\begin{Arabic}
\quranayah[45][27]
\end{Arabic}}
\flushleft{\begin{hindi}
और सारे आसमान व ज़मीन की बादशाहत ख़ास ख़ुदा की है और जिस रोज़ क़यामत बरपा होगी उस रोज़ अहले बातिल बड़े घाटे में रहेंगे
\end{hindi}}
\flushright{\begin{Arabic}
\quranayah[45][28]
\end{Arabic}}
\flushleft{\begin{hindi}
और (ऐ रसूल) तुम हर उम्मत को देखोगे कि (फैसले की मुन्तज़िर अदब से) घूटनों के बल बैठी होगी और हर उम्मत अपने नामाए आमाल की तरफ बुलाइ जाएगी जो कुछ तुम लोग करते थे आज तुमको उसका बदला दिया जाएगा
\end{hindi}}
\flushright{\begin{Arabic}
\quranayah[45][29]
\end{Arabic}}
\flushleft{\begin{hindi}
ये हमारी किताब (जिसमें आमाल लिखे हैं) तुम्हारे मुक़ाबले में ठीक ठीक बोल रही है जो कुछ भी तुम करते थे हम लिखवाते जाते थे
\end{hindi}}
\flushright{\begin{Arabic}
\quranayah[45][30]
\end{Arabic}}
\flushleft{\begin{hindi}
ग़रज़ जिन लोगों ने ईमान क़ुबूल किया और अच्छे (अच्छे) काम किये तो उनको उनका परवरदिगार अपनी रहमत (से बेहिश्त) में दाख़िल करेगा यही तो सरीही कामयाबी है
\end{hindi}}
\flushright{\begin{Arabic}
\quranayah[45][31]
\end{Arabic}}
\flushleft{\begin{hindi}
और जिन्होंने कुफ्र एख्तेयार किया (उनसे कहा जाएगा) तो क्या तुम्हारे सामने हमारी आयतें नहीं पढ़ी जाती थीं (ज़रूर) तो तुमने तकब्बुर किया और तुम लोग तो गुनेहगार हो गए
\end{hindi}}
\flushright{\begin{Arabic}
\quranayah[45][32]
\end{Arabic}}
\flushleft{\begin{hindi}
और जब (तुम से) कहा जाता था कि ख़ुदा का वायदा सच्चा है और क़यामत (के आने) में कुछ शुबहा नहीं तो तुम कहते थे कि हम नहीं जानते कि क़यामत क्या चीज़ है हम तो बस (उसे) एक ख्याली बात समझते हैं और हम तो (उसका) यक़ीन नहीं रखते
\end{hindi}}
\flushright{\begin{Arabic}
\quranayah[45][33]
\end{Arabic}}
\flushleft{\begin{hindi}
और उनके करतूतों की बुराईयाँ उस पर ज़ाहिर हो जाएँगी और जिस (अज़ाब) की ये हँसी उड़ाया करते थे उन्हें (हर तरफ से) घेर लेगा
\end{hindi}}
\flushright{\begin{Arabic}
\quranayah[45][34]
\end{Arabic}}
\flushleft{\begin{hindi}
और (उनसे) कहा जाएगा कि जिस तरह तुमने उस दिन के आने को भुला दिया था उसी तरह आज हम तुमको अपनी रहमत से अमदन भुला देंगे और तुम्हारा ठिकाना दोज़ख़ है और कोई तुम्हारा मददगार नहीं
\end{hindi}}
\flushright{\begin{Arabic}
\quranayah[45][35]
\end{Arabic}}
\flushleft{\begin{hindi}
ये इस सबब से कि तुम लोगों ने ख़ुदा की आयतों को हँसी ठट्ठा बना रखा था और दुनयावी ज़िन्दगी ने तुमको धोखे में डाल दिया था ग़रज़ ये लोग न तो आज दुनिया से निकाले जाएँगे और न उनको इसका मौका दिया जाएगा कि (तौबा करके ख़ुदा को) राज़ी कर ले
\end{hindi}}
\flushright{\begin{Arabic}
\quranayah[45][36]
\end{Arabic}}
\flushleft{\begin{hindi}
पस सब तारीफ ख़ुदा ही के लिए सज़ावार है जो सारे आसमान का मालिक और ज़मीन का मालिक (ग़रज़) सारे जहॉन का मालिक है
\end{hindi}}
\flushright{\begin{Arabic}
\quranayah[45][37]
\end{Arabic}}
\flushleft{\begin{hindi}
और सारे आसमान व ज़मीन में उसके लिए बड़ाई है और वही (सब पर) ग़ालिब हिकमत वाला है
\end{hindi}}
\chapter{Al-Ahqaf (The Sandhills)}
\begin{Arabic}
\Huge{\centerline{\basmalah}}\end{Arabic}
\flushright{\begin{Arabic}
\quranayah[46][1]
\end{Arabic}}
\flushleft{\begin{hindi}
हा मीम
\end{hindi}}
\flushright{\begin{Arabic}
\quranayah[46][2]
\end{Arabic}}
\flushleft{\begin{hindi}
ये किताब ग़ालिब (व) हकीम ख़ुदा की तरफ से नाज़िल हुई है
\end{hindi}}
\flushright{\begin{Arabic}
\quranayah[46][3]
\end{Arabic}}
\flushleft{\begin{hindi}
हमने तो सारे आसमान व ज़मीन और जो कुछ इन दोनों के दरमियान है हिकमत ही से एक ख़ास वक्त तक के लिए ही पैदा किया है और कुफ्फ़ार जिन चीज़ों से डराए जाते हैं उन से मुँह फेर लेते हैं
\end{hindi}}
\flushright{\begin{Arabic}
\quranayah[46][4]
\end{Arabic}}
\flushleft{\begin{hindi}
(ऐ रसूल) तुम पूछो कि ख़ुदा को छोड़ कर जिनकी तुम इबादत करते हो क्या तुमने उनको देखा है मुझे भी तो दिखाओ कि उन लोगों ने ज़मीन में क्या चीज़े पैदा की हैं या आसमानों (के बनाने) में उनकी शिरकत है तो अगर तुम सच्चे हो तो उससे पहले की कोई किताब (या अगलों के) इल्म का बक़िया हो तो मेरे सामने पेश करो
\end{hindi}}
\flushright{\begin{Arabic}
\quranayah[46][5]
\end{Arabic}}
\flushleft{\begin{hindi}
और उस शख़्श से बढ़ कर कौन गुमराह हो सकता है जो ख़ुदा के सिवा ऐसे शख़्श को पुकारे जो उसे क़यामत तक जवाब ही न दे और उनको उनके पुकारने की ख़बरें तक न हों
\end{hindi}}
\flushright{\begin{Arabic}
\quranayah[46][6]
\end{Arabic}}
\flushleft{\begin{hindi}
और जब लोग (क़यामत) में जमा किये जाएगें तो वह (माबूद) उनके दुशमन हो जाएंगे और उनकी परसतिश से इन्कार करेंगे
\end{hindi}}
\flushright{\begin{Arabic}
\quranayah[46][7]
\end{Arabic}}
\flushleft{\begin{hindi}
और जब हमारी खुली खुली आयतें उनके सामने पढ़ी जाती हैं तो जो लोग काफिर हैं हक़ के बारे में जब उनके पास आ चुका तो कहते हैं ये तो सरीही जादू है
\end{hindi}}
\flushright{\begin{Arabic}
\quranayah[46][8]
\end{Arabic}}
\flushleft{\begin{hindi}
क्या ये कहते हैं कि इसने इसको ख़ुद गढ़ लिया है तो (ऐ रसूल) तुम कह दो कि अगर मैं इसको (अपने जी से) गढ़ लेता तो तुम ख़ुदा के सामने मेरे कुछ भी काम न आओगे जो जो बातें तुम लोग उसके बारे में करते रहते हो वह ख़ूब जानता है मेरे और तुम्हारे दरमियान वही गवाही को काफ़ी है और वही बड़ा बख्शने वाला है मेहरबान है
\end{hindi}}
\flushright{\begin{Arabic}
\quranayah[46][9]
\end{Arabic}}
\flushleft{\begin{hindi}
(ऐ रसूल) तुम कह दो कि मैं कोई नया रसूल तो आया नहीं हूँ और मैं कुछ नहीं जानता कि आइन्दा मेरे साथ क्या किया जाएगा और न (ये कि) तुम्हारे साथ क्या किया जाएगा मैं तो बस उसी का पाबन्द हूँ जो मेरे पास वही आयी है और मैं तो बस एलानिया डराने वाला हूँ
\end{hindi}}
\flushright{\begin{Arabic}
\quranayah[46][10]
\end{Arabic}}
\flushleft{\begin{hindi}
(ऐ रसूल) तुम कह दो कि भला देखो तो कि अगर ये (क़ुरान) ख़ुदा की तरफ से हो और तुम उससे इन्कार कर बैठे हालॉकि (बनी इसराईल में से) एक गवाह उसके मिसल की गवाही भी दे चुका और ईमान भी ले आया और तुमने सरकशी की (तो तुम्हारे ज़ालिम होने में क्या शक़ है) बेशक ख़ुदा ज़ालिम लोगों को मन्ज़िल मक़सूद तक नहीं पहुँचाता
\end{hindi}}
\flushright{\begin{Arabic}
\quranayah[46][11]
\end{Arabic}}
\flushleft{\begin{hindi}
और काफिर लोग मोमिनों के बारे में कहते हैं कि अगर ये (दीन) बेहतर होता तो ये लोग उसकी तरफ हमसे पहले न दौड़ पड़ते और जब क़ुरान के ज़रिए से उनकी हिदायत न हुई तो अब भी कहेंगे ये तो एक क़दीमी झूठ है
\end{hindi}}
\flushright{\begin{Arabic}
\quranayah[46][12]
\end{Arabic}}
\flushleft{\begin{hindi}
और इसके क़ब्ल मूसा की किताब पेशवा और (सरासर) रहमत थी और ये (क़ुरान) वह किताब है जो अरबी ज़बान में (उसकी) तसदीक़ करती है ताकि (इसके ज़रिए से) ज़ालिमों को डराए और नेकी कारों के लिए (अज़सरतापा) ख़ुशख़बरी है
\end{hindi}}
\flushright{\begin{Arabic}
\quranayah[46][13]
\end{Arabic}}
\flushleft{\begin{hindi}
बेशक जिन लोगों ने कहा कि हमारा परवरदिगार ख़ुदा है फिर वह इस पर क़ायम रहे तो (क़यामत में) उनको न कुछ ख़ौफ़ होगा और न वह ग़मग़ीन होंगे
\end{hindi}}
\flushright{\begin{Arabic}
\quranayah[46][14]
\end{Arabic}}
\flushleft{\begin{hindi}
यही तो अहले जन्नत हैं कि हमेशा उसमें रहेंगे (ये) उसका सिला है जो ये लोग (दुनिया में) किया करते थे
\end{hindi}}
\flushright{\begin{Arabic}
\quranayah[46][15]
\end{Arabic}}
\flushleft{\begin{hindi}
और हमने इन्सान को अपने माँ बाप के साथ भलाई करने का हुक्म दिया (क्यों कि) उसकी माँ ने रंज ही की हालत में उसको पेट में रखा और रंज ही से उसको जना और उसका पेट में रहना और उसको दूध बढ़ाई के तीस महीने हुए यहाँ तक कि जब अपनी पूरी जवानी को पहुँचता और चालीस बरस (के सिन) को पहुँचता है तो (ख़ुदा से) अर्ज़ करता है परवरदिगार तो मुझे तौफ़ीक़ अता फरमा कि तूने जो एहसानात मुझ पर और मेरे वालदैन पर किये हैं मैं उन एहसानों का शुक्रिया अदा करूँ और ये (भी तौफीक दे) कि मैं ऐसा नेक काम करूँ जिसे तू पसन्द करे और मेरे लिए मेरी औलाद में सुलाह व तक़वा पैदा करे तेरी तरफ रूजू करता हूँ और मैं यक़ीनन फरमाबरदारो में हूँ
\end{hindi}}
\flushright{\begin{Arabic}
\quranayah[46][16]
\end{Arabic}}
\flushleft{\begin{hindi}
यही वह लोग हैं जिनके नेक अमल हम क़ुबूल फरमाएँगे और बेहिश्त (के जाने) वालों में उनके गुनाहों से दरग़ुज़र करेंगे (ये वह) सच्चा वायदा है जो उन से किया जाता था
\end{hindi}}
\flushright{\begin{Arabic}
\quranayah[46][17]
\end{Arabic}}
\flushleft{\begin{hindi}
और जिसने अपने माँ बाप से कहा कि तुम्हारा बुरा हो, क्या तुम मुझे धमकी देते हो कि मैं दोबारा (कब्र से) निकाला जाऊँगा हालॉकि बहुत से लोग मुझसे पहले गुज़र चुके (और कोई ज़िन्दा न हुआ) और दोनों फ़रियाद कर रहे थे कि तुझ पर वाए हो ईमान ले आ ख़ुदा का वायदा ज़रूर सच्चा है तो वह बोल उठा कि ये तो बस अगले लोगों के अफ़साने हैं
\end{hindi}}
\flushright{\begin{Arabic}
\quranayah[46][18]
\end{Arabic}}
\flushleft{\begin{hindi}
ये वही लोग हैं कि जिन्नात और आदमियों की (दूसरी) उम्मतें जो उनसे पहले गुज़र चुकी हैं उन ही के शुमूल में उन पर भी अज़ाब का वायदा मुस्तहक़ हो चुका है ये लोग बेशक घाटा उठाने वाले थे
\end{hindi}}
\flushright{\begin{Arabic}
\quranayah[46][19]
\end{Arabic}}
\flushleft{\begin{hindi}
और लोगों ने जैसे काम किये होंगे उसी के मुताबिक सबके दर्जे होंगे और ये इसलिए कि ख़ुदा उनके आमाल का उनको पूरा पूरा बदला दे और उन पर कुछ भी ज़ुल्म न किया जाएं
\end{hindi}}
\flushright{\begin{Arabic}
\quranayah[46][20]
\end{Arabic}}
\flushleft{\begin{hindi}
और जिस दिन कुफ्फार जहन्नुम के सामने लाएँ जाएँगे (तो उनसे कहा जाएगा कि) तुमने अपनी दुनिया की ज़िन्दगी में अपने मज़े उड़ा चुके और उसमें ख़ूब चैन कर चुके तो आज तुम पर ज़िल्लत का अज़ाब किया जाएगा इसलिए कि तुम अपनी ज़मीन में अकड़ा करते थे और इसलिए कि तुम बदकारियां करते थे
\end{hindi}}
\flushright{\begin{Arabic}
\quranayah[46][21]
\end{Arabic}}
\flushleft{\begin{hindi}
और (ऐ रसूल) तुम आद को भाई (हूद) को याद करो जब उन्होंने अपनी क़ौम को (सरज़मीन) अहक़ाफ में डराया और उनके पहले और उनके बाद भी बहुत से डराने वाले पैग़म्बर गुज़र चुके थे (और हूद ने अपनी क़ौम से कहा) कि ख़ुदा के सिवा किसी की इबादत न करो क्योंकि तुम्हारे बारे में एक बड़े सख्त दिन के अज़ाब से डरता हूँ
\end{hindi}}
\flushright{\begin{Arabic}
\quranayah[46][22]
\end{Arabic}}
\flushleft{\begin{hindi}
वह बोले क्या तुम हमारे पास इसलिए आए हो कि हमको हमारे माबूदों से फेर दो तो अगर तुम सच्चे हो तो जिस अज़ाब की तुम हमें धमकी देते हो ले आओ
\end{hindi}}
\flushright{\begin{Arabic}
\quranayah[46][23]
\end{Arabic}}
\flushleft{\begin{hindi}
हूद ने कहा (इसका) इल्म तो बस ख़ुदा के पास है और (मैं जो एहकाम देकर भेजा गया हूँ) वह तुम्हें पहुँचाए देता हूँ मगर मैं तुमको देखता हूँ कि तुम जाहिल लोग हो
\end{hindi}}
\flushright{\begin{Arabic}
\quranayah[46][24]
\end{Arabic}}
\flushleft{\begin{hindi}
तो जब उन लोगों ने इस (अज़ाब) को देखा कि वबाल की तरह उनके मैदानों की तरफ उम्ड़ा आ रहा है तो कहने लगे ये तो बादल है जो हम पर बरस कर रहेगा (नहीं) बल्कि ये वह (अज़ाब) जिसकी तुम जल्दी मचा रहे थे (ये) वह ऑंधी है जिसमें दर्दनाक (अज़ाब) है
\end{hindi}}
\flushright{\begin{Arabic}
\quranayah[46][25]
\end{Arabic}}
\flushleft{\begin{hindi}
जो अपने परवरदिगार के हुक्म से हर चीज़ को तबाह व बरबाद कर देगी तो वह ऐसे (तबाह) हुए कि उनके घरों के सिवा कुछ नज़र ही नहीं आता था हम गुनाहगारों की यूँ ही सज़ा किया करते हैं
\end{hindi}}
\flushright{\begin{Arabic}
\quranayah[46][26]
\end{Arabic}}
\flushleft{\begin{hindi}
और हमने उनको ऐसे कामों में मक़दूर दिये थे जिनमें तुम्हें (कुछ भी) मक़दूर नहीं दिया और उन्हें कान और ऑंख और दिल (सब कुछ दिए थे) तो चूँकि वह लोग ख़ुदा की आयतों से इन्कार करने लगे तो न उनके कान ही कुछ काम आए और न उनकी ऑंखें और न उनके दिल और जिस (अज़ाब) की ये लोग हँसी उड़ाया करते थे उसने उनको हर तरफ से घेर लिया
\end{hindi}}
\flushright{\begin{Arabic}
\quranayah[46][27]
\end{Arabic}}
\flushleft{\begin{hindi}
और (ऐ अहले मक्का) हमने तुम्हारे इर्द गिर्द की बस्तियों को हलाक कर मारा और (अपनी क़ुदरत की) बहुत सी निशानियाँ तरह तरह से दिखा दी ताकि ये लोग बाज़ आएँ (मगर कौन सुनता है)
\end{hindi}}
\flushright{\begin{Arabic}
\quranayah[46][28]
\end{Arabic}}
\flushleft{\begin{hindi}
तो ख़ुदा के सिवा जिन को उन लोगों ने तक़र्रुब (ख़ुदा) के लिए माबूद बना रखा था उन्होंने (अज़ाब के वक्त) उनकी क्यों न मदद की बल्कि वह तो उनसे ग़ायब हो गये और उनके झूठ और उनकी (इफ़तेरा) परदाज़ियों की ये हक़ीक़त थी
\end{hindi}}
\flushright{\begin{Arabic}
\quranayah[46][29]
\end{Arabic}}
\flushleft{\begin{hindi}
और जब हमने जिनों में से कई शख़्शों को तुम्हारी तरफ मुतावज्जे किया कि वह दिल लगाकर क़ुरान सुनें तो जब उनके पास हाज़िर हुए तो एक दुसरे से कहने लगे ख़ामोश बैठे (सुनते) रहो फिर जब (पढ़ना) तमाम हुआ तो अपनी क़ौम की तरफ़ वापस गए
\end{hindi}}
\flushright{\begin{Arabic}
\quranayah[46][30]
\end{Arabic}}
\flushleft{\begin{hindi}
कि (उनको अज़ाब से) डराएं तो उन से कहना शुरू किया कि ऐ भाइयों हम एक किताब सुन आए हैं जो मूसा के बाद नाज़िल हुई है (और) जो किताबें, पहले (नाज़िल हुयीं) हैं उनकी तसदीक़ करती हैं सच्चे (दीन) और सीधी राह की हिदायत करती हैं
\end{hindi}}
\flushright{\begin{Arabic}
\quranayah[46][31]
\end{Arabic}}
\flushleft{\begin{hindi}
ऐ हमारी क़ौम ख़ुदा की तरफ बुलाने वाले की बात मानों और ख़ुदा पर ईमान लाओ वह तुम्हारे गुनाह बख्श देगा और (क़यामत) में तुम्हें दर्दनाक अज़ाब से पनाह में रखेगा
\end{hindi}}
\flushright{\begin{Arabic}
\quranayah[46][32]
\end{Arabic}}
\flushleft{\begin{hindi}
और जिसने ख़ुदा की तरफ बुलाने वाले की बात न मानी तो (याद रहे कि) वह (ख़ुदा को रूए) ज़मीन में आजिज़ नहीं कर सकता और न उस के सिवा कोई सरपरस्त होगा यही लोग गुमराही में हैं
\end{hindi}}
\flushright{\begin{Arabic}
\quranayah[46][33]
\end{Arabic}}
\flushleft{\begin{hindi}
क्या इन लोगों ने ये ग़ौर नहीं किया कि जिस ख़ुदा ने सारे आसमान और ज़मीन को पैदा किया और उनके पैदा करने से ज़रा भी थका नहीं वह इस बात पर क़ादिर है कि मुर्दो को ज़िन्दा करेगा हाँ (ज़रूर) वह हर चीज़ पर क़ादिर है
\end{hindi}}
\flushright{\begin{Arabic}
\quranayah[46][34]
\end{Arabic}}
\flushleft{\begin{hindi}
जिस दिन कुफ्फ़ार (जहन्नुम की) आग के सामने पेश किए जाएँगे (तो उन से पूछा जाएगा) क्या अब भी ये बरहक़ नहीं है वह लोग कहेंगे अपने परवरदिगार की क़सम हाँ (हक़ है) ख़ुदा फ़रमाएगा तो लो अब अपने इन्कार व कुफ्र के बदले अज़ाब के मज़े चखो
\end{hindi}}
\flushright{\begin{Arabic}
\quranayah[46][35]
\end{Arabic}}
\flushleft{\begin{hindi}
तो (ऐ रसूल) पैग़म्बरों में से जिस तरह अव्वलुल अज्म (आली हिम्मत), सब्र करते रहे तुम भी सब्र करो और उनके लिए (अज़ाब) की ताज़ील की ख्वाहिश न करो जिस दिन यह लोग उस कयामत को देखेंगे जिसको उनसे वायदा किया जाता है तो (उनको मालूम होगा कि) गोया ये लोग (दुनिया में) बहुत रहे होगें तो सारे दिन में से एक घड़ी भर तो बस वही लोग हलाक होंगे जो बदकार थे
\end{hindi}}
\chapter{Muhammad (Muhammad)}
\begin{Arabic}
\Huge{\centerline{\basmalah}}\end{Arabic}
\flushright{\begin{Arabic}
\quranayah[47][1]
\end{Arabic}}
\flushleft{\begin{hindi}
जिन लोगों ने कुफ़्र एख्तेयार किया और (लोगों को) ख़ुदा के रास्ते से रोका ख़ुदा ने उनके आमाल अकारत कर दिए
\end{hindi}}
\flushright{\begin{Arabic}
\quranayah[47][2]
\end{Arabic}}
\flushleft{\begin{hindi}
और जिन लोगों ने ईमान क़ुबूल किया और अच्छे (अच्छे) काम किए और जो (किताब) मोहम्मद पर उनके परवरदिगार की तरफ से नाज़िल हुई है और वह बरहक़ है उस पर ईमान लाए तो ख़ुदा ने उनके गुनाह उनसे दूर कर दिए और उनकी हालत संवार दी
\end{hindi}}
\flushright{\begin{Arabic}
\quranayah[47][3]
\end{Arabic}}
\flushleft{\begin{hindi}
ये इस वजह से कि काफिरों ने झूठी बात की पैरवी की और ईमान वालों ने अपने परवरदिगार का सच्चा दीन एख्तेयार किया यूँ ख़ुदा लोगों के समझाने के लिए मिसालें बयान करता है
\end{hindi}}
\flushright{\begin{Arabic}
\quranayah[47][4]
\end{Arabic}}
\flushleft{\begin{hindi}
तो जब तुम काफिरों से भिड़ो तो (उनकी) गर्दनें मारो यहाँ तक कि जब तुम उन्हें ज़ख्मों से चूर कर डालो तो उनकी मुश्कें कस लो फिर उसके बाद या तो एहसान रख (कर छोड़ दे) या मुआवेज़ा लेकर, यहाँ तक कि (दुशमन) लड़ाई के हथियार रख दे तो (याद रखो) अगर ख़ुदा चाहता तो (और तरह) उनसे बदला लेता मगर उसने चाहा कि तुम्हारी आज़माइश एक दूसरे से (लड़वा कर) करे और जो लोग ख़ुदा की राह में यहीद किये गए उनकी कारगुज़ारियों को ख़ुदा हरगिज़ अकारत न करेगा
\end{hindi}}
\flushright{\begin{Arabic}
\quranayah[47][5]
\end{Arabic}}
\flushleft{\begin{hindi}
उन्हें अनक़रीब मंज़िले मक़सूद तक पहुँचाएगा
\end{hindi}}
\flushright{\begin{Arabic}
\quranayah[47][6]
\end{Arabic}}
\flushleft{\begin{hindi}
और उनकी हालत सवार देगा और उनको उस बेहिश्त में दाख़िल करेगा जिसका उन्हें (पहले से) येनासा कर रखा है
\end{hindi}}
\flushright{\begin{Arabic}
\quranayah[47][7]
\end{Arabic}}
\flushleft{\begin{hindi}
ऐ ईमानदारों अगर तुम ख़ुदा (के दीन) की मदद करोगे तो वह भी तुम्हारी मदद करेगा और तुम्हें साबित क़दम रखेगा
\end{hindi}}
\flushright{\begin{Arabic}
\quranayah[47][8]
\end{Arabic}}
\flushleft{\begin{hindi}
और जो लोग काफ़िर हैं उनके लिए तो डगमगाहट है और ख़ुदा (उनके) आमाल बरबाद कर देगा
\end{hindi}}
\flushright{\begin{Arabic}
\quranayah[47][9]
\end{Arabic}}
\flushleft{\begin{hindi}
ये इसलिए कि ख़ुदा ने जो चीज़ नाज़िल फ़रमायी उसे उन्होने (नापसन्द किया) तो ख़ुदा ने उनकी कारस्तानियों को अकारत कर दिया
\end{hindi}}
\flushright{\begin{Arabic}
\quranayah[47][10]
\end{Arabic}}
\flushleft{\begin{hindi}
तो क्या ये लोग रूए ज़मीन पर चले फिरे नहीं तो देखते जो लोग उनसे पहले थे उनका अन्जाम क्या (ख़राब) हुआ कि ख़ुदा ने उन पर तबाही डाल दी और इसी तरह (उन) काफिरों को भी (सज़ा मिलेगी)
\end{hindi}}
\flushright{\begin{Arabic}
\quranayah[47][11]
\end{Arabic}}
\flushleft{\begin{hindi}
ये इस वजह से कि ईमानदारों का ख़ुदा सरपरस्त है और काफिरों का हरगिज़ कोई सरपरस्त नहीं
\end{hindi}}
\flushright{\begin{Arabic}
\quranayah[47][12]
\end{Arabic}}
\flushleft{\begin{hindi}
ख़ुदा उन लोगों को जो ईमान लाए और अच्छे (अच्छे) काम करते रहे ज़रूर बेहिश्त के उन बाग़ों में जा पहुँचाएगा जिनके नीचे नहरें जारी हैं और जो काफ़िर हैं वह (दुनिया में) चैन करते हैं और इस तरह (बेफिक्री से खाते (पीते) हैं जैसे चारपाए खाते पीते हैं और आख़िर) उनका ठिकाना जहन्नुम है
\end{hindi}}
\flushright{\begin{Arabic}
\quranayah[47][13]
\end{Arabic}}
\flushleft{\begin{hindi}
और जिस बस्ती से तुम लोगों ने निकाल दिया उससे ज़ोर में कहीं बढ़ चढ़ के बहुत सी बस्तियाँ थीं जिनको हमने तबाह बर्बाद कर दिया तो उनका कोई मददगार भी न हुआ
\end{hindi}}
\flushright{\begin{Arabic}
\quranayah[47][14]
\end{Arabic}}
\flushleft{\begin{hindi}
क्या जो शख़्श अपने परवरदिगार की तरफ से रौशन दलील पर हो उस शख़्श के बराबर हो सकता है जिसकी बदकारियाँ उसे भली कर दिखायीं गयीं हों वह अपनी नफ़सियानी ख्वाहिशों पर चलते हैं
\end{hindi}}
\flushright{\begin{Arabic}
\quranayah[47][15]
\end{Arabic}}
\flushleft{\begin{hindi}
जिस बेहिश्त का परहेज़गारों से वायदा किया जाता है उसकी सिफत ये है कि उसमें पानी की नहरें जिनमें ज़रा बू नहीं और दूध की नहरें हैं जिनका मज़ा तक नहीं बदला और शराब की नहरें हैं जो पीने वालों के लिए (सरासर) लज्ज़त है और साफ़ शफ्फ़ाफ़ यहद की नहरें हैं और वहाँ उनके लिए हर किस्म के मेवे हैं और उनके परवरदिगार की तरफ से बख़्शिस है (भला ये लोग) उनके बराबर हो सकते हैं जो हमेशा दोज़ख़ में रहेंगे और उनको खौलता हुआ पानी पिलाया जाएगा तो वह ऑंतों के टुकड़े टुकड़े कर डालेगा
\end{hindi}}
\flushright{\begin{Arabic}
\quranayah[47][16]
\end{Arabic}}
\flushleft{\begin{hindi}
और (ऐ रसूल) उनमें से बाज़ ऐसे भी हैं जो तुम्हारी तरफ कान लगाए रहते हैं यहाँ तक कि सब सुना कर जब तुम्हारे पास से निकलते हैं तो जिन लोगों को इल्म (कुरान) दिया गया है उनसे कहते हैं (क्यों भई) अभी उस शख़्श ने क्या कहा था ये वही लोग हैं जिनके दिलों पर ख़ुदा ने (कुफ़्र की) अलामत मुक़र्रर कर दी है और ये अपनी नफसियानी ख्वाहिशों पर चल रहे हैं
\end{hindi}}
\flushright{\begin{Arabic}
\quranayah[47][17]
\end{Arabic}}
\flushleft{\begin{hindi}
और जो लोग हिदायत याफ़ता हैं उनको ख़ुदा (क़ुरान के ज़रिए से) मज़ीद हिदायत करता है और उनको परहेज़गारी अता फरमाता है
\end{hindi}}
\flushright{\begin{Arabic}
\quranayah[47][18]
\end{Arabic}}
\flushleft{\begin{hindi}
तो क्या ये लोग बस क़यामत ही के मुनतज़िर हैं कि उन पर एक बारगी आ जाए तो उसकी निशानियाँ आ चुकी हैं तो जिस वक्त क़यामत उन (के सर) पर आ पहुँचेगी फिर उन्हें नसीहत कहाँ मुफीद हो सकती है
\end{hindi}}
\flushright{\begin{Arabic}
\quranayah[47][19]
\end{Arabic}}
\flushleft{\begin{hindi}
तो फिर समझ लो कि ख़ुदा के सिवा कोई माबूद नहीं और (हम से) अपने और ईमानदार मर्दों और ईमानदार औरतों के गुनाहों की माफ़ी मांगते रहो और ख़ुदा तुम्हारे चलने फिरने और ठहरने से (ख़ूब) वाक़िफ़ है
\end{hindi}}
\flushright{\begin{Arabic}
\quranayah[47][20]
\end{Arabic}}
\flushleft{\begin{hindi}
और मोमिनीन कहते हैं कि (जेहाद के बारे में) कोई सूरा क्यों नहीं नाज़िल होता लेकिन जब कोई साफ़ सरीही मायनों का सूरा नाज़िल हुआ और उसमें जेहाद का बयान हो तो जिन लोगों के दिल में (नेफ़ाक़) का मर्ज़ है तुम उनको देखोगे कि तुम्हारी तरफ़ इस तरह देखते हैं जैसे किसी पर मौत की बेहोशी (छायी) हो (कि उसकी ऑंखें पथरा जाएं) तो उन पर वाए हो
\end{hindi}}
\flushright{\begin{Arabic}
\quranayah[47][21]
\end{Arabic}}
\flushleft{\begin{hindi}
(उनके लिए अच्छा काम तो) फरमाबरदारी और पसन्दीदा बात है फिर जब लड़ाई ठन जाए तो अगर ये लोग ख़ुदा से सच्चे रहें तो उनके हक़ में बहुत बेहतर है
\end{hindi}}
\flushright{\begin{Arabic}
\quranayah[47][22]
\end{Arabic}}
\flushleft{\begin{hindi}
(मुनाफ़िक़ों) क्या तुमसे कुछ दूर है कि अगर तुम हाक़िम बनो तो रूए ज़मीन में फसाद फैलाने और अपने रिश्ते नातों को तोड़ने लगो ये वही लोग हैं जिन पर ख़ुदा ने लानत की है
\end{hindi}}
\flushright{\begin{Arabic}
\quranayah[47][23]
\end{Arabic}}
\flushleft{\begin{hindi}
और (गोया ख़ुद उसने) उन (के कानों) को बहरा और ऑंखों को अंधा कर दिया है
\end{hindi}}
\flushright{\begin{Arabic}
\quranayah[47][24]
\end{Arabic}}
\flushleft{\begin{hindi}
तो क्या लोग क़ुरान में (ज़रा भी) ग़ौर नहीं करते या (उनके) दिलों पर ताले लगे हुए हैं
\end{hindi}}
\flushright{\begin{Arabic}
\quranayah[47][25]
\end{Arabic}}
\flushleft{\begin{hindi}
बेशक जो लोग राहे हिदायत साफ़ साफ़ मालूम होने के बाद उलटे पाँव (कुफ़्र की तरफ) फिर गये शैतान ने उन्हें (बुते देकर) ढील दे रखी है और उनकी (तमन्नाओं) की रस्सियाँ दराज़ कर दी हैं
\end{hindi}}
\flushright{\begin{Arabic}
\quranayah[47][26]
\end{Arabic}}
\flushleft{\begin{hindi}
यह इसलिए जो लोग ख़ुदा की नाज़िल की हुई(किताब) से बेज़ार हैं ये उनसे कहते हैं कि बाज़ कामों में हम तुम्हारी ही बात मानेंगे और ख़ुदा उनके पोशीदा मशवरों से वाक़िफ है
\end{hindi}}
\flushright{\begin{Arabic}
\quranayah[47][27]
\end{Arabic}}
\flushleft{\begin{hindi}
तो जब फ़रिश्तें उनकी जान निकालेंगे उस वक्त उनका क्या हाल होगा कि उनके चेहरों पर और उनकी पुश्त पर मारते जाएँगे
\end{hindi}}
\flushright{\begin{Arabic}
\quranayah[47][28]
\end{Arabic}}
\flushleft{\begin{hindi}
ये इस सबब से कि जिस चीज़ों से ख़ुदा नाख़ुश है उसकी तो ये लोग पैरवी करते हैं और जिसमें ख़ुदा की ख़ुशी है उससे बेज़ार हैं तो ख़ुदा ने भी उनकी कारस्तानियों को अकारत कर दिया
\end{hindi}}
\flushright{\begin{Arabic}
\quranayah[47][29]
\end{Arabic}}
\flushleft{\begin{hindi}
क्या वह लोग जिनके दिलों में (नेफ़ाक़ का) मर्ज़ है ये ख्याल करते हैं कि ख़ुदा दिल के कीनों को भी न ज़ाहिर करेगा
\end{hindi}}
\flushright{\begin{Arabic}
\quranayah[47][30]
\end{Arabic}}
\flushleft{\begin{hindi}
तो हम चाहते तो हम तुम्हें इन लोगों को दिखा देते तो तुम उनकी पेशानी ही से उनको पहचान लेते अगर तुम उन्हें उनके अन्दाज़े गुफ्तगू ही से ज़रूर पहचान लोगे और ख़ुदा तो तुम्हारे आमाल से वाक़िफ है
\end{hindi}}
\flushright{\begin{Arabic}
\quranayah[47][31]
\end{Arabic}}
\flushleft{\begin{hindi}
और हम तुम लोगों को ज़रूर आज़माएँगे ताकि तुममें जो लोग जेहाद करने वाले और (तकलीफ़) झेलने वाले हैं उनको देख लें और तुम्हारे हालात जाँच लें
\end{hindi}}
\flushright{\begin{Arabic}
\quranayah[47][32]
\end{Arabic}}
\flushleft{\begin{hindi}
बेशक जिन लोगों पर (दीन की) सीधी राह साफ़ ज़ाहिर हो गयी उसके बाद इन्कार कर बैठे और (लोगों को) ख़ुदा की राह से रोका और पैग़म्बर की मुख़ालेफ़त की तो ख़ुदा का कुछ भी नहीं बिगाड़ सकेंगे और वह उनका सब किया कराया अक़ारत कर देगा
\end{hindi}}
\flushright{\begin{Arabic}
\quranayah[47][33]
\end{Arabic}}
\flushleft{\begin{hindi}
ऐ ईमानदारों ख़ुदा का हुक्म मानों और रसूल की फरमाँबरदारी करो और अपने आमाल को ज़ाया न करो
\end{hindi}}
\flushright{\begin{Arabic}
\quranayah[47][34]
\end{Arabic}}
\flushleft{\begin{hindi}
बेशक जो लोग काफ़िर हो गए और लोगों को ख़ुदा की राह से रोका, फिर काफिर ही मर गए तो ख़ुदा उनको हरगिज़ नहीं बख्शेगा तो तुम हिम्मत न हारो
\end{hindi}}
\flushright{\begin{Arabic}
\quranayah[47][35]
\end{Arabic}}
\flushleft{\begin{hindi}
और (दुशमनों को)े सुलह की दावत न दो तुम ग़ालिब हो ही और ख़ुदा तो तुम्हारे साथ है और हरगिज़ तुम्हारे आमाल (के सवाब को कम न करेगा)
\end{hindi}}
\flushright{\begin{Arabic}
\quranayah[47][36]
\end{Arabic}}
\flushleft{\begin{hindi}
दुनियावी ज़िन्दगी तो बस खेल तमाशा है और अगर तुम (ख़ुदा पर) ईमान रखोगे और परहेज़गारी करोगे तो वह तुमको तुम्हारे अज्र इनायत फरमाएगा और तुमसे तुम्हारे माल नहीं तलब करेगा
\end{hindi}}
\flushright{\begin{Arabic}
\quranayah[47][37]
\end{Arabic}}
\flushleft{\begin{hindi}
और अगर वह तुमसे माल तलब करे और तुमसे चिमट कर माँगे भी तो तुम (ज़रूर) बुख्ल करने लगो
\end{hindi}}
\flushright{\begin{Arabic}
\quranayah[47][38]
\end{Arabic}}
\flushleft{\begin{hindi}
और ख़ुदा तो तुम्हारे कीने को ज़रूर ज़ाहिर करके रहेगा देखो तुम लोग वही तो हो कि ख़ुदा की राह में ख़र्च के लिए बुलाए जाते हो तो बाज़ तुम में ऐसे भी हैं जो बुख्ल करते हैं और (याद रहे कि) जो बुख्ल करता है तो ख़ुद अपने ही से बुख्ल करता है और ख़ुदा तो बेनियाज़ है और तुम (उसके) मोहताज हो और अगर तुम (ख़ुदा के हुक्म से) मुँह फेरोगे तो ख़ुदा (तुम्हारे सिवा) दूसरों बदल देगा और वह तुम्हारे ऐसे (बख़ील) न होंगे
\end{hindi}}
\chapter{Al-Fath (The Victory)}
\begin{Arabic}
\Huge{\centerline{\basmalah}}\end{Arabic}
\flushright{\begin{Arabic}
\quranayah[48][1]
\end{Arabic}}
\flushleft{\begin{hindi}
(ऐ रसूल) ये हुबैदिया की सुलह नहीं बल्कि हमने हक़ीक़तन तुमको खुल्लम खुल्ला फतेह अता की
\end{hindi}}
\flushright{\begin{Arabic}
\quranayah[48][2]
\end{Arabic}}
\flushleft{\begin{hindi}
ताकि ख़ुदा तुम्हारी उम्मत के अगले और पिछले गुनाह माफ़ कर दे और तुम पर अपनी नेअमत पूरी करे और तुम्हें सीधी राह पर साबित क़दम रखे
\end{hindi}}
\flushright{\begin{Arabic}
\quranayah[48][3]
\end{Arabic}}
\flushleft{\begin{hindi}
और ख़ुदा तुम्हारी ज़बरदस्त मदद करे
\end{hindi}}
\flushright{\begin{Arabic}
\quranayah[48][4]
\end{Arabic}}
\flushleft{\begin{hindi}
वह (वही) ख़ुदा तो है जिसने मोमिनीन के दिलों में तसल्ली नाज़िल फरमाई ताकि अपने (पहले) ईमान के साथ और ईमान को बढ़ाएँ और सारे आसमान व ज़मीन के लशकर तो ख़ुदा ही के हैं और ख़ुदा बड़ा वाक़िफकार हकीम है
\end{hindi}}
\flushright{\begin{Arabic}
\quranayah[48][5]
\end{Arabic}}
\flushleft{\begin{hindi}
ताकि मोमिन मर्द और मोमिना औरतों को (बेहिश्त) के बाग़ों में जा पहुँचाए जिनके नीचे नहरें जारी हैं और ये वहाँ हमेशा रहेंगे और उनके गुनाहों को उनसे दूर कर दे और ये ख़ुदा के नज़दीक बड़ी कामयाबी है
\end{hindi}}
\flushright{\begin{Arabic}
\quranayah[48][6]
\end{Arabic}}
\flushleft{\begin{hindi}
और मुनाफिक़ मर्द और मुनाफ़िक़ औरतों और मुशरिक मर्द और मुशरिक औरतों पर जो ख़ुदा के हक़ में बुरे बुरे ख्याल रखते हैं अज़ाब नाज़िल करे उन पर (मुसीबत की) बड़ी गर्दिश है (और ख़ुदा) उन पर ग़ज़बनाक है और उसने उस पर लानत की है और उनके लिए जहन्नुम को तैयार कर रखा है और वह (क्या) बुरी जगह है
\end{hindi}}
\flushright{\begin{Arabic}
\quranayah[48][7]
\end{Arabic}}
\flushleft{\begin{hindi}
और सारे आसमान व ज़मीन के लश्कर ख़ुदा ही के हैं और ख़ुदा तो बड़ा वाक़िफ़कार (और) ग़ालिब है
\end{hindi}}
\flushright{\begin{Arabic}
\quranayah[48][8]
\end{Arabic}}
\flushleft{\begin{hindi}
(ऐ रसूल) हमने तुमको (तमाम आलम का) गवाह और ख़ुशख़बरी देने वाला और धमकी देने वाला (पैग़म्बर बनाकर) भेजा
\end{hindi}}
\flushright{\begin{Arabic}
\quranayah[48][9]
\end{Arabic}}
\flushleft{\begin{hindi}
ताकि (मुसलमानों) तुम लोग ख़ुदा और उसके रसूल पर ईमान लाओ और उसकी मदद करो और उसको बुज़ुर्ग़ समझो और सुबह और शाम उसी की तस्बीह करो
\end{hindi}}
\flushright{\begin{Arabic}
\quranayah[48][10]
\end{Arabic}}
\flushleft{\begin{hindi}
बेशक जो लोग तुमसे बैयत करते हैं वह ख़ुदा ही से बैयत करते हैं ख़ुदा की क़ूवत (कुदरत तो बस सबकी कूवत पर) ग़ालिब है तो जो अहद को तोड़ेगा तो अपने अपने नुक़सान के लिए अहद तोड़ता है और जिसने उस बात को जिसका उसने ख़ुदा से अहद किया है पूरा किया तो उसको अनक़रीब ही अज्रे अज़ीम अता फ़रमाएगा
\end{hindi}}
\flushright{\begin{Arabic}
\quranayah[48][11]
\end{Arabic}}
\flushleft{\begin{hindi}
जो गंवार देहाती (हुदैबिया से) पीछे रह गए अब वह तुमसे कहेंगे कि हमको हमारे माल और लड़के वालों ने रोक रखा तो आप हमारे वास्ते (ख़ुदा से) मग़फिरत की दुआ माँगें ये लोग अपनी ज़बान से ऐसी बातें कहते हैं जो उनके दिल में नहीं (ऐ रसूल) तुम कह दो कि अगर ख़ुदा तुम लोगों को नुक़सान पहुँचाना चाहे या तुम्हें फायदा पहुँचाने का इरादा करे तो ख़ुदा के मुक़ाबले में तुम्हारे लिए किसका बस चल सकता है बल्कि जो कुछ तुम करते हो ख़ुदा उससे ख़ूब वाक़िफ है
\end{hindi}}
\flushright{\begin{Arabic}
\quranayah[48][12]
\end{Arabic}}
\flushleft{\begin{hindi}
(ये फ़क़त तुम्हारे हीले हैं) बात ये है कि तुम ये समझे बैठे थे कि रसूल और मोमिनीन हरगिज़ कभी अपने लड़के वालों में पलट कर आने ही के नहीं (और सब मार डाले जाएँगे) और यही बात तुम्हारे दिलों में भी खप गयी थी और इसी वजह से, तुम तरह तरह की बदगुमानियाँ करने लगे थे और (आख़िरकार) तुम लोग आप बरबाद हुए
\end{hindi}}
\flushright{\begin{Arabic}
\quranayah[48][13]
\end{Arabic}}
\flushleft{\begin{hindi}
और जो शख़्श ख़ुदा और उसके रसूल पर ईमान न लाए तो हमने (ऐसे) काफ़िरों के लिए जहन्नुम की आग तैयार कर रखी है
\end{hindi}}
\flushright{\begin{Arabic}
\quranayah[48][14]
\end{Arabic}}
\flushleft{\begin{hindi}
और सारे आसमान व ज़मीन की बादशाहत ख़ुदा ही की है जिसे चाहे बख्श दे और जिसे चाहे सज़ा दे और ख़ुदा तो बड़ा बख्शने वाला मेहरबान है
\end{hindi}}
\flushright{\begin{Arabic}
\quranayah[48][15]
\end{Arabic}}
\flushleft{\begin{hindi}
(मुसलमानों) अब जो तुम (ख़ैबर की) ग़नीमतों के लेने को जाने लगोगे तो जो लोग (हुदैबिया से) पीछे रह गये थे तुम से कहेंगे कि हमें भी अपने साथ चलने दो ये चाहते हैं कि ख़ुदा के क़ौल को बदल दें तुम (साफ) कह दो कि तुम हरगिज़ हमारे साथ नहीं चलने पाओगे ख़ुदा ने पहले ही से ऐसा फ़रमा दिया है तो ये लोग कहेंगे कि तुम लोग तो हमसे हसद रखते हो (ख़ुदा ऐसा क्या कहेगा) बात ये है कि ये लोग बहुत ही कम समझते हैं
\end{hindi}}
\flushright{\begin{Arabic}
\quranayah[48][16]
\end{Arabic}}
\flushleft{\begin{hindi}
कि जो गवॉर पीछे रह गए थे उनसे कह दो कि अनक़रीब ही तुम सख्त जंगजू क़ौम के (साथ लड़ने के लिए) बुलाए जाओगे कि तुम (या तो) उनसे लड़ते ही रहोगे या मुसलमान ही हो जाएँगे पस अगर तुम (ख़ुदा का) हुक्म मानोगे तो ख़ुदा तुमको अच्छा बदला देगा और अगर तुमने जिस तरह पहली दफा सरताबी की थी अब भी सरताबी करोगे तो वह तुमको दर्दनाक अज़ाब की सज़ा देगा
\end{hindi}}
\flushright{\begin{Arabic}
\quranayah[48][17]
\end{Arabic}}
\flushleft{\begin{hindi}
(जेहाद से पीछे रह जाने का) न तो अन्धे ही पर कुछ गुनाह है और न लँगड़े पर गुनाह है और न बीमार पर गुनाह है और जो शख़्श ख़ुदा और उसके रसूल का हुक्म मानेगा तो वह उसको (बेहिश्त के) उन सदाबहार बाग़ों में दाख़िल करेगा जिनके नीचे नहरें जारी होंगी और जो सरताबी करेगा वह उसको दर्दनाक अज़ाब की सज़ा देगा
\end{hindi}}
\flushright{\begin{Arabic}
\quranayah[48][18]
\end{Arabic}}
\flushleft{\begin{hindi}
जिस वक्त मोमिनीन तुमसे दरख्त के नीचे (लड़ने मरने) की बैयत कर रहे थे तो ख़ुदा उनसे इस (बात पर) ज़रूर ख़ुश हुआ ग़रज़ जो कुछ उनके दिलों में था ख़ुदा ने उसे देख लिया फिर उन पर तस्सली नाज़िल फरमाई और उन्हें उसके एवज़ में बहुत जल्द फ़तेह इनायत की
\end{hindi}}
\flushright{\begin{Arabic}
\quranayah[48][19]
\end{Arabic}}
\flushleft{\begin{hindi}
और (इसके अलावा) बहुत सी ग़नीमतें (भी) जो उन्होने हासिल की (अता फरमाई) और ख़ुदा तो ग़ालिब (और) हिकमत वाला है
\end{hindi}}
\flushright{\begin{Arabic}
\quranayah[48][20]
\end{Arabic}}
\flushleft{\begin{hindi}
ख़ुदा ने तुमसे बहुत सी ग़नीमतों का वायदा फरमाया था कि तुम उन पर काबिज़ हो गए तो उसने तुम्हें ये (ख़ैबर की ग़नीमत) जल्दी से दिलवा दीं और (हुबैदिया से) लोगों की दराज़ी को तुमसे रोक दिया और ग़रज़ ये थी कि मोमिनीन के लिए (क़ुदरत) का नमूना हो और ख़ुदा तुमको सीधी राह पर ले चले
\end{hindi}}
\flushright{\begin{Arabic}
\quranayah[48][21]
\end{Arabic}}
\flushleft{\begin{hindi}
और दूसरी (ग़नीमतें भी दी) जिन पर तुम क़ुदरत नहीं रखते थे (और) ख़ुदा ही उन पर हावी था और ख़ुदा तो हर चीज़ पर क़ादिर है
\end{hindi}}
\flushright{\begin{Arabic}
\quranayah[48][22]
\end{Arabic}}
\flushleft{\begin{hindi}
(और) अगर कुफ्फ़ार तुमसे लड़ते तो ज़रूर पीठ फेर कर भाग जाते फिर वह न (अपना) किसी को सरपरस्त ही पाते न मददगार
\end{hindi}}
\flushright{\begin{Arabic}
\quranayah[48][23]
\end{Arabic}}
\flushleft{\begin{hindi}
यही ख़ुदा की आदत है जो पहले ही से चली आती है और तुम ख़ुदा की आदत को बदलते न देखोगे
\end{hindi}}
\flushright{\begin{Arabic}
\quranayah[48][24]
\end{Arabic}}
\flushleft{\begin{hindi}
और वह वही तो है जिसने तुमको उन कुफ्फ़ार पर फ़तेह देने के बाद मक्के की सरहद पर उनके हाथ तुमसे और तुम्हारे हाथ उनसे रोक दिए और तुम लोग जो कुछ भी करते थे ख़ुदा उसे देख रहा था
\end{hindi}}
\flushright{\begin{Arabic}
\quranayah[48][25]
\end{Arabic}}
\flushleft{\begin{hindi}
ये वही लोग तो हैं जिन्होने कुफ़्र किया और तुमको मस्जिदुल हराम (में जाने) से रोका और क़ुरबानी के जानवरों को भी (न आने दिया) कि वह अपनी (मुक़र्रर) जगह (में) पहुँचने से रूके रहे और अगर कुछ ऐसे ईमानदार मर्द और ईमानदार औरतें न होती जिनसे तुम वाक़िफ न थे कि तुम उनको (लड़ाई में कुफ्फ़ार के साथ) पामाल कर डालते पस तुमको उनकी तरफ से बेख़बरी में नुकसान पहँच जाता (तो उसी वक्त तुमको फतेह हुई मगर ताख़ीर) इसलिए (हुई) कि ख़ुदा जिसे चाहे अपनी रहमत में दाख़िल करे और अगर वह (ईमानदार कुफ्फ़ार से) अलग हो जाते तो उनमें से जो लोग काफ़िर थे हम उन्हें दर्दनाक अज़ाब की ज़रूर सज़ा देते
\end{hindi}}
\flushright{\begin{Arabic}
\quranayah[48][26]
\end{Arabic}}
\flushleft{\begin{hindi}
(ये वह वक्त) था जब काफ़िरों ने अपने दिलों में ज़िद ठान ली थी और ज़िद भी तो जाहिलियत की सी तो ख़ुदा ने अपने रसूल और मोमिनीन (के दिलों) पर अपनी तरफ़ से तसकीन नाज़िल फ़रमाई और उनको परहेज़गारी की बात पर क़ायम रखा और ये लोग उसी के सज़ावार और अहल भी थे और ख़ुदा तो हर चीज़ से ख़बरदार है
\end{hindi}}
\flushright{\begin{Arabic}
\quranayah[48][27]
\end{Arabic}}
\flushleft{\begin{hindi}
बेशक ख़ुदा ने अपने रसूल को सच्चा मुताबिके वाक़ेया ख्वाब दिखाया था कि तुम लोग इन्शाअल्लाह मस्जिदुल हराम में अपने सर मुँडवा कर और अपने थोड़े से बाल कतरवा कर बहुत अमन व इत्मेनान से दाख़िल होंगे (और) किसी तरह का ख़ौफ न करोगे तो जो बात तुम नहीं जानते थे उसको मालूम थी तो उसने फ़तेह मक्का से पहले ही बहुत जल्द फतेह अता की
\end{hindi}}
\flushright{\begin{Arabic}
\quranayah[48][28]
\end{Arabic}}
\flushleft{\begin{hindi}
वह वही तो है जिसने अपने रसूल को हिदायत और सच्चा दीन देकर भेजा ताकि उसको तमाम दीनों पर ग़ालिब रखे और गवाही के लिए तो बस ख़ुदा ही काफ़ी है
\end{hindi}}
\flushright{\begin{Arabic}
\quranayah[48][29]
\end{Arabic}}
\flushleft{\begin{hindi}
मोहम्मद ख़ुदा के रसूल हैं और जो लोग उनके साथ हैं काफ़िरों पर बड़े सख्त और आपस में बड़े रहम दिल हैं तू उनको देखेगा (कि ख़ुदा के सामने) झुके सर बसजूद हैं ख़ुदा के फज़ल और उसकी ख़ुशनूदी के ख्वास्तगार हैं कसरते सुजूद के असर से उनकी पेशानियों में घट्टे पड़े हुए हैं यही औसाफ़ उनके तौरेत में भी हैं और यही हालात इंजील में (भी मज़कूर) हैं गोया एक खेती है जिसने (पहले ज़मीन से) अपनी सूई निकाली फिर (अजज़ा ज़मीन को गेज़ा बनाकर) उसी सूई को मज़बूत किया तो वह मोटी हुई फिर अपनी जड़ पर सीधी खड़ी हो गयी और अपनी ताज़गी से किसानों को ख़ुश करने लगी और इतनी जल्दी तरक्क़ी इसलिए दी ताकि उनके ज़रिए काफ़िरों का जी जलाएँ जो लोग ईमान लाए और अच्छे (अच्छे) काम करते रहे ख़ुदा ने उनसे बख़्शिस और अज्रे अज़ीम का वायदा किया है
\end{hindi}}
\chapter{Al-Hujurat (The Apartments)}
\begin{Arabic}
\Huge{\centerline{\basmalah}}\end{Arabic}
\flushright{\begin{Arabic}
\quranayah[49][1]
\end{Arabic}}
\flushleft{\begin{hindi}
ऐ ईमानदारों ख़ुदा और उसके रसूल के सामने किसी बात में आगे न बढ़ जाया करो और ख़ुदा से डरते रहो बेशक ख़ुदा बड़ा सुनने वाला वाक़िफ़कार है
\end{hindi}}
\flushright{\begin{Arabic}
\quranayah[49][2]
\end{Arabic}}
\flushleft{\begin{hindi}
ऐ ईमानदारों (बोलने में) अपनी आवाज़े पैग़म्बर की आवाज़ से ऊँची न किया करो और जिस तरह तुम आपस में एक दूसरे से ज़ोर (ज़ोर) से बोला करते हो उनके रूबरू ज़ोर से न बोला करो (ऐसा न हो कि) तुम्हारा किया कराया सब अकारत हो जाए और तुमको ख़बर भी न हो
\end{hindi}}
\flushright{\begin{Arabic}
\quranayah[49][3]
\end{Arabic}}
\flushleft{\begin{hindi}
बेशक जो लोग रसूले ख़ुदा के सामने अपनी आवाज़ें धीमी कर लिया करते हैं यही लोग हैं जिनके दिलों को ख़ुदा ने परहेज़गारी के लिए जाँच लिया है उनके लिए (आख़ेरत में) बख़्शिस और बड़ा अज्र है
\end{hindi}}
\flushright{\begin{Arabic}
\quranayah[49][4]
\end{Arabic}}
\flushleft{\begin{hindi}
(ऐ रसूल) जो लोग तुमको हुजरों के बाहर से आवाज़ देते हैं उनमें के अक्सर बे अक्ल हैं
\end{hindi}}
\flushright{\begin{Arabic}
\quranayah[49][5]
\end{Arabic}}
\flushleft{\begin{hindi}
और अगर ये लोग इतना ताम्मुल करते कि तुम ख़ुद निकल कर उनके पास आ जाते (तब बात करते) तो ये उनके लिए बेहतर था और ख़ुदा तो बड़ा बख्शने वाला मेहरबान है
\end{hindi}}
\flushright{\begin{Arabic}
\quranayah[49][6]
\end{Arabic}}
\flushleft{\begin{hindi}
ऐ ईमानदारों अगर कोई बदकिरदार तुम्हारे पास कोई ख़बर लेकर आए तो ख़ूब तहक़ीक़ कर लिया करो (ऐसा न हो) कि तुम किसी क़ौम को नादानी से नुक़सान पहुँचाओ फिर अपने किए पर नादिम हो
\end{hindi}}
\flushright{\begin{Arabic}
\quranayah[49][7]
\end{Arabic}}
\flushleft{\begin{hindi}
और जान रखो कि तुम में ख़ुदा के पैग़म्बर (मौजूद) हैं बहुत सी बातें ऐसी हैं कि अगर रसूल उनमें तुम्हारा कहा मान लिया करें तो (उलटे) तुम ही मुश्किल में पड़ जाओ लेकिन ख़ुदा ने तुम्हें ईमान की मोहब्बत दे दी है और उसको तुम्हारे दिलों में उमदा कर दिखाया है और कुफ़्र और बदकारी और नाफ़रमानी से तुमको बेज़ार कर दिया है यही लोग ख़ुदा के फ़ज़ल व एहसान से राहे हिदायत पर हैं
\end{hindi}}
\flushright{\begin{Arabic}
\quranayah[49][8]
\end{Arabic}}
\flushleft{\begin{hindi}
और ख़ुदा तो बड़ा वाक़िफ़कार और हिकमत वाला है
\end{hindi}}
\flushright{\begin{Arabic}
\quranayah[49][9]
\end{Arabic}}
\flushleft{\begin{hindi}
और अगर मोमिनीन में से दो फिरक़े आपस में लड़ पड़े तो उन दोनों में सुलह करा दो फिर अगर उनमें से एक (फ़रीक़) दूसरे पर ज्यादती करे तो जो (फिरक़ा) ज्यादती करे तुम (भी) उससे लड़ो यहाँ तक वह ख़ुदा के हुक्म की तरफ रूझू करे फिर जब रूजू करे तो फरीकैन में मसावात के साथ सुलह करा दो और इन्साफ़ से काम लो बेशक ख़ुदा इन्साफ़ करने वालों को दोस्त रखता है
\end{hindi}}
\flushright{\begin{Arabic}
\quranayah[49][10]
\end{Arabic}}
\flushleft{\begin{hindi}
मोमिनीन तो आपस में बस भाई भाई हैं तो अपने दो भाईयों में मेल जोल करा दिया करो और ख़ुदा से डरते रहो ताकि तुम पर रहम किया जाए
\end{hindi}}
\flushright{\begin{Arabic}
\quranayah[49][11]
\end{Arabic}}
\flushleft{\begin{hindi}
ऐ ईमानदारों (तुम किसी क़ौम का) कोई मर्द ( दूसरी क़ौम के मर्दों की हँसी न उड़ाये मुमकिन है कि वह लोग (ख़ुदा के नज़दीक) उनसे अच्छे हों और न औरते औरतों से (तमसख़ुर करें) क्या अजब है कि वह उनसे अच्छी हों और तुम आपस में एक दूसरे को मिलने न दो न एक दूसरे का बुरा नाम धरो ईमान लाने के बाद बदकारी (का) नाम ही बुरा है और जो लोग बाज़ न आएँ तो ऐसे ही लोग ज़ालिम हैं
\end{hindi}}
\flushright{\begin{Arabic}
\quranayah[49][12]
\end{Arabic}}
\flushleft{\begin{hindi}
ऐ ईमानदारों बहुत से गुमान (बद) से बचे रहो क्यों कि बाज़ बदगुमानी गुनाह हैं और आपस में एक दूसरे के हाल की टोह में न रहा करो और न तुममें से एक दूसरे की ग़ीबत करे क्या तुममें से कोई इस बात को पसन्द करेगा कि अपने मरे हुए भाई का गोश्त खाए तो तुम उससे (ज़रूर) नफरत करोगे और ख़ुदा से डरो, बेशक ख़ुदा बड़ा तौबा क़ुबूल करने वाला मेहरबान है
\end{hindi}}
\flushright{\begin{Arabic}
\quranayah[49][13]
\end{Arabic}}
\flushleft{\begin{hindi}
लोगों हमने तो तुम सबको एक मर्द और एक औरत से पैदा किया और हम ही ने तुम्हारे कबीले और बिरादरियाँ बनायीं ताकि एक दूसरे की शिनाख्त करे इसमें शक़ नहीं कि ख़ुदा के नज़दीक तुम सबमें बड़ा इज्ज़तदार वही है जो बड़ा परहेज़गार हो बेशक ख़ुदा बड़ा वाक़िफ़कार ख़बरदार है
\end{hindi}}
\flushright{\begin{Arabic}
\quranayah[49][14]
\end{Arabic}}
\flushleft{\begin{hindi}
अरब के देहाती कहते हैं कि हम ईमान लाए (ऐ रसूल) तुम कह दो कि तुम ईमान नहीं लाए बल्कि (यूँ) कह दो कि इस्लाम लाए हालॉकि ईमान का अभी तक तुम्हारे दिल में गुज़र हुआ ही नहीं और अगर तुम ख़ुदा की और उसके रसूल की फरमाबरदारी करोगे तो ख़ुदा तुम्हारे आमाल में से कुछ कम नहीं करेगा - बेशक ख़ुदा बड़ा बख्शने वाला मेहरबान है
\end{hindi}}
\flushright{\begin{Arabic}
\quranayah[49][15]
\end{Arabic}}
\flushleft{\begin{hindi}
(सच्चे मोमिन) तो बस वही हैं जो ख़ुदा और उसके रसूल पर ईमान लाए फिर उन्होंने उसमें किसी तरह का शक़ शुबह न किया और अपने माल से और अपनी जानों से ख़ुदा की राह में जेहाद किया यही लोग (दावाए ईमान में) सच्चे हैं
\end{hindi}}
\flushright{\begin{Arabic}
\quranayah[49][16]
\end{Arabic}}
\flushleft{\begin{hindi}
(ऐ रसूल इनसे) पूछो तो कि क्या तुम ख़ुदा को अपनी दीदारी जताते हो और ख़ुदा तो जो कुछ आसमानों मे है और जो कुछ ज़मीन में है (ग़रज़ सब कुछ) जानता है और ख़ुदा हर चीज़ से ख़बरदार है
\end{hindi}}
\flushright{\begin{Arabic}
\quranayah[49][17]
\end{Arabic}}
\flushleft{\begin{hindi}
(ऐ रसूल) तुम पर ये लोग (इसलाम लाने का) एहसान जताते हैं तुम (साफ़) कह दो कि तुम अपने इसलाम का मुझ पर एहसान न जताओ (बल्कि) अगर तुम (दावाए ईमान में) सच्चे हो तो समझो कि, ख़ुदा ने तुम पर एहसान किया कि उसने तुमको ईमान का रास्ता दिखाया
\end{hindi}}
\flushright{\begin{Arabic}
\quranayah[49][18]
\end{Arabic}}
\flushleft{\begin{hindi}
बेशक ख़ुदा तो सारे आसमानों और ज़मीन की छिपी हुई बातों को जानता है और जो तुम करते हो ख़ुदा उसे देख रहा है
\end{hindi}}
\chapter{Qaf (Qaf)}
\begin{Arabic}
\Huge{\centerline{\basmalah}}\end{Arabic}
\flushright{\begin{Arabic}
\quranayah[50][1]
\end{Arabic}}
\flushleft{\begin{hindi}
क़ाफ़ क़ुरान मजीद की क़सम (मोहम्मद पैग़म्बर हैं)
\end{hindi}}
\flushright{\begin{Arabic}
\quranayah[50][2]
\end{Arabic}}
\flushleft{\begin{hindi}
लेकिन इन (काफिरों) को ताज्जुब है कि उन ही में एक (अज़ाब से) डराने वाला (पैग़म्बर) उनके पास आ गया तो कुफ्फ़ार कहने लगे ये तो एक अजीब बात है
\end{hindi}}
\flushright{\begin{Arabic}
\quranayah[50][3]
\end{Arabic}}
\flushleft{\begin{hindi}
भला जब हम मर जाएँगे और (सड़ गल कर) मिटटी हो जाएँगे तो फिर ये दोबार ज़िन्दा होना (अक्ल से) बईद (बात है)
\end{hindi}}
\flushright{\begin{Arabic}
\quranayah[50][4]
\end{Arabic}}
\flushleft{\begin{hindi}
उनके जिस्मों से ज़मीन जिस चीज़ को (खा खा कर) कम करती है वह हमको मालूम है और हमारे पास तो तहरीरी याददाश्त किताब लौहे महफूज़ मौजूद है
\end{hindi}}
\flushright{\begin{Arabic}
\quranayah[50][5]
\end{Arabic}}
\flushleft{\begin{hindi}
मगर जब उनके पास दीन (हक़) आ पहुँचा तो उन्होने उसे झुठलाया तो वह लोग एक ऐसी बात में उलझे हुए हैं जिसे क़रार नहीं
\end{hindi}}
\flushright{\begin{Arabic}
\quranayah[50][6]
\end{Arabic}}
\flushleft{\begin{hindi}
तो क्या इन लोगों ने अपने ऊपर आसमान की नज़र नहीं की कि हमने उसको क्यों कर बनाया और उसको कैसी ज़ीनत दी और उनसे कहीं शिगाफ्त तक नहीं
\end{hindi}}
\flushright{\begin{Arabic}
\quranayah[50][7]
\end{Arabic}}
\flushleft{\begin{hindi}
और ज़मीन को हमने फैलाया और उस पर बोझल पहाड़ रख दिये और इसमें हर तरह की ख़ुशनुमा चीज़ें उगाई ताकि तमाम रूजू लाने वाले
\end{hindi}}
\flushright{\begin{Arabic}
\quranayah[50][8]
\end{Arabic}}
\flushleft{\begin{hindi}
(बन्दे) हिदायत और इबरत हासिल करें
\end{hindi}}
\flushright{\begin{Arabic}
\quranayah[50][9]
\end{Arabic}}
\flushleft{\begin{hindi}
और हमने आसमान से बरकत वाला पानी बरसाया तो उससे बाग़ (के दरख्त) उगाए और खेती का अनाज और लम्बी लम्बी खजूरें
\end{hindi}}
\flushright{\begin{Arabic}
\quranayah[50][10]
\end{Arabic}}
\flushleft{\begin{hindi}
जिसका बौर बाहम गुथा हुआ है
\end{hindi}}
\flushright{\begin{Arabic}
\quranayah[50][11]
\end{Arabic}}
\flushleft{\begin{hindi}
(ये सब कुछ) बन्दों की रोज़ी देने के लिए (पैदा किया) और पानी ही से हमने मुर्दा शहर (उफ़तादा ज़मीन) को ज़िन्दा किया
\end{hindi}}
\flushright{\begin{Arabic}
\quranayah[50][12]
\end{Arabic}}
\flushleft{\begin{hindi}
इसी तरह (क़यामत में मुर्दों को) निकलना होगा उनसे पहले नूह की क़ौम और ख़न्दक़ वालों और (क़ौम) समूद ने अपने अपने पैग़म्बरों को झुठलाया
\end{hindi}}
\flushright{\begin{Arabic}
\quranayah[50][13]
\end{Arabic}}
\flushleft{\begin{hindi}
और (क़ौम) आद और फिरऔन और लूत की क़ौम
\end{hindi}}
\flushright{\begin{Arabic}
\quranayah[50][14]
\end{Arabic}}
\flushleft{\begin{hindi}
और बन के रहने वालों (क़ौम शुऐब) और तुब्बा की क़ौम और (उन) सबने अपने (अपने) पैग़म्बरों को झुठलाया तो हमारा (अज़ाब का) वायदा पूरा हो कर रहा
\end{hindi}}
\flushright{\begin{Arabic}
\quranayah[50][15]
\end{Arabic}}
\flushleft{\begin{hindi}
तो क्या हम पहली बार पैदा करके थक गये हैं (हरगिज़ नहीं) मगर ये लोग अज़ सरे नौ (दोबारा) पैदा करने की निस्बत शक़ में पड़े हैं
\end{hindi}}
\flushright{\begin{Arabic}
\quranayah[50][16]
\end{Arabic}}
\flushleft{\begin{hindi}
और बेशक हम ही ने इन्सान को पैदा किया और जो ख्यालात उसके दिल में गुज़रते हैं हम उनको जानते हैं और हम तो उसकी शहरग से भी ज्यादा क़रीब हैं
\end{hindi}}
\flushright{\begin{Arabic}
\quranayah[50][17]
\end{Arabic}}
\flushleft{\begin{hindi}
जब (वह कोई काम करता हैं तो) दो लिखने वाले (केरामन क़ातेबीन) जो उसके दाहिने बाएं बैठे हैं लिख लेते हैं
\end{hindi}}
\flushright{\begin{Arabic}
\quranayah[50][18]
\end{Arabic}}
\flushleft{\begin{hindi}
कोई बात उसकी ज़बान पर नहीं आती मगर एक निगेहबान उसके पास तैयार रहता है
\end{hindi}}
\flushright{\begin{Arabic}
\quranayah[50][19]
\end{Arabic}}
\flushleft{\begin{hindi}
मौत की बेहोशी यक़ीनन तारी होगी (जो हम बता देंगे कि) यही तो वह (हालात है) जिससे तू भागा करता था
\end{hindi}}
\flushright{\begin{Arabic}
\quranayah[50][20]
\end{Arabic}}
\flushleft{\begin{hindi}
और सूर फूँका जाएगा यही (अज़ाब) के वायदे का दिन है और हर शख़्श (हमारे सामने) (इस तरह) हाज़िर होगा
\end{hindi}}
\flushright{\begin{Arabic}
\quranayah[50][21]
\end{Arabic}}
\flushleft{\begin{hindi}
कि उसके साथ एक (फरिश्ता) हॅका लाने वाला होगा
\end{hindi}}
\flushright{\begin{Arabic}
\quranayah[50][22]
\end{Arabic}}
\flushleft{\begin{hindi}
और एक (आमाल का) गवाह उससे कहा जाएगा कि उस (दिन) से तू ग़फ़लत में पड़ा था तो अब हमने तेरे सामने से पर्दे को हटा दिया तो आज तेरी निगाह बड़ी तेज़ है
\end{hindi}}
\flushright{\begin{Arabic}
\quranayah[50][23]
\end{Arabic}}
\flushleft{\begin{hindi}
और उसका साथी (फ़रिश्ता) कहेगा ये (उसका अमल) जो मेरे पास है
\end{hindi}}
\flushright{\begin{Arabic}
\quranayah[50][24]
\end{Arabic}}
\flushleft{\begin{hindi}
(तब हुक्म होगा कि) तुम दोनों हर सरकश नाशुक्रे को दोज़ख़ में डाल दो
\end{hindi}}
\flushright{\begin{Arabic}
\quranayah[50][25]
\end{Arabic}}
\flushleft{\begin{hindi}
जो (वाजिब हुकूक से) माल में बुख्ल करने वाला हद से बढ़ने वाला (दीन में) शक़ करने वाला था
\end{hindi}}
\flushright{\begin{Arabic}
\quranayah[50][26]
\end{Arabic}}
\flushleft{\begin{hindi}
जिसने ख़ुदा के साथ दूसरे माबूद बना रखे थे तो अब तुम दोनों इसको सख्त अज़ाब में डाल ही दो
\end{hindi}}
\flushright{\begin{Arabic}
\quranayah[50][27]
\end{Arabic}}
\flushleft{\begin{hindi}
(उस वक्त) उसका साथी (शैतान) कहेगा परवरदिगार हमने इसको गुमराह नहीं किया था बल्कि ये तो ख़ुद सख्त गुमराही में मुब्तिला था
\end{hindi}}
\flushright{\begin{Arabic}
\quranayah[50][28]
\end{Arabic}}
\flushleft{\begin{hindi}
इस पर ख़ुदा फ़रमाएगा हमारे सामने झगड़े न करो मैं तो तुम लोगों को पहले ही (अज़ाब से) डरा चुका था
\end{hindi}}
\flushright{\begin{Arabic}
\quranayah[50][29]
\end{Arabic}}
\flushleft{\begin{hindi}
मेरे यहाँ बात बदला नहीं करती और न मैं बन्दों पर (ज़र्रा बराबर) ज़ुल्म करने वाला हूँ
\end{hindi}}
\flushright{\begin{Arabic}
\quranayah[50][30]
\end{Arabic}}
\flushleft{\begin{hindi}
उस दिन हम दोज़ख़ से पूछेंगे कि तू भर चुकी और वह कहेगी क्या कुछ और भी हैं
\end{hindi}}
\flushright{\begin{Arabic}
\quranayah[50][31]
\end{Arabic}}
\flushleft{\begin{hindi}
और बेहिश्त परहेज़गारों के बिलकुल करीब कर दी जाएगी
\end{hindi}}
\flushright{\begin{Arabic}
\quranayah[50][32]
\end{Arabic}}
\flushleft{\begin{hindi}
यही तो वह बेहिश्त है जिसका तुममें से हर एक (ख़ुदा की तरफ़) रूजू करने वाले (हुदूद की) हिफाज़त करने वाले से वायदा किया जाता है
\end{hindi}}
\flushright{\begin{Arabic}
\quranayah[50][33]
\end{Arabic}}
\flushleft{\begin{hindi}
तो जो शख़्श ख़ुदा से बे देखे डरता रहा और ख़ुदा की तरफ़ रूजू करने वाला दिल लेकर आया
\end{hindi}}
\flushright{\begin{Arabic}
\quranayah[50][34]
\end{Arabic}}
\flushleft{\begin{hindi}
(उसको हुक्म होगा कि) इसमें सही सलामत दाख़िल हो जाओ यहीं तो हमेशा रहने का दिन है
\end{hindi}}
\flushright{\begin{Arabic}
\quranayah[50][35]
\end{Arabic}}
\flushleft{\begin{hindi}
इसमें ये लोग जो चाहेंगे उनके लिए हाज़िर है और हमारे यहॉ तो इससे भी ज्यादा है
\end{hindi}}
\flushright{\begin{Arabic}
\quranayah[50][36]
\end{Arabic}}
\flushleft{\begin{hindi}
और हमने तो इनसे पहले कितनी उम्मतें हलाक कर डाली जो इनसे क़ूवत में कहीं बढ़ कर थीं तो उन लोगों ने (मौत के ख़ौफ से) तमाम शहरों को छान मारा कि भला कहीं भी भागने का ठिकाना है
\end{hindi}}
\flushright{\begin{Arabic}
\quranayah[50][37]
\end{Arabic}}
\flushleft{\begin{hindi}
इसमें शक़ नहीं कि जो शख़्श (आगाह) दिल रखता है या कान लगाकर हुज़ूरे क़ल्ब से सुनता है उसके लिए इसमें (काफ़ी) नसीहत है
\end{hindi}}
\flushright{\begin{Arabic}
\quranayah[50][38]
\end{Arabic}}
\flushleft{\begin{hindi}
और हमने ही यक़ीनन सारे आसमान और ज़मीन और जो कुछ उन दोनों के बीच में है छह: दिन में पैदा किए और थकान तो हमको छुकर भी नहीं गयी
\end{hindi}}
\flushright{\begin{Arabic}
\quranayah[50][39]
\end{Arabic}}
\flushleft{\begin{hindi}
तो (ऐ रसूल) जो कुछ ये (काफ़िर) लोग किया करते हैं उस पर तुम सब्र करो और आफ़ताब के निकलने से पहले अपने परवरदिगार के हम्द की तस्बीह किया करो
\end{hindi}}
\flushright{\begin{Arabic}
\quranayah[50][40]
\end{Arabic}}
\flushleft{\begin{hindi}
और थोड़ी देर रात को भी और नमाज़ के बाद भी उसकी तस्बीह करो
\end{hindi}}
\flushright{\begin{Arabic}
\quranayah[50][41]
\end{Arabic}}
\flushleft{\begin{hindi}
और कान लगा कर सुन रखो कि जिस दिन पुकारने वाला (इसराफ़ील) नज़दीक ही जगह से आवाज़ देगा
\end{hindi}}
\flushright{\begin{Arabic}
\quranayah[50][42]
\end{Arabic}}
\flushleft{\begin{hindi}
(कि उठो) जिस दिन लोग एक सख्त चीख़ को बाख़ूबी सुन लेगें वही दिन (लोगों) के कब्रों से निकलने का होगा
\end{hindi}}
\flushright{\begin{Arabic}
\quranayah[50][43]
\end{Arabic}}
\flushleft{\begin{hindi}
बेशक हम ही (लोगों को) ज़िन्दा करते हैं और हम ही मारते हैं
\end{hindi}}
\flushright{\begin{Arabic}
\quranayah[50][44]
\end{Arabic}}
\flushleft{\begin{hindi}
और हमारी ही तरफ फिर कर आना है जिस दिन ज़मीन (उनके ऊपर से) फट जाएगी और ये झट पट निकल खड़े होंगे ये उठाना और जमा करना
\end{hindi}}
\flushright{\begin{Arabic}
\quranayah[50][45]
\end{Arabic}}
\flushleft{\begin{hindi}
और हम पर बहुत आसान है (ऐ रसूल) ये लोग जो कुछ कहते हैं हम (उसे) ख़ूब जानते हैं और तुम उन पर जब्र तो देते नहीं हो तो जो हमारे (अज़ाब के) वायदे से डरे उसको तुम क़ुरान के ज़रिए नसीहत करते रहो
\end{hindi}}
\chapter{Ad-Dhariyat (The Scatterers)}
\begin{Arabic}
\Huge{\centerline{\basmalah}}\end{Arabic}
\flushright{\begin{Arabic}
\quranayah[51][1]
\end{Arabic}}
\flushleft{\begin{hindi}
उन (हवाओं की क़सम) जो (बादलों को) उड़ा कर तितर बितर कर देती हैं
\end{hindi}}
\flushright{\begin{Arabic}
\quranayah[51][2]
\end{Arabic}}
\flushleft{\begin{hindi}
फिर (पानी का) बोझ उठाती हैं
\end{hindi}}
\flushright{\begin{Arabic}
\quranayah[51][3]
\end{Arabic}}
\flushleft{\begin{hindi}
फिर आहिस्ता आहिस्ता चलती हैं
\end{hindi}}
\flushright{\begin{Arabic}
\quranayah[51][4]
\end{Arabic}}
\flushleft{\begin{hindi}
फिर एक ज़रूरी चीज़ (बारिश) को तक़सीम करती हैं
\end{hindi}}
\flushright{\begin{Arabic}
\quranayah[51][5]
\end{Arabic}}
\flushleft{\begin{hindi}
कि तुम से जो वायदा किया जाता है ज़रूर बिल्कुल सच्चा है
\end{hindi}}
\flushright{\begin{Arabic}
\quranayah[51][6]
\end{Arabic}}
\flushleft{\begin{hindi}
और (आमाल की) जज़ा (सज़ा) ज़रूर होगी
\end{hindi}}
\flushright{\begin{Arabic}
\quranayah[51][7]
\end{Arabic}}
\flushleft{\begin{hindi}
और आसमान की क़सम जिसमें रहते हैं
\end{hindi}}
\flushright{\begin{Arabic}
\quranayah[51][8]
\end{Arabic}}
\flushleft{\begin{hindi}
कि (ऐ अहले मक्का) तुम लोग एक ऐसी मुख्तलिफ़ बेजोड़ बात में पड़े हो
\end{hindi}}
\flushright{\begin{Arabic}
\quranayah[51][9]
\end{Arabic}}
\flushleft{\begin{hindi}
कि उससे वही फेरा जाएगा (गुमराह होगा)
\end{hindi}}
\flushright{\begin{Arabic}
\quranayah[51][10]
\end{Arabic}}
\flushleft{\begin{hindi}
जो (ख़ुदा के इल्म में) फेरा जा चुका है अटकल दौड़ाने वाले हलाक हों
\end{hindi}}
\flushright{\begin{Arabic}
\quranayah[51][11]
\end{Arabic}}
\flushleft{\begin{hindi}
जो ग़फलत में भूले हुए (पड़े) हैं पूछते हैं कि जज़ा का दिन कब होगा
\end{hindi}}
\flushright{\begin{Arabic}
\quranayah[51][12]
\end{Arabic}}
\flushleft{\begin{hindi}
उस दिन (होगा)
\end{hindi}}
\flushright{\begin{Arabic}
\quranayah[51][13]
\end{Arabic}}
\flushleft{\begin{hindi}
जब इनको (जहन्नुम की) आग में अज़ाब दिया जाएगा
\end{hindi}}
\flushright{\begin{Arabic}
\quranayah[51][14]
\end{Arabic}}
\flushleft{\begin{hindi}
(और उनसे कहा जाएगा) अपने अज़ाब का मज़ा चखो ये वही है जिसकी तुम जल्दी मचाया करते थे
\end{hindi}}
\flushright{\begin{Arabic}
\quranayah[51][15]
\end{Arabic}}
\flushleft{\begin{hindi}
बेशक परहेज़गार लोग (बेहिश्त के) बाग़ों और चश्मों में (ऐश करते) होगें
\end{hindi}}
\flushright{\begin{Arabic}
\quranayah[51][16]
\end{Arabic}}
\flushleft{\begin{hindi}
जो उनका परवरदिगार उन्हें अता करता है ये (ख़ुश ख़ुश) ले रहे हैं ये लोग इससे पहले (दुनिया में) नेको कार थे
\end{hindi}}
\flushright{\begin{Arabic}
\quranayah[51][17]
\end{Arabic}}
\flushleft{\begin{hindi}
(इबादत की वजह से) रात को बहुत ही कम सोते थे
\end{hindi}}
\flushright{\begin{Arabic}
\quranayah[51][18]
\end{Arabic}}
\flushleft{\begin{hindi}
और पिछले पहर को अपनी मग़फ़िरत की दुआएं करते थे
\end{hindi}}
\flushright{\begin{Arabic}
\quranayah[51][19]
\end{Arabic}}
\flushleft{\begin{hindi}
और उनके माल में माँगने वाले और न माँगने वाले (दोनों) का हिस्सा था
\end{hindi}}
\flushright{\begin{Arabic}
\quranayah[51][20]
\end{Arabic}}
\flushleft{\begin{hindi}
और यक़ीन करने वालों के लिए ज़मीन में (क़ुदरते ख़ुदा की) बहुत सी निशानियाँ हैं
\end{hindi}}
\flushright{\begin{Arabic}
\quranayah[51][21]
\end{Arabic}}
\flushleft{\begin{hindi}
और ख़ुदा तुम में भी हैं तो क्या तुम देखते नहीं
\end{hindi}}
\flushright{\begin{Arabic}
\quranayah[51][22]
\end{Arabic}}
\flushleft{\begin{hindi}
और तुम्हारी रोज़ी और जिस चीज़ का तुमसे वायदा किया जाता है आसमान में है
\end{hindi}}
\flushright{\begin{Arabic}
\quranayah[51][23]
\end{Arabic}}
\flushleft{\begin{hindi}
तो आसमान व ज़मीन के मालिक की क़सम ये (क़ुरान) बिल्कुल ठीक है जिस तरह तुम बातें करते हो
\end{hindi}}
\flushright{\begin{Arabic}
\quranayah[51][24]
\end{Arabic}}
\flushleft{\begin{hindi}
क्या तुम्हारे पास इबराहीम के मुअज़िज़ मेहमानो (फ़रिश्तों) की भी ख़बर पहुँची है कि जब वह लोग उनके पास आए
\end{hindi}}
\flushright{\begin{Arabic}
\quranayah[51][25]
\end{Arabic}}
\flushleft{\begin{hindi}
तो कहने लगे (सलामुन अलैकुम) तो इबराहीम ने भी (अलैकुम) सलाम किया (देखा तो) ऐसे लोग जिनसे न जान न पहचान
\end{hindi}}
\flushright{\begin{Arabic}
\quranayah[51][26]
\end{Arabic}}
\flushleft{\begin{hindi}
फिर अपने घर जाकर जल्दी से (भुना हुआ) एक मोटा ताज़ा बछड़ा ले आए
\end{hindi}}
\flushright{\begin{Arabic}
\quranayah[51][27]
\end{Arabic}}
\flushleft{\begin{hindi}
और उसे उनके आगे रख दिया (फिर) कहने लगे आप लोग तनाउल क्यों नहीं करते
\end{hindi}}
\flushright{\begin{Arabic}
\quranayah[51][28]
\end{Arabic}}
\flushleft{\begin{hindi}
(इस पर भी न खाया) तो इबराहीम उनसे जो ही जी में डरे वह लोग बोले आप अन्देशा न करें और उनको एक दानिशमन्द लड़के की ख़ुशख़बरी दी
\end{hindi}}
\flushright{\begin{Arabic}
\quranayah[51][29]
\end{Arabic}}
\flushleft{\begin{hindi}
तो (ये सुनते ही) इबराहीम की बीवी (सारा) चिल्लाती हुई उनके सामने आयीं और अपना मुँह पीट लिया कहने लगीं (ऐ है) एक तो (मैं) बुढ़िया (उस पर) बांझ
\end{hindi}}
\flushright{\begin{Arabic}
\quranayah[51][30]
\end{Arabic}}
\flushleft{\begin{hindi}
लड़का क्यों कर होगा फ़रिश्ते बोले तुम्हारे परवरदिगार ने यूँ ही फरमाया है वह बेशक हिकमत वाला वाक़िफ़कार है
\end{hindi}}
\flushright{\begin{Arabic}
\quranayah[51][31]
\end{Arabic}}
\flushleft{\begin{hindi}
तब इबराहीम ने पूछा कि (ऐ ख़ुदा के) भेजे हुए फरिश्तों आख़िर तुम्हें क्या मुहिम दर पेश है
\end{hindi}}
\flushright{\begin{Arabic}
\quranayah[51][32]
\end{Arabic}}
\flushleft{\begin{hindi}
वह बोले हम तो गुनाहगारों (क़ौमे लूत) की तरफ भेजे गए हैं
\end{hindi}}
\flushright{\begin{Arabic}
\quranayah[51][33]
\end{Arabic}}
\flushleft{\begin{hindi}
ताकि उन पर मिटटी के पथरीले खरन्जे बरसाएँ
\end{hindi}}
\flushright{\begin{Arabic}
\quranayah[51][34]
\end{Arabic}}
\flushleft{\begin{hindi}
जिन पर हद से बढ़ जाने वालों के लिए तुम्हारे परवरदिगार की तरफ से निशान लगा दिए गए हैं
\end{hindi}}
\flushright{\begin{Arabic}
\quranayah[51][35]
\end{Arabic}}
\flushleft{\begin{hindi}
ग़रज़ वहाँ जितने लोग मोमिनीन थे उनको हमने निकाल दिया
\end{hindi}}
\flushright{\begin{Arabic}
\quranayah[51][36]
\end{Arabic}}
\flushleft{\begin{hindi}
और वहाँ तो हमने एक के सिवा मुसलमानों का कोई घर पाया भी नहीं
\end{hindi}}
\flushright{\begin{Arabic}
\quranayah[51][37]
\end{Arabic}}
\flushleft{\begin{hindi}
और जो लोग दर्दनाक अज़ाब से डरते हैं उनके लिए वहाँ (इबरत की) निशानी छोड़ दी और मूसा (के हाल) में भी (निशानी है)
\end{hindi}}
\flushright{\begin{Arabic}
\quranayah[51][38]
\end{Arabic}}
\flushleft{\begin{hindi}
जब हमने उनको फिरऔन के पास खुला हुआ मौजिज़ा देकर भेजा
\end{hindi}}
\flushright{\begin{Arabic}
\quranayah[51][39]
\end{Arabic}}
\flushleft{\begin{hindi}
तो उसने अपने लशकर के बिरते पर मुँह मोड़ लिया और कहने लगा ये तो (अच्छा ख़ासा) जादूगर या सौदाई है
\end{hindi}}
\flushright{\begin{Arabic}
\quranayah[51][40]
\end{Arabic}}
\flushleft{\begin{hindi}
तो हमने उसको और उसके लशकर को ले डाला फिर उन सबको दरिया में पटक दिया
\end{hindi}}
\flushright{\begin{Arabic}
\quranayah[51][41]
\end{Arabic}}
\flushleft{\begin{hindi}
और वह तो क़ाबिले मलामत काम करता ही था और आद की क़ौम (के हाल) में भी निशानी है हमने उन पर एक बे बरकत ऑंधी चलायी
\end{hindi}}
\flushright{\begin{Arabic}
\quranayah[51][42]
\end{Arabic}}
\flushleft{\begin{hindi}
कि जिस चीज़ पर चलती उसको बोसीदा हडडी की तरह रेज़ा रेज़ा किए बग़ैर न छोड़ती
\end{hindi}}
\flushright{\begin{Arabic}
\quranayah[51][43]
\end{Arabic}}
\flushleft{\begin{hindi}
और समूद (के हाल) में भी (क़ुदरत की निशानी) है जब उससे कहा गया कि एक ख़ास वक्त तक ख़ूब चैन कर लो
\end{hindi}}
\flushright{\begin{Arabic}
\quranayah[51][44]
\end{Arabic}}
\flushleft{\begin{hindi}
तो उन्होने अपने परवरदिगार के हुक्म से सरकशी की तो उन्हें एक रोज़ कड़क और बिजली ने ले डाला और देखते ही रह गए
\end{hindi}}
\flushright{\begin{Arabic}
\quranayah[51][45]
\end{Arabic}}
\flushleft{\begin{hindi}
फिर न वह उठने की ताक़त रखते थे और न बदला ही ले सकते थे
\end{hindi}}
\flushright{\begin{Arabic}
\quranayah[51][46]
\end{Arabic}}
\flushleft{\begin{hindi}
और (उनसे) पहले (हम) नूह की क़ौम को (हलाक कर चुके थे) बेशक वह बदकार लोग थे
\end{hindi}}
\flushright{\begin{Arabic}
\quranayah[51][47]
\end{Arabic}}
\flushleft{\begin{hindi}
और हमने आसमानों को अपने बल बूते से बनाया और बेशक हममें सब क़ुदरत है
\end{hindi}}
\flushright{\begin{Arabic}
\quranayah[51][48]
\end{Arabic}}
\flushleft{\begin{hindi}
और ज़मीन को भी हम ही ने बिछाया तो हम कैसे अच्छे बिछाने वाले हैं
\end{hindi}}
\flushright{\begin{Arabic}
\quranayah[51][49]
\end{Arabic}}
\flushleft{\begin{hindi}
और हम ही ने हर चीज़ की दो दो क़िस्में बनायीं ताकि तुम लोग नसीहत हासिल करो
\end{hindi}}
\flushright{\begin{Arabic}
\quranayah[51][50]
\end{Arabic}}
\flushleft{\begin{hindi}
तो ख़ुदा ही की तरफ़ भागो मैं तुमको यक़ीनन उसकी तरफ से खुल्लम खुल्ला डराने वाला हूँ
\end{hindi}}
\flushright{\begin{Arabic}
\quranayah[51][51]
\end{Arabic}}
\flushleft{\begin{hindi}
और ख़ुदा के साथ दूसरा माबूद न बनाओ मैं तुमको यक़ीनन उसकी तरफ से खुल्लम खुल्ला डराने वाला हूँ
\end{hindi}}
\flushright{\begin{Arabic}
\quranayah[51][52]
\end{Arabic}}
\flushleft{\begin{hindi}
इसी तरह उनसे पहले लोगों के पास जो पैग़म्बर आता तो वह उसको जादूगर कहते या सिड़ी दीवाना (बताते)
\end{hindi}}
\flushright{\begin{Arabic}
\quranayah[51][53]
\end{Arabic}}
\flushleft{\begin{hindi}
ये लोग एक दूसरे को ऐसी बात की वसीयत करते आते हैं (नहीं) बल्कि ये लोग हैं ही सरकश
\end{hindi}}
\flushright{\begin{Arabic}
\quranayah[51][54]
\end{Arabic}}
\flushleft{\begin{hindi}
तो (ऐ रसूल) तुम इनसे मुँह फेर लो तुम पर तो कुछ इल्ज़ाम नहीं है
\end{hindi}}
\flushright{\begin{Arabic}
\quranayah[51][55]
\end{Arabic}}
\flushleft{\begin{hindi}
और नसीहत किए जाओ क्योंकि नसीहत मोमिनीन को फायदा देती है
\end{hindi}}
\flushright{\begin{Arabic}
\quranayah[51][56]
\end{Arabic}}
\flushleft{\begin{hindi}
और मैने जिनों और आदमियों को इसी ग़रज़ से पैदा किया कि वह मेरी इबादत करें
\end{hindi}}
\flushright{\begin{Arabic}
\quranayah[51][57]
\end{Arabic}}
\flushleft{\begin{hindi}
न तो मैं उनसे रोज़ी का तालिब हूँ और न ये चाहता हूँ कि मुझे खाना खिलाएँ
\end{hindi}}
\flushright{\begin{Arabic}
\quranayah[51][58]
\end{Arabic}}
\flushleft{\begin{hindi}
ख़ुदा ख़ुद बड़ा रोज़ी देने वाला ज़ोरावर (और) ज़बरदस्त है
\end{hindi}}
\flushright{\begin{Arabic}
\quranayah[51][59]
\end{Arabic}}
\flushleft{\begin{hindi}
तो (इन) ज़ालिमों के वास्ते भी अज़ाब का कुछ हिस्सा है जिस तरह उनके साथियों के लिए हिस्सा था तो इनको हम से जल्दी न करनी चाहिए
\end{hindi}}
\flushright{\begin{Arabic}
\quranayah[51][60]
\end{Arabic}}
\flushleft{\begin{hindi}
तो जिस दिन का इन काफ़िरों से वायदा किया जाता है इससे इनके लिए ख़राबी है
\end{hindi}}
\chapter{At-Tur (The Mountain)}
\begin{Arabic}
\Huge{\centerline{\basmalah}}\end{Arabic}
\flushright{\begin{Arabic}
\quranayah[52][1]
\end{Arabic}}
\flushleft{\begin{hindi}
(कोहे) तूर की क़सम
\end{hindi}}
\flushright{\begin{Arabic}
\quranayah[52][2]
\end{Arabic}}
\flushleft{\begin{hindi}
और उसकी किताब (लौहे महफूज़) की
\end{hindi}}
\flushright{\begin{Arabic}
\quranayah[52][3]
\end{Arabic}}
\flushleft{\begin{hindi}
जो क़ुशादा औराक़ में लिखी हुई है
\end{hindi}}
\flushright{\begin{Arabic}
\quranayah[52][4]
\end{Arabic}}
\flushleft{\begin{hindi}
और बैतुल मामूर की (जो काबा के सामने फरिश्तों का क़िब्ला है)
\end{hindi}}
\flushright{\begin{Arabic}
\quranayah[52][5]
\end{Arabic}}
\flushleft{\begin{hindi}
और ऊँची छत (आसमान) की
\end{hindi}}
\flushright{\begin{Arabic}
\quranayah[52][6]
\end{Arabic}}
\flushleft{\begin{hindi}
और जोश व ख़रोश वाले समन्दर की
\end{hindi}}
\flushright{\begin{Arabic}
\quranayah[52][7]
\end{Arabic}}
\flushleft{\begin{hindi}
कि तुम्हारे परवरदिगार का अज़ाब बेशक वाकेए होकर रहेगा
\end{hindi}}
\flushright{\begin{Arabic}
\quranayah[52][8]
\end{Arabic}}
\flushleft{\begin{hindi}
(और) इसका कोई रोकने वाला नहीं
\end{hindi}}
\flushright{\begin{Arabic}
\quranayah[52][9]
\end{Arabic}}
\flushleft{\begin{hindi}
जिस दिन आसमान चक्कर खाने लगेगा
\end{hindi}}
\flushright{\begin{Arabic}
\quranayah[52][10]
\end{Arabic}}
\flushleft{\begin{hindi}
और पहाड़ उड़ने लगेंगे
\end{hindi}}
\flushright{\begin{Arabic}
\quranayah[52][11]
\end{Arabic}}
\flushleft{\begin{hindi}
तो उस दिन झुठलाने वालों की ख़राबी है
\end{hindi}}
\flushright{\begin{Arabic}
\quranayah[52][12]
\end{Arabic}}
\flushleft{\begin{hindi}
जो लोग बातिल में पड़े खेल रहे हैं
\end{hindi}}
\flushright{\begin{Arabic}
\quranayah[52][13]
\end{Arabic}}
\flushleft{\begin{hindi}
जिस दिन जहन्नुम की आग की तरफ उनको ढकेल ढकेल ले जाएँगे
\end{hindi}}
\flushright{\begin{Arabic}
\quranayah[52][14]
\end{Arabic}}
\flushleft{\begin{hindi}
(और उनसे कहा जाएगा) यही वह जहन्नुम है जिसे तुम झुठलाया करते थे
\end{hindi}}
\flushright{\begin{Arabic}
\quranayah[52][15]
\end{Arabic}}
\flushleft{\begin{hindi}
तो क्या ये जादू है या तुमको नज़र ही नहीं आता
\end{hindi}}
\flushright{\begin{Arabic}
\quranayah[52][16]
\end{Arabic}}
\flushleft{\begin{hindi}
इसी में घुसो फिर सब्र करो या बेसब्री करो (दोनों) तुम्हारे लिए यकसाँ हैं तुम्हें तो बस उन्हीं कामों का बदला मिलेगा जो तुम किया करते थे
\end{hindi}}
\flushright{\begin{Arabic}
\quranayah[52][17]
\end{Arabic}}
\flushleft{\begin{hindi}
बेशक परहेज़गार लोग बाग़ों और नेअमतों में होंगे
\end{hindi}}
\flushright{\begin{Arabic}
\quranayah[52][18]
\end{Arabic}}
\flushleft{\begin{hindi}
जो (जो नेअमतें) उनके परवरदिगार ने उन्हें दी हैं उनके मज़े ले रहे हैं और उनका परवरदिगार उन्हें दोज़ख़ के अज़ाब से बचाएगा
\end{hindi}}
\flushright{\begin{Arabic}
\quranayah[52][19]
\end{Arabic}}
\flushleft{\begin{hindi}
जो जो कारगुज़ारियाँ तुम कर चुके हो उनके सिले में (आराम से) तख्तों पर जो बराबर बिछे हुए हैं
\end{hindi}}
\flushright{\begin{Arabic}
\quranayah[52][20]
\end{Arabic}}
\flushleft{\begin{hindi}
तकिए लगाकर ख़ूब मज़े से खाओ पियो और हम बड़ी बड़ी ऑंखों वाली हूर से उनका ब्याह रचाएँगे
\end{hindi}}
\flushright{\begin{Arabic}
\quranayah[52][21]
\end{Arabic}}
\flushleft{\begin{hindi}
और जिन लोगों ने ईमान क़ुबूल किया और उनकी औलाद ने भी ईमान में उनका साथ दिया तो हम उनकी औलाद को भी उनके दर्जे पहुँचा देंगे और हम उनकी कारगुज़ारियों में से कुछ भी कम न करेंगे हर शख़्श अपने आमाल के बदले में गिरवी है
\end{hindi}}
\flushright{\begin{Arabic}
\quranayah[52][22]
\end{Arabic}}
\flushleft{\begin{hindi}
और जिस क़िस्म के मेवे और गोश्त को उनका जी चाहेगा हम उन्हें बढ़ाकर अता करेंगे
\end{hindi}}
\flushright{\begin{Arabic}
\quranayah[52][23]
\end{Arabic}}
\flushleft{\begin{hindi}
वहाँ एक दूसरे से शराब का जाम ले लिया करेंगे जिसमें न कोई बेहूदगी है और न गुनाह
\end{hindi}}
\flushright{\begin{Arabic}
\quranayah[52][24]
\end{Arabic}}
\flushleft{\begin{hindi}
(और ख़िदमत के लिए) नौजवान लड़के उनके आस पास चक्कर लगाया करेंगे वह (हुस्न व जमाल में) गोया एहतियात से रखे हुए मोती हैं
\end{hindi}}
\flushright{\begin{Arabic}
\quranayah[52][25]
\end{Arabic}}
\flushleft{\begin{hindi}
और एक दूसरे की तरफ रूख़ करके (लुत्फ की) बातें करेंगे
\end{hindi}}
\flushright{\begin{Arabic}
\quranayah[52][26]
\end{Arabic}}
\flushleft{\begin{hindi}
(उनमें से कुछ) कहेंगे कि हम इससे पहले अपने घर में (ख़ुदा से बहुत) डरा करते थे
\end{hindi}}
\flushright{\begin{Arabic}
\quranayah[52][27]
\end{Arabic}}
\flushleft{\begin{hindi}
तो ख़ुदा ने हम पर बड़ा एहसान किया और हमको (जहन्नुम की) लौ के अज़ाब से बचा लिया
\end{hindi}}
\flushright{\begin{Arabic}
\quranayah[52][28]
\end{Arabic}}
\flushleft{\begin{hindi}
इससे क़ब्ल हम उनसे दुआएँ किया करते थे बेशक वह एहसान करने वाला मेहरबान है
\end{hindi}}
\flushright{\begin{Arabic}
\quranayah[52][29]
\end{Arabic}}
\flushleft{\begin{hindi}
तो (ऐ रसूल) तुम नसीहत किए जाओ तो तुम अपने परवरदिगार के फज़ल से न काहिन हो और न मजनून किया
\end{hindi}}
\flushright{\begin{Arabic}
\quranayah[52][30]
\end{Arabic}}
\flushleft{\begin{hindi}
क्या (तुमको) ये लोग कहते हैं कि (ये) शायर हैं (और) हम तो उसके बारे में ज़माने के हवादिस का इन्तेज़ार कर रहे हैं
\end{hindi}}
\flushright{\begin{Arabic}
\quranayah[52][31]
\end{Arabic}}
\flushleft{\begin{hindi}
तुम कह दो कि (अच्छा) तुम भी इन्तेज़ार करो मैं भी इन्तेज़ार करता हूँ
\end{hindi}}
\flushright{\begin{Arabic}
\quranayah[52][32]
\end{Arabic}}
\flushleft{\begin{hindi}
क्या उनकी अक्लें उन्हें ये (बातें) बताती हैं या ये लोग हैं ही सरकश
\end{hindi}}
\flushright{\begin{Arabic}
\quranayah[52][33]
\end{Arabic}}
\flushleft{\begin{hindi}
क्या ये लोग कहते हैं कि इसने क़ुरान ख़ुद गढ़ लिया है बात ये है कि ये लोग ईमान ही नहीं रखते
\end{hindi}}
\flushright{\begin{Arabic}
\quranayah[52][34]
\end{Arabic}}
\flushleft{\begin{hindi}
तो अगर ये लोग सच्चे हैं तो ऐसा ही कलाम बना तो लाएँ
\end{hindi}}
\flushright{\begin{Arabic}
\quranayah[52][35]
\end{Arabic}}
\flushleft{\begin{hindi}
क्या ये लोग किसी के (पैदा किये) बग़ैर ही पैदा हो गए हैं या यही लोग (मख़लूक़ात के) पैदा करने वाले हैं
\end{hindi}}
\flushright{\begin{Arabic}
\quranayah[52][36]
\end{Arabic}}
\flushleft{\begin{hindi}
या इन्होने ही ने सारे आसमान व ज़मीन पैदा किए हैं (नहीं) बल्कि ये लोग यक़ीन ही नहीं रखते
\end{hindi}}
\flushright{\begin{Arabic}
\quranayah[52][37]
\end{Arabic}}
\flushleft{\begin{hindi}
क्या तुम्हारे परवरदिगार के ख़ज़ाने इन्हीं के पास हैं या यही लोग हाकिम हैं
\end{hindi}}
\flushright{\begin{Arabic}
\quranayah[52][38]
\end{Arabic}}
\flushleft{\begin{hindi}
या उनके पास कोई सीढ़ी है जिस पर (चढ़ कर आसमान से) सुन आते हैं जो सुन आया करता हो तो वह कोई सरीही दलील पेश करे
\end{hindi}}
\flushright{\begin{Arabic}
\quranayah[52][39]
\end{Arabic}}
\flushleft{\begin{hindi}
क्या ख़ुदा के लिए बेटियाँ हैं और तुम लोगों के लिए बेटे
\end{hindi}}
\flushright{\begin{Arabic}
\quranayah[52][40]
\end{Arabic}}
\flushleft{\begin{hindi}
या तुम उनसे (तबलीग़े रिसालत की) उजरत माँगते हो कि ये लोग कर्ज़ के बोझ से दबे जाते हैं
\end{hindi}}
\flushright{\begin{Arabic}
\quranayah[52][41]
\end{Arabic}}
\flushleft{\begin{hindi}
या इन लोगों के पास ग़ैब (का इल्म) है कि वह लिख लेते हैं
\end{hindi}}
\flushright{\begin{Arabic}
\quranayah[52][42]
\end{Arabic}}
\flushleft{\begin{hindi}
या ये लोग कुछ दाँव चलाना चाहते हैं तो जो लोग काफ़िर हैं वह ख़ुद अपने दांव में फँसे हैं
\end{hindi}}
\flushright{\begin{Arabic}
\quranayah[52][43]
\end{Arabic}}
\flushleft{\begin{hindi}
या ख़ुदा के सिवा इनका कोई (दूसरा) माबूद है जिन चीज़ों को ये लोग (ख़ुदा का) शरीक बनाते हैं वह उससे पाक और पाक़ीज़ा है
\end{hindi}}
\flushright{\begin{Arabic}
\quranayah[52][44]
\end{Arabic}}
\flushleft{\begin{hindi}
और अगर ये लोग आसमान से कोई अज़ाब (अज़ाब का) टुकड़ा गिरते हुए देखें तो बोल उठेंगे ये तो दलदार बादल है
\end{hindi}}
\flushright{\begin{Arabic}
\quranayah[52][45]
\end{Arabic}}
\flushleft{\begin{hindi}
तो (ऐ रसूल) तुम इनको इनकी हालत पर छोड़ दो यहाँ तक कि वह जिसमें ये बेहोश हो जाएँगे
\end{hindi}}
\flushright{\begin{Arabic}
\quranayah[52][46]
\end{Arabic}}
\flushleft{\begin{hindi}
इनके सामने आ जाए जिस दिन न इनकी मक्कारी ही कुछ काम आएगी और न इनकी मदद ही की जाएगी
\end{hindi}}
\flushright{\begin{Arabic}
\quranayah[52][47]
\end{Arabic}}
\flushleft{\begin{hindi}
और इसमें शक़ नहीं कि ज़ालिमों के लिए इसके अलावा और भी अज़ाब है मगर उनमें बहुतेरे नहीं जानते हैं
\end{hindi}}
\flushright{\begin{Arabic}
\quranayah[52][48]
\end{Arabic}}
\flushleft{\begin{hindi}
और (ऐ रसूल) तुम अपने परवरदिगार के हुक्म से इन्तेज़ार में सब्र किए रहो तो तुम बिल्कुल हमारी निगेहदाश्त में हो तो जब तुम उठा करो तो अपने परवरदिगार की हम्द की तस्बीह किया करो
\end{hindi}}
\flushright{\begin{Arabic}
\quranayah[52][49]
\end{Arabic}}
\flushleft{\begin{hindi}
और कुछ रात को भी और सितारों के ग़ुरूब होने के बाद तस्बीह किया करो
\end{hindi}}
\chapter{An-Najm (The Star)}
\begin{Arabic}
\Huge{\centerline{\basmalah}}\end{Arabic}
\flushright{\begin{Arabic}
\quranayah[53][1]
\end{Arabic}}
\flushleft{\begin{hindi}
तारे की क़सम जब टूटा
\end{hindi}}
\flushright{\begin{Arabic}
\quranayah[53][2]
\end{Arabic}}
\flushleft{\begin{hindi}
कि तुम्हारे रफ़ीक़ (मोहम्मद) न गुमराह हुए और न बहके
\end{hindi}}
\flushright{\begin{Arabic}
\quranayah[53][3]
\end{Arabic}}
\flushleft{\begin{hindi}
और वह तो अपनी नफ़सियानी ख्वाहिश से कुछ भी नहीं कहते
\end{hindi}}
\flushright{\begin{Arabic}
\quranayah[53][4]
\end{Arabic}}
\flushleft{\begin{hindi}
ये तो बस वही है जो भेजी जाती है
\end{hindi}}
\flushright{\begin{Arabic}
\quranayah[53][5]
\end{Arabic}}
\flushleft{\begin{hindi}
इनको निहायत ताक़तवर (फ़रिश्ते जिबरील) ने तालीम दी है
\end{hindi}}
\flushright{\begin{Arabic}
\quranayah[53][6]
\end{Arabic}}
\flushleft{\begin{hindi}
जो बड़ा ज़बरदस्त है और जब ये (आसमान के) ऊँचे (मुशरक़ो) किनारे पर था तो वह अपनी (असली सूरत में) सीधा खड़ा हुआ
\end{hindi}}
\flushright{\begin{Arabic}
\quranayah[53][7]
\end{Arabic}}
\flushleft{\begin{hindi}
फिर करीब हो (और आगे) बढ़ा
\end{hindi}}
\flushright{\begin{Arabic}
\quranayah[53][8]
\end{Arabic}}
\flushleft{\begin{hindi}
(फिर जिबरील व मोहम्मद में) दो कमान का फ़ासला रह गया
\end{hindi}}
\flushright{\begin{Arabic}
\quranayah[53][9]
\end{Arabic}}
\flushleft{\begin{hindi}
बल्कि इससे भी क़रीब था
\end{hindi}}
\flushright{\begin{Arabic}
\quranayah[53][10]
\end{Arabic}}
\flushleft{\begin{hindi}
ख़ुदा ने अपने बन्दे की तरफ जो 'वही' भेजी सो भेजी
\end{hindi}}
\flushright{\begin{Arabic}
\quranayah[53][11]
\end{Arabic}}
\flushleft{\begin{hindi}
तो जो कुछ उन्होने देखा उनके दिल ने झूठ न जाना
\end{hindi}}
\flushright{\begin{Arabic}
\quranayah[53][12]
\end{Arabic}}
\flushleft{\begin{hindi}
तो क्या वह (रसूल) जो कुछ देखता है तुम लोग उसमें झगड़ते हो
\end{hindi}}
\flushright{\begin{Arabic}
\quranayah[53][13]
\end{Arabic}}
\flushleft{\begin{hindi}
और उन्होने तो उस (जिबरील) को एक बार (शबे मेराज) और देखा है
\end{hindi}}
\flushright{\begin{Arabic}
\quranayah[53][14]
\end{Arabic}}
\flushleft{\begin{hindi}
सिदरतुल मुनतहा के नज़दीक
\end{hindi}}
\flushright{\begin{Arabic}
\quranayah[53][15]
\end{Arabic}}
\flushleft{\begin{hindi}
उसी के पास तो रहने की बेहिश्त है
\end{hindi}}
\flushright{\begin{Arabic}
\quranayah[53][16]
\end{Arabic}}
\flushleft{\begin{hindi}
जब छा रहा था सिदरा पर जो छा रहा था
\end{hindi}}
\flushright{\begin{Arabic}
\quranayah[53][17]
\end{Arabic}}
\flushleft{\begin{hindi}
(उस वक्त भी) उनकी ऑंख न तो और तरफ़ माएल हुई और न हद से आगे बढ़ी
\end{hindi}}
\flushright{\begin{Arabic}
\quranayah[53][18]
\end{Arabic}}
\flushleft{\begin{hindi}
और उन्होने यक़ीनन अपने परवरदिगार (की क़ुदरत) की बड़ी बड़ी निशानियाँ देखीं
\end{hindi}}
\flushright{\begin{Arabic}
\quranayah[53][19]
\end{Arabic}}
\flushleft{\begin{hindi}
तो भला तुम लोगों ने लात व उज्ज़ा और तीसरे पिछले मनात को देखा
\end{hindi}}
\flushright{\begin{Arabic}
\quranayah[53][20]
\end{Arabic}}
\flushleft{\begin{hindi}
(भला ये ख़ुदा हो सकते हैं)
\end{hindi}}
\flushright{\begin{Arabic}
\quranayah[53][21]
\end{Arabic}}
\flushleft{\begin{hindi}
क्या तुम्हारे तो बेटे हैं और उसके लिए बेटियाँ
\end{hindi}}
\flushright{\begin{Arabic}
\quranayah[53][22]
\end{Arabic}}
\flushleft{\begin{hindi}
ये तो बहुत बेइन्साफ़ी की तक़सीम है
\end{hindi}}
\flushright{\begin{Arabic}
\quranayah[53][23]
\end{Arabic}}
\flushleft{\begin{hindi}
ये तो बस सिर्फ नाम ही नाम है जो तुमने और तुम्हारे बाप दादाओं ने गढ़ लिए हैं, ख़ुदा ने तो इसकी कोई सनद नाज़िल नहीं की ये लोग तो बस अटकल और अपनी नफ़सानी ख्वाहिश के पीछे चल रहे हैं हालॉकि उनके पास उनके परवरदिगार की तरफ से हिदायत भी आ चुकी है
\end{hindi}}
\flushright{\begin{Arabic}
\quranayah[53][24]
\end{Arabic}}
\flushleft{\begin{hindi}
क्या जिस चीज़ की इन्सान तमन्ना करे वह उसे ज़रूर मिलती है
\end{hindi}}
\flushright{\begin{Arabic}
\quranayah[53][25]
\end{Arabic}}
\flushleft{\begin{hindi}
आख़ेरत और दुनिया तो ख़ास ख़ुदा ही के एख्तेयार में हैं
\end{hindi}}
\flushright{\begin{Arabic}
\quranayah[53][26]
\end{Arabic}}
\flushleft{\begin{hindi}
और आसमानों में बहुत से फरिश्ते हैं जिनकी सिफ़ारिश कुछ भी काम न आती, मगर ख़ुदा जिसके लिए चाहे इजाज़त दे दे और पसन्द करे उसके बाद (सिफ़ारिश कर सकते हैं)
\end{hindi}}
\flushright{\begin{Arabic}
\quranayah[53][27]
\end{Arabic}}
\flushleft{\begin{hindi}
जो लोग आख़ेरत पर ईमान नहीं रखते वह फ़रिश्तों के नाम रखते हैं औरतों के से नाम हालॉकि उन्हें इसकी कुछ ख़बर नहीं
\end{hindi}}
\flushright{\begin{Arabic}
\quranayah[53][28]
\end{Arabic}}
\flushleft{\begin{hindi}
वह लोग तो बस गुमान (ख्याल) के पीछे चल रहे हैं, हालॉकि गुमान यक़ीन के बदले में कुछ भी काम नहीं आया करता,
\end{hindi}}
\flushright{\begin{Arabic}
\quranayah[53][29]
\end{Arabic}}
\flushleft{\begin{hindi}
तो जो हमारी याद से रदगिरदानी करे ओर सिर्फ दुनिया की ज़िन्दगी ही का तालिब हो तुम भी उससे मुँह फेर लो
\end{hindi}}
\flushright{\begin{Arabic}
\quranayah[53][30]
\end{Arabic}}
\flushleft{\begin{hindi}
उनके इल्म की यही इन्तिहा है तुम्हारा परवरदिगार, जो उसके रास्ते से भटक गया उसको भी ख़ूब जानता है, और जो राहे रास्त पर है उनसे भी ख़ूब वाक़िफ है
\end{hindi}}
\flushright{\begin{Arabic}
\quranayah[53][31]
\end{Arabic}}
\flushleft{\begin{hindi}
और जो कुछ आसमानों में है और जो कुछ ज़मीन में है (ग़रज़ सब कुछ) ख़ुदा ही का है, ताकि जिन लोगों ने बुराई की हो उनको उनकी कारस्तानियों की सज़ा दे और जिन लोगों ने नेकी की है (उनकी नेकी की जज़ा दे)
\end{hindi}}
\flushright{\begin{Arabic}
\quranayah[53][32]
\end{Arabic}}
\flushleft{\begin{hindi}
जो सग़ीरा गुनाहों के सिवा कबीरा गुनाहों से और बेहयाई की बातों से बचे रहते हैं बेशक तुम्हारा परवरदिगार बड़ी बख्यिश वाला है वही तुमको ख़ूब जानता है जब उसने तुमको मिटटी से पैदा किया और जब तुम अपनी माँ के पेट में बच्चे थे तो (तकब्बुर) से अपने नफ्स की पाकीज़गी न जताया करो जो परहेज़गार है उसको वह ख़ूब जानता है
\end{hindi}}
\flushright{\begin{Arabic}
\quranayah[53][33]
\end{Arabic}}
\flushleft{\begin{hindi}
भला (ऐ रसूल) तुमने उस शख़्श को भी देखा जिसने रदगिरदानी की
\end{hindi}}
\flushright{\begin{Arabic}
\quranayah[53][34]
\end{Arabic}}
\flushleft{\begin{hindi}
और थोड़ा सा (ख़ुदा की राह में) दिया और फिर बन्द कर दिया
\end{hindi}}
\flushright{\begin{Arabic}
\quranayah[53][35]
\end{Arabic}}
\flushleft{\begin{hindi}
क्या उसके पास इल्मे ग़ैब है कि वह देख रहा है
\end{hindi}}
\flushright{\begin{Arabic}
\quranayah[53][36]
\end{Arabic}}
\flushleft{\begin{hindi}
क्या उसको उन बातों की ख़बर नहीं पहुँची जो मूसा के सहीफ़ों में है
\end{hindi}}
\flushright{\begin{Arabic}
\quranayah[53][37]
\end{Arabic}}
\flushleft{\begin{hindi}
और इबराहीम के (सहीफ़ों में)
\end{hindi}}
\flushright{\begin{Arabic}
\quranayah[53][38]
\end{Arabic}}
\flushleft{\begin{hindi}
जिन्होने (अपना हक़) (पूरा अदा) किया इन सहीफ़ों में ये है, कि कोई शख़्श दूसरे (के गुनाह) का बोझ नहीं उठाएगा
\end{hindi}}
\flushright{\begin{Arabic}
\quranayah[53][39]
\end{Arabic}}
\flushleft{\begin{hindi}
और ये कि इन्सान को वही मिलता है जिसकी वह कोशिश करता है
\end{hindi}}
\flushright{\begin{Arabic}
\quranayah[53][40]
\end{Arabic}}
\flushleft{\begin{hindi}
और ये कि उनकी कोशिश अनक़रीेब ही (क़यामत में) देखी जाएगी
\end{hindi}}
\flushright{\begin{Arabic}
\quranayah[53][41]
\end{Arabic}}
\flushleft{\begin{hindi}
फिर उसका पूरा पूरा बदला दिया जाएगा
\end{hindi}}
\flushright{\begin{Arabic}
\quranayah[53][42]
\end{Arabic}}
\flushleft{\begin{hindi}
और ये कि (सबको आख़िर) तुम्हारे परवरदिगार ही के पास पहुँचना है
\end{hindi}}
\flushright{\begin{Arabic}
\quranayah[53][43]
\end{Arabic}}
\flushleft{\begin{hindi}
और ये कि वही हँसाता और रूलाता है
\end{hindi}}
\flushright{\begin{Arabic}
\quranayah[53][44]
\end{Arabic}}
\flushleft{\begin{hindi}
और ये कि वही मारता और जिलाता है
\end{hindi}}
\flushright{\begin{Arabic}
\quranayah[53][45]
\end{Arabic}}
\flushleft{\begin{hindi}
और ये कि वही नर और मादा दो किस्म (के हैवान) नुत्फे से जब (रहम में) डाला जाता है
\end{hindi}}
\flushright{\begin{Arabic}
\quranayah[53][46]
\end{Arabic}}
\flushleft{\begin{hindi}
पैदा करता है
\end{hindi}}
\flushright{\begin{Arabic}
\quranayah[53][47]
\end{Arabic}}
\flushleft{\begin{hindi}
और ये कि उसी पर (कयामत में) दोबारा उठाना लाज़िम है
\end{hindi}}
\flushright{\begin{Arabic}
\quranayah[53][48]
\end{Arabic}}
\flushleft{\begin{hindi}
और ये कि वही मालदार बनाता है और सरमाया अता करता है,
\end{hindi}}
\flushright{\begin{Arabic}
\quranayah[53][49]
\end{Arabic}}
\flushleft{\begin{hindi}
और ये कि वही योअराए का मालिक है
\end{hindi}}
\flushright{\begin{Arabic}
\quranayah[53][50]
\end{Arabic}}
\flushleft{\begin{hindi}
और ये कि उसी ने पहले (क़ौमे) आद को हलाक किया
\end{hindi}}
\flushright{\begin{Arabic}
\quranayah[53][51]
\end{Arabic}}
\flushleft{\begin{hindi}
और समूद को भी ग़रज़ किसी को बाक़ी न छोड़ा
\end{hindi}}
\flushright{\begin{Arabic}
\quranayah[53][52]
\end{Arabic}}
\flushleft{\begin{hindi}
और (उसके) पहले नूह की क़ौम को बेशक ये लोग बड़े ही ज़ालिम और बड़े ही सरकश थे
\end{hindi}}
\flushright{\begin{Arabic}
\quranayah[53][53]
\end{Arabic}}
\flushleft{\begin{hindi}
और उसी ने (क़ौमे लूत की) उलटी हुई बस्तियों को दे पटका
\end{hindi}}
\flushright{\begin{Arabic}
\quranayah[53][54]
\end{Arabic}}
\flushleft{\begin{hindi}
(फिर उन पर) जो छाया सो छाया
\end{hindi}}
\flushright{\begin{Arabic}
\quranayah[53][55]
\end{Arabic}}
\flushleft{\begin{hindi}
तो तू (ऐ इन्सान आख़िर) अपने परवरदिगार की कौन सी नेअमत पर शक़ किया करेगा
\end{hindi}}
\flushright{\begin{Arabic}
\quranayah[53][56]
\end{Arabic}}
\flushleft{\begin{hindi}
ये (मोहम्मद भी अगले डराने वाले पैग़म्बरों में से एक डरने वाला) पैग़म्बर है
\end{hindi}}
\flushright{\begin{Arabic}
\quranayah[53][57]
\end{Arabic}}
\flushleft{\begin{hindi}
कयामत क़रीब आ गयी
\end{hindi}}
\flushright{\begin{Arabic}
\quranayah[53][58]
\end{Arabic}}
\flushleft{\begin{hindi}
ख़ुदा के सिवा उसे कोई टाल नहीं सकता
\end{hindi}}
\flushright{\begin{Arabic}
\quranayah[53][59]
\end{Arabic}}
\flushleft{\begin{hindi}
तो क्या तुम लोग इस बात से ताज्जुब करते हो और हँसते हो
\end{hindi}}
\flushright{\begin{Arabic}
\quranayah[53][60]
\end{Arabic}}
\flushleft{\begin{hindi}
और रोते नहीं हो
\end{hindi}}
\flushright{\begin{Arabic}
\quranayah[53][61]
\end{Arabic}}
\flushleft{\begin{hindi}
और तुम इस क़दर ग़ाफ़िल हो तो ख़ुदा के आगे सजदे किया करो
\end{hindi}}
\flushright{\begin{Arabic}
\quranayah[53][62]
\end{Arabic}}
\flushleft{\begin{hindi}
और (उसी की) इबादत किया करो (62) सजदा
\end{hindi}}
\chapter{Al-Qamar (The Moon)}
\begin{Arabic}
\Huge{\centerline{\basmalah}}\end{Arabic}
\flushright{\begin{Arabic}
\quranayah[54][1]
\end{Arabic}}
\flushleft{\begin{hindi}
क़यामत क़रीब आ गयी और चाँद दो टुकड़े हो गया
\end{hindi}}
\flushright{\begin{Arabic}
\quranayah[54][2]
\end{Arabic}}
\flushleft{\begin{hindi}
और अगर ये कुफ्फ़ार कोई मौजिज़ा देखते हैं, तो मुँह फेर लेते हैं, और कहते हैं कि ये तो बड़ा ज़बरदस्त जादू है
\end{hindi}}
\flushright{\begin{Arabic}
\quranayah[54][3]
\end{Arabic}}
\flushleft{\begin{hindi}
और उन लोगों ने झुठलाया और अपनी नफ़सियानी ख्वाहिशों की पैरवी की, और हर काम का वक्त मुक़र्रर है
\end{hindi}}
\flushright{\begin{Arabic}
\quranayah[54][4]
\end{Arabic}}
\flushleft{\begin{hindi}
और उनके पास तो वह हालात पहुँच चुके हैं जिनमें काफी तम्बीह थीं
\end{hindi}}
\flushright{\begin{Arabic}
\quranayah[54][5]
\end{Arabic}}
\flushleft{\begin{hindi}
और इन्तेहा दर्जे की दानाई मगर (उनको तो) डराना कुछ फ़ायदा नहीं देता
\end{hindi}}
\flushright{\begin{Arabic}
\quranayah[54][6]
\end{Arabic}}
\flushleft{\begin{hindi}
तो (ऐ रसूल) तुम भी उनसे किनाराकश रहो, जिस दिन एक बुलाने वाला (इसराफ़ील) एक अजनबी और नागवार चीज़ की तरफ़ बुलाएगा
\end{hindi}}
\flushright{\begin{Arabic}
\quranayah[54][7]
\end{Arabic}}
\flushleft{\begin{hindi}
तो (निदामत से) ऑंखें नीचे किए हुए कब्रों से निकल पड़ेंगे गोया वह फैली हुई टिड्डियाँ हैं
\end{hindi}}
\flushright{\begin{Arabic}
\quranayah[54][8]
\end{Arabic}}
\flushleft{\begin{hindi}
(और) बुलाने वाले की तरफ गर्दनें बढ़ाए दौड़ते जाते होंगे, कुफ्फ़ार कहेंगे ये तो बड़ा सख्त दिन है
\end{hindi}}
\flushright{\begin{Arabic}
\quranayah[54][9]
\end{Arabic}}
\flushleft{\begin{hindi}
इनसे पहले नूह की क़ौम ने भी झुठलाया था, तो उन्होने हमारे (ख़ास) बन्दे (नूह) को झुठलाया, और कहने लगे ये तो दीवाना है
\end{hindi}}
\flushright{\begin{Arabic}
\quranayah[54][10]
\end{Arabic}}
\flushleft{\begin{hindi}
और उनको झिड़कियाँ भी दी गयीं, तो उन्होंने अपने परवरदिगार से दुआ की कि (बारे इलाहा मैं) इनके मुक़ाबले में कमज़ोर हूँ
\end{hindi}}
\flushright{\begin{Arabic}
\quranayah[54][11]
\end{Arabic}}
\flushleft{\begin{hindi}
तो अब तू ही (इनसे) बदला ले तो हमने मूसलाधार पानी से आसमान के दरवाज़े खोल दिए
\end{hindi}}
\flushright{\begin{Arabic}
\quranayah[54][12]
\end{Arabic}}
\flushleft{\begin{hindi}
और ज़मीन से चश्में जारी कर दिए, तो एक काम के लिए जो मुक़र्रर हो चुका था (दोनों) पानी मिलकर एक हो गया
\end{hindi}}
\flushright{\begin{Arabic}
\quranayah[54][13]
\end{Arabic}}
\flushleft{\begin{hindi}
और हमने एक कश्ती पर जो तख्तों और कीलों से तैयार की गयी थी सवार किया
\end{hindi}}
\flushright{\begin{Arabic}
\quranayah[54][14]
\end{Arabic}}
\flushleft{\begin{hindi}
और वह हमारी निगरानी में चल रही थी (ये) उस शख़्श (नूह) का बदला लेने के लिए जिसको लोग न मानते थे
\end{hindi}}
\flushright{\begin{Arabic}
\quranayah[54][15]
\end{Arabic}}
\flushleft{\begin{hindi}
और हमने उसको एक इबरत बना कर छोड़ा तो कोई है जो इबरत हासिल करे
\end{hindi}}
\flushright{\begin{Arabic}
\quranayah[54][16]
\end{Arabic}}
\flushleft{\begin{hindi}
तो (उनको) मेरा अज़ाब और डराना कैसा था
\end{hindi}}
\flushright{\begin{Arabic}
\quranayah[54][17]
\end{Arabic}}
\flushleft{\begin{hindi}
और हमने तो क़ुरान को नसीहत हासिल करने के वास्ते आसान कर दिया है तो कोई है जो नसीहत हासिल करे
\end{hindi}}
\flushright{\begin{Arabic}
\quranayah[54][18]
\end{Arabic}}
\flushleft{\begin{hindi}
आद (की क़ौम ने) (अपने पैग़म्बर) को झुठलाया तो (उनका) मेरा अज़ाब और डराना कैसा था,
\end{hindi}}
\flushright{\begin{Arabic}
\quranayah[54][19]
\end{Arabic}}
\flushleft{\begin{hindi}
हमने उन पर बहुत सख्त मनहूस दिन में बड़े ज़न्नाटे की ऑंधी चलायी
\end{hindi}}
\flushright{\begin{Arabic}
\quranayah[54][20]
\end{Arabic}}
\flushleft{\begin{hindi}
जो लोगों को (अपनी जगह से) इस तरह उखाड़ फेकती थी गोया वह उखड़े हुए खजूर के तने हैं
\end{hindi}}
\flushright{\begin{Arabic}
\quranayah[54][21]
\end{Arabic}}
\flushleft{\begin{hindi}
तो (उनको) मेरा अज़ाब और डराना कैसा था
\end{hindi}}
\flushright{\begin{Arabic}
\quranayah[54][22]
\end{Arabic}}
\flushleft{\begin{hindi}
और हमने तो क़ुरान को नसीहत हासिल करने के वास्ते आसान कर दिया, तो कोई है जो नसीहत हासिल करे
\end{hindi}}
\flushright{\begin{Arabic}
\quranayah[54][23]
\end{Arabic}}
\flushleft{\begin{hindi}
(क़ौम) समूद ने डराने वाले (पैग़म्बरों) को झुठलाया
\end{hindi}}
\flushright{\begin{Arabic}
\quranayah[54][24]
\end{Arabic}}
\flushleft{\begin{hindi}
तो कहने लगे कि भला एक आदमी की जो हम ही में से हो उसकी पैरवीं करें ऐसा करें तो गुमराही और दीवानगी में पड़ गए
\end{hindi}}
\flushright{\begin{Arabic}
\quranayah[54][25]
\end{Arabic}}
\flushleft{\begin{hindi}
क्या हम सबमें बस उसी पर वही नाज़िल हुई है (नहीं) बल्कि ये तो बड़ा झूठा तअल्ली करने वाला है
\end{hindi}}
\flushright{\begin{Arabic}
\quranayah[54][26]
\end{Arabic}}
\flushleft{\begin{hindi}
उनको अनक़रीब कल ही मालूम हो जाएगा कि कौन बड़ा झूठा तकब्बुर करने वाला है
\end{hindi}}
\flushright{\begin{Arabic}
\quranayah[54][27]
\end{Arabic}}
\flushleft{\begin{hindi}
(ऐ सालेह) हम उनकी आज़माइश के लिए ऊँटनी भेजने वाले हैं तो तुम उनको देखते रहो और (थोड़ा) सब्र करो
\end{hindi}}
\flushright{\begin{Arabic}
\quranayah[54][28]
\end{Arabic}}
\flushleft{\begin{hindi}
और उनको ख़बर कर दो कि उनमें पानी की बारी मुक़र्रर कर दी गयी है हर (बारी वाले को अपनी) बारी पर हाज़िर होना चाहिए
\end{hindi}}
\flushright{\begin{Arabic}
\quranayah[54][29]
\end{Arabic}}
\flushleft{\begin{hindi}
तो उन लोगों ने अपने रफीक़ (क़ेदार) को बुलाया तो उसने पकड़ कर (ऊँटनी की) कूंचे काट डालीं
\end{hindi}}
\flushright{\begin{Arabic}
\quranayah[54][30]
\end{Arabic}}
\flushleft{\begin{hindi}
तो (देखो) मेरा अज़ाब और डराना कैसा था
\end{hindi}}
\flushright{\begin{Arabic}
\quranayah[54][31]
\end{Arabic}}
\flushleft{\begin{hindi}
हमने उन पर एक सख्त चिंघाड़ (का अज़ाब) भेज दिया तो वह बाड़े वालो के सूखे हुए चूर चूर भूसे की तरह हो गए
\end{hindi}}
\flushright{\begin{Arabic}
\quranayah[54][32]
\end{Arabic}}
\flushleft{\begin{hindi}
और हमने क़ुरान को नसीहत हासिल करने के वास्ते आसान कर दिया है तो कोई है जो नसीहत हासिल करे
\end{hindi}}
\flushright{\begin{Arabic}
\quranayah[54][33]
\end{Arabic}}
\flushleft{\begin{hindi}
लूत की क़ौम ने भी डराने वाले (पैग़म्बरों) को झुठलाया
\end{hindi}}
\flushright{\begin{Arabic}
\quranayah[54][34]
\end{Arabic}}
\flushleft{\begin{hindi}
तो हमने उन पर कंकर भरी हवा चलाई मगर लूत के लड़के बाले को हमने उनको अपने फज़ल व करम से पिछले ही को बचा लिया
\end{hindi}}
\flushright{\begin{Arabic}
\quranayah[54][35]
\end{Arabic}}
\flushleft{\begin{hindi}
हम शुक्र करने वालों को ऐसा ही बदला दिया करते हैं
\end{hindi}}
\flushright{\begin{Arabic}
\quranayah[54][36]
\end{Arabic}}
\flushleft{\begin{hindi}
और लूत ने उनको हमारी पकड़ से भी डराया था मगर उन लोगों ने डराते ही में शक़ किया
\end{hindi}}
\flushright{\begin{Arabic}
\quranayah[54][37]
\end{Arabic}}
\flushleft{\begin{hindi}
और उनसे उनके मेहमान (फ़रिश्ते) के बारे में नाजायज़ मतलब की ख्वाहिश की तो हमने उनकी ऑंखें अन्धी कर दीं तो मेरे अज़ाब और डराने का मज़ा चखो
\end{hindi}}
\flushright{\begin{Arabic}
\quranayah[54][38]
\end{Arabic}}
\flushleft{\begin{hindi}
और सुबह सवेरे ही उन पर अज़ाब आ गया जो किसी तरह टल ही नहीं सकता था
\end{hindi}}
\flushright{\begin{Arabic}
\quranayah[54][39]
\end{Arabic}}
\flushleft{\begin{hindi}
तो मेरे अज़ाब और डराने के (पड़े) मज़े चखो
\end{hindi}}
\flushright{\begin{Arabic}
\quranayah[54][40]
\end{Arabic}}
\flushleft{\begin{hindi}
और हमने तो क़ुरान को नसीहत हासिल करने के वास्ते आसान कर दिया तो कोई है जो नसीहत हासिल करे
\end{hindi}}
\flushright{\begin{Arabic}
\quranayah[54][41]
\end{Arabic}}
\flushleft{\begin{hindi}
और फिरऔन के पास भी डराने वाले (पैग़म्बर) आए
\end{hindi}}
\flushright{\begin{Arabic}
\quranayah[54][42]
\end{Arabic}}
\flushleft{\begin{hindi}
तो उन लोगों ने हमारी कुल निशानियों को झुठलाया तो हमने उनको इस तरह सख्त पकड़ा जिस तरह एक ज़बरदस्त साहिबे क़ुदरत पकड़ा करता है
\end{hindi}}
\flushright{\begin{Arabic}
\quranayah[54][43]
\end{Arabic}}
\flushleft{\begin{hindi}
(ऐ अहले मक्का) क्या उन लोगों से भी तुम्हारे कुफ्फार बढ़ कर हैं या तुम्हारे वास्ते (पहली) किताबों में माफी (लिखी हुई) है
\end{hindi}}
\flushright{\begin{Arabic}
\quranayah[54][44]
\end{Arabic}}
\flushleft{\begin{hindi}
क्या ये लोग कहते हैं कि हम बहुत क़वी जमाअत हैं
\end{hindi}}
\flushright{\begin{Arabic}
\quranayah[54][45]
\end{Arabic}}
\flushleft{\begin{hindi}
अनक़रीब ही ये जमाअत शिकस्त खाएगी और ये लोग पीठ फेर कर भाग जाएँगे
\end{hindi}}
\flushright{\begin{Arabic}
\quranayah[54][46]
\end{Arabic}}
\flushleft{\begin{hindi}
बात ये है कि इनके वायदे का वक्त क़यामत है और क़यामत बड़ी सख्त और बड़ी तल्ख़ (चीज़) है
\end{hindi}}
\flushright{\begin{Arabic}
\quranayah[54][47]
\end{Arabic}}
\flushleft{\begin{hindi}
बेशक गुनाहगार लोग गुमराही और दीवानगी में (मुब्तिला) हैं
\end{hindi}}
\flushright{\begin{Arabic}
\quranayah[54][48]
\end{Arabic}}
\flushleft{\begin{hindi}
उस रोज़ ये लोग अपने अपने मुँह के बल (जहन्नुम की) आग में घसीटे जाएँगे (और उनसे कहा जाएगा) अब जहन्नुम की आग का मज़ा चखो
\end{hindi}}
\flushright{\begin{Arabic}
\quranayah[54][49]
\end{Arabic}}
\flushleft{\begin{hindi}
बेशक हमने हर चीज़ एक मुक़र्रर अन्दाज़ से पैदा की है
\end{hindi}}
\flushright{\begin{Arabic}
\quranayah[54][50]
\end{Arabic}}
\flushleft{\begin{hindi}
और हमारा हुक्म तो बस ऑंख के झपकने की तरह एक बात होती है
\end{hindi}}
\flushright{\begin{Arabic}
\quranayah[54][51]
\end{Arabic}}
\flushleft{\begin{hindi}
और हम तुम्हारे हम मशरबो को हलाक कर चुके हैं तो कोई है जो नसीहत हासिल करे
\end{hindi}}
\flushright{\begin{Arabic}
\quranayah[54][52]
\end{Arabic}}
\flushleft{\begin{hindi}
और अगर चे ये लोग जो कुछ कर चुके हैं (इनके) आमाल नामों में (दर्ज) है
\end{hindi}}
\flushright{\begin{Arabic}
\quranayah[54][53]
\end{Arabic}}
\flushleft{\begin{hindi}
(यानि) हर छोटा और बड़ा काम लिख दिया गया है
\end{hindi}}
\flushright{\begin{Arabic}
\quranayah[54][54]
\end{Arabic}}
\flushleft{\begin{hindi}
बेशक परहेज़गार लोग (बेहिश्त के) बाग़ों और नहरों में
\end{hindi}}
\flushright{\begin{Arabic}
\quranayah[54][55]
\end{Arabic}}
\flushleft{\begin{hindi}
(यानि) पसन्दीदा मक़ाम में हर तरह की कुदरत रखने वाले बादशाह की बारगाह में (मुक़र्रिब) होंगे
\end{hindi}}
\chapter{Ar-Rahman (The Beneficent)}
\begin{Arabic}
\Huge{\centerline{\basmalah}}\end{Arabic}
\flushright{\begin{Arabic}
\quranayah[55][1]
\end{Arabic}}
\flushleft{\begin{hindi}
बड़ा मेहरबान (ख़ुदा)
\end{hindi}}
\flushright{\begin{Arabic}
\quranayah[55][2]
\end{Arabic}}
\flushleft{\begin{hindi}
उसी ने क़ुरान की तालीम फरमाई
\end{hindi}}
\flushright{\begin{Arabic}
\quranayah[55][3]
\end{Arabic}}
\flushleft{\begin{hindi}
उसी ने इन्सान को पैदा किया
\end{hindi}}
\flushright{\begin{Arabic}
\quranayah[55][4]
\end{Arabic}}
\flushleft{\begin{hindi}
उसी ने उनको (अपना मतलब) बयान करना सिखाया
\end{hindi}}
\flushright{\begin{Arabic}
\quranayah[55][5]
\end{Arabic}}
\flushleft{\begin{hindi}
सूरज और चाँद एक मुक़र्रर हिसाब से चल रहे हैं
\end{hindi}}
\flushright{\begin{Arabic}
\quranayah[55][6]
\end{Arabic}}
\flushleft{\begin{hindi}
और बूटियाँ बेलें, और दरख्त (उसी को) सजदा करते हैं
\end{hindi}}
\flushright{\begin{Arabic}
\quranayah[55][7]
\end{Arabic}}
\flushleft{\begin{hindi}
और उसी ने आसमान बुलन्द किया और तराजू (इन्साफ) को क़ायम किया
\end{hindi}}
\flushright{\begin{Arabic}
\quranayah[55][8]
\end{Arabic}}
\flushleft{\begin{hindi}
ताकि तुम लोग तराज़ू (से तौलने) में हद से तजाउज़ न करो
\end{hindi}}
\flushright{\begin{Arabic}
\quranayah[55][9]
\end{Arabic}}
\flushleft{\begin{hindi}
और ईन्साफ के साथ ठीक तौलो और तौल कम न करो
\end{hindi}}
\flushright{\begin{Arabic}
\quranayah[55][10]
\end{Arabic}}
\flushleft{\begin{hindi}
और उसी ने लोगों के नफे क़े लिए ज़मीन बनायी
\end{hindi}}
\flushright{\begin{Arabic}
\quranayah[55][11]
\end{Arabic}}
\flushleft{\begin{hindi}
कि उसमें मेवे और खजूर के दरख्त हैं जिसके ख़ोशों में ग़िलाफ़ होते हैं
\end{hindi}}
\flushright{\begin{Arabic}
\quranayah[55][12]
\end{Arabic}}
\flushleft{\begin{hindi}
और अनाज जिसके साथ भुस होता है और ख़ुशबूदार फूल
\end{hindi}}
\flushright{\begin{Arabic}
\quranayah[55][13]
\end{Arabic}}
\flushleft{\begin{hindi}
तो (ऐ गिरोह जिन व इन्स) तुम दोनों अपने परवरदिगार की कौन कौन सी नेअमतों को न मानोगे
\end{hindi}}
\flushright{\begin{Arabic}
\quranayah[55][14]
\end{Arabic}}
\flushleft{\begin{hindi}
उसी ने इन्सान को ठीकरे की तरह खन खनाती हुई मिटटी से पैदा किया
\end{hindi}}
\flushright{\begin{Arabic}
\quranayah[55][15]
\end{Arabic}}
\flushleft{\begin{hindi}
और उसी ने जिन्नात को आग के शोले से पैदा किया
\end{hindi}}
\flushright{\begin{Arabic}
\quranayah[55][16]
\end{Arabic}}
\flushleft{\begin{hindi}
तो (ऐ गिरोह जिन व इन्स) तुम अपने परवरदिगार की कौन कौन सी नेअमतों से मुकरोगे
\end{hindi}}
\flushright{\begin{Arabic}
\quranayah[55][17]
\end{Arabic}}
\flushleft{\begin{hindi}
वही जाड़े गर्मी के दोनों मशरिकों का मालिक है और दोनों मग़रिबों का (भी) मालिक है
\end{hindi}}
\flushright{\begin{Arabic}
\quranayah[55][18]
\end{Arabic}}
\flushleft{\begin{hindi}
तो (ऐ जिनों) और (आदमियों) तुम अपने परवरदिगार की किस किस नेअमत से इन्कार करोगे
\end{hindi}}
\flushright{\begin{Arabic}
\quranayah[55][19]
\end{Arabic}}
\flushleft{\begin{hindi}
उसी ने दरिया बहाए जो बाहम मिल जाते हैं
\end{hindi}}
\flushright{\begin{Arabic}
\quranayah[55][20]
\end{Arabic}}
\flushleft{\begin{hindi}
दो के दरमियान एक हद्दे फ़ासिल (आड़) है जिससे तजाउज़ नहीं कर सकते
\end{hindi}}
\flushright{\begin{Arabic}
\quranayah[55][21]
\end{Arabic}}
\flushleft{\begin{hindi}
तो (ऐ जिन व इन्स) तुम दोनों अपने परवरदिगार की किस किस नेअमत को झुठलाओगे
\end{hindi}}
\flushright{\begin{Arabic}
\quranayah[55][22]
\end{Arabic}}
\flushleft{\begin{hindi}
इन दोनों दरियाओं से मोती और मूँगे निकलते हैं
\end{hindi}}
\flushright{\begin{Arabic}
\quranayah[55][23]
\end{Arabic}}
\flushleft{\begin{hindi}
(तो जिन व इन्स) तुम दोनों अपने परवरदिगार की कौन कौन सी नेअमत को न मानोगे
\end{hindi}}
\flushright{\begin{Arabic}
\quranayah[55][24]
\end{Arabic}}
\flushleft{\begin{hindi}
और जहाज़ जो दरिया में पहाड़ों की तरह ऊँचे खड़े रहते हैं उसी के हैं
\end{hindi}}
\flushright{\begin{Arabic}
\quranayah[55][25]
\end{Arabic}}
\flushleft{\begin{hindi}
तो (ऐ जिन व इन्स) तुम अपने परवरदिगार की किस किस नेअमत को झुठलाओगे
\end{hindi}}
\flushright{\begin{Arabic}
\quranayah[55][26]
\end{Arabic}}
\flushleft{\begin{hindi}
जो (मख़लूक) ज़मीन पर है सब फ़ना होने वाली है
\end{hindi}}
\flushright{\begin{Arabic}
\quranayah[55][27]
\end{Arabic}}
\flushleft{\begin{hindi}
और सिर्फ तुम्हारे परवरदिगार की ज़ात जो अज़मत और करामत वाली है बाक़ी रहेगी
\end{hindi}}
\flushright{\begin{Arabic}
\quranayah[55][28]
\end{Arabic}}
\flushleft{\begin{hindi}
तो तुम दोनों अपने मालिक की किस किस नेअमत से इन्कार करोगे
\end{hindi}}
\flushright{\begin{Arabic}
\quranayah[55][29]
\end{Arabic}}
\flushleft{\begin{hindi}
और जितने लोग सारे आसमान व ज़मीन में हैं (सब) उसी से माँगते हैं वह हर रोज़ (हर वक्त) मख़लूक के एक न एक काम में है
\end{hindi}}
\flushright{\begin{Arabic}
\quranayah[55][30]
\end{Arabic}}
\flushleft{\begin{hindi}
तो तुम दोनों अपने सरपरस्त की कौन कौन सी नेअमत से मुकरोगे
\end{hindi}}
\flushright{\begin{Arabic}
\quranayah[55][31]
\end{Arabic}}
\flushleft{\begin{hindi}
(ऐ दोनों गिरोहों) हम अनक़रीब ही तुम्हारी तरफ मुतावज्जे होंगे
\end{hindi}}
\flushright{\begin{Arabic}
\quranayah[55][32]
\end{Arabic}}
\flushleft{\begin{hindi}
तो तुम दोनों अपने पालने वाले की किस किस नेअमत को न मानोगे
\end{hindi}}
\flushright{\begin{Arabic}
\quranayah[55][33]
\end{Arabic}}
\flushleft{\begin{hindi}
ऐ गिरोह जिन व इन्स अगर तुममें क़ुदरत है कि आसमानों और ज़मीन के किनारों से (होकर कहीं) निकल (कर मौत या अज़ाब से भाग) सको तो निकल जाओ (मगर) तुम तो बग़ैर क़ूवत और ग़लबे के निकल ही नहीं सकते (हालॉ कि तुममें न क़ूवत है और न ही ग़लबा)
\end{hindi}}
\flushright{\begin{Arabic}
\quranayah[55][34]
\end{Arabic}}
\flushleft{\begin{hindi}
तो तुम अपने परवरदिगार की किस किस नेअमत को झुठलाओगे
\end{hindi}}
\flushright{\begin{Arabic}
\quranayah[55][35]
\end{Arabic}}
\flushleft{\begin{hindi}
(गुनाहगार जिनों और आदमियों जहन्नुम में) तुम दोनो पर आग का सब्ज़ शोला और सियाह धुऑं छोड़ दिया जाएगा तो तुम दोनों (किस तरह) रोक नहीं सकोगे
\end{hindi}}
\flushright{\begin{Arabic}
\quranayah[55][36]
\end{Arabic}}
\flushleft{\begin{hindi}
फिर तुम दोनों अपने परवरदिगार की किस किस नेअमत से इन्कार करोगे
\end{hindi}}
\flushright{\begin{Arabic}
\quranayah[55][37]
\end{Arabic}}
\flushleft{\begin{hindi}
फिर जब आसमान फट कर (क़यामत में) तेल की तरह लाल हो जाऐगा
\end{hindi}}
\flushright{\begin{Arabic}
\quranayah[55][38]
\end{Arabic}}
\flushleft{\begin{hindi}
तो तुम दोनों अपने परवरदिगार की किस किस नेअमत से मुकरोगे
\end{hindi}}
\flushright{\begin{Arabic}
\quranayah[55][39]
\end{Arabic}}
\flushleft{\begin{hindi}
तो उस दिन न तो किसी इन्सान से उसके गुनाह के बारे में पूछा जाएगा न किसी जिन से
\end{hindi}}
\flushright{\begin{Arabic}
\quranayah[55][40]
\end{Arabic}}
\flushleft{\begin{hindi}
तो तुम दोनों अपने मालिक की किस किस नेअमत को न मानोगे
\end{hindi}}
\flushright{\begin{Arabic}
\quranayah[55][41]
\end{Arabic}}
\flushleft{\begin{hindi}
गुनाहगार लोग तो अपने चेहरों ही से पहचान लिए जाएँगे तो पेशानी के पटटे और पाँव पकड़े (जहन्नुम में डाल दिये जाएँगे)
\end{hindi}}
\flushright{\begin{Arabic}
\quranayah[55][42]
\end{Arabic}}
\flushleft{\begin{hindi}
आख़िर तुम दोनों अपने परवरदिगार की किस किस नेअमत से इन्कार करोगे
\end{hindi}}
\flushright{\begin{Arabic}
\quranayah[55][43]
\end{Arabic}}
\flushleft{\begin{hindi}
(फिर उनसे कहा जाएगा) यही वह जहन्नुम है जिसे गुनाहगार लोग झुठलाया करते थे
\end{hindi}}
\flushright{\begin{Arabic}
\quranayah[55][44]
\end{Arabic}}
\flushleft{\begin{hindi}
ये लोग दोज़ख़ और हद दरजा खौलते हुए पानी के दरमियान (बेक़रार दौड़ते) चक्कर लगाते फिरेंगे
\end{hindi}}
\flushright{\begin{Arabic}
\quranayah[55][45]
\end{Arabic}}
\flushleft{\begin{hindi}
तो तुम दोनों अपने परवरदिगार की किस किस नेअमत को न मानोगे
\end{hindi}}
\flushright{\begin{Arabic}
\quranayah[55][46]
\end{Arabic}}
\flushleft{\begin{hindi}
और जो शख्स अपने परवरदिगार के सामने खड़े होने से डरता रहा उसके लिए दो दो बाग़ हैं
\end{hindi}}
\flushright{\begin{Arabic}
\quranayah[55][47]
\end{Arabic}}
\flushleft{\begin{hindi}
तो तुम दोनों अपने परवरदिगार की कौन कौन सी नेअमत से इन्कार करोगे
\end{hindi}}
\flushright{\begin{Arabic}
\quranayah[55][48]
\end{Arabic}}
\flushleft{\begin{hindi}
दोनों बाग़ (दरख्तों की) टहनियों से हरे भरे (मेवों से लदे) हुए
\end{hindi}}
\flushright{\begin{Arabic}
\quranayah[55][49]
\end{Arabic}}
\flushleft{\begin{hindi}
फिर दोनों अपने सरपरस्त की किस किस नेअमतों को झुठलाओगे
\end{hindi}}
\flushright{\begin{Arabic}
\quranayah[55][50]
\end{Arabic}}
\flushleft{\begin{hindi}
इन दोनों में दो चश्में जारी होंगें
\end{hindi}}
\flushright{\begin{Arabic}
\quranayah[55][51]
\end{Arabic}}
\flushleft{\begin{hindi}
तो तुम दोनों अपने परवरदिगार की किस किस नेअमत से मुकरोगे
\end{hindi}}
\flushright{\begin{Arabic}
\quranayah[55][52]
\end{Arabic}}
\flushleft{\begin{hindi}
इन दोनों बाग़ों में सब मेवे दो दो किस्म के होंगे
\end{hindi}}
\flushright{\begin{Arabic}
\quranayah[55][53]
\end{Arabic}}
\flushleft{\begin{hindi}
तुम दोनों अपने परवरदिगार की किस किस नेअमत से इन्कार करोगे
\end{hindi}}
\flushright{\begin{Arabic}
\quranayah[55][54]
\end{Arabic}}
\flushleft{\begin{hindi}
यह लोग उन फ़र्शों पर जिनके असतर अतलस के होंगे तकिये लगाकर बैठे होंगे तो दोनों बाग़ों के मेवे (इस क़दर) क़रीब होंगे (कि अगर चाहे तो लगे हुए खालें)
\end{hindi}}
\flushright{\begin{Arabic}
\quranayah[55][55]
\end{Arabic}}
\flushleft{\begin{hindi}
तो तुम दोनों अपने मालिक की किस किस नेअमत को न मानोगे
\end{hindi}}
\flushright{\begin{Arabic}
\quranayah[55][56]
\end{Arabic}}
\flushleft{\begin{hindi}
इसमें (पाक दामन ग़ैर की तरफ ऑंख उठा कर न देखने वाली औरतें होंगी जिनको उन से पहले न किसी इन्सान ने हाथ लगाया होगा) और जिन ने
\end{hindi}}
\flushright{\begin{Arabic}
\quranayah[55][57]
\end{Arabic}}
\flushleft{\begin{hindi}
तो तुम दोनों अपने परवरदिगार की किन किन नेअमतों को झुठलाओगे
\end{hindi}}
\flushright{\begin{Arabic}
\quranayah[55][58]
\end{Arabic}}
\flushleft{\begin{hindi}
(ऐसी हसीन) गोया वह (मुजस्सिम) याक़ूत व मूँगे हैं
\end{hindi}}
\flushright{\begin{Arabic}
\quranayah[55][59]
\end{Arabic}}
\flushleft{\begin{hindi}
तो तुम दोनों अपने परवरदिगार की किन किन नेअमतों से मुकरोगे
\end{hindi}}
\flushright{\begin{Arabic}
\quranayah[55][60]
\end{Arabic}}
\flushleft{\begin{hindi}
भला नेकी का बदला नेकी के सिवा कुछ और भी है
\end{hindi}}
\flushright{\begin{Arabic}
\quranayah[55][61]
\end{Arabic}}
\flushleft{\begin{hindi}
फिर तुम दोनों अपने मालिक की किस किस नेअमत को झुठलाओगे
\end{hindi}}
\flushright{\begin{Arabic}
\quranayah[55][62]
\end{Arabic}}
\flushleft{\begin{hindi}
उन दोनों बाग़ों के अलावा दो बाग़ और हैं
\end{hindi}}
\flushright{\begin{Arabic}
\quranayah[55][63]
\end{Arabic}}
\flushleft{\begin{hindi}
तो तुम दोनों अपने पालने वाले की किस किस नेअमत से इन्कार करोगे
\end{hindi}}
\flushright{\begin{Arabic}
\quranayah[55][64]
\end{Arabic}}
\flushleft{\begin{hindi}
दोनों निहायत गहरे सब्ज़ व शादाब
\end{hindi}}
\flushright{\begin{Arabic}
\quranayah[55][65]
\end{Arabic}}
\flushleft{\begin{hindi}
तो तुम दोनों अपने सरपरस्त की किन किन नेअमतों को न मानोगे
\end{hindi}}
\flushright{\begin{Arabic}
\quranayah[55][66]
\end{Arabic}}
\flushleft{\begin{hindi}
उन दोनों बाग़ों में दो चश्में जोश मारते होंगे
\end{hindi}}
\flushright{\begin{Arabic}
\quranayah[55][67]
\end{Arabic}}
\flushleft{\begin{hindi}
तो तुम दोनों अपने परवरदिगार की किस किस नेअमत से मुकरोगे
\end{hindi}}
\flushright{\begin{Arabic}
\quranayah[55][68]
\end{Arabic}}
\flushleft{\begin{hindi}
उन दोनों में मेवें हैं खुरमें और अनार
\end{hindi}}
\flushright{\begin{Arabic}
\quranayah[55][69]
\end{Arabic}}
\flushleft{\begin{hindi}
तो तुम दोनों अपने मालिक की किन किन नेअमतों को झुठलाओगे
\end{hindi}}
\flushright{\begin{Arabic}
\quranayah[55][70]
\end{Arabic}}
\flushleft{\begin{hindi}
उन बाग़ों में ख़ुश ख़ुल्क और ख़ूबसूरत औरतें होंगी
\end{hindi}}
\flushright{\begin{Arabic}
\quranayah[55][71]
\end{Arabic}}
\flushleft{\begin{hindi}
तो तुम दोनों अपने मालिक की किन किन नेअमतों को झुठलाओगे
\end{hindi}}
\flushright{\begin{Arabic}
\quranayah[55][72]
\end{Arabic}}
\flushleft{\begin{hindi}
वह हूरें हैं जो ख़ेमों में छुपी बैठी हैं
\end{hindi}}
\flushright{\begin{Arabic}
\quranayah[55][73]
\end{Arabic}}
\flushleft{\begin{hindi}
फिर तुम दोनों अपने परवरदिगार की कौन कौन सी नेअमत से इन्कार करोगे
\end{hindi}}
\flushright{\begin{Arabic}
\quranayah[55][74]
\end{Arabic}}
\flushleft{\begin{hindi}
उनसे पहले उनको किसी इन्सान ने उनको छुआ तक नहीं और न जिन ने
\end{hindi}}
\flushright{\begin{Arabic}
\quranayah[55][75]
\end{Arabic}}
\flushleft{\begin{hindi}
फिर तुम दोनों अपने मालिक की किस किस नेअमत से मुकरोगे
\end{hindi}}
\flushright{\begin{Arabic}
\quranayah[55][76]
\end{Arabic}}
\flushleft{\begin{hindi}
ये लोग सब्ज़ कालीनों और नफीस व हसीन मसनदों पर तकिए लगाए (बैठे) होंगे
\end{hindi}}
\flushright{\begin{Arabic}
\quranayah[55][77]
\end{Arabic}}
\flushleft{\begin{hindi}
फिर तुम अपने परवरदिगार की किन किन नेअमतों से इन्कार करोगे
\end{hindi}}
\flushright{\begin{Arabic}
\quranayah[55][78]
\end{Arabic}}
\flushleft{\begin{hindi}
(ऐ रसूल) तुम्हारा परवरदिगार जो साहिबे जलाल व करामत है उसी का नाम बड़ा बाबरकत है
\end{hindi}}
\chapter{Al-Waqi'ah (The Event)}
\begin{Arabic}
\Huge{\centerline{\basmalah}}\end{Arabic}
\flushright{\begin{Arabic}
\quranayah[56][1]
\end{Arabic}}
\flushleft{\begin{hindi}
जब क़यामत बरपा होगी और उसके वाक़िया होने में ज़रा झूट नहीं
\end{hindi}}
\flushright{\begin{Arabic}
\quranayah[56][2]
\end{Arabic}}
\flushleft{\begin{hindi}
(उस वक्त लोगों में फ़र्क ज़ाहिर होगा)
\end{hindi}}
\flushright{\begin{Arabic}
\quranayah[56][3]
\end{Arabic}}
\flushleft{\begin{hindi}
कि किसी को पस्त करेगी किसी को बुलन्द
\end{hindi}}
\flushright{\begin{Arabic}
\quranayah[56][4]
\end{Arabic}}
\flushleft{\begin{hindi}
जब ज़मीन बड़े ज़ोरों में हिलने लगेगी
\end{hindi}}
\flushright{\begin{Arabic}
\quranayah[56][5]
\end{Arabic}}
\flushleft{\begin{hindi}
और पहाड़ (टकरा कर) बिल्कुल चूर चूर हो जाएँगे
\end{hindi}}
\flushright{\begin{Arabic}
\quranayah[56][6]
\end{Arabic}}
\flushleft{\begin{hindi}
फिर ज़र्रे बन कर उड़ने लगेंगे
\end{hindi}}
\flushright{\begin{Arabic}
\quranayah[56][7]
\end{Arabic}}
\flushleft{\begin{hindi}
और तुम लोग तीन किस्म हो जाओगे
\end{hindi}}
\flushright{\begin{Arabic}
\quranayah[56][8]
\end{Arabic}}
\flushleft{\begin{hindi}
तो दाहिने हाथ (में आमाल नामा लेने) वाले (वाह) दाहिने हाथ वाले क्या (चैन में) हैं
\end{hindi}}
\flushright{\begin{Arabic}
\quranayah[56][9]
\end{Arabic}}
\flushleft{\begin{hindi}
और बाएं हाथ (में आमाल नामा लेने) वाले (अफ़सोस) बाएं हाथ वाले क्या (मुसीबत में) हैं
\end{hindi}}
\flushright{\begin{Arabic}
\quranayah[56][10]
\end{Arabic}}
\flushleft{\begin{hindi}
और जो आगे बढ़ जाने वाले हैं (वाह क्या कहना) वह आगे ही बढ़ने वाले थे
\end{hindi}}
\flushright{\begin{Arabic}
\quranayah[56][11]
\end{Arabic}}
\flushleft{\begin{hindi}
यही लोग (ख़ुदा के) मुक़र्रिब हैं
\end{hindi}}
\flushright{\begin{Arabic}
\quranayah[56][12]
\end{Arabic}}
\flushleft{\begin{hindi}
आराम व आसाइश के बाग़ों में बहुत से
\end{hindi}}
\flushright{\begin{Arabic}
\quranayah[56][13]
\end{Arabic}}
\flushleft{\begin{hindi}
तो अगले लोगों में से होंगे
\end{hindi}}
\flushright{\begin{Arabic}
\quranayah[56][14]
\end{Arabic}}
\flushleft{\begin{hindi}
और कुछ थोडे से पिछले लोगों में से मोती
\end{hindi}}
\flushright{\begin{Arabic}
\quranayah[56][15]
\end{Arabic}}
\flushleft{\begin{hindi}
और याक़ूत से जड़े हुए सोने के तारों से बने हुए
\end{hindi}}
\flushright{\begin{Arabic}
\quranayah[56][16]
\end{Arabic}}
\flushleft{\begin{hindi}
तख्ते पर एक दूसरे के सामने तकिए लगाए (बैठे) होंगे
\end{hindi}}
\flushright{\begin{Arabic}
\quranayah[56][17]
\end{Arabic}}
\flushleft{\begin{hindi}
नौजवान लड़के जो (बेहिश्त में) हमेशा (लड़के ही बने) रहेंगे
\end{hindi}}
\flushright{\begin{Arabic}
\quranayah[56][18]
\end{Arabic}}
\flushleft{\begin{hindi}
(शरबत वग़ैरह के) सागर और चमकदार टोंटीदार कंटर और शफ्फ़ाफ़ शराब के जाम लिए हुए उनके पास चक्कर लगाते होंगे
\end{hindi}}
\flushright{\begin{Arabic}
\quranayah[56][19]
\end{Arabic}}
\flushleft{\begin{hindi}
जिसके (पीने) से न तो उनको (ख़ुमार से) दर्दसर होगा और न वह बदहवास मदहोश होंगे
\end{hindi}}
\flushright{\begin{Arabic}
\quranayah[56][20]
\end{Arabic}}
\flushleft{\begin{hindi}
और जिस क़िस्म के मेवे पसन्द करें
\end{hindi}}
\flushright{\begin{Arabic}
\quranayah[56][21]
\end{Arabic}}
\flushleft{\begin{hindi}
और जिस क़िस्म के परिन्दे का गोश्त उनका जी चाहे (सब मौजूद है)
\end{hindi}}
\flushright{\begin{Arabic}
\quranayah[56][22]
\end{Arabic}}
\flushleft{\begin{hindi}
और बड़ी बड़ी ऑंखों वाली हूरें
\end{hindi}}
\flushright{\begin{Arabic}
\quranayah[56][23]
\end{Arabic}}
\flushleft{\begin{hindi}
जैसे एहतेयात से रखे हुए मोती
\end{hindi}}
\flushright{\begin{Arabic}
\quranayah[56][24]
\end{Arabic}}
\flushleft{\begin{hindi}
ये बदला है उनके (नेक) आमाल का
\end{hindi}}
\flushright{\begin{Arabic}
\quranayah[56][25]
\end{Arabic}}
\flushleft{\begin{hindi}
वहाँ न तो बेहूदा बात सुनेंगे और न गुनाह की बात
\end{hindi}}
\flushright{\begin{Arabic}
\quranayah[56][26]
\end{Arabic}}
\flushleft{\begin{hindi}
(फहश) बस उनका कलाम सलाम ही सलाम होगा
\end{hindi}}
\flushright{\begin{Arabic}
\quranayah[56][27]
\end{Arabic}}
\flushleft{\begin{hindi}
और दाहिने हाथ वाले (वाह) दाहिने हाथ वालों का क्या कहना है
\end{hindi}}
\flushright{\begin{Arabic}
\quranayah[56][28]
\end{Arabic}}
\flushleft{\begin{hindi}
बे काँटे की बेरो और लदे गुथे हुए
\end{hindi}}
\flushright{\begin{Arabic}
\quranayah[56][29]
\end{Arabic}}
\flushleft{\begin{hindi}
केलों और लम्बी लम्बी छाँव
\end{hindi}}
\flushright{\begin{Arabic}
\quranayah[56][30]
\end{Arabic}}
\flushleft{\begin{hindi}
और झरनो के पानी
\end{hindi}}
\flushright{\begin{Arabic}
\quranayah[56][31]
\end{Arabic}}
\flushleft{\begin{hindi}
और अनारों
\end{hindi}}
\flushright{\begin{Arabic}
\quranayah[56][32]
\end{Arabic}}
\flushleft{\begin{hindi}
मेवो में होंगें
\end{hindi}}
\flushright{\begin{Arabic}
\quranayah[56][33]
\end{Arabic}}
\flushleft{\begin{hindi}
जो न कभी खत्म होंगे और न उनकी कोई रोक टोक
\end{hindi}}
\flushright{\begin{Arabic}
\quranayah[56][34]
\end{Arabic}}
\flushleft{\begin{hindi}
और ऊँचे ऊँचे (नरम गद्दो के) फ़र्शों में (मज़े करते) होंगे
\end{hindi}}
\flushright{\begin{Arabic}
\quranayah[56][35]
\end{Arabic}}
\flushleft{\begin{hindi}
(उनको) वह हूरें मिलेंगी जिसको हमने नित नया पैदा किया है
\end{hindi}}
\flushright{\begin{Arabic}
\quranayah[56][36]
\end{Arabic}}
\flushleft{\begin{hindi}
तो हमने उन्हें कुँवारियाँ प्यारी प्यारी हमजोलियाँ बनाया
\end{hindi}}
\flushright{\begin{Arabic}
\quranayah[56][37]
\end{Arabic}}
\flushleft{\begin{hindi}
(ये सब सामान)
\end{hindi}}
\flushright{\begin{Arabic}
\quranayah[56][38]
\end{Arabic}}
\flushleft{\begin{hindi}
दाहिने हाथ (में नामए आमाल लेने) वालों के वास्ते है
\end{hindi}}
\flushright{\begin{Arabic}
\quranayah[56][39]
\end{Arabic}}
\flushleft{\begin{hindi}
(इनमें) बहुत से तो अगले लोगों में से
\end{hindi}}
\flushright{\begin{Arabic}
\quranayah[56][40]
\end{Arabic}}
\flushleft{\begin{hindi}
और बहुत से पिछले लोगों में से
\end{hindi}}
\flushright{\begin{Arabic}
\quranayah[56][41]
\end{Arabic}}
\flushleft{\begin{hindi}
और बाएं हाथ (में नामए आमाल लेने) वाले (अफसोस) बाएं हाथ वाले क्या (मुसीबत में) हैं
\end{hindi}}
\flushright{\begin{Arabic}
\quranayah[56][42]
\end{Arabic}}
\flushleft{\begin{hindi}
(दोज़ख़ की) लौ और खौलते हुए पानी
\end{hindi}}
\flushright{\begin{Arabic}
\quranayah[56][43]
\end{Arabic}}
\flushleft{\begin{hindi}
और काले सियाह धुएँ के साये में होंगे
\end{hindi}}
\flushright{\begin{Arabic}
\quranayah[56][44]
\end{Arabic}}
\flushleft{\begin{hindi}
जो न ठन्डा और न ख़ुश आइन्द
\end{hindi}}
\flushright{\begin{Arabic}
\quranayah[56][45]
\end{Arabic}}
\flushleft{\begin{hindi}
ये लोग इससे पहले (दुनिया में) ख़ूब ऐश उड़ा चुके थे
\end{hindi}}
\flushright{\begin{Arabic}
\quranayah[56][46]
\end{Arabic}}
\flushleft{\begin{hindi}
और बड़े गुनाह (शिर्क) पर अड़े रहते थे
\end{hindi}}
\flushright{\begin{Arabic}
\quranayah[56][47]
\end{Arabic}}
\flushleft{\begin{hindi}
और कहा करते थे कि भला जब हम मर जाएँगे और (सड़ गल कर) मिटटी और हडिडयाँ (ही हडिडयाँ) रह जाएँगे
\end{hindi}}
\flushright{\begin{Arabic}
\quranayah[56][48]
\end{Arabic}}
\flushleft{\begin{hindi}
तो क्या हमें या हमारे अगले बाप दादाओं को फिर उठना है
\end{hindi}}
\flushright{\begin{Arabic}
\quranayah[56][49]
\end{Arabic}}
\flushleft{\begin{hindi}
(ऐ रसूल) तुम कह दो कि अगले और पिछले
\end{hindi}}
\flushright{\begin{Arabic}
\quranayah[56][50]
\end{Arabic}}
\flushleft{\begin{hindi}
सब के सब रोजे मुअय्यन की मियाद पर ज़रूर इकट्ठे किए जाएँगे
\end{hindi}}
\flushright{\begin{Arabic}
\quranayah[56][51]
\end{Arabic}}
\flushleft{\begin{hindi}
फिर तुमको बेशक ऐ गुमराहों झुठलाने वालों
\end{hindi}}
\flushright{\begin{Arabic}
\quranayah[56][52]
\end{Arabic}}
\flushleft{\begin{hindi}
यक़ीनन (जहन्नुम में) थोहड़ के दरख्तों में से खाना होगा
\end{hindi}}
\flushright{\begin{Arabic}
\quranayah[56][53]
\end{Arabic}}
\flushleft{\begin{hindi}
तो तुम लोगों को उसी से (अपना) पेट भरना होगा
\end{hindi}}
\flushright{\begin{Arabic}
\quranayah[56][54]
\end{Arabic}}
\flushleft{\begin{hindi}
फिर उसके ऊपर खौलता हुआ पानी पीना होगा
\end{hindi}}
\flushright{\begin{Arabic}
\quranayah[56][55]
\end{Arabic}}
\flushleft{\begin{hindi}
और पियोगे भी तो प्यासे ऊँट का सा (डग डगा के) पीना
\end{hindi}}
\flushright{\begin{Arabic}
\quranayah[56][56]
\end{Arabic}}
\flushleft{\begin{hindi}
क़यामत के दिन यही उनकी मेहमानी होगी
\end{hindi}}
\flushright{\begin{Arabic}
\quranayah[56][57]
\end{Arabic}}
\flushleft{\begin{hindi}
तुम लोगों को (पहली बार भी) हम ही ने पैदा किया है
\end{hindi}}
\flushright{\begin{Arabic}
\quranayah[56][58]
\end{Arabic}}
\flushleft{\begin{hindi}
फिर तुम लोग (दोबार की) क्यों नहीं तस्दीक़ करते
\end{hindi}}
\flushright{\begin{Arabic}
\quranayah[56][59]
\end{Arabic}}
\flushleft{\begin{hindi}
तो जिस नुत्फे क़ो तुम (औरतों के रहम में डालते हो) क्या तुमने देख भाल लिया है क्या तुम उससे आदमी बनाते हो या हम बनाते हैं
\end{hindi}}
\flushright{\begin{Arabic}
\quranayah[56][60]
\end{Arabic}}
\flushleft{\begin{hindi}
हमने तुम लोगों में मौत को मुक़र्रर कर दिया है और हम उससे आजिज़ नहीं हैं
\end{hindi}}
\flushright{\begin{Arabic}
\quranayah[56][61]
\end{Arabic}}
\flushleft{\begin{hindi}
कि तुम्हारे ऐसे और लोग बदल डालें और तुम लोगों को इस (सूरत) में पैदा करें जिसे तुम मुत्तलक़ नहीं जानते
\end{hindi}}
\flushright{\begin{Arabic}
\quranayah[56][62]
\end{Arabic}}
\flushleft{\begin{hindi}
और तुमने पैहली पैदाइश तो समझ ही ली है (कि हमने की) फिर तुम ग़ौर क्यों नहीं करते
\end{hindi}}
\flushright{\begin{Arabic}
\quranayah[56][63]
\end{Arabic}}
\flushleft{\begin{hindi}
भला देखो तो कि जो कुछ तुम लोग बोते हो क्या
\end{hindi}}
\flushright{\begin{Arabic}
\quranayah[56][64]
\end{Arabic}}
\flushleft{\begin{hindi}
तुम लोग उसे उगाते हो या हम उगाते हैं अगर हम चाहते
\end{hindi}}
\flushright{\begin{Arabic}
\quranayah[56][65]
\end{Arabic}}
\flushleft{\begin{hindi}
तो उसे चूर चूर कर देते तो तुम बातें ही बनाते रह जाते
\end{hindi}}
\flushright{\begin{Arabic}
\quranayah[56][66]
\end{Arabic}}
\flushleft{\begin{hindi}
कि (हाए) हम तो (मुफ्त) तावान में फॅसे (नहीं)
\end{hindi}}
\flushright{\begin{Arabic}
\quranayah[56][67]
\end{Arabic}}
\flushleft{\begin{hindi}
हम तो बदनसीब हैं
\end{hindi}}
\flushright{\begin{Arabic}
\quranayah[56][68]
\end{Arabic}}
\flushleft{\begin{hindi}
तो क्या तुमने पानी पर भी नज़र डाली जो (दिन रात) पीते हो
\end{hindi}}
\flushright{\begin{Arabic}
\quranayah[56][69]
\end{Arabic}}
\flushleft{\begin{hindi}
क्या उसको बादल से तुमने बरसाया है या हम बरसाते हैं
\end{hindi}}
\flushright{\begin{Arabic}
\quranayah[56][70]
\end{Arabic}}
\flushleft{\begin{hindi}
अगर हम चाहें तो उसे खारी बना दें तो तुम लोग यक्र क्यों नहीं करते
\end{hindi}}
\flushright{\begin{Arabic}
\quranayah[56][71]
\end{Arabic}}
\flushleft{\begin{hindi}
तो क्या तुमने आग पर भी ग़ौर किया जिसे तुम लोग लकड़ी से निकालते हो
\end{hindi}}
\flushright{\begin{Arabic}
\quranayah[56][72]
\end{Arabic}}
\flushleft{\begin{hindi}
क्या उसके दरख्त को तुमने पैदा किया या हम पैदा करते हैं
\end{hindi}}
\flushright{\begin{Arabic}
\quranayah[56][73]
\end{Arabic}}
\flushleft{\begin{hindi}
हमने आग को (जहन्नुम की) याद देहानी और मुसाफिरों के नफे के (वास्ते पैदा किया)
\end{hindi}}
\flushright{\begin{Arabic}
\quranayah[56][74]
\end{Arabic}}
\flushleft{\begin{hindi}
तो (ऐ रसूल) तुम अपने बुज़ुर्ग परवरदिगार की तस्बीह करो
\end{hindi}}
\flushright{\begin{Arabic}
\quranayah[56][75]
\end{Arabic}}
\flushleft{\begin{hindi}
तो मैं तारों के मनाज़िल की क़सम खाता हूँ
\end{hindi}}
\flushright{\begin{Arabic}
\quranayah[56][76]
\end{Arabic}}
\flushleft{\begin{hindi}
और अगर तुम समझो तो ये बड़ी क़सम है
\end{hindi}}
\flushright{\begin{Arabic}
\quranayah[56][77]
\end{Arabic}}
\flushleft{\begin{hindi}
कि बेशक ये बड़े रूतबे का क़ुरान है
\end{hindi}}
\flushright{\begin{Arabic}
\quranayah[56][78]
\end{Arabic}}
\flushleft{\begin{hindi}
जो किताब (लौहे महफूज़) में (लिखा हुआ) है
\end{hindi}}
\flushright{\begin{Arabic}
\quranayah[56][79]
\end{Arabic}}
\flushleft{\begin{hindi}
इसको बस वही लोग छूते हैं जो पाक हैं
\end{hindi}}
\flushright{\begin{Arabic}
\quranayah[56][80]
\end{Arabic}}
\flushleft{\begin{hindi}
सारे जहाँ के परवरदिगार की तरफ से (मोहम्मद पर) नाज़िल हुआ है
\end{hindi}}
\flushright{\begin{Arabic}
\quranayah[56][81]
\end{Arabic}}
\flushleft{\begin{hindi}
तो क्या तुम लोग इस कलाम से इन्कार रखते हो
\end{hindi}}
\flushright{\begin{Arabic}
\quranayah[56][82]
\end{Arabic}}
\flushleft{\begin{hindi}
और तुमने अपनी रोज़ी ये करार दे ली है कि (उसको) झुठलाते हो
\end{hindi}}
\flushright{\begin{Arabic}
\quranayah[56][83]
\end{Arabic}}
\flushleft{\begin{hindi}
तो क्या जब जान गले तक पहुँचती है
\end{hindi}}
\flushright{\begin{Arabic}
\quranayah[56][84]
\end{Arabic}}
\flushleft{\begin{hindi}
और तुम उस वक्त (क़ी हालत) पड़े देखा करते हो
\end{hindi}}
\flushright{\begin{Arabic}
\quranayah[56][85]
\end{Arabic}}
\flushleft{\begin{hindi}
और हम इस (मरने वाले) से तुमसे भी ज्यादा नज़दीक होते हैं लेकिन तुमको दिखाई नहीं देता
\end{hindi}}
\flushright{\begin{Arabic}
\quranayah[56][86]
\end{Arabic}}
\flushleft{\begin{hindi}
तो अगर तुम किसी के दबाव में नहीं हो
\end{hindi}}
\flushright{\begin{Arabic}
\quranayah[56][87]
\end{Arabic}}
\flushleft{\begin{hindi}
तो अगर (अपने दावे में) तुम सच्चे हो तो रूह को फेर क्यों नहीं देते
\end{hindi}}
\flushright{\begin{Arabic}
\quranayah[56][88]
\end{Arabic}}
\flushleft{\begin{hindi}
पस अगर वह (मरने वाला ख़ुदा के) मुक़र्रेबीन से है
\end{hindi}}
\flushright{\begin{Arabic}
\quranayah[56][89]
\end{Arabic}}
\flushleft{\begin{hindi}
तो (उस के लिए) आराम व आसाइश है और ख़ुशबूदार फूल और नेअमत के बाग़
\end{hindi}}
\flushright{\begin{Arabic}
\quranayah[56][90]
\end{Arabic}}
\flushleft{\begin{hindi}
और अगर वह दाहिने हाथ वालों में से है
\end{hindi}}
\flushright{\begin{Arabic}
\quranayah[56][91]
\end{Arabic}}
\flushleft{\begin{hindi}
तो (उससे कहा जाएगा कि) तुम पर दाहिने हाथ वालों की तरफ़ से सलाम हो
\end{hindi}}
\flushright{\begin{Arabic}
\quranayah[56][92]
\end{Arabic}}
\flushleft{\begin{hindi}
और अगर झुठलाने वाले गुमराहों में से है
\end{hindi}}
\flushright{\begin{Arabic}
\quranayah[56][93]
\end{Arabic}}
\flushleft{\begin{hindi}
तो (उसकी) मेहमानी खौलता हुआ पानी है
\end{hindi}}
\flushright{\begin{Arabic}
\quranayah[56][94]
\end{Arabic}}
\flushleft{\begin{hindi}
और जहन्नुम में दाखिल कर देना
\end{hindi}}
\flushright{\begin{Arabic}
\quranayah[56][95]
\end{Arabic}}
\flushleft{\begin{hindi}
बेशक ये (ख़बर) यक़ीनन सही है
\end{hindi}}
\flushright{\begin{Arabic}
\quranayah[56][96]
\end{Arabic}}
\flushleft{\begin{hindi}
तो (ऐ रसूल) तुम अपने बुज़ुर्ग परवरदिगार की तस्बीह करो
\end{hindi}}
\chapter{Al-Hadid (Iron)}
\begin{Arabic}
\Huge{\centerline{\basmalah}}\end{Arabic}
\flushright{\begin{Arabic}
\quranayah[57][1]
\end{Arabic}}
\flushleft{\begin{hindi}
जो जो चीज़ सारे आसमान व ज़मीन में है सब ख़ुदा की तसबीह करती है और वही ग़ालिब हिकमत वाला है
\end{hindi}}
\flushright{\begin{Arabic}
\quranayah[57][2]
\end{Arabic}}
\flushleft{\begin{hindi}
सारे आसमान व ज़मीन की बादशाही उसी की है वही जिलाता है वही मारता है और वही हर चीज़ पर कादिर है
\end{hindi}}
\flushright{\begin{Arabic}
\quranayah[57][3]
\end{Arabic}}
\flushleft{\begin{hindi}
वही सबसे पहले और सबसे आख़िर है और (अपनी क़ूवतों से) सब पर ज़ाहिर और (निगाहों से) पोशीदा है और वही सब चीज़ों को जानता
\end{hindi}}
\flushright{\begin{Arabic}
\quranayah[57][4]
\end{Arabic}}
\flushleft{\begin{hindi}
वह वही तो है जिसने सारे आसमान व ज़मीन को छह: दिन में पैदा किए फिर अर्श (के बनाने) पर आमादा हुआ जो चीज़ ज़मीन में दाखिल होती है और जो उससे निकलती है और जो चीज़ आसमान से नाज़िल होती है और जो उसकी तरफ चढ़ती है (सब) उसको मालूम है और तुम (चाहे) जहाँ कहीं रहो वह तुम्हारे साथ है और जो कुछ भी तुम करते हो ख़ुदा उसे देख रहा है
\end{hindi}}
\flushright{\begin{Arabic}
\quranayah[57][5]
\end{Arabic}}
\flushleft{\begin{hindi}
सारे आसमान व ज़मीन की बादशाही ख़ास उसी की है और ख़ुदा ही की तरफ कुल उमूर की रूजू होती है
\end{hindi}}
\flushright{\begin{Arabic}
\quranayah[57][6]
\end{Arabic}}
\flushleft{\begin{hindi}
वही रात को (घटा कर) दिन में दाखिल करता है तो दिन बढ़ जाता है और दिन को (घटाकर) रात में दाख़िल करता है (तो रात बढ़ जाती है) और दिलों के भेदों तक से ख़ूब वाक़िफ है
\end{hindi}}
\flushright{\begin{Arabic}
\quranayah[57][7]
\end{Arabic}}
\flushleft{\begin{hindi}
(लोगों) ख़ुदा और उसके रसूल पर ईमान लाओ और जिस (माल) में उसने तुमको अपना नायब बनाया है उसमें से से कुछ (ख़ुदा की राह में) ख़र्च करो तो तुम में से जो लोग ईमान लाए और (राहे ख़ुदा में) ख़र्च करते रहें उनके लिए बड़ा अज्र है
\end{hindi}}
\flushright{\begin{Arabic}
\quranayah[57][8]
\end{Arabic}}
\flushleft{\begin{hindi}
और तुम्हें क्या हो गया है कि ख़ुदा पर ईमान नहीं लाते हो हालॉकि रसूल तुम्हें बुला रहें हैं कि अपने परवरदिगार पर ईमान लाओ और अगर तुमको बावर हो तो (यक़ीन करो कि) ख़ुदा तुम से (इसका) इक़रार ले चुका
\end{hindi}}
\flushright{\begin{Arabic}
\quranayah[57][9]
\end{Arabic}}
\flushleft{\begin{hindi}
वही तो है जो अपने बन्दे (मोहम्मद) पर वाज़ेए व रौशन आयतें नाज़िल करता है ताकि तुम लोगों को (कुफ़्र की) तारिक़ीयों से निकाल कर (ईमान की) रौशनी में ले जाए और बेशक ख़ुदा तुम पर बड़ा मेहरबान और निहायत रहम वाला है
\end{hindi}}
\flushright{\begin{Arabic}
\quranayah[57][10]
\end{Arabic}}
\flushleft{\begin{hindi}
और तुमको क्या हो गया कि (अपना माल) ख़ुदा की राह में ख़र्च नहीं करते हालॉकि सारे आसमान व ज़मीन का मालिक व वारिस ख़ुदा ही है तुममें से जिस शख़्श ने फतेह (मक्का) से पहले (अपना माल) ख़र्च किया और जेहाद किया (और जिसने बाद में किया) वह बराबर नहीं उनका दर्जा उन लोगों से कहीं बढ़ कर है जिन्होंने बाद में ख़र्च किया और जेहाद किया और (यूँ तो) ख़ुदा ने नेकी और सवाब का वायदा तो सबसे किया है और जो कुछ तुम करते हो ख़ुदा उससे ख़ूब वाक़िफ़ है
\end{hindi}}
\flushright{\begin{Arabic}
\quranayah[57][11]
\end{Arabic}}
\flushleft{\begin{hindi}
कौन ऐसा है जो ख़ुदा को ख़ालिस नियत से कर्जे हसना दे तो ख़ुदा उसके लिए (अज्र को) दूना कर दे और उसके लिए बहुत मुअज्ज़िज़ सिला (जन्नत) तो है ही
\end{hindi}}
\flushright{\begin{Arabic}
\quranayah[57][12]
\end{Arabic}}
\flushleft{\begin{hindi}
जिस दिन तुम मोमिन मर्द और मोमिन औरतों को देखोगे कि उन (के ईमान) का नूर उनके आगे आगे और दाहिने तरफ चल रहा होगा तो उनसे कहा (जाएगा) तुमको बशारत हो कि आज तुम्हारे लिए वह बाग़ है जिनके नीचे नहरें जारी हैं जिनमें हमेशा रहोगे यही तो बड़ी कामयाबी है
\end{hindi}}
\flushright{\begin{Arabic}
\quranayah[57][13]
\end{Arabic}}
\flushleft{\begin{hindi}
उस दिन मुनाफ़िक मर्द और मुनाफ़िक औरतें ईमानदारों से कहेंगे एक नज़र (शफ़क्क़त) हमारी तरफ़ भी करो कि हम भी तुम्हारे नूर से कुछ रौशनी हासिल करें तो (उनसे) कहा जाएगा कि तुम अपने पीछे (दुनिया में) लौट जाओ और (वही) किसी और नूर की तलाश करो फिर उनके बीच में एक दीवार खड़ी कर दी जाएगी जिसमें एक दरवाज़ा होगा (और) उसके अन्दर की जानिब तो रहमत है और बाहर की तरफ अज़ाब तो मुनाफ़िक़ीन मोमिनीन से पुकार कर कहेंगे
\end{hindi}}
\flushright{\begin{Arabic}
\quranayah[57][14]
\end{Arabic}}
\flushleft{\begin{hindi}
(क्यों भाई) क्या हम कभी तुम्हारे साथ न थे तो मोमिनीन कहेंगे थे तो ज़रूर मगर तुम ने तो ख़ुद अपने आपको बला में डाला और (हमारे हक़ में गर्दिशों के) मुन्तज़िर हैं और (दीन में) शक़ किया किए और तुम्हें (तुम्हारी) तमन्नाओं ने धोखे में रखा यहाँ तक कि ख़ुदा का हुक्म आ पहुँचा और एक बड़े दग़ाबाज़ (शैतान) ने ख़ुदा के बारे में तुमको फ़रेब दिया
\end{hindi}}
\flushright{\begin{Arabic}
\quranayah[57][15]
\end{Arabic}}
\flushleft{\begin{hindi}
तो आज न तो तुमसे कोई मुआवज़ा लिया जाएगा और न काफ़िरों से तुम सबका ठिकाना (बस) जहन्नुम है वही तुम्हारे वास्ते सज़ावार है और (क्या) बुरी जगह है
\end{hindi}}
\flushright{\begin{Arabic}
\quranayah[57][16]
\end{Arabic}}
\flushleft{\begin{hindi}
क्या ईमानदारों के लिए अभी तक इसका वक्त नहीं आया कि ख़ुदा की याद और क़ुरान के लिए जो (ख़ुदा की तरफ से) नाज़िल हुआ है उनके दिल नरम हों और वह उन लोगों के से न हो जाएँ जिनको उन से पहले किताब (तौरेत, इन्जील) दी गयी थी तो (जब) एक ज़माना दराज़ गुज़र गया तो उनके दिल सख्त हो गए और इनमें से बहुतेरे बदकार हैं
\end{hindi}}
\flushright{\begin{Arabic}
\quranayah[57][17]
\end{Arabic}}
\flushleft{\begin{hindi}
जान रखो कि ख़ुदा ही ज़मीन को उसके मरने (उफ़तादा होने) के बाद ज़िन्दा (आबाद) करता है हमने तुमसे अपनी (क़ुदरत की) निशानियाँ खोल खोल कर बयान कर दी हैं ताकि तुम समझो
\end{hindi}}
\flushright{\begin{Arabic}
\quranayah[57][18]
\end{Arabic}}
\flushleft{\begin{hindi}
बेशक ख़ैरात देने वाले मर्द और ख़ैरात देने वाली औरतें और (जो लोग) ख़ुदा की नीयत से ख़ालिस कर्ज़ देते हैं उनको दोगुना (अज्र) दिया जाएगा और उनका बहुत मुअज़िज़ सिला (जन्नत) तो है ही
\end{hindi}}
\flushright{\begin{Arabic}
\quranayah[57][19]
\end{Arabic}}
\flushleft{\begin{hindi}
और जो लोग ख़ुदा और उसके रसूलों पर ईमान लाए यही लोग अपने परवरदिगार के नज़दीक सिद्दीक़ों और शहीदों के दरजे में होंगे उनके लिए उन्ही (सिद्दीकों और शहीदों) का अज्र और उन्हीं का नूर होगा और जिन लोगों ने कुफ्र किया और हमारी आयतों को झुठलाया वही लोग जहन्नुमी हैं
\end{hindi}}
\flushright{\begin{Arabic}
\quranayah[57][20]
\end{Arabic}}
\flushleft{\begin{hindi}
जान रखो कि दुनियावी ज़िन्दगी महज़ खेल और तमाशा और ज़ाहिरी ज़ीनत (व आसाइश) और आपस में एक दूसरे पर फ़ख्र क़रना और माल और औलाद की एक दूसरे से ज्यादा ख्वाहिश है (दुनयावी ज़िन्दगी की मिसाल तो) बारिश की सी मिसाल है जिस (की वजह) से किसानों की खेती (लहलहाती और) उनको ख़ुश कर देती थी फिर सूख जाती है तो तू उसको देखता है कि ज़र्द हो जाती है फिर चूर चूर हो जाती है और आख़िरत में (कुफ्फार के लिए) सख्त अज़ाब है और (मोमिनों के लिए) ख़ुदा की तरफ से बख़्शिस और ख़ुशनूदी और दुनयावी ज़िन्दगी तो बस फ़रेब का साज़ो सामान है
\end{hindi}}
\flushright{\begin{Arabic}
\quranayah[57][21]
\end{Arabic}}
\flushleft{\begin{hindi}
तुम अपने परवरदिगार के (सबब) बख़्शिस की और बेहिश्त की तरफ लपक के आगे बढ़ जाओ जिसका अर्ज़ आसमान और ज़मीन के अर्ज़ के बराबर है जो उन लोगों के लिए तैयार की गयी है जो ख़ुदा पर और उसके रसूलों पर ईमान लाए हैं ये ख़ुदा का फज़ल है जिसे चाहे अता करे और ख़ुदा का फज़ल (व क़रम) तो बहुत बड़ा है
\end{hindi}}
\flushright{\begin{Arabic}
\quranayah[57][22]
\end{Arabic}}
\flushleft{\begin{hindi}
जितनी मुसीबतें रूए ज़मीन पर और ख़ुद तुम लोगों पर नाज़िल होती हैं (वह सब) क़ब्ल इसके कि हम उन्हें पैदा करें किताब (लौह महफूज़) में (लिखी हुई) हैं बेशक ये ख़ुदा पर आसान है
\end{hindi}}
\flushright{\begin{Arabic}
\quranayah[57][23]
\end{Arabic}}
\flushleft{\begin{hindi}
ताकि जब कोई चीज़ तुमसे जाती रहे तो तुम उसका रंज न किया करो और जब कोई चीज़ (नेअमत) ख़ुदा तुमको दे तो उस पर न इतराया करो और ख़ुदा किसी इतराने वाले येख़ी बाज़ को दोस्त नहीं रखता
\end{hindi}}
\flushright{\begin{Arabic}
\quranayah[57][24]
\end{Arabic}}
\flushleft{\begin{hindi}
जो ख़ुद भी बुख्ल करते हैं और दूसरे लोगों को भी बुख्ल करना सिखाते हैं और जो शख़्श (इन बातों से) रूगरदानी करे तो ख़ुदा भी बेपरवा सज़ावारे हम्दोसना है
\end{hindi}}
\flushright{\begin{Arabic}
\quranayah[57][25]
\end{Arabic}}
\flushleft{\begin{hindi}
हमने यक़ीनन अपने पैग़म्बरों को वाज़े व रौशन मोजिज़े देकर भेजा और उनके साथ किताब और (इन्साफ़ की) तराज़ू नाज़िल किया ताकि लोग इन्साफ़ पर क़ायम रहे और हम ही ने लोहे को नाज़िल किया जिसके ज़रिए से सख्त लड़ाई और लोगों के बहुत से नफे (की बातें) हैं और ताकि ख़ुदा देख ले कि बेदेखे भाले ख़ुदा और उसके रसूलों की कौन मदद करता है बेशक ख़ुदा बहुत ज़बरदस्त ग़ालिब है
\end{hindi}}
\flushright{\begin{Arabic}
\quranayah[57][26]
\end{Arabic}}
\flushleft{\begin{hindi}
और बेशक हम ही ने नूह और इबराहीम को (पैग़म्बर बनाकर) भेजा और उनही दोनों की औलाद में नबूवत और किताब मुक़र्रर की तो उनमें के बाज़ हिदायत याफ्ता हैं और उन के बहुतेरे बदकार हैं
\end{hindi}}
\flushright{\begin{Arabic}
\quranayah[57][27]
\end{Arabic}}
\flushleft{\begin{hindi}
फिर उनके पीछे ही उनके क़दम ब क़दम अपने और पैग़म्बर भेजे और उनके पीछे मरियम के बेटे ईसा को भेजा और उनको इन्जील अता की और जिन लोगों ने उनकी पैरवी की उनके दिलों में यफ़क्क़त और मेहरबानी डाल दी और रहबानियत (लज्ज़ात से किनाराकशी) उन लोगों ने ख़ुद एक नयी बात निकाली थी हमने उनको उसका हुक्म नहीं दिया था मगर (उन लोगों ने) ख़ुदा की ख़ुशनूदी हासिल करने की ग़रज़ से (ख़ुद ईजाद किया) तो उसको भी जैसा बनाना चाहिए था न बना सके तो जो लोग उनमें से ईमान लाए उनको हमने उनका अज्र दिया उनमें के बहुतेरे तो बदकार ही हैं
\end{hindi}}
\flushright{\begin{Arabic}
\quranayah[57][28]
\end{Arabic}}
\flushleft{\begin{hindi}
ऐ ईमानदारों ख़ुदा से डरो और उसके रसूल (मोहम्मद) पर ईमान लाओ तो ख़ुदा तुमको अपनी रहमत के दो हिस्से अज्र अता फरमाएगा और तुमको ऐसा नूर इनायत फ़रमाएगा जिस (की रौशनी) में तुम चलोगे और तुमको बख्श भी देगा और ख़ुदा तो बड़ा बख्शने वाला मेहरबान है
\end{hindi}}
\flushright{\begin{Arabic}
\quranayah[57][29]
\end{Arabic}}
\flushleft{\begin{hindi}
(ये इसलिए कहा जाता है) ताकि अहले किताब ये न समझें कि ये मोमिनीन ख़ुदा के फज़ल (व क़रम) पर कुछ भी कुदरत नहीं रखते और ये तो यक़ीनी बात है कि फज़ल ख़ुदा ही के कब्ज़े में है वह जिसको चाहे अता फरमाए और ख़ुदा तो बड़े फज़ल (व क़रम) का मालिक है
\end{hindi}}
\chapter{Al-Mujadilah (The Pleading Woman)}
\begin{Arabic}
\Huge{\centerline{\basmalah}}\end{Arabic}
\flushright{\begin{Arabic}
\quranayah[58][1]
\end{Arabic}}
\flushleft{\begin{hindi}
ऐ रसूल जो औरत (ख़ुला) तुमसे अपने शौहर (उस) के बारे में तुमसे झगड़ती और ख़ुदा से गिले शिकवे करती है ख़ुदा ने उसकी बात सुन ली और ख़ुदा तुम दोनों की ग़ुफ्तगू सुन रहा है बेशक ख़ुदा बड़ा सुनने वाला देखने वाला है
\end{hindi}}
\flushright{\begin{Arabic}
\quranayah[58][2]
\end{Arabic}}
\flushleft{\begin{hindi}
तुम में से जो लोग अपनी बीवियों के साथ ज़हार करते हैं अपनी बीवी को माँ कहते हैं वह कुछ उनकी माएँ नहीं (हो जातीं) उनकी माएँ तो बस वही हैं जो उनको जनती हैं और वह बेशक एक नामाक़ूल और झूठी बात कहते हैं और ख़ुदा बेशक माफ करने वाला और बड़ा बख्शने वाला है
\end{hindi}}
\flushright{\begin{Arabic}
\quranayah[58][3]
\end{Arabic}}
\flushleft{\begin{hindi}
और जो लोग अपनी बीवियों से ज़हार कर बैठे फिर अपनी बात वापस लें तो दोनों के हमबिस्तर होने से पहले (कफ्फ़ारे में) एक ग़ुलाम का आज़ाद करना (ज़रूरी) है उसकी तुमको नसीहत की जाती है और तुम जो कुछ भी करते हो (ख़ुदा) उससे आगाह है
\end{hindi}}
\flushright{\begin{Arabic}
\quranayah[58][4]
\end{Arabic}}
\flushleft{\begin{hindi}
फिर जिसको ग़ुलाम न मिले तो दोनों की मुक़ारबत के क़ब्ल दो महीने के पै दर पै रोज़े रखें और जिसको इसकी भी क़ुदरत न हो साठ मोहताजों को खाना खिलाना फर्ज़ है ये (हुक्म इसलिए है) ताकि तुम ख़ुदा और उसके रसूल की (पैरवी) तसदीक़ करो और ये ख़ुदा की मुक़र्रर हदें हैं और काफ़िरों के लिए दर्दनाक अज़ाब है
\end{hindi}}
\flushright{\begin{Arabic}
\quranayah[58][5]
\end{Arabic}}
\flushleft{\begin{hindi}
बेशक जो लोग ख़ुदा की और उसके रसूल की मुख़ालेफ़त करते हैं वह (उसी तरह) ज़लील किए जाएँगे जिस तरह उनके पहले लोग किए जा चुके हैं और हम तो अपनी साफ़ और सरीही आयतें नाज़िल कर चुके और काफिरों के लिए ज़लील करने वाला अज़ाब है
\end{hindi}}
\flushright{\begin{Arabic}
\quranayah[58][6]
\end{Arabic}}
\flushleft{\begin{hindi}
जिस दिन ख़ुदा उन सबको दोबारा उठाएगा तो उनके आमाल से उनको आगाह कर देगा ये लोग (अगरचे) उनको भूल गये हैं मगर ख़ुदा ने उनको याद रखा है और ख़ुदा तो हर चीज़ का गवाह है
\end{hindi}}
\flushright{\begin{Arabic}
\quranayah[58][7]
\end{Arabic}}
\flushleft{\begin{hindi}
क्या तुमको मालूम नहीं कि जो कुछ आसमानों में है और जो कुछ ज़मीन में है (ग़रज़ सब कुछ) ख़ुदा जानता है जब तीन (आदमियों) का ख़ुफिया मशवेरा होता है तो वह (ख़ुद) उनका ज़रूर चौथा है और जब पाँच का मशवेरा होता है तो वह उनका छठा है और उससे कम हो या ज्यादा और चाहे जहाँ कहीं हो वह उनके साथ ज़रूर होता है फिर जो कुछ वह (दुनिया में) करते रहे क़यामत के दिन उनको उससे आगाह कर देगा बेशक ख़ुदा हर चीज़ से ख़ूब वाक़िफ़ है
\end{hindi}}
\flushright{\begin{Arabic}
\quranayah[58][8]
\end{Arabic}}
\flushleft{\begin{hindi}
क्या तुमने उन लोगों को नहीं देखा जिनको सरगोशियाँ करने से मना किया गया ग़रज़ जिस काम की उनको मुमानिअत की गयी थी उसी को फिर करते हैं और (लुत्फ़ तो ये है कि) गुनाह और (बेजा) ज्यादती और रसूल की नाफरमानी की सरगोशियाँ करते हैं और जब तुम्हारे पास आते हैं तो जिन लफ्ज़ों से ख़ुदा ने भी तुम को सलाम नहीं किया उन लफ्ज़ों से सलाम करते हैं और अपने जी में कहते हैं कि (अगर ये वाक़ई पैग़म्बर हैं तो) जो कुछ हम कहते हैं ख़ुदा हमें उसकी सज़ा क्यों नहीं देता (ऐ रसूल) उनको दोज़ख़ ही (की सज़ा) काफी है जिसमें ये दाख़िल होंगे तो वह (क्या) बुरी जगह है
\end{hindi}}
\flushright{\begin{Arabic}
\quranayah[58][9]
\end{Arabic}}
\flushleft{\begin{hindi}
ऐ ईमानदारों जब तुम आपस में सरगोशी करो तो गुनाह और ज्यादती और रसूल की नाफरमानी की सरगोशी न करो बल्कि नेकीकारी और परहेज़गारी की सरगोशी करो और ख़ुदा से डरते रहो जिसके सामने (एक दिन) जमा किए जाओगे
\end{hindi}}
\flushright{\begin{Arabic}
\quranayah[58][10]
\end{Arabic}}
\flushleft{\begin{hindi}
(बरी बातों की) सरगोशी तो बस एक शैतानी काम है (और इसलिए करते हैं) ताकि ईमानदारों को उससे रंज पहुँचे हालॉकि ख़ुदा की तरफ से आज़ादी दिए बग़ैर सरगोशी उनका कुछ बिगाड़ नहीं सकती और मोमिनीन को तो ख़ुदा ही पर भरोसा रखना चाहिए
\end{hindi}}
\flushright{\begin{Arabic}
\quranayah[58][11]
\end{Arabic}}
\flushleft{\begin{hindi}
ऐ ईमानदारों जब तुमसे कहा जाए कि मजलिस में जगह कुशादा करो वह तो कुशादा कर दिया करो ख़ुदा तुमको कुशादगी अता करेगा और जब तुमसे कहा जाए कि उठ खड़े हो तो उठ खड़े हुआ करो जो लोग तुमसे ईमानदार हैं और जिनको इल्म अता हुआ है ख़ुदा उनके दर्जे बुलन्द करेगा और ख़ुदा तुम्हारे सब कामों से बेख़बर है
\end{hindi}}
\flushright{\begin{Arabic}
\quranayah[58][12]
\end{Arabic}}
\flushleft{\begin{hindi}
ऐ ईमानदारों जब पैग़म्बर से कोई बात कान में कहनी चाहो तो कुछ ख़ैरात अपनी सरगोशी से पहले दे दिया करो यही तुम्हारे वास्ते बेहतर और पाकीज़ा बात है पस अगर तुमको इसका मुक़दूर न हो तो बेशक ख़ुदा बड़ा बख्शने वाला मेहरबान है
\end{hindi}}
\flushright{\begin{Arabic}
\quranayah[58][13]
\end{Arabic}}
\flushleft{\begin{hindi}
(मुसलमानों) क्या तुम इस बात से डर गए कि (रसूल के) कान में बात कहने से पहले ख़ैरात कर लो तो जब तुम लोग (इतना सा काम) न कर सके और ख़ुदा ने तुम्हें माफ़ कर दिया तो पाबन्दी से नमाज़ पढ़ो और ज़कात देते रहो और ख़ुदा उसके रसूल की इताअत करो और जो कुछ तुम करते हो ख़ुदा उससे बाख़बर है
\end{hindi}}
\flushright{\begin{Arabic}
\quranayah[58][14]
\end{Arabic}}
\flushleft{\begin{hindi}
क्या तुमने उन लोगों की हालत पर ग़ौर नहीं किया जो उन लोगों से दोस्ती करते हैं जिन पर ख़ुदा ने ग़ज़ब ढाया है तो अब वह न तुम मे हैं और न उनमें ये लोग जानबूझ कर झूठी बातो पर क़समें खाते हैं और वह जानते हैं
\end{hindi}}
\flushright{\begin{Arabic}
\quranayah[58][15]
\end{Arabic}}
\flushleft{\begin{hindi}
ख़ुदा ने उनके लिए सख्त अज़ाब तैयार कर रखा है इसमें शक़ नहीं कि ये लोग जो कुछ करते हैं बहुत ही बुरा है
\end{hindi}}
\flushright{\begin{Arabic}
\quranayah[58][16]
\end{Arabic}}
\flushleft{\begin{hindi}
उन लोगों ने अपनी क़समों को सिपर बना लिया है और (लोगों को) ख़ुदा की राह से रोक दिया तो उनके लिए रूसवा करने वाला अज़ाब है
\end{hindi}}
\flushright{\begin{Arabic}
\quranayah[58][17]
\end{Arabic}}
\flushleft{\begin{hindi}
ख़ुदा सामने हरगिज़ न उनके माल ही कुछ काम आएँगे और न उनकी औलाद ही काम आएगी यही लोग जहन्नुमी हैं कि हमेशा उसमें रहेंगे
\end{hindi}}
\flushright{\begin{Arabic}
\quranayah[58][18]
\end{Arabic}}
\flushleft{\begin{hindi}
जिस दिन ख़ुदा उन सबको दोबार उठा खड़ा करेगा तो ये लोग जिस तरह तुम्हारे सामने क़समें खाते हैं उसी तरह उसके सामने भी क़समें खाएँगे और ख्याल करते हैं कि वह राहे सवाब पर हैं आगाह रहो ये लोग यक़ीनन झूठे हैं
\end{hindi}}
\flushright{\begin{Arabic}
\quranayah[58][19]
\end{Arabic}}
\flushleft{\begin{hindi}
शैतान ने इन पर क़ाबू पा लिया है और ख़ुदा की याद उनसे भुला दी है ये लोग शैतान के गिरोह है सुन रखो कि शैतान का गिरोह घाटा उठाने वाला है
\end{hindi}}
\flushright{\begin{Arabic}
\quranayah[58][20]
\end{Arabic}}
\flushleft{\begin{hindi}
जो लोग ख़ुदा और उसके रसूल से मुख़ालेफ़त करते हैं वह सब ज़लील लोगों में हैं
\end{hindi}}
\flushright{\begin{Arabic}
\quranayah[58][21]
\end{Arabic}}
\flushleft{\begin{hindi}
ख़ुदा ने हुक्म नातिक दे दिया है कि मैं और मेरे पैग़म्बर ज़रूर ग़ालिब रहेंगे बेशक ख़ुदा बड़ा ज़बरदस्त ग़ालिब है
\end{hindi}}
\flushright{\begin{Arabic}
\quranayah[58][22]
\end{Arabic}}
\flushleft{\begin{hindi}
जो लोग ख़ुदा और रोज़े आख़ेरत पर ईमान रखते हैं तुम उनको ख़ुदा और उसके रसूल के दुश्मनों से दोस्ती करते हुए न देखोगे अगरचे वह उनके बाप या बेटे या भाई या ख़ानदान ही के लोग (क्यों न हों) यही वह लोग हैं जिनके दिलों में ख़ुदा ने ईमान को साबित कर दिया है और ख़ास अपने नूर से उनकी ताईद की है और उनको (बेहिश्त में) उन (हरे भरे) बाग़ों में दाखिल करेगा जिनके नीचे नहरे जारी है (और वह) हमेश उसमें रहेंगे ख़ुदा उनसे राज़ी और वह ख़ुदा से ख़ुश यही ख़ुदा का गिरोह है सुन रखो कि ख़ुदा के गिरोग के लोग दिली मुरादें पाएँगे
\end{hindi}}
\chapter{Al-Hashr (The Banishment)}
\begin{Arabic}
\Huge{\centerline{\basmalah}}\end{Arabic}
\flushright{\begin{Arabic}
\quranayah[59][1]
\end{Arabic}}
\flushleft{\begin{hindi}
जो चीज़ आसमानों में है और जो चीज़ ज़मीन में है (सब) ख़ुदा की तस्बीह करती हैं और वही ग़ालिब हिकमत वाला है
\end{hindi}}
\flushright{\begin{Arabic}
\quranayah[59][2]
\end{Arabic}}
\flushleft{\begin{hindi}
वही तो है जिसने कुफ्फ़ार अहले किताब (बनी नुजैर) को पहले हश्र (ज़िलाए वतन) में उनके घरों से निकाल बाहर किया (मुसलमानों) तुमको तो ये वहम भी न था कि वह निकल जाएँगे और वह लोग ये समझे हुये थे कि उनके क़िले उनको ख़ुदा (के अज़ाब) से बचा लेंगे मगर जहाँ से उनको ख्याल भी न था ख़ुदा ने उनको आ लिया और उनके दिलों में (मुसलमानों) को रौब डाल दिया कि वह लोग ख़ुद अपने हाथों से और मोमिनीन के हाथों से अपने घरों को उजाड़ने लगे तो ऐ ऑंख वालों इबरत हासिल करो
\end{hindi}}
\flushright{\begin{Arabic}
\quranayah[59][3]
\end{Arabic}}
\flushleft{\begin{hindi}
और ख़ुदा ने उनकी किसमत में ज़िला वतनी न लिखा होता तो उन पर दुनिया में भी (दूसरी तरह) अज़ाब करता और आख़ेरत में तो उन पर जहन्नुम का अज़ाब है ही
\end{hindi}}
\flushright{\begin{Arabic}
\quranayah[59][4]
\end{Arabic}}
\flushleft{\begin{hindi}
ये इसलिए कि उन लोगों ने ख़ुदा और उसके रसूल की मुख़ालेफ़त की और जिसने ख़ुदा की मुख़ालेफ़त की तो (याद रहे कि) ख़ुदा बड़ा सख्त अज़ाब देने वाला है
\end{hindi}}
\flushright{\begin{Arabic}
\quranayah[59][5]
\end{Arabic}}
\flushleft{\begin{hindi}
(मोमिनों) खजूर का दरख्त जो तुमने काट डाला या जूँ का तँ से उनकी जड़ों पर खड़ा रहने दिया तो ख़ुदा ही के हुक्म से और मतलब ये था कि वह नाफरमानों को रूसवा करे
\end{hindi}}
\flushright{\begin{Arabic}
\quranayah[59][6]
\end{Arabic}}
\flushleft{\begin{hindi}
(तो) जो माल ख़ुदा ने अपने रसूल को उन लोगों से बे लड़े दिलवा दिया उसमें तुम्हार हक़ नहीं क्योंकि तुमने उसके लिए कुछ दौड़ धूप तो की ही नहीं, न घोड़ों से न ऊँटों से, मगर ख़ुदा अपने पैग़म्बरों को जिस पर चाहता है ग़लबा अता फरमाता है और ख़ुदा हर चीज़ पर क़ादिर है
\end{hindi}}
\flushright{\begin{Arabic}
\quranayah[59][7]
\end{Arabic}}
\flushleft{\begin{hindi}
तो जो माल ख़ुदा ने अपने रसूल को देहात वालों से बे लड़े दिलवाया है वह ख़ास ख़ुदा और उसके रसूल और (रसूल के) क़राबतदारों और यतीमों और मोहताजों और परदेसियों का है ताकि जो लोग तुममें से दौलतमन्द हैं हिर फिर कर दौलत उन्हीं में न रहे, हाँ जो तुमको रसूल दें दें वह ले लिया करो और जिससे मना करें उससे बाज़ रहो और ख़ुदा से डरते रहो बेशक ख़ुदा सख्त अज़ाब देने वाला है
\end{hindi}}
\flushright{\begin{Arabic}
\quranayah[59][8]
\end{Arabic}}
\flushleft{\begin{hindi}
(इस माल में) उन मुफलिस मुहाजिरों का हिस्सा भी है जो अपने घरों से और मालों से निकाले (और अलग किए) गए (और) ख़ुदा के फ़ज़ल व ख़ुशनूदी के तलबगार हैं और ख़ुदा की और उसके रसूल की मदद करते हैं यही लोग सच्चे ईमानदार हैं और (उनका भी हिस्सा है)
\end{hindi}}
\flushright{\begin{Arabic}
\quranayah[59][9]
\end{Arabic}}
\flushleft{\begin{hindi}
जो लोग मोहाजेरीन से पहले (हिजरत के) घर (मदीना) में मुक़ीम हैं और ईमान में (मुसतक़िल) रहे और जो लोग हिजरत करके उनके पास आए उनसे मोहब्बत करते हैं और जो कुछ उनको मिला उसके लिए अपने दिलों में कुछ ग़रज़ नहीं पाते और अगरचे अपने ऊपर तंगी ही क्यों न हो दूसरों को अपने नफ्स पर तरजीह देते हैं और जो शख़्श अपने नफ्स की हिर्स से बचा लिया गया तो ऐसे ही लोग अपनी दिली मुरादें पाएँगे
\end{hindi}}
\flushright{\begin{Arabic}
\quranayah[59][10]
\end{Arabic}}
\flushleft{\begin{hindi}
और उनका भी हिस्सा है और जो लोग उन (मोहाजेरीन) के बाद आए (और) दुआ करते हैं कि परवरदिगारा हमारी और उन लोगों की जो हमसे पहले ईमान ला चुके मग़फेरत कर और मोमिनों की तरफ से हमारे दिलों में किसी तरह का कीना न आने दे परवरदिगार बेशक तू बड़ा शफीक़ निहायत रहम वाला है
\end{hindi}}
\flushright{\begin{Arabic}
\quranayah[59][11]
\end{Arabic}}
\flushleft{\begin{hindi}
क्या तुमने उन मुनाफ़िकों की हालत पर नज़र नहीं की जो अपने काफ़िर भाइयों अहले किताब से कहा करते हैं कि अगर कहीं तुम (घरों से) निकाले गए तो यक़ीन जानों कि हम भी तुम्हारे साथ (ज़रूर) निकल खड़े होंगे और तुम्हारे बारे में कभी किसी की इताअत न करेंगे और अगर तुमसे लड़ाई होगी तो ज़रूर तुम्हारी मदद करेंगे, मगर ख़ुदा बयान किए देता है कि ये लोग यक़ीनन झूठे हैं
\end{hindi}}
\flushright{\begin{Arabic}
\quranayah[59][12]
\end{Arabic}}
\flushleft{\begin{hindi}
अगर कुफ्फ़ार निकाले भी जाएँ तो ये मुनाफेक़ीन उनके साथ न निकलेंगे और अगर उनसे लड़ाई हुई तो उनकी मदद भी न करेंगे और यक़ीनन करेंगे भी तो पीठ फेर कर भाग जाएँगे
\end{hindi}}
\flushright{\begin{Arabic}
\quranayah[59][13]
\end{Arabic}}
\flushleft{\begin{hindi}
फिर उनको कहीं से कुमक भी न मिलेगी (मोमिनों) तुम्हारी हैबत उनके दिलों में ख़ुदा से भी बढ़कर है, ये इस वजह से कि ये लोग समझ नहीं रखते
\end{hindi}}
\flushright{\begin{Arabic}
\quranayah[59][14]
\end{Arabic}}
\flushleft{\begin{hindi}
ये सब के सब मिलकर भी तुमसे नहीं लड़ सकते, मगर हर तरफ से महफूज़ बस्तियों में या (शहर पनाह की) दीवारों की आड़ में इनकी आपस में तो बड़ी धाक है कि तुम ख्याल करोगे कि सब के सब (एक जान) हैं मगर उनके दिल एक दूसरे से फटे हुए हैं ये इस वजह से कि ये लोग बेअक्ल हैं
\end{hindi}}
\flushright{\begin{Arabic}
\quranayah[59][15]
\end{Arabic}}
\flushleft{\begin{hindi}
उनका हाल उन लोगों का सा है जो उनसे कुछ ही पेशतर अपने कामों की सज़ा का मज़ा चख चुके हैं और उनके लिए दर्दनाक अज़ाब है
\end{hindi}}
\flushright{\begin{Arabic}
\quranayah[59][16]
\end{Arabic}}
\flushleft{\begin{hindi}
(मुनाफ़िकों) की मिसाल शैतान की सी है कि इन्सान से कहता रहा कि काफ़िर हो जाओ, फिर जब वह काफ़िर हो गया तो कहने लगा मैं तुमसे बेज़ार हूँ मैं सारे जहाँ के परवरदिगार से डरता हूँ
\end{hindi}}
\flushright{\begin{Arabic}
\quranayah[59][17]
\end{Arabic}}
\flushleft{\begin{hindi}
तो दोनों का नतीजा ये हुआ कि दोनों दोज़ख़ में (डाले) जाएँगे और उसमें हमेशा रहेंगे और यही तमाम ज़ालिमों की सज़ा है
\end{hindi}}
\flushright{\begin{Arabic}
\quranayah[59][18]
\end{Arabic}}
\flushleft{\begin{hindi}
ऐ ईमानदारों ख़ुदा से डरो, और हर शख़्श को ग़ौर करना चाहिए कि कल क़यामत के वास्ते उसने पहले से क्या भेजा है और ख़ुदा ही से डरते रहो बेशक जो कुछ तुम करते हो ख़ुदा उससे बाख़बर है
\end{hindi}}
\flushright{\begin{Arabic}
\quranayah[59][19]
\end{Arabic}}
\flushleft{\begin{hindi}
और उन लोगों के जैसे न हो जाओ जो ख़ुदा को भुला बैठे तो ख़ुदा ने उन्हें ऐसा कर दिया कि वह अपने आपको भूल गए यही लोग तो बद किरदार हैं
\end{hindi}}
\flushright{\begin{Arabic}
\quranayah[59][20]
\end{Arabic}}
\flushleft{\begin{hindi}
जहन्नुमी और जन्नती किसी तरह बराबर नहीं हो सकते जन्नती लोग ही तो कामयाबी हासिल करने वाले हैं
\end{hindi}}
\flushright{\begin{Arabic}
\quranayah[59][21]
\end{Arabic}}
\flushleft{\begin{hindi}
अगर हम इस क़ुरान को किसी पहाड़ पर (भी) नाज़िल करते तो तुम उसको देखते कि ख़ुदा के डर से झुका और फटा जाता है ये मिसालें हम लोगों (के समझाने) के लिए बयान करते हैं ताकि वह ग़ौर करें
\end{hindi}}
\flushright{\begin{Arabic}
\quranayah[59][22]
\end{Arabic}}
\flushleft{\begin{hindi}
वही ख़ुदा है जिसके सिवा कोई माबूद नहीं, पोशीदा और ज़ाहिर का जानने वाला वही बड़ा मेहरबान निहायत रहम वाला है
\end{hindi}}
\flushright{\begin{Arabic}
\quranayah[59][23]
\end{Arabic}}
\flushleft{\begin{hindi}
वही वह ख़ुदा है जिसके सिवा कोई क़ाबिले इबादत नहीं (हक़ीक़ी) बादशाह, पाक ज़ात (हर ऐब से) बरी अमन देने वाला निगेहबान, ग़ालिब ज़बरदस्त बड़ाई वाला ये लोग जिसको (उसका) शरीक ठहराते हैं उससे पाक है
\end{hindi}}
\flushright{\begin{Arabic}
\quranayah[59][24]
\end{Arabic}}
\flushleft{\begin{hindi}
वही ख़ुदा (तमाम चीज़ों का ख़ालिक) मुजिद सूरतों का बनाने वाला उसी के अच्छे अच्छे नाम हैं जो चीज़े सारे आसमान व ज़मीन में हैं सब उसी की तसबीह करती हैं, और वही ग़ालिब हिकमत वाला है
\end{hindi}}
\chapter{Al-Mumtahanah (The Woman who is Examined)}
\begin{Arabic}
\Huge{\centerline{\basmalah}}\end{Arabic}
\flushright{\begin{Arabic}
\quranayah[60][1]
\end{Arabic}}
\flushleft{\begin{hindi}
ऐ ईमानदारों अगर तुम मेरी राह में जेहाद करने और मेरी ख़ुशनूदी की तमन्ना में (घर से) निकलते हो तो मेरे और अपने दुशमनों को दोस्त न बनाओ तुम उनके पास दोस्ती का पैग़ाम भेजते हो और जो दीन हक़ तुम्हारे पास आया है उससे वह लोग इनकार करते हैं वह लोग रसूल को और तुमको इस बात पर (घर से) निकालते हैं कि तुम अपने परवरदिगार ख़ुदा पर ईमान ले आए हो (और) तुम हो कि उनके पास छुप छुप के दोस्ती का पैग़ाम भेजते हो हालॉकि तुम कुछ भी छुपा कर या बिल एलान करते हो मैं उससे ख़ूब वाक़िफ़ हूँ और तुममें से जो शख़्श ऐसा करे तो वह सीधी राह से यक़ीनन भटक गया
\end{hindi}}
\flushright{\begin{Arabic}
\quranayah[60][2]
\end{Arabic}}
\flushleft{\begin{hindi}
अगर ये लोग तुम पर क़ाबू पा जाएँ तो तुम्हारे दुश्मन हो जाएँ और ईज़ा के लिए तुम्हारी तरफ अपने हाथ भी बढ़ाएँगे और अपनी ज़बाने भी और चाहते हैं कि काश तुम भी काफिर हो जाओ
\end{hindi}}
\flushright{\begin{Arabic}
\quranayah[60][3]
\end{Arabic}}
\flushleft{\begin{hindi}
क़यामत के दिन न तुम्हारे रिश्ते नाते ही कुछ काम आएँगे न तुम्हारी औलाद (उस दिन) तो वही फ़ैसला कर देगा और जो कुछ भी तुम करते हो ख़ुदा उसे देख रहा है
\end{hindi}}
\flushright{\begin{Arabic}
\quranayah[60][4]
\end{Arabic}}
\flushleft{\begin{hindi}
(मुसलमानों) तुम्हारे वास्ते तो इबराहीम और उनके साथियों (के क़ौल व फेल का अच्छा नमूना मौजूद है) कि जब उन्होने अपनी क़ौम से कहा कि हम तुमसे और उन (बुतों) से जिन्हें तुम ख़ुदा के सिवा पूजते हो बेज़ार हैं हम तो तुम्हारे (दीन के) मुनकिर हैं और जब तक तुम यकता ख़ुदा पर ईमान न लाओ हमारे तुम्हारे दरमियान खुल्लम खुल्ला अदावत व दुशमनी क़ायम हो गयी मगर (हाँ) इबराहीम ने अपने (मुँह बोले) बाप से ये (अलबत्ता) कहा कि मैं आपके लिए मग़फ़िरत की दुआ ज़रूर करूँगा और ख़ुदा के सामने तो मैं आपके वास्ते कुछ एख्तेयार नहीं रखता ऐ हमारे पालने वाले (ख़ुदा) हमने तुझी पर भरोसा कर लिया है और तेरी ही तरफ हम रूजू करते हैं
\end{hindi}}
\flushright{\begin{Arabic}
\quranayah[60][5]
\end{Arabic}}
\flushleft{\begin{hindi}
और तेरी तरफ हमें लौट कर जाना है ऐ हमारे पालने वाले तू हम लोगों को काफ़िरों की आज़माइश (का ज़रिया) न क़रार दे और परवरदिगार तू हमें बख्श दे बेशक तू ग़ालिब (और) हिकमत वाला है
\end{hindi}}
\flushright{\begin{Arabic}
\quranayah[60][6]
\end{Arabic}}
\flushleft{\begin{hindi}
(मुसलमानों) उन लोगों के (अफ़आल) का तुम्हारे वास्ते जो ख़ुदा और रोज़े आख़ेरत की उम्मीद रखता हो अच्छा नमूना है और जो (इससे) मुँह मोड़े तो ख़ुदा भी यक़ीनन बेपरवा (और) सज़ावारे हम्द है
\end{hindi}}
\flushright{\begin{Arabic}
\quranayah[60][7]
\end{Arabic}}
\flushleft{\begin{hindi}
करीब है कि ख़ुदा तुम्हारे और उनमें से तुम्हारे दुश्मनों के दरमियान दोस्ती पैदा कर दे और ख़ुदा तो क़ादिर है और ख़ुदा बड़ा बख्शने वाला मेहरबान है
\end{hindi}}
\flushright{\begin{Arabic}
\quranayah[60][8]
\end{Arabic}}
\flushleft{\begin{hindi}
जो लोग तुमसे तुम्हारे दीन के बारे में नहीं लड़े भिड़े और न तुम्हें घरों से निकाले उन लोगों के साथ एहसान करने और उनके साथ इन्साफ़ से पेश आने से ख़ुदा तुम्हें मना नहीं करता बेशक ख़ुदा इन्साफ़ करने वालों को दोस्त रखता है
\end{hindi}}
\flushright{\begin{Arabic}
\quranayah[60][9]
\end{Arabic}}
\flushleft{\begin{hindi}
ख़ुदा तो बस उन लोगों के साथ दोस्ती करने से मना करता है जिन्होने तुमसे दीन के बारे में लड़ाई की और तुमको तुम्हारे घरों से निकाल बाहर किया, और तुम्हारे निकालने में (औरों की) मदद की और जो लोग ऐसों से दोस्ती करेंगे वह लोग ज़ालिम हैं
\end{hindi}}
\flushright{\begin{Arabic}
\quranayah[60][10]
\end{Arabic}}
\flushleft{\begin{hindi}
ऐ ईमानदारों जब तुम्हारे पास ईमानदार औरतें वतन छोड़ कर आएँ तो तुम उनको आज़मा लो, ख़ुदा तो उनके ईमान से वाकिफ़ है ही, पस अगर तुम भी उनको ईमानदार समझो तो उन्ही काफ़िरों के पास वापस न फेरो न ये औरतें उनके लिए हलाल हैं और न वह कुफ्फ़ार उन औरतों के लिए हलाल हैं और उन कुफ्फार ने जो कुछ (उन औरतों के मेहर में) ख़र्च किया हो उनको दे दो, और जब उनका महर उन्हें दे दिया करो तो इसका तुम पर कुछ गुनाह नहीं कि तुम उससे निकाह कर लो और काफिर औरतों की आबरू (जो तुम्हारी बीवियाँ हों) अपने कब्ज़े में न रखो (छोड़ दो कि कुफ्फ़ार से जा मिलें) और तुमने जो कुछ (उन पर) ख़र्च किया हो (कुफ्फ़ार से) लो, और उन्होने भी जो कुछ ख़र्च किया हो तुम से माँग लें यही ख़ुदा का हुक्म है जो तुम्हारे दरमियान सादिर करता है और ख़ुदा वाक़िफ़कार हकीम है
\end{hindi}}
\flushright{\begin{Arabic}
\quranayah[60][11]
\end{Arabic}}
\flushleft{\begin{hindi}
और अगर तुम्हारी बीवियों में से कोई औरत तुम्हारे हाथ से निकल कर काफिरों के पास चली जाए और (ख़र्च न मिले) और तुम (उन काफ़िरों से लड़ो और लूटो तो (माले ग़नीमत से) जिनकी औरतें चली गयीं हैं उनको इतना दे दो जितना उनका ख़र्च हुआ है) और जिस ख़ुदा पर तुम लोग ईमान लाए हो उससे डरते रहो
\end{hindi}}
\flushright{\begin{Arabic}
\quranayah[60][12]
\end{Arabic}}
\flushleft{\begin{hindi}
(ऐ रसूल) जब तुम्हारे पास ईमानदार औरतें तुमसे इस बात पर बैयत करने आएँ कि वह न किसी को ख़ुदा का शरीक बनाएँगी और न चोरी करेंगी और न जेना करेंगी और न अपनी औलाद को मार डालेंगी और न अपने हाथ पाँव के सामने कोई बोहतान (लड़के का शौहर पर) गढ़ के लाएँगी, और न किसी नेक काम में तुम्हारी नाफ़रमानी करेंगी तो तुम उनसे बैयत ले लो और ख़ुदा से उनके मग़फ़िरत की दुआ माँगो बेशक बड़ा ख़ुदा बख्शने वाला मेहरबान है
\end{hindi}}
\flushright{\begin{Arabic}
\quranayah[60][13]
\end{Arabic}}
\flushleft{\begin{hindi}
ऐ ईमानदारों जिन लोगों पर ख़ुदा ने अपना ग़ज़ब ढाया उनसे दोस्ती न करो (क्योंकि) जिस तरह काफ़िरों को मुर्दों (के दोबारा ज़िन्दा होने) की उम्मीद नहीं उसी तरह आख़ेरत से भी ये लोग न उम्मीद हैं
\end{hindi}}
\chapter{As-Saff (The Ranks)}
\begin{Arabic}
\Huge{\centerline{\basmalah}}\end{Arabic}
\flushright{\begin{Arabic}
\quranayah[61][1]
\end{Arabic}}
\flushleft{\begin{hindi}
जो चीज़े आसमानों में है और जो चीज़े ज़मीन में हैं (सब) ख़ुदा की तस्बीह करती हैं और वह ग़ालिब हिकमत वाला है
\end{hindi}}
\flushright{\begin{Arabic}
\quranayah[61][2]
\end{Arabic}}
\flushleft{\begin{hindi}
ऐ ईमानदारों तुम ऐसी बातें क्यों कहा करते हो जो किया नहीं करते
\end{hindi}}
\flushright{\begin{Arabic}
\quranayah[61][3]
\end{Arabic}}
\flushleft{\begin{hindi}
ख़ुदा के नज़दीक ये ग़ज़ब की बात है कि तुम ऐसी बात कहो जो करो नहीं
\end{hindi}}
\flushright{\begin{Arabic}
\quranayah[61][4]
\end{Arabic}}
\flushleft{\begin{hindi}
ख़ुदा तो उन लोगों से उलफ़त रखता है जो उसकी राह में इस तरह परा बाँध के लड़ते हैं कि गोया वह सीसा पिलाई हुई दीवारें हैं
\end{hindi}}
\flushright{\begin{Arabic}
\quranayah[61][5]
\end{Arabic}}
\flushleft{\begin{hindi}
और जब मूसा ने अपनी क़ौम से कहा कि भाइयों तुम मुझे क्यों अज़ीयत देते हो हालॉकि तुम तो जानते हो कि मैं तुम्हारे पास ख़ुदा का (भेजा हुआ) रसूल हूँ तो जब वह टेढ़े हुए तो ख़ुदा ने भी उनके दिलों को टेढ़ा ही रहने दिया और ख़ुदा बदकार लोगों को मंज़िले मक़सूद तक नहीं पहुँचाया करता
\end{hindi}}
\flushright{\begin{Arabic}
\quranayah[61][6]
\end{Arabic}}
\flushleft{\begin{hindi}
और (याद करो) जब मरियम के बेटे ईसा ने कहा ऐ बनी इसराइल मैं तुम्हारे पास ख़ुदा का भेजा हुआ (आया) हूँ (और जो किताब तौरेत मेरे सामने मौजूद है उसकी तसदीक़ करता हूँ और एक पैग़म्बर जिनका नाम अहमद होगा (और) मेरे बाद आएँगे उनकी ख़ुशख़बरी सुनाता हूँ जो जब वह (पैग़म्बर अहमद) उनके पास वाज़ेए व रौशन मौजिज़े लेकर आया तो कहने लगे ये तो खुला हुआ जादू है
\end{hindi}}
\flushright{\begin{Arabic}
\quranayah[61][7]
\end{Arabic}}
\flushleft{\begin{hindi}
और जो शख़्श इस्लाम की तरफ बुलाया जाए (और) वह कुबूल के बदले उलटा ख़ुदा पर झूठ (तूफान) जोड़े उससे बढ़ कर ज़ालिम और कौन होगा और ख़ुदा ज़ालिम लोगों को मंज़िले मक़सूद तक नहीं पहुँचाया करता
\end{hindi}}
\flushright{\begin{Arabic}
\quranayah[61][8]
\end{Arabic}}
\flushleft{\begin{hindi}
ये लोग अपने मुँह से (फूँक मारकर) ख़ुदा के नूर को बुझाना चाहते हैं हालॉकि ख़ुदा अपने नूर को पूरा करके रहेगा अगरचे कुफ्फ़ार बुरा ही (क्यों न) मानें
\end{hindi}}
\flushright{\begin{Arabic}
\quranayah[61][9]
\end{Arabic}}
\flushleft{\begin{hindi}
वह वही है जिसने अपने रसूल को हिदायत और सच्चे दीन के साथ भेजा ताकि उसे और तमाम दीनों पर ग़ालिब करे अगरचे मुशरेकीन बुरा ही (क्यों न) माने
\end{hindi}}
\flushright{\begin{Arabic}
\quranayah[61][10]
\end{Arabic}}
\flushleft{\begin{hindi}
ऐ ईमानदारों क्या मैं नहीं ऐसी तिजारत बता दूँ जो तुमको (आख़ेरत के) दर्दनाक अज़ाब से निजात दे
\end{hindi}}
\flushright{\begin{Arabic}
\quranayah[61][11]
\end{Arabic}}
\flushleft{\begin{hindi}
(वह ये है कि) ख़ुदा और उसके रसूल पर ईमान लाओ और अपने माल व जान से ख़ुदा की राह में जेहाद करो अगर तुम समझो तो यही तुम्हारे हक़ में बेहतर है
\end{hindi}}
\flushright{\begin{Arabic}
\quranayah[61][12]
\end{Arabic}}
\flushleft{\begin{hindi}
(ऐसा करोगे) तो वह भी इसके ऐवज़ में तुम्हारे गुनाह बख्श देगा और तुम्हें उन बाग़ों में दाख़िल करेगा जिनके नीचे नहरें जारी हैं और पाकीजा मकानात में (जगह देगा) जो जावेदानी बेहिश्त में हैं यही तो बड़ी कामयाबी है
\end{hindi}}
\flushright{\begin{Arabic}
\quranayah[61][13]
\end{Arabic}}
\flushleft{\begin{hindi}
और एक चीज़ जिसके तुम दिल दादा हो (यानि तुमको) ख़ुदा की तरफ से मदद (मिलेगी और अनक़रीब फतेह (होगी) और (ऐ रसूल) मोमिनीन को ख़ुशख़बरी (इसकी) दे दो
\end{hindi}}
\flushright{\begin{Arabic}
\quranayah[61][14]
\end{Arabic}}
\flushleft{\begin{hindi}
ऐ ईमानदारों ख़ुदा के मददगार बन जाओ जिस तरह मरियम के बेटे ईसा ने हवारियों से कहा था कि (भला) ख़ुदा की तरफ (बुलाने में) मेरे मददगार कौन लोग हैं तो हवारीन बोल उठे थे कि हम ख़ुदा के अनसार हैं तो बनी इसराईल में से एक गिरोह (उन पर) ईमान लाया और एक गिरोह काफिर रहा, तो जो लोग ईमान लाए हमने उनको उनके दुशमनों के मुक़ाबले में मदद दी तो आख़िर वही ग़ालिब रहे
\end{hindi}}
\chapter{Al-Jumu'ah (The Congregation)}
\begin{Arabic}
\Huge{\centerline{\basmalah}}\end{Arabic}
\flushright{\begin{Arabic}
\quranayah[62][1]
\end{Arabic}}
\flushleft{\begin{hindi}
जो चीज़ आसमानों में है और जो चीज़ ज़मीन में है (सब) ख़ुदा की तस्बीह करती हैं जो (हक़ीक़ी) बादशाह पाक ज़ात ग़ालिब हिकमत वाला है
\end{hindi}}
\flushright{\begin{Arabic}
\quranayah[62][2]
\end{Arabic}}
\flushleft{\begin{hindi}
वही तो जिसने जाहिलों में उन्हीं में का एक रसूल (मोहम्मद) भेजा जो उनके सामने उसकी आयतें पढ़ते और उनको पाक करते और उनको किताब और अक्ल की बातें सिखाते हैं अगरचे इसके पहले तो ये लोग सरीही गुमराही में (पड़े हुए) थे
\end{hindi}}
\flushright{\begin{Arabic}
\quranayah[62][3]
\end{Arabic}}
\flushleft{\begin{hindi}
और उनमें से उन लोगों की तरफ़ (भेजा) जो अभी तक उनसे मुलहिक़ नहीं हुए और वह तो ग़ालिब हिकमत वाला है
\end{hindi}}
\flushright{\begin{Arabic}
\quranayah[62][4]
\end{Arabic}}
\flushleft{\begin{hindi}
ख़ुदा का फज़ल है जिसको चाहता है अता फरमाता है और ख़ुदा तो बड़े फज़ल (व करम) का मालिक है
\end{hindi}}
\flushright{\begin{Arabic}
\quranayah[62][5]
\end{Arabic}}
\flushleft{\begin{hindi}
जिन लोगों (के सरों) पर तौरेत लदवायी गयी है उन्होने उस (के बार) को न उठाया उनकी मिसाल गधे की सी है जिस पर बड़ी बड़ी किताबें लदी हों जिन लोगों ने ख़ुदा की आयतों को झुठलाया उनकी भी क्या बुरी मिसाल है और ख़ुदा ज़ालिम लोगों को मंज़िल मकसूद तक नहीं पहुँचाया करता
\end{hindi}}
\flushright{\begin{Arabic}
\quranayah[62][6]
\end{Arabic}}
\flushleft{\begin{hindi}
(ऐ रसूल) तुम कह दो कि ऐ यहूदियों अगर तुम ये ख्याल करते हो कि तुम ही ख़ुदा के दोस्त हो और लोग नहीं तो अगर तुम (अपने दावे में) सच्चे हो तो मौत की तमन्ना करो
\end{hindi}}
\flushright{\begin{Arabic}
\quranayah[62][7]
\end{Arabic}}
\flushleft{\begin{hindi}
और ये लोग उन आमाल के सबब जो ये पहले कर चुके हैं कभी उसकी आरज़ू न करेंगे और ख़ुदा तो ज़ालिमों को जानता है
\end{hindi}}
\flushright{\begin{Arabic}
\quranayah[62][8]
\end{Arabic}}
\flushleft{\begin{hindi}
(ऐ रसूल) तुम कह दो कि मौत जिससे तुम लोग भागते हो वह तो ज़रूर तुम्हारे सामने आएगी फिर तुम पोशीदा और ज़ाहिर के जानने वाले (ख़ुदा) की तरफ लौटा दिए जाओगे फिर जो कुछ भी तुम करते थे वह तुम्हें बता देगा
\end{hindi}}
\flushright{\begin{Arabic}
\quranayah[62][9]
\end{Arabic}}
\flushleft{\begin{hindi}
ऐ ईमानदारों जब जुमा का दिन नमाज़ (जुमा) के लिए अज़ान दी जाए तो ख़ुदा की याद (नमाज़) की तरफ दौड़ पड़ो और (ख़रीद) व फरोख्त छोड़ दो अगर तुम समझते हो तो यही तुम्हारे हक़ में बेहतर है
\end{hindi}}
\flushright{\begin{Arabic}
\quranayah[62][10]
\end{Arabic}}
\flushleft{\begin{hindi}
फिर जब नमाज़ हो चुके तो ज़मीन में (जहाँ चाहो) जाओ और ख़ुदा के फज़ल (अपनी रोज़ी) की तलाश करो और ख़ुदा को बहुत याद करते रहो ताकि तुम दिली मुरादें पाओ
\end{hindi}}
\flushright{\begin{Arabic}
\quranayah[62][11]
\end{Arabic}}
\flushleft{\begin{hindi}
और (उनकी हालत तो ये है कि) जब ये लोग सौदा बिकता या तमाशा होता देखें तो उसकी तरफ टूट पड़े और तुमको खड़ा हुआ छोड़ दें (ऐ रसूल) तुम कह दो कि जो चीज़ ख़ुदा के यहाँ है वह तमाशे और सौदे से कहीं बेहतर है और ख़ुदा सबसे बेहतर रिज्क़ देने वाला है
\end{hindi}}
\chapter{Al-Munafiqun (The Hypocrites)}
\begin{Arabic}
\Huge{\centerline{\basmalah}}\end{Arabic}
\flushright{\begin{Arabic}
\quranayah[63][1]
\end{Arabic}}
\flushleft{\begin{hindi}
(ऐ रसूल) जब तुम्हारे पास मुनाफेक़ीन आते हैं तो कहते हैं कि हम तो इक़रार करते हैं कि आप यक़नीन ख़ुदा के रसूल हैं और ख़ुदा भी जानता है तुम यक़ीनी उसके रसूल हो मगर ख़ुदा ज़ाहिर किए देता है कि ये लोग अपने (एतक़ाद के लिहाज़ से) ज़रूर झूठे हैं
\end{hindi}}
\flushright{\begin{Arabic}
\quranayah[63][2]
\end{Arabic}}
\flushleft{\begin{hindi}
इन लोगों ने अपनी क़समों को सिपर बना रखा है तो (इसी के ज़रिए से) लोगों को ख़ुदा की राह से रोकते हैं बेशक ये लोग जो काम करते हैं बुरे हैं
\end{hindi}}
\flushright{\begin{Arabic}
\quranayah[63][3]
\end{Arabic}}
\flushleft{\begin{hindi}
इस सबब से कि (ज़ाहिर में) ईमान लाए फिर काफ़िर हो गए, तो उनके दिलों पर (गोया) मोहर लगा दी गयी है तो अब ये समझते ही नहीं
\end{hindi}}
\flushright{\begin{Arabic}
\quranayah[63][4]
\end{Arabic}}
\flushleft{\begin{hindi}
और जब तुम उनको देखोगे तो तनासुबे आज़ा की वजह से उनका क़द व क़ामत तुम्हें बहुत अच्छा मालूम होगा और गुफ्तगू करेंगे तो ऐसी कि तुम तवज्जो से सुनो (मगर अक्ल से ख़ाली) गोया दीवारों से लगायी हुयीं बेकार लकड़ियाँ हैं हर चीख़ की आवाज़ को समझते हैं कि उन्हीं पर आ पड़ी ये लोग तुम्हारे दुश्मन हैं तुम उनसे बचे रहो ख़ुदा इन्हें मार डाले ये कहाँ बहके फिरते हैं
\end{hindi}}
\flushright{\begin{Arabic}
\quranayah[63][5]
\end{Arabic}}
\flushleft{\begin{hindi}
और जब उनसे कहा जाता है कि आओ रसूलअल्लाह तुम्हारे वास्ते मग़फेरत की दुआ करें तो वह लोग अपने सर फेर लेते हैं और तुम उनको देखोगे कि तकब्बुर करते हुए मुँह फेर लेते हैं
\end{hindi}}
\flushright{\begin{Arabic}
\quranayah[63][6]
\end{Arabic}}
\flushleft{\begin{hindi}
तो तुम उनकी मग़फेरत की दुआ माँगो या न माँगो उनके हक़ में बराबर है (क्यों कि) ख़ुदा तो उन्हें हरगिज़ बख्शेगा नहीं ख़ुदा तो हरगिज़ बदकारों को मंज़िले मक़सूद तक नहीं पहुँचाया करता
\end{hindi}}
\flushright{\begin{Arabic}
\quranayah[63][7]
\end{Arabic}}
\flushleft{\begin{hindi}
ये वही लोग तो हैं जो (अन्सार से) कहते हैं कि जो (मुहाजिरीन) रसूले ख़ुदा के पास रहते हैं उन पर ख़र्च न करो यहाँ तक कि ये लोग ख़ुद तितर बितर हो जाएँ हालॉकि सारे आसमान और ज़मीन के ख़ज़ाने ख़ुदा ही के पास हैं मगर मुनाफेक़ीन नहीं समझते
\end{hindi}}
\flushright{\begin{Arabic}
\quranayah[63][8]
\end{Arabic}}
\flushleft{\begin{hindi}
ये लोग तो कहते हैं कि अगर हम लौट कर मदीने पहुँचे तो इज्ज़दार लोग (ख़ुद) ज़लील (रसूल) को ज़रूर निकाल बाहर कर देंगे हालॉकि इज्ज़त तो ख़ास ख़ुदा और उसके रसूल और मोमिनीन के लिए है मगर मुनाफेक़ीन नहीं जानते
\end{hindi}}
\flushright{\begin{Arabic}
\quranayah[63][9]
\end{Arabic}}
\flushleft{\begin{hindi}
ऐ ईमानदारों तुम्हारे माल और तुम्हारी औलाद तुमको ख़ुदा की याद से ग़ाफिल न करे और जो ऐसा करेगा तो वही लोग घाटे में रहेंगे
\end{hindi}}
\flushright{\begin{Arabic}
\quranayah[63][10]
\end{Arabic}}
\flushleft{\begin{hindi}
और हमने जो कुछ तुम्हें दिया है उसमें से क़ब्ल इसके (ख़ुदा की राह में) ख़र्च कर डालो कि तुममें से किसी की मौत आ जाए तो (इसकी नौबत न आए कि) कहने लगे कि परवरदिगार तूने मुझे थोड़ी सी मोहलत और क्यों न दी ताकि ख़ैरात करता और नेकीकारों से हो जाता
\end{hindi}}
\flushright{\begin{Arabic}
\quranayah[63][11]
\end{Arabic}}
\flushleft{\begin{hindi}
और जब किसी की मौत आ जाती है तो ख़ुदा उसको हरगिज़ मोहलत नहीं देता और जो कुछ तुम करते हो ख़ुदा उससे ख़बरदार है
\end{hindi}}
\chapter{At-Taghabun (The Manifestation of Losses)}
\begin{Arabic}
\Huge{\centerline{\basmalah}}\end{Arabic}
\flushright{\begin{Arabic}
\quranayah[64][1]
\end{Arabic}}
\flushleft{\begin{hindi}
जो चीज़ आसमानों में है और जो चीज़ ज़मीन में है (सब) ख़ुदा ही की तस्बीह करती है उसी की बादशाहत है और तारीफ़ उसी के लिए सज़ावार है और वही हर चीज़ पर कादिर है
\end{hindi}}
\flushright{\begin{Arabic}
\quranayah[64][2]
\end{Arabic}}
\flushleft{\begin{hindi}
वही तो है जिसने तुम लोगों को पैदा किया कोई तुममें काफ़िर है और कोई मोमिन और जो कुछ तुम करते हो ख़ुदा उसको देख रहा है
\end{hindi}}
\flushright{\begin{Arabic}
\quranayah[64][3]
\end{Arabic}}
\flushleft{\begin{hindi}
उसी ने सारे आसमान व ज़मीन को हिकमत व मसलेहत से पैदा किया और उसी ने तुम्हारी सूरतें बनायीं तो सबसे अच्छी सूरतें बनायीं और उसी की तरफ लौटकर जाना हैं
\end{hindi}}
\flushright{\begin{Arabic}
\quranayah[64][4]
\end{Arabic}}
\flushleft{\begin{hindi}
जो कुछ सारे आसमान व ज़मीन में है वह (सब) जानता है और जो कुछ तुम छुपा कर या खुल्लम खुल्ला करते हो उससे (भी) वाकिफ़ है और ख़ुदा तो दिल के भेद तक से आगाह है
\end{hindi}}
\flushright{\begin{Arabic}
\quranayah[64][5]
\end{Arabic}}
\flushleft{\begin{hindi}
क्या तुम्हें उनकी ख़बर नहीं पहुँची जिन्होंने (तुम से) पहले कुफ़्र किया तो उन्होने अपने काम की सज़ा का (दुनिया में) मज़ा चखा और (आख़िरत में तो) उनके लिए दर्दनाक अज़ाब है
\end{hindi}}
\flushright{\begin{Arabic}
\quranayah[64][6]
\end{Arabic}}
\flushleft{\begin{hindi}
ये इस वजह से कि उनके पास पैग़म्बर वाज़ेए व रौशन मौजिज़े लेकर आ चुके थे तो कहने लगे कि क्या आदमी हमारे हादी बनेंगें ग़रज़ ये लोग काफ़िर हो बैठे और मुँह फेर बैठे और ख़ुदा ने भी (उनकी) परवाह न की और ख़ुदा तो बे परवा सज़ावारे हम्द है
\end{hindi}}
\flushright{\begin{Arabic}
\quranayah[64][7]
\end{Arabic}}
\flushleft{\begin{hindi}
काफ़िरों का ख्याल ये है कि ये लोग दोबारा न उठाए जाएँगे (ऐ रसूल) तुम कह दो वहाँ अपने परवरदिगार की क़सम तुम ज़रूर उठाए जाओगे फिर जो जो काम तुम करते रहे वह तुम्हें बता देगा और ये तो ख़ुदा पर आसान है
\end{hindi}}
\flushright{\begin{Arabic}
\quranayah[64][8]
\end{Arabic}}
\flushleft{\begin{hindi}
तो तुम ख़ुदा और उसके रसूल पर उसी नूर पर ईमान लाओ जिसको हमने नाज़िल किया है और जो कुछ तुम करते हो ख़ुदा उससे ख़बरदार है
\end{hindi}}
\flushright{\begin{Arabic}
\quranayah[64][9]
\end{Arabic}}
\flushleft{\begin{hindi}
जब वह क़यामत के दिन तुम सबको जमा करेगा फिर यही हार जीत का दिन होगा और जो शख़्श ख़ुदा पर ईमान लाए और अच्छा काम करे वह उससे उसकी बुराइयाँ दूर कर देगा और उसको (बेहिश्त में) उन बाग़ों में दाख़िल करेगा जिनके नीचे नहरें जारी हैं वह उनमें अबादुल आबाद हमेशा रहेगा, यही तो बड़ी कामयाबी है
\end{hindi}}
\flushright{\begin{Arabic}
\quranayah[64][10]
\end{Arabic}}
\flushleft{\begin{hindi}
और जो लोग काफ़िर हैं और हमारी आयतों को झुठलाते रहे यही लोग जहन्नुमी हैं कि हमेशा उसी में रहेंगे और वह क्या बुरा ठिकाना है
\end{hindi}}
\flushright{\begin{Arabic}
\quranayah[64][11]
\end{Arabic}}
\flushleft{\begin{hindi}
जब कोई मुसीबत आती है तो ख़ुदा के इज़न से और जो शख़्श ख़ुदा पर ईमान लाता है तो ख़ुदा उसके कल्ब की हिदायत करता है और ख़ुदा हर चीज़ से ख़ूब आगाह है
\end{hindi}}
\flushright{\begin{Arabic}
\quranayah[64][12]
\end{Arabic}}
\flushleft{\begin{hindi}
और ख़ुदा की इताअत करो और रसूल की इताअत करो फिर अगर तुमने मुँह फेरा तो हमारे रसूल पर सिर्फ पैग़ाम का वाज़ेए करके पहुँचा देना फर्ज़ है
\end{hindi}}
\flushright{\begin{Arabic}
\quranayah[64][13]
\end{Arabic}}
\flushleft{\begin{hindi}
ख़ुदा (वह है कि) उसके सिवा कोई माबूद नहीं और मोमिनो को ख़ुदा ही पर भरोसा करना चाहिए
\end{hindi}}
\flushright{\begin{Arabic}
\quranayah[64][14]
\end{Arabic}}
\flushleft{\begin{hindi}
ऐ ईमानदारों तुम्हारी बीवियों और तुम्हारी औलाद में से बाज़ तुम्हारे दुशमन हैं तो तुम उनसे बचे रहो और अगर तुम माफ कर दो दरगुज़र करो और बख्श दो तो ख़ुदा बड़ा बख्शने वाला मेहरबान है
\end{hindi}}
\flushright{\begin{Arabic}
\quranayah[64][15]
\end{Arabic}}
\flushleft{\begin{hindi}
तुम्हारे माल और तुम्हारी औलादे बस आज़माइश है और ख़ुदा के यहाँ तो बड़ा अज्र (मौजूद) है
\end{hindi}}
\flushright{\begin{Arabic}
\quranayah[64][16]
\end{Arabic}}
\flushleft{\begin{hindi}
तो जहाँ तक तुम से हो सके ख़ुदा से डरते रहो और (उसके एहकाम) सुनो और मानों और अपनी बेहतरी के वास्ते (उसकी राह में) ख़र्च करो और जो शख़्श अपने नफ्स की हिरस से बचा लिया गया तो ऐसे ही लोग मुरादें पाने वाले हैं
\end{hindi}}
\flushright{\begin{Arabic}
\quranayah[64][17]
\end{Arabic}}
\flushleft{\begin{hindi}
अगर तुम ख़ुदा को कर्जे हसना दोगे तो वह उसको तुम्हारे वास्ते दूना कर देगा और तुमको बख्श देगा और ख़ुदा तो बड़ा क़द्रदान व बुर्दबार है
\end{hindi}}
\flushright{\begin{Arabic}
\quranayah[64][18]
\end{Arabic}}
\flushleft{\begin{hindi}
पोशीदा और ज़ाहिर का जानने वाला ग़ालिब हिकमत वाला है
\end{hindi}}
\chapter{At-Talaq (Divorce)}
\begin{Arabic}
\Huge{\centerline{\basmalah}}\end{Arabic}
\flushright{\begin{Arabic}
\quranayah[65][1]
\end{Arabic}}
\flushleft{\begin{hindi}
ऐ रसूल (मुसलमानों से कह दो) जब तुम अपनी बीवियों को तलाक़ दो तो उनकी इद्दत (पाकी) के वक्त तलाक़ दो और इद्दा का शुमार रखो और अपने परवरदिगार ख़ुदा से डरो और (इद्दे के अन्दर) उनके घर से उन्हें न निकालो और वह ख़ुद भी घर से न निकलें मगर जब वह कोई सरीही बेहयाई का काम कर बैठें (तो निकाल देने में मुज़ायका नहीं) और ये ख़ुदा की (मुक़र्रर की हुई) हदें हैं और जो ख़ुदा की हदों से तजाउज़ करेगा तो उसने अपने ऊपर आप ज़ुल्म किया तो तू नहीं जानता यायद ख़ुदा उसके बाद कोई बात पैदा करे (जिससे मर्द पछताए और मेल हो जाए)
\end{hindi}}
\flushright{\begin{Arabic}
\quranayah[65][2]
\end{Arabic}}
\flushleft{\begin{hindi}
तो जब ये अपना इद्दा पूरा करने के करीब पहुँचे तो या तुम उन्हें उनवाने शाइस्ता से रोक लो या अच्छी तरह रूख़सत ही कर दो और (तलाक़ के वक्त) अपने लोगों में से दो आदिलों को गवाह क़रार दे लो और गवाहों तुम ख़ुदा के वास्ते ठीक ठीक गवाही देना इन बातों से उस शख़्श को नसीहत की जाती है जो ख़ुदा और रोजे अाख़ेरत पर ईमान रखता हो और जो ख़ुदा से डरेगा तो ख़ुदा उसके लिए नजात की सूरत निकाल देगा
\end{hindi}}
\flushright{\begin{Arabic}
\quranayah[65][3]
\end{Arabic}}
\flushleft{\begin{hindi}
और उसको ऐसी जगह से रिज़क़ देगा जहाँ से वहम भी न हो और जिसने ख़ुदा पर भरोसा किया तो वह उसके लिए काफी है बेशक ख़ुदा अपने काम को पूरा करके रहता है ख़ुदा ने हर चीज़ का एक अन्दाज़ा मुक़र्रर कर रखा है
\end{hindi}}
\flushright{\begin{Arabic}
\quranayah[65][4]
\end{Arabic}}
\flushleft{\begin{hindi}
और जो औरतें हैज़ से मायूस हो चुकी अगर तुम को उनके इद्दे में शक़ होवे तो उनका इद्दा तीन महीने है और (अला हाज़ल क़यास) वह औरतें जिनको हैज़ हुआ ही नहीं और हामेला औरतों का इद्दा उनका बच्चा जनना है और जो ख़ुदा से डरता है ख़ुदा उसके काम मे सहूलित पैदा करेगा
\end{hindi}}
\flushright{\begin{Arabic}
\quranayah[65][5]
\end{Arabic}}
\flushleft{\begin{hindi}
ये ख़ुदा का हुक्म है जो ख़ुदा ने तुम पर नाज़िल किया है और जो ख़ुदा डरता रहेगा तो वह उसके गुनाह उससे दूर कर देगा और उसे बड़ा दरजा देगा
\end{hindi}}
\flushright{\begin{Arabic}
\quranayah[65][6]
\end{Arabic}}
\flushleft{\begin{hindi}
मुतलक़ा औरतों को (इद्दे तक) अपने मक़दूर मुताबिक दे रखो जहाँ तुम ख़ुद रहते हो और उनको तंग करने के लिए उनको तकलीफ न पहुँचाओ और अगर वह हामेला हो तो बच्चा जनने तक उनका खर्च देते रहो फिर (जनने के बाद) अगर वह बच्चे को तुम्हारी ख़ातिर दूध पिलाए तो उन्हें उनकी (मुनासिब) उजरत दे दो और बाहम सलाहियत से दस्तूर के मुताबिक बात चीत करो और अगर तुम बाहम कश म कश करो तो बच्चे को उसके (बाप की) ख़ातिर से कोई और औरत दूध पिला देगी
\end{hindi}}
\flushright{\begin{Arabic}
\quranayah[65][7]
\end{Arabic}}
\flushleft{\begin{hindi}
गुन्जाइश वाले को अपनी गुन्जाइश के मुताबिक़ ख़र्च करना चाहिए और जिसकी रोज़ी तंग हो वह जितना ख़ुदा ने उसे दिया है उसमें से खर्च करे ख़ुदा ने जिसको जितना दिया है बस उसी के मुताबिक़ तकलीफ़ दिया करता है ख़ुदा अनकरीब ही तंगी के बाद फ़राख़ी अता करेगा
\end{hindi}}
\flushright{\begin{Arabic}
\quranayah[65][8]
\end{Arabic}}
\flushleft{\begin{hindi}
और बहुत सी बस्तियों (वाले) ने अपने परवरदिगार और उसके रसूलों के हुक्म से सरकशी की तो हमने उनका बड़ी सख्ती से हिसाब लिया और उन्हें बुरे अज़ाब की सज़ा दी
\end{hindi}}
\flushright{\begin{Arabic}
\quranayah[65][9]
\end{Arabic}}
\flushleft{\begin{hindi}
तो उन्होने अपने काम की सज़ा का मज़ा चख लिया और उनके काम का अन्जाम घाटा ही था
\end{hindi}}
\flushright{\begin{Arabic}
\quranayah[65][10]
\end{Arabic}}
\flushleft{\begin{hindi}
ख़ुदा ने उनके लिए सख्त अज़ाब तैयार कर रखा है तो ऐ अक्लमन्दों जो ईमान ला चुके हो ख़ुदा से डरते रहो ख़ुदा ने तुम्हारे पास (अपनी) याद क़ुरान और अपना रसूल भेज दिया है
\end{hindi}}
\flushright{\begin{Arabic}
\quranayah[65][11]
\end{Arabic}}
\flushleft{\begin{hindi}
जो तुम्हारे सामने वाज़ेए आयतें पढ़ता है ताकि जो लोग ईमान लाए और अच्छे अच्छे काम करते रहे उनको (कुफ़्र की) तारिक़ियों से ईमान की रौशनी की तरफ़ निकाल लाए और जो ख़ुदा पर ईमान लाए और अच्छे अच्छे काम करे तो ख़ुदा उसको (बेहिश्त के) उन बाग़ों में दाखिल करेगा जिनके नीचे नहरें जारी हैं और वह उसमें अबादुल आबाद तक रहेंगे ख़ुदा ने उनको अच्छी अच्छी रोज़ी दी है
\end{hindi}}
\flushright{\begin{Arabic}
\quranayah[65][12]
\end{Arabic}}
\flushleft{\begin{hindi}
ख़ुदा ही तो है जिसने सात आसमान पैदा किए और उन्हीं के बराबर ज़मीन को भी उनमें ख़ुदा का हुक्म नाज़िल होता रहता है - ताकि तुम लोग जान लो कि ख़ुदा हर चीज़ पर कादिर है और बेशक ख़ुदा अपने इल्म से हर चीज़ पर हावी है
\end{hindi}}
\chapter{At-Tahrim (The Prohibition)}
\begin{Arabic}
\Huge{\centerline{\basmalah}}\end{Arabic}
\flushright{\begin{Arabic}
\quranayah[66][1]
\end{Arabic}}
\flushleft{\begin{hindi}
ऐ रसूल जो चीज़ ख़ुदा ने तुम्हारे लिए हलाल की है तुम उससे अपनी बीवियों की ख़ुशनूदी के लिए क्यों किनारा कशी करो और ख़ुदा तो बड़ा बख्शने वाला मेहरबान है
\end{hindi}}
\flushright{\begin{Arabic}
\quranayah[66][2]
\end{Arabic}}
\flushleft{\begin{hindi}
ख़ुदा ने तुम लोगों के लिए क़समों को तोड़ डालने का कफ्फ़ार मुक़र्रर कर दिया है और ख़ुदा ही तुम्हारा कारसाज़ है और वही वाक़िफ़कार हिकमत वाला है
\end{hindi}}
\flushright{\begin{Arabic}
\quranayah[66][3]
\end{Arabic}}
\flushleft{\begin{hindi}
और जब पैग़म्बर ने अपनी बाज़ बीवी (हफ़सा) से चुपके से कोई बात कही फिर जब उसने (बावजूद मुमानियत) उस बात की (आयशा को) ख़बर दे दी और ख़ुदा ने इस अम्र को रसूल पर ज़ाहिर कर दिया तो रसूल ने (आयशा को) बाज़ बात (किस्सा मारिया) जता दी और बाज़ बात (किस्साए यहद) टाल दी ग़रज़ जब रसूल ने इस वाक़िये (हफ़सा के अफ़शाए राज़) कि उस (आयशा) को ख़बर दी तो हैरत से बोल उठीं आपको इस बात (अफ़शाए राज़) की किसने ख़बर दी रसूल ने कहा मुझे बड़े वाक़िफ़कार ख़बरदार (ख़ुदा) ने बता दिया
\end{hindi}}
\flushright{\begin{Arabic}
\quranayah[66][4]
\end{Arabic}}
\flushleft{\begin{hindi}
(तो ऐ हफ़सा व आयशा) अगर तुम दोनों (इस हरकत से) तौबा करो तो ख़ैर क्योंकि तुम दोनों के दिल टेढ़े हैं और अगर तुम दोनों रसूल की मुख़ालेफ़त में एक दूसरे की अयानत करती रहोगी तो कुछ परवा नहीं (क्यों कि) ख़ुदा और जिबरील और तमाम ईमानदारों में नेक शख़्श उनके मददगार हैं और उनके अलावा कुल फरिश्ते मददगार हैं
\end{hindi}}
\flushright{\begin{Arabic}
\quranayah[66][5]
\end{Arabic}}
\flushleft{\begin{hindi}
अगर रसूल तुम लोगों को तलाक़ दे दे तो अनक़रीब ही उनका परवरदिगार तुम्हारे बदले उनको तुमसे अच्छी बीवियाँ अता करे जो फ़रमाबरदार ईमानदार ख़ुदा रसूल की मुतीय (गुनाहों से) तौबा करने वालियाँ इबादत गुज़ार रोज़ा रखने वालियाँ ब्याही हुई
\end{hindi}}
\flushright{\begin{Arabic}
\quranayah[66][6]
\end{Arabic}}
\flushleft{\begin{hindi}
और बिन ब्याही कुंवारियाँ हो ऐ ईमानदारों अपने आपको और अपने लड़के बालों को (जहन्नुम की) आग से बचाओ जिसके इंधन आदमी और पत्थर होंगे उन पर वह तन्दख़ू सख्त मिजाज़ फ़रिश्ते (मुक़र्रर) हैं कि ख़ुदा जिस बात का हुक्म देता है उसकी नाफरमानी नहीं करते और जो हुक्म उन्हें मिलता है उसे बजा लाते हैं
\end{hindi}}
\flushright{\begin{Arabic}
\quranayah[66][7]
\end{Arabic}}
\flushleft{\begin{hindi}
(जब कुफ्फ़ार दोज़ख़ के सामने आएँगे तो कहा जाएगा) काफ़िरों आज बहाने न ढूँढो जो कुछ तुम करते थे तुम्हें उसकी सज़ा दी जाएगी
\end{hindi}}
\flushright{\begin{Arabic}
\quranayah[66][8]
\end{Arabic}}
\flushleft{\begin{hindi}
ऐ ईमानदारों ख़ुदा की बारगाह में साफ़ ख़ालिस दिल से तौबा करो तो (उसकी वजह से) उम्मीद है कि तुम्हारा परवरदिगार तुमसे तुम्हारे गुनाह दूर कर दे और तुमको (बेहिश्त के) उन बाग़ों में दाखिल करे जिनके नीचे नहरें जारी हैं उस दिन जब ख़ुदा रसूल को और उन लोगों को जो उनके साथ ईमान लाए हैं रूसवा नहीं करेगा (बल्कि) उनका नूर उनके आगे आगे और उनके दाहिने तरफ़ (रौशनी करता) चल रहा होगा और ये लोग ये दुआ करते होंगे परवरदिगार हमारे लिए हमारा नूर पूरा कर और हमें बख्य दे बेशक तू हर चीज़ पर कादिर है
\end{hindi}}
\flushright{\begin{Arabic}
\quranayah[66][9]
\end{Arabic}}
\flushleft{\begin{hindi}
ऐ रसूल काफ़िरों और मुनाफ़िकों से जेहाद करो और उन पर सख्ती करो और उनका ठिकाना जहन्नुम है और वह क्या बुरा ठिकाना है
\end{hindi}}
\flushright{\begin{Arabic}
\quranayah[66][10]
\end{Arabic}}
\flushleft{\begin{hindi}
ख़ुदा ने काफिरों (की इबरत) के वास्ते नूह की बीवी (वाएला) और लूत की बीवी (वाहेला) की मसल बयान की है कि ये दोनो हमारे बन्दों के तसर्रुफ़ थीं तो दोनों ने अपने शौहरों से दगा की तो उनके शौहर ख़ुदा के मुक़ाबले में उनके कुछ भी काम न आए और उनको हुक्म दिया गया कि और जाने वालों के साथ जहन्नुम में तुम दोनों भी दाखिल हो जाओ
\end{hindi}}
\flushright{\begin{Arabic}
\quranayah[66][11]
\end{Arabic}}
\flushleft{\begin{hindi}
और ख़ुदा ने मोमिनीन (की तसल्ली) के लिए फिरऔन की बीवी (आसिया) की मिसाल बयान फ़रमायी है कि जब उसने दुआ की परवरदिगार मेरे लिए अपने यहाँ बेहिश्त में एक घर बना और मुझे फिरऔन और उसकी कारस्तानी से नजात दे और मुझे ज़ालिम लोगो (के हाथ) से छुटकारा अता फ़रमा
\end{hindi}}
\flushright{\begin{Arabic}
\quranayah[66][12]
\end{Arabic}}
\flushleft{\begin{hindi}
और (दूसरी मिसाल) इमरान की बेटी मरियम जिसने अपनी शर्मगाह को महफूज़ रखा तो हमने उसमें रूह फूंक दी और उसने अपने परवरदिगार की बातों और उसकी किताबों की तस्दीक़ की और फरमाबरदारों में थी
\end{hindi}}
\chapter{Al-Mulk (The Kingdom)}
\begin{Arabic}
\Huge{\centerline{\basmalah}}\end{Arabic}
\flushright{\begin{Arabic}
\quranayah[67][1]
\end{Arabic}}
\flushleft{\begin{hindi}
जिस (ख़ुदा) के कब्ज़े में (सारे जहाँन की) बादशाहत है वह बड़ी बरकत वाला है और वह हर चीज़ पर कादिर है
\end{hindi}}
\flushright{\begin{Arabic}
\quranayah[67][2]
\end{Arabic}}
\flushleft{\begin{hindi}
जिसने मौत और ज़िन्दगी को पैदा किया ताकि तुम्हें आज़माए कि तुममें से काम में सबसे अच्छा कौन है और वह ग़ालिब (और) बड़ा बख्शने वाला है
\end{hindi}}
\flushright{\begin{Arabic}
\quranayah[67][3]
\end{Arabic}}
\flushleft{\begin{hindi}
जिसने सात आसमान तले ऊपर बना डाले भला तुझे ख़ुदा की आफ़रिनश में कोई कसर नज़र आती है तो फिर ऑंख उठाकर देख भला तुझे कोई शिग़ाफ़ नज़र आता है
\end{hindi}}
\flushright{\begin{Arabic}
\quranayah[67][4]
\end{Arabic}}
\flushleft{\begin{hindi}
फिर दुबारा ऑंख उठा कर देखो तो (हर बार तेरी) नज़र नाकाम और थक कर तेरी तरफ पलट आएगी
\end{hindi}}
\flushright{\begin{Arabic}
\quranayah[67][5]
\end{Arabic}}
\flushleft{\begin{hindi}
और हमने नीचे वाले (पहले) आसमान को (तारों के) चिराग़ों से ज़ीनत दी है और हमने उनको शैतानों के मारने का आला बनाया और हमने उनके लिए दहकती हुई आग का अज़ाब तैयार कर रखा है
\end{hindi}}
\flushright{\begin{Arabic}
\quranayah[67][6]
\end{Arabic}}
\flushleft{\begin{hindi}
और जो लोग अपने परवरदिगार के मुनकिर हैं उनके लिए जहन्नुम का अज़ाब है और वह (बहुत) बुरा ठिकाना है
\end{hindi}}
\flushright{\begin{Arabic}
\quranayah[67][7]
\end{Arabic}}
\flushleft{\begin{hindi}
जब ये लोग इसमें डाले जाएँगे तो उसकी बड़ी चीख़ सुनेंगे और वह जोश मार रही होगी
\end{hindi}}
\flushright{\begin{Arabic}
\quranayah[67][8]
\end{Arabic}}
\flushleft{\begin{hindi}
बल्कि गोया मारे जोश के फट पड़ेगी जब उसमें (उनका) कोई गिरोह डाला जाएगा तो उनसे दारोग़ए जहन्नुम पूछेगा क्या तुम्हारे पास कोई डराने वाला पैग़म्बर नहीं आया था
\end{hindi}}
\flushright{\begin{Arabic}
\quranayah[67][9]
\end{Arabic}}
\flushleft{\begin{hindi}
वह कहेंगे हॉ हमारे पास डराने वाला तो ज़रूर आया था मगर हमने उसको झुठला दिया और कहा कि ख़ुदा ने तो कुछ नाज़िल ही नहीं किया तुम तो बड़ी (गहरी) गुमराही में (पड़े) हो
\end{hindi}}
\flushright{\begin{Arabic}
\quranayah[67][10]
\end{Arabic}}
\flushleft{\begin{hindi}
और (ये भी) कहेंगे कि अगर (उनकी बात) सुनते या समझते तब तो (आज) दोज़ख़ियों में न होते
\end{hindi}}
\flushright{\begin{Arabic}
\quranayah[67][11]
\end{Arabic}}
\flushleft{\begin{hindi}
ग़रज़ वह अपने गुनाह का इक़रार कर लेंगे तो दोज़ख़ियों को ख़ुदा की रहमत से दूरी है
\end{hindi}}
\flushright{\begin{Arabic}
\quranayah[67][12]
\end{Arabic}}
\flushleft{\begin{hindi}
बेशक जो लोग अपने परवरदिगार से बेदेखे भाले डरते हैं उनके लिए मग़फेरत और बड़ा भारी अज्र है
\end{hindi}}
\flushright{\begin{Arabic}
\quranayah[67][13]
\end{Arabic}}
\flushleft{\begin{hindi}
और तुम अपनी बात छिपकर कहो या खुल्लम खुल्ला वह तो दिल के भेदों तक से ख़ूब वाक़िफ़ है
\end{hindi}}
\flushright{\begin{Arabic}
\quranayah[67][14]
\end{Arabic}}
\flushleft{\begin{hindi}
भला जिसने पैदा किया वह तो बेख़बर और वह तो बड़ा बारीकबीन वाक़िफ़कार है
\end{hindi}}
\flushright{\begin{Arabic}
\quranayah[67][15]
\end{Arabic}}
\flushleft{\begin{hindi}
वही तो है जिसने ज़मीन को तुम्हारे लिए नरम (व हमवार) कर दिया तो उसके अतराफ़ व जवानिब में चलो फिरो और उसकी (दी हुई) रोज़ी खाओ
\end{hindi}}
\flushright{\begin{Arabic}
\quranayah[67][16]
\end{Arabic}}
\flushleft{\begin{hindi}
और फिर उसी की तरफ क़ब्र से उठ कर जाना है क्या तुम उस शख़्श से जो आसमान में (हुकूमत करता है) इस बात से बेख़ौफ़ हो कि तुमको ज़मीन में धॅसा दे फिर वह एकबारगी उलट पुलट करने लगे
\end{hindi}}
\flushright{\begin{Arabic}
\quranayah[67][17]
\end{Arabic}}
\flushleft{\begin{hindi}
या तुम इस बात से बेख़ौफ हो कि जो आसमान में (सल्तनत करता) है कि तुम पर पत्थर भरी ऑंधी चलाए तो तुम्हें अनक़रीेब ही मालूम हो जाएगा कि मेरा डराना कैसा है
\end{hindi}}
\flushright{\begin{Arabic}
\quranayah[67][18]
\end{Arabic}}
\flushleft{\begin{hindi}
और जो लोग उनसे पहले थे उन्होने झुठलाया था तो (देखो) कि मेरी नाख़ुशी कैसी थी
\end{hindi}}
\flushright{\begin{Arabic}
\quranayah[67][19]
\end{Arabic}}
\flushleft{\begin{hindi}
क्या उन लोगों ने अपने सरों पर चिड़ियों को उड़ते नहीं देखा जो परों को फैलाए रहती हैं और समेट लेती हैं कि ख़ुदा के सिवा उन्हें कोई रोके नहीं रह सकता बेशक वह हर चीज़ को देख रहा है
\end{hindi}}
\flushright{\begin{Arabic}
\quranayah[67][20]
\end{Arabic}}
\flushleft{\begin{hindi}
भला ख़ुदा के सिवा ऐसा कौन है जो तुम्हारी फ़ौज बनकर तुम्हारी मदद करे काफ़िर लोग तो धोखे ही (धोखे) में हैं भला ख़ुदा अगर अपनी (दी हुई) रोज़ी रोक ले तो कौन ऐसा है जो तुम्हें रिज़क़ दे
\end{hindi}}
\flushright{\begin{Arabic}
\quranayah[67][21]
\end{Arabic}}
\flushleft{\begin{hindi}
मगर ये कुफ्फ़ार तो सरकशी और नफ़रत (के भँवर) में फँसे हुए हैं भला जो शख़्श औंधे मुँह के बाल चले वह ज्यादा हिदायत याफ्ता होगा
\end{hindi}}
\flushright{\begin{Arabic}
\quranayah[67][22]
\end{Arabic}}
\flushleft{\begin{hindi}
या वह शख़्श जो सीधा बराबर राहे रास्त पर चल रहा हो (ऐ रसूल) तुम कह दो कि ख़ुदा तो वही है जिसने तुमको नित नया पैदा किया
\end{hindi}}
\flushright{\begin{Arabic}
\quranayah[67][23]
\end{Arabic}}
\flushleft{\begin{hindi}
और तुम्हारे वास्ते कान और ऑंख और दिल बनाए (मगर) तुम तो बहुत कम शुक्र अदा करते हो
\end{hindi}}
\flushright{\begin{Arabic}
\quranayah[67][24]
\end{Arabic}}
\flushleft{\begin{hindi}
कह दो कि वही तो है जिसने तुमको ज़मीन में फैला दिया और उसी के सामने जमा किए जाओगे
\end{hindi}}
\flushright{\begin{Arabic}
\quranayah[67][25]
\end{Arabic}}
\flushleft{\begin{hindi}
और कुफ्फ़ार कहते हैं कि अगर तुम सच्चे हो तो (आख़िर) ये वायदा कब (पूरा) होगा
\end{hindi}}
\flushright{\begin{Arabic}
\quranayah[67][26]
\end{Arabic}}
\flushleft{\begin{hindi}
(ऐ रसूल) तुम कह दो कि (इसका) इल्म तो बस ख़ुदा ही को है और मैं तो सिर्फ साफ़ साफ़ (अज़ाब से) डराने वाला हूँ
\end{hindi}}
\flushright{\begin{Arabic}
\quranayah[67][27]
\end{Arabic}}
\flushleft{\begin{hindi}
तो जब ये लोग उसे करीब से देख लेंगे (ख़ौफ के मारे) काफिरों के चेहरे बिगड़ जाएँगे और उनसे कहा जाएगा ये वही है जिसके तुम ख़वास्तग़ार थे
\end{hindi}}
\flushright{\begin{Arabic}
\quranayah[67][28]
\end{Arabic}}
\flushleft{\begin{hindi}
(ऐ रसूल) तुम कह दो भला देखो तो कि अगर ख़ुदा मुझको और मेरे साथियों को हलाक कर दे या हम पर रहम फरमाए तो काफ़िरों को दर्दनाक अज़ाब से कौन पनाह देगा
\end{hindi}}
\flushright{\begin{Arabic}
\quranayah[67][29]
\end{Arabic}}
\flushleft{\begin{hindi}
तुम कह दो कि वही (ख़ुदा) बड़ा रहम करने वाला है जिस पर हम ईमान लाए हैं और हमने तो उसी पर भरोसा कर लिया है तो अनक़रीब ही तुम्हें मालूम हो जाएगा कि कौन सरीही गुमराही में (पड़ा) है
\end{hindi}}
\flushright{\begin{Arabic}
\quranayah[67][30]
\end{Arabic}}
\flushleft{\begin{hindi}
ऐ रसूल तुम कह दो कि भला देखो तो कि अगर तुम्हारा पानी ज़मीन के अन्दर चला जाए कौन ऐसा है जो तुम्हारे लिए पानी का चश्मा बहा लाए
\end{hindi}}
\chapter{Al-Qalam (The Pen)}
\begin{Arabic}
\Huge{\centerline{\basmalah}}\end{Arabic}
\flushright{\begin{Arabic}
\quranayah[68][1]
\end{Arabic}}
\flushleft{\begin{hindi}
नून क़लम की और उस चीज़ की जो लिखती हैं (उसकी) क़सम है
\end{hindi}}
\flushright{\begin{Arabic}
\quranayah[68][2]
\end{Arabic}}
\flushleft{\begin{hindi}
कि तुम अपने परवरदिगार के फ़ज़ल (व करम) से दीवाने नहीं हो
\end{hindi}}
\flushright{\begin{Arabic}
\quranayah[68][3]
\end{Arabic}}
\flushleft{\begin{hindi}
और तुम्हारे वास्ते यक़ीनन वह अज्र है जो कभी ख़त्म ही न होगा
\end{hindi}}
\flushright{\begin{Arabic}
\quranayah[68][4]
\end{Arabic}}
\flushleft{\begin{hindi}
और बेशक तुम्हारे एख़लाक़ बड़े आला दर्जे के हैं
\end{hindi}}
\flushright{\begin{Arabic}
\quranayah[68][5]
\end{Arabic}}
\flushleft{\begin{hindi}
तो अनक़रीब ही तुम भी देखोगे और ये कुफ्फ़ार भी देख लेंगे
\end{hindi}}
\flushright{\begin{Arabic}
\quranayah[68][6]
\end{Arabic}}
\flushleft{\begin{hindi}
कि तुममें दीवाना कौन है
\end{hindi}}
\flushright{\begin{Arabic}
\quranayah[68][7]
\end{Arabic}}
\flushleft{\begin{hindi}
बेशक तुम्हारा परवरदिगार इनसे ख़ूब वाक़िफ़ है जो उसकी राह से भटके हुए हैं और वही हिदायत याफ्ता लोगों को भी ख़ूब जानता है
\end{hindi}}
\flushright{\begin{Arabic}
\quranayah[68][8]
\end{Arabic}}
\flushleft{\begin{hindi}
तो तुम झुठलाने वालों का कहना न मानना
\end{hindi}}
\flushright{\begin{Arabic}
\quranayah[68][9]
\end{Arabic}}
\flushleft{\begin{hindi}
वह लोग ये चाहते हैं कि अगर तुम नरमी एख्तेयार करो तो वह भी नरम हो जाएँ
\end{hindi}}
\flushright{\begin{Arabic}
\quranayah[68][10]
\end{Arabic}}
\flushleft{\begin{hindi}
और तुम (कहीं) ऐसे के कहने में न आना जो बहुत क़समें खाता ज़लील औक़ात ऐबजू
\end{hindi}}
\flushright{\begin{Arabic}
\quranayah[68][11]
\end{Arabic}}
\flushleft{\begin{hindi}
जो आला दर्जे का चुग़लख़ोर माल का बहुत बख़ील
\end{hindi}}
\flushright{\begin{Arabic}
\quranayah[68][12]
\end{Arabic}}
\flushleft{\begin{hindi}
हद से बढ़ने वाला गुनेहगार तुन्द मिजाज़
\end{hindi}}
\flushright{\begin{Arabic}
\quranayah[68][13]
\end{Arabic}}
\flushleft{\begin{hindi}
और उसके अलावा बदज़ात (हरमज़ादा) भी है
\end{hindi}}
\flushright{\begin{Arabic}
\quranayah[68][14]
\end{Arabic}}
\flushleft{\begin{hindi}
चूँकि माल बहुत से बेटे रखता है
\end{hindi}}
\flushright{\begin{Arabic}
\quranayah[68][15]
\end{Arabic}}
\flushleft{\begin{hindi}
जब उसके सामने हमारी आयतें पढ़ी जाती हैं तो बोल उठता है कि ये तो अगलों के अफ़साने हैं
\end{hindi}}
\flushright{\begin{Arabic}
\quranayah[68][16]
\end{Arabic}}
\flushleft{\begin{hindi}
हम अनक़रीब इसकी नाक पर दाग़ लगाएँगे
\end{hindi}}
\flushright{\begin{Arabic}
\quranayah[68][17]
\end{Arabic}}
\flushleft{\begin{hindi}
जिस तरह हमने एक बाग़ वालों का इम्तेहान लिया था उसी तरह उनका इम्तेहान लिया जब उन्होने क़समें खा खाकर कहा कि सुबह होते हम उसका मेवा ज़रूर तोड़ डालेंगे
\end{hindi}}
\flushright{\begin{Arabic}
\quranayah[68][18]
\end{Arabic}}
\flushleft{\begin{hindi}
और इन्शाअल्लाह न कहा
\end{hindi}}
\flushright{\begin{Arabic}
\quranayah[68][19]
\end{Arabic}}
\flushleft{\begin{hindi}
तो ये लोग पड़े सो ही रहे थे कि तुम्हारे परवरदिगार की तरफ से (रातों रात) एक बला चक्कर लगा गयी
\end{hindi}}
\flushright{\begin{Arabic}
\quranayah[68][20]
\end{Arabic}}
\flushleft{\begin{hindi}
तो वह (सारा बाग़ जलकर) ऐसा हो गया जैसे बहुत काली रात
\end{hindi}}
\flushright{\begin{Arabic}
\quranayah[68][21]
\end{Arabic}}
\flushleft{\begin{hindi}
फिर ये लोग नूर के तड़के लगे बाहम गुल मचाने
\end{hindi}}
\flushright{\begin{Arabic}
\quranayah[68][22]
\end{Arabic}}
\flushleft{\begin{hindi}
कि अगर तुमको फल तोड़ना है तो अपने बाग़ में सवेरे से चलो
\end{hindi}}
\flushright{\begin{Arabic}
\quranayah[68][23]
\end{Arabic}}
\flushleft{\begin{hindi}
ग़रज़ वह लोग चले और आपस में चुपके चुपके कहते जाते थे
\end{hindi}}
\flushright{\begin{Arabic}
\quranayah[68][24]
\end{Arabic}}
\flushleft{\begin{hindi}
कि आज यहाँ तुम्हारे पास कोई फ़क़ीर न आने पाए
\end{hindi}}
\flushright{\begin{Arabic}
\quranayah[68][25]
\end{Arabic}}
\flushleft{\begin{hindi}
तो वह लोग रोक थाम के एहतमाम के साथ फल तोड़ने की ठाने हुए सवेरे ही जा पहुँचे
\end{hindi}}
\flushright{\begin{Arabic}
\quranayah[68][26]
\end{Arabic}}
\flushleft{\begin{hindi}
फिर जब उसे (जला हुआ सियाह) देखा तो कहने लगे हम लोग भटक गए
\end{hindi}}
\flushright{\begin{Arabic}
\quranayah[68][27]
\end{Arabic}}
\flushleft{\begin{hindi}
(ये हमारा बाग़ नहीं फिर ये सोचकर बोले) बात ये है कि हम लोग बड़े बदनसीब हैं
\end{hindi}}
\flushright{\begin{Arabic}
\quranayah[68][28]
\end{Arabic}}
\flushleft{\begin{hindi}
जो उनमें से मुनसिफ़ मिजाज़ था कहने लगा क्यों मैंने तुमसे नहीं कहा था कि तुम लोग (ख़ुदा की) तसबीह क्यों नहीं करते
\end{hindi}}
\flushright{\begin{Arabic}
\quranayah[68][29]
\end{Arabic}}
\flushleft{\begin{hindi}
वह बोले हमारा परवरदिगार पाक है बेशक हमीं ही कुसूरवार हैं
\end{hindi}}
\flushright{\begin{Arabic}
\quranayah[68][30]
\end{Arabic}}
\flushleft{\begin{hindi}
फिर लगे एक दूसरे के मुँह दर मुँह मलामत करने
\end{hindi}}
\flushright{\begin{Arabic}
\quranayah[68][31]
\end{Arabic}}
\flushleft{\begin{hindi}
(आख़िर) सबने इक़रार किया कि हाए अफसोस बेशक हम ही ख़ुद सरकश थे
\end{hindi}}
\flushright{\begin{Arabic}
\quranayah[68][32]
\end{Arabic}}
\flushleft{\begin{hindi}
उम्मीद है कि हमारा परवरदिगार हमें इससे बेहतर बाग़ इनायत फ़रमाए हम अपने परवरदिगार की तरफ रूजू करते हैं
\end{hindi}}
\flushright{\begin{Arabic}
\quranayah[68][33]
\end{Arabic}}
\flushleft{\begin{hindi}
(देखो) यूँ अज़ाब होता है और आख़ेरत का अज़ाब तो इससे कहीं बढ़ कर है अगर ये लोग समझते हों
\end{hindi}}
\flushright{\begin{Arabic}
\quranayah[68][34]
\end{Arabic}}
\flushleft{\begin{hindi}
बेशक परहेज़गार लोग अपने परवरदिगार के यहाँ ऐशो आराम के बाग़ों में होंगे
\end{hindi}}
\flushright{\begin{Arabic}
\quranayah[68][35]
\end{Arabic}}
\flushleft{\begin{hindi}
तो क्या हम फरमाबरदारों को नाफ़रमानो के बराबर कर देंगे
\end{hindi}}
\flushright{\begin{Arabic}
\quranayah[68][36]
\end{Arabic}}
\flushleft{\begin{hindi}
(हरगिज़ नहीं) तुम्हें क्या हो गया है तुम तुम कैसा हुक्म लगाते हो
\end{hindi}}
\flushright{\begin{Arabic}
\quranayah[68][37]
\end{Arabic}}
\flushleft{\begin{hindi}
या तुम्हारे पास कोई ईमानी किताब है जिसमें तुम पढ़ लेते हो
\end{hindi}}
\flushright{\begin{Arabic}
\quranayah[68][38]
\end{Arabic}}
\flushleft{\begin{hindi}
कि जो चीज़ पसन्द करोगे तुम को वहाँ ज़रूर मिलेगी
\end{hindi}}
\flushright{\begin{Arabic}
\quranayah[68][39]
\end{Arabic}}
\flushleft{\begin{hindi}
या तुमने हमसे क़समें ले रखी हैं जो रोज़े क़यामत तक चली जाएगी कि जो कुछ तुम हुक्म दोगे वही तुम्हारे लिए ज़रूर हाज़िर होगा
\end{hindi}}
\flushright{\begin{Arabic}
\quranayah[68][40]
\end{Arabic}}
\flushleft{\begin{hindi}
उनसे पूछो तो कि उनमें इसका कौन ज़िम्मेदार है
\end{hindi}}
\flushright{\begin{Arabic}
\quranayah[68][41]
\end{Arabic}}
\flushleft{\begin{hindi}
या (इस बाब में) उनके और लोग भी शरीक हैं तो अगर ये लोग सच्चे हैं तो अपने शरीकों को सामने लाएँ
\end{hindi}}
\flushright{\begin{Arabic}
\quranayah[68][42]
\end{Arabic}}
\flushleft{\begin{hindi}
जिस दिन पिंडली खोल दी जाए और (काफ़िर) लोग सजदे के लिए बुलाए जाएँगे तो (सजदा) न कर सकेंगे
\end{hindi}}
\flushright{\begin{Arabic}
\quranayah[68][43]
\end{Arabic}}
\flushleft{\begin{hindi}
उनकी ऑंखें झुकी हुई होंगी रूसवाई उन पर छाई होगी और (दुनिया में) ये लोग सजदे के लिए बुलाए जाते और हटटे कटटे तन्दरूस्त थे
\end{hindi}}
\flushright{\begin{Arabic}
\quranayah[68][44]
\end{Arabic}}
\flushleft{\begin{hindi}
तो मुझे उस कलाम के झुठलाने वाले से समझ लेने दो हम उनको आहिस्ता आहिस्ता इस तरह पकड़ लेंगे कि उनको ख़बर भी न होगी
\end{hindi}}
\flushright{\begin{Arabic}
\quranayah[68][45]
\end{Arabic}}
\flushleft{\begin{hindi}
और मैं उनको मोहलत दिये जाता हूँ बेशक मेरी तदबीर मज़बूत है
\end{hindi}}
\flushright{\begin{Arabic}
\quranayah[68][46]
\end{Arabic}}
\flushleft{\begin{hindi}
(ऐ रसूल) क्या तुम उनसे (तबलीग़े रिसालत का) कुछ सिला माँगते हो कि उन पर तावान का बोझ पड़ रहा है
\end{hindi}}
\flushright{\begin{Arabic}
\quranayah[68][47]
\end{Arabic}}
\flushleft{\begin{hindi}
या उनके इस ग़ैब (की ख़बर) है कि ये लोग लिख लिया करते हैं
\end{hindi}}
\flushright{\begin{Arabic}
\quranayah[68][48]
\end{Arabic}}
\flushleft{\begin{hindi}
तो तुम अपने परवरदिगार के हुक्म के इन्तेज़ार में सब्र करो और मछली (का निवाला होने) वाले (यूनुस) के ऐसे न हो जाओ कि जब वह ग़ुस्से में भरे हुए थे और अपने परवरदिगार को पुकारा
\end{hindi}}
\flushright{\begin{Arabic}
\quranayah[68][49]
\end{Arabic}}
\flushleft{\begin{hindi}
अगर तुम्हारे परवरदिगार की मेहरबानी उनकी यावरी न करती तो चटियल मैदान में डाल दिए जाते और उनका बुरा हाल होता
\end{hindi}}
\flushright{\begin{Arabic}
\quranayah[68][50]
\end{Arabic}}
\flushleft{\begin{hindi}
तो उनके परवरदिगार ने उनको बरगुज़ीदा करके नेकोकारों से बना दिया
\end{hindi}}
\flushright{\begin{Arabic}
\quranayah[68][51]
\end{Arabic}}
\flushleft{\begin{hindi}
और कुफ्फ़ार जब क़ुरान को सुनते हैं तो मालूम होता है कि ये लोग तुम्हें घूर घूर कर (राह रास्त से) ज़रूर फिसला देंगे
\end{hindi}}
\flushright{\begin{Arabic}
\quranayah[68][52]
\end{Arabic}}
\flushleft{\begin{hindi}
और कहते हैं कि ये तो सिड़ी हैं और ये (क़ुरान) तो सारे जहाँन की नसीहत है
\end{hindi}}
\chapter{Al-Haqqah (The Sure Truth)}
\begin{Arabic}
\Huge{\centerline{\basmalah}}\end{Arabic}
\flushright{\begin{Arabic}
\quranayah[69][1]
\end{Arabic}}
\flushleft{\begin{hindi}
सच मुच होने वाली (क़यामत)
\end{hindi}}
\flushright{\begin{Arabic}
\quranayah[69][2]
\end{Arabic}}
\flushleft{\begin{hindi}
और सच मुच होने वाली क्या चीज़ है
\end{hindi}}
\flushright{\begin{Arabic}
\quranayah[69][3]
\end{Arabic}}
\flushleft{\begin{hindi}
और तुम्हें क्या मालूम कि वह सच मुच होने वाली क्या है
\end{hindi}}
\flushright{\begin{Arabic}
\quranayah[69][4]
\end{Arabic}}
\flushleft{\begin{hindi}
(वही) खड़ खड़ाने वाली (जिस) को आद व समूद ने झुठलाया
\end{hindi}}
\flushright{\begin{Arabic}
\quranayah[69][5]
\end{Arabic}}
\flushleft{\begin{hindi}
ग़रज़ समूद तो चिंघाड़ से हलाक कर दिए गए
\end{hindi}}
\flushright{\begin{Arabic}
\quranayah[69][6]
\end{Arabic}}
\flushleft{\begin{hindi}
रहे आद तो वह बहुत शदीद तेज़ ऑंधी से हलाक कर दिए गए
\end{hindi}}
\flushright{\begin{Arabic}
\quranayah[69][7]
\end{Arabic}}
\flushleft{\begin{hindi}
ख़ुदा ने उसे सात रात और आठ दिन लगाकर उन पर चलाया तो लोगों को इस तरह ढहे (मुर्दे) पड़े देखता कि गोया वह खजूरों के खोखले तने हैं
\end{hindi}}
\flushright{\begin{Arabic}
\quranayah[69][8]
\end{Arabic}}
\flushleft{\begin{hindi}
तू क्या इनमें से किसी को भी बचा खुचा देखता है
\end{hindi}}
\flushright{\begin{Arabic}
\quranayah[69][9]
\end{Arabic}}
\flushleft{\begin{hindi}
और फिरऔन और जो लोग उससे पहले थे और वह लोग (क़ौमे लूत) जो उलटी हुई बस्तियों के रहने वाले थे सब गुनाह के काम करते थे
\end{hindi}}
\flushright{\begin{Arabic}
\quranayah[69][10]
\end{Arabic}}
\flushleft{\begin{hindi}
तो उन लोगों ने अपने परवरदिगार के रसूल की नाफ़रमानी की तो ख़ुदा ने भी उनकी बड़ी सख्ती से ले दे कर डाली
\end{hindi}}
\flushright{\begin{Arabic}
\quranayah[69][11]
\end{Arabic}}
\flushleft{\begin{hindi}
जब पानी चढ़ने लगा तो हमने तुमको कशती पर सवार किया
\end{hindi}}
\flushright{\begin{Arabic}
\quranayah[69][12]
\end{Arabic}}
\flushleft{\begin{hindi}
ताकि हम उसे तुम्हारे लिए यादगार बनाएं और उसे याद रखने वाले कान सुनकर याद रखें
\end{hindi}}
\flushright{\begin{Arabic}
\quranayah[69][13]
\end{Arabic}}
\flushleft{\begin{hindi}
फिर जब सूर में एक (बार) फूँक मार दी जाएगी
\end{hindi}}
\flushright{\begin{Arabic}
\quranayah[69][14]
\end{Arabic}}
\flushleft{\begin{hindi}
और ज़मीन और पहाड़ उठाकर एक बारगी (टकरा कर) रेज़ा रेज़ा कर दिए जाएँगे तो उस रोज़ क़यामत आ ही जाएगी
\end{hindi}}
\flushright{\begin{Arabic}
\quranayah[69][15]
\end{Arabic}}
\flushleft{\begin{hindi}
और आसमान फट जाएगा
\end{hindi}}
\flushright{\begin{Arabic}
\quranayah[69][16]
\end{Arabic}}
\flushleft{\begin{hindi}
तो वह उस दिन बहुत फुस फुसा होगा और फ़रिश्ते उनके किनारे पर होंगे
\end{hindi}}
\flushright{\begin{Arabic}
\quranayah[69][17]
\end{Arabic}}
\flushleft{\begin{hindi}
और तुम्हारे परवरदिगार के अर्श को उस दिन आठ फ़रिश्ते अपने सरों पर उठाए होंगे
\end{hindi}}
\flushright{\begin{Arabic}
\quranayah[69][18]
\end{Arabic}}
\flushleft{\begin{hindi}
उस दिन तुम सब के सब (ख़ुदा के सामने) पेश किए जाओगे और तुम्हारी कोई पोशीदा बात छुपी न रहेगी
\end{hindi}}
\flushright{\begin{Arabic}
\quranayah[69][19]
\end{Arabic}}
\flushleft{\begin{hindi}
तो जिसको (उसका नामए आमाल) दाहिने हाथ में दिया जाएगा तो वह (लोगो से) कहेगा लीजिए मेरा नामए आमाल पढ़िए
\end{hindi}}
\flushright{\begin{Arabic}
\quranayah[69][20]
\end{Arabic}}
\flushleft{\begin{hindi}
तो मैं तो जानता था कि मुझे मेरा हिसाब (किताब) ज़रूर मिलेगा
\end{hindi}}
\flushright{\begin{Arabic}
\quranayah[69][21]
\end{Arabic}}
\flushleft{\begin{hindi}
फिर वह दिल पसन्द ऐश में होगा
\end{hindi}}
\flushright{\begin{Arabic}
\quranayah[69][22]
\end{Arabic}}
\flushleft{\begin{hindi}
बड़े आलीशान बाग़ में
\end{hindi}}
\flushright{\begin{Arabic}
\quranayah[69][23]
\end{Arabic}}
\flushleft{\begin{hindi}
जिनके फल बहुत झुके हुए क़रीब होंगे
\end{hindi}}
\flushright{\begin{Arabic}
\quranayah[69][24]
\end{Arabic}}
\flushleft{\begin{hindi}
जो कारगुज़ारियाँ तुम गुज़िशता अय्याम में करके आगे भेज चुके हो उसके सिले में मज़े से खाओ पियो
\end{hindi}}
\flushright{\begin{Arabic}
\quranayah[69][25]
\end{Arabic}}
\flushleft{\begin{hindi}
और जिसका नामए आमाल उनके बाएँ हाथ में दिया जाएगा तो वह कहेगा ऐ काश मुझे मेरा नामए अमल न दिया जाता
\end{hindi}}
\flushright{\begin{Arabic}
\quranayah[69][26]
\end{Arabic}}
\flushleft{\begin{hindi}
और मुझे न मालूल होता कि मेरा हिसाब क्या है
\end{hindi}}
\flushright{\begin{Arabic}
\quranayah[69][27]
\end{Arabic}}
\flushleft{\begin{hindi}
ऐ काश मौत ने (हमेशा के लिए मेरा) काम तमाम कर दिया होता
\end{hindi}}
\flushright{\begin{Arabic}
\quranayah[69][28]
\end{Arabic}}
\flushleft{\begin{hindi}
(अफ़सोस) मेरा माल मेरे कुछ भी काम न आया
\end{hindi}}
\flushright{\begin{Arabic}
\quranayah[69][29]
\end{Arabic}}
\flushleft{\begin{hindi}
(हाए) मेरी सल्तनत ख़ाक में मिल गयी (फिर हुक्म होगा)
\end{hindi}}
\flushright{\begin{Arabic}
\quranayah[69][30]
\end{Arabic}}
\flushleft{\begin{hindi}
इसे गिरफ्तार करके तौक़ पहना दो
\end{hindi}}
\flushright{\begin{Arabic}
\quranayah[69][31]
\end{Arabic}}
\flushleft{\begin{hindi}
फिर इसे जहन्नुम में झोंक दो,
\end{hindi}}
\flushright{\begin{Arabic}
\quranayah[69][32]
\end{Arabic}}
\flushleft{\begin{hindi}
फिर एक ज़ंजीर में जिसकी नाप सत्तर गज़ की है उसे ख़ूब जकड़ दो
\end{hindi}}
\flushright{\begin{Arabic}
\quranayah[69][33]
\end{Arabic}}
\flushleft{\begin{hindi}
(क्यों कि) ये न तो बुज़ुर्ग ख़ुदा ही पर ईमान लाता था और न मोहताज के खिलाने पर आमादा (लोगों को) करता था
\end{hindi}}
\flushright{\begin{Arabic}
\quranayah[69][34]
\end{Arabic}}
\flushleft{\begin{hindi}
तो आज न उसका कोई ग़मख्वार है
\end{hindi}}
\flushright{\begin{Arabic}
\quranayah[69][35]
\end{Arabic}}
\flushleft{\begin{hindi}
और न पीप के सिवा (उसके लिए) कुछ खाना है
\end{hindi}}
\flushright{\begin{Arabic}
\quranayah[69][36]
\end{Arabic}}
\flushleft{\begin{hindi}
जिसको गुनेहगारों के सिवा कोई नहीं खाएगा
\end{hindi}}
\flushright{\begin{Arabic}
\quranayah[69][37]
\end{Arabic}}
\flushleft{\begin{hindi}
तो मुझे उन चीज़ों की क़सम है
\end{hindi}}
\flushright{\begin{Arabic}
\quranayah[69][38]
\end{Arabic}}
\flushleft{\begin{hindi}
जो तुम्हें दिखाई देती हैं
\end{hindi}}
\flushright{\begin{Arabic}
\quranayah[69][39]
\end{Arabic}}
\flushleft{\begin{hindi}
और जो तुम्हें नहीं सुझाई देती कि बेशक ये (क़ुरान)
\end{hindi}}
\flushright{\begin{Arabic}
\quranayah[69][40]
\end{Arabic}}
\flushleft{\begin{hindi}
एक मोअज़िज़ फरिश्ते का लाया हुआ पैग़ाम है
\end{hindi}}
\flushright{\begin{Arabic}
\quranayah[69][41]
\end{Arabic}}
\flushleft{\begin{hindi}
और ये किसी शायर की तुक बन्दी नहीं तुम लोग तो बहुत कम ईमान लाते हो
\end{hindi}}
\flushright{\begin{Arabic}
\quranayah[69][42]
\end{Arabic}}
\flushleft{\begin{hindi}
और न किसी काहिन की (ख्याली) बात है तुम लोग तो बहुत कम ग़ौर करते हो
\end{hindi}}
\flushright{\begin{Arabic}
\quranayah[69][43]
\end{Arabic}}
\flushleft{\begin{hindi}
सारे जहाँन के परवरदिगार का नाज़िल किया हुआ (क़लाम) है
\end{hindi}}
\flushright{\begin{Arabic}
\quranayah[69][44]
\end{Arabic}}
\flushleft{\begin{hindi}
अगर रसूल हमारी निस्बत कोई झूठ बात बना लाते
\end{hindi}}
\flushright{\begin{Arabic}
\quranayah[69][45]
\end{Arabic}}
\flushleft{\begin{hindi}
तो हम उनका दाहिना हाथ पकड़ लेते
\end{hindi}}
\flushright{\begin{Arabic}
\quranayah[69][46]
\end{Arabic}}
\flushleft{\begin{hindi}
फिर हम ज़रूर उनकी गर्दन उड़ा देते
\end{hindi}}
\flushright{\begin{Arabic}
\quranayah[69][47]
\end{Arabic}}
\flushleft{\begin{hindi}
तो तुममें से कोई उनसे (मुझे रोक न सकता)
\end{hindi}}
\flushright{\begin{Arabic}
\quranayah[69][48]
\end{Arabic}}
\flushleft{\begin{hindi}
ये तो परहेज़गारों के लिए नसीहत है
\end{hindi}}
\flushright{\begin{Arabic}
\quranayah[69][49]
\end{Arabic}}
\flushleft{\begin{hindi}
और हम ख़ूब जानते हैं कि तुम में से कुछ लोग (इसके) झुठलाने वाले हैं
\end{hindi}}
\flushright{\begin{Arabic}
\quranayah[69][50]
\end{Arabic}}
\flushleft{\begin{hindi}
और इसमें शक़ नहीं कि ये काफ़िरों की हसरत का बाएस है
\end{hindi}}
\flushright{\begin{Arabic}
\quranayah[69][51]
\end{Arabic}}
\flushleft{\begin{hindi}
और इसमें शक़ नहीं कि ये यक़ीनन बरहक़ है
\end{hindi}}
\flushright{\begin{Arabic}
\quranayah[69][52]
\end{Arabic}}
\flushleft{\begin{hindi}
तो तुम अपने परवरदिगार की तसबीह करो
\end{hindi}}
\chapter{Al-Ma'arij (The Ways of Ascent)}
\begin{Arabic}
\Huge{\centerline{\basmalah}}\end{Arabic}
\flushright{\begin{Arabic}
\quranayah[70][1]
\end{Arabic}}
\flushleft{\begin{hindi}
एक माँगने वाले ने काफिरों के लिए होकर रहने वाले अज़ाब को माँगा
\end{hindi}}
\flushright{\begin{Arabic}
\quranayah[70][2]
\end{Arabic}}
\flushleft{\begin{hindi}
जिसको कोई टाल नहीं सकता
\end{hindi}}
\flushright{\begin{Arabic}
\quranayah[70][3]
\end{Arabic}}
\flushleft{\begin{hindi}
जो दर्जे वाले ख़ुदा की तरफ से (होने वाला) था
\end{hindi}}
\flushright{\begin{Arabic}
\quranayah[70][4]
\end{Arabic}}
\flushleft{\begin{hindi}
जिसकी तरफ फ़रिश्ते और रूहुल अमीन चढ़ते हैं (और ये) एक दिन में इतनी मुसाफ़त तय करते हैं जिसका अन्दाज़ा पचास हज़ार बरस का होगा
\end{hindi}}
\flushright{\begin{Arabic}
\quranayah[70][5]
\end{Arabic}}
\flushleft{\begin{hindi}
तो तुम अच्छी तरह इन तक़लीफों को बरदाश्त करते रहो
\end{hindi}}
\flushright{\begin{Arabic}
\quranayah[70][6]
\end{Arabic}}
\flushleft{\begin{hindi}
वह (क़यामत) उनकी निगाह में बहुत दूर है
\end{hindi}}
\flushright{\begin{Arabic}
\quranayah[70][7]
\end{Arabic}}
\flushleft{\begin{hindi}
और हमारी नज़र में नज़दीक है
\end{hindi}}
\flushright{\begin{Arabic}
\quranayah[70][8]
\end{Arabic}}
\flushleft{\begin{hindi}
जिस दिन आसमान पिघले हुए ताँबे का सा हो जाएगा
\end{hindi}}
\flushright{\begin{Arabic}
\quranayah[70][9]
\end{Arabic}}
\flushleft{\begin{hindi}
और पहाड़ धुनके हुए ऊन का सा
\end{hindi}}
\flushright{\begin{Arabic}
\quranayah[70][10]
\end{Arabic}}
\flushleft{\begin{hindi}
बावजूद कि एक दूसरे को देखते होंगे
\end{hindi}}
\flushright{\begin{Arabic}
\quranayah[70][11]
\end{Arabic}}
\flushleft{\begin{hindi}
कोई किसी दोस्त को न पूछेगा गुनेहगार तो आरज़ू करेगा कि काश उस दिन के अज़ाब के बदले उसके बेटों
\end{hindi}}
\flushright{\begin{Arabic}
\quranayah[70][12]
\end{Arabic}}
\flushleft{\begin{hindi}
और उसकी बीवी और उसके भाई
\end{hindi}}
\flushright{\begin{Arabic}
\quranayah[70][13]
\end{Arabic}}
\flushleft{\begin{hindi}
और उसके कुनबे को जिसमें वह रहता था
\end{hindi}}
\flushright{\begin{Arabic}
\quranayah[70][14]
\end{Arabic}}
\flushleft{\begin{hindi}
और जितने आदमी ज़मीन पर हैं सब को ले ले और उसको छुटकारा दे दें
\end{hindi}}
\flushright{\begin{Arabic}
\quranayah[70][15]
\end{Arabic}}
\flushleft{\begin{hindi}
(मगर) ये हरगिज़ न होगा
\end{hindi}}
\flushright{\begin{Arabic}
\quranayah[70][16]
\end{Arabic}}
\flushleft{\begin{hindi}
जहन्नुम की वह भड़कती आग है कि खाल उधेड़ कर रख देगी
\end{hindi}}
\flushright{\begin{Arabic}
\quranayah[70][17]
\end{Arabic}}
\flushleft{\begin{hindi}
(और) उन लोगों को अपनी तरफ बुलाती होगी
\end{hindi}}
\flushright{\begin{Arabic}
\quranayah[70][18]
\end{Arabic}}
\flushleft{\begin{hindi}
जिन्होंने (दीन से) पीठ फेरी और मुँह मोड़ा और (माल जमा किया)
\end{hindi}}
\flushright{\begin{Arabic}
\quranayah[70][19]
\end{Arabic}}
\flushleft{\begin{hindi}
और बन्द कर रखा बेशक इन्सान बड़ा लालची पैदा हुआ है
\end{hindi}}
\flushright{\begin{Arabic}
\quranayah[70][20]
\end{Arabic}}
\flushleft{\begin{hindi}
जब उसे तक़लीफ छू भी गयी तो घबरा गया
\end{hindi}}
\flushright{\begin{Arabic}
\quranayah[70][21]
\end{Arabic}}
\flushleft{\begin{hindi}
और जब उसे ज़रा फराग़ी हासिल हुई तो बख़ील बन बैठा
\end{hindi}}
\flushright{\begin{Arabic}
\quranayah[70][22]
\end{Arabic}}
\flushleft{\begin{hindi}
मगर जो लोग नमाज़ पढ़ते हैं
\end{hindi}}
\flushright{\begin{Arabic}
\quranayah[70][23]
\end{Arabic}}
\flushleft{\begin{hindi}
जो अपनी नमाज़ का इल्तज़ाम रखते हैं
\end{hindi}}
\flushright{\begin{Arabic}
\quranayah[70][24]
\end{Arabic}}
\flushleft{\begin{hindi}
और जिनके माल में माँगने वाले और न माँगने वाले के
\end{hindi}}
\flushright{\begin{Arabic}
\quranayah[70][25]
\end{Arabic}}
\flushleft{\begin{hindi}
लिए एक मुक़र्रर हिस्सा है
\end{hindi}}
\flushright{\begin{Arabic}
\quranayah[70][26]
\end{Arabic}}
\flushleft{\begin{hindi}
और जो लोग रोज़े जज़ा की तस्दीक़ करते हैं
\end{hindi}}
\flushright{\begin{Arabic}
\quranayah[70][27]
\end{Arabic}}
\flushleft{\begin{hindi}
और जो लोग अपने परवरदिगार के अज़ाब से डरते रहते हैं
\end{hindi}}
\flushright{\begin{Arabic}
\quranayah[70][28]
\end{Arabic}}
\flushleft{\begin{hindi}
बेशक उनको परवरदिगार के अज़ाब से बेख़ौफ न होना चाहिए
\end{hindi}}
\flushright{\begin{Arabic}
\quranayah[70][29]
\end{Arabic}}
\flushleft{\begin{hindi}
और जो लोग अपनी शर्मगाहों को अपनी बीवियों और अपनी लौन्डियों के सिवा से हिफाज़त करते हैं
\end{hindi}}
\flushright{\begin{Arabic}
\quranayah[70][30]
\end{Arabic}}
\flushleft{\begin{hindi}
तो इन लोगों की हरगिज़ मलामत न की जाएगी
\end{hindi}}
\flushright{\begin{Arabic}
\quranayah[70][31]
\end{Arabic}}
\flushleft{\begin{hindi}
तो जो लोग उनके सिवा और के ख़ास्तगार हों तो यही लोग हद से बढ़ जाने वाले हैं
\end{hindi}}
\flushright{\begin{Arabic}
\quranayah[70][32]
\end{Arabic}}
\flushleft{\begin{hindi}
और जो लोग अपनी अमानतों और अहदों का लेहाज़ रखते हैं
\end{hindi}}
\flushright{\begin{Arabic}
\quranayah[70][33]
\end{Arabic}}
\flushleft{\begin{hindi}
और जो लोग अपनी यहादतों पर क़ायम रहते हैं
\end{hindi}}
\flushright{\begin{Arabic}
\quranayah[70][34]
\end{Arabic}}
\flushleft{\begin{hindi}
और जो लोग अपनी नमाज़ो का ख्याल रखते हैं
\end{hindi}}
\flushright{\begin{Arabic}
\quranayah[70][35]
\end{Arabic}}
\flushleft{\begin{hindi}
यही लोग बेहिश्त के बाग़ों में इज्ज़त से रहेंगे
\end{hindi}}
\flushright{\begin{Arabic}
\quranayah[70][36]
\end{Arabic}}
\flushleft{\begin{hindi}
तो (ऐ रसूल) काफिरों को क्या हो गया है
\end{hindi}}
\flushright{\begin{Arabic}
\quranayah[70][37]
\end{Arabic}}
\flushleft{\begin{hindi}
कि तुम्हारे पास गिरोह गिरोह दाहिने से बाएँ से दौड़े चले आ रहे हैं
\end{hindi}}
\flushright{\begin{Arabic}
\quranayah[70][38]
\end{Arabic}}
\flushleft{\begin{hindi}
क्या इनमें से हर शख़्श इस का मुतमइनी है कि चैन के बाग़ (बेहिश्त) में दाख़िल होगा
\end{hindi}}
\flushright{\begin{Arabic}
\quranayah[70][39]
\end{Arabic}}
\flushleft{\begin{hindi}
हरगिज़ नहीं हमने उनको जिस (गन्दी) चीज़ से पैदा किया ये लोग जानते हैं
\end{hindi}}
\flushright{\begin{Arabic}
\quranayah[70][40]
\end{Arabic}}
\flushleft{\begin{hindi}
तो मैं मशरिकों और मग़रिबों के परवरदिगार की क़सम खाता हूँ कि हम ज़रूर इस बात की कुदरत रखते हैं
\end{hindi}}
\flushright{\begin{Arabic}
\quranayah[70][41]
\end{Arabic}}
\flushleft{\begin{hindi}
कि उनके बदले उनसे बेहतर लोग ला (बसाएँ) और हम आजिज़ नहीं हैं
\end{hindi}}
\flushright{\begin{Arabic}
\quranayah[70][42]
\end{Arabic}}
\flushleft{\begin{hindi}
तो तुम उनको छोड़ दो कि बातिल में पड़े खेलते रहें यहाँ तक कि जिस दिन का उनसे वायदा किया जाता है उनके सामने आ मौजूद हो
\end{hindi}}
\flushright{\begin{Arabic}
\quranayah[70][43]
\end{Arabic}}
\flushleft{\begin{hindi}
उसी दिन ये लोग कब्रों से निकल कर इस तरह दौड़ेंगे गोया वह किसी झन्डे की तरफ दौड़े चले जाते हैं
\end{hindi}}
\flushright{\begin{Arabic}
\quranayah[70][44]
\end{Arabic}}
\flushleft{\begin{hindi}
(निदामत से) उनकी ऑंखें झुकी होंगी उन पर रूसवाई छाई हुई होगी ये वही दिन है जिसका उनसे वायदा किया जाता था
\end{hindi}}
\chapter{Nuh (Noah)}
\begin{Arabic}
\Huge{\centerline{\basmalah}}\end{Arabic}
\flushright{\begin{Arabic}
\quranayah[71][1]
\end{Arabic}}
\flushleft{\begin{hindi}
हमने नूह को उसकी क़ौम के पास (पैग़म्बर बनाकर) भेजा कि क़ब्ल उसके कि उनकी क़ौम पर दर्दनाक अज़ाब आए उनको उससे डराओ
\end{hindi}}
\flushright{\begin{Arabic}
\quranayah[71][2]
\end{Arabic}}
\flushleft{\begin{hindi}
तो नूह (अपनी क़ौम से) कहने लगे ऐ मेरी क़ौम मैं तो तुम्हें साफ़ साफ़ डराता (और समझाता) हूँ
\end{hindi}}
\flushright{\begin{Arabic}
\quranayah[71][3]
\end{Arabic}}
\flushleft{\begin{hindi}
कि तुम लोग ख़ुदा की इबादत करो और उसी से डरो और मेरी इताअत करो
\end{hindi}}
\flushright{\begin{Arabic}
\quranayah[71][4]
\end{Arabic}}
\flushleft{\begin{hindi}
ख़ुदा तुम्हारे गुनाह बख्श देगा और तुम्हें (मौत के) मुक़र्रर वक्त तक बाक़ी रखेगा, बेशक जब ख़ुदा का मुक़र्रर किया हुआ वक्त अा जाता है तो पीछे हटाया नहीं जा सकता अगर तुम समझते होते
\end{hindi}}
\flushright{\begin{Arabic}
\quranayah[71][5]
\end{Arabic}}
\flushleft{\begin{hindi}
(जब लोगों ने न माना तो) अर्ज़ की परवरदिगार मैं अपनी क़ौम को (ईमान की तरफ) बुलाता रहा
\end{hindi}}
\flushright{\begin{Arabic}
\quranayah[71][6]
\end{Arabic}}
\flushleft{\begin{hindi}
लेकिन वह मेरे बुलाने से और ज्यादा गुरेज़ ही करते रहे
\end{hindi}}
\flushright{\begin{Arabic}
\quranayah[71][7]
\end{Arabic}}
\flushleft{\begin{hindi}
और मैने जब उनको बुलाया कि (ये तौबा करें और) तू उन्हें माफ कर दे तो उन्होने अपने कानों में उंगलियां दे लीं और मुझसे छिपने को कपड़े ओढ़ लिए और अड़ गए और बहुत शिद्दत से अकड़ बैठे
\end{hindi}}
\flushright{\begin{Arabic}
\quranayah[71][8]
\end{Arabic}}
\flushleft{\begin{hindi}
फिर मैंने उनको बिल एलान बुलाया फिर उनको ज़ाहिर ब ज़ाहिर समझाया
\end{hindi}}
\flushright{\begin{Arabic}
\quranayah[71][9]
\end{Arabic}}
\flushleft{\begin{hindi}
और उनकी पोशीदा भी फ़हमाईश की कि मैंने उनसे कहा
\end{hindi}}
\flushright{\begin{Arabic}
\quranayah[71][10]
\end{Arabic}}
\flushleft{\begin{hindi}
अपने परवरदिगार से मग़फेरत की दुआ माँगो बेशक वह बड़ा बख्शने वाला है
\end{hindi}}
\flushright{\begin{Arabic}
\quranayah[71][11]
\end{Arabic}}
\flushleft{\begin{hindi}
(और) तुम पर आसमान से मूसलाधार पानी बरसाएगा
\end{hindi}}
\flushright{\begin{Arabic}
\quranayah[71][12]
\end{Arabic}}
\flushleft{\begin{hindi}
और माल और औलाद में तरक्क़ी देगा, और तुम्हारे लिए बाग़ बनाएगा, और तुम्हारे लिए नहरें जारी करेगा
\end{hindi}}
\flushright{\begin{Arabic}
\quranayah[71][13]
\end{Arabic}}
\flushleft{\begin{hindi}
तुम्हें क्या हो गया है कि तुम ख़ुदा की अज़मत का ज़रा भी ख्याल नहीं करते
\end{hindi}}
\flushright{\begin{Arabic}
\quranayah[71][14]
\end{Arabic}}
\flushleft{\begin{hindi}
हालॉकि उसी ने तुमको तरह तरह का पैदा किया
\end{hindi}}
\flushright{\begin{Arabic}
\quranayah[71][15]
\end{Arabic}}
\flushleft{\begin{hindi}
क्या तुमने ग़ौर नहीं किया कि ख़ुदा ने सात आसमान ऊपर तलें क्यों कर बनाए
\end{hindi}}
\flushright{\begin{Arabic}
\quranayah[71][16]
\end{Arabic}}
\flushleft{\begin{hindi}
और उसी ने उसमें चाँद को नूर बनाया और सूरज को रौशन चिराग़ बना दिया
\end{hindi}}
\flushright{\begin{Arabic}
\quranayah[71][17]
\end{Arabic}}
\flushleft{\begin{hindi}
और ख़ुदा ही तुमको ज़मीन से पैदा किया
\end{hindi}}
\flushright{\begin{Arabic}
\quranayah[71][18]
\end{Arabic}}
\flushleft{\begin{hindi}
फिर तुमको उसी में दोबारा ले जाएगा और (क़यामत में उसी से) निकाल कर खड़ा करेगा
\end{hindi}}
\flushright{\begin{Arabic}
\quranayah[71][19]
\end{Arabic}}
\flushleft{\begin{hindi}
और ख़ुदा ही ने ज़मीन को तुम्हारे लिए फ़र्श बनाया
\end{hindi}}
\flushright{\begin{Arabic}
\quranayah[71][20]
\end{Arabic}}
\flushleft{\begin{hindi}
ताकि तुम उसके बड़े बड़े कुशादा रास्तों में चलो फिरो
\end{hindi}}
\flushright{\begin{Arabic}
\quranayah[71][21]
\end{Arabic}}
\flushleft{\begin{hindi}
(फिर) नूह ने अर्ज़ की परवरदिगार इन लोगों ने मेरी नाफ़रमानी की उस शख़्श के ताबेदार बन के जिसने उनके माल और औलाद में नुक़सान के सिवा फ़ायदा न पहुँचाया
\end{hindi}}
\flushright{\begin{Arabic}
\quranayah[71][22]
\end{Arabic}}
\flushleft{\begin{hindi}
और उन्होंने (मेरे साथ) बड़ी मक्कारियाँ की
\end{hindi}}
\flushright{\begin{Arabic}
\quranayah[71][23]
\end{Arabic}}
\flushleft{\begin{hindi}
और (उलटे) कहने लगे कि आपने माबूदों को हरगिज़ न छोड़ना और न वद को और सुआ को और न यगूस और यऊक़ व नस्र को छोड़ना
\end{hindi}}
\flushright{\begin{Arabic}
\quranayah[71][24]
\end{Arabic}}
\flushleft{\begin{hindi}
और उन्होंने बहुतेरों को गुमराह कर छोड़ा और तू (उन) ज़ालिमों की गुमराही को और बढ़ा दे
\end{hindi}}
\flushright{\begin{Arabic}
\quranayah[71][25]
\end{Arabic}}
\flushleft{\begin{hindi}
(आख़िर) वह अपने गुनाहों की बदौलत (पहले तो) डुबाए गए फिर जहन्नुम में झोंके गए तो उन लोगों ने ख़ुदा के सिवा किसी को अपना मददगार न पाया
\end{hindi}}
\flushright{\begin{Arabic}
\quranayah[71][26]
\end{Arabic}}
\flushleft{\begin{hindi}
और नूह ने अर्ज़ की परवरदिगार (इन) काफ़िरों में रूए ज़मीन पर किसी को बसा हुआ न रहने दे
\end{hindi}}
\flushright{\begin{Arabic}
\quranayah[71][27]
\end{Arabic}}
\flushleft{\begin{hindi}
क्योंकि अगर तू उनको छोड़ देगा तो ये (फिर) तेरे बन्दों को गुमराह करेंगे और उनकी औलाद भी गुनाहगार और कट्टी काफिर ही होगी
\end{hindi}}
\flushright{\begin{Arabic}
\quranayah[71][28]
\end{Arabic}}
\flushleft{\begin{hindi}
परवरदिगार मुझको और मेरे माँ बाप को और जो मोमिन मेरे घर में आए उनको और तमाम ईमानदार मर्दों और मोमिन औरतों को बख्श दे और (इन) ज़ालिमों की बस तबाही को और ज्यादा कर
\end{hindi}}
\chapter{Al-Jinn (The Jinn)}
\begin{Arabic}
\Huge{\centerline{\basmalah}}\end{Arabic}
\flushright{\begin{Arabic}
\quranayah[72][1]
\end{Arabic}}
\flushleft{\begin{hindi}
(ऐ रसूल लोगों से) कह दो कि मेरे पास 'वही' आयी है कि जिनों की एक जमाअत ने (क़ुरान को) जी लगाकर सुना तो कहने लगे कि हमने एक अजीब क़ुरान सुना है
\end{hindi}}
\flushright{\begin{Arabic}
\quranayah[72][2]
\end{Arabic}}
\flushleft{\begin{hindi}
जो भलाई की राह दिखाता है तो हम उस पर ईमान ले आए और अब तो हम किसी को अपने परवरदिगार का शरीक न बनाएँगे
\end{hindi}}
\flushright{\begin{Arabic}
\quranayah[72][3]
\end{Arabic}}
\flushleft{\begin{hindi}
और ये कि हमारे परवरदिगार की शान बहुत बड़ी है उसने न (किसी को) बीवी बनाया और न बेटा बेटी
\end{hindi}}
\flushright{\begin{Arabic}
\quranayah[72][4]
\end{Arabic}}
\flushleft{\begin{hindi}
और ये कि हममें से बाज़ बेवकूफ ख़ुदा के बारे में हद से ज्यादा लग़ो बातें निकाला करते थे
\end{hindi}}
\flushright{\begin{Arabic}
\quranayah[72][5]
\end{Arabic}}
\flushleft{\begin{hindi}
और ये कि हमारा तो ख्याल था कि आदमी और जिन ख़ुदा की निस्बत झूठी बात नहीं बोल सकते
\end{hindi}}
\flushright{\begin{Arabic}
\quranayah[72][6]
\end{Arabic}}
\flushleft{\begin{hindi}
और ये कि आदमियों में से कुछ लोग जिन्नात में से बाज़ लोगों की पनाह पकड़ा करते थे तो (इससे) उनकी सरकशी और बढ़ गयी
\end{hindi}}
\flushright{\begin{Arabic}
\quranayah[72][7]
\end{Arabic}}
\flushleft{\begin{hindi}
और ये कि जैसा तुम्हारा ख्याल है वैसा उनका भी एतक़ाद था कि ख़ुदा हरगिज़ किसी को दोबारा नहीं ज़िन्दा करेगा
\end{hindi}}
\flushright{\begin{Arabic}
\quranayah[72][8]
\end{Arabic}}
\flushleft{\begin{hindi}
और ये कि हमने आसमान को टटोला तो उसको भी बहुत क़वी निगेहबानों और शोलो से भरा हुआ पाया
\end{hindi}}
\flushright{\begin{Arabic}
\quranayah[72][9]
\end{Arabic}}
\flushleft{\begin{hindi}
और ये कि पहले हम वहाँ बहुत से मक़ामात में (बातें) सुनने के लिए बैठा करते थे मगर अब कोई सुनना चाहे तो अपने लिए शोले तैयार पाएगा
\end{hindi}}
\flushright{\begin{Arabic}
\quranayah[72][10]
\end{Arabic}}
\flushleft{\begin{hindi}
और ये कि हम नहीं समझते कि उससे अहले ज़मीन के हक़ में बुराई मक़सूद है या उनके परवरदिगार ने उनकी भलाई का इरादा किया है
\end{hindi}}
\flushright{\begin{Arabic}
\quranayah[72][11]
\end{Arabic}}
\flushleft{\begin{hindi}
और ये कि हममें से कुछ लोग तो नेकोकार हैं और कुछ लोग और तरह के हम लोगों के भी तो कई तरह के फिरकें हैं
\end{hindi}}
\flushright{\begin{Arabic}
\quranayah[72][12]
\end{Arabic}}
\flushleft{\begin{hindi}
और ये कि हम समझते थे कि हम ज़मीन में (रह कर) ख़ुदा को हरगिज़ हरा नहीं सकते हैं और न भाग कर उसको आजिज़ कर सकते हैं
\end{hindi}}
\flushright{\begin{Arabic}
\quranayah[72][13]
\end{Arabic}}
\flushleft{\begin{hindi}
और ये कि जब हमने हिदायत (की किताब) सुनी तो उन पर ईमान लाए तो जो शख़्श अपने परवरदिगार पर ईमान लाएगा तो उसको न नुक़सान का ख़ौफ़ है और न ज़ुल्म का
\end{hindi}}
\flushright{\begin{Arabic}
\quranayah[72][14]
\end{Arabic}}
\flushleft{\begin{hindi}
और ये कि हम में से कुछ लोग तो फ़रमाबरदार हैं और कुछ लोग नाफ़रमान तो जो लोग फ़रमाबरदार हैं तो वह सीधे रास्ते पर चलें और रहें
\end{hindi}}
\flushright{\begin{Arabic}
\quranayah[72][15]
\end{Arabic}}
\flushleft{\begin{hindi}
नाफरमान तो वह जहन्नुम के कुन्दे बने
\end{hindi}}
\flushright{\begin{Arabic}
\quranayah[72][16]
\end{Arabic}}
\flushleft{\begin{hindi}
और (ऐ रसूल तुम कह दो) कि अगर ये लोग सीधी राह पर क़ायम रहते तो हम ज़रूर उनको अलग़ारों पानी से सेराब करते
\end{hindi}}
\flushright{\begin{Arabic}
\quranayah[72][17]
\end{Arabic}}
\flushleft{\begin{hindi}
ताकि उससे उनकी आज़माईश करें और जो शख़्श अपने परवरदिगार की याद से मुँह मोड़ेगा तो वह उसको सख्त अज़ाब में झोंक देगा
\end{hindi}}
\flushright{\begin{Arabic}
\quranayah[72][18]
\end{Arabic}}
\flushleft{\begin{hindi}
और ये कि मस्जिदें ख़ास ख़ुदा की हैं तो लोगों ख़ुदा के साथ किसी की इबादन न करना
\end{hindi}}
\flushright{\begin{Arabic}
\quranayah[72][19]
\end{Arabic}}
\flushleft{\begin{hindi}
और ये कि जब उसका बन्दा (मोहम्मद) उसकी इबादत को खड़ा होता है तो लोग उसके गिर्द हुजूम करके गिर पड़ते हैं
\end{hindi}}
\flushright{\begin{Arabic}
\quranayah[72][20]
\end{Arabic}}
\flushleft{\begin{hindi}
(ऐ रसूल) तुम कह दो कि मैं तो अपने परवरदिगार की इबादत करता हूँ और उसका किसी को शरीक नहीं बनाता
\end{hindi}}
\flushright{\begin{Arabic}
\quranayah[72][21]
\end{Arabic}}
\flushleft{\begin{hindi}
(ये भी) कह दो कि मैं तुम्हारे हक़ में न बुराई ही का एख्तेयार रखता हूँ और न भलाई का
\end{hindi}}
\flushright{\begin{Arabic}
\quranayah[72][22]
\end{Arabic}}
\flushleft{\begin{hindi}
(ये भी) कह दो कि मुझे ख़ुदा (के अज़ाब) से कोई भी पनाह नहीं दे सकता और न मैं उसके सिवा कहीं पनाह की जगह देखता हूँ
\end{hindi}}
\flushright{\begin{Arabic}
\quranayah[72][23]
\end{Arabic}}
\flushleft{\begin{hindi}
ख़ुदा की तरफ से (एहकाम के) पहुँचा देने और उसके पैग़ामों के सिवा (कुछ नहीं कर सकता) और जिसने ख़ुदा और उसके रसूल की नाफरमानी की तो उसके लिए यक़ीनन जहन्नुम की आग है जिसमें वह हमेशा और अबादुल आबाद तक रहेगा
\end{hindi}}
\flushright{\begin{Arabic}
\quranayah[72][24]
\end{Arabic}}
\flushleft{\begin{hindi}
यहाँ तक कि जब ये लोग उन चीज़ों को देख लेंगे जिनका उनसे वायदा किया जाता है तो उनको मालूम हो जाएगा कि किसके मददगार कमज़ोर और किसका शुमार कम है
\end{hindi}}
\flushright{\begin{Arabic}
\quranayah[72][25]
\end{Arabic}}
\flushleft{\begin{hindi}
(ऐ रसूल) तुम कह दो कि मैं नहीं जानता कि जिस दिन का तुमसे वायदा किया जाता है क़रीब है या मेरे परवरदिगार ने उसकी मुद्दत दराज़ कर दी है
\end{hindi}}
\flushright{\begin{Arabic}
\quranayah[72][26]
\end{Arabic}}
\flushleft{\begin{hindi}
(वही) ग़ैबवॉ है और अपनी ग़ैब की बाते किसी पर ज़ाहिर नहीं करता
\end{hindi}}
\flushright{\begin{Arabic}
\quranayah[72][27]
\end{Arabic}}
\flushleft{\begin{hindi}
मगर जिस पैग़म्बर को पसन्द फरमाए तो उसके आगे और पीछे निगेहबान फरिश्ते मुक़र्रर कर देता है
\end{hindi}}
\flushright{\begin{Arabic}
\quranayah[72][28]
\end{Arabic}}
\flushleft{\begin{hindi}
ताकि देख ले कि उन्होंने अपने परवरदिगार के पैग़ामात पहुँचा दिए और (यूँ तो) जो कुछ उनके पास है वह सब पर हावी है और उसने तो एक एक चीज़ गिन रखी हैं
\end{hindi}}
\chapter{Al-Muzzammil (The One Covering Himself)}
\begin{Arabic}
\Huge{\centerline{\basmalah}}\end{Arabic}
\flushright{\begin{Arabic}
\quranayah[73][1]
\end{Arabic}}
\flushleft{\begin{hindi}
ऐ (मेरे) चादर लपेटे रसूल
\end{hindi}}
\flushright{\begin{Arabic}
\quranayah[73][2]
\end{Arabic}}
\flushleft{\begin{hindi}
रात को (नमाज़ के वास्ते) खड़े रहो मगर (पूरी रात नहीं)
\end{hindi}}
\flushright{\begin{Arabic}
\quranayah[73][3]
\end{Arabic}}
\flushleft{\begin{hindi}
थोड़ी रात या आधी रात या इससे भी कुछ कम कर दो या उससे कुछ बढ़ा दो
\end{hindi}}
\flushright{\begin{Arabic}
\quranayah[73][4]
\end{Arabic}}
\flushleft{\begin{hindi}
और क़ुरान को बाक़ायदा ठहर ठहर कर पढ़ा करो
\end{hindi}}
\flushright{\begin{Arabic}
\quranayah[73][5]
\end{Arabic}}
\flushleft{\begin{hindi}
हम अनक़रीब तुम पर एक भारी हुक्म नाज़िल करेंगे इसमें शक़ नहीं कि रात को उठना
\end{hindi}}
\flushright{\begin{Arabic}
\quranayah[73][6]
\end{Arabic}}
\flushleft{\begin{hindi}
ख़ूब (नफ्स का) पामाल करना और बहुत ठिकाने से ज़िक्र का वक्त है
\end{hindi}}
\flushright{\begin{Arabic}
\quranayah[73][7]
\end{Arabic}}
\flushleft{\begin{hindi}
दिन को तो तुम्हारे बहुत बड़े बड़े अशग़ाल हैं
\end{hindi}}
\flushright{\begin{Arabic}
\quranayah[73][8]
\end{Arabic}}
\flushleft{\begin{hindi}
तो तुम अपने परवरदिगार के नाम का ज़िक्र करो और सबसे टूट कर उसी के हो रहो
\end{hindi}}
\flushright{\begin{Arabic}
\quranayah[73][9]
\end{Arabic}}
\flushleft{\begin{hindi}
(वही) मशरिक और मग़रिब का मालिक है उसके सिवा कोई माबूद नहीं तो तुम उसी को कारसाज़ बनाओ
\end{hindi}}
\flushright{\begin{Arabic}
\quranayah[73][10]
\end{Arabic}}
\flushleft{\begin{hindi}
और जो कुछ लोग बका करते हैं उस पर सब्र करो और उनसे बा उनवाने शाएस्ता अलग थलग रहो
\end{hindi}}
\flushright{\begin{Arabic}
\quranayah[73][11]
\end{Arabic}}
\flushleft{\begin{hindi}
और मुझे उन झुठलाने वालों से जो दौलतमन्द हैं समझ लेने दो और उनको थोड़ी सी मोहलत दे दो
\end{hindi}}
\flushright{\begin{Arabic}
\quranayah[73][12]
\end{Arabic}}
\flushleft{\begin{hindi}
बेशक हमारे पास बेड़ियाँ (भी) हैं और जलाने वाली आग (भी)
\end{hindi}}
\flushright{\begin{Arabic}
\quranayah[73][13]
\end{Arabic}}
\flushleft{\begin{hindi}
और गले में फँसने वाला खाना (भी) और दुख देने वाला अज़ाब (भी)
\end{hindi}}
\flushright{\begin{Arabic}
\quranayah[73][14]
\end{Arabic}}
\flushleft{\begin{hindi}
जिस दिन ज़मीन और पहाड़ लरज़ने लगेंगे और पहाड़ रेत के टीले से भुर भुरे हो जाएँगे
\end{hindi}}
\flushright{\begin{Arabic}
\quranayah[73][15]
\end{Arabic}}
\flushleft{\begin{hindi}
(ऐ मक्का वालों) हमने तुम्हारे पास (उसी तरह) एक रसूल (मोहम्मद) को भेजा जो तुम्हारे मामले में गवाही दे जिस तरह फिरऔन के पास एक रसूल (मूसा) को भेजा था
\end{hindi}}
\flushright{\begin{Arabic}
\quranayah[73][16]
\end{Arabic}}
\flushleft{\begin{hindi}
तो फिरऔन ने उस रसूल की नाफ़रमानी की तो हमने भी (उसकी सज़ा में) उसको बहुत सख्त पकड़ा
\end{hindi}}
\flushright{\begin{Arabic}
\quranayah[73][17]
\end{Arabic}}
\flushleft{\begin{hindi}
तो अगर तुम भी न मानोगे तो उस दिन (के अज़ाब) से क्यों कर बचोगे जो बच्चों को बूढ़ा बना देगा
\end{hindi}}
\flushright{\begin{Arabic}
\quranayah[73][18]
\end{Arabic}}
\flushleft{\begin{hindi}
जिस दिन आसमान फट पड़ेगा (ये) उसका वायदा पूरा होकर रहेगा
\end{hindi}}
\flushright{\begin{Arabic}
\quranayah[73][19]
\end{Arabic}}
\flushleft{\begin{hindi}
बेशक ये नसीहत है तो जो शख़्श चाहे अपने परवरदिगार की राह एख्तेयार करे
\end{hindi}}
\flushright{\begin{Arabic}
\quranayah[73][20]
\end{Arabic}}
\flushleft{\begin{hindi}
(ऐ रसूल) तुम्हारा परवरदिगार चाहता है कि तुम और तुम्हारे चन्द साथ के लोग (कभी) दो तिहाई रात के करीब और (कभी) आधी रात और (कभी) तिहाई रात (नमाज़ में) खड़े रहते हो और ख़ुदा ही रात और दिन का अच्छी तरह अन्दाज़ा कर सकता है उसे मालूम है कि तुम लोग उस पर पूरी तरह से हावी नहीं हो सकते तो उसने तुम पर मेहरबानी की तो जितना आसानी से हो सके उतना (नमाज़ में) क़ुरान पढ़ लिया करो और वह जानता है कि अनक़रीब तुममें से बाज़ बीमार हो जाएँगे और बाज़ ख़ुदा के फ़ज़ल की तलाश में रूए ज़मीन पर सफर एख्तेयार करेंगे और कुछ लोग ख़ुदा की राह में जेहाद करेंगे तो जितना तुम आसानी से हो सके पढ़ लिया करो और नमाज़ पाबन्दी से पढ़ो और ज़कात देते रहो और ख़ुदा को कर्ज़े हसना दो और जो नेक अमल अपने वास्ते (ख़ुदा के सामने) पेश करोगे उसको ख़ुदा के हाँ बेहतर और सिले में बुर्ज़ुग तर पाओगे और ख़ुदा से मग़फेरत की दुआ माँगो बेशक ख़ुदा बड़ा बख्शने वाला मेहरबान है
\end{hindi}}
\chapter{Al-Muddaththir (The One Wrapping Himself Up)}
\begin{Arabic}
\Huge{\centerline{\basmalah}}\end{Arabic}
\flushright{\begin{Arabic}
\quranayah[74][1]
\end{Arabic}}
\flushleft{\begin{hindi}
ऐ (मेरे) कपड़ा ओढ़ने वाले (रसूल) उठो
\end{hindi}}
\flushright{\begin{Arabic}
\quranayah[74][2]
\end{Arabic}}
\flushleft{\begin{hindi}
और लोगों को (अज़ाब से) डराओ
\end{hindi}}
\flushright{\begin{Arabic}
\quranayah[74][3]
\end{Arabic}}
\flushleft{\begin{hindi}
और अपने परवरदिगार की बड़ाई करो
\end{hindi}}
\flushright{\begin{Arabic}
\quranayah[74][4]
\end{Arabic}}
\flushleft{\begin{hindi}
और अपने कपड़े पाक रखो
\end{hindi}}
\flushright{\begin{Arabic}
\quranayah[74][5]
\end{Arabic}}
\flushleft{\begin{hindi}
और गन्दगी से अलग रहो
\end{hindi}}
\flushright{\begin{Arabic}
\quranayah[74][6]
\end{Arabic}}
\flushleft{\begin{hindi}
और इसी तरह एहसान न करो कि ज्यादा के ख़ास्तगार बनो
\end{hindi}}
\flushright{\begin{Arabic}
\quranayah[74][7]
\end{Arabic}}
\flushleft{\begin{hindi}
और अपने परवरदिगार के लिए सब्र करो
\end{hindi}}
\flushright{\begin{Arabic}
\quranayah[74][8]
\end{Arabic}}
\flushleft{\begin{hindi}
फिर जब सूर फूँका जाएगा
\end{hindi}}
\flushright{\begin{Arabic}
\quranayah[74][9]
\end{Arabic}}
\flushleft{\begin{hindi}
तो वह दिन काफ़िरों पर सख्त दिन होगा
\end{hindi}}
\flushright{\begin{Arabic}
\quranayah[74][10]
\end{Arabic}}
\flushleft{\begin{hindi}
आसान नहीं होगा
\end{hindi}}
\flushright{\begin{Arabic}
\quranayah[74][11]
\end{Arabic}}
\flushleft{\begin{hindi}
(ऐ रसूल) मुझे और उस शख़्श को छोड़ दो जिसे मैने अकेला पैदा किया
\end{hindi}}
\flushright{\begin{Arabic}
\quranayah[74][12]
\end{Arabic}}
\flushleft{\begin{hindi}
और उसे बहुत सा माल दिया
\end{hindi}}
\flushright{\begin{Arabic}
\quranayah[74][13]
\end{Arabic}}
\flushleft{\begin{hindi}
और नज़र के सामने रहने वाले बेटे (दिए)
\end{hindi}}
\flushright{\begin{Arabic}
\quranayah[74][14]
\end{Arabic}}
\flushleft{\begin{hindi}
और उसे हर तरह के सामान से वुसअत दी
\end{hindi}}
\flushright{\begin{Arabic}
\quranayah[74][15]
\end{Arabic}}
\flushleft{\begin{hindi}
फिर उस पर भी वह तमाअ रखता है कि मैं और बढ़ाऊँ
\end{hindi}}
\flushright{\begin{Arabic}
\quranayah[74][16]
\end{Arabic}}
\flushleft{\begin{hindi}
ये हरगिज़ न होगा ये तो मेरी आयतों का दुश्मन था
\end{hindi}}
\flushright{\begin{Arabic}
\quranayah[74][17]
\end{Arabic}}
\flushleft{\begin{hindi}
तो मैं अनक़रीब उस सख्त अज़ाब में मुब्तिला करूँगा
\end{hindi}}
\flushright{\begin{Arabic}
\quranayah[74][18]
\end{Arabic}}
\flushleft{\begin{hindi}
उसने फिक्र की और ये तजवीज़ की
\end{hindi}}
\flushright{\begin{Arabic}
\quranayah[74][19]
\end{Arabic}}
\flushleft{\begin{hindi}
तो ये (कम्बख्त) मार डाला जाए
\end{hindi}}
\flushright{\begin{Arabic}
\quranayah[74][20]
\end{Arabic}}
\flushleft{\begin{hindi}
उसने क्यों कर तजवीज़ की
\end{hindi}}
\flushright{\begin{Arabic}
\quranayah[74][21]
\end{Arabic}}
\flushleft{\begin{hindi}
फिर ग़ौर किया
\end{hindi}}
\flushright{\begin{Arabic}
\quranayah[74][22]
\end{Arabic}}
\flushleft{\begin{hindi}
फिर त्योरी चढ़ाई और मुँह बना लिया
\end{hindi}}
\flushright{\begin{Arabic}
\quranayah[74][23]
\end{Arabic}}
\flushleft{\begin{hindi}
फिर पीठ फेर कर चला गया और अकड़ बैठा
\end{hindi}}
\flushright{\begin{Arabic}
\quranayah[74][24]
\end{Arabic}}
\flushleft{\begin{hindi}
फिर कहने लगा ये बस जादू है जो (अगलों से) चला आता है
\end{hindi}}
\flushright{\begin{Arabic}
\quranayah[74][25]
\end{Arabic}}
\flushleft{\begin{hindi}
ये तो बस आदमी का कलाम है
\end{hindi}}
\flushright{\begin{Arabic}
\quranayah[74][26]
\end{Arabic}}
\flushleft{\begin{hindi}
(ख़ुदा का नहीं) मैं उसे अनक़रीब जहन्नुम में झोंक दूँगा
\end{hindi}}
\flushright{\begin{Arabic}
\quranayah[74][27]
\end{Arabic}}
\flushleft{\begin{hindi}
और तुम क्या जानों कि जहन्नुम क्या है
\end{hindi}}
\flushright{\begin{Arabic}
\quranayah[74][28]
\end{Arabic}}
\flushleft{\begin{hindi}
वह न बाक़ी रखेगी न छोड़ देगी
\end{hindi}}
\flushright{\begin{Arabic}
\quranayah[74][29]
\end{Arabic}}
\flushleft{\begin{hindi}
और बदन को जला कर सियाह कर देगी
\end{hindi}}
\flushright{\begin{Arabic}
\quranayah[74][30]
\end{Arabic}}
\flushleft{\begin{hindi}
उस पर उन्नीस (फ़रिश्ते मुअय्यन) हैं
\end{hindi}}
\flushright{\begin{Arabic}
\quranayah[74][31]
\end{Arabic}}
\flushleft{\begin{hindi}
और हमने जहन्नुम का निगेहबान तो बस फरिश्तों को बनाया है और उनका ये शुमार भी काफिरों की आज़माइश के लिए मुक़र्रर किया ताकि अहले किताब (फौरन) यक़ीन कर लें और मोमिनो का ईमान और ज्यादा हो और अहले किताब और मोमिनीन (किसी तरह) शक़ न करें और जिन लोगों के दिल में (निफ़ाक का) मर्ज़ है (वह) और काफिर लोग कह बैठे कि इस मसल (के बयान करने) से ख़ुदा का क्या मतलब है यूँ ख़ुदा जिसे चाहता है गुमराही में छोड़ देता है और जिसे चाहता है हिदायत करता है और तुम्हारे परवरदिगार के लशकरों को उसके सिवा कोई नहीं जानता और ये तो आदमियों के लिए बस नसीहत है
\end{hindi}}
\flushright{\begin{Arabic}
\quranayah[74][32]
\end{Arabic}}
\flushleft{\begin{hindi}
सुन रखो (हमें) चाँद की क़सम
\end{hindi}}
\flushright{\begin{Arabic}
\quranayah[74][33]
\end{Arabic}}
\flushleft{\begin{hindi}
और रात की जब जाने लगे
\end{hindi}}
\flushright{\begin{Arabic}
\quranayah[74][34]
\end{Arabic}}
\flushleft{\begin{hindi}
और सुबह की जब रौशन हो जाए
\end{hindi}}
\flushright{\begin{Arabic}
\quranayah[74][35]
\end{Arabic}}
\flushleft{\begin{hindi}
कि वह (जहन्नुम) भी एक बहुत बड़ी (आफ़त) है
\end{hindi}}
\flushright{\begin{Arabic}
\quranayah[74][36]
\end{Arabic}}
\flushleft{\begin{hindi}
(और) लोगों के डराने वाली है
\end{hindi}}
\flushright{\begin{Arabic}
\quranayah[74][37]
\end{Arabic}}
\flushleft{\begin{hindi}
(सबके लिए नहीें बल्कि) तुममें से वह जो शख़्श (नेकी की तरफ़) आगे बढ़ना
\end{hindi}}
\flushright{\begin{Arabic}
\quranayah[74][38]
\end{Arabic}}
\flushleft{\begin{hindi}
और (बुराई से) पीछे हटना चाहे हर शख़्श अपने आमाल के बदले गिर्द है
\end{hindi}}
\flushright{\begin{Arabic}
\quranayah[74][39]
\end{Arabic}}
\flushleft{\begin{hindi}
मगर दाहिने हाथ (में नामए अमल लेने) वाले
\end{hindi}}
\flushright{\begin{Arabic}
\quranayah[74][40]
\end{Arabic}}
\flushleft{\begin{hindi}
(बेहिश्त के) बाग़ों में गुनेहगारों से बाहम पूछ रहे होंगे
\end{hindi}}
\flushright{\begin{Arabic}
\quranayah[74][41]
\end{Arabic}}
\flushleft{\begin{hindi}
कि आख़िर तुम्हें दोज़ख़ में कौन सी चीज़ (घसीट) लायी
\end{hindi}}
\flushright{\begin{Arabic}
\quranayah[74][42]
\end{Arabic}}
\flushleft{\begin{hindi}
वह लोग कहेंगे
\end{hindi}}
\flushright{\begin{Arabic}
\quranayah[74][43]
\end{Arabic}}
\flushleft{\begin{hindi}
कि हम न तो नमाज़ पढ़ा करते थे
\end{hindi}}
\flushright{\begin{Arabic}
\quranayah[74][44]
\end{Arabic}}
\flushleft{\begin{hindi}
और न मोहताजों को खाना खिलाते थे
\end{hindi}}
\flushright{\begin{Arabic}
\quranayah[74][45]
\end{Arabic}}
\flushleft{\begin{hindi}
और अहले बातिल के साथ हम भी बड़े काम में घुस पड़ते थे
\end{hindi}}
\flushright{\begin{Arabic}
\quranayah[74][46]
\end{Arabic}}
\flushleft{\begin{hindi}
और रोज़ जज़ा को झुठलाया करते थे (और यूँ ही रहे)
\end{hindi}}
\flushright{\begin{Arabic}
\quranayah[74][47]
\end{Arabic}}
\flushleft{\begin{hindi}
यहाँ तक कि हमें मौत आ गयी
\end{hindi}}
\flushright{\begin{Arabic}
\quranayah[74][48]
\end{Arabic}}
\flushleft{\begin{hindi}
तो (उस वक्त) उन्हें सिफ़ारिश करने वालों की सिफ़ारिश कुछ काम न आएगी
\end{hindi}}
\flushright{\begin{Arabic}
\quranayah[74][49]
\end{Arabic}}
\flushleft{\begin{hindi}
और उन्हें क्या हो गया है कि नसीहत से मुँह मोड़े हुए हैं
\end{hindi}}
\flushright{\begin{Arabic}
\quranayah[74][50]
\end{Arabic}}
\flushleft{\begin{hindi}
गोया वह वहशी गधे हैं
\end{hindi}}
\flushright{\begin{Arabic}
\quranayah[74][51]
\end{Arabic}}
\flushleft{\begin{hindi}
कि येर से (दुम दबा कर) भागते हैं
\end{hindi}}
\flushright{\begin{Arabic}
\quranayah[74][52]
\end{Arabic}}
\flushleft{\begin{hindi}
असल ये है कि उनमें से हर शख़्श इसका मुतमइनी है कि उसे खुली हुई (आसमानी) किताबें अता की जाएँ
\end{hindi}}
\flushright{\begin{Arabic}
\quranayah[74][53]
\end{Arabic}}
\flushleft{\begin{hindi}
ये तो हरगिज़ न होगा बल्कि ये तो आख़ेरत ही से नहीं डरते
\end{hindi}}
\flushright{\begin{Arabic}
\quranayah[74][54]
\end{Arabic}}
\flushleft{\begin{hindi}
हाँ हाँ बेशक ये (क़ुरान सरा सर) नसीहत है
\end{hindi}}
\flushright{\begin{Arabic}
\quranayah[74][55]
\end{Arabic}}
\flushleft{\begin{hindi}
तो जो चाहे उसे याद रखे
\end{hindi}}
\flushright{\begin{Arabic}
\quranayah[74][56]
\end{Arabic}}
\flushleft{\begin{hindi}
और ख़ुदा की मशीयत के बग़ैर ये लोग याद रखने वाले नहीं वही (बन्दों के) डराने के क़ाबिल और बख्यिश का मालिक है
\end{hindi}}
\chapter{Al-Qiyamah (The Resurrection)}
\begin{Arabic}
\Huge{\centerline{\basmalah}}\end{Arabic}
\flushright{\begin{Arabic}
\quranayah[75][1]
\end{Arabic}}
\flushleft{\begin{hindi}
मैं रोजे क़यामत की क़सम खाता हूँ
\end{hindi}}
\flushright{\begin{Arabic}
\quranayah[75][2]
\end{Arabic}}
\flushleft{\begin{hindi}
(और बुराई से) मलामत करने वाले जी की क़सम खाता हूँ (कि तुम सब दोबारा) ज़रूर ज़िन्दा किए जाओगे
\end{hindi}}
\flushright{\begin{Arabic}
\quranayah[75][3]
\end{Arabic}}
\flushleft{\begin{hindi}
क्या इन्सान ये ख्याल करता है (कि हम उसकी हड्डियों को बोसीदा होने के बाद) जमा न करेंगे हाँ (ज़रूर करेंगें)
\end{hindi}}
\flushright{\begin{Arabic}
\quranayah[75][4]
\end{Arabic}}
\flushleft{\begin{hindi}
हम इस पर क़ादिर हैं कि हम उसकी पोर पोर दुरूस्त करें
\end{hindi}}
\flushright{\begin{Arabic}
\quranayah[75][5]
\end{Arabic}}
\flushleft{\begin{hindi}
मगर इन्सान तो ये जानता है कि अपने आगे भी (हमेशा) बुराई करता जाए
\end{hindi}}
\flushright{\begin{Arabic}
\quranayah[75][6]
\end{Arabic}}
\flushleft{\begin{hindi}
पूछता है कि क़यामत का दिन कब होगा
\end{hindi}}
\flushright{\begin{Arabic}
\quranayah[75][7]
\end{Arabic}}
\flushleft{\begin{hindi}
तो जब ऑंखे चकाचौन्ध में आ जाएँगी
\end{hindi}}
\flushright{\begin{Arabic}
\quranayah[75][8]
\end{Arabic}}
\flushleft{\begin{hindi}
और चाँद गहन में लग जाएगा
\end{hindi}}
\flushright{\begin{Arabic}
\quranayah[75][9]
\end{Arabic}}
\flushleft{\begin{hindi}
और सूरज और चाँद इकट्ठा कर दिए जाएँगे
\end{hindi}}
\flushright{\begin{Arabic}
\quranayah[75][10]
\end{Arabic}}
\flushleft{\begin{hindi}
तो इन्सान कहेगा आज कहाँ भाग कर जाऊँ
\end{hindi}}
\flushright{\begin{Arabic}
\quranayah[75][11]
\end{Arabic}}
\flushleft{\begin{hindi}
यक़ीन जानों कहीं पनाह नहीं
\end{hindi}}
\flushright{\begin{Arabic}
\quranayah[75][12]
\end{Arabic}}
\flushleft{\begin{hindi}
उस रोज़ तुम्हारे परवरदिगार ही के पास ठिकाना है
\end{hindi}}
\flushright{\begin{Arabic}
\quranayah[75][13]
\end{Arabic}}
\flushleft{\begin{hindi}
उस दिन आदमी को जो कुछ उसके आगे पीछे किया है बता दिया जाएगा
\end{hindi}}
\flushright{\begin{Arabic}
\quranayah[75][14]
\end{Arabic}}
\flushleft{\begin{hindi}
बल्कि इन्सान तो अपने ऊपर आप गवाह है
\end{hindi}}
\flushright{\begin{Arabic}
\quranayah[75][15]
\end{Arabic}}
\flushleft{\begin{hindi}
अगरचे वह अपने गुनाहों की उज्र व ज़रूर माज़ेरत पढ़ा करता रहे
\end{hindi}}
\flushright{\begin{Arabic}
\quranayah[75][16]
\end{Arabic}}
\flushleft{\begin{hindi}
(ऐ रसूल) वही के जल्दी याद करने वास्ते अपनी ज़बान को हरकत न दो
\end{hindi}}
\flushright{\begin{Arabic}
\quranayah[75][17]
\end{Arabic}}
\flushleft{\begin{hindi}
उसका जमा कर देना और पढ़वा देना तो यक़ीनी हमारे ज़िम्मे है
\end{hindi}}
\flushright{\begin{Arabic}
\quranayah[75][18]
\end{Arabic}}
\flushleft{\begin{hindi}
तो जब हम उसको (जिबरील की ज़बानी) पढ़ें तो तुम भी (पूरा) सुनने के बाद इसी तरह पढ़ा करो
\end{hindi}}
\flushright{\begin{Arabic}
\quranayah[75][19]
\end{Arabic}}
\flushleft{\begin{hindi}
फिर उस (के मुश्किलात का समझा देना भी हमारे ज़िम्में है)
\end{hindi}}
\flushright{\begin{Arabic}
\quranayah[75][20]
\end{Arabic}}
\flushleft{\begin{hindi}
मगर (लोगों) हक़ तो ये है कि तुम लोग दुनिया को दोस्त रखते हो
\end{hindi}}
\flushright{\begin{Arabic}
\quranayah[75][21]
\end{Arabic}}
\flushleft{\begin{hindi}
और आख़ेरत को छोड़े बैठे हो
\end{hindi}}
\flushright{\begin{Arabic}
\quranayah[75][22]
\end{Arabic}}
\flushleft{\begin{hindi}
उस रोज़ बहुत से चेहरे तो तरो ताज़ा बशबाब होंगे
\end{hindi}}
\flushright{\begin{Arabic}
\quranayah[75][23]
\end{Arabic}}
\flushleft{\begin{hindi}
(और) अपने परवरदिगार (की नेअमत) को देख रहे होंगे
\end{hindi}}
\flushright{\begin{Arabic}
\quranayah[75][24]
\end{Arabic}}
\flushleft{\begin{hindi}
और बहुतेरे मुँह उस दिन उदास होंगे
\end{hindi}}
\flushright{\begin{Arabic}
\quranayah[75][25]
\end{Arabic}}
\flushleft{\begin{hindi}
समझ रहें हैं कि उन पर मुसीबत पड़ने वाली है कि कमर तोड़ देगी
\end{hindi}}
\flushright{\begin{Arabic}
\quranayah[75][26]
\end{Arabic}}
\flushleft{\begin{hindi}
सुन लो जब जान (बदन से खिंच के) हँसली तक आ पहुँचेगी
\end{hindi}}
\flushright{\begin{Arabic}
\quranayah[75][27]
\end{Arabic}}
\flushleft{\begin{hindi}
और कहा जाएगा कि (इस वक्त) क़ोई झाड़ फूँक करने वाला है
\end{hindi}}
\flushright{\begin{Arabic}
\quranayah[75][28]
\end{Arabic}}
\flushleft{\begin{hindi}
और मरने वाले ने समझा कि अब (सबसे) जुदाई है
\end{hindi}}
\flushright{\begin{Arabic}
\quranayah[75][29]
\end{Arabic}}
\flushleft{\begin{hindi}
और (मौत की तकलीफ़ से) पिन्डली से पिन्डली लिपट जाएगी
\end{hindi}}
\flushright{\begin{Arabic}
\quranayah[75][30]
\end{Arabic}}
\flushleft{\begin{hindi}
उस दिन तुमको अपने परवरदिगार की बारगाह में चलना है
\end{hindi}}
\flushright{\begin{Arabic}
\quranayah[75][31]
\end{Arabic}}
\flushleft{\begin{hindi}
तो उसने (ग़फलत में) न (कलामे ख़ुदा की) तसदीक़ की न नमाज़ पढ़ी
\end{hindi}}
\flushright{\begin{Arabic}
\quranayah[75][32]
\end{Arabic}}
\flushleft{\begin{hindi}
मगर झुठलाया और (ईमान से) मुँह फेरा
\end{hindi}}
\flushright{\begin{Arabic}
\quranayah[75][33]
\end{Arabic}}
\flushleft{\begin{hindi}
अपने घर की तरफ इतराता हुआ चला
\end{hindi}}
\flushright{\begin{Arabic}
\quranayah[75][34]
\end{Arabic}}
\flushleft{\begin{hindi}
अफसोस है तुझ पर फिर अफसोस है फिर तुफ़ है
\end{hindi}}
\flushright{\begin{Arabic}
\quranayah[75][35]
\end{Arabic}}
\flushleft{\begin{hindi}
तुझ पर फिर तुफ़ है
\end{hindi}}
\flushright{\begin{Arabic}
\quranayah[75][36]
\end{Arabic}}
\flushleft{\begin{hindi}
क्या इन्सान ये समझता है कि वह यूँ ही छोड़ दिया जाएगा
\end{hindi}}
\flushright{\begin{Arabic}
\quranayah[75][37]
\end{Arabic}}
\flushleft{\begin{hindi}
क्या वह (इब्तेदन) मनी का एक क़तरा न था जो रहम में डाली जाती है
\end{hindi}}
\flushright{\begin{Arabic}
\quranayah[75][38]
\end{Arabic}}
\flushleft{\begin{hindi}
फिर लोथड़ा हुआ फिर ख़ुदा ने उसे बनाया
\end{hindi}}
\flushright{\begin{Arabic}
\quranayah[75][39]
\end{Arabic}}
\flushleft{\begin{hindi}
फिर उसे दुरूस्त किया फिर उसकी दो किस्में बनायीं (एक) मर्द और (एक) औरत
\end{hindi}}
\flushright{\begin{Arabic}
\quranayah[75][40]
\end{Arabic}}
\flushleft{\begin{hindi}
क्या इस पर क़ादिर नहीं कि (क़यामत में) मुर्दों को ज़िन्दा कर दे
\end{hindi}}
\chapter{Al-Insan (The Man)}
\begin{Arabic}
\Huge{\centerline{\basmalah}}\end{Arabic}
\flushright{\begin{Arabic}
\quranayah[76][1]
\end{Arabic}}
\flushleft{\begin{hindi}
बेशक इन्सान पर एक ऐसा वक्त अा चुका है कि वह कोई चीज़ क़ाबिले ज़िक्र न था
\end{hindi}}
\flushright{\begin{Arabic}
\quranayah[76][2]
\end{Arabic}}
\flushleft{\begin{hindi}
हमने इन्सान को मख़लूत नुत्फे से पैदा किया कि उसे आज़माये तो हमने उसे सुनता देखता बनाया
\end{hindi}}
\flushright{\begin{Arabic}
\quranayah[76][3]
\end{Arabic}}
\flushleft{\begin{hindi}
और उसको रास्ता भी दिखा दिया (अब वह) ख्वाह शुक्र गुज़ार हो ख्वाह नाशुक्रा
\end{hindi}}
\flushright{\begin{Arabic}
\quranayah[76][4]
\end{Arabic}}
\flushleft{\begin{hindi}
हमने काफ़िरों के ज़ंजीरे, तौक और दहकती हुई आग तैयार कर रखी है
\end{hindi}}
\flushright{\begin{Arabic}
\quranayah[76][5]
\end{Arabic}}
\flushleft{\begin{hindi}
बेशक नेकोकार लोग शराब के वह सागर पियेंगे जिसमें काफूर की आमेज़िश होगी ये एक चश्मा है जिसमें से ख़ुदा के (ख़ास) बन्दे पियेंगे
\end{hindi}}
\flushright{\begin{Arabic}
\quranayah[76][6]
\end{Arabic}}
\flushleft{\begin{hindi}
और जहाँ चाहेंगे बहा ले जाएँगे
\end{hindi}}
\flushright{\begin{Arabic}
\quranayah[76][7]
\end{Arabic}}
\flushleft{\begin{hindi}
ये वह लोग हैं जो नज़रें पूरी करते हैं और उस दिन से जिनकी सख्ती हर तरह फैली होगी डरते हैं
\end{hindi}}
\flushright{\begin{Arabic}
\quranayah[76][8]
\end{Arabic}}
\flushleft{\begin{hindi}
और उसकी मोहब्बत में मोहताज और यतीम और असीर को खाना खिलाते हैं
\end{hindi}}
\flushright{\begin{Arabic}
\quranayah[76][9]
\end{Arabic}}
\flushleft{\begin{hindi}
(और कहते हैं कि) हम तो तुमको बस ख़ालिस ख़ुदा के लिए खिलाते हैं हम न तुम से बदले के ख़ास्तगार हैं और न शुक्र गुज़ारी के
\end{hindi}}
\flushright{\begin{Arabic}
\quranayah[76][10]
\end{Arabic}}
\flushleft{\begin{hindi}
हमको तो अपने परवरदिगार से उस दिन का डर है जिसमें मुँह बन जाएँगे (और) चेहरे पर हवाइयाँ उड़ती होंगी
\end{hindi}}
\flushright{\begin{Arabic}
\quranayah[76][11]
\end{Arabic}}
\flushleft{\begin{hindi}
तो ख़ुदा उन्हें उस दिन की तकलीफ़ से बचा लेगा और उनको ताज़गी और ख़ुशदिली अता फरमाएगा
\end{hindi}}
\flushright{\begin{Arabic}
\quranayah[76][12]
\end{Arabic}}
\flushleft{\begin{hindi}
और उनके सब्र के बदले (बेहिश्त के) बाग़ और रेशम (की पोशाक) अता फ़रमाएगा
\end{hindi}}
\flushright{\begin{Arabic}
\quranayah[76][13]
\end{Arabic}}
\flushleft{\begin{hindi}
वहाँ वह तख्तों पर तकिए लगाए (बैठे) होंगे न वहाँ (आफताब की) धूप देखेंगे और न शिद्दत की सर्दी
\end{hindi}}
\flushright{\begin{Arabic}
\quranayah[76][14]
\end{Arabic}}
\flushleft{\begin{hindi}
और घने दरख्तों के साए उन पर झुके हुए होंगे और मेवों के गुच्छे उनके बहुत क़रीब हर तरह उनके एख्तेयार में
\end{hindi}}
\flushright{\begin{Arabic}
\quranayah[76][15]
\end{Arabic}}
\flushleft{\begin{hindi}
और उनके सामने चाँदी के साग़र और शीशे के निहायत शफ्फ़ाफ़ गिलास का दौर चल रहा होगा
\end{hindi}}
\flushright{\begin{Arabic}
\quranayah[76][16]
\end{Arabic}}
\flushleft{\begin{hindi}
और शीशे भी (काँच के नहीं) चाँदी के जो ठीक अन्दाज़े के मुताबिक बनाए गए हैं
\end{hindi}}
\flushright{\begin{Arabic}
\quranayah[76][17]
\end{Arabic}}
\flushleft{\begin{hindi}
और वहाँ उन्हें ऐसी शराब पिलाई जाएगी जिसमें जनजबील (के पानी) की आमेज़िश होगी
\end{hindi}}
\flushright{\begin{Arabic}
\quranayah[76][18]
\end{Arabic}}
\flushleft{\begin{hindi}
ये बेहश्त में एक चश्मा है जिसका नाम सलसबील है
\end{hindi}}
\flushright{\begin{Arabic}
\quranayah[76][19]
\end{Arabic}}
\flushleft{\begin{hindi}
और उनके सामने हमेशा एक हालत पर रहने वाले नौजवाल लड़के चक्कर लगाते होंगे कि जब तुम उनको देखो तो समझो कि बिखरे हुए मोती हैं
\end{hindi}}
\flushright{\begin{Arabic}
\quranayah[76][20]
\end{Arabic}}
\flushleft{\begin{hindi}
और जब तुम वहाँ निगाह उठाओगे तो हर तरह की नेअमत और अज़ीमुश शान सल्तनत देखोगे
\end{hindi}}
\flushright{\begin{Arabic}
\quranayah[76][21]
\end{Arabic}}
\flushleft{\begin{hindi}
उनके ऊपर सब्ज़ क्रेब और अतलस की पोशाक होगी और उन्हें चाँदी के कंगन पहनाए जाएँगे और उनका परवरदिगार उन्हें निहायत पाकीज़ा शराब पिलाएगा
\end{hindi}}
\flushright{\begin{Arabic}
\quranayah[76][22]
\end{Arabic}}
\flushleft{\begin{hindi}
ये यक़ीनी तुम्हारे लिए होगा और तुम्हारी (कारगुज़ारियों के) सिले में और तुम्हारी कोशिश क़ाबिले शुक्र गुज़ारी है
\end{hindi}}
\flushright{\begin{Arabic}
\quranayah[76][23]
\end{Arabic}}
\flushleft{\begin{hindi}
(ऐ रसूल) हमने तुम पर क़ुरान को रफ्ता रफ्ता करके नाज़िल किया
\end{hindi}}
\flushright{\begin{Arabic}
\quranayah[76][24]
\end{Arabic}}
\flushleft{\begin{hindi}
तो तुम अपने परवरदिगार के हुक्म के इन्तज़ार में सब्र किए रहो और उन लोगों में से गुनाहगार और नाशुक्रे की पैरवी न करना
\end{hindi}}
\flushright{\begin{Arabic}
\quranayah[76][25]
\end{Arabic}}
\flushleft{\begin{hindi}
सुबह शाम अपने परवरदिगार का नाम लेते रहो
\end{hindi}}
\flushright{\begin{Arabic}
\quranayah[76][26]
\end{Arabic}}
\flushleft{\begin{hindi}
और कुछ रात गए उसका सजदा करो और बड़ी रात तक उसकी तस्बीह करते रहो
\end{hindi}}
\flushright{\begin{Arabic}
\quranayah[76][27]
\end{Arabic}}
\flushleft{\begin{hindi}
ये लोग यक़ीनन दुनिया को पसन्द करते हैं और बड़े भारी दिन को अपने पसे पुश्त छोड़ बैठे हैं
\end{hindi}}
\flushright{\begin{Arabic}
\quranayah[76][28]
\end{Arabic}}
\flushleft{\begin{hindi}
हमने उनको पैदा किया और उनके आज़ा को मज़बूत बनाया और अगर हम चाहें तो उनके बदले उन्हीं के जैसे लोग ले आएँ
\end{hindi}}
\flushright{\begin{Arabic}
\quranayah[76][29]
\end{Arabic}}
\flushleft{\begin{hindi}
बेशक ये कुरान सरासर नसीहत है तो जो शख़्श चाहे अपने परवरदिगार की राह ले
\end{hindi}}
\flushright{\begin{Arabic}
\quranayah[76][30]
\end{Arabic}}
\flushleft{\begin{hindi}
और जब तक ख़ुदा को मंज़ूर न हो तुम लोग कुछ भी चाह नहीं सकते बेशक ख़ुदा बड़ा वाक़िफकार दाना है
\end{hindi}}
\flushright{\begin{Arabic}
\quranayah[76][31]
\end{Arabic}}
\flushleft{\begin{hindi}
जिसको चाहे अपनी रहमत में दाख़िल कर ले और ज़ालिमों के वास्ते उसने दर्दनाक अज़ाब तैयार कर रखा है
\end{hindi}}
\chapter{Al-Mursalat (Those Sent Forth)}
\begin{Arabic}
\Huge{\centerline{\basmalah}}\end{Arabic}
\flushright{\begin{Arabic}
\quranayah[77][1]
\end{Arabic}}
\flushleft{\begin{hindi}
हवाओं की क़सम जो (पहले) धीमी चलती हैं
\end{hindi}}
\flushright{\begin{Arabic}
\quranayah[77][2]
\end{Arabic}}
\flushleft{\begin{hindi}
फिर ज़ोर पकड़ के ऑंधी हो जाती हैं
\end{hindi}}
\flushright{\begin{Arabic}
\quranayah[77][3]
\end{Arabic}}
\flushleft{\begin{hindi}
और (बादलों को) उभार कर फैला देती हैं
\end{hindi}}
\flushright{\begin{Arabic}
\quranayah[77][4]
\end{Arabic}}
\flushleft{\begin{hindi}
फिर (उनको) फाड़ कर जुदा कर देती हैं
\end{hindi}}
\flushright{\begin{Arabic}
\quranayah[77][5]
\end{Arabic}}
\flushleft{\begin{hindi}
फिर फरिश्तों की क़सम जो वही लाते हैं
\end{hindi}}
\flushright{\begin{Arabic}
\quranayah[77][6]
\end{Arabic}}
\flushleft{\begin{hindi}
ताकि हुज्जत तमाम हो और डरा दिया जाए
\end{hindi}}
\flushright{\begin{Arabic}
\quranayah[77][7]
\end{Arabic}}
\flushleft{\begin{hindi}
कि जिस बात का तुमसे वायदा किया जाता है वह ज़रूर होकर रहेगा
\end{hindi}}
\flushright{\begin{Arabic}
\quranayah[77][8]
\end{Arabic}}
\flushleft{\begin{hindi}
फिर जब तारों की चमक जाती रहेगी
\end{hindi}}
\flushright{\begin{Arabic}
\quranayah[77][9]
\end{Arabic}}
\flushleft{\begin{hindi}
और जब आसमान फट जाएगा
\end{hindi}}
\flushright{\begin{Arabic}
\quranayah[77][10]
\end{Arabic}}
\flushleft{\begin{hindi}
और जब पहाड़ (रूई की तरह) उड़े उड़े फिरेंगे
\end{hindi}}
\flushright{\begin{Arabic}
\quranayah[77][11]
\end{Arabic}}
\flushleft{\begin{hindi}
और जब पैग़म्बर लोग एक मुअय्यन वक्त पर जमा किए जाएँगे
\end{hindi}}
\flushright{\begin{Arabic}
\quranayah[77][12]
\end{Arabic}}
\flushleft{\begin{hindi}
(फिर) भला इन (बातों) में किस दिन के लिए ताख़ीर की गयी है
\end{hindi}}
\flushright{\begin{Arabic}
\quranayah[77][13]
\end{Arabic}}
\flushleft{\begin{hindi}
फ़ैसले के दिन के लिए
\end{hindi}}
\flushright{\begin{Arabic}
\quranayah[77][14]
\end{Arabic}}
\flushleft{\begin{hindi}
और तुमको क्या मालूम की फ़ैसले का दिन क्या है
\end{hindi}}
\flushright{\begin{Arabic}
\quranayah[77][15]
\end{Arabic}}
\flushleft{\begin{hindi}
उस दिन झुठलाने वालों की मिट्टी ख़राब है
\end{hindi}}
\flushright{\begin{Arabic}
\quranayah[77][16]
\end{Arabic}}
\flushleft{\begin{hindi}
क्या हमने अगलों को हलाक नहीं किया
\end{hindi}}
\flushright{\begin{Arabic}
\quranayah[77][17]
\end{Arabic}}
\flushleft{\begin{hindi}
फिर उनके पीछे पीछे पिछलों को भी चलता करेंगे
\end{hindi}}
\flushright{\begin{Arabic}
\quranayah[77][18]
\end{Arabic}}
\flushleft{\begin{hindi}
हम गुनेहगारों के साथ ऐसा ही किया करते हैं
\end{hindi}}
\flushright{\begin{Arabic}
\quranayah[77][19]
\end{Arabic}}
\flushleft{\begin{hindi}
उस दिन झुठलाने वालों की मिट्टी ख़राब है
\end{hindi}}
\flushright{\begin{Arabic}
\quranayah[77][20]
\end{Arabic}}
\flushleft{\begin{hindi}
क्या हमने तुमको ज़लील पानी (मनी) से पैदा नहीं किया
\end{hindi}}
\flushright{\begin{Arabic}
\quranayah[77][21]
\end{Arabic}}
\flushleft{\begin{hindi}
फिर हमने उसको एक मुअय्यन वक्त तक
\end{hindi}}
\flushright{\begin{Arabic}
\quranayah[77][22]
\end{Arabic}}
\flushleft{\begin{hindi}
एक महफूज़ मक़ाम (रहम) में रखा
\end{hindi}}
\flushright{\begin{Arabic}
\quranayah[77][23]
\end{Arabic}}
\flushleft{\begin{hindi}
फिर (उसका) एक अन्दाज़ा मुक़र्रर किया तो हम कैसा अच्छा अन्दाज़ा मुक़र्रर करने वाले हैं
\end{hindi}}
\flushright{\begin{Arabic}
\quranayah[77][24]
\end{Arabic}}
\flushleft{\begin{hindi}
उन दिन झुठलाने वालों की ख़राबी है
\end{hindi}}
\flushright{\begin{Arabic}
\quranayah[77][25]
\end{Arabic}}
\flushleft{\begin{hindi}
क्या हमने ज़मीन को ज़िन्दों और मुर्दों को समेटने वाली नहीं बनाया
\end{hindi}}
\flushright{\begin{Arabic}
\quranayah[77][26]
\end{Arabic}}
\flushleft{\begin{hindi}
और उसमें ऊँचे ऊँचे अटल पहाड़ रख दिए
\end{hindi}}
\flushright{\begin{Arabic}
\quranayah[77][27]
\end{Arabic}}
\flushleft{\begin{hindi}
और तुम लोगों को मीठा पानी पिलाया
\end{hindi}}
\flushright{\begin{Arabic}
\quranayah[77][28]
\end{Arabic}}
\flushleft{\begin{hindi}
उस दिन झुठलाने वालों की ख़राबी है
\end{hindi}}
\flushright{\begin{Arabic}
\quranayah[77][29]
\end{Arabic}}
\flushleft{\begin{hindi}
जिस चीज़ को तुम झुठलाया करते थे अब उसकी तरफ़ चलो
\end{hindi}}
\flushright{\begin{Arabic}
\quranayah[77][30]
\end{Arabic}}
\flushleft{\begin{hindi}
(धुएँ के) साये की तरफ़ चलो जिसके तीन हिस्से हैं
\end{hindi}}
\flushright{\begin{Arabic}
\quranayah[77][31]
\end{Arabic}}
\flushleft{\begin{hindi}
जिसमें न ठन्डक है और न जहन्नुम की लपक से बचाएगा
\end{hindi}}
\flushright{\begin{Arabic}
\quranayah[77][32]
\end{Arabic}}
\flushleft{\begin{hindi}
उससे इतने बड़े बड़े अंगारे बरसते होंगे जैसे महल
\end{hindi}}
\flushright{\begin{Arabic}
\quranayah[77][33]
\end{Arabic}}
\flushleft{\begin{hindi}
गोया ज़र्द रंग के ऊँट हैं
\end{hindi}}
\flushright{\begin{Arabic}
\quranayah[77][34]
\end{Arabic}}
\flushleft{\begin{hindi}
उस दिन झुठलाने वालों की ख़राबी है
\end{hindi}}
\flushright{\begin{Arabic}
\quranayah[77][35]
\end{Arabic}}
\flushleft{\begin{hindi}
ये वह दिन होगा कि लोग लब तक न हिला सकेंगे
\end{hindi}}
\flushright{\begin{Arabic}
\quranayah[77][36]
\end{Arabic}}
\flushleft{\begin{hindi}
और उनको इजाज़त दी जाएगी कि कुछ उज्र माअज़ेरत कर सकें
\end{hindi}}
\flushright{\begin{Arabic}
\quranayah[77][37]
\end{Arabic}}
\flushleft{\begin{hindi}
उस दिन झुठलाने वालों की तबाही है
\end{hindi}}
\flushright{\begin{Arabic}
\quranayah[77][38]
\end{Arabic}}
\flushleft{\begin{hindi}
यही फैसले का दिन है (जिस में) हमने तुमको और अगलों को इकट्ठा किया है
\end{hindi}}
\flushright{\begin{Arabic}
\quranayah[77][39]
\end{Arabic}}
\flushleft{\begin{hindi}
तो अगर तुम्हें कोई दाँव करना हो तो आओ चल चुको
\end{hindi}}
\flushright{\begin{Arabic}
\quranayah[77][40]
\end{Arabic}}
\flushleft{\begin{hindi}
उस दिन झुठलाने वालों की ख़राबी है
\end{hindi}}
\flushright{\begin{Arabic}
\quranayah[77][41]
\end{Arabic}}
\flushleft{\begin{hindi}
बेशक परहेज़गार लोग (दरख्तों की) घनी छाँव में होंगे
\end{hindi}}
\flushright{\begin{Arabic}
\quranayah[77][42]
\end{Arabic}}
\flushleft{\begin{hindi}
और चश्मों और आदमियों में जो उन्हें मरग़ूब हो
\end{hindi}}
\flushright{\begin{Arabic}
\quranayah[77][43]
\end{Arabic}}
\flushleft{\begin{hindi}
(दुनिया में) जो अमल करते थे उसके बदले में मज़े से खाओ पियो
\end{hindi}}
\flushright{\begin{Arabic}
\quranayah[77][44]
\end{Arabic}}
\flushleft{\begin{hindi}
मुबारक हम नेकोकारों को ऐसा ही बदला दिया करते हैं
\end{hindi}}
\flushright{\begin{Arabic}
\quranayah[77][45]
\end{Arabic}}
\flushleft{\begin{hindi}
उस दिन झुठलाने वालों की ख़राबी है
\end{hindi}}
\flushright{\begin{Arabic}
\quranayah[77][46]
\end{Arabic}}
\flushleft{\begin{hindi}
(झुठलाने वालों) चन्द दिन चैन से खा पी लो तुम बेशक गुनेहगार हो
\end{hindi}}
\flushright{\begin{Arabic}
\quranayah[77][47]
\end{Arabic}}
\flushleft{\begin{hindi}
उस दिन झुठलाने वालों की मिट्टी ख़राब है
\end{hindi}}
\flushright{\begin{Arabic}
\quranayah[77][48]
\end{Arabic}}
\flushleft{\begin{hindi}
और जब उनसे कहा जाता है कि रूकूउ करों तो रूकूउ नहीं करते
\end{hindi}}
\flushright{\begin{Arabic}
\quranayah[77][49]
\end{Arabic}}
\flushleft{\begin{hindi}
उस दिन झुठलाने वालों की ख़राबी है
\end{hindi}}
\flushright{\begin{Arabic}
\quranayah[77][50]
\end{Arabic}}
\flushleft{\begin{hindi}
अब इसके बाद ये किस बात पर ईमान लाएँगे
\end{hindi}}
\chapter{An-Naba' (The Announcement)}
\begin{Arabic}
\Huge{\centerline{\basmalah}}\end{Arabic}
\flushright{\begin{Arabic}
\quranayah[78][1]
\end{Arabic}}
\flushleft{\begin{hindi}
ये लोग आपस में किस चीज़ का हाल पूछते हैं
\end{hindi}}
\flushright{\begin{Arabic}
\quranayah[78][2]
\end{Arabic}}
\flushleft{\begin{hindi}
एक बड़ी ख़बर का हाल
\end{hindi}}
\flushright{\begin{Arabic}
\quranayah[78][3]
\end{Arabic}}
\flushleft{\begin{hindi}
जिसमें लोग एख्तेलाफ कर रहे हैं
\end{hindi}}
\flushright{\begin{Arabic}
\quranayah[78][4]
\end{Arabic}}
\flushleft{\begin{hindi}
देखो उन्हें अनक़रीब ही मालूम हो जाएगा
\end{hindi}}
\flushright{\begin{Arabic}
\quranayah[78][5]
\end{Arabic}}
\flushleft{\begin{hindi}
फिर इन्हें अनक़रीब ही ज़रूर मालूम हो जाएगा
\end{hindi}}
\flushright{\begin{Arabic}
\quranayah[78][6]
\end{Arabic}}
\flushleft{\begin{hindi}
क्या हमने ज़मीन को बिछौना
\end{hindi}}
\flushright{\begin{Arabic}
\quranayah[78][7]
\end{Arabic}}
\flushleft{\begin{hindi}
और पहाड़ों को (ज़मीन) की मेख़े नहीं बनाया
\end{hindi}}
\flushright{\begin{Arabic}
\quranayah[78][8]
\end{Arabic}}
\flushleft{\begin{hindi}
और हमने तुम लोगों को जोड़ा जोड़ा पैदा किया
\end{hindi}}
\flushright{\begin{Arabic}
\quranayah[78][9]
\end{Arabic}}
\flushleft{\begin{hindi}
और तुम्हारी नींद को आराम (का बाइस) क़रार दिया
\end{hindi}}
\flushright{\begin{Arabic}
\quranayah[78][10]
\end{Arabic}}
\flushleft{\begin{hindi}
और रात को परदा बनाया
\end{hindi}}
\flushright{\begin{Arabic}
\quranayah[78][11]
\end{Arabic}}
\flushleft{\begin{hindi}
और हम ही ने दिन को (कसब) मआश (का वक्त) बनाया
\end{hindi}}
\flushright{\begin{Arabic}
\quranayah[78][12]
\end{Arabic}}
\flushleft{\begin{hindi}
और तुम्हारे ऊपर सात मज़बूत (आसमान) बनाए
\end{hindi}}
\flushright{\begin{Arabic}
\quranayah[78][13]
\end{Arabic}}
\flushleft{\begin{hindi}
और हम ही ने (सूरज) को रौशन चिराग़ बनाया
\end{hindi}}
\flushright{\begin{Arabic}
\quranayah[78][14]
\end{Arabic}}
\flushleft{\begin{hindi}
और हम ही ने बादलों से मूसलाधार पानी बरसाया
\end{hindi}}
\flushright{\begin{Arabic}
\quranayah[78][15]
\end{Arabic}}
\flushleft{\begin{hindi}
ताकि उसके ज़रिए से दाने और सबज़ी
\end{hindi}}
\flushright{\begin{Arabic}
\quranayah[78][16]
\end{Arabic}}
\flushleft{\begin{hindi}
और घने घने बाग़ पैदा करें
\end{hindi}}
\flushright{\begin{Arabic}
\quranayah[78][17]
\end{Arabic}}
\flushleft{\begin{hindi}
बेशक फैसले का दिन मुक़र्रर है
\end{hindi}}
\flushright{\begin{Arabic}
\quranayah[78][18]
\end{Arabic}}
\flushleft{\begin{hindi}
जिस दिन सूर फूँका जाएगा और तुम लोग गिरोह गिरोह हाज़िर होगे
\end{hindi}}
\flushright{\begin{Arabic}
\quranayah[78][19]
\end{Arabic}}
\flushleft{\begin{hindi}
और आसमान खोल दिए जाएँगे
\end{hindi}}
\flushright{\begin{Arabic}
\quranayah[78][20]
\end{Arabic}}
\flushleft{\begin{hindi}
तो (उसमें) दरवाज़े हो जाएँगे और पहाड़ (अपनी जगह से) चलाए जाएँगे तो रेत होकर रह जाएँगे
\end{hindi}}
\flushright{\begin{Arabic}
\quranayah[78][21]
\end{Arabic}}
\flushleft{\begin{hindi}
बेशक जहन्नुम घात में है
\end{hindi}}
\flushright{\begin{Arabic}
\quranayah[78][22]
\end{Arabic}}
\flushleft{\begin{hindi}
सरकशों का (वही) ठिकाना है
\end{hindi}}
\flushright{\begin{Arabic}
\quranayah[78][23]
\end{Arabic}}
\flushleft{\begin{hindi}
उसमें मुद्दतों पड़े झींकते रहेंगें
\end{hindi}}
\flushright{\begin{Arabic}
\quranayah[78][24]
\end{Arabic}}
\flushleft{\begin{hindi}
न वहाँ ठन्डक का मज़ा चखेंगे और न खौलते हुए पानी
\end{hindi}}
\flushright{\begin{Arabic}
\quranayah[78][25]
\end{Arabic}}
\flushleft{\begin{hindi}
और बहती हुई पीप के सिवा कुछ पीने को मिलेगा
\end{hindi}}
\flushright{\begin{Arabic}
\quranayah[78][26]
\end{Arabic}}
\flushleft{\begin{hindi}
(ये उनकी कारस्तानियों का) पूरा पूरा बदला है
\end{hindi}}
\flushright{\begin{Arabic}
\quranayah[78][27]
\end{Arabic}}
\flushleft{\begin{hindi}
बेशक ये लोग आख़ेरत के हिसाब की उम्मीद ही न रखते थे
\end{hindi}}
\flushright{\begin{Arabic}
\quranayah[78][28]
\end{Arabic}}
\flushleft{\begin{hindi}
और इन लोगो हमारी आयतों को बुरी तरह झुठलाया
\end{hindi}}
\flushright{\begin{Arabic}
\quranayah[78][29]
\end{Arabic}}
\flushleft{\begin{hindi}
और हमने हर चीज़ को लिख कर मनज़बत कर रखा है
\end{hindi}}
\flushright{\begin{Arabic}
\quranayah[78][30]
\end{Arabic}}
\flushleft{\begin{hindi}
तो अब तुम मज़ा चखो हमतो तुम पर अज़ाब ही बढ़ाते जाएँगे
\end{hindi}}
\flushright{\begin{Arabic}
\quranayah[78][31]
\end{Arabic}}
\flushleft{\begin{hindi}
बेशक परहेज़गारों के लिए बड़ी कामयाबी है
\end{hindi}}
\flushright{\begin{Arabic}
\quranayah[78][32]
\end{Arabic}}
\flushleft{\begin{hindi}
(यानि बेहश्त के) बाग़ और अंगूर
\end{hindi}}
\flushright{\begin{Arabic}
\quranayah[78][33]
\end{Arabic}}
\flushleft{\begin{hindi}
और वह औरतें जिनकी उठती हुई जवानियाँ
\end{hindi}}
\flushright{\begin{Arabic}
\quranayah[78][34]
\end{Arabic}}
\flushleft{\begin{hindi}
और बाहम हमजोलियाँ हैं और शराब के लबरेज़ साग़र
\end{hindi}}
\flushright{\begin{Arabic}
\quranayah[78][35]
\end{Arabic}}
\flushleft{\begin{hindi}
और शराब के लबरेज़ साग़र वहाँ न बेहूदा बात सुनेंगे और न झूठ
\end{hindi}}
\flushright{\begin{Arabic}
\quranayah[78][36]
\end{Arabic}}
\flushleft{\begin{hindi}
(ये) तुम्हारे परवरदिगार की तरफ से काफ़ी इनाम और सिला है
\end{hindi}}
\flushright{\begin{Arabic}
\quranayah[78][37]
\end{Arabic}}
\flushleft{\begin{hindi}
जो सारे आसमान और ज़मीन और जो इन दोनों के बीच में है सबका मालिक है बड़ा मेहरबान लोगों को उससे बात का पूरा न होगा
\end{hindi}}
\flushright{\begin{Arabic}
\quranayah[78][38]
\end{Arabic}}
\flushleft{\begin{hindi}
जिस दिन जिबरील और फरिश्ते (उसके सामने) पर बाँध कर खड़े होंगे (उस दिन) उससे कोई बात न कर सकेगा मगर जिसे ख़ुदा इजाज़त दे और वह ठिकाने की बात कहे
\end{hindi}}
\flushright{\begin{Arabic}
\quranayah[78][39]
\end{Arabic}}
\flushleft{\begin{hindi}
वह दिन बरहक़ है तो जो शख़्श चाहे अपने परवरदिगार की बारगाह में (अपना) ठिकाना बनाए
\end{hindi}}
\flushright{\begin{Arabic}
\quranayah[78][40]
\end{Arabic}}
\flushleft{\begin{hindi}
हमने तुम लोगों को अनक़रीब आने वाले अज़ाब से डरा दिया जिस दिन आदमी अपने हाथों पहले से भेजे हुए (आमाल) को देखेगा और काफ़िर कहेगा काश मैं ख़ाक हो जाता
\end{hindi}}
\chapter{An-Nazi'at (Those Who Yearn)}
\begin{Arabic}
\Huge{\centerline{\basmalah}}\end{Arabic}
\flushright{\begin{Arabic}
\quranayah[79][1]
\end{Arabic}}
\flushleft{\begin{hindi}
उन (फ़रिश्तों) की क़सम
\end{hindi}}
\flushright{\begin{Arabic}
\quranayah[79][2]
\end{Arabic}}
\flushleft{\begin{hindi}
जो (कुफ्फ़ार की रूह) डूब कर सख्ती से खींच लेते हैं
\end{hindi}}
\flushright{\begin{Arabic}
\quranayah[79][3]
\end{Arabic}}
\flushleft{\begin{hindi}
और उनकी क़सम जो (मोमिनीन की जान) आसानी से खोल देते हैं
\end{hindi}}
\flushright{\begin{Arabic}
\quranayah[79][4]
\end{Arabic}}
\flushleft{\begin{hindi}
और उनकी क़सम जो (आसमान ज़मीन के दरमियान) पैरते फिरते हैं
\end{hindi}}
\flushright{\begin{Arabic}
\quranayah[79][5]
\end{Arabic}}
\flushleft{\begin{hindi}
फिर एक के आगे बढ़ते हैं
\end{hindi}}
\flushright{\begin{Arabic}
\quranayah[79][6]
\end{Arabic}}
\flushleft{\begin{hindi}
फिर (दुनिया के) इन्तज़ाम करते हैं (उनकी क़सम) कि क़यामत हो कर रहेगी
\end{hindi}}
\flushright{\begin{Arabic}
\quranayah[79][7]
\end{Arabic}}
\flushleft{\begin{hindi}
जिस दिन ज़मीन को भूचाल आएगा फिर उसके पीछे और ज़लज़ला आएगा
\end{hindi}}
\flushright{\begin{Arabic}
\quranayah[79][8]
\end{Arabic}}
\flushleft{\begin{hindi}
उस दिन दिलों को धड़कन होगी
\end{hindi}}
\flushright{\begin{Arabic}
\quranayah[79][9]
\end{Arabic}}
\flushleft{\begin{hindi}
उनकी ऑंखें (निदामत से) झुकी हुई होंगी
\end{hindi}}
\flushright{\begin{Arabic}
\quranayah[79][10]
\end{Arabic}}
\flushleft{\begin{hindi}
कुफ्फ़ार कहते हैं कि क्या हम उलटे पाँव (ज़िन्दगी की तरफ़) फिर लौटेंगे
\end{hindi}}
\flushright{\begin{Arabic}
\quranayah[79][11]
\end{Arabic}}
\flushleft{\begin{hindi}
क्या जब हम खोखल हड्डियाँ हो जाएँगे
\end{hindi}}
\flushright{\begin{Arabic}
\quranayah[79][12]
\end{Arabic}}
\flushleft{\begin{hindi}
कहते हैं कि ये लौटना तो बड़ा नुक़सान देह है
\end{hindi}}
\flushright{\begin{Arabic}
\quranayah[79][13]
\end{Arabic}}
\flushleft{\begin{hindi}
वह (क़यामत) तो (गोया) बस एक सख्त चीख़ होगी
\end{hindi}}
\flushright{\begin{Arabic}
\quranayah[79][14]
\end{Arabic}}
\flushleft{\begin{hindi}
और लोग शक़ बारगी एक मैदान (हश्र) में मौजूद होंगे
\end{hindi}}
\flushright{\begin{Arabic}
\quranayah[79][15]
\end{Arabic}}
\flushleft{\begin{hindi}
(ऐ रसूल) क्या तुम्हारे पास मूसा का किस्सा भी पहुँचा है
\end{hindi}}
\flushright{\begin{Arabic}
\quranayah[79][16]
\end{Arabic}}
\flushleft{\begin{hindi}
जब उनको परवरदिगार ने तूवा के मैदान में पुकारा
\end{hindi}}
\flushright{\begin{Arabic}
\quranayah[79][17]
\end{Arabic}}
\flushleft{\begin{hindi}
कि फिरऔन के पास जाओ वह सरकश हो गया है
\end{hindi}}
\flushright{\begin{Arabic}
\quranayah[79][18]
\end{Arabic}}
\flushleft{\begin{hindi}
(और उससे) कहो कि क्या तेरी ख्वाहिश है कि (कुफ्र से) पाक हो जाए
\end{hindi}}
\flushright{\begin{Arabic}
\quranayah[79][19]
\end{Arabic}}
\flushleft{\begin{hindi}
और मैं तुझे तेरे परवरदिगार की राह बता दूँ तो तुझको ख़ौफ (पैदा) हो
\end{hindi}}
\flushright{\begin{Arabic}
\quranayah[79][20]
\end{Arabic}}
\flushleft{\begin{hindi}
ग़रज़ मूसा ने उसे (असा का बड़ा) मौजिज़ा दिखाया
\end{hindi}}
\flushright{\begin{Arabic}
\quranayah[79][21]
\end{Arabic}}
\flushleft{\begin{hindi}
तो उसने झुठला दिया और न माना
\end{hindi}}
\flushright{\begin{Arabic}
\quranayah[79][22]
\end{Arabic}}
\flushleft{\begin{hindi}
फिर पीठ फेर कर (ख़िलाफ़ की) तदबीर करने लगा
\end{hindi}}
\flushright{\begin{Arabic}
\quranayah[79][23]
\end{Arabic}}
\flushleft{\begin{hindi}
फिर (लोगों को) जमा किया और बुलन्द आवाज़ से चिल्लाया
\end{hindi}}
\flushright{\begin{Arabic}
\quranayah[79][24]
\end{Arabic}}
\flushleft{\begin{hindi}
तो कहने लगा मैं तुम लोगों का सबसे बड़ा परवरदिगार हूँ
\end{hindi}}
\flushright{\begin{Arabic}
\quranayah[79][25]
\end{Arabic}}
\flushleft{\begin{hindi}
तो ख़ुदा ने उसे दुनिया और आख़ेरत (दोनों) के अज़ाब में गिरफ्तार किया
\end{hindi}}
\flushright{\begin{Arabic}
\quranayah[79][26]
\end{Arabic}}
\flushleft{\begin{hindi}
बेशक जो शख़्श (ख़ुदा से) डरे उसके लिए इस (किस्से) में इबरत है
\end{hindi}}
\flushright{\begin{Arabic}
\quranayah[79][27]
\end{Arabic}}
\flushleft{\begin{hindi}
भला तुम्हारा पैदा करना ज्यादा मुश्किल है या आसमान का
\end{hindi}}
\flushright{\begin{Arabic}
\quranayah[79][28]
\end{Arabic}}
\flushleft{\begin{hindi}
कि उसी ने उसको बनाया उसकी छत को ख़ूब ऊँचा रखा
\end{hindi}}
\flushright{\begin{Arabic}
\quranayah[79][29]
\end{Arabic}}
\flushleft{\begin{hindi}
फिर उसे दुरूस्त किया और उसकी रात को तारीक बनाया और (दिन को) उसकी धूप निकाली
\end{hindi}}
\flushright{\begin{Arabic}
\quranayah[79][30]
\end{Arabic}}
\flushleft{\begin{hindi}
और उसके बाद ज़मीन को फैलाया
\end{hindi}}
\flushright{\begin{Arabic}
\quranayah[79][31]
\end{Arabic}}
\flushleft{\begin{hindi}
उसी में से उसका पानी और उसका चारा निकाला
\end{hindi}}
\flushright{\begin{Arabic}
\quranayah[79][32]
\end{Arabic}}
\flushleft{\begin{hindi}
और पहाड़ों को उसमें गाड़ दिया
\end{hindi}}
\flushright{\begin{Arabic}
\quranayah[79][33]
\end{Arabic}}
\flushleft{\begin{hindi}
(ये सब सामान) तुम्हारे और तुम्हारे चारपायो के फ़ायदे के लिए है
\end{hindi}}
\flushright{\begin{Arabic}
\quranayah[79][34]
\end{Arabic}}
\flushleft{\begin{hindi}
तो जब बड़ी सख्त मुसीबत (क़यामत) आ मौजूद होगी
\end{hindi}}
\flushright{\begin{Arabic}
\quranayah[79][35]
\end{Arabic}}
\flushleft{\begin{hindi}
जिस दिन इन्सान अपने कामों को कुछ याद करेगा
\end{hindi}}
\flushright{\begin{Arabic}
\quranayah[79][36]
\end{Arabic}}
\flushleft{\begin{hindi}
और जहन्नुम देखने वालों के सामने ज़ाहिर कर दी जाएगी
\end{hindi}}
\flushright{\begin{Arabic}
\quranayah[79][37]
\end{Arabic}}
\flushleft{\begin{hindi}
तो जिसने (दुनिया में) सर उठाया था
\end{hindi}}
\flushright{\begin{Arabic}
\quranayah[79][38]
\end{Arabic}}
\flushleft{\begin{hindi}
और दुनियावी ज़िन्दगी को तरजीह दी थी
\end{hindi}}
\flushright{\begin{Arabic}
\quranayah[79][39]
\end{Arabic}}
\flushleft{\begin{hindi}
उसका ठिकाना तो यक़ीनन दोज़ख़ है
\end{hindi}}
\flushright{\begin{Arabic}
\quranayah[79][40]
\end{Arabic}}
\flushleft{\begin{hindi}
मगर जो शख़्श अपने परवरदिगार के सामने खड़े होने से डरता और जी को नाजायज़ ख्वाहिशों से रोकता रहा
\end{hindi}}
\flushright{\begin{Arabic}
\quranayah[79][41]
\end{Arabic}}
\flushleft{\begin{hindi}
तो उसका ठिकाना यक़ीनन बेहश्त है
\end{hindi}}
\flushright{\begin{Arabic}
\quranayah[79][42]
\end{Arabic}}
\flushleft{\begin{hindi}
(ऐ रसूल) लोग तुम से क़यामत के बारे में पूछते हैं
\end{hindi}}
\flushright{\begin{Arabic}
\quranayah[79][43]
\end{Arabic}}
\flushleft{\begin{hindi}
कि उसका कहीं थल बेड़ा भी है
\end{hindi}}
\flushright{\begin{Arabic}
\quranayah[79][44]
\end{Arabic}}
\flushleft{\begin{hindi}
तो तुम उसके ज़िक्र से किस फ़िक्र में हो
\end{hindi}}
\flushright{\begin{Arabic}
\quranayah[79][45]
\end{Arabic}}
\flushleft{\begin{hindi}
उस (के इल्म) की इन्तेहा तुम्हारे परवरदिगार ही तक है तो तुम बस जो उससे डरे उसको डराने वाले हो
\end{hindi}}
\flushright{\begin{Arabic}
\quranayah[79][46]
\end{Arabic}}
\flushleft{\begin{hindi}
जिस दिन वह लोग इसको देखेंगे तो (समझेंगे कि दुनिया में) बस एक शाम या सुबह ठहरे थे
\end{hindi}}
\chapter{'Abasa (He Frowned)}
\begin{Arabic}
\Huge{\centerline{\basmalah}}\end{Arabic}
\flushright{\begin{Arabic}
\quranayah[80][1]
\end{Arabic}}
\flushleft{\begin{hindi}
वह अपनी बात पर चीं ब जबीं हो गया
\end{hindi}}
\flushright{\begin{Arabic}
\quranayah[80][2]
\end{Arabic}}
\flushleft{\begin{hindi}
और मुँह फेर बैठा कि उसके पास नाबीना आ गया
\end{hindi}}
\flushright{\begin{Arabic}
\quranayah[80][3]
\end{Arabic}}
\flushleft{\begin{hindi}
और तुमको क्या मालूम यायद वह (तालीम से) पाकीज़गी हासिल करता
\end{hindi}}
\flushright{\begin{Arabic}
\quranayah[80][4]
\end{Arabic}}
\flushleft{\begin{hindi}
या वह नसीहत सुनता तो नसीहत उसके काम आती
\end{hindi}}
\flushright{\begin{Arabic}
\quranayah[80][5]
\end{Arabic}}
\flushleft{\begin{hindi}
तो जो कुछ परवाह नहीं करता
\end{hindi}}
\flushright{\begin{Arabic}
\quranayah[80][6]
\end{Arabic}}
\flushleft{\begin{hindi}
उसके तो तुम दरपै हो जाते हो हालॉकि अगर वह न सुधरे
\end{hindi}}
\flushright{\begin{Arabic}
\quranayah[80][7]
\end{Arabic}}
\flushleft{\begin{hindi}
तो तुम ज़िम्मेदार नहीं
\end{hindi}}
\flushright{\begin{Arabic}
\quranayah[80][8]
\end{Arabic}}
\flushleft{\begin{hindi}
और जो तुम्हारे पास लपकता हुआ आता है
\end{hindi}}
\flushright{\begin{Arabic}
\quranayah[80][9]
\end{Arabic}}
\flushleft{\begin{hindi}
और (ख़ुदा से) डरता है
\end{hindi}}
\flushright{\begin{Arabic}
\quranayah[80][10]
\end{Arabic}}
\flushleft{\begin{hindi}
तो तुम उससे बेरूख़ी करते हो
\end{hindi}}
\flushright{\begin{Arabic}
\quranayah[80][11]
\end{Arabic}}
\flushleft{\begin{hindi}
देखो ये (क़ुरान) तो सरासर नसीहत है
\end{hindi}}
\flushright{\begin{Arabic}
\quranayah[80][12]
\end{Arabic}}
\flushleft{\begin{hindi}
तो जो चाहे इसे याद रखे
\end{hindi}}
\flushright{\begin{Arabic}
\quranayah[80][13]
\end{Arabic}}
\flushleft{\begin{hindi}
(लौहे महफूज़ के) बहुत मोअज़ज़िज औराक़ में (लिखा हुआ) है
\end{hindi}}
\flushright{\begin{Arabic}
\quranayah[80][14]
\end{Arabic}}
\flushleft{\begin{hindi}
बुलन्द मरतबा और पाक हैं
\end{hindi}}
\flushright{\begin{Arabic}
\quranayah[80][15]
\end{Arabic}}
\flushleft{\begin{hindi}
(ऐसे) लिखने वालों के हाथों में है
\end{hindi}}
\flushright{\begin{Arabic}
\quranayah[80][16]
\end{Arabic}}
\flushleft{\begin{hindi}
जो बुज़ुर्ग नेकोकार हैं
\end{hindi}}
\flushright{\begin{Arabic}
\quranayah[80][17]
\end{Arabic}}
\flushleft{\begin{hindi}
इन्सान हलाक हो जाए वह क्या कैसा नाशुक्रा है
\end{hindi}}
\flushright{\begin{Arabic}
\quranayah[80][18]
\end{Arabic}}
\flushleft{\begin{hindi}
(ख़ुदा ने) उसे किस चीज़ से पैदा किया
\end{hindi}}
\flushright{\begin{Arabic}
\quranayah[80][19]
\end{Arabic}}
\flushleft{\begin{hindi}
नुत्फे से उसे पैदा किया फिर उसका अन्दाज़ा मुक़र्रर किया
\end{hindi}}
\flushright{\begin{Arabic}
\quranayah[80][20]
\end{Arabic}}
\flushleft{\begin{hindi}
फिर उसका रास्ता आसान कर दिया
\end{hindi}}
\flushright{\begin{Arabic}
\quranayah[80][21]
\end{Arabic}}
\flushleft{\begin{hindi}
फिर उसे मौत दी फिर उसे कब्र में दफ़न कराया
\end{hindi}}
\flushright{\begin{Arabic}
\quranayah[80][22]
\end{Arabic}}
\flushleft{\begin{hindi}
फिर जब चाहेगा उठा खड़ा करेगा
\end{hindi}}
\flushright{\begin{Arabic}
\quranayah[80][23]
\end{Arabic}}
\flushleft{\begin{hindi}
सच तो यह है कि ख़ुदा ने जो हुक्म उसे दिया उसने उसको पूरा न किया
\end{hindi}}
\flushright{\begin{Arabic}
\quranayah[80][24]
\end{Arabic}}
\flushleft{\begin{hindi}
तो इन्सान को अपने घाटे ही तरफ ग़ौर करना चाहिए
\end{hindi}}
\flushright{\begin{Arabic}
\quranayah[80][25]
\end{Arabic}}
\flushleft{\begin{hindi}
कि हम ही ने (बादल) से पानी बरसाया
\end{hindi}}
\flushright{\begin{Arabic}
\quranayah[80][26]
\end{Arabic}}
\flushleft{\begin{hindi}
फिर हम ही ने ज़मीन (दरख्त उगाकर) चीरी फाड़ी
\end{hindi}}
\flushright{\begin{Arabic}
\quranayah[80][27]
\end{Arabic}}
\flushleft{\begin{hindi}
फिर हमने उसमें अनाज उगाया
\end{hindi}}
\flushright{\begin{Arabic}
\quranayah[80][28]
\end{Arabic}}
\flushleft{\begin{hindi}
और अंगूर और तरकारियाँ
\end{hindi}}
\flushright{\begin{Arabic}
\quranayah[80][29]
\end{Arabic}}
\flushleft{\begin{hindi}
और ज़ैतून और खजूरें
\end{hindi}}
\flushright{\begin{Arabic}
\quranayah[80][30]
\end{Arabic}}
\flushleft{\begin{hindi}
और घने घने बाग़ और मेवे
\end{hindi}}
\flushright{\begin{Arabic}
\quranayah[80][31]
\end{Arabic}}
\flushleft{\begin{hindi}
और चारा (ये सब कुछ) तुम्हारे और तुम्हारे
\end{hindi}}
\flushright{\begin{Arabic}
\quranayah[80][32]
\end{Arabic}}
\flushleft{\begin{hindi}
चारपायों के फायदे के लिए (बनाया)
\end{hindi}}
\flushright{\begin{Arabic}
\quranayah[80][33]
\end{Arabic}}
\flushleft{\begin{hindi}
तो जब कानों के परदे फाड़ने वाली (क़यामत) आ मौजूद होगी
\end{hindi}}
\flushright{\begin{Arabic}
\quranayah[80][34]
\end{Arabic}}
\flushleft{\begin{hindi}
उस दिन आदमी अपने भाई
\end{hindi}}
\flushright{\begin{Arabic}
\quranayah[80][35]
\end{Arabic}}
\flushleft{\begin{hindi}
और अपनी माँ और अपने बाप
\end{hindi}}
\flushright{\begin{Arabic}
\quranayah[80][36]
\end{Arabic}}
\flushleft{\begin{hindi}
और अपने लड़के बालों से भागेगा
\end{hindi}}
\flushright{\begin{Arabic}
\quranayah[80][37]
\end{Arabic}}
\flushleft{\begin{hindi}
उस दिन हर शख़्श (अपनी नजात की) ऐसी फ़िक्र में होगा जो उसके (मशग़ूल होने के) लिए काफ़ी हों
\end{hindi}}
\flushright{\begin{Arabic}
\quranayah[80][38]
\end{Arabic}}
\flushleft{\begin{hindi}
बहुत से चेहरे तो उस दिन चमकते होंगे
\end{hindi}}
\flushright{\begin{Arabic}
\quranayah[80][39]
\end{Arabic}}
\flushleft{\begin{hindi}
ख़न्दाँ शांदाँ (यही नेको कार हैं)
\end{hindi}}
\flushright{\begin{Arabic}
\quranayah[80][40]
\end{Arabic}}
\flushleft{\begin{hindi}
और बहुत से चेहरे ऐसे होंगे जिन पर गर्द पड़ी होगी
\end{hindi}}
\flushright{\begin{Arabic}
\quranayah[80][41]
\end{Arabic}}
\flushleft{\begin{hindi}
उस पर सियाही छाई हुई होगी
\end{hindi}}
\flushright{\begin{Arabic}
\quranayah[80][42]
\end{Arabic}}
\flushleft{\begin{hindi}
यही कुफ्फ़ार बदकार हैं
\end{hindi}}
\chapter{At-Takwir (The Folding Up)}
\begin{Arabic}
\Huge{\centerline{\basmalah}}\end{Arabic}
\flushright{\begin{Arabic}
\quranayah[81][1]
\end{Arabic}}
\flushleft{\begin{hindi}
जिस वक्त आफ़ताब की चादर को लपेट लिया जाएगा
\end{hindi}}
\flushright{\begin{Arabic}
\quranayah[81][2]
\end{Arabic}}
\flushleft{\begin{hindi}
और जिस वक्त तारे गिर पडेग़ें
\end{hindi}}
\flushright{\begin{Arabic}
\quranayah[81][3]
\end{Arabic}}
\flushleft{\begin{hindi}
और जब पहाड़ चलाए जाएंगें
\end{hindi}}
\flushright{\begin{Arabic}
\quranayah[81][4]
\end{Arabic}}
\flushleft{\begin{hindi}
और जब अनक़रीब जनने वाली ऊंटनियों बेकार कर दी जाएंगी
\end{hindi}}
\flushright{\begin{Arabic}
\quranayah[81][5]
\end{Arabic}}
\flushleft{\begin{hindi}
और जिस वक्त वहशी जानवर इकट्ठा किये जायेंगे
\end{hindi}}
\flushright{\begin{Arabic}
\quranayah[81][6]
\end{Arabic}}
\flushleft{\begin{hindi}
और जिस वक्त दरिया आग हो जायेंगे
\end{hindi}}
\flushright{\begin{Arabic}
\quranayah[81][7]
\end{Arabic}}
\flushleft{\begin{hindi}
और जिस वक्त रुहें हवियों से मिला दी जाएंगी
\end{hindi}}
\flushright{\begin{Arabic}
\quranayah[81][8]
\end{Arabic}}
\flushleft{\begin{hindi}
और जिस वक्त ज़िन्दा दर गोर लड़की से पूछा जाएगा
\end{hindi}}
\flushright{\begin{Arabic}
\quranayah[81][9]
\end{Arabic}}
\flushleft{\begin{hindi}
कि वह किस गुनाह के बदले मारी गयी
\end{hindi}}
\flushright{\begin{Arabic}
\quranayah[81][10]
\end{Arabic}}
\flushleft{\begin{hindi}
और जिस वक्त (आमाल के) दफ्तर खोले जाएं
\end{hindi}}
\flushright{\begin{Arabic}
\quranayah[81][11]
\end{Arabic}}
\flushleft{\begin{hindi}
और जिस वक्त आसमान का छिलका उतारा जाएगा
\end{hindi}}
\flushright{\begin{Arabic}
\quranayah[81][12]
\end{Arabic}}
\flushleft{\begin{hindi}
और जब दोज़ख़ (की आग) भड़कायी जाएगी
\end{hindi}}
\flushright{\begin{Arabic}
\quranayah[81][13]
\end{Arabic}}
\flushleft{\begin{hindi}
और जब बेहिश्त क़रीब कर दी जाएगी
\end{hindi}}
\flushright{\begin{Arabic}
\quranayah[81][14]
\end{Arabic}}
\flushleft{\begin{hindi}
तब हर शख़्श मालूम करेगा कि वह क्या (आमाल) लेकर आया
\end{hindi}}
\flushright{\begin{Arabic}
\quranayah[81][15]
\end{Arabic}}
\flushleft{\begin{hindi}
तो मुझे उन सितारों की क़सम जो चलते चलते पीछे हट जाते
\end{hindi}}
\flushright{\begin{Arabic}
\quranayah[81][16]
\end{Arabic}}
\flushleft{\begin{hindi}
और ग़ायब होते हैं
\end{hindi}}
\flushright{\begin{Arabic}
\quranayah[81][17]
\end{Arabic}}
\flushleft{\begin{hindi}
और रात की क़सम जब ख़त्म होने को आए
\end{hindi}}
\flushright{\begin{Arabic}
\quranayah[81][18]
\end{Arabic}}
\flushleft{\begin{hindi}
और सुबह की क़सम जब रौशन हो जाए
\end{hindi}}
\flushright{\begin{Arabic}
\quranayah[81][19]
\end{Arabic}}
\flushleft{\begin{hindi}
कि बेशक यें (क़ुरान) एक मुअज़िज़ फरिश्ता (जिबरील की ज़बान का पैग़ाम है
\end{hindi}}
\flushright{\begin{Arabic}
\quranayah[81][20]
\end{Arabic}}
\flushleft{\begin{hindi}
जो बड़े क़वी अर्श के मालिक की बारगाह में बुलन्द रुतबा है
\end{hindi}}
\flushright{\begin{Arabic}
\quranayah[81][21]
\end{Arabic}}
\flushleft{\begin{hindi}
वहाँ (सब फरिश्तों का) सरदार अमानतदार है
\end{hindi}}
\flushright{\begin{Arabic}
\quranayah[81][22]
\end{Arabic}}
\flushleft{\begin{hindi}
और (मक्के वालों) तुम्हारे साथी मोहम्मद दीवाने नहीं हैं
\end{hindi}}
\flushright{\begin{Arabic}
\quranayah[81][23]
\end{Arabic}}
\flushleft{\begin{hindi}
और बेशक उन्होनें जिबरील को (आसमान के) खुले (शरक़ी) किनारे पर देखा है
\end{hindi}}
\flushright{\begin{Arabic}
\quranayah[81][24]
\end{Arabic}}
\flushleft{\begin{hindi}
और वह ग़ैब की बातों के ज़ाहिर करने में बख़ील नहीं
\end{hindi}}
\flushright{\begin{Arabic}
\quranayah[81][25]
\end{Arabic}}
\flushleft{\begin{hindi}
और न यह मरदूद शैतान का क़ौल है
\end{hindi}}
\flushright{\begin{Arabic}
\quranayah[81][26]
\end{Arabic}}
\flushleft{\begin{hindi}
फिर तुम कहाँ जाते हो
\end{hindi}}
\flushright{\begin{Arabic}
\quranayah[81][27]
\end{Arabic}}
\flushleft{\begin{hindi}
ये सारे जहॉन के लोगों के लिए बस नसीहत है
\end{hindi}}
\flushright{\begin{Arabic}
\quranayah[81][28]
\end{Arabic}}
\flushleft{\begin{hindi}
(मगर) उसी के लिए जो तुममें सीधी राह चले
\end{hindi}}
\flushright{\begin{Arabic}
\quranayah[81][29]
\end{Arabic}}
\flushleft{\begin{hindi}
और तुम तो सारे जहॉन के पालने वाले ख़ुदा के चाहे बग़ैर कुछ भी चाह नहीं सकते
\end{hindi}}
\chapter{Al-Infitar (The Cleaving)}
\begin{Arabic}
\Huge{\centerline{\basmalah}}\end{Arabic}
\flushright{\begin{Arabic}
\quranayah[82][1]
\end{Arabic}}
\flushleft{\begin{hindi}
जब आसमान तर्ख़ जाएगा
\end{hindi}}
\flushright{\begin{Arabic}
\quranayah[82][2]
\end{Arabic}}
\flushleft{\begin{hindi}
और जब तारे झड़ पड़ेंगे
\end{hindi}}
\flushright{\begin{Arabic}
\quranayah[82][3]
\end{Arabic}}
\flushleft{\begin{hindi}
और जब दरिया बह (कर एक दूसरे से मिल) जाएँगे
\end{hindi}}
\flushright{\begin{Arabic}
\quranayah[82][4]
\end{Arabic}}
\flushleft{\begin{hindi}
और जब कब्रें उखाड़ दी जाएँगी
\end{hindi}}
\flushright{\begin{Arabic}
\quranayah[82][5]
\end{Arabic}}
\flushleft{\begin{hindi}
तब हर शख़्श को मालूम हो जाएगा कि उसने आगे क्या भेजा था और पीछे क्या छोड़ा था
\end{hindi}}
\flushright{\begin{Arabic}
\quranayah[82][6]
\end{Arabic}}
\flushleft{\begin{hindi}
ऐ इन्सान तुम्हें अपने परवरदिगार के बारे में किस चीज़ ने धोका दिया
\end{hindi}}
\flushright{\begin{Arabic}
\quranayah[82][7]
\end{Arabic}}
\flushleft{\begin{hindi}
जिसने तुझे पैदा किया तो तुझे दुरूस्त बनाया और मुनासिब आज़ा दिए
\end{hindi}}
\flushright{\begin{Arabic}
\quranayah[82][8]
\end{Arabic}}
\flushleft{\begin{hindi}
और जिस सूरत में उसने चाहा तेरे जोड़ बन्द मिलाए
\end{hindi}}
\flushright{\begin{Arabic}
\quranayah[82][9]
\end{Arabic}}
\flushleft{\begin{hindi}
हाँ बात ये है कि तुम लोग जज़ा (के दिन) को झुठलाते हो
\end{hindi}}
\flushright{\begin{Arabic}
\quranayah[82][10]
\end{Arabic}}
\flushleft{\begin{hindi}
हालॉकि तुम पर निगेहबान मुक़र्रर हैं
\end{hindi}}
\flushright{\begin{Arabic}
\quranayah[82][11]
\end{Arabic}}
\flushleft{\begin{hindi}
बुर्ज़ुग लोग (फरिश्ते सब बातों को) लिखने वाले (केरामन क़ातेबीन)
\end{hindi}}
\flushright{\begin{Arabic}
\quranayah[82][12]
\end{Arabic}}
\flushleft{\begin{hindi}
जो कुछ तुम करते हो वह सब जानते हैं
\end{hindi}}
\flushright{\begin{Arabic}
\quranayah[82][13]
\end{Arabic}}
\flushleft{\begin{hindi}
बेशक नेको कार (बेहिश्त की) नेअमतों में होंगे
\end{hindi}}
\flushright{\begin{Arabic}
\quranayah[82][14]
\end{Arabic}}
\flushleft{\begin{hindi}
और बदकार लोग यक़ीनन जहन्नुम में जज़ा के दिन
\end{hindi}}
\flushright{\begin{Arabic}
\quranayah[82][15]
\end{Arabic}}
\flushleft{\begin{hindi}
उसी में झोंके जाएँगे
\end{hindi}}
\flushright{\begin{Arabic}
\quranayah[82][16]
\end{Arabic}}
\flushleft{\begin{hindi}
और वह लोग उससे छुप न सकेंगे
\end{hindi}}
\flushright{\begin{Arabic}
\quranayah[82][17]
\end{Arabic}}
\flushleft{\begin{hindi}
और तुम्हें क्या मालूम कि जज़ा का दिन क्या है
\end{hindi}}
\flushright{\begin{Arabic}
\quranayah[82][18]
\end{Arabic}}
\flushleft{\begin{hindi}
फिर तुम्हें क्या मालूम कि जज़ा का दिन क्या चीज़ है
\end{hindi}}
\flushright{\begin{Arabic}
\quranayah[82][19]
\end{Arabic}}
\flushleft{\begin{hindi}
उस दिन कोई शख़्श किसी शख़्श की भलाई न कर सकेगा और उस दिन हुक्म सिर्फ ख़ुदा ही का होगा
\end{hindi}}
\chapter{At-Tatfif (Default in Duty)}
\begin{Arabic}
\Huge{\centerline{\basmalah}}\end{Arabic}
\flushright{\begin{Arabic}
\quranayah[83][1]
\end{Arabic}}
\flushleft{\begin{hindi}
नाप तौल में कमी करने वालों की ख़राबी है
\end{hindi}}
\flushright{\begin{Arabic}
\quranayah[83][2]
\end{Arabic}}
\flushleft{\begin{hindi}
जो औरें से नाप कर लें तो पूरा पूरा लें
\end{hindi}}
\flushright{\begin{Arabic}
\quranayah[83][3]
\end{Arabic}}
\flushleft{\begin{hindi}
और जब उनकी नाप या तौल कर दें तो कम कर दें
\end{hindi}}
\flushright{\begin{Arabic}
\quranayah[83][4]
\end{Arabic}}
\flushleft{\begin{hindi}
क्या ये लोग इतना भी ख्याल नहीं करते
\end{hindi}}
\flushright{\begin{Arabic}
\quranayah[83][5]
\end{Arabic}}
\flushleft{\begin{hindi}
कि एक बड़े (सख्त) दिन (क़यामत) में उठाए जाएँगे
\end{hindi}}
\flushright{\begin{Arabic}
\quranayah[83][6]
\end{Arabic}}
\flushleft{\begin{hindi}
जिस दिन तमाम लोग सारे जहाँन के परवरदिगार के सामने खड़े होंगे
\end{hindi}}
\flushright{\begin{Arabic}
\quranayah[83][7]
\end{Arabic}}
\flushleft{\begin{hindi}
सुन रखो कि बदकारों के नाम ए अमाल सिज्जीन में हैं
\end{hindi}}
\flushright{\begin{Arabic}
\quranayah[83][8]
\end{Arabic}}
\flushleft{\begin{hindi}
तुमको क्या मालूम सिज्जीन क्या चीज़ है
\end{hindi}}
\flushright{\begin{Arabic}
\quranayah[83][9]
\end{Arabic}}
\flushleft{\begin{hindi}
एक लिखा हुआ दफ़तर है जिसमें शयातीन के (आमाल दर्ज हैं)
\end{hindi}}
\flushright{\begin{Arabic}
\quranayah[83][10]
\end{Arabic}}
\flushleft{\begin{hindi}
उस दिन झुठलाने वालों की ख़राबी है
\end{hindi}}
\flushright{\begin{Arabic}
\quranayah[83][11]
\end{Arabic}}
\flushleft{\begin{hindi}
जो लोग रोजे ज़ज़ा को झुठलाते हैं
\end{hindi}}
\flushright{\begin{Arabic}
\quranayah[83][12]
\end{Arabic}}
\flushleft{\begin{hindi}
हालॉकि उसको हद से निकल जाने वाले गुनाहगार के सिवा कोई नहीं झुठलाता
\end{hindi}}
\flushright{\begin{Arabic}
\quranayah[83][13]
\end{Arabic}}
\flushleft{\begin{hindi}
जब उसके सामने हमारी आयतें पढ़ी जाती हैं तो कहता है कि ये तो अगलों के अफसाने हैं
\end{hindi}}
\flushright{\begin{Arabic}
\quranayah[83][14]
\end{Arabic}}
\flushleft{\begin{hindi}
नहीं नहीं बात ये है कि ये लोग जो आमाल (बद) करते हैं उनका उनके दिलों पर जंग बैठ गया है
\end{hindi}}
\flushright{\begin{Arabic}
\quranayah[83][15]
\end{Arabic}}
\flushleft{\begin{hindi}
बेशक ये लोग उस दिन अपने परवरदिगार (की रहमत से) रोक दिए जाएँगे
\end{hindi}}
\flushright{\begin{Arabic}
\quranayah[83][16]
\end{Arabic}}
\flushleft{\begin{hindi}
फिर ये लोग ज़रूर जहन्नुम वासिल होंगे
\end{hindi}}
\flushright{\begin{Arabic}
\quranayah[83][17]
\end{Arabic}}
\flushleft{\begin{hindi}
फिर उनसे कहा जाएगा कि ये वही चीज़ तो है जिसे तुम झुठलाया करते थे
\end{hindi}}
\flushright{\begin{Arabic}
\quranayah[83][18]
\end{Arabic}}
\flushleft{\begin{hindi}
ये भी सुन रखो कि नेको के नाम ए अमाल इल्लीयीन में होंगे
\end{hindi}}
\flushright{\begin{Arabic}
\quranayah[83][19]
\end{Arabic}}
\flushleft{\begin{hindi}
और तुमको क्या मालूम कि इल्लीयीन क्या है वह एक लिखा हुआ दफ़तर है
\end{hindi}}
\flushright{\begin{Arabic}
\quranayah[83][20]
\end{Arabic}}
\flushleft{\begin{hindi}
जिसमें नेकों के आमाल दर्ज हैं
\end{hindi}}
\flushright{\begin{Arabic}
\quranayah[83][21]
\end{Arabic}}
\flushleft{\begin{hindi}
उसके पास मुक़र्रिब (फ़रिश्ते) हाज़िर हैं
\end{hindi}}
\flushright{\begin{Arabic}
\quranayah[83][22]
\end{Arabic}}
\flushleft{\begin{hindi}
बेशक नेक लोग नेअमतों में होंगे
\end{hindi}}
\flushright{\begin{Arabic}
\quranayah[83][23]
\end{Arabic}}
\flushleft{\begin{hindi}
तख्तों पर बैठे नज़ारे करेंगे
\end{hindi}}
\flushright{\begin{Arabic}
\quranayah[83][24]
\end{Arabic}}
\flushleft{\begin{hindi}
तुम उनके चेहरों ही से राहत की ताज़गी मालूम कर लोगे
\end{hindi}}
\flushright{\begin{Arabic}
\quranayah[83][25]
\end{Arabic}}
\flushleft{\begin{hindi}
उनको सर ब मोहर ख़ालिस शराब पिलायी जाएगी
\end{hindi}}
\flushright{\begin{Arabic}
\quranayah[83][26]
\end{Arabic}}
\flushleft{\begin{hindi}
जिसकी मोहर मिश्क की होगी और उसकी तरफ अलबत्ता शायक़ीन को रग़बत करनी चाहिए
\end{hindi}}
\flushright{\begin{Arabic}
\quranayah[83][27]
\end{Arabic}}
\flushleft{\begin{hindi}
और उस (शराब) में तसनीम के पानी की आमेज़िश होगी
\end{hindi}}
\flushright{\begin{Arabic}
\quranayah[83][28]
\end{Arabic}}
\flushleft{\begin{hindi}
वह एक चश्मा है जिसमें मुक़रेबीन पियेंगे
\end{hindi}}
\flushright{\begin{Arabic}
\quranayah[83][29]
\end{Arabic}}
\flushleft{\begin{hindi}
बेशक जो गुनाहगार मोमिनों से हँसी किया करते थे
\end{hindi}}
\flushright{\begin{Arabic}
\quranayah[83][30]
\end{Arabic}}
\flushleft{\begin{hindi}
और जब उनके पास से गुज़रते तो उन पर चशमक करते थे
\end{hindi}}
\flushright{\begin{Arabic}
\quranayah[83][31]
\end{Arabic}}
\flushleft{\begin{hindi}
और जब अपने लड़के वालों की तरफ़ लौट कर आते थे तो इतराते हुए
\end{hindi}}
\flushright{\begin{Arabic}
\quranayah[83][32]
\end{Arabic}}
\flushleft{\begin{hindi}
और जब उन मोमिनीन को देखते तो कह बैठते थे कि ये तो यक़ीनी गुमराह हैं
\end{hindi}}
\flushright{\begin{Arabic}
\quranayah[83][33]
\end{Arabic}}
\flushleft{\begin{hindi}
हालॉकि ये लोग उन पर कुछ निगराँ बना के तो भेजे नहीं गए थे
\end{hindi}}
\flushright{\begin{Arabic}
\quranayah[83][34]
\end{Arabic}}
\flushleft{\begin{hindi}
तो आज (क़यामत में) ईमानदार लोग काफ़िरों से हँसी करेंगे
\end{hindi}}
\flushright{\begin{Arabic}
\quranayah[83][35]
\end{Arabic}}
\flushleft{\begin{hindi}
(और) तख्तों पर बैठे नज़ारे करेंगे
\end{hindi}}
\flushright{\begin{Arabic}
\quranayah[83][36]
\end{Arabic}}
\flushleft{\begin{hindi}
कि अब तो काफ़िरों को उनके किए का पूरा पूरा बदला मिल गया
\end{hindi}}
\chapter{Al-Inshiqaq (The Bursting Asunder)}
\begin{Arabic}
\Huge{\centerline{\basmalah}}\end{Arabic}
\flushright{\begin{Arabic}
\quranayah[84][1]
\end{Arabic}}
\flushleft{\begin{hindi}
जब आसमान फट जाएगा
\end{hindi}}
\flushright{\begin{Arabic}
\quranayah[84][2]
\end{Arabic}}
\flushleft{\begin{hindi}
और अपने परवरदिगार का हुक्म बजा लाएगा और उसे वाजिब भी यही है
\end{hindi}}
\flushright{\begin{Arabic}
\quranayah[84][3]
\end{Arabic}}
\flushleft{\begin{hindi}
और जब ज़मीन (बराबर करके) तान दी जाएगी
\end{hindi}}
\flushright{\begin{Arabic}
\quranayah[84][4]
\end{Arabic}}
\flushleft{\begin{hindi}
और जो कुछ उसमें है उगल देगी और बिल्कुल ख़ाली हो जाएगी
\end{hindi}}
\flushright{\begin{Arabic}
\quranayah[84][5]
\end{Arabic}}
\flushleft{\begin{hindi}
और अपने परवरदिगार का हुक्म बजा लाएगी
\end{hindi}}
\flushright{\begin{Arabic}
\quranayah[84][6]
\end{Arabic}}
\flushleft{\begin{hindi}
और उस पर लाज़िम भी यही है (तो क़यामत आ जाएगी) ऐ इन्सान तू अपने परवरदिगार की हुज़ूरी की कोशिश करता है
\end{hindi}}
\flushright{\begin{Arabic}
\quranayah[84][7]
\end{Arabic}}
\flushleft{\begin{hindi}
तो तू (एक न एक दिन) उसके सामने हाज़िर होगा फिर (उस दिन) जिसका नामाए आमाल उसके दाहिने हाथ में दिया जाएगा
\end{hindi}}
\flushright{\begin{Arabic}
\quranayah[84][8]
\end{Arabic}}
\flushleft{\begin{hindi}
उससे तो हिसाब आसान तरीके से लिया जाएगा
\end{hindi}}
\flushright{\begin{Arabic}
\quranayah[84][9]
\end{Arabic}}
\flushleft{\begin{hindi}
और (फिर) वह अपने (मोमिनीन के) क़बीले की तरफ ख़ुश ख़ुश पलटेगा
\end{hindi}}
\flushright{\begin{Arabic}
\quranayah[84][10]
\end{Arabic}}
\flushleft{\begin{hindi}
लेकिन जिस शख़्श को उसका नामए आमल उसकी पीठ के पीछे से दिया जाएगा
\end{hindi}}
\flushright{\begin{Arabic}
\quranayah[84][11]
\end{Arabic}}
\flushleft{\begin{hindi}
वह तो मौत की दुआ करेगा
\end{hindi}}
\flushright{\begin{Arabic}
\quranayah[84][12]
\end{Arabic}}
\flushleft{\begin{hindi}
और जहन्नुम वासिल होगा
\end{hindi}}
\flushright{\begin{Arabic}
\quranayah[84][13]
\end{Arabic}}
\flushleft{\begin{hindi}
ये शख़्श तो अपने लड़के बालों में मस्त रहता था
\end{hindi}}
\flushright{\begin{Arabic}
\quranayah[84][14]
\end{Arabic}}
\flushleft{\begin{hindi}
और समझता था कि कभी (ख़ुदा की तरफ) फिर कर जाएगा ही नहीं
\end{hindi}}
\flushright{\begin{Arabic}
\quranayah[84][15]
\end{Arabic}}
\flushleft{\begin{hindi}
हाँ उसका परवरदिगार यक़ीनी उसको देख भाल कर रहा है
\end{hindi}}
\flushright{\begin{Arabic}
\quranayah[84][16]
\end{Arabic}}
\flushleft{\begin{hindi}
तो मुझे शाम की मुर्ख़ी की क़सम
\end{hindi}}
\flushright{\begin{Arabic}
\quranayah[84][17]
\end{Arabic}}
\flushleft{\begin{hindi}
और रात की और उन चीज़ों की जिन्हें ये ढाँक लेती है
\end{hindi}}
\flushright{\begin{Arabic}
\quranayah[84][18]
\end{Arabic}}
\flushleft{\begin{hindi}
और चाँद की जब पूरा हो जाए
\end{hindi}}
\flushright{\begin{Arabic}
\quranayah[84][19]
\end{Arabic}}
\flushleft{\begin{hindi}
कि तुम लोग ज़रूर एक सख्ती के बाद दूसरी सख्ती में फँसोगे
\end{hindi}}
\flushright{\begin{Arabic}
\quranayah[84][20]
\end{Arabic}}
\flushleft{\begin{hindi}
तो उन लोगों को क्या हो गया है कि ईमान नहीं ईमान नहीं लाते
\end{hindi}}
\flushright{\begin{Arabic}
\quranayah[84][21]
\end{Arabic}}
\flushleft{\begin{hindi}
और जब उनके सामने क़ुरान पढ़ा जाता है तो (ख़ुदा का) सजदा नहीं करते (21) (सजदा)
\end{hindi}}
\flushright{\begin{Arabic}
\quranayah[84][22]
\end{Arabic}}
\flushleft{\begin{hindi}
बल्कि काफ़िर लोग तो (और उसे) झुठलाते हैं
\end{hindi}}
\flushright{\begin{Arabic}
\quranayah[84][23]
\end{Arabic}}
\flushleft{\begin{hindi}
और जो बातें ये लोग अपने दिलों में छिपाते हैं ख़ुदा उसे ख़ूब जानता है
\end{hindi}}
\flushright{\begin{Arabic}
\quranayah[84][24]
\end{Arabic}}
\flushleft{\begin{hindi}
तो (ऐ रसूल) उन्हें दर्दनाक अज़ाब की ख़ुशख़बरी दे दो
\end{hindi}}
\flushright{\begin{Arabic}
\quranayah[84][25]
\end{Arabic}}
\flushleft{\begin{hindi}
मगर जो लोग ईमान लाए और उन्होंने अच्छे अच्छे काम किए उनके लिए बेइन्तिहा अज्र (व सवाब है)
\end{hindi}}
\chapter{Al-Buruj (The Stars)}
\begin{Arabic}
\Huge{\centerline{\basmalah}}\end{Arabic}
\flushright{\begin{Arabic}
\quranayah[85][1]
\end{Arabic}}
\flushleft{\begin{hindi}
बुर्ज़ों वाले आसमानों की क़सम
\end{hindi}}
\flushright{\begin{Arabic}
\quranayah[85][2]
\end{Arabic}}
\flushleft{\begin{hindi}
और उस दिन की जिसका वायदा किया गया है
\end{hindi}}
\flushright{\begin{Arabic}
\quranayah[85][3]
\end{Arabic}}
\flushleft{\begin{hindi}
और गवाह की और जिसकी गवाही दे जाएगी
\end{hindi}}
\flushright{\begin{Arabic}
\quranayah[85][4]
\end{Arabic}}
\flushleft{\begin{hindi}
उसकी (कि कुफ्फ़ार मक्का हलाक हुए) जिस तरह ख़न्दक़ वाले हलाक कर दिए गए
\end{hindi}}
\flushright{\begin{Arabic}
\quranayah[85][5]
\end{Arabic}}
\flushleft{\begin{hindi}
जो ख़न्दक़ें आग की थीं
\end{hindi}}
\flushright{\begin{Arabic}
\quranayah[85][6]
\end{Arabic}}
\flushleft{\begin{hindi}
जिसमें (उन्होंने मुसलमानों के लिए) ईंधन झोंक रखा था
\end{hindi}}
\flushright{\begin{Arabic}
\quranayah[85][7]
\end{Arabic}}
\flushleft{\begin{hindi}
जब वह उन (ख़न्दक़ों) पर बैठे हुए और जो सुलूक ईमानदारों के साथ करते थे उसको सामने देख रहे थे
\end{hindi}}
\flushright{\begin{Arabic}
\quranayah[85][8]
\end{Arabic}}
\flushleft{\begin{hindi}
और उनको मोमिनीन की यही बात बुरी मालूम हुई कि वह लोग ख़ुदा पर ईमान लाए थे जो ज़बरदस्त और सज़ावार हम्द है
\end{hindi}}
\flushright{\begin{Arabic}
\quranayah[85][9]
\end{Arabic}}
\flushleft{\begin{hindi}
वह (ख़ुदा) जिसकी सारे आसमान ज़मीन में बादशाहत है और ख़ुदा हर चीज़ से वाक़िफ़ है
\end{hindi}}
\flushright{\begin{Arabic}
\quranayah[85][10]
\end{Arabic}}
\flushleft{\begin{hindi}
बेशक जिन लोगों ने ईमानदार मर्दों और औरतों को तकलीफें दीं फिर तौबा न की उनके लिए जहन्नुम का अज़ाब तो है ही (इसके अलावा) जलने का भी अज़ाब होगा
\end{hindi}}
\flushright{\begin{Arabic}
\quranayah[85][11]
\end{Arabic}}
\flushleft{\begin{hindi}
बेशक जो लोग ईमान लाए और अच्छे काम करते रहे उनके लिए वह बाग़ात हैं जिनके नीचे नहरें जारी हैं यही तो बड़ी कामयाबी है
\end{hindi}}
\flushright{\begin{Arabic}
\quranayah[85][12]
\end{Arabic}}
\flushleft{\begin{hindi}
बेशक तुम्हारे परवरदिगार की पकड़ बहुत सख्त है
\end{hindi}}
\flushright{\begin{Arabic}
\quranayah[85][13]
\end{Arabic}}
\flushleft{\begin{hindi}
वही पहली दफ़ा पैदा करता है और वही दोबारा (क़यामत में ज़िन्दा) करेगा
\end{hindi}}
\flushright{\begin{Arabic}
\quranayah[85][14]
\end{Arabic}}
\flushleft{\begin{hindi}
और वही बड़ा बख्शने वाला मोहब्बत करने वाला है
\end{hindi}}
\flushright{\begin{Arabic}
\quranayah[85][15]
\end{Arabic}}
\flushleft{\begin{hindi}
अर्श का मालिक बड़ा आलीशान है
\end{hindi}}
\flushright{\begin{Arabic}
\quranayah[85][16]
\end{Arabic}}
\flushleft{\begin{hindi}
जो चाहता है करता है
\end{hindi}}
\flushright{\begin{Arabic}
\quranayah[85][17]
\end{Arabic}}
\flushleft{\begin{hindi}
क्या तुम्हारे पास लशकरों की ख़बर पहुँची है
\end{hindi}}
\flushright{\begin{Arabic}
\quranayah[85][18]
\end{Arabic}}
\flushleft{\begin{hindi}
(यानि) फिरऔन व समूद की (ज़रूर पहुँची है)
\end{hindi}}
\flushright{\begin{Arabic}
\quranayah[85][19]
\end{Arabic}}
\flushleft{\begin{hindi}
मगर कुफ्फ़ार तो झुठलाने ही (की फ़िक्र) में हैं
\end{hindi}}
\flushright{\begin{Arabic}
\quranayah[85][20]
\end{Arabic}}
\flushleft{\begin{hindi}
और ख़ुदा उनको पीछे से घेरे हुए है (ये झुठलाने के क़ाबिल नहीं)
\end{hindi}}
\flushright{\begin{Arabic}
\quranayah[85][21]
\end{Arabic}}
\flushleft{\begin{hindi}
बल्कि ये तो क़ुरान मजीद है
\end{hindi}}
\flushright{\begin{Arabic}
\quranayah[85][22]
\end{Arabic}}
\flushleft{\begin{hindi}
जो लौहे महफूज़ में लिखा हुआ है
\end{hindi}}
\chapter{At-Tariq (The Comer by Night)}
\begin{Arabic}
\Huge{\centerline{\basmalah}}\end{Arabic}
\flushright{\begin{Arabic}
\quranayah[86][1]
\end{Arabic}}
\flushleft{\begin{hindi}
आसमान और रात को आने वाले की क़सम
\end{hindi}}
\flushright{\begin{Arabic}
\quranayah[86][2]
\end{Arabic}}
\flushleft{\begin{hindi}
और तुमको क्या मालूम रात को आने वाला क्या है
\end{hindi}}
\flushright{\begin{Arabic}
\quranayah[86][3]
\end{Arabic}}
\flushleft{\begin{hindi}
(वह) चमकता हुआ तारा है
\end{hindi}}
\flushright{\begin{Arabic}
\quranayah[86][4]
\end{Arabic}}
\flushleft{\begin{hindi}
(इस बात की क़सम) कि कोई शख़्श ऐसा नहीं जिस पर निगेहबान मुक़र्रर नहीं
\end{hindi}}
\flushright{\begin{Arabic}
\quranayah[86][5]
\end{Arabic}}
\flushleft{\begin{hindi}
तो इन्सान को देखना चाहिए कि वह किस चीज़ से पैदा हुआ हैं
\end{hindi}}
\flushright{\begin{Arabic}
\quranayah[86][6]
\end{Arabic}}
\flushleft{\begin{hindi}
वह उछलते हुए पानी (मनी) से पैदा हुआ है
\end{hindi}}
\flushright{\begin{Arabic}
\quranayah[86][7]
\end{Arabic}}
\flushleft{\begin{hindi}
जो पीठ और सीने की हड्डियों के बीच में से निकलता है
\end{hindi}}
\flushright{\begin{Arabic}
\quranayah[86][8]
\end{Arabic}}
\flushleft{\begin{hindi}
बेशक ख़ुदा उसके दोबारा (पैदा) करने पर ज़रूर कुदरत रखता है
\end{hindi}}
\flushright{\begin{Arabic}
\quranayah[86][9]
\end{Arabic}}
\flushleft{\begin{hindi}
जिस दिन दिलों के भेद जाँचे जाएँगे
\end{hindi}}
\flushright{\begin{Arabic}
\quranayah[86][10]
\end{Arabic}}
\flushleft{\begin{hindi}
तो (उस दिन) उसका न कुछ ज़ोर चलेगा और न कोई मददगार होगा
\end{hindi}}
\flushright{\begin{Arabic}
\quranayah[86][11]
\end{Arabic}}
\flushleft{\begin{hindi}
चक्कर (खाने) वाले आसमान की क़सम
\end{hindi}}
\flushright{\begin{Arabic}
\quranayah[86][12]
\end{Arabic}}
\flushleft{\begin{hindi}
और फटने वाली (ज़मीन की क़सम)
\end{hindi}}
\flushright{\begin{Arabic}
\quranayah[86][13]
\end{Arabic}}
\flushleft{\begin{hindi}
बेशक ये क़ुरान क़ौले फ़ैसल है
\end{hindi}}
\flushright{\begin{Arabic}
\quranayah[86][14]
\end{Arabic}}
\flushleft{\begin{hindi}
और लग़ो नहीं है
\end{hindi}}
\flushright{\begin{Arabic}
\quranayah[86][15]
\end{Arabic}}
\flushleft{\begin{hindi}
बेशक ये कुफ्फ़ार अपनी तदबीर कर रहे हैं
\end{hindi}}
\flushright{\begin{Arabic}
\quranayah[86][16]
\end{Arabic}}
\flushleft{\begin{hindi}
और मैं अपनी तद्बीर कर रहा हूँ
\end{hindi}}
\flushright{\begin{Arabic}
\quranayah[86][17]
\end{Arabic}}
\flushleft{\begin{hindi}
तो काफ़िरों को मोहलत दो बस उनको थोड़ी सी मोहलत दो
\end{hindi}}
\chapter{Al-A'la (The Most High)}
\begin{Arabic}
\Huge{\centerline{\basmalah}}\end{Arabic}
\flushright{\begin{Arabic}
\quranayah[87][1]
\end{Arabic}}
\flushleft{\begin{hindi}
ऐ रसूल अपने आलीशान परवरदिगार के नाम की तस्बीह करो
\end{hindi}}
\flushright{\begin{Arabic}
\quranayah[87][2]
\end{Arabic}}
\flushleft{\begin{hindi}
जिसने (हर चीज़ को) पैदा किया
\end{hindi}}
\flushright{\begin{Arabic}
\quranayah[87][3]
\end{Arabic}}
\flushleft{\begin{hindi}
और दुरूस्त किया और जिसने (उसका) अन्दाज़ा मुक़र्रर किया फिर राह बतायी
\end{hindi}}
\flushright{\begin{Arabic}
\quranayah[87][4]
\end{Arabic}}
\flushleft{\begin{hindi}
और जिसने (हैवानात के लिए) चारा उगाया
\end{hindi}}
\flushright{\begin{Arabic}
\quranayah[87][5]
\end{Arabic}}
\flushleft{\begin{hindi}
फिर ख़ुश्क उसे सियाह रंग का कूड़ा कर दिया
\end{hindi}}
\flushright{\begin{Arabic}
\quranayah[87][6]
\end{Arabic}}
\flushleft{\begin{hindi}
हम तुम्हें (ऐसा) पढ़ा देंगे कि कभी भूलो ही नहीं
\end{hindi}}
\flushright{\begin{Arabic}
\quranayah[87][7]
\end{Arabic}}
\flushleft{\begin{hindi}
मगर जो ख़ुदा चाहे (मन्सूख़ कर दे) बेशक वह खुली बात को भी जानता है और छुपे हुए को भी
\end{hindi}}
\flushright{\begin{Arabic}
\quranayah[87][8]
\end{Arabic}}
\flushleft{\begin{hindi}
और हम तुमको आसान तरीके की तौफ़ीक़ देंगे
\end{hindi}}
\flushright{\begin{Arabic}
\quranayah[87][9]
\end{Arabic}}
\flushleft{\begin{hindi}
तो जहाँ तक समझाना मुफ़ीद हो समझते रहो
\end{hindi}}
\flushright{\begin{Arabic}
\quranayah[87][10]
\end{Arabic}}
\flushleft{\begin{hindi}
जो खौफ रखता हो वह तो फौरी समझ जाएगा
\end{hindi}}
\flushright{\begin{Arabic}
\quranayah[87][11]
\end{Arabic}}
\flushleft{\begin{hindi}
और बदबख्त उससे पहलू तही करेगा
\end{hindi}}
\flushright{\begin{Arabic}
\quranayah[87][12]
\end{Arabic}}
\flushleft{\begin{hindi}
जो (क़यामत में) बड़ी (तेज़) आग में दाख़िल होगा
\end{hindi}}
\flushright{\begin{Arabic}
\quranayah[87][13]
\end{Arabic}}
\flushleft{\begin{hindi}
फिर न वहाँ मरेगा ही न जीयेगा
\end{hindi}}
\flushright{\begin{Arabic}
\quranayah[87][14]
\end{Arabic}}
\flushleft{\begin{hindi}
वह यक़ीनन मुराद दिली को पहुँचा जो (शिर्क से) पाक हो
\end{hindi}}
\flushright{\begin{Arabic}
\quranayah[87][15]
\end{Arabic}}
\flushleft{\begin{hindi}
और अपने परवरदिगार का ज़िक्र करता और नमाज़ पढ़ता रहा
\end{hindi}}
\flushright{\begin{Arabic}
\quranayah[87][16]
\end{Arabic}}
\flushleft{\begin{hindi}
मगर तुम लोग दुनियावी ज़िन्दगी को तरजीह देते हो
\end{hindi}}
\flushright{\begin{Arabic}
\quranayah[87][17]
\end{Arabic}}
\flushleft{\begin{hindi}
हालॉकि आख़ोरत कहीं बेहतर और देर पा है
\end{hindi}}
\flushright{\begin{Arabic}
\quranayah[87][18]
\end{Arabic}}
\flushleft{\begin{hindi}
बेशक यही बात अगले सहीफ़ों
\end{hindi}}
\flushright{\begin{Arabic}
\quranayah[87][19]
\end{Arabic}}
\flushleft{\begin{hindi}
इबराहीम और मूसा के सहीफ़ों में भी है
\end{hindi}}
\chapter{Al-Ghashiyah (The Overwhelming Event)}
\begin{Arabic}
\Huge{\centerline{\basmalah}}\end{Arabic}
\flushright{\begin{Arabic}
\quranayah[88][1]
\end{Arabic}}
\flushleft{\begin{hindi}
भला तुमको ढाँप लेने वाली मुसीबत (क़यामत) का हाल मालुम हुआ है
\end{hindi}}
\flushright{\begin{Arabic}
\quranayah[88][2]
\end{Arabic}}
\flushleft{\begin{hindi}
उस दिन बहुत से चेहरे ज़लील रूसवा होंगे
\end{hindi}}
\flushright{\begin{Arabic}
\quranayah[88][3]
\end{Arabic}}
\flushleft{\begin{hindi}
(तौक़ व जंज़ीर से) मयक्क़त करने वाले
\end{hindi}}
\flushright{\begin{Arabic}
\quranayah[88][4]
\end{Arabic}}
\flushleft{\begin{hindi}
थके माँदे दहकती हुई आग में दाखिल होंगे
\end{hindi}}
\flushright{\begin{Arabic}
\quranayah[88][5]
\end{Arabic}}
\flushleft{\begin{hindi}
उन्हें एक खौलते हुए चशमें का पानी पिलाया जाएगा
\end{hindi}}
\flushright{\begin{Arabic}
\quranayah[88][6]
\end{Arabic}}
\flushleft{\begin{hindi}
ख़ारदार झाड़ी के सिवा उनके लिए कोई खाना नहीं
\end{hindi}}
\flushright{\begin{Arabic}
\quranayah[88][7]
\end{Arabic}}
\flushleft{\begin{hindi}
जो मोटाई पैदा करे न भूख में कुछ काम आएगा
\end{hindi}}
\flushright{\begin{Arabic}
\quranayah[88][8]
\end{Arabic}}
\flushleft{\begin{hindi}
(और) बहुत से चेहरे उस दिन तरो ताज़ा होंगे
\end{hindi}}
\flushright{\begin{Arabic}
\quranayah[88][9]
\end{Arabic}}
\flushleft{\begin{hindi}
अपनी कोशिश (के नतीजे) पर शादमान
\end{hindi}}
\flushright{\begin{Arabic}
\quranayah[88][10]
\end{Arabic}}
\flushleft{\begin{hindi}
एक आलीशान बाग़ में
\end{hindi}}
\flushright{\begin{Arabic}
\quranayah[88][11]
\end{Arabic}}
\flushleft{\begin{hindi}
वहाँ कोई लग़ो बात सुनेंगे ही नहीं
\end{hindi}}
\flushright{\begin{Arabic}
\quranayah[88][12]
\end{Arabic}}
\flushleft{\begin{hindi}
उसमें चश्में जारी होंगें
\end{hindi}}
\flushright{\begin{Arabic}
\quranayah[88][13]
\end{Arabic}}
\flushleft{\begin{hindi}
उसमें ऊँचे ऊँचे तख्त बिछे होंगे
\end{hindi}}
\flushright{\begin{Arabic}
\quranayah[88][14]
\end{Arabic}}
\flushleft{\begin{hindi}
और (उनके किनारे) गिलास रखे होंगे
\end{hindi}}
\flushright{\begin{Arabic}
\quranayah[88][15]
\end{Arabic}}
\flushleft{\begin{hindi}
और गाँव तकिए क़तार की क़तार लगे होंगे
\end{hindi}}
\flushright{\begin{Arabic}
\quranayah[88][16]
\end{Arabic}}
\flushleft{\begin{hindi}
और नफ़ीस मसनदे बिछी हुई
\end{hindi}}
\flushright{\begin{Arabic}
\quranayah[88][17]
\end{Arabic}}
\flushleft{\begin{hindi}
तो क्या ये लोग ऊँट की तरह ग़ौर नहीं करते कि कैसा अजीब पैदा किया गया है
\end{hindi}}
\flushright{\begin{Arabic}
\quranayah[88][18]
\end{Arabic}}
\flushleft{\begin{hindi}
और आसमान की तरफ कि क्या बुलन्द बनाया गया है
\end{hindi}}
\flushright{\begin{Arabic}
\quranayah[88][19]
\end{Arabic}}
\flushleft{\begin{hindi}
और पहाड़ों की तरफ़ कि किस तरह खड़े किए गए हैं
\end{hindi}}
\flushright{\begin{Arabic}
\quranayah[88][20]
\end{Arabic}}
\flushleft{\begin{hindi}
और ज़मीन की तरफ कि किस तरह बिछायी गयी है
\end{hindi}}
\flushright{\begin{Arabic}
\quranayah[88][21]
\end{Arabic}}
\flushleft{\begin{hindi}
तो तुम नसीहत करते रहो तुम तो बस नसीहत करने वाले हो
\end{hindi}}
\flushright{\begin{Arabic}
\quranayah[88][22]
\end{Arabic}}
\flushleft{\begin{hindi}
तुम कुछ उन पर दरोग़ा तो हो नहीं
\end{hindi}}
\flushright{\begin{Arabic}
\quranayah[88][23]
\end{Arabic}}
\flushleft{\begin{hindi}
हाँ जिसने मुँह फेर लिया
\end{hindi}}
\flushright{\begin{Arabic}
\quranayah[88][24]
\end{Arabic}}
\flushleft{\begin{hindi}
और न माना तो ख़ुदा उसको बहुत बड़े अज़ाब की सज़ा देगा
\end{hindi}}
\flushright{\begin{Arabic}
\quranayah[88][25]
\end{Arabic}}
\flushleft{\begin{hindi}
बेशक उनको हमारी तरफ़ लौट कर आना है
\end{hindi}}
\flushright{\begin{Arabic}
\quranayah[88][26]
\end{Arabic}}
\flushleft{\begin{hindi}
फिर उनका हिसाब हमारे ज़िम्मे है
\end{hindi}}
\chapter{Al-Fajr (The Daybreak)}
\begin{Arabic}
\Huge{\centerline{\basmalah}}\end{Arabic}
\flushright{\begin{Arabic}
\quranayah[89][1]
\end{Arabic}}
\flushleft{\begin{hindi}
सुबह की क़सम
\end{hindi}}
\flushright{\begin{Arabic}
\quranayah[89][2]
\end{Arabic}}
\flushleft{\begin{hindi}
और दस रातों की
\end{hindi}}
\flushright{\begin{Arabic}
\quranayah[89][3]
\end{Arabic}}
\flushleft{\begin{hindi}
और ज़ुफ्त व ताक़ की
\end{hindi}}
\flushright{\begin{Arabic}
\quranayah[89][4]
\end{Arabic}}
\flushleft{\begin{hindi}
और रात की जब आने लगे
\end{hindi}}
\flushright{\begin{Arabic}
\quranayah[89][5]
\end{Arabic}}
\flushleft{\begin{hindi}
अक्लमन्द के वास्ते तो ज़रूर बड़ी क़सम है (कि कुफ्फ़ार पर ज़रूर अज़ाब होगा)
\end{hindi}}
\flushright{\begin{Arabic}
\quranayah[89][6]
\end{Arabic}}
\flushleft{\begin{hindi}
क्या तुमने देखा नहीं कि तुम्हारे आद के साथ क्या किया
\end{hindi}}
\flushright{\begin{Arabic}
\quranayah[89][7]
\end{Arabic}}
\flushleft{\begin{hindi}
यानि इरम वाले दराज़ क़द
\end{hindi}}
\flushright{\begin{Arabic}
\quranayah[89][8]
\end{Arabic}}
\flushleft{\begin{hindi}
जिनका मिसल तमाम (दुनिया के) शहरों में कोई पैदा ही नहीं किया गया
\end{hindi}}
\flushright{\begin{Arabic}
\quranayah[89][9]
\end{Arabic}}
\flushleft{\begin{hindi}
और समूद के साथ (क्या किया) जो वादी (क़रा) में पत्थर तराश कर घर बनाते थे
\end{hindi}}
\flushright{\begin{Arabic}
\quranayah[89][10]
\end{Arabic}}
\flushleft{\begin{hindi}
और फिरऔन के साथ (क्या किया) जो (सज़ा के लिए) मेख़े रखता था
\end{hindi}}
\flushright{\begin{Arabic}
\quranayah[89][11]
\end{Arabic}}
\flushleft{\begin{hindi}
ये लोग मुख़तलिफ़ शहरों में सरकश हो रहे थे
\end{hindi}}
\flushright{\begin{Arabic}
\quranayah[89][12]
\end{Arabic}}
\flushleft{\begin{hindi}
और उनमें बहुत से फ़साद फैला रखे थे
\end{hindi}}
\flushright{\begin{Arabic}
\quranayah[89][13]
\end{Arabic}}
\flushleft{\begin{hindi}
तो तुम्हारे परवरदिगार ने उन पर अज़ाब का कोड़ा लगाया
\end{hindi}}
\flushright{\begin{Arabic}
\quranayah[89][14]
\end{Arabic}}
\flushleft{\begin{hindi}
बेशक तुम्हारा परवरदिगार ताक में है
\end{hindi}}
\flushright{\begin{Arabic}
\quranayah[89][15]
\end{Arabic}}
\flushleft{\begin{hindi}
लेकिन इन्सान जब उसको उसका परवरदिगार (इस तरह) आज़माता है कि उसको इज्ज़त व नेअमत देता है, तो कहता है कि मेरे परवरदिगार ने मुझे इज्ज़त दी है
\end{hindi}}
\flushright{\begin{Arabic}
\quranayah[89][16]
\end{Arabic}}
\flushleft{\begin{hindi}
मगर जब उसको (इस तरह) आज़माता है कि उस पर रोज़ी को तंग कर देता है बोल उठता है कि मेरे परवरदिगार ने मुझे ज़लील किया
\end{hindi}}
\flushright{\begin{Arabic}
\quranayah[89][17]
\end{Arabic}}
\flushleft{\begin{hindi}
हरगिज़ नहीं बल्कि तुम लोग न यतीम की ख़ातिरदारी करते हो
\end{hindi}}
\flushright{\begin{Arabic}
\quranayah[89][18]
\end{Arabic}}
\flushleft{\begin{hindi}
और न मोहताज को खाना खिलाने की तरग़ीब देते हो
\end{hindi}}
\flushright{\begin{Arabic}
\quranayah[89][19]
\end{Arabic}}
\flushleft{\begin{hindi}
और मीरारा के माल (हलाल व हराम) को समेट कर चख जाते हो
\end{hindi}}
\flushright{\begin{Arabic}
\quranayah[89][20]
\end{Arabic}}
\flushleft{\begin{hindi}
और माल को बहुत ही अज़ीज़ रखते हो
\end{hindi}}
\flushright{\begin{Arabic}
\quranayah[89][21]
\end{Arabic}}
\flushleft{\begin{hindi}
सुन रखो कि जब ज़मीन कूट कूट कर रेज़ा रेज़ा कर दी जाएगी
\end{hindi}}
\flushright{\begin{Arabic}
\quranayah[89][22]
\end{Arabic}}
\flushleft{\begin{hindi}
और तुम्हारे परवरदिगार का हुक्म और फ़रिश्ते कतार के कतार आ जाएँगे
\end{hindi}}
\flushright{\begin{Arabic}
\quranayah[89][23]
\end{Arabic}}
\flushleft{\begin{hindi}
और उस दिन जहन्नुम सामने कर दी जाएगी उस दिन इन्सान चौंकेगा मगर अब चौंकना कहाँ (फ़ायदा देगा)
\end{hindi}}
\flushright{\begin{Arabic}
\quranayah[89][24]
\end{Arabic}}
\flushleft{\begin{hindi}
(उस वक्त) क़हेगा कि काश मैने अपनी (इस) ज़िन्दगी के वास्ते कुछ पहले भेजा होता
\end{hindi}}
\flushright{\begin{Arabic}
\quranayah[89][25]
\end{Arabic}}
\flushleft{\begin{hindi}
तो उस दिन ख़ुदा ऐसा अज़ाब करेगा कि किसी ने वैसा अज़ाब न किया होगा
\end{hindi}}
\flushright{\begin{Arabic}
\quranayah[89][26]
\end{Arabic}}
\flushleft{\begin{hindi}
और न कोई उसके जकड़ने की तरह जकड़ेगा
\end{hindi}}
\flushright{\begin{Arabic}
\quranayah[89][27]
\end{Arabic}}
\flushleft{\begin{hindi}
(और कुछ लोगों से कहेगा) ऐ इत्मेनान पाने वाली जान
\end{hindi}}
\flushright{\begin{Arabic}
\quranayah[89][28]
\end{Arabic}}
\flushleft{\begin{hindi}
अपने परवरदिगार की तरफ़ चल तू उससे ख़ुश वह तुझ से राज़ी
\end{hindi}}
\flushright{\begin{Arabic}
\quranayah[89][29]
\end{Arabic}}
\flushleft{\begin{hindi}
तो मेरे (ख़ास) बन्दों में शामिल हो जा
\end{hindi}}
\flushright{\begin{Arabic}
\quranayah[89][30]
\end{Arabic}}
\flushleft{\begin{hindi}
और मेरे बेहिश्त में दाख़िल हो जा
\end{hindi}}
\chapter{Al-Balad (The City)}
\begin{Arabic}
\Huge{\centerline{\basmalah}}\end{Arabic}
\flushright{\begin{Arabic}
\quranayah[90][1]
\end{Arabic}}
\flushleft{\begin{hindi}
मुझे इस शहर (मक्का) की कसम
\end{hindi}}
\flushright{\begin{Arabic}
\quranayah[90][2]
\end{Arabic}}
\flushleft{\begin{hindi}
और तुम इसी शहर में तो रहते हो
\end{hindi}}
\flushright{\begin{Arabic}
\quranayah[90][3]
\end{Arabic}}
\flushleft{\begin{hindi}
और (तुम्हारे) बाप (आदम) और उसकी औलाद की क़सम
\end{hindi}}
\flushright{\begin{Arabic}
\quranayah[90][4]
\end{Arabic}}
\flushleft{\begin{hindi}
हमने इन्सान को मशक्क़त में (रहने वाला) पैदा किया है
\end{hindi}}
\flushright{\begin{Arabic}
\quranayah[90][5]
\end{Arabic}}
\flushleft{\begin{hindi}
क्या वह ये समझता है कि उस पर कोई काबू न पा सकेगा
\end{hindi}}
\flushright{\begin{Arabic}
\quranayah[90][6]
\end{Arabic}}
\flushleft{\begin{hindi}
वह कहता है कि मैने अलग़ारों माल उड़ा दिया
\end{hindi}}
\flushright{\begin{Arabic}
\quranayah[90][7]
\end{Arabic}}
\flushleft{\begin{hindi}
क्या वह ये ख्याल रखता है कि उसको किसी ने देखा ही नहीं
\end{hindi}}
\flushright{\begin{Arabic}
\quranayah[90][8]
\end{Arabic}}
\flushleft{\begin{hindi}
क्या हमने उसे दोनों ऑंखें और ज़बान
\end{hindi}}
\flushright{\begin{Arabic}
\quranayah[90][9]
\end{Arabic}}
\flushleft{\begin{hindi}
और दोनों लब नहीं दिए (ज़रूर दिए)
\end{hindi}}
\flushright{\begin{Arabic}
\quranayah[90][10]
\end{Arabic}}
\flushleft{\begin{hindi}
और उसको (अच्छी बुरी) दोनों राहें भी दिखा दीं
\end{hindi}}
\flushright{\begin{Arabic}
\quranayah[90][11]
\end{Arabic}}
\flushleft{\begin{hindi}
फिर वह घाटी पर से होकर (क्यों) नहीं गुज़रा
\end{hindi}}
\flushright{\begin{Arabic}
\quranayah[90][12]
\end{Arabic}}
\flushleft{\begin{hindi}
और तुमको क्या मालूम कि घाटी क्या है
\end{hindi}}
\flushright{\begin{Arabic}
\quranayah[90][13]
\end{Arabic}}
\flushleft{\begin{hindi}
किसी (की) गर्दन का (गुलामी या कर्ज से) छुड़ाना
\end{hindi}}
\flushright{\begin{Arabic}
\quranayah[90][14]
\end{Arabic}}
\flushleft{\begin{hindi}
या भूख के दिन रिश्तेदार यतीम या ख़ाकसार
\end{hindi}}
\flushright{\begin{Arabic}
\quranayah[90][15]
\end{Arabic}}
\flushleft{\begin{hindi}
मोहताज को
\end{hindi}}
\flushright{\begin{Arabic}
\quranayah[90][16]
\end{Arabic}}
\flushleft{\begin{hindi}
खाना खिलाना
\end{hindi}}
\flushright{\begin{Arabic}
\quranayah[90][17]
\end{Arabic}}
\flushleft{\begin{hindi}
फिर तो उन लोगों में (शामिल) हो जाता जो ईमान लाए और सब्र की नसीहत और तरस खाने की वसीयत करते रहे
\end{hindi}}
\flushright{\begin{Arabic}
\quranayah[90][18]
\end{Arabic}}
\flushleft{\begin{hindi}
यही लोग ख़ुश नसीब हैं
\end{hindi}}
\flushright{\begin{Arabic}
\quranayah[90][19]
\end{Arabic}}
\flushleft{\begin{hindi}
और जिन लोगों ने हमारी आयतों से इन्कार किया है यही लोग बदबख्त हैं
\end{hindi}}
\flushright{\begin{Arabic}
\quranayah[90][20]
\end{Arabic}}
\flushleft{\begin{hindi}
कि उनको आग में डाल कर हर तरफ से बन्द कर दिया जाएगा
\end{hindi}}
\chapter{Ash-Shams (The Sun)}
\begin{Arabic}
\Huge{\centerline{\basmalah}}\end{Arabic}
\flushright{\begin{Arabic}
\quranayah[91][1]
\end{Arabic}}
\flushleft{\begin{hindi}
सूरज की क़सम और उसकी रौशनी की
\end{hindi}}
\flushright{\begin{Arabic}
\quranayah[91][2]
\end{Arabic}}
\flushleft{\begin{hindi}
और चाँद की जब उसके पीछे निकले
\end{hindi}}
\flushright{\begin{Arabic}
\quranayah[91][3]
\end{Arabic}}
\flushleft{\begin{hindi}
और दिन की जब उसे चमका दे
\end{hindi}}
\flushright{\begin{Arabic}
\quranayah[91][4]
\end{Arabic}}
\flushleft{\begin{hindi}
और रात की जब उसे ढाँक ले
\end{hindi}}
\flushright{\begin{Arabic}
\quranayah[91][5]
\end{Arabic}}
\flushleft{\begin{hindi}
और आसमान की और जिसने उसे बनाया
\end{hindi}}
\flushright{\begin{Arabic}
\quranayah[91][6]
\end{Arabic}}
\flushleft{\begin{hindi}
और ज़मीन की जिसने उसे बिछाया
\end{hindi}}
\flushright{\begin{Arabic}
\quranayah[91][7]
\end{Arabic}}
\flushleft{\begin{hindi}
और जान की और जिसने उसे दुरूस्त किया
\end{hindi}}
\flushright{\begin{Arabic}
\quranayah[91][8]
\end{Arabic}}
\flushleft{\begin{hindi}
फिर उसकी बदकारी और परहेज़गारी को उसे समझा दिया
\end{hindi}}
\flushright{\begin{Arabic}
\quranayah[91][9]
\end{Arabic}}
\flushleft{\begin{hindi}
(क़सम है) जिसने उस (जान) को (गनाह से) पाक रखा वह तो कामयाब हुआ
\end{hindi}}
\flushright{\begin{Arabic}
\quranayah[91][10]
\end{Arabic}}
\flushleft{\begin{hindi}
और जिसने उसे (गुनाह करके) दबा दिया वह नामुराद रहा
\end{hindi}}
\flushright{\begin{Arabic}
\quranayah[91][11]
\end{Arabic}}
\flushleft{\begin{hindi}
क़ौम मसूद ने अपनी सरकशी से (सालेह पैग़म्बर को) झुठलाया,
\end{hindi}}
\flushright{\begin{Arabic}
\quranayah[91][12]
\end{Arabic}}
\flushleft{\begin{hindi}
जब उनमें का एक बड़ा बदबख्त उठ खड़ा हुआ
\end{hindi}}
\flushright{\begin{Arabic}
\quranayah[91][13]
\end{Arabic}}
\flushleft{\begin{hindi}
तो ख़ुदा के रसूल (सालेह) ने उनसे कहा कि ख़ुदा की ऊँटनी और उसके पानी पीने से तअर्रुज़ न करना
\end{hindi}}
\flushright{\begin{Arabic}
\quranayah[91][14]
\end{Arabic}}
\flushleft{\begin{hindi}
मगर उन लोगों पैग़म्बर को झुठलाया और उसकी कूँचे काट डाली तो ख़ुदा ने उनके गुनाहों सबब से उन पर अज़ाब नाज़िल किया फिर (हलाक करके) बराबर कर दिया
\end{hindi}}
\flushright{\begin{Arabic}
\quranayah[91][15]
\end{Arabic}}
\flushleft{\begin{hindi}
और उसको उनके बदले का कोई ख़ौफ तो है नहीं
\end{hindi}}
\chapter{Al-Lail (The Night)}
\begin{Arabic}
\Huge{\centerline{\basmalah}}\end{Arabic}
\flushright{\begin{Arabic}
\quranayah[92][1]
\end{Arabic}}
\flushleft{\begin{hindi}
रात की क़सम जब (सूरज को) छिपा ले
\end{hindi}}
\flushright{\begin{Arabic}
\quranayah[92][2]
\end{Arabic}}
\flushleft{\begin{hindi}
और दिन की क़सम जब ख़ूब रौशन हो
\end{hindi}}
\flushright{\begin{Arabic}
\quranayah[92][3]
\end{Arabic}}
\flushleft{\begin{hindi}
और उस (ज़ात) की जिसने नर व मादा को पैदा किया
\end{hindi}}
\flushright{\begin{Arabic}
\quranayah[92][4]
\end{Arabic}}
\flushleft{\begin{hindi}
कि बेशक तुम्हारी कोशिश तरह तरह की है
\end{hindi}}
\flushright{\begin{Arabic}
\quranayah[92][5]
\end{Arabic}}
\flushleft{\begin{hindi}
तो जिसने सख़ावत की और अच्छी बात (इस्लाम) की तस्दीक़ की
\end{hindi}}
\flushright{\begin{Arabic}
\quranayah[92][6]
\end{Arabic}}
\flushleft{\begin{hindi}
तो हम उसके लिए राहत व आसानी
\end{hindi}}
\flushright{\begin{Arabic}
\quranayah[92][7]
\end{Arabic}}
\flushleft{\begin{hindi}
(जन्नत) के असबाब मुहय्या कर देंगे
\end{hindi}}
\flushright{\begin{Arabic}
\quranayah[92][8]
\end{Arabic}}
\flushleft{\begin{hindi}
और जिसने बुख्ल किया, और बेपरवाई की
\end{hindi}}
\flushright{\begin{Arabic}
\quranayah[92][9]
\end{Arabic}}
\flushleft{\begin{hindi}
और अच्छी बात को झुठलाया
\end{hindi}}
\flushright{\begin{Arabic}
\quranayah[92][10]
\end{Arabic}}
\flushleft{\begin{hindi}
तो हम उसे सख्ती (जहन्नुम) में पहुँचा देंगे,
\end{hindi}}
\flushright{\begin{Arabic}
\quranayah[92][11]
\end{Arabic}}
\flushleft{\begin{hindi}
और जब वह हलाक होगा तो उसका माल उसके कुछ भी काम न आएगा
\end{hindi}}
\flushright{\begin{Arabic}
\quranayah[92][12]
\end{Arabic}}
\flushleft{\begin{hindi}
हमें राह दिखा देना ज़रूर है
\end{hindi}}
\flushright{\begin{Arabic}
\quranayah[92][13]
\end{Arabic}}
\flushleft{\begin{hindi}
और आख़ेरत और दुनिया (दोनों) ख़ास हमारी चीज़े हैं
\end{hindi}}
\flushright{\begin{Arabic}
\quranayah[92][14]
\end{Arabic}}
\flushleft{\begin{hindi}
तो हमने तुम्हें भड़कती हुई आग से डरा दिया
\end{hindi}}
\flushright{\begin{Arabic}
\quranayah[92][15]
\end{Arabic}}
\flushleft{\begin{hindi}
उसमें बस वही दाख़िल होगा जो बड़ा बदबख्त है
\end{hindi}}
\flushright{\begin{Arabic}
\quranayah[92][16]
\end{Arabic}}
\flushleft{\begin{hindi}
जिसने झुठलाया और मुँह फेर लिया और जो बड़ा परहेज़गार है
\end{hindi}}
\flushright{\begin{Arabic}
\quranayah[92][17]
\end{Arabic}}
\flushleft{\begin{hindi}
वह उससे बचा लिया जाएगा
\end{hindi}}
\flushright{\begin{Arabic}
\quranayah[92][18]
\end{Arabic}}
\flushleft{\begin{hindi}
जो अपना माल (ख़ुदा की राह) में देता है ताकि पाक हो जाए
\end{hindi}}
\flushright{\begin{Arabic}
\quranayah[92][19]
\end{Arabic}}
\flushleft{\begin{hindi}
और लुत्फ ये है कि किसी का उस पर कोई एहसान नहीं जिसका उसे बदला दिया जाता है
\end{hindi}}
\flushright{\begin{Arabic}
\quranayah[92][20]
\end{Arabic}}
\flushleft{\begin{hindi}
बल्कि (वह तो) सिर्फ अपने आलीशान परवरदिगार की ख़ुशनूदी हासिल करने के लिए (देता है)
\end{hindi}}
\flushright{\begin{Arabic}
\quranayah[92][21]
\end{Arabic}}
\flushleft{\begin{hindi}
और वह अनक़रीब भी ख़ुश हो जाएगा
\end{hindi}}
\chapter{Ad-Duha (The Brightness of the Day)}
\begin{Arabic}
\Huge{\centerline{\basmalah}}\end{Arabic}
\flushright{\begin{Arabic}
\quranayah[93][1]
\end{Arabic}}
\flushleft{\begin{hindi}
(ऐ रसूल) पहर दिन चढ़े की क़सम
\end{hindi}}
\flushright{\begin{Arabic}
\quranayah[93][2]
\end{Arabic}}
\flushleft{\begin{hindi}
और रात की जब (चीज़ों को) छुपा ले
\end{hindi}}
\flushright{\begin{Arabic}
\quranayah[93][3]
\end{Arabic}}
\flushleft{\begin{hindi}
कि तुम्हारा परवरदिगार न तुमको छोड़ बैठा और (न तुमसे) नाराज़ हुआ
\end{hindi}}
\flushright{\begin{Arabic}
\quranayah[93][4]
\end{Arabic}}
\flushleft{\begin{hindi}
और तुम्हारे वास्ते आख़ेरत दुनिया से यक़ीनी कहीं बेहतर है
\end{hindi}}
\flushright{\begin{Arabic}
\quranayah[93][5]
\end{Arabic}}
\flushleft{\begin{hindi}
और तुम्हारा परवरदिगार अनक़रीब इस क़दर अता करेगा कि तुम ख़ुश हो जाओ
\end{hindi}}
\flushright{\begin{Arabic}
\quranayah[93][6]
\end{Arabic}}
\flushleft{\begin{hindi}
क्या उसने तुम्हें यतीम पाकर (अबू तालिब की) पनाह न दी (ज़रूर दी)
\end{hindi}}
\flushright{\begin{Arabic}
\quranayah[93][7]
\end{Arabic}}
\flushleft{\begin{hindi}
और तुमको एहकाम से नावाकिफ़ देखा तो मंज़िले मक़सूद तक पहुँचा दिया
\end{hindi}}
\flushright{\begin{Arabic}
\quranayah[93][8]
\end{Arabic}}
\flushleft{\begin{hindi}
और तुमको तंगदस्त देखकर ग़नी कर दिया
\end{hindi}}
\flushright{\begin{Arabic}
\quranayah[93][9]
\end{Arabic}}
\flushleft{\begin{hindi}
तो तुम भी यतीम पर सितम न करना
\end{hindi}}
\flushright{\begin{Arabic}
\quranayah[93][10]
\end{Arabic}}
\flushleft{\begin{hindi}
माँगने वाले को झिड़की न देना
\end{hindi}}
\flushright{\begin{Arabic}
\quranayah[93][11]
\end{Arabic}}
\flushleft{\begin{hindi}
और अपने परवरदिगार की नेअमतों का ज़िक्र करते रहना
\end{hindi}}
\chapter{Al-Inshirah (The Expansion)}
\begin{Arabic}
\Huge{\centerline{\basmalah}}\end{Arabic}
\flushright{\begin{Arabic}
\quranayah[94][1]
\end{Arabic}}
\flushleft{\begin{hindi}
(ऐ रसूल) क्या हमने तुम्हारा सीना इल्म से कुशादा नहीं कर दिया (जरूर किया)
\end{hindi}}
\flushright{\begin{Arabic}
\quranayah[94][2]
\end{Arabic}}
\flushleft{\begin{hindi}
और तुम पर से वह बोझ उतार दिया
\end{hindi}}
\flushright{\begin{Arabic}
\quranayah[94][3]
\end{Arabic}}
\flushleft{\begin{hindi}
जिसने तुम्हारी कमर तोड़ रखी थी
\end{hindi}}
\flushright{\begin{Arabic}
\quranayah[94][4]
\end{Arabic}}
\flushleft{\begin{hindi}
और तुम्हारा ज़िक्र भी बुलन्द कर दिया
\end{hindi}}
\flushright{\begin{Arabic}
\quranayah[94][5]
\end{Arabic}}
\flushleft{\begin{hindi}
तो (हाँ) पस बेशक दुशवारी के साथ ही आसानी है
\end{hindi}}
\flushright{\begin{Arabic}
\quranayah[94][6]
\end{Arabic}}
\flushleft{\begin{hindi}
यक़ीनन दुश्वारी के साथ आसानी है
\end{hindi}}
\flushright{\begin{Arabic}
\quranayah[94][7]
\end{Arabic}}
\flushleft{\begin{hindi}
तो जब तुम फारिग़ हो जाओ तो मुक़र्रर कर दो
\end{hindi}}
\flushright{\begin{Arabic}
\quranayah[94][8]
\end{Arabic}}
\flushleft{\begin{hindi}
और फिर अपने परवरदिगार की तरफ रग़बत करो
\end{hindi}}
\chapter{At-Tin (The Fig)}
\begin{Arabic}
\Huge{\centerline{\basmalah}}\end{Arabic}
\flushright{\begin{Arabic}
\quranayah[95][1]
\end{Arabic}}
\flushleft{\begin{hindi}
इन्जीर और ज़ैतून की क़सम
\end{hindi}}
\flushright{\begin{Arabic}
\quranayah[95][2]
\end{Arabic}}
\flushleft{\begin{hindi}
और तूर सीनीन की
\end{hindi}}
\flushright{\begin{Arabic}
\quranayah[95][3]
\end{Arabic}}
\flushleft{\begin{hindi}
और उस अमन वाले शहर (मक्का) की
\end{hindi}}
\flushright{\begin{Arabic}
\quranayah[95][4]
\end{Arabic}}
\flushleft{\begin{hindi}
कि हमने इन्सान बहुत अच्छे कैड़े का पैदा किया
\end{hindi}}
\flushright{\begin{Arabic}
\quranayah[95][5]
\end{Arabic}}
\flushleft{\begin{hindi}
फिर हमने उसे (बूढ़ा करके रफ्ता रफ्ता) पस्त से पस्त हालत की तरफ फेर दिया
\end{hindi}}
\flushright{\begin{Arabic}
\quranayah[95][6]
\end{Arabic}}
\flushleft{\begin{hindi}
मगर जो लोग ईमान लाए और अच्छे (अच्छे) काम करते रहे उनके लिए तो बे इन्तेहा अज्र व सवाब है
\end{hindi}}
\flushright{\begin{Arabic}
\quranayah[95][7]
\end{Arabic}}
\flushleft{\begin{hindi}
तो (ऐ रसूल) इन दलीलों के बाद तुमको (रोज़े) जज़ा के बारे में कौन झुठला सकता है
\end{hindi}}
\flushright{\begin{Arabic}
\quranayah[95][8]
\end{Arabic}}
\flushleft{\begin{hindi}
क्या ख़ुदा सबसे बड़ा हाकिम नहीं है (हाँ ज़रूर है)
\end{hindi}}
\chapter{Al-'Alaq (The Clot)}
\begin{Arabic}
\Huge{\centerline{\basmalah}}\end{Arabic}
\flushright{\begin{Arabic}
\quranayah[96][1]
\end{Arabic}}
\flushleft{\begin{hindi}
(ऐ रसूल) अपने परवरदिगार का नाम लेकर पढ़ो जिसने हर (चीज़ को) पैदा किया
\end{hindi}}
\flushright{\begin{Arabic}
\quranayah[96][2]
\end{Arabic}}
\flushleft{\begin{hindi}
उस ने इन्सान को जमे हुए ख़ून से पैदा किया पढ़ो
\end{hindi}}
\flushright{\begin{Arabic}
\quranayah[96][3]
\end{Arabic}}
\flushleft{\begin{hindi}
और तुम्हारा परवरदिगार बड़ा क़रीम है
\end{hindi}}
\flushright{\begin{Arabic}
\quranayah[96][4]
\end{Arabic}}
\flushleft{\begin{hindi}
जिसने क़लम के ज़रिए तालीम दी
\end{hindi}}
\flushright{\begin{Arabic}
\quranayah[96][5]
\end{Arabic}}
\flushleft{\begin{hindi}
उसीने इन्सान को वह बातें बतायीं जिनको वह कुछ जानता ही न था
\end{hindi}}
\flushright{\begin{Arabic}
\quranayah[96][6]
\end{Arabic}}
\flushleft{\begin{hindi}
सुन रखो बेशक इन्सान जो अपने को ग़नी देखता है
\end{hindi}}
\flushright{\begin{Arabic}
\quranayah[96][7]
\end{Arabic}}
\flushleft{\begin{hindi}
तो सरकश हो जाता है
\end{hindi}}
\flushright{\begin{Arabic}
\quranayah[96][8]
\end{Arabic}}
\flushleft{\begin{hindi}
बेशक तुम्हारे परवरदिगार की तरफ (सबको) पलटना है
\end{hindi}}
\flushright{\begin{Arabic}
\quranayah[96][9]
\end{Arabic}}
\flushleft{\begin{hindi}
भला तुमने उस शख़्श को भी देखा
\end{hindi}}
\flushright{\begin{Arabic}
\quranayah[96][10]
\end{Arabic}}
\flushleft{\begin{hindi}
जो एक बन्दे को जब वह नमाज़ पढ़ता है तो वह रोकता है
\end{hindi}}
\flushright{\begin{Arabic}
\quranayah[96][11]
\end{Arabic}}
\flushleft{\begin{hindi}
भला देखो तो कि अगर ये राहे रास्त पर हो या परहेज़गारी का हुक्म करे
\end{hindi}}
\flushright{\begin{Arabic}
\quranayah[96][12]
\end{Arabic}}
\flushleft{\begin{hindi}
(तो रोकना कैसा)
\end{hindi}}
\flushright{\begin{Arabic}
\quranayah[96][13]
\end{Arabic}}
\flushleft{\begin{hindi}
भला देखो तो कि अगर उसने (सच्चे को) झुठला दिया और (उसने) मुँह फेरा
\end{hindi}}
\flushright{\begin{Arabic}
\quranayah[96][14]
\end{Arabic}}
\flushleft{\begin{hindi}
(तो नतीजा क्या होगा) क्या उसको ये मालूम नहीं कि ख़ुदा यक़ीनन देख रहा है
\end{hindi}}
\flushright{\begin{Arabic}
\quranayah[96][15]
\end{Arabic}}
\flushleft{\begin{hindi}
देखो अगर वह बाज़ न आएगा तो हम परेशानी के पट्टे पकड़ के घसीटेंगे
\end{hindi}}
\flushright{\begin{Arabic}
\quranayah[96][16]
\end{Arabic}}
\flushleft{\begin{hindi}
झूठे ख़तावार की पेशानी के पट्टे
\end{hindi}}
\flushright{\begin{Arabic}
\quranayah[96][17]
\end{Arabic}}
\flushleft{\begin{hindi}
तो वह अपने याराने जलसा को बुलाए हम भी जल्लाद फ़रिश्ते को बुलाएँगे
\end{hindi}}
\flushright{\begin{Arabic}
\quranayah[96][18]
\end{Arabic}}
\flushleft{\begin{hindi}
(ऐ रसूल) देखो हरगिज़ उनका कहना न मानना
\end{hindi}}
\flushright{\begin{Arabic}
\quranayah[96][19]
\end{Arabic}}
\flushleft{\begin{hindi}
और सजदे करते रहो और कुर्ब हासिल करो (19) (सजदा)
\end{hindi}}
\chapter{Al-Qadr (The Majesty)}
\begin{Arabic}
\Huge{\centerline{\basmalah}}\end{Arabic}
\flushright{\begin{Arabic}
\quranayah[97][1]
\end{Arabic}}
\flushleft{\begin{hindi}
हमने (इस कुरान) को शबे क़द्र में नाज़िल (करना शुरू) किया
\end{hindi}}
\flushright{\begin{Arabic}
\quranayah[97][2]
\end{Arabic}}
\flushleft{\begin{hindi}
और तुमको क्या मालूम शबे क़द्र क्या है
\end{hindi}}
\flushright{\begin{Arabic}
\quranayah[97][3]
\end{Arabic}}
\flushleft{\begin{hindi}
शबे क़द्र (मरतबा और अमल में) हज़ार महीनो से बेहतर है
\end{hindi}}
\flushright{\begin{Arabic}
\quranayah[97][4]
\end{Arabic}}
\flushleft{\begin{hindi}
इस (रात) में फ़रिश्ते और जिबरील (साल भर की) हर बात का हुक्म लेकर अपने परवरदिगार के हुक्म से नाज़िल होते हैं
\end{hindi}}
\flushright{\begin{Arabic}
\quranayah[97][5]
\end{Arabic}}
\flushleft{\begin{hindi}
ये रात सुबह के तुलूअ होने तक (अज़सरतापा) सलामती है
\end{hindi}}
\chapter{Al-Bayyinah (The Clear Evidence)}
\begin{Arabic}
\Huge{\centerline{\basmalah}}\end{Arabic}
\flushright{\begin{Arabic}
\quranayah[98][1]
\end{Arabic}}
\flushleft{\begin{hindi}
अहले किताब और मुशरिकों से जो लोग काफिर थे जब तक कि उनके पास खुली हुई दलीलें न पहुँचे वह (अपने कुफ्र से) बाज़ आने वाले न थे
\end{hindi}}
\flushright{\begin{Arabic}
\quranayah[98][2]
\end{Arabic}}
\flushleft{\begin{hindi}
(यानि) ख़ुदा के रसूल जो पाक औराक़ पढ़ते हैं (आए और)
\end{hindi}}
\flushright{\begin{Arabic}
\quranayah[98][3]
\end{Arabic}}
\flushleft{\begin{hindi}
उनमें (जो) पुरज़ोर और दरूस्त बातें लिखी हुई हैं (सुनाये)
\end{hindi}}
\flushright{\begin{Arabic}
\quranayah[98][4]
\end{Arabic}}
\flushleft{\begin{hindi}
अहले किताब मुताफ़र्रिक़ हुए भी तो जब उनके पास खुली हुई दलील आ चुकी
\end{hindi}}
\flushright{\begin{Arabic}
\quranayah[98][5]
\end{Arabic}}
\flushleft{\begin{hindi}
(तब) और उन्हें तो बस ये हुक्म दिया गया था कि निरा ख़ुरा उसी का एतक़ाद रख के बातिल से कतरा के ख़ुदा की इबादत करे और पाबन्दी से नमाज़ पढ़े और ज़कात अदा करता रहे और यही सच्चा दीन है
\end{hindi}}
\flushright{\begin{Arabic}
\quranayah[98][6]
\end{Arabic}}
\flushleft{\begin{hindi}
बेशक अहले किताब और मुशरेकीन से जो लोग (अब तक) काफ़िर हैं वह दोज़ख़ की आग में (होंगे) हमेशा उसी में रहेंगे यही लोग बदतरीन ख़लाएक़ हैं
\end{hindi}}
\flushright{\begin{Arabic}
\quranayah[98][7]
\end{Arabic}}
\flushleft{\begin{hindi}
बेशक जो लोग ईमान लाए और अच्छे काम करते रहे यही लोग बेहतरीन ख़लाएक़ हैं
\end{hindi}}
\flushright{\begin{Arabic}
\quranayah[98][8]
\end{Arabic}}
\flushleft{\begin{hindi}
उनकी जज़ा उनके परवरदिगार के यहाँ हमेशा रहने (सहने) के बाग़ हैं जिनके नीचे नहरें जारी हैं और वह आबादुल आबाद हमेशा उसी में रहेंगे ख़ुदा उनसे राज़ी और वह ख़ुदा से ख़ुश ये (जज़ा) ख़ास उस शख़्श की है जो अपने परवरदिगार से डरे
\end{hindi}}
\chapter{Al-Zilzal (The Shaking)}
\begin{Arabic}
\Huge{\centerline{\basmalah}}\end{Arabic}
\flushright{\begin{Arabic}
\quranayah[99][1]
\end{Arabic}}
\flushleft{\begin{hindi}
जब ज़मीन बड़े ज़ोरों के साथ ज़लज़ले में आ जाएगी
\end{hindi}}
\flushright{\begin{Arabic}
\quranayah[99][2]
\end{Arabic}}
\flushleft{\begin{hindi}
और ज़मीन अपने अन्दर के बोझे (मादनयात मुर्दे वग़ैरह) निकाल डालेगी
\end{hindi}}
\flushright{\begin{Arabic}
\quranayah[99][3]
\end{Arabic}}
\flushleft{\begin{hindi}
और एक इन्सान कहेगा कि उसको क्या हो गया है
\end{hindi}}
\flushright{\begin{Arabic}
\quranayah[99][4]
\end{Arabic}}
\flushleft{\begin{hindi}
उस रोज़ वह अपने सब हालात बयान कर देगी
\end{hindi}}
\flushright{\begin{Arabic}
\quranayah[99][5]
\end{Arabic}}
\flushleft{\begin{hindi}
क्योंकि तुम्हारे परवरदिगार ने उसको हुक्म दिया होगा
\end{hindi}}
\flushright{\begin{Arabic}
\quranayah[99][6]
\end{Arabic}}
\flushleft{\begin{hindi}
उस दिन लोग गिरोह गिरोह (अपनी कब्रों से) निकलेंगे ताकि अपने आमाल को देखे
\end{hindi}}
\flushright{\begin{Arabic}
\quranayah[99][7]
\end{Arabic}}
\flushleft{\begin{hindi}
तो जिस शख्स ने ज़र्रा बराबर नेकी की वह उसे देख लेगा
\end{hindi}}
\flushright{\begin{Arabic}
\quranayah[99][8]
\end{Arabic}}
\flushleft{\begin{hindi}
और जिस शख्स ने ज़र्रा बराबर बदी की है तो उसे देख लेगा
\end{hindi}}
\chapter{Al-'Adiyat (The Assaulters)}
\begin{Arabic}
\Huge{\centerline{\basmalah}}\end{Arabic}
\flushright{\begin{Arabic}
\quranayah[100][1]
\end{Arabic}}
\flushleft{\begin{hindi}
(ग़ाज़ियों के) सरपट दौड़ने वाले घोड़ो की क़सम
\end{hindi}}
\flushright{\begin{Arabic}
\quranayah[100][2]
\end{Arabic}}
\flushleft{\begin{hindi}
जो नथनों से फ़रराटे लेते हैं
\end{hindi}}
\flushright{\begin{Arabic}
\quranayah[100][3]
\end{Arabic}}
\flushleft{\begin{hindi}
फिर पत्थर पर टाप मारकर चिंगारियाँ निकालते हैं फिर सुबह को छापा मारते हैं
\end{hindi}}
\flushright{\begin{Arabic}
\quranayah[100][4]
\end{Arabic}}
\flushleft{\begin{hindi}
(तो दौड़ धूप से) बुलन्द कर देते हैं
\end{hindi}}
\flushright{\begin{Arabic}
\quranayah[100][5]
\end{Arabic}}
\flushleft{\begin{hindi}
फिर उस वक्त (दुश्मन के) दिल में घुस जाते हैं
\end{hindi}}
\flushright{\begin{Arabic}
\quranayah[100][6]
\end{Arabic}}
\flushleft{\begin{hindi}
(ग़रज़ क़सम है) कि बेशक इन्सान अपने परवरदिगार का नाशुक्रा है
\end{hindi}}
\flushright{\begin{Arabic}
\quranayah[100][7]
\end{Arabic}}
\flushleft{\begin{hindi}
और यक़ीनी ख़ुदा भी उससे वाक़िफ़ है
\end{hindi}}
\flushright{\begin{Arabic}
\quranayah[100][8]
\end{Arabic}}
\flushleft{\begin{hindi}
और बेशक वह माल का सख्त हरीस है
\end{hindi}}
\flushright{\begin{Arabic}
\quranayah[100][9]
\end{Arabic}}
\flushleft{\begin{hindi}
तो क्या वह ये नहीं जानता कि जब मुर्दे क़ब्रों से निकाले जाएँगे
\end{hindi}}
\flushright{\begin{Arabic}
\quranayah[100][10]
\end{Arabic}}
\flushleft{\begin{hindi}
और दिलों के भेद ज़ाहिर कर दिए जाएँगे
\end{hindi}}
\flushright{\begin{Arabic}
\quranayah[100][11]
\end{Arabic}}
\flushleft{\begin{hindi}
बेशक उस दिन उनका परवरदिगार उनसे ख़ूब वाक़िफ़ होगा
\end{hindi}}
\chapter{Al-Qari'ah (The Calamity)}
\begin{Arabic}
\Huge{\centerline{\basmalah}}\end{Arabic}
\flushright{\begin{Arabic}
\quranayah[101][1]
\end{Arabic}}
\flushleft{\begin{hindi}
खड़खड़ाने वाली
\end{hindi}}
\flushright{\begin{Arabic}
\quranayah[101][2]
\end{Arabic}}
\flushleft{\begin{hindi}
वह खड़खड़ाने वाली क्या है
\end{hindi}}
\flushright{\begin{Arabic}
\quranayah[101][3]
\end{Arabic}}
\flushleft{\begin{hindi}
और तुम को क्या मालूम कि वह खड़खड़ाने वाली क्या है
\end{hindi}}
\flushright{\begin{Arabic}
\quranayah[101][4]
\end{Arabic}}
\flushleft{\begin{hindi}
जिस दिन लोग (मैदाने हश्र में) टिड्डियों की तरह फैले होंगे
\end{hindi}}
\flushright{\begin{Arabic}
\quranayah[101][5]
\end{Arabic}}
\flushleft{\begin{hindi}
और पहाड़ धुनकी हुई रूई के से हो जाएँगे
\end{hindi}}
\flushright{\begin{Arabic}
\quranayah[101][6]
\end{Arabic}}
\flushleft{\begin{hindi}
तो जिसके (नेक आमाल) के पल्ले भारी होंगे
\end{hindi}}
\flushright{\begin{Arabic}
\quranayah[101][7]
\end{Arabic}}
\flushleft{\begin{hindi}
वह मन भाते ऐश में होंगे
\end{hindi}}
\flushright{\begin{Arabic}
\quranayah[101][8]
\end{Arabic}}
\flushleft{\begin{hindi}
और जिनके आमाल के पल्ले हल्के होंगे
\end{hindi}}
\flushright{\begin{Arabic}
\quranayah[101][9]
\end{Arabic}}
\flushleft{\begin{hindi}
तो उनका ठिकाना न रहा
\end{hindi}}
\flushright{\begin{Arabic}
\quranayah[101][10]
\end{Arabic}}
\flushleft{\begin{hindi}
और तुमको क्या मालूम हाविया क्या है
\end{hindi}}
\flushright{\begin{Arabic}
\quranayah[101][11]
\end{Arabic}}
\flushleft{\begin{hindi}
वह दहकती हुई आग है
\end{hindi}}
\chapter{At-Takathur (The Abundance of Wealth)}
\begin{Arabic}
\Huge{\centerline{\basmalah}}\end{Arabic}
\flushright{\begin{Arabic}
\quranayah[102][1]
\end{Arabic}}
\flushleft{\begin{hindi}
कुल व माल की बहुतायत ने तुम लोगों को ग़ाफ़िल रखा
\end{hindi}}
\flushright{\begin{Arabic}
\quranayah[102][2]
\end{Arabic}}
\flushleft{\begin{hindi}
यहाँ तक कि तुम लोगों ने कब्रें देखी (मर गए)
\end{hindi}}
\flushright{\begin{Arabic}
\quranayah[102][3]
\end{Arabic}}
\flushleft{\begin{hindi}
देखो तुमको अनक़रीब ही मालुम हो जाएगा
\end{hindi}}
\flushright{\begin{Arabic}
\quranayah[102][4]
\end{Arabic}}
\flushleft{\begin{hindi}
फिर देखो तुम्हें अनक़रीब ही मालूम हो जाएगा
\end{hindi}}
\flushright{\begin{Arabic}
\quranayah[102][5]
\end{Arabic}}
\flushleft{\begin{hindi}
देखो अगर तुमको यक़ीनी तौर पर मालूम होता (तो हरगिज़ ग़ाफिल न होते)
\end{hindi}}
\flushright{\begin{Arabic}
\quranayah[102][6]
\end{Arabic}}
\flushleft{\begin{hindi}
तुम लोग ज़रूर दोज़ख़ को देखोगे
\end{hindi}}
\flushright{\begin{Arabic}
\quranayah[102][7]
\end{Arabic}}
\flushleft{\begin{hindi}
फिर तुम लोग यक़ीनी देखना देखोगे
\end{hindi}}
\flushright{\begin{Arabic}
\quranayah[102][8]
\end{Arabic}}
\flushleft{\begin{hindi}
फिर तुमसे नेअमतों के बारें ज़रूर बाज़ पुर्स की जाएगी
\end{hindi}}
\chapter{Al-'Asr (The Time)}
\begin{Arabic}
\Huge{\centerline{\basmalah}}\end{Arabic}
\flushright{\begin{Arabic}
\quranayah[103][1]
\end{Arabic}}
\flushleft{\begin{hindi}
नमाज़े अस्र की क़सम
\end{hindi}}
\flushright{\begin{Arabic}
\quranayah[103][2]
\end{Arabic}}
\flushleft{\begin{hindi}
बेशक इन्सान घाटे में है
\end{hindi}}
\flushright{\begin{Arabic}
\quranayah[103][3]
\end{Arabic}}
\flushleft{\begin{hindi}
मगर जो लोग ईमान लाए, और अच्छे काम करते रहे और आपस में हक़ का हुक्म और सब्र की वसीयत करते रहे
\end{hindi}}
\chapter{Al-Humazah (The Slanderer)}
\begin{Arabic}
\Huge{\centerline{\basmalah}}\end{Arabic}
\flushright{\begin{Arabic}
\quranayah[104][1]
\end{Arabic}}
\flushleft{\begin{hindi}
हर ताना देने वाले चुग़लख़ोर की ख़राबी है
\end{hindi}}
\flushright{\begin{Arabic}
\quranayah[104][2]
\end{Arabic}}
\flushleft{\begin{hindi}
जो माल को जमा करता है और गिन गिन कर रखता है
\end{hindi}}
\flushright{\begin{Arabic}
\quranayah[104][3]
\end{Arabic}}
\flushleft{\begin{hindi}
वह समझता है कि उसका माल उसे हमेशा ज़िन्दा बाक़ी रखेगा
\end{hindi}}
\flushright{\begin{Arabic}
\quranayah[104][4]
\end{Arabic}}
\flushleft{\begin{hindi}
हरगिज़ नहीं वह तो ज़रूर हुतमा में डाला जाएगा
\end{hindi}}
\flushright{\begin{Arabic}
\quranayah[104][5]
\end{Arabic}}
\flushleft{\begin{hindi}
और तुमको क्या मालूम हतमा क्या है
\end{hindi}}
\flushright{\begin{Arabic}
\quranayah[104][6]
\end{Arabic}}
\flushleft{\begin{hindi}
वह ख़ुदा की भड़काई हुई आग है जो (तलवे से लगी तो) दिलों तक चढ़ जाएगी
\end{hindi}}
\flushright{\begin{Arabic}
\quranayah[104][7]
\end{Arabic}}
\flushleft{\begin{hindi}
ये लोग आग के लम्बे सुतूनो
\end{hindi}}
\flushright{\begin{Arabic}
\quranayah[104][8]
\end{Arabic}}
\flushleft{\begin{hindi}
में डाल कर बन्द कर दिए
\end{hindi}}
\flushright{\begin{Arabic}
\quranayah[104][9]
\end{Arabic}}
\flushleft{\begin{hindi}
जाएँगे
\end{hindi}}
\chapter{Al-Fil (The Elephant)}
\begin{Arabic}
\Huge{\centerline{\basmalah}}\end{Arabic}
\flushright{\begin{Arabic}
\quranayah[105][1]
\end{Arabic}}
\flushleft{\begin{hindi}
ऐ रसूल क्या तुमने नहीं देखा कि तुम्हारे परवरदिगार ने हाथी वालों के साथ क्या किया
\end{hindi}}
\flushright{\begin{Arabic}
\quranayah[105][2]
\end{Arabic}}
\flushleft{\begin{hindi}
क्या उसने उनकी तमाम तद्बीरें ग़लत नहीं कर दीं (ज़रूर)
\end{hindi}}
\flushright{\begin{Arabic}
\quranayah[105][3]
\end{Arabic}}
\flushleft{\begin{hindi}
और उन पर झुन्ड की झुन्ड चिड़ियाँ भेज दीं
\end{hindi}}
\flushright{\begin{Arabic}
\quranayah[105][4]
\end{Arabic}}
\flushleft{\begin{hindi}
जो उन पर खरन्जों की कंकरियाँ फेकती थीं
\end{hindi}}
\flushright{\begin{Arabic}
\quranayah[105][5]
\end{Arabic}}
\flushleft{\begin{hindi}
तो उन्हें चबाए हुए भूस की (तबाह) कर दिया
\end{hindi}}
\chapter{Al-Quraish (The Quraish)}
\begin{Arabic}
\Huge{\centerline{\basmalah}}\end{Arabic}
\flushright{\begin{Arabic}
\quranayah[106][1]
\end{Arabic}}
\flushleft{\begin{hindi}
चूँकि क़ुरैश को जाड़े और गर्मी के सफ़र से मानूस कर दिया है
\end{hindi}}
\flushright{\begin{Arabic}
\quranayah[106][2]
\end{Arabic}}
\flushleft{\begin{hindi}
तो उनको मानूस कर देने की वजह से
\end{hindi}}
\flushright{\begin{Arabic}
\quranayah[106][3]
\end{Arabic}}
\flushleft{\begin{hindi}
इस घर (काबा) के मालिक की इबादत करनी चाहिए
\end{hindi}}
\flushright{\begin{Arabic}
\quranayah[106][4]
\end{Arabic}}
\flushleft{\begin{hindi}
जिसने उनको भूख में खाना दिया और उनको खौफ़ से अमन अता किया
\end{hindi}}
\chapter{Al-Ma'un (Acts of Kindness)}
\begin{Arabic}
\Huge{\centerline{\basmalah}}\end{Arabic}
\flushright{\begin{Arabic}
\quranayah[107][1]
\end{Arabic}}
\flushleft{\begin{hindi}
क्या तुमने उस शख़्श को भी देखा है जो रोज़ जज़ा को झुठलाता है
\end{hindi}}
\flushright{\begin{Arabic}
\quranayah[107][2]
\end{Arabic}}
\flushleft{\begin{hindi}
ये तो वही (कम्बख्त) है जो यतीम को धक्के देता है
\end{hindi}}
\flushright{\begin{Arabic}
\quranayah[107][3]
\end{Arabic}}
\flushleft{\begin{hindi}
और मोहताजों को खिलाने के लिए (लोगों को) आमादा नहीं करता
\end{hindi}}
\flushright{\begin{Arabic}
\quranayah[107][4]
\end{Arabic}}
\flushleft{\begin{hindi}
तो उन नमाज़ियों की तबाही है
\end{hindi}}
\flushright{\begin{Arabic}
\quranayah[107][5]
\end{Arabic}}
\flushleft{\begin{hindi}
जो अपनी नमाज़ से ग़ाफिल रहते हैं
\end{hindi}}
\flushright{\begin{Arabic}
\quranayah[107][6]
\end{Arabic}}
\flushleft{\begin{hindi}
जो दिखाने के वास्ते करते हैं
\end{hindi}}
\flushright{\begin{Arabic}
\quranayah[107][7]
\end{Arabic}}
\flushleft{\begin{hindi}
और रोज़मर्रा की मालूली चीज़ें भी आरियत नहीं देते
\end{hindi}}
\chapter{Al-Kauthar (The Abundance of Good)}
\begin{Arabic}
\Huge{\centerline{\basmalah}}\end{Arabic}
\flushright{\begin{Arabic}
\quranayah[108][1]
\end{Arabic}}
\flushleft{\begin{hindi}
(ऐ रसूल) हमनें तुमको को कौसर अता किया,
\end{hindi}}
\flushright{\begin{Arabic}
\quranayah[108][2]
\end{Arabic}}
\flushleft{\begin{hindi}
तो तुम अपने परवरदिगार की नमाज़ पढ़ा करो
\end{hindi}}
\flushright{\begin{Arabic}
\quranayah[108][3]
\end{Arabic}}
\flushleft{\begin{hindi}
और क़ुर्बानी दिया करो बेशक तुम्हारा दुश्मन बे औलाद रहेगा
\end{hindi}}
\chapter{Al-Kafirun (The Disbelievers)}
\begin{Arabic}
\Huge{\centerline{\basmalah}}\end{Arabic}
\flushright{\begin{Arabic}
\quranayah[109][1]
\end{Arabic}}
\flushleft{\begin{hindi}
(ऐ रसूल) तुम कह दो कि ऐ काफिरों
\end{hindi}}
\flushright{\begin{Arabic}
\quranayah[109][2]
\end{Arabic}}
\flushleft{\begin{hindi}
तुम जिन चीज़ों को पूजते हो, मैं उनको नहीं पूजता
\end{hindi}}
\flushright{\begin{Arabic}
\quranayah[109][3]
\end{Arabic}}
\flushleft{\begin{hindi}
और जिस (ख़ुदा) की मैं इबादत करता हूँ उसकी तुम इबादत नहीं करते
\end{hindi}}
\flushright{\begin{Arabic}
\quranayah[109][4]
\end{Arabic}}
\flushleft{\begin{hindi}
और जिन्हें तुम पूजते हो मैं उनका पूजने वाला नहीं
\end{hindi}}
\flushright{\begin{Arabic}
\quranayah[109][5]
\end{Arabic}}
\flushleft{\begin{hindi}
और जिसकी मैं इबादत करता हूँ उसकी तुम इबादत करने वाले नहीं
\end{hindi}}
\flushright{\begin{Arabic}
\quranayah[109][6]
\end{Arabic}}
\flushleft{\begin{hindi}
तुम्हारे लिए तुम्हारा दीन मेरे लिए मेरा दीन
\end{hindi}}
\chapter{An-Nasr (The Help)}
\begin{Arabic}
\Huge{\centerline{\basmalah}}\end{Arabic}
\flushright{\begin{Arabic}
\quranayah[110][1]
\end{Arabic}}
\flushleft{\begin{hindi}
ऐ रसूल जब ख़ुदा की मदद आ पहँचेगी
\end{hindi}}
\flushright{\begin{Arabic}
\quranayah[110][2]
\end{Arabic}}
\flushleft{\begin{hindi}
और फतेह (मक्का) हो जाएगी और तुम लोगों को देखोगे कि गोल के गोल ख़ुदा के दीन में दाख़िल हो रहे हैं
\end{hindi}}
\flushright{\begin{Arabic}
\quranayah[110][3]
\end{Arabic}}
\flushleft{\begin{hindi}
तो तुम अपने परवरदिगार की तारीफ़ के साथ तसबीह करना और उसी से मग़फेरत की दुआ माँगना वह बेशक बड़ा माफ़ करने वाला है
\end{hindi}}
\chapter{Al-Lahab (The Flame)}
\begin{Arabic}
\Huge{\centerline{\basmalah}}\end{Arabic}
\flushright{\begin{Arabic}
\quranayah[111][1]
\end{Arabic}}
\flushleft{\begin{hindi}
अबु लहब के हाथ टूट जाएँ और वह ख़ुद सत्यानास हो जाए
\end{hindi}}
\flushright{\begin{Arabic}
\quranayah[111][2]
\end{Arabic}}
\flushleft{\begin{hindi}
(आख़िर) न उसका माल ही उसके हाथ आया और (न) उसने कमाया
\end{hindi}}
\flushright{\begin{Arabic}
\quranayah[111][3]
\end{Arabic}}
\flushleft{\begin{hindi}
वह बहुत भड़कती हुई आग में दाख़िल होगा
\end{hindi}}
\flushright{\begin{Arabic}
\quranayah[111][4]
\end{Arabic}}
\flushleft{\begin{hindi}
और उसकी जोरू भी जो सर पर ईंधन उठाए फिरती है
\end{hindi}}
\flushright{\begin{Arabic}
\quranayah[111][5]
\end{Arabic}}
\flushleft{\begin{hindi}
और उसके गले में बटी हुई रस्सी बँधी है
\end{hindi}}
\chapter{Al-Ikhlas (The Unity)}
\begin{Arabic}
\Huge{\centerline{\basmalah}}\end{Arabic}
\flushright{\begin{Arabic}
\quranayah[112][1]
\end{Arabic}}
\flushleft{\begin{hindi}
(ऐ रसूल) तुम कह दो कि ख़ुदा एक है
\end{hindi}}
\flushright{\begin{Arabic}
\quranayah[112][2]
\end{Arabic}}
\flushleft{\begin{hindi}
ख़ुदा बरहक़ बेनियाज़ है
\end{hindi}}
\flushright{\begin{Arabic}
\quranayah[112][3]
\end{Arabic}}
\flushleft{\begin{hindi}
न उसने किसी को जना न उसको किसी ने जना,
\end{hindi}}
\flushright{\begin{Arabic}
\quranayah[112][4]
\end{Arabic}}
\flushleft{\begin{hindi}
और उसका कोई हमसर नहीं
\end{hindi}}
\chapter{Al-Falaq (The Dawn)}
\begin{Arabic}
\Huge{\centerline{\basmalah}}\end{Arabic}
\flushright{\begin{Arabic}
\quranayah[113][1]
\end{Arabic}}
\flushleft{\begin{hindi}
(ऐ रसूल) तुम कह दो कि मैं सुबह के मालिक की
\end{hindi}}
\flushright{\begin{Arabic}
\quranayah[113][2]
\end{Arabic}}
\flushleft{\begin{hindi}
हर चीज़ की बुराई से जो उसने पैदा की पनाह माँगता हूँ
\end{hindi}}
\flushright{\begin{Arabic}
\quranayah[113][3]
\end{Arabic}}
\flushleft{\begin{hindi}
और अंधेरीरात की बुराई से जब उसका अंधेरा छा जाए
\end{hindi}}
\flushright{\begin{Arabic}
\quranayah[113][4]
\end{Arabic}}
\flushleft{\begin{hindi}
और गन्डों पर फूँकने वालियों की बुराई से
\end{hindi}}
\flushright{\begin{Arabic}
\quranayah[113][5]
\end{Arabic}}
\flushleft{\begin{hindi}
(जब फूँके) और हसद करने वाले की बुराई से
\end{hindi}}
\chapter{An-Nas (The Men)}
\begin{Arabic}
\Huge{\centerline{\basmalah}}\end{Arabic}
\flushright{\begin{Arabic}
\quranayah[114][1]
\end{Arabic}}
\flushleft{\begin{hindi}
(ऐ रसूल) तुम कह दो मैं लोगों के परवरदिगार
\end{hindi}}
\flushright{\begin{Arabic}
\quranayah[114][2]
\end{Arabic}}
\flushleft{\begin{hindi}
लोगों के बादशाह
\end{hindi}}
\flushright{\begin{Arabic}
\quranayah[114][3]
\end{Arabic}}
\flushleft{\begin{hindi}
लोगों के माबूद की (शैतानी)
\end{hindi}}
\flushright{\begin{Arabic}
\quranayah[114][4]
\end{Arabic}}
\flushleft{\begin{hindi}
वसवसे की बुराई से पनाह माँगता हूँ
\end{hindi}}
\flushright{\begin{Arabic}
\quranayah[114][5]
\end{Arabic}}
\flushleft{\begin{hindi}
जो (ख़ुदा के नाम से) पीछे हट जाता है जो लोगों के दिलों में वसवसे डाला करता है
\end{hindi}}
\flushright{\begin{Arabic}
\quranayah[114][6]
\end{Arabic}}
\flushleft{\begin{hindi}
जिन्नात में से ख्वाह आदमियों में से
\end{hindi}}
