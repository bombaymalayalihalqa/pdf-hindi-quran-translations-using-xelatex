\chapter{Al-Fatihah (The Opening)}
\begin{Arabic}
\Huge{\centerline{\basmalah}}\end{Arabic}
\flushright{\begin{Arabic}
\quranayah[1][1]
\end{Arabic}}
\flushleft{\begin{hindi}
अल्लाह के नाम से जो बड़ा कृपालु और अत्यन्त दयावान हैं।
\end{hindi}}
\flushright{\begin{Arabic}
\quranayah[1][2]
\end{Arabic}}
\flushleft{\begin{hindi}
प्रशंसा अल्लाह ही के लिए हैं जो सारे संसार का रब हैं
\end{hindi}}
\flushright{\begin{Arabic}
\quranayah[1][3]
\end{Arabic}}
\flushleft{\begin{hindi}
बड़ा कृपालु, अत्यन्त दयावान हैं
\end{hindi}}
\flushright{\begin{Arabic}
\quranayah[1][4]
\end{Arabic}}
\flushleft{\begin{hindi}
बदला दिए जाने के दिन का मालिक हैं
\end{hindi}}
\flushright{\begin{Arabic}
\quranayah[1][5]
\end{Arabic}}
\flushleft{\begin{hindi}
हम तेरी बन्दगी करते हैं और तुझी से मदद माँगते हैं
\end{hindi}}
\flushright{\begin{Arabic}
\quranayah[1][6]
\end{Arabic}}
\flushleft{\begin{hindi}
हमें सीधे मार्ग पर चला
\end{hindi}}
\flushright{\begin{Arabic}
\quranayah[1][7]
\end{Arabic}}
\flushleft{\begin{hindi}
उन लोगों के मार्ग पर जो तेरे कृपापात्र हुए, जो न प्रकोप के भागी हुए और न पथभ्रष्ट
\end{hindi}}
\chapter{Al-Baqarah (The Cow)}
\begin{Arabic}
\Huge{\centerline{\basmalah}}\end{Arabic}
\flushright{\begin{Arabic}
\quranayah[2][1]
\end{Arabic}}
\flushleft{\begin{hindi}
अलीफ़॰ लाम॰ मीम॰
\end{hindi}}
\flushright{\begin{Arabic}
\quranayah[2][2]
\end{Arabic}}
\flushleft{\begin{hindi}
वह किताब यही हैं, जिसमें कोई सन्देह नहीं, मार्गदर्शन हैं डर रखनेवालों के लिए,
\end{hindi}}
\flushright{\begin{Arabic}
\quranayah[2][3]
\end{Arabic}}
\flushleft{\begin{hindi}
जो अनदेखे ईमान लाते हैं, नमाज़ क़ायम करते हैं और जो कुछ हमने उन्हें दिया हैं उसमें से कुछ खर्च करते हैं;
\end{hindi}}
\flushright{\begin{Arabic}
\quranayah[2][4]
\end{Arabic}}
\flushleft{\begin{hindi}
और जो उस पर ईमान लाते हैं जो तुम पर उतरा और जो तुमसे पहले अवतरित हुआ हैं और आख़िरत पर वही लोग विश्वास रखते हैं;
\end{hindi}}
\flushright{\begin{Arabic}
\quranayah[2][5]
\end{Arabic}}
\flushleft{\begin{hindi}
वही लोग हैं जो अपने रब के सीधे मार्ग पर हैं और वही सफलता प्राप्त करनेवाले हैं
\end{hindi}}
\flushright{\begin{Arabic}
\quranayah[2][6]
\end{Arabic}}
\flushleft{\begin{hindi}
जिन लोगों ने कुफ़्र (इनकार) किया उनके लिए बराबर हैं, चाहे तुमने उन्हें सचेत किया हो या सचेत न किया हो, वे ईमान नहीं लाएँगे
\end{hindi}}
\flushright{\begin{Arabic}
\quranayah[2][7]
\end{Arabic}}
\flushleft{\begin{hindi}
अल्लाह ने उनके दिलों पर और कानों पर मुहर लगा दी है और उनकी आँखों पर परदा पड़ा है, और उनके लिए बड़ी यातना है
\end{hindi}}
\flushright{\begin{Arabic}
\quranayah[2][8]
\end{Arabic}}
\flushleft{\begin{hindi}
कुछ लोग ऐसे हैं जो कहते हैं कि हम अल्लाह और अन्तिम दिन पर ईमान रखते हैं, हालाँकि वे ईमान नहीं रखते
\end{hindi}}
\flushright{\begin{Arabic}
\quranayah[2][9]
\end{Arabic}}
\flushleft{\begin{hindi}
वे अल्लाह और ईमानवालों के साथ धोखेबाज़ी कर रहे हैं, हालाँकि धोखा वे स्वयं अपने-आपको ही दे रहे हैं, परन्तु वे इसको महसूस नहीं करते
\end{hindi}}
\flushright{\begin{Arabic}
\quranayah[2][10]
\end{Arabic}}
\flushleft{\begin{hindi}
उनके दिलों में रोग था तो अल्लाह ने उनके रोग को और बढ़ा दिया और उनके लिए झूठ बोलते रहने के कारण उनके लिए एक दुखद यातना है
\end{hindi}}
\flushright{\begin{Arabic}
\quranayah[2][11]
\end{Arabic}}
\flushleft{\begin{hindi}
और जब उनसे कहा जाता है कि "ज़मीन में बिगाड़ पैदा न करो", तो कहते हैं, "हम तो केवल सुधारक है।""
\end{hindi}}
\flushright{\begin{Arabic}
\quranayah[2][12]
\end{Arabic}}
\flushleft{\begin{hindi}
जान लो! वही हैं जो बिगाड़ पैदा करते हैं, परन्तु उन्हें एहसास नहीं होता
\end{hindi}}
\flushright{\begin{Arabic}
\quranayah[2][13]
\end{Arabic}}
\flushleft{\begin{hindi}
और जब उनसे कहा जाता है, "ईमान लाओ जैसे लोग ईमान लाए हैं", कहते हैं, "क्या हम ईमान लाए जैसे कम समझ लोग ईमान लाए हैं?" जान लो, वही कम समझ हैं परन्तु जानते नहीं
\end{hindi}}
\flushright{\begin{Arabic}
\quranayah[2][14]
\end{Arabic}}
\flushleft{\begin{hindi}
और जब ईमान लानेवालों से मिलते हैं तो कहते, "हम भी ईमान लाए हैं," और जब एकान्त में अपने शैतानों के पास पहुँचते हैं, तो कहते हैं, "हम तो तुम्हारे साथ हैं और यह तो हम केवल परिहास कर रहे हैं।"
\end{hindi}}
\flushright{\begin{Arabic}
\quranayah[2][15]
\end{Arabic}}
\flushleft{\begin{hindi}
अल्लाह उनके साथ परिहास कर रहा है और उन्हें उनकी सरकशी में ढील दिए जाता है, वे भटकते फिर रहे हैं
\end{hindi}}
\flushright{\begin{Arabic}
\quranayah[2][16]
\end{Arabic}}
\flushleft{\begin{hindi}
यही वे लोग हैं, जिन्होंने मार्गदर्शन के बदले में गुमराही मोल ली, किन्तु उनके इस व्यापार में न कोई लाभ पहुँचाया, और न ही वे सीधा मार्ग पा सके
\end{hindi}}
\flushright{\begin{Arabic}
\quranayah[2][17]
\end{Arabic}}
\flushleft{\begin{hindi}
उनकी मिसाल ऐसी हैं जैसे किसी व्यक्ति ने आग जलाई, फिर जब उसने उसके वातावरण को प्रकाशित कर दिया, तो अल्लाह ने उसका प्रकाश ही छीन लिया और उन्हें अँधेरों में छोड़ दिया जिससे उन्हें कुछ सुझाई नहीं दे रहा हैं
\end{hindi}}
\flushright{\begin{Arabic}
\quranayah[2][18]
\end{Arabic}}
\flushleft{\begin{hindi}
वे बहरे हैं, गूँगें हैं, अन्धे हैं, अब वे लौटने के नहीं
\end{hindi}}
\flushright{\begin{Arabic}
\quranayah[2][19]
\end{Arabic}}
\flushleft{\begin{hindi}
या (उनकी मिसाल ऐसी है) जैसे आकाश से वर्षा हो रही हो जिसके साथ अँधेरे हों और गरज और चमक भी हो, वे बिजली की कड़क के कारण मृत्यु के भय से अपने कानों में उँगलियाँ दे ले रहे हों - और अल्लाह ने तो इनकार करनेवालों को घेर रखा हैं
\end{hindi}}
\flushright{\begin{Arabic}
\quranayah[2][20]
\end{Arabic}}
\flushleft{\begin{hindi}
मानो शीघ्र ही बिजली उनकी आँखों की रौशनी उचक लेने को है; जब भी उनपर चमकती हो, वे चल पड़ते हो और जब उनपर अँधेरा छा जाता हैं तो खड़े हो जाते हो; अगर अल्लाह चाहता तो उनकी सुनने और देखने की शक्ति बिलकुल ही छीन लेता। निस्सन्देह अल्लाह को हर चीज़ की सामर्थ्य प्राप्त है
\end{hindi}}
\flushright{\begin{Arabic}
\quranayah[2][21]
\end{Arabic}}
\flushleft{\begin{hindi}
ऐ लोगो! बन्दगी करो अपने रब की जिसने तुम्हें और तुमसे पहले के लोगों को पैदा किया, ताकि तुम बच सको;
\end{hindi}}
\flushright{\begin{Arabic}
\quranayah[2][22]
\end{Arabic}}
\flushleft{\begin{hindi}
वही है जिसने तुम्हारे लिए ज़मीन को फर्श और आकाश को छत बनाया, और आकाश से पानी उतारा, फिर उसके द्वारा हर प्रकार की पैदावार की और फल तुम्हारी रोजी के लिए पैदा किए, अतः जब तुम जानते हो तो अल्लाह के समकक्ष न ठहराओ
\end{hindi}}
\flushright{\begin{Arabic}
\quranayah[2][23]
\end{Arabic}}
\flushleft{\begin{hindi}
और अगर उसके विषय में जो हमने अपने बन्दे पर उतारा हैं, तुम किसी सन्देह में न हो तो उस जैसी कोई सूरा ले आओ और अल्लाह से हटकर अपने सहायकों को बुला लो जिनके आ मौजूद होने पर तुम्हें विश्वास हैं, यदि तुम सच्चे हो
\end{hindi}}
\flushright{\begin{Arabic}
\quranayah[2][24]
\end{Arabic}}
\flushleft{\begin{hindi}
फिर अगर तुम ऐसा न कर सको और तुम कदापि नहीं कर सकते, तो डरो उस आग से जिसका ईधन इनसान और पत्थर हैं, जो इनकार करनेवालों के लिए तैयार की गई है
\end{hindi}}
\flushright{\begin{Arabic}
\quranayah[2][25]
\end{Arabic}}
\flushleft{\begin{hindi}
जो लोग ईमान लाए और उन्होंने अच्छे कर्म किए उन्हें शुभ सूचना दे दो कि उनके लिए ऐसे बाग़ है जिनके नीचे नहरें बह रहीं होगी; जब भी उनमें से कोई फल उन्हें रोजी के रूप में मिलेगा, तो कहेंगे, "यह तो वही हैं जो पहले हमें मिला था," और उन्हें मिलता-जुलता ही (फल) मिलेगा; उनके लिए वहाँ पाक-साफ़ पत्नि याँ होगी, और वे वहाँ सदैव रहेंगे
\end{hindi}}
\flushright{\begin{Arabic}
\quranayah[2][26]
\end{Arabic}}
\flushleft{\begin{hindi}
निस्संदेह अल्लाह नहीं शरमाता कि वह कोई मिसाल पेश करे चाहे वह हो मच्छर की, बल्कि उससे भी बढ़कर किसी तुच्छ चीज़ की। फिर जो ईमान लाए है वे तो जानते है कि वह उनके रब की ओर से सत्य हैं; रहे इनकार करनेवाले तो वे कहते है, "इस मिसाल से अल्लाह का अभिप्राय क्या है?" इससे वह बहुतों को भटकने देता है और बहुतों को सीधा मार्ग दिखा देता है, मगर इससे वह केवल अवज्ञाकारियों ही को भटकने देता है
\end{hindi}}
\flushright{\begin{Arabic}
\quranayah[2][27]
\end{Arabic}}
\flushleft{\begin{hindi}
जो अल्लाह की प्रतिज्ञा को उसे सुदृढ़ करने के पश्चात भंग कर देते हैं और जिसे अल्लाह ने जोड़ने का आदेश दिया है उसे काट डालते हैं, और ज़मीन में बिगाड़ पैदा करते हैं, वही हैं जो घाटे में हैं
\end{hindi}}
\flushright{\begin{Arabic}
\quranayah[2][28]
\end{Arabic}}
\flushleft{\begin{hindi}
तुम अल्लाह के साथ अविश्वास की नीति कैसे अपनाते हो, जबकि तुम निर्जीव थे तो उसने तुम्हें जीवित किया, फिर वही तुम्हें मौत देता हैं, फिर वही तुम्हें जीवित करेगा, फिर उसी की ओर तुम्हें लौटना हैं?
\end{hindi}}
\flushright{\begin{Arabic}
\quranayah[2][29]
\end{Arabic}}
\flushleft{\begin{hindi}
वही तो है जिसने तुम्हारे लिए ज़मीन की सारी चीज़े पैदा की, फिर आकाश की ओर रुख़ किया और ठीक तौर पर सात आकाश बनाए और वह हर चीज़ को जानता है
\end{hindi}}
\flushright{\begin{Arabic}
\quranayah[2][30]
\end{Arabic}}
\flushleft{\begin{hindi}
और याद करो जब तुम्हारे रब ने फरिश्तों से कहा कि "मैं धरती में (मनुष्य को) खलीफ़ा (सत्ताधारी) बनानेवाला हूँ।" उन्होंने कहा, "क्या उसमें उसको रखेगा, जो उसमें बिगाड़ पैदा करे और रक्तपात करे और हम तेरा गुणगान करते और तुझे पवित्र कहते हैं?" उसने कहा, "मैं जानता हूँ जो तुम नहीं जानते।"
\end{hindi}}
\flushright{\begin{Arabic}
\quranayah[2][31]
\end{Arabic}}
\flushleft{\begin{hindi}
उसने (अल्लाह ने) आदम को सारे नाम सिखाए, फिर उन्हें फ़रिश्तों के सामने पेश किया और कहा, "अगर तुम सच्चे हो तो मुझे इनके नाम बताओ।"
\end{hindi}}
\flushright{\begin{Arabic}
\quranayah[2][32]
\end{Arabic}}
\flushleft{\begin{hindi}
वे बोले, "पाक और महिमावान है तू! तूने जो कुछ हमें बताया उसके सिवा हमें कोई ज्ञान नहीं। निस्संदेह तू सर्वज्ञ, तत्वदर्शी है।"
\end{hindi}}
\flushright{\begin{Arabic}
\quranayah[2][33]
\end{Arabic}}
\flushleft{\begin{hindi}
उसने कहा, "ऐ आदम! उन्हें उन लोगों के नाम बताओ।" फिर जब उसने उन्हें उनके नाम बता दिए तो (अल्लाह ने) कहा, "क्या मैंने तुमसे कहा न था कि मैं आकाशों और धरती की छिपी बातों को जानता हूँ और मैं जानता हूँ जो कुछ तुम ज़ाहिर करते हो और जो कुछ छिपाते हो।"
\end{hindi}}
\flushright{\begin{Arabic}
\quranayah[2][34]
\end{Arabic}}
\flushleft{\begin{hindi}
और याद करो जब हमने फ़रिश्तों से कहा कि "आदम को सजदा करो" तो, उन्होंने सजदा किया सिवाय इबलील के; उसने इनकार कर दिया और लगा बड़ा बनने और काफ़िर हो रहा
\end{hindi}}
\flushright{\begin{Arabic}
\quranayah[2][35]
\end{Arabic}}
\flushleft{\begin{hindi}
और हमने कहा, "ऐ आदम! तुम और तुम्हारी पत्नी जन्नत में रहो और वहाँ जी भर बेरोक-टोक जहाँ से तुम दोनों का जी चाहे खाओ, लेकिन इस वृक्ष के पास न जाना, अन्यथा तुम ज़ालिम ठहरोगे।"
\end{hindi}}
\flushright{\begin{Arabic}
\quranayah[2][36]
\end{Arabic}}
\flushleft{\begin{hindi}
अन्ततः शैतान ने उन्हें वहाँ से फिसला दिया, फिर उन दोनों को वहाँ से निकलवाकर छोड़ा, जहाँ वे थे। हमने कहा कि "उतरो, तुम एक-दूसरे के शत्रु होगे और तुम्हें एक समय तक धरती में ठहरना और बिसलना है।"
\end{hindi}}
\flushright{\begin{Arabic}
\quranayah[2][37]
\end{Arabic}}
\flushleft{\begin{hindi}
फिर आदम ने अपने रब से कुछ शब्द पा लिए, तो अल्लाह ने उसकी तौबा क़बूल कर ली; निस्संदेह वही तौबा क़बूल करने वाला, अत्यन्त दयावान है
\end{hindi}}
\flushright{\begin{Arabic}
\quranayah[2][38]
\end{Arabic}}
\flushleft{\begin{hindi}
हमने कहा, "तुम सब यहाँ से उतरो, फिर यदि तुम्हारे पास मेरी ओर से कोई मार्गदर्शन पहुँचे तो जिस किसी ने मेरे मार्गदर्शन का अनुसरण किया, तो ऐसे लोगों को न तो कोई भय होगा और न वे शोकाकुल होंगे।"
\end{hindi}}
\flushright{\begin{Arabic}
\quranayah[2][39]
\end{Arabic}}
\flushleft{\begin{hindi}
और जिन लोगों ने इनकार किया और हमारी आयतों को झुठलाया, वहीं आग में पड़नेवाले हैं, वे उसमें सदैव रहेंगे
\end{hindi}}
\flushright{\begin{Arabic}
\quranayah[2][40]
\end{Arabic}}
\flushleft{\begin{hindi}
ऐ इसराईल का सन्तान! याद करो मेरे उस अनुग्रह को जो मैंने तुमपर किया था। और मेरी प्रतिज्ञा को पूरा करो, मैं तुमसे की हुई प्रतिज्ञा को पूरा करूँगा और हाँ मुझी से डरो
\end{hindi}}
\flushright{\begin{Arabic}
\quranayah[2][41]
\end{Arabic}}
\flushleft{\begin{hindi}
और ईमान लाओ उस चीज़ पर जो मैंने उतारी है, जो उसकी पुष्टि में है, जो तुम्हारे पास है, और सबसे पहले तुम ही उसके इनकार करनेवाले न बनो। और मेरी आयतों को थोड़ा मूल्य प्राप्त करने का साधन न बनाओ, मुझसे ही तुम डरो
\end{hindi}}
\flushright{\begin{Arabic}
\quranayah[2][42]
\end{Arabic}}
\flushleft{\begin{hindi}
और सत्य में असत्य का घाल-मेल न करो और जानते-बुझते सत्य को छिपाओ मत
\end{hindi}}
\flushright{\begin{Arabic}
\quranayah[2][43]
\end{Arabic}}
\flushleft{\begin{hindi}
और नमाज़ क़ायम करो और ज़कात दो और (मेरे समक्ष) झुकनेवालों के साथ झुको
\end{hindi}}
\flushright{\begin{Arabic}
\quranayah[2][44]
\end{Arabic}}
\flushleft{\begin{hindi}
क्या तुम लोगों को तो नेकी और एहसान का उपदेश देते हो और अपने आपको भूल जाते हो, हालाँकि तुम किताब भी पढ़ते हो? फिर क्या तुम बुद्धि से काम नहीं लेते?
\end{hindi}}
\flushright{\begin{Arabic}
\quranayah[2][45]
\end{Arabic}}
\flushleft{\begin{hindi}
धैर्य और नमाज़ से मदद लो, और निस्संदेह यह (नमाज) बहुत कठिन है, किन्तु उन लोगों के लिए नहीं जिनके दिल पिघले हुए हो;
\end{hindi}}
\flushright{\begin{Arabic}
\quranayah[2][46]
\end{Arabic}}
\flushleft{\begin{hindi}
जो समझते है कि उन्हें अपने रब से मिलना हैं और उसी की ओर उन्हें पलटकर जाना है
\end{hindi}}
\flushright{\begin{Arabic}
\quranayah[2][47]
\end{Arabic}}
\flushleft{\begin{hindi}
ऐ इसराईल की सन्तान! याद करो मेरे उस अनुग्रह को जो मैंने तुमपर किया और इसे भी कि मैंने तुम्हें सारे संसार पर श्रेष्ठता प्रदान की थी;
\end{hindi}}
\flushright{\begin{Arabic}
\quranayah[2][48]
\end{Arabic}}
\flushleft{\begin{hindi}
और डरो उस दिन से जब न कोई किसी भी ओर से कुछ तावान भरेगा और न किसी की ओर से कोई सिफ़ारिश ही क़बूल की जाएगी और न किसी की ओर से कोई फ़िद्‌या (अर्थदंड) लिया जाएगा और न वे सहायता ही पा सकेंगे।
\end{hindi}}
\flushright{\begin{Arabic}
\quranayah[2][49]
\end{Arabic}}
\flushleft{\begin{hindi}
और याद करो जब हमने तुम्हें फ़िरऔनियों से छुटकारा दिलाया जो तुम्हें अत्यन्त बुरी यातना देते थे, तुम्हारे बेटों को मार डालते थे और तुम्हारी स्त्रियों को जीवित रहने देते थे; और इसमं तुम्हारे रब की ओर से बड़ी परीक्षा थी
\end{hindi}}
\flushright{\begin{Arabic}
\quranayah[2][50]
\end{Arabic}}
\flushleft{\begin{hindi}
याद करो जब हमने तुम्हें सागर में अलग-अलग चौड़े रास्ते से ले जाकर छुटकारा दिया और फ़िरऔनियों को तुम्हारी आँखों के सामने डूबो दिया
\end{hindi}}
\flushright{\begin{Arabic}
\quranayah[2][51]
\end{Arabic}}
\flushleft{\begin{hindi}
और याद करो जब हमने मूसा से चालीस रातों का वादा ठहराया तो उसके पीछे तुम बछड़े को अपना देवता बना बैठे, तुम अत्याचारी थे
\end{hindi}}
\flushright{\begin{Arabic}
\quranayah[2][52]
\end{Arabic}}
\flushleft{\begin{hindi}
फिर इसके पश्चात भी हमने तुम्हें क्षमा किया, ताकि तुम कृतज्ञता दिखालाओ
\end{hindi}}
\flushright{\begin{Arabic}
\quranayah[2][53]
\end{Arabic}}
\flushleft{\begin{hindi}
और याद करो जब मूसा को हमने किताब और कसौटी प्रदान की, ताकि तुम मार्ग पा सको
\end{hindi}}
\flushright{\begin{Arabic}
\quranayah[2][54]
\end{Arabic}}
\flushleft{\begin{hindi}
और जब मूसा ने अपनी क़ौम से कहा, "ऐ मेरी कौम के लोगो! बछड़े को देवता बनाकर तुमने अपने ऊपर ज़ुल्म किया है, तो तुम अपने पैदा करनेवाले की ओर पलटो, अतः अपने लोगों को स्वयं क़त्ल करो। यही तुम्हारे पैदा करनेवाले की स्पष्ट में तुम्हारे लिए अच्छा है, फिर उसने तुम्हारी तौबा क़बूल कर ली। निस्संदेह वह बड़ी तौबा क़बूल करनेवाला, अत्यन्त दयावान है।"
\end{hindi}}
\flushright{\begin{Arabic}
\quranayah[2][55]
\end{Arabic}}
\flushleft{\begin{hindi}
और याद करो जब तुमने कहा था, "ऐ मूसा, हम तुमपर ईमान नहीं लाएँगे जब तक अल्लाह को खुल्लम-खुल्ला न देख लें।" फिर एक कड़क ने तुम्हें आ दबोचा, तुम देखते रहे
\end{hindi}}
\flushright{\begin{Arabic}
\quranayah[2][56]
\end{Arabic}}
\flushleft{\begin{hindi}
फिर तुम्हारे निर्जीव हो जाने के पश्चात हमने तुम्हें जिला उठाया, ताकि तुम कृतज्ञता दिखलाओ
\end{hindi}}
\flushright{\begin{Arabic}
\quranayah[2][57]
\end{Arabic}}
\flushleft{\begin{hindi}
और हमने तुमपर बादलों की छाया की और तुमपर 'मन्न' और 'सलबा' उतारा - "खाओ, जो अच्छी पाक चीजें हमने तुम्हें प्रदान की है।" उन्होंने हमारा तो कुछ भी नहीं बिगाड़ा, बल्कि वे अपने ही ऊपर अत्याचार करते रहे
\end{hindi}}
\flushright{\begin{Arabic}
\quranayah[2][58]
\end{Arabic}}
\flushleft{\begin{hindi}
और जब हमने कहा था, "इस बस्ती में प्रवेश करो फिर उसमें से जहाँ से चाहो जी भर खाओ, और बस्ती के द्वार में सजदागुज़ार बनकर प्रवेश करो और कहो, "छूट हैं।" हम तुम्हारी खताओं को क्षमा कर देंगे और अच्छे से अच्छा काम करनेवालों पर हम और अधिक अनुग्रह करेंगे।"
\end{hindi}}
\flushright{\begin{Arabic}
\quranayah[2][59]
\end{Arabic}}
\flushleft{\begin{hindi}
फिर जो बात उनसे कहीं गई थी ज़ालिमों ने उसे दूसरी बात से बदल दिया। अन्ततः ज़ालिमों पर हमने, जो अवज्ञा वे कर रहे थे उसके कारण, आकाश से यातना उतारी
\end{hindi}}
\flushright{\begin{Arabic}
\quranayah[2][60]
\end{Arabic}}
\flushleft{\begin{hindi}
और याद करो जब मूसा ने अपनी क़ौम के लिए पानी की प्रार्थना को तो हमने कहा, "चट्टान पर अपनी लाठी मारो," तो उससे बारह स्रोत फूट निकले और हर गिरोह ने अपना-अपना घाट जान लिया - "खाओ और पियो अल्लाह का दिया और धरती में बिगाड़ फैलाते न फिरो।"
\end{hindi}}
\flushright{\begin{Arabic}
\quranayah[2][61]
\end{Arabic}}
\flushleft{\begin{hindi}
और याद करो जब तुमने कहा था, "ऐ मूसा, हम एक ही प्रकार के खाने पर कदापि संतोष नहीं कर सकते, अतः हमारे लिए अपने रब से प्रार्थना करो कि हमारे वास्ते धरती की उपज से साग-पात और ककड़ियाँ और लहसुन और मसूर और प्याज़ निकाले।" और मूसा ने कहा, "क्या तुम जो घटिया चीज़ है उसको उससे बदलकर लेना चाहते हो जो उत्तम है? किसी नगर में उतरो, फिर जो कुछ तुमने माँगा हैं, तुम्हें मिल जाएगा" - और उनपर अपमान और हीन दशा थोप दी गई, और अल्लाह के प्रकोप के भागी हुए। यह इसलिए कि वे अल्लाह की आयतों का इनकार करते रहे और नबियों की अकारण हत्या करते थे। यह इसलिए कि उन्होंने अवज्ञा की और वे सीमा का उल्लंघन करते रहे
\end{hindi}}
\flushright{\begin{Arabic}
\quranayah[2][62]
\end{Arabic}}
\flushleft{\begin{hindi}
निस्संदेह, ईमानवाले और जो यहूदी हुए और ईसाई और साबिई, जो भी अल्लाह और अन्तिम दिन पर ईमान लाया और अच्छा कर्म किया तो ऐसे लोगों का उनके अपने रब के पास (अच्छा) बदला है, उनको न तो कोई भय होगा और न वे शोकाकुल होंगे -
\end{hindi}}
\flushright{\begin{Arabic}
\quranayah[2][63]
\end{Arabic}}
\flushleft{\begin{hindi}
और याद करो जब हमने इस हाल में कि तूर पर्वत को तुम्हारे ऊपर ऊँचा कर रखा था, तुमसे दृढ़ वचन लिया था, "जो चीज़ हमने तुम्हें दी हैं उसे मजबूती के साथ पकड़ो और जो कुछ उसमें हैं उसे याद रखो ताकि तुम बच सको।"
\end{hindi}}
\flushright{\begin{Arabic}
\quranayah[2][64]
\end{Arabic}}
\flushleft{\begin{hindi}
फिर इसके पश्चात भी तुम फिर गए, तो यदि अल्लाह की कृपा और उसकी दयालुता तुम पर न होती, तो तुम घाटे में पड़ गए होते
\end{hindi}}
\flushright{\begin{Arabic}
\quranayah[2][65]
\end{Arabic}}
\flushleft{\begin{hindi}
और तुम उन लोगों के विषय में तो जानते ही हो जिन्होंने तुममें से 'सब्त' के दिन के मामले में मर्यादा का उल्लंघन किया था, तो हमने उनसे कह दिया, "बन्दर हो जाओ, धिक्कारे और फिटकारे हुए!"
\end{hindi}}
\flushright{\begin{Arabic}
\quranayah[2][66]
\end{Arabic}}
\flushleft{\begin{hindi}
फिर हमने इसे सामनेवालों और बाद के लोगों के लिए शिक्षा-सामग्री और डर रखनेवालों के लिए नसीहत बनाकर छोड़ा
\end{hindi}}
\flushright{\begin{Arabic}
\quranayah[2][67]
\end{Arabic}}
\flushleft{\begin{hindi}
और याद करो जब मूसा ने अपनी क़ौम से कहा, "निश्चय ही अल्लाह तुम्हें आदेश देता है कि एक गाय जब्ह करो।" कहने लगे, "क्या तुम हमसे परिहास करते हो?" उसने कहा, "मैं इससे अल्लाह की पनाह माँगता हूँ कि जाहिल बनूँ।"
\end{hindi}}
\flushright{\begin{Arabic}
\quranayah[2][68]
\end{Arabic}}
\flushleft{\begin{hindi}
बोले, "हमारे लिए अपने रब से निवेदन करो कि वह हम पर स्पष्टा कर दे कि वह गाय कौन-सी है?" उसने कहा, "वह कहता है कि वह ऐसी गाय है जो न बूढ़ी है, न बछिया, इनके बीच की रास है; तो जो तुम्हें हुक्म दिया जा रहा है, करो।"
\end{hindi}}
\flushright{\begin{Arabic}
\quranayah[2][69]
\end{Arabic}}
\flushleft{\begin{hindi}
कहने लगे, "हमारे लिए अपने रब से निवेदन करो कि वह हमें बता दे कि उसका रंग कैसा है?" कहा, "वह कहता है कि वह गाय सुनहरी है, गहरे चटकीले रंग की कि देखनेवालों को प्रसन्न कर देती है।"
\end{hindi}}
\flushright{\begin{Arabic}
\quranayah[2][70]
\end{Arabic}}
\flushleft{\begin{hindi}
बोले, "हमारे लिए अपने रब से निवेदन करो कि वह हमें बता दे कि वह कौन-सी है, गायों का निर्धारण हमारे लिए संदिग्ध हो रहा है। यदि अल्लाह ने चाहा तो हम अवश्य। पता लगा लेंगे।"
\end{hindi}}
\flushright{\begin{Arabic}
\quranayah[2][71]
\end{Arabic}}
\flushleft{\begin{hindi}
उसने कहा, " वह कहता हैं कि वह ऐसा गाय है जो सधाई हुई नहीं है कि भूमि जोतती हो, और न वह खेत को पानी देती है, ठीक-ठाक है, उसमें किसी दूसरे रंग की मिलावट नहीं है।" बोले, "अब तुमने ठीक बात बताई है।" फिर उन्होंने उसे ज़ब्ह किया, जबकि वे करना नहीं चाहते थे
\end{hindi}}
\flushright{\begin{Arabic}
\quranayah[2][72]
\end{Arabic}}
\flushleft{\begin{hindi}
और याद करो जब तुमने एक व्यक्ति की हत्या कर दी, फिर उस सिलसिले में तुमने टाल-मटोल से काम लिया - जबकि जिसको तुम छिपा रहे थे, अल्लाह उसे खोल देनेवाला था
\end{hindi}}
\flushright{\begin{Arabic}
\quranayah[2][73]
\end{Arabic}}
\flushleft{\begin{hindi}
तो हमने कहा, "उसे उसके एक हिस्से से मारो।" इस प्रकार अल्लाह मुर्दों को जीवित करता है और तुम्हें अपनी निशानियाँ दिखाता है, ताकि तुम समझो
\end{hindi}}
\flushright{\begin{Arabic}
\quranayah[2][74]
\end{Arabic}}
\flushleft{\begin{hindi}
फिर इसके पश्चात भी तुम्हारे दिल कठोर हो गए, तो वे पत्थरों की तरह हो गए बल्कि उनसे भी अधिक कठोर; क्योंकि कुछ पत्थर ऐसे भी होते है जिनसे नहरें फूट निकलती है, और कुछ ऐसे भी होते है कि फट जाते है तो उनमें से पानी निकलने लगता है, और उनमें से कुछ ऐसे भी होते है जो अल्लाह के भय से गिर जाते है। और अल्लाह, जो कुछ तुम कर रहे हो, उससे बेखबर नहीं है
\end{hindi}}
\flushright{\begin{Arabic}
\quranayah[2][75]
\end{Arabic}}
\flushleft{\begin{hindi}
तो क्या तुम इस लालच में हो कि वे तुम्हारी बात मान लेंगे, जबकि उनमें से कुछ लोग अल्लाह का कलाम सुनते रहे हैं, फिर उसे भली-भाँति समझ लेने के पश्चात जान-बूझकर उसमें परिवर्तन करते रहे?
\end{hindi}}
\flushright{\begin{Arabic}
\quranayah[2][76]
\end{Arabic}}
\flushleft{\begin{hindi}
और जब वे ईमान लानेवाले से मिलते है तो कहते हैं, "हम भी ईमान रखते हैं", और जब आपस में एक-दूसरे से एकान्त में मिलते है तो कहते है, "क्या तुम उन्हें वे बातें, जो अल्लाह ने तुम पर खोली, बता देते हो कि वे उनके द्वारा तुम्हारे रब के यहाँ हुज्जत में तुम्हारा मुक़ाबिला करें? तो क्या तुम समझते नहीं!"
\end{hindi}}
\flushright{\begin{Arabic}
\quranayah[2][77]
\end{Arabic}}
\flushleft{\begin{hindi}
क्या वे जानते नहीं कि अल्लाह वह सब कुछ जानता है, जो कुछ वे छिपाते और जो कुछ ज़ाहिर करते हैं?
\end{hindi}}
\flushright{\begin{Arabic}
\quranayah[2][78]
\end{Arabic}}
\flushleft{\begin{hindi}
और उनमें सामान्य बेपढ़े भी हैं जिन्हें किताब का ज्ञान नहीं है, बस कुछ कामनाओं एवं आशाओं को धर्म जानते हैं, और वे तो बस अटकल से काम लेते हैं
\end{hindi}}
\flushright{\begin{Arabic}
\quranayah[2][79]
\end{Arabic}}
\flushleft{\begin{hindi}
तो विनाश और तबाही है उन लोगों के लिए जो अपने हाथों से किताब लिखते हैं फिर कहते हैं, "यह अल्लाह की ओर से है", ताकि उसके द्वारा थोड़ा मूल्य प्राप्त कर लें। तो तबाही है उनके हाथों ने लिखा और तबाही है उनके लिए उसके कारण जो वे कमा रहे हैं
\end{hindi}}
\flushright{\begin{Arabic}
\quranayah[2][80]
\end{Arabic}}
\flushleft{\begin{hindi}
वे कहते है, "जहन्नम की आग हमें नहीं छू सकती, हाँ, कुछ गिने-चुने दिनों की बात और है।" कहो, "क्या तुमने अल्लाह से कोई वचन ले रखा है? फिर तो अल्लाह कदापि अपने वचन के विरुद्ध नहीं जा सकता? या तुम अल्लाह के ज़िम्मे डालकर ऐसी बात कहते हो जिसका तुम्हें ज्ञान नहीं?
\end{hindi}}
\flushright{\begin{Arabic}
\quranayah[2][81]
\end{Arabic}}
\flushleft{\begin{hindi}
क्यों नहीं; जिसने भी कोई बदी कमाई और उसकी खताकारी ने उसे अपने घरे में ले लिया, तो ऐसे ही लोग आग (जहन्नम) में पड़नेवाले है; वे उसी में सदैव रहेंगे
\end{hindi}}
\flushright{\begin{Arabic}
\quranayah[2][82]
\end{Arabic}}
\flushleft{\begin{hindi}
रहे वे लोग जो ईमान लाए और उन्होंने अच्छे कर्म किए, वही जन्नतवाले हैं, वे सदैव उसी में रहेंगे।"
\end{hindi}}
\flushright{\begin{Arabic}
\quranayah[2][83]
\end{Arabic}}
\flushleft{\begin{hindi}
और याद करो जब इसराईल की सन्तान से हमने वचन लिया, "अल्लाह के अतिरिक्त किसी की बन्दगी न करोगे; और माँ-बाप के साथ और नातेदारों के साथ और अनाथों और मुहताजों के साथ अच्छा व्यवहार करोगे; और यह कि लोगों से भली बात कहो और नमाज़ क़ायम करो और ज़कात दो।" तो तुम फिर गए, बस तुममें से बचे थोड़े ही, और तुम उपेक्षा की नीति ही अपनाए रहे
\end{hindi}}
\flushright{\begin{Arabic}
\quranayah[2][84]
\end{Arabic}}
\flushleft{\begin{hindi}
और याद करो जब तुमसे वचन लिया, "अपने ख़ून न बहाओगे और न अपने लोगों को अपनी बस्तियों से निकालोगे।" फिर तुमने इक़रार किया और तुम स्वयं इसके गवाह हो
\end{hindi}}
\flushright{\begin{Arabic}
\quranayah[2][85]
\end{Arabic}}
\flushleft{\begin{hindi}
फिर तुम वही हो कि अपने लोगों की हत्या करते हो और अपने ही एक गिरोह के लोगों को उनकी बस्तियों से निकालते हो; तुम गुनाह और ज़्यादती के साथ उनके विरुद्ध एक-दूसरे के पृष्ठपोषक बन जाते हो; और यदि वे बन्दी बनकर तुम्हारे पास आते है, तो उनकी रिहाई के लिए फिद्ए (अर्थदंड) का लेन-देन करते हो जबकि उनको उनके घरों से निकालना ही तुम पर हराम था, तो क्या तुम किताब के एक हिस्से को मानते हो और एक को नहीं मानते? फिर तुममें जो ऐसा करें उसका बदला इसके सिवा और क्या हो सकता है कि सांसारिक जीवन में अपमान हो? और क़यामत के दिन ऐसे लोगों को कठोर से कठोर यातना की ओर फेर दिया जाएगा। अल्लाह उससे बेखबर नहीं है जो कुछ तुम कर रहे हो
\end{hindi}}
\flushright{\begin{Arabic}
\quranayah[2][86]
\end{Arabic}}
\flushleft{\begin{hindi}
यही वे लोग है जो आख़िरात के बदले सांसारिक जीवन के ख़रीदार हुए, तो न उनकी यातना हल्की की जाएगी और न उन्हें कोई सहायता पहुँच सकेगी
\end{hindi}}
\flushright{\begin{Arabic}
\quranayah[2][87]
\end{Arabic}}
\flushleft{\begin{hindi}
और हमने मूसा को किताब दी थी, और उसके पश्चात आगे-पीछे निरन्तर रसूल भेजते रहे; और मरयम के बेटे ईसा को खुली-खुली निशानियाँ प्रदान की और पवित्र-आत्मा के द्वारा उसे शक्ति प्रदान की; तो यही तो हुआ कि जब भी कोई रसूल तुम्हारे पास वह कुछ लेकर आया जो तुम्हारे जी को पसन्द न था, तो तुम अकड़ बैठे, तो एक गिरोह को तो तुमने झुठलाया और एक गिरोह को क़त्ल करते हो?
\end{hindi}}
\flushright{\begin{Arabic}
\quranayah[2][88]
\end{Arabic}}
\flushleft{\begin{hindi}
वे कहते हैं, "हमारे दिलों पर तो प्राकृतिक आवरण चढ़े है" नहीं, बल्कि उनके इनकार के कारण अल्लाह ने उनपर लानत की है; अतः वे ईमान थोड़े ही लाएँगे
\end{hindi}}
\flushright{\begin{Arabic}
\quranayah[2][89]
\end{Arabic}}
\flushleft{\begin{hindi}
और जब उनके पास एक किताब अल्लाह की ओर से आई है जो उसकी पुष्टि करती है जो उनके पास मौजूद है - और इससे पहले तो वे न माननेवाले लोगों पर विजय पाने के इच्छुक रहे है - फिर जब वह चीज़ उनके पास आ गई जिसे वे पहचान भी गए हैं, तो उसका इनकार कर बैठे; तो अल्लाह की फिटकार इनकार करने वालों पर!
\end{hindi}}
\flushright{\begin{Arabic}
\quranayah[2][90]
\end{Arabic}}
\flushleft{\begin{hindi}
क्या ही बुरी चीज़ है जिसके बदले उन्होंने अपनी जानों का सौदा किया, अर्थात जो कुछ अल्लाह ने उतारा है उसे सरकशी और इस अप्रियता के कारण नहीं मानते कि अल्लाह अपना फ़ज़्ल (कृपा) अपने बन्दों में से जिसपर चाहता है क्यों उतारता है, अतः वे प्रकोप पर प्रकोप के अधिकारी हो गए है। और ऐसे इनकार करनेवालों के लिए अपमानजनक यातना है
\end{hindi}}
\flushright{\begin{Arabic}
\quranayah[2][91]
\end{Arabic}}
\flushleft{\begin{hindi}
जब उनसे कहा जाता है, "अल्लाह ने जो कुछ उतारा है उस पर ईमान लाओ", तो कहते है, "हम तो उसपर ईमान रखते है जो हम पर उतरा है," और उसे मानने से इनकार करते हैं जो उसके पीछे है, जबकि वही सत्य है, उसकी पुष्टि करता है जो उसके पास है। कहो, "अच्छा तो इससे पहले अल्लाह के पैग़म्बरों की हत्या क्यों करते रहे हो, यदि तुम ईमानवाले हो?"
\end{hindi}}
\flushright{\begin{Arabic}
\quranayah[2][92]
\end{Arabic}}
\flushleft{\begin{hindi}
तुम्हारे पास मूसा खुली-खुली निशानियाँ लेकर आया, फिर भी उसके बाद तुम ज़ालिम बनकर बछड़े को देवता बना बैठे
\end{hindi}}
\flushright{\begin{Arabic}
\quranayah[2][93]
\end{Arabic}}
\flushleft{\begin{hindi}
कहो, "यदि अल्लाह के निकट आख़िरत का घर सारे इनसानों को छोड़कर केवल तुम्हारे ही लिए है, फिर तो मृत्यु की कामना करो, यदि तुम सच्चे हो।"
\end{hindi}}
\flushright{\begin{Arabic}
\quranayah[2][94]
\end{Arabic}}
\flushleft{\begin{hindi}
अपने हाथों इन्होंने जो कुछ आगे भेजा है उसके कारण वे कदापि उसकी कामना न करेंगे; अल्लाह तो ज़ालिमों को भली-भाँति जानता है
\end{hindi}}
\flushright{\begin{Arabic}
\quranayah[2][95]
\end{Arabic}}
\flushleft{\begin{hindi}
अपने हाथों इन्होंने जो कुछ आगे भेजा है उसके कारण वे कदापि उसकी कामना न करेंगे; अल्लाह तो जालिमों को भली-भाँति जानता है
\end{hindi}}
\flushright{\begin{Arabic}
\quranayah[2][96]
\end{Arabic}}
\flushleft{\begin{hindi}
तुम उन्हें सब लोगों से बढ़कर जीवन का लोभी पाओगे, यहाँ तक कि वे इस सम्बन्ध में शिर्क करनेवालो से भी बढ़े हुए है। उनका तो प्रत्येक व्यक्ति यह इच्छा रखता है कि क्या ही अच्छा होता कि उस हज़ार वर्ष की आयु मिले, जबकि यदि उसे यह आयु प्राप्त भी जाए, तो भी वह अपने आपको यातना से नहीं बचा सकता। अल्लाह देख रहा है, जो कुछ वे कर रहे है
\end{hindi}}
\flushright{\begin{Arabic}
\quranayah[2][97]
\end{Arabic}}
\flushleft{\begin{hindi}
कहो, "जो कोई जिबरील का शत्रु हो, (तो वह अल्लाह का शत्रु है) क्योंकि उसने तो उसे अल्लाह ही के हुक्म से तम्हारे दिल पर उतारा है, जो उन (भविष्यवाणियों) के सर्वथा अनुकूल है जो उससे पहले से मौजूद हैं; और ईमानवालों के लिए मार्गदर्शन और शुभ-सूचना है
\end{hindi}}
\flushright{\begin{Arabic}
\quranayah[2][98]
\end{Arabic}}
\flushleft{\begin{hindi}
जो कोई अल्लाह और उसके फ़रिश्तों और उसके रसूलों और जिबरील और मीकाईल का शत्रु हो, तो ऐसे इनकार करनेवालों का अल्लाह शत्रु है।"
\end{hindi}}
\flushright{\begin{Arabic}
\quranayah[2][99]
\end{Arabic}}
\flushleft{\begin{hindi}
और हमने तुम्हारी ओर खुली-खुली आयतें उतारी है और उनका इनकार तो बस वही लोग करते हैस जो उल्लंघनकारी हैं
\end{hindi}}
\flushright{\begin{Arabic}
\quranayah[2][100]
\end{Arabic}}
\flushleft{\begin{hindi}
क्या यह एक निश्चित नीति है कि जब कि उन्होंने कोई वचन दिया तो उनके एक गिरोह ने उसे उठा फेंका? बल्कि उनमें अधिकतर ईमान ही नहीं रखते
\end{hindi}}
\flushright{\begin{Arabic}
\quranayah[2][101]
\end{Arabic}}
\flushleft{\begin{hindi}
और जब उनके पास अल्लाह की ओर से एक रसूल आया, जिससे उस (भविष्यवाणी) की पुष्टि हो रही है जो उनके पास थी, तो उनके एक गिरोह ने, जिन्हें किताब मिली थी, अल्लाह की किताब को अपने पीठ पीछे डाल दिया, मानो वे कुछ जानते ही नही
\end{hindi}}
\flushright{\begin{Arabic}
\quranayah[2][102]
\end{Arabic}}
\flushleft{\begin{hindi}
और जो वे उस चीज़ के पीछे पड़ गए जिसे शैतान सुलैमान की बादशाही पर थोपकर पढ़ते थे - हालाँकि सुलैमान ने कोई कुफ़्र नहीं किया था, बल्कि कुफ़्र तो शैतानों ने किया था; वे लोगों को जादू सिखाते थे - और उस चीज़ में पड़ गए जो बाबिल में दोनों फ़रिश्तों हारूत और मारूत पर उतारी गई थी। और वे किसी को भी सिखाते न थे जब तक कि कह न देते, "हम तो बस एक परीक्षा है; तो तुम कुफ़्र में न पड़ना।" तो लोग उन दोनों से वह कुछ सीखते है, जिसके द्वारा पति और पत्नी में अलगाव पैदा कर दे - यद्यपि वे उससे किसी को भी हानि नहीं पहुँचा सकते थे। हाँ, यह और बात है कि अल्लाह के हुक्म से किसी को हानि पहुँचनेवाली ही हो - और वह कुछ सीखते है जो उन्हें हानि ही पहुँचाए और उन्हें कोई लाभ न पहुँचाए। और उन्हें भली-भाँति मालूम है कि जो उसका ग्राहक बना, उसका आखिरत में कोई हिस्सा नहीं। कितनी बुरी चीज़ के बदले उन्होंने प्राणों का सौदा किया, यदि वे जानते (तो ठीक मार्ग अपनाते)
\end{hindi}}
\flushright{\begin{Arabic}
\quranayah[2][103]
\end{Arabic}}
\flushleft{\begin{hindi}
और यदि वे ईमान लाते और डर रखते, तो अल्लाह के यहाँ से मिलनेवाला बदला कहीं अच्छा था, यदि वे जानते (तो इसे समझ सकते)
\end{hindi}}
\flushright{\begin{Arabic}
\quranayah[2][104]
\end{Arabic}}
\flushleft{\begin{hindi}
ऐ ईमान लानेवालो! 'राइना' न कहा करो, बल्कि 'उनज़ुरना' कहा और सुना करो। और इनकार करनेवालों के लिए दुखद यातना है
\end{hindi}}
\flushright{\begin{Arabic}
\quranayah[2][105]
\end{Arabic}}
\flushleft{\begin{hindi}
इनकार करनेवाले नहीं चाहते, न किताबवाले और न मुशरिक (बहुदेववादी) कि तुम्हारे रब की ओर से तुमपर कोई भलाई उतरे, हालाँकि अल्लाह जिसे चाहे अपनी दयालुता के लिए ख़ास कर ले; अल्लाह बड़ा अनुग्रह करनेवाला है
\end{hindi}}
\flushright{\begin{Arabic}
\quranayah[2][106]
\end{Arabic}}
\flushleft{\begin{hindi}
हम जिस आयत (और निशान) को भी मिटा दें या उसे भुला देते है, तो उससे बेहतर लाते है या उस जैसा दूसरा ही। क्या तुम नहीं जानते हो कि अल्लाह को हर चीज़ का सामर्थ्य प्राप्त है?
\end{hindi}}
\flushright{\begin{Arabic}
\quranayah[2][107]
\end{Arabic}}
\flushleft{\begin{hindi}
क्या तुम नहीं जानते कि आकाशों और धरती का राज्य अल्लाह ही का है और अल्लाह से हटकर न तुम्हारा कोई मित्र है और न सहायक?
\end{hindi}}
\flushright{\begin{Arabic}
\quranayah[2][108]
\end{Arabic}}
\flushleft{\begin{hindi}
(ऐ ईमानवालों! तुम अपने रसूल के आदर का ध्यान रखो) या तुम चाहते हो कि अपने रसूल से उसी प्रकार से प्रश्न और बात करो, जिस प्रकार इससे पहले मूसा से बात की गई है? हालाँकि जिस व्यक्ति न ईमान के बदले इनकार की नीति अपनाई, तो वह सीधे रास्ते से भटक गया
\end{hindi}}
\flushright{\begin{Arabic}
\quranayah[2][109]
\end{Arabic}}
\flushleft{\begin{hindi}
बहुत-से किताबवाले अपने भीतर की ईर्ष्या से चाहते है कि किसी प्रकार वे तुम्हारे ईमान लाने के बाद फेरकर तुम्हे इनकार कर देनेवाला बना दें, यद्यपि सत्य उनपर प्रकट हो चुका है, तो तुम दरगुज़र (क्षमा) से काम लो और जाने दो यहाँ तक कि अल्लाह अपना फ़ैसला लागू न कर दे। निस्संदेह अल्लाह को हर चीज़ की सामर्थ्य प्राप्त है
\end{hindi}}
\flushright{\begin{Arabic}
\quranayah[2][110]
\end{Arabic}}
\flushleft{\begin{hindi}
और नमाज़ कायम करो और ज़कात दो और तुम स्वयं अपने लिए जो भलाई भी पेश करोगे, उसे अल्लाह के यहाँ मौजूद पाओगे। निस्संदेह जो कुछ तुम करते हो, अल्लाह उसे देख रहा है
\end{hindi}}
\flushright{\begin{Arabic}
\quranayah[2][111]
\end{Arabic}}
\flushleft{\begin{hindi}
और उनका कहना है, "कोई व्यक्ति जन्नत में प्रवेश नहीं करता सिवाय उससे जो यहूदी है या ईसाई है।" ये उनकी अपनी निराधार कामनाएँ है। कहो, "यदि तुम सच्चे हो तो अपने प्रमाण पेश करो।"
\end{hindi}}
\flushright{\begin{Arabic}
\quranayah[2][112]
\end{Arabic}}
\flushleft{\begin{hindi}
क्यों नहीं, जिसने भी अपने-आपको अल्लाह के प्रति समर्पित कर दिया और उसका कर्म भी अच्छे से अच्छा हो तो उसका प्रतिदान उसके रब के पास है और ऐसे लोगो के लिए न तो कोई भय होगा और न वे शोकाकुल होंगे
\end{hindi}}
\flushright{\begin{Arabic}
\quranayah[2][113]
\end{Arabic}}
\flushleft{\begin{hindi}
यहूदियों ने कहा, "ईसाईयों की कोई बुनियाद नहीं।" और ईसाइयों ने कहा, "यहूदियों की कोई बुनियाद नहीं।" हालाँकि वे किताब का पाठ करते है। इसी तरह की बात उन्होंने भी कही है जो ज्ञान से वंचित है। तो अल्लाह क़यामत के दिन उनके बीच उस चीज़ के विषय में निर्णय कर देगा, जिसके विषय में वे विभेद कर रहे है
\end{hindi}}
\flushright{\begin{Arabic}
\quranayah[2][114]
\end{Arabic}}
\flushleft{\begin{hindi}
और उससे बढ़कर अत्याचारी और कौन होगा जिसने अल्लाह की मस्जिदों को उसके नाम के स्मरण से वंचित रखा और उन्हें उजाडने पर उतारू रहा? ऐसे लोगों को तो बस डरते हुए ही उसमें प्रवेश करना चाहिए था। उनके लिए संसार में रुसवाई (अपमान) है और उनके लिए आख़िरत में बड़ी यातना नियत है
\end{hindi}}
\flushright{\begin{Arabic}
\quranayah[2][115]
\end{Arabic}}
\flushleft{\begin{hindi}
पूरब और पश्चिम अल्लाह ही के है, अतः जिस ओर भी तुम रुख करो उसी ओर अल्लाह का रुख़ है। निस्संदेह अल्लाह बड़ा समाईवाला (सर्वव्यापी) सर्वज्ञ है
\end{hindi}}
\flushright{\begin{Arabic}
\quranayah[2][116]
\end{Arabic}}
\flushleft{\begin{hindi}
कहते है, अल्लाह औलाद रखता है - महिमावाला है वह! (पूरब और पश्चिम हीं नहीं, बल्कि) आकाशों और धरती में जो कुछ भी है, उसी का है। सभी उसके आज्ञाकारी है
\end{hindi}}
\flushright{\begin{Arabic}
\quranayah[2][117]
\end{Arabic}}
\flushleft{\begin{hindi}
वह आकाशों और धरती का प्रथमतः पैदा करनेवाला है। वह तो जब किसी काम का निर्णय करता है, तो उसके लिए बस कह देता है कि "हो जा" और वह हो जाता है
\end{hindi}}
\flushright{\begin{Arabic}
\quranayah[2][118]
\end{Arabic}}
\flushleft{\begin{hindi}
जिन्हें ज्ञान नहीं हैं, वे कहते है, "अल्लाह हमसे बात क्यों नहीं करता? या कोई निशानी हमारे पास आ जाए।" इसी प्रकार इनसे पहले के लोग भी कह चुके है। इन सबके दिल एक जैसे है। हम खोल-खोलकर निशानियाँ उन लोगों के लिए बयान कर चुके है जो विश्वास करें
\end{hindi}}
\flushright{\begin{Arabic}
\quranayah[2][119]
\end{Arabic}}
\flushleft{\begin{hindi}
निश्चित रूप से हमने तुम्हें हक़ के साथ शुभ-सूचना देनेवाला और डरानेवाला बनाकर भेजा। भड़कती आग में पड़नेवालों के विषय में तुमसे कुछ न पूछा जाएगा
\end{hindi}}
\flushright{\begin{Arabic}
\quranayah[2][120]
\end{Arabic}}
\flushleft{\begin{hindi}
न यहूदी तुमसे कभी राज़ी होनेवाले है और न ईसाई जब तक कि तुम अनके पंथ पर न चलने लग जाओ। कह दो, "अल्लाह का मार्गदर्शन ही वास्तविक मार्गदर्शन है।" और यदि उस ज्ञान के पश्चात जो तुम्हारे पास आ चुका है, तुमने उनकी इच्छाओं का अनुसरण किया, तो अल्लाह से बचानेवाला न तो तुम्हारा कोई मित्र होगा और न सहायक
\end{hindi}}
\flushright{\begin{Arabic}
\quranayah[2][121]
\end{Arabic}}
\flushleft{\begin{hindi}
जिन लोगों को हमने किताब दी है उनमें वे लोग जो उसे उस तरह पढ़ते है जैसा कि उसके पढ़ने का हक़ है, वही उसपर ईमान ला रहे है, और जो उसका इनकार करेंगे, वही घाटे में रहनेवाले है
\end{hindi}}
\flushright{\begin{Arabic}
\quranayah[2][122]
\end{Arabic}}
\flushleft{\begin{hindi}
ऐ इसराईल की सन्तान! मेरी उस कृपा को याद करो जो मैंने तुमपर की थी और यह कि मैंने तुम्हें संसारवालों पर श्रेष्ठता प्रदान की
\end{hindi}}
\flushright{\begin{Arabic}
\quranayah[2][123]
\end{Arabic}}
\flushleft{\begin{hindi}
और उस दिन से डरो, जब कोई न किसी के काम आएगा, न किसी की ओर से अर्थदंड स्वीकार किया जाएगा, और न कोई सिफ़ारिश ही उसे लाभ पहुँचा सकेगी, और न उनको कोई सहायता ही पहुँच सकेगी
\end{hindi}}
\flushright{\begin{Arabic}
\quranayah[2][124]
\end{Arabic}}
\flushleft{\begin{hindi}
और याद करो जब इबराहीम की उसके रब से कुछ बातों में परीक्षा ली तो उसने उसको पूरा कर दिखाया। उसने कहा, "मैं तुझे सारे इनसानों का पेशवा बनानेवाला हूँ।" उसने निवेदन किया, " और मेरी सन्तान में भी।" उसने कहा, "ज़ालिम मेरे इस वादे के अन्तर्गत नहीं आ सकते।"
\end{hindi}}
\flushright{\begin{Arabic}
\quranayah[2][125]
\end{Arabic}}
\flushleft{\begin{hindi}
और याद करो जब हमने इस घर (काबा) को लोगों को लिए केन्द्र और शान्तिस्थल बनाया - और, "इबराहीम के स्थल में से किसी जगह को नमाज़ की जगह बना लो!" - और इबराहीम और इसमाईल को ज़िम्मेदार बनाया। "तुम मेरे इस घर को तवाफ़ करनेवालों और एतिकाफ़ करनेवालों के लिए और रुकू और सजदा करनेवालों के लिए पाक-साफ़ रखो।"
\end{hindi}}
\flushright{\begin{Arabic}
\quranayah[2][126]
\end{Arabic}}
\flushleft{\begin{hindi}
और याद करो जब इबराहीम ने कहा, "ऐ मेरे रब! इसे शान्तिमय भू-भाग बना दे और इसके उन निवासियों को फलों की रोज़ी दे जो उनमें से अल्लाह और अन्तिम दिन पर ईमान लाएँ।" कहा, "और जो इनकार करेगा थोड़ा फ़ायदा तो उसे भी दूँगा, फिर उसे घसीटकर आग की यातना की ओर पहुँचा दूँगा और वह बहुत-ही बुरा ठिकाना है!"
\end{hindi}}
\flushright{\begin{Arabic}
\quranayah[2][127]
\end{Arabic}}
\flushleft{\begin{hindi}
और याद करो जब इबराहीम और इसमाईल इस घर की बुनियादें उठा रहे थे, (तो उन्होंने प्रार्थना की), "ऐ हमारे रब! हमारी ओर से इसे स्वीकार कर ले, निस्संदेह तू सुनता-जानता है
\end{hindi}}
\flushright{\begin{Arabic}
\quranayah[2][128]
\end{Arabic}}
\flushleft{\begin{hindi}
ऐ हमारे रब! हम दोनों को अपना आज्ञाकारी बना और हमारी संतान में से अपना एक आज्ञाकारी समुदाय बना; और हमें हमारे इबादत के तरीक़े बता और हमारी तौबा क़बूल कर। निस्संदेह तू तौबा क़बूल करनेवाला, अत्यन्त दयावान है
\end{hindi}}
\flushright{\begin{Arabic}
\quranayah[2][129]
\end{Arabic}}
\flushleft{\begin{hindi}
ऐ हमारे रब! उनमें उन्हीं में से एक ऐसा रसूल उठा जो उन्हें तेरी आयतें सुनाए और उनको किताब और तत्वदर्शिता की शिक्षा दे और उन (की आत्मा) को विकसित करे। निस्संदेह तू प्रभुत्वशाली, तत्वदर्शी है
\end{hindi}}
\flushright{\begin{Arabic}
\quranayah[2][130]
\end{Arabic}}
\flushleft{\begin{hindi}
कौन है जो इबराहीम के पंथ से मुँह मोड़े सिवाय उसके जिसने स्वयं को पतित कर लिया? और उसे तो हमने दुनिया में चुन लिया था और निस्संदेह आख़िरत में उसकी गणना योग्य लोगों में होगी
\end{hindi}}
\flushright{\begin{Arabic}
\quranayah[2][131]
\end{Arabic}}
\flushleft{\begin{hindi}
क्योंकि जब उससे रब ने कहा, "मुस्लिम (आज्ञाकारी) हो जा।" उसने कहा, "मैं सारे संसार के रब का मुस्लिम हो गया।"
\end{hindi}}
\flushright{\begin{Arabic}
\quranayah[2][132]
\end{Arabic}}
\flushleft{\begin{hindi}
और इसी की वसीयत इबराहीम ने अपने बेटों को की और याक़ूब ने भी (अपनी सन्तानों को की) कि, "ऐ मेरे बेटों! अल्लाह ने तुम्हारे लिए यही दीन (धर्म) चुना है, तो इस्लाम (ईश-आज्ञापालन) को अतिरिक्त किसी और दशा में तुम्हारी मृत्यु न हो।"
\end{hindi}}
\flushright{\begin{Arabic}
\quranayah[2][133]
\end{Arabic}}
\flushleft{\begin{hindi}
(क्या तुम इबराहीम के वसीयत करते समय मौजूद थे? या तुम मौजूद थे जब याक़ूब की मृत्यु का समय आया? जब उसने बेटों से कहा, "तुम मेरे पश्चात किसकी इबादत करोगे?" उन्होंने कहा, "हम आपके इष्ट-पूज्य और आपके पूर्वज इबराहीम और इसमाईल और इसहाक़ के इष्ट-पूज्य की बन्दगी करेंगे - जो अकेला इष्ट-पूज्य है, और हम उसी के आज्ञाकारी (मुस्लिम) हैं।"
\end{hindi}}
\flushright{\begin{Arabic}
\quranayah[2][134]
\end{Arabic}}
\flushleft{\begin{hindi}
वह एक गिरोह था जो गुज़र चुका, जो कुछ उसने कमाया वह उसका है, और जो कुछ तुमने कमाया वह तुम्हारा है। और जो कुछ वे करते रहे उसके विषय में तुमसे कोई पूछताछ न की जाएगी
\end{hindi}}
\flushright{\begin{Arabic}
\quranayah[2][135]
\end{Arabic}}
\flushleft{\begin{hindi}
वे कहते हैं, "यहूदी या ईसाई हो जाओ तो मार्ग पर लोगे।" कहो, "नहीं, बल्कि इबराहीम का पंथ अपनाओ जो एक (अल्लाह) का हो गया था, और वह बहुदेववादियों में से न था।"
\end{hindi}}
\flushright{\begin{Arabic}
\quranayah[2][136]
\end{Arabic}}
\flushleft{\begin{hindi}
कहो, "हम ईमान लाए अल्लाह पर और उस चीज़ पर जो हमारी ओर से उतरी और जो इबराहीम और इसमाईल और इसहाक़ और याक़ूब और उसकी संतान की ओर उतरी, और जो मूसा और ईसा को मिली, और जो सभी नबियों को उनके रब की ओर से प्रदान की गई। हम उनमें से किसी के बीच अन्तर नहीं करते और हम केवल उसी के आज्ञाकारी हैं।"
\end{hindi}}
\flushright{\begin{Arabic}
\quranayah[2][137]
\end{Arabic}}
\flushleft{\begin{hindi}
फिर यदि वे उसी तरह ईमान लाएँ जिस तरह तुम ईमान लाए हो, तो उन्होंने मार्ग पा लिया। और यदि वे मुँह मोड़े, तो फिर वही विरोध में पड़े हुए है। अतः तुम्हारी जगह स्वयं अल्लाह उनसे निबटने के लिए काफ़ी है; वह सब कुछ सुनता, जानता है
\end{hindi}}
\flushright{\begin{Arabic}
\quranayah[2][138]
\end{Arabic}}
\flushleft{\begin{hindi}
(कहो,) "अल्लाह का रंग ग्रहण करो, उसके रंग से अच्छा और किसका रंह हो सकता है? और हम तो उसी की बन्दगी करते हैं।"
\end{hindi}}
\flushright{\begin{Arabic}
\quranayah[2][139]
\end{Arabic}}
\flushleft{\begin{hindi}
कहो, "क्या तुम अल्लाह के विषय में हमसे झगड़ते हो, हालाँकि वही हमारा रब भी है, और तुम्हारा रब भी? और हमारे लिए हमारे कर्म हैं और तुम्हारे लिए तुम्हारे कर्म। और हम तो बस निरे उसी के है।"
\end{hindi}}
\flushright{\begin{Arabic}
\quranayah[2][140]
\end{Arabic}}
\flushleft{\begin{hindi}
या तुम कहते हो कि इबराहीम और इसमाईल और इसहाक़ और याक़ूब और उनकी संतान सब के सब यहूदी या ईसाई थे? कहो, "तुम अधिक जानते हो या अल्लाह? और उससे बढ़कर ज़ालिम कौन होगा, जिसके पास अल्लाह की ओर से आई हुई कोई गवाही हो, और वह उसे छिपाए? और जो कुछ तुम कर रहे हो, अल्लाह उससे बेखबर नहीं है।"
\end{hindi}}
\flushright{\begin{Arabic}
\quranayah[2][141]
\end{Arabic}}
\flushleft{\begin{hindi}
वह एक गिरोह थो जो गुज़र चुका, जो कुछ उसने कमाया वह उसके लिए है और जो कुछ तुमने कमाया वह तुम्हारे लिए है। और तुमसे उसके विषय में न पूछा जाएगा, जो कुछ वे करते रहे है
\end{hindi}}
\flushright{\begin{Arabic}
\quranayah[2][142]
\end{Arabic}}
\flushleft{\begin{hindi}
मूर्ख लोग अब कहेंगे, "उन्हें उनके उस क़िबले (उपासना-दिशा) से, जिस पर वे थे किस ची़ज़ ने फेर दिया?" कहो, "पूरब और पश्चिम अल्लाह ही के है, वह जिसे चाहता है सीधा मार्ग दिखाता है।"
\end{hindi}}
\flushright{\begin{Arabic}
\quranayah[2][143]
\end{Arabic}}
\flushleft{\begin{hindi}
और इसी प्रकार हमने तुम्हें बीच का एक उत्तम समुदाय बनाया है, ताकि तुम सारे मनुष्यों पर गवाह हो, और रसूल तुमपर गवाह हो। और जिस (क़िबले) पर तुम रहे हो उसे तो हमने केवल इसलिए क़िबला बनाया था कि जो लोग पीठ-पीछे फिर जानेवाले है, उनसे हम उनको अलग जान लें जो रसूल का अनुसरण करते है। और यह बात बहुत भारी (अप्रिय) है, किन्तु उन लोगों के लिए नहीं जिन्हें अल्लाह ने मार्ग दिखाया है। और अल्लाह ऐसा नहीं कि वह तुम्हारे ईमान को अकारथ कर दे, अल्लाह तो इनसानों के लिए अत्यन्त करूणामय, दयावान है
\end{hindi}}
\flushright{\begin{Arabic}
\quranayah[2][144]
\end{Arabic}}
\flushleft{\begin{hindi}
हम आकाश में तुम्हारे मुँह की गर्दिश देख रहे है, तो हम अवश्य ही तुम्हें उसी क़िबले का अधिकारी बना देंगे जिसे तुम पसन्द करते हो। अतः मस्जिदे हराम (काबा) की ओर अपना रूख़ करो। और जहाँ कहीं भी हो अपने मुँह उसी की ओर करो - निश्चय ही जिन लोगों को किताब मिली थी, वे भली-भाँति जानते है कि वही उनके रब की ओर से हक़ है, इसके बावजूद जो कुछ वे कर रहे है अल्लाह उससे बेखबर नहीं है
\end{hindi}}
\flushright{\begin{Arabic}
\quranayah[2][145]
\end{Arabic}}
\flushleft{\begin{hindi}
यदि तुम उन लोगों के पास, जिन्हें किताब दी गई थी, कोई भी निशानी ले आओ, फिर भी वे तुम्हारे क़िबले का अनुसरण नहीं करेंगे और तुम भी उसके क़िबले का अनुसरण करने वाले नहीं हो। और वे स्वयं परस्पर एक-दूसरे के क़िबले का अनुसरण करनेवाले नहीं हैं। और यदि तुमने उस ज्ञान के पश्चात, जो तुम्हारे पास आ चुका है, उनकी इच्छाओं का अनुसरण किया, तो निश्चय ही तुम्हारी गणना ज़ालिमों में होगी
\end{hindi}}
\flushright{\begin{Arabic}
\quranayah[2][146]
\end{Arabic}}
\flushleft{\begin{hindi}
जिन लोगों को हमने किताब दी है वे उसे पहचानते है, जैसे अपने बेटों को पहचानते है और उनमें से कुछ सत्य को जान-बूझकर छिपा रहे हैं
\end{hindi}}
\flushright{\begin{Arabic}
\quranayah[2][147]
\end{Arabic}}
\flushleft{\begin{hindi}
सत्य तुम्हारे रब की ओर से है। अतः तुम सन्देह करनेवालों में से कदापि न होगा
\end{hindi}}
\flushright{\begin{Arabic}
\quranayah[2][148]
\end{Arabic}}
\flushleft{\begin{hindi}
प्रत्येक की एक ही दिशा है, वह उसी की ओर मुख किेए हुए है, तो तुम भलाईयों में अग्रसरता दिखाओ। जहाँ कहीं भी तुम होगे अल्लाह तुम सबको एकत्र करेगा। निस्संदेह अल्लाह को हर चीज़ की सामर्थ्य प्राप्त है
\end{hindi}}
\flushright{\begin{Arabic}
\quranayah[2][149]
\end{Arabic}}
\flushleft{\begin{hindi}
और जहाँ से भी तुम निकलों, 'मस्जिदे हराम' (काबा) की ओर अपना मुँह फेर लिया करो। निस्संदेह यही तुम्हारे रब की ओर से हक़ है। जो कुछ तुम करते हो, अल्लाह उससे बेख़बर नहीं है
\end{hindi}}
\flushright{\begin{Arabic}
\quranayah[2][150]
\end{Arabic}}
\flushleft{\begin{hindi}
जहाँ से भी तुम निकलो, 'मस्जिदे हराम' की ओर अपना मुँह फेर लिया करो, और जहाँ कहीं भी तुम हो उसी की ओर मुँह कर लिया करो, ताकि लोगों के पास तुम्हारे ख़िलाफ़ कोई हुज्जत बाक़ी न रहे - सिवाय उन लोगों के जो उनमें ज़ालिम हैं, तुम उनसे न डरो, मुझसे ही डरो - और ताकि मैं तुमपर अपनी नेमत पूरी कर दूँ, और ताकि तुम सीधी राह चलो
\end{hindi}}
\flushright{\begin{Arabic}
\quranayah[2][151]
\end{Arabic}}
\flushleft{\begin{hindi}
जैसाकि हमने तुम्हारे बीच एक रसूल तुम्हीं में से भेजा जो तुम्हें हमारी आयतें सुनाता है, तुम्हें निखारता है, और तुम्हें किताब और हिकमत (तत्वदर्शिता) की शिक्षा देता है और तुम्हें वह कुछ सिखाता है, जो तुम जानते न थे
\end{hindi}}
\flushright{\begin{Arabic}
\quranayah[2][152]
\end{Arabic}}
\flushleft{\begin{hindi}
अतः तुम मुझे याद रखो, मैं भी तुम्हें याद रखूँगा। और मेरा आभार स्वीकार करते रहना, मेरे प्रति अकृतज्ञता न दिखलाना
\end{hindi}}
\flushright{\begin{Arabic}
\quranayah[2][153]
\end{Arabic}}
\flushleft{\begin{hindi}
ऐ ईमान लानेवालो! धैर्य और नमाज़ से मदद प्राप्त। करो। निस्संदेह अल्लाह उन लोगों के साथ है जो धैर्य और दृढ़ता से काम लेते है
\end{hindi}}
\flushright{\begin{Arabic}
\quranayah[2][154]
\end{Arabic}}
\flushleft{\begin{hindi}
और जो लोग अल्लाह के मार्ग में मारे जाएँ उन्हें मुर्दा न कहो, बल्कि वे जीवित है, परन्तु तुम्हें एहसास नहीं होता
\end{hindi}}
\flushright{\begin{Arabic}
\quranayah[2][155]
\end{Arabic}}
\flushleft{\begin{hindi}
और हम अवश्य ही कुछ भय से, और कुछ भूख से, और कुछ जान-माल और पैदावार की कमी से तुम्हारी परीक्षा लेंगे। और धैर्य से काम लेनेवालों को शुभ-सूचना दे दो
\end{hindi}}
\flushright{\begin{Arabic}
\quranayah[2][156]
\end{Arabic}}
\flushleft{\begin{hindi}
जो लोग उस समय, जबकि उनपर कोई मुसीबत आती है, कहते है, "निस्संदेह हम अल्लाह ही के है और हम उसी की ओर लौटने वाले है।"
\end{hindi}}
\flushright{\begin{Arabic}
\quranayah[2][157]
\end{Arabic}}
\flushleft{\begin{hindi}
यही लोग है जिनपर उनके रब की विशेष कृपाएँ है और दयालुता भी; और यही लोग है जो सीधे मार्ग पर हैं
\end{hindi}}
\flushright{\begin{Arabic}
\quranayah[2][158]
\end{Arabic}}
\flushleft{\begin{hindi}
निस्संदेह सफ़ा और मरवा अल्लाह की विशेष निशानियों में से हैं; अतः जो इस घर (काबा) का हज या उमपा करे, उसके लिए इसमें कोई दोष नहीं कि वह इन दोनों (पहाडियों) के बीच फेरा लगाए। और जो कोई स्वेच्छा और रुचि से कोई भलाई का कार्य करे तो अल्लाह भी गुणग्राहक, सर्वज्ञ है
\end{hindi}}
\flushright{\begin{Arabic}
\quranayah[2][159]
\end{Arabic}}
\flushleft{\begin{hindi}
जो लोग हमारी उतारी हुई खुली निशानियों और मार्गदर्शन को छिपाते है, इसके बाद कि हम उन्हें लोगों के लिए किताब में स्पष्ट कर चुके है; वही है जिन्हें अल्लाह धिक्कारता है - और सभी धिक्कारने वाले भी उन्हें धिक्कारते है
\end{hindi}}
\flushright{\begin{Arabic}
\quranayah[2][160]
\end{Arabic}}
\flushleft{\begin{hindi}
सिवाय उनके जिन्होंने तौबा कर ली और सुधार कर लिया, और साफ़-साफ़ बयान कर दिया, तो उनकी तौबा मैं क़बूल करूँगा; मैं बड़ा तौबा क़बूल करनेवाला, अत्यन्त दयावान हूँ
\end{hindi}}
\flushright{\begin{Arabic}
\quranayah[2][161]
\end{Arabic}}
\flushleft{\begin{hindi}
जिन लोगों ने कुफ़्र किया और काफ़िर (इनकार करनेवाले) ही रहकर मरे, वही हैं जिनपर अल्लाह की, फ़रिश्तों की और सारे मनुष्यों की, सबकी फिटकार है
\end{hindi}}
\flushright{\begin{Arabic}
\quranayah[2][162]
\end{Arabic}}
\flushleft{\begin{hindi}
इसी दशा में वे सदैव रहेंगे, न उनकी यातना हल्की की जाएगी और न उन्हें मुहलत ही मिलेगी
\end{hindi}}
\flushright{\begin{Arabic}
\quranayah[2][163]
\end{Arabic}}
\flushleft{\begin{hindi}
तुम्हारा पूज्य-प्रभु अकेला पूज्य-प्रभु है, उस कृपाशील और दयावान के अतिरिक्त कोई पूज्य-प्रभु नहीं
\end{hindi}}
\flushright{\begin{Arabic}
\quranayah[2][164]
\end{Arabic}}
\flushleft{\begin{hindi}
निस्संदेह आकाशों और धरती की संरचना में, और रात और दिन की अदला-बदली में, और उन नौकाओं में जो लोगों की लाभप्रद चीज़े लेकर समुद्र (और नदी) में चलती है, और उस पानी में जिसे अल्लाह ने आकाश से उतारा, फिर जिसके द्वारा धरती को उसके निर्जीव हो जाने के पश्चात जीवित किया और उसमें हर एक (प्रकार के) जीवधारी को फैलाया और हवाओं को गर्दिश देने में और उन बादलों में जो आकाश और धरती के बीच (काम पर) नियुक्त होते है, उन लोगों के लिए कितनी ही निशानियाँ है जो बुद्धि से काम लें
\end{hindi}}
\flushright{\begin{Arabic}
\quranayah[2][165]
\end{Arabic}}
\flushleft{\begin{hindi}
कुछ लोग ऐसे भी है जो अल्लाह से हटकर दूसरों को उसके समकक्ष ठहराते है, उनसे ऐसा प्रेम करते है जैसा अल्लाह से प्रेम करना चाहिए। और कुछ ईमानवाले है उन्हें सबसे बढ़कर अल्लाह से प्रेम होता है। और ये अत्याचारी (बहुदेववादी) जबकि यातना देखते है, यदि इस तथ्य को जान लेते कि शक्ति सारी की सारी अल्लाह ही को प्राप्त हो और यह कि अल्लाह अत्यन्त कठोर यातना देनेवाला है (तो इनकी नीति कुछ और होती)
\end{hindi}}
\flushright{\begin{Arabic}
\quranayah[2][166]
\end{Arabic}}
\flushleft{\begin{hindi}
जब वे लोग जिनके पीछे वे चलते थे, यातना को देखकर अपने अनुयायियों से विरक्त हो जाएँगे और उनके सम्बन्ध और सम्पर्क टूट जाएँगे
\end{hindi}}
\flushright{\begin{Arabic}
\quranayah[2][167]
\end{Arabic}}
\flushleft{\begin{hindi}
वे लोग जो उनके पीछे चले थे कहेंगे, "काश! हमें एक बार (फिर संसार में लौटना होता तो जिस तरह आज ये हमसे विरक्त हो रहे हैं, हम भी इनसे विरक्त हो जाते।" इस प्रकार अल्लाह उनके लिए संताप बनाकर उन्हें कर्म दिखाएगा और वे आग (जहन्नम) से निकल न सकेंगे
\end{hindi}}
\flushright{\begin{Arabic}
\quranayah[2][168]
\end{Arabic}}
\flushleft{\begin{hindi}
ऐ लोगों! धरती में जो हलाल और अच्छी-सुथरी चीज़ें हैं उन्हें खाओ और शैतान के पदचिन्हों पर न चलो। निस्संदेह वह तुम्हारा खुला शत्रु है
\end{hindi}}
\flushright{\begin{Arabic}
\quranayah[2][169]
\end{Arabic}}
\flushleft{\begin{hindi}
वह तो बस तुम्हें बुराई और अश्लीलता पर उकसाता है और इसपर कि तुम अल्लाह पर थोपकर वे बातें कहो जो तुम नहीं जानते
\end{hindi}}
\flushright{\begin{Arabic}
\quranayah[2][170]
\end{Arabic}}
\flushleft{\begin{hindi}
और जब उनसे कहा जाता है, "अल्लाह ने जो कुछ उतारा है उसका अनुसरण करो।" तो कहते है, "नहीं बल्कि हम तो उसका अनुसरण करेंगे जिसपर हमने अपने बाप-दादा को पाया है।" क्या उस दशा में भी जबकि उनके बाप-दादा कुछ भी बुद्धि से काम न लेते रहे हों और न सीधे मार्ग पर रहे हों?
\end{hindi}}
\flushright{\begin{Arabic}
\quranayah[2][171]
\end{Arabic}}
\flushleft{\begin{hindi}
इन इनकार करनेवालों की मिसाल ऐसी है जैसे कोई ऐसी चीज़ों को पुकारे जो पुकार और आवाज़ के सिवा कुछ न सुनती और समझती हो। ये बहरे हैं, गूँगें हैं, अन्धें हैं; इसलिए ये कुछ भी नहीं समझ सकते
\end{hindi}}
\flushright{\begin{Arabic}
\quranayah[2][172]
\end{Arabic}}
\flushleft{\begin{hindi}
ऐ ईमान लानेवालो! जो अच्छी-सुथरी चीज़ें हमने तुम्हें प्रदान की हैं उनमें से खाओ और अल्लाह के आगे कृतज्ञता दिखलाओ, यदि तुम उसी की बन्दगी करते हो
\end{hindi}}
\flushright{\begin{Arabic}
\quranayah[2][173]
\end{Arabic}}
\flushleft{\begin{hindi}
उसने तो तुमपर केवल मुर्दार और ख़ून और सूअर का माँस और जिस पर अल्लाह के अतिरिक्त किसी और का नाम लिया गया हो, हराम ठहराया है। इसपर भी जो बहुत मजबूर और विवश हो जाए, वह अवज्ञा करनेवाला न हो और न सीमा से आगे बढ़नेवाला हो तो उसपर कोई गुनाह नहीं। निस्संदेह अल्लाह अत्यन्त क्षमाशील, दयावान है
\end{hindi}}
\flushright{\begin{Arabic}
\quranayah[2][174]
\end{Arabic}}
\flushleft{\begin{hindi}
जो लोग उस चीज़ को छिपाते है जो अल्लाह ने अपनी किताब में से उतारी है और उसके बदले थोड़े मूल्य का सौदा करते है, वे तो बस आग खाकर अपने पेट भर रहे है; और क़ियामत के दिन अल्लाह न तो उनसे बात करेगा और न उन्हें निखारेगा; और उनके लिए दुखद यातना है
\end{hindi}}
\flushright{\begin{Arabic}
\quranayah[2][175]
\end{Arabic}}
\flushleft{\begin{hindi}
यहीं लोग हैं जिन्होंने मार्गदर्शन के बदले पथभ्रष्टका मोल ली; और क्षमा के बदले यातना के ग्राहक बने। तो आग को सहन करने के लिए उनका उत्साह कितना बढ़ा हुआ है!
\end{hindi}}
\flushright{\begin{Arabic}
\quranayah[2][176]
\end{Arabic}}
\flushleft{\begin{hindi}
वह (यातना) इसलिए होगी कि अल्लाह ने तो हक़ के साथ किताब उतारी, किन्तु जिन लोगों ने किताब के मामले में विभेद किया वे हठ और विरोध में बहुत दूर निकल गए
\end{hindi}}
\flushright{\begin{Arabic}
\quranayah[2][177]
\end{Arabic}}
\flushleft{\begin{hindi}
नेकी केवल यह नहीं है कि तुम अपने मुँह पूरब और पश्चिम की ओर कर लो, बल्कि नेकी तो उसकी नेकी है जो अल्लाह अन्तिम दिन, फ़रिश्तों, किताब और नबियों पर ईमान लाया और माल, उसके प्रति प्रेम के बावजूद नातेदारों, अनाथों, मुहताजों, मुसाफ़िरों और माँगनेवालों को दिया और गर्दनें छुड़ाने में भी, और नमाज़ क़ायम की और ज़कात दी और अपने वचन को ऐसे लोग पूरा करनेवाले है जब वचन दें; और तंगी और विशेष रूप से शारीरिक कष्टों में और लड़ाई के समय में जमनेवाले हैं, तो ऐसे ही लोग है जो सच्चे सिद्ध हुए और वही लोग डर रखनेवाले हैं
\end{hindi}}
\flushright{\begin{Arabic}
\quranayah[2][178]
\end{Arabic}}
\flushleft{\begin{hindi}
ऐ ईमान लानेवालो! मारे जानेवालों के विषय में हत्यादंड (क़िसास) तुमपर अनिवार्य किया गया, स्वतंत्र-स्वतंत्र बराबर है और ग़़ुलाम-ग़ुलाम बराबर है और औरत-औरत बराबर है। फिर यदि किसी को उसके भाई की ओर से कुछ छूट मिल जाए तो सामान्य रीति का पालन करना चाहिए; और भले तरीके से उसे अदा करना चाहिए। यह तुम्हारें रब की ओर से एक छूट और दयालुता है। फिर इसके बाद भो जो ज़्यादती करे तो उसके लिए दुखद यातना है
\end{hindi}}
\flushright{\begin{Arabic}
\quranayah[2][179]
\end{Arabic}}
\flushleft{\begin{hindi}
ऐ बुद्धि और समझवालों! तुम्हारे लिए हत्यादंड (क़िसास) में जीवन है, ताकि तुम बचो
\end{hindi}}
\flushright{\begin{Arabic}
\quranayah[2][180]
\end{Arabic}}
\flushleft{\begin{hindi}
जब तुममें से किसी की मृत्यु का समय आ जाए, यदि वह कुछ माल छोड़ रहा हो, तो माँ-बाप और नातेदारों को भलाई की वसीयत करना तुमपर अनिवार्य किया गया। यह हक़ है डर रखनेवालों पर
\end{hindi}}
\flushright{\begin{Arabic}
\quranayah[2][181]
\end{Arabic}}
\flushleft{\begin{hindi}
तो जो कोई उसके सुनने के पश्चात उसे बदल डाले तो उसका गुनाह उन्हीं लोगों पर होगा जो इसे बदलेंगे। निस्संदेह अल्लाह सब कुछ सुननेवाला और जाननेवाला है
\end{hindi}}
\flushright{\begin{Arabic}
\quranayah[2][182]
\end{Arabic}}
\flushleft{\begin{hindi}
फिर जिस किसी वसीयत करनेवाले को न्याय से किसी प्रकार के हटने या हक़़ मारने की आशंका हो, इस कारण उनके (वारिसों के) बीच सुधार की व्यवस्था कर दें, तो उसपर कोई गुनाह नहीं। निस्संदेह अल्लाह क्षमाशील, अत्यन्त दयावान है
\end{hindi}}
\flushright{\begin{Arabic}
\quranayah[2][183]
\end{Arabic}}
\flushleft{\begin{hindi}
ऐ ईमान लानेवालो! तुमपर रोज़े अनिवार्य किए गए, जिस प्रकार तुमसे पहले के लोगों पर किए गए थे, ताकि तुम डर रखनेवाले बन जाओ
\end{hindi}}
\flushright{\begin{Arabic}
\quranayah[2][184]
\end{Arabic}}
\flushleft{\begin{hindi}
गिनती के कुछ दिनों के लिए - इसपर भी तुममें कोई बीमार हो, या सफ़र में हो तो दूसरे दिनों में संख्या पूरी कर ले। और जिन (बीमार और मुसाफ़िरों) को इसकी (मुहताजों को खिलाने की) सामर्थ्य हो, उनके ज़िम्मे बदलें में एक मुहताज का खाना है। फिर जो अपनी ख़ुशी से कुछ और नेकी करे तो यह उसी के लिए अच्छा है और यह कि तुम रोज़ा रखो तो तुम्हारे लिए अधिक उत्तम है, यदि तुम जानो
\end{hindi}}
\flushright{\begin{Arabic}
\quranayah[2][185]
\end{Arabic}}
\flushleft{\begin{hindi}
रमज़ान का महीना जिसमें कुरआन उतारा गया लोगों के मार्गदर्शन के लिए, और मार्गदर्शन और सत्य-असत्य के अन्तर के प्रमाणों के साथा। अतः तुममें जो कोई इस महीने में मौजूद हो उसे चाहिए कि उसके रोज़े रखे और जो बीमार हो या सफ़र में हो तो दूसरे दिनों में गिनती पूरी कर ले। अल्लाह तुम्हारे साथ आसानी चाहता है, वह तुम्हारे साथ सख़्ती और कठिनाई नहीं चाहता, (वह तुम्हारे लिए आसानी पैदा कर रहा है) और चाहता है कि तुम संख्या पूरी कर लो और जो सीधा मार्ग तुम्हें दिखाया गया है, उस पर अल्लाह की बड़ाई प्रकट करो और ताकि तुम कृतज्ञ बनो
\end{hindi}}
\flushright{\begin{Arabic}
\quranayah[2][186]
\end{Arabic}}
\flushleft{\begin{hindi}
और जब तुमसे मेरे बन्दे मेरे सम्बन्ध में पूछें, तो मैं तो निकट ही हूँ, पुकार का उत्तर देता हूँ, जब वह मुझे पुकारता है, तो उन्हें चाहिए कि वे मेरा हुक्म मानें और मुझपर ईमान रखें, ताकि वे सीधा मार्ग पा लें
\end{hindi}}
\flushright{\begin{Arabic}
\quranayah[2][187]
\end{Arabic}}
\flushleft{\begin{hindi}
तुम्हारे लिए रोज़ो की रातों में अपनी औरतों के पास जाना जायज़ (वैध) हुआ। वे तुम्हारे परिधान (लिबास) हैं और तुम उनका परिधान हो। अल्लाह को मालूम हो गया कि तुम लोग अपने-आपसे कपट कर रहे थे, तो उसने तुमपर कृपा की और तुम्हें क्षमा कर दिया। तो अब तुम उनसे मिलो-जुलो और अल्लाह ने जो कुछ तुम्हारे लिए लिख रखा है, उसे तलब करो। और खाओ और पियो यहाँ तक कि तुम्हें उषाकाल की सफ़ेद धारी (रात की) काली धारी से स्पष्टा दिखाई दे जाए। फिर रात तक रोज़ा पूरा करो और जब तुम मस्जिदों में 'एतकाफ़' की हालत में हो, तो तुम उनसे न मिलो। ये अल्लाह की सीमाएँ हैं। अतः इनके निकट न जाना। इस प्रकार अल्लाह अपनी आयतें लोगों के लिए खोल-खोलकर बयान करता है, ताकि वे डर रखनेवाले बनें
\end{hindi}}
\flushright{\begin{Arabic}
\quranayah[2][188]
\end{Arabic}}
\flushleft{\begin{hindi}
और आपस में तुम एक-दूसरे के माल को अवैध रूप से न खाओ, और न उन्हें हाकिमों के आगे ले जाओ कि (हक़ मारकर) लोगों के कुछ माल जानते-बूझते हड़प सको
\end{hindi}}
\flushright{\begin{Arabic}
\quranayah[2][189]
\end{Arabic}}
\flushleft{\begin{hindi}
वे तुमसे (प्रतिष्ठित) महीनों के विषय में पूछते है। कहो, "वे तो लोगों के लिए और हज के लिए नियत है। और यह कोई ख़ूबी और नेकी नहीं हैं कि तुम घरों में उनके पीछे से आओ, बल्कि नेकी तो उसकी है जो (अल्लाह का) डर रखे। तुम घरों में उनके दरवाड़ों से आओ और अल्लाह से डरते रहो, ताकि तुम्हें सफलता प्राप्त हो
\end{hindi}}
\flushright{\begin{Arabic}
\quranayah[2][190]
\end{Arabic}}
\flushleft{\begin{hindi}
और अल्लाह के मार्ग में उन लोगों से लड़ो जो तुमसे लड़े, किन्तु ज़्यादती न करो। निस्संदेह अल्लाह ज़्यादती करनेवालों को पसन्द नहीं करता
\end{hindi}}
\flushright{\begin{Arabic}
\quranayah[2][191]
\end{Arabic}}
\flushleft{\begin{hindi}
और जहाँ कहीं उनपर क़ाबू पाओ, क़त्ल करो और उन्हें निकालो जहाँ से उन्होंने तुम्हें निकाला है, इसलिए कि फ़ितना (उत्पीड़न) क़त्ल से भी बढ़कर गम्भीर है। लेकिन मस्जिदे हराम (काबा) के निकट तुम उनसे न लड़ो जब तक कि वे स्वयं तुमसे वहाँ युद्ध न करें। अतः यदि वे तुमसे युद्ध करें तो उन्हें क़त्ल करो - ऐसे इनकारियों का ऐसा ही बदला है
\end{hindi}}
\flushright{\begin{Arabic}
\quranayah[2][192]
\end{Arabic}}
\flushleft{\begin{hindi}
फिर यदि वे बाज़ आ जाएँ तो अल्लाह भी क्षमा करनेवाला, अत्यन्त दयावान है
\end{hindi}}
\flushright{\begin{Arabic}
\quranayah[2][193]
\end{Arabic}}
\flushleft{\begin{hindi}
तुम उनसे लड़ो यहाँ तक कि फ़ितना शेष न रह जाए और दीन (धर्म) अल्लाह के लिए हो जाए। अतः यदि वे बाज़ आ जाएँ तो अत्याचारियों के अतिरिक्त किसी के विरुद्ध कोई क़दम उठाना ठीक नहीं
\end{hindi}}
\flushright{\begin{Arabic}
\quranayah[2][194]
\end{Arabic}}
\flushleft{\begin{hindi}
प्रतिष्ठित महीना बराबर है प्रतिष्ठित महिने के, और समस्त प्रतिष्ठाओं का भी बराबरी का बदला है। अतः जो तुमपर ज़्यादती करे, तो जैसी ज़्यादती वह तुम पर के, तुम भी उसी प्रकार उससे ज़्यादती का बदला लो। और अल्लाह का डर रखो और जान लो कि अल्लाह डर रखनेवालों के साथ है
\end{hindi}}
\flushright{\begin{Arabic}
\quranayah[2][195]
\end{Arabic}}
\flushleft{\begin{hindi}
और अल्लाह के मार्ग में ख़र्च करो और अपने ही हाथों से अपने-आपकोतबाही में न डालो, और अच्छे से अच्छा तरीक़ा अपनाओ। निस्संदेह अल्लाह अच्छे से अच्छा काम करनेवालों को पसन्द करता है
\end{hindi}}
\flushright{\begin{Arabic}
\quranayah[2][196]
\end{Arabic}}
\flushleft{\begin{hindi}
और हज और उमरा जो कि अल्लाह के लिए है, पूरे करो। फिर यदि तुम घिर जाओ, तो जो क़ुरबानी उपलब्ध हो पेश कर दो। और अपने सिर न मूड़ो जब तक कि क़ुरबानी अपने ठिकाने न पहुँच जाए, किन्तु जो व्यक्ति तुममें बीमार हो या उसके सिर में कोई तकलीफ़ हो, तो रोज़े या सदक़ा या क़रबानी के रूप में फ़िद्याी देना होगा। फिर जब तुम पर से ख़तरा टल जाए, तो जो व्यक्ति हज तक उमरा से लाभान्वित हो, जो जो क़ुरबानी उपलब्ध हो पेश करे, और जिसको उपलब्ध न हो तो हज के दिनों में तीन दिन के रोज़े रखे और सात दिन के रोज़े जब तुम वापस हो, ये पूरे दस हुए। यह उसके लिए है जिसके बाल-बच्चे मस्जिदे हराम के निकट न रहते हों। अल्लाह का डर रखो और भली-भाँति जान लो कि अल्लाह कठोर दंड देनेवाला है
\end{hindi}}
\flushright{\begin{Arabic}
\quranayah[2][197]
\end{Arabic}}
\flushleft{\begin{hindi}
हज के महीने जाने-पहचाने और निश्चित हैं, तो जो इनमें हज करने का निश्चय करे, को हज में न तो काम-वासना की बातें हो सकती है और न अवज्ञा और न लड़ाई-झगड़े की कोई बात। और जो भलाई के काम भी तुम करोंगे अल्लाह उसे जानता होगा। और (ईश-भय) पाथेय ले लो, क्योंकि सबसे उत्तम पाथेय ईश-भय है। और ऐ बुद्धि और समझवालो! मेरा डर रखो
\end{hindi}}
\flushright{\begin{Arabic}
\quranayah[2][198]
\end{Arabic}}
\flushleft{\begin{hindi}
इसमे तुम्हारे लिए कोई गुनाह नहीं कि अपने रब का अनुग्रह तलब करो। फिर जब तुम अरफ़ात से चलो तो 'मशअरे हराम' (मुज़दल्फ़ा) के निकट ठहरकर अल्लाह को याद करो, और उसे याद करो जैसाकि उसने तुम्हें बताया है, और इससे पहले तुम पथभ्रष्ट थे
\end{hindi}}
\flushright{\begin{Arabic}
\quranayah[2][199]
\end{Arabic}}
\flushleft{\begin{hindi}
इसके पश्चात जहाँ से और सब लोग चलें, वहीं से तुम भी चलो, और अल्लाह से क्षमा की प्रार्थना करो। निस्संदेह अल्लाह अत्यन्त क्षमाशील, दयावान है
\end{hindi}}
\flushright{\begin{Arabic}
\quranayah[2][200]
\end{Arabic}}
\flushleft{\begin{hindi}
फिर जब तुम अपनी हज सम्बन्धी रीतियों को पूरा कर चुको तो अल्लाह को याद करो जैसे अपने बाप-दादा को याद करते रहे हो, बल्कि उससे भी बढ़कर याद करो। फिर लोगों सें कोई तो ऐसा है जो कहता है, "हमारे रब! हमें दुनिया में दे दो।" ऐसी हालत में आख़िरत में उसका कोई हिस्सा नहीं
\end{hindi}}
\flushright{\begin{Arabic}
\quranayah[2][201]
\end{Arabic}}
\flushleft{\begin{hindi}
और उनमें कोई ऐसा है जो कहता है, "हमारे रब! हमें प्रदान कर दुनिया में भी अच्छी दशा और आख़िरत में भी अच्छा दशा, और हमें आग (जहन्नम) की यातना से बचा ले।"
\end{hindi}}
\flushright{\begin{Arabic}
\quranayah[2][202]
\end{Arabic}}
\flushleft{\begin{hindi}
ऐसे ही लोग है कि उन्होंने जो कुछ कमाया है उसकी जिन्स का हिस्सा उनके लिए नियत है। और अल्लाह जल्द ही हिसाब चुकानेवाला है
\end{hindi}}
\flushright{\begin{Arabic}
\quranayah[2][203]
\end{Arabic}}
\flushleft{\begin{hindi}
और अल्लाह की याद में गिनती के ये कुछ दिन व्यतीत करो। फिर जो कोई जल्दी करके दो ही दिन में कूच करे तो इसमें उसपर कोई गुनाह नहीं। और जो ठहरा रहे तो इसमें भी उसपर कोई गुनाह नहीं। यह उसके लिेए है जो अल्लाह का डर रखे। और अल्लाह का डर रखो और जान रखो कि उसी के पास तुम इकट्ठा होगे
\end{hindi}}
\flushright{\begin{Arabic}
\quranayah[2][204]
\end{Arabic}}
\flushleft{\begin{hindi}
लोगों में कोई तो ऐसा है कि इस सांसारिक जीवन के विषय में उसकी बाते तुम्हें बहुत भाती है, उस (खोट) के बावजूद जो उसके दिल में होती है, वह अल्लाह को गवाह ठहराता है और झगड़े में वह बड़ा हठी है
\end{hindi}}
\flushright{\begin{Arabic}
\quranayah[2][205]
\end{Arabic}}
\flushleft{\begin{hindi}
और जब वह लौटता है, तो धरती में इसलिए दौड़-धूप करता है कि इसमें बिगाड़ पैदा करे और खेती और नस्ल को तबाह करे, जबकि अल्लाह बिगाड़ को पसन्द नहीं करता
\end{hindi}}
\flushright{\begin{Arabic}
\quranayah[2][206]
\end{Arabic}}
\flushleft{\begin{hindi}
और जब उससे कहा जाता है, "अल्लाह से डर", तो अहंकार उसे और गुनाह पर जमा देता है। अतः उसके लिए तो जहन्नम ही काफ़ी है, और वह बहुत-ही बुरी शय्या है!
\end{hindi}}
\flushright{\begin{Arabic}
\quranayah[2][207]
\end{Arabic}}
\flushleft{\begin{hindi}
और लोगों में वह भी है जो अल्लाह की प्रसन्नता के संसाधन की चाह में अपनी जान खता देता है। अल्लाह भी अपने ऐसे बन्दों के प्रति अत्यन्त करुणाशील है
\end{hindi}}
\flushright{\begin{Arabic}
\quranayah[2][208]
\end{Arabic}}
\flushleft{\begin{hindi}
ऐ ईमान लानेवालो! तुम सब इस्लाम में दाख़िल हो जाओ और शैतान के पदचिन्ह पर न चलो। वह तो तुम्हारा खुला हुआ शत्रु है
\end{hindi}}
\flushright{\begin{Arabic}
\quranayah[2][209]
\end{Arabic}}
\flushleft{\begin{hindi}
फिर यदि तुम उन स्पष्टा दलीलों के पश्चात भी, जो तुम्हारे पास आ चुकी है, फिसल गए, तो भली-भाँति जान रखो कि अल्लाह अत्यन्त प्रभुत्वशाली, तत्वदर्शी है
\end{hindi}}
\flushright{\begin{Arabic}
\quranayah[2][210]
\end{Arabic}}
\flushleft{\begin{hindi}
क्या वे (इसराईल की सन्तान) बस इसकी प्रतीक्षा कर रहे है कि अल्लाह स्वयं ही बादलों की छायों में उनके सामने आ जाए और फ़रिश्ते भी, हालाँकि बात तय कर दी गई है? मामले तो अल्लाह ही की ओर लौटते है
\end{hindi}}
\flushright{\begin{Arabic}
\quranayah[2][211]
\end{Arabic}}
\flushleft{\begin{hindi}
इसराईल की सन्तान से पूछो, हमने उन्हें कितनी खुली-खुली निशानियाँ प्रदान की। और जो अल्लाह की नेमत को इसके बाद कि वह उसे पहुँच चुकी हो बदल डाले, तो निस्संदेह अल्लाह भी कठोर दंड देनेवाला है
\end{hindi}}
\flushright{\begin{Arabic}
\quranayah[2][212]
\end{Arabic}}
\flushleft{\begin{hindi}
इनकार करनेवाले सांसारिक जीवन पर रीझे हुए है और ईमानवालों का उपहास करते है, जबकि जो लोग अल्लाह का डर रखते है, वे क़ियामत के दिन उनसे ऊपर होंगे। अल्लाह जिस चाहता है बेहिसाब देता है
\end{hindi}}
\flushright{\begin{Arabic}
\quranayah[2][213]
\end{Arabic}}
\flushleft{\begin{hindi}
सारे मनुष्य एक ही समुदाय थे (उन्होंने विभेद किया) तो अल्लाह ने नबियों को भेजा, जो शुभ-सूचना देनेवाले और डरानवाले थे; और उनके साथ हक़ पर आधारित किताब उतारी, ताकि लोगों में उन बातों का जिनमें वे विभेद कर रहे है, फ़ैसला कर दे। इसमें विभेद तो बस उन्हीं लोगों ने, जिन्हें वह मिली थी, परस्पर ज़्यादती करने के लिए इसके पश्चात किया, जबकि खुली निशानियाँ उनके पास आ चुकी थी। अतः ईमानवालों को अल्लाह ने अपनी अनूज्ञा से उस सत्य के विषय में मार्गदर्शन किया, जिसमें उन्होंने विभेद किया था। अल्लाह जिसे चाहता है, सीधे मार्ग पर चलाता है
\end{hindi}}
\flushright{\begin{Arabic}
\quranayah[2][214]
\end{Arabic}}
\flushleft{\begin{hindi}
क्या तुमने यह समझ रखा है कि जन्नत में प्रवेश पा जाओगे, जबकि अभी तुम पर वह सब कुछ नहीं बीता है जो तुमसे पहले के लोगों पर बीत चुका? उनपर तंगियाँ और तकलीफ़े आई और उन्हें हिला मारा गया यहाँ तक कि रसूल बोल उठे और उनके साथ ईमानवाले भी कि अल्लाह की सहायता कब आएगी? जान लो! अल्लाह की सहायता निकट है
\end{hindi}}
\flushright{\begin{Arabic}
\quranayah[2][215]
\end{Arabic}}
\flushleft{\begin{hindi}
वे तुमसे पूछते है, "कितना ख़र्च करें?" कहो, "(पहले यह समझ लो कि) जो माल भी तुमने ख़र्च किया है, वह तो माँ-बाप, नातेदारों और अनाथों, और मुहताजों और मुसाफ़िरों के लिए ख़र्च हुआ है। और जो भलाई भी तुम करो, निस्संदेह अल्लाह उसे भली-भाँति जान लेगा।
\end{hindi}}
\flushright{\begin{Arabic}
\quranayah[2][216]
\end{Arabic}}
\flushleft{\begin{hindi}
तुम पर युद्ध अनिवार्य किया गया और वह तुम्हें अप्रिय है, और बहुत सम्भव है कि कोई चीज़ तुम्हें अप्रिय हो और वह तुम्हारे लिए अच्छी हो। और बहुत सम्भव है कि कोई चीज़ तुम्हें प्रिय हो और वह तुम्हारे लिए बुरी हो। और जानता अल्लाह है, और तुम नहीं जानते।"
\end{hindi}}
\flushright{\begin{Arabic}
\quranayah[2][217]
\end{Arabic}}
\flushleft{\begin{hindi}
वे तुमसे आदरणीय महीने में युद्ध के विषय में पूछते है। कहो, "उसमें लड़ना बड़ी गम्भीर बात है, परन्तु अल्लाह के मार्ग से रोकना, उसके साथ अविश्वास करना, मस्जिदे हराम (काबा) से रोकना और उसके लोगों को उससे निकालना, अल्लाह की स्पष्ट में इससे भी अधिक गम्भीर है और फ़ितना (उत्पीड़न), रक्तपात से भी बुरा है।" और उसका बस चले तो वे तो तुमसे बराबर लड़ते रहे, ताकि तुम्हें तुम्हारे दीन (धर्म) से फेर दें। और तुममे से जो कोई अपने दीन से फिर जाए और अविश्वासी होकर मरे, तो ऐसे ही लोग है जिनके कर्म दुनिया और आख़िरत में नष्ट हो गए, और वही आग (जहन्नम) में पड़नेवाले है, वे उसी में सदैव रहेंगे
\end{hindi}}
\flushright{\begin{Arabic}
\quranayah[2][218]
\end{Arabic}}
\flushleft{\begin{hindi}
रहे वे लोग जो ईमान लाए और जिन्होंने अल्लाह के मार्ग में घर-बार छोड़ा और जिहाद किया, वहीं अल्लाह की दयालुता की आशा रखते है। निस्संदेह अल्लाह अत्यन्त क्षमाशील, दयावान है
\end{hindi}}
\flushright{\begin{Arabic}
\quranayah[2][219]
\end{Arabic}}
\flushleft{\begin{hindi}
तुमसे शराब और जुए के विषय में पूछते है। कहो, "उन दोनों चीज़ों में बड़ा गुनाह है, यद्यपि लोगों के लिए कुछ फ़ायदे भी है, परन्तु उनका गुनाह उनके फ़ायदे से कहीं बढकर है।" और वे तुमसे पूछते है, "कितना ख़र्च करें?" कहो, "जो आवश्यकता से अधिक हो।" इस प्रकार अल्लाह दुनिया और आख़िरत के विषय में तुम्हारे लिए अपनी आयते खोल-खोलकर बयान करता है, ताकि तुम सोच-विचार करो।
\end{hindi}}
\flushright{\begin{Arabic}
\quranayah[2][220]
\end{Arabic}}
\flushleft{\begin{hindi}
और वे तुमसे अनाथों के विषय में पूछते है। कहो, "उनके सुधार की जो रीति अपनाई जाए अच्छी है। और यदि तुम उन्हें अपने साथ सम्मिलित कर लो तो वे तुम्हारे भाई-बन्धु ही हैं। और अल्लाह बिगाड़ पैदा करनेवाले को बचाव पैदा करनेवाले से अलग पहचानता है। और यदि अल्लाह चाहता तो तुमको ज़हमत (कठिनाई) में डाल देता। निस्संदेह अल्लाह प्रभुत्वशाली, तत्वदर्शी है।"
\end{hindi}}
\flushright{\begin{Arabic}
\quranayah[2][221]
\end{Arabic}}
\flushleft{\begin{hindi}
और मुशरिक (बहुदेववादी) स्त्रियों से विवाह न करो जब तक कि वे ईमान न लाएँ। एक ईमानदारी बांदी (दासी), मुशरिक स्त्री से कहीं उत्तम है; चाहे वह तुम्हें कितनी ही अच्छी क्यों न लगे। और न (ईमानवाली स्त्रियाँ) मुशरिक पुरुषों से विवाह करो, जब तक कि वे ईमान न लाएँ। एक ईमानवाला गुलाम आज़ाद मुशरिक से कहीं उत्तम है, चाहे वह तुम्हें कितना ही अच्छा क्यों न लगे। ऐसे लोग आग (जहन्नम) की ओर बुलाते है और अल्लाह अपनी अनुज्ञा से जन्नत और क्षमा की ओर बुलाता है। और वह अपनी आयतें लोगों के सामने खोल-खोलकर बयान करता है, ताकि वे चेतें
\end{hindi}}
\flushright{\begin{Arabic}
\quranayah[2][222]
\end{Arabic}}
\flushleft{\begin{hindi}
और वे तुमसे मासिक-धर्म के विषय में पूछते है। कहो, "वह एक तकलीफ़ और गन्दगी की चीज़ है। अतः मासिक-धर्म के दिनों में स्त्रियों से अलग रहो और उनके पास न जाओ, जबतक कि वे पाक-साफ़ न हो जाएँ। फिर जब वे भली-भाँति पाक-साफ़ हो जाए, तो जिस प्रकार अल्लाह ने तुम्हें बताया है, उनके पास आओ। निस्संदेह अल्लाह बहुत तौबा करनेवालों को पसन्द करता है और वह उन्हें पसन्द करता है जो स्वच्छता को पसन्द करते है
\end{hindi}}
\flushright{\begin{Arabic}
\quranayah[2][223]
\end{Arabic}}
\flushleft{\begin{hindi}
तुम्हारी स्त्रियों तुम्हारी खेती है। अतः जिस प्रकार चाहो तुम अपनी खेती में आओ और अपने लिए आगे भेजो; और अल्लाह से डरते रहो; भली-भाँति जान ले कि तुम्हें उससे मिलना है; और ईमान लानेवालों को शुभ-सूचना दे दो
\end{hindi}}
\flushright{\begin{Arabic}
\quranayah[2][224]
\end{Arabic}}
\flushleft{\begin{hindi}
अपने नेक और धर्मपरायण होने और लोगों के मध्य सुधारक होने के सिलसिले में अपनी क़समों के द्वारा अल्लाह को आड़ और निशाना न बनाओ कि इन कामों को छोड़ दो। अल्लाह सब कुछ सुनता, जानता है
\end{hindi}}
\flushright{\begin{Arabic}
\quranayah[2][225]
\end{Arabic}}
\flushleft{\begin{hindi}
अल्लाह तुम्हें तुम्हारी ऐसी कसमों पर नहीं पकड़ेगा जो यूँ ही मुँह से निकल गई हो, लेकिन उन क़समों पर वह तुम्हें अवश्य पकड़ेगा जो तुम्हारे दिल के इरादे का नतीजा हों। अल्लाह बहुत क्षमा करनेवाला, सहनशील है
\end{hindi}}
\flushright{\begin{Arabic}
\quranayah[2][226]
\end{Arabic}}
\flushleft{\begin{hindi}
जो लोग अपनी स्त्रियों से अलग रहने की क़सम खा बैठें, उनके लिए चार महीने की प्रतिक्षा है। फिर यदि वे पलट आएँ, तो अल्लाह अत्यन्त क्षमाशील, दयावान है
\end{hindi}}
\flushright{\begin{Arabic}
\quranayah[2][227]
\end{Arabic}}
\flushleft{\begin{hindi}
और यदि वे तलाक़ ही की ठान लें, तो अल्लाह भी सुननेवाला भली-भाँति जाननेवाला है
\end{hindi}}
\flushright{\begin{Arabic}
\quranayah[2][228]
\end{Arabic}}
\flushleft{\begin{hindi}
और तलाक़ पाई हुई स्त्रियाँ तीन हैज़ (मासिक-धर्म) गुज़रने तक अपने-आप को रोके रखे, और यदि वे अल्लाह और अन्तिम दिन पर ईमान रखती है तो उनके लिए यह वैध न होगा कि अल्लाह ने उनके गर्भाशयों में जो कुछ पैदा किया हो उसे छिपाएँ। इस बीच उनके पति, यदि सम्बन्धों को ठीक कर लेने का इरादा रखते हों, तो वे उन्हें लौटा लेने के ज़्यादा हक़दार है। और उन पत्नियों के भी सामान्य नियम के अनुसार वैसे ही अधिकार हैं, जैसी उन पर ज़िम्मेदारियाँ डाली गई है। और पतियों को उनपर एक दर्जा प्राप्त है। अल्लाह अत्यन्त प्रभुत्वशाली, तत्वदर्शी है
\end{hindi}}
\flushright{\begin{Arabic}
\quranayah[2][229]
\end{Arabic}}
\flushleft{\begin{hindi}
तलाक़ दो बार है। फिर सामान्य नियम के अनुसार (स्त्री को) रोक लिया जाए या भले तरीक़े से विदा कर दिया जाए। और तुम्हारे लिए वैध नहीं है कि जो कुछ तुम उन्हें दे चुके हो, उसमें से कुछ ले लो, सिवाय इस स्थिति के कि दोनों को डर हो कि अल्लाह की (निर्धारित) सीमाओं पर क़ायम न रह सकेंगे तो यदि तुमको यह डर हो कि वे अल्लाह की सीमाओ पर क़ायम न रहेंगे तो स्त्री जो कुछ देकर छुटकारा प्राप्त करना चाहे उसमें उन दोनो के लिए कोई गुनाह नहीं। ये अल्लाह की सीमाएँ है। अतः इनका उल्लंघन न करो। और जो कोई अल्लाह की सीमाओं का उल्लंघन करे तो ऐसे लोग अत्याचारी है
\end{hindi}}
\flushright{\begin{Arabic}
\quranayah[2][230]
\end{Arabic}}
\flushleft{\begin{hindi}
(दो तलाक़ो के पश्चात) फिर यदि वह उसे तलाक़ दे दे, तो इसके पश्चात वह उसके लिए वैध न होगी, जबतक कि वह उसके अतिरिक्त किसी दूसरे पति से निकाह न कर ले। अतः यदि वह उसे तलाक़ दे दे तो फिर उन दोनों के लिए एक-दूसरे को पलट आने में कोई गुनाह न होगा, यदि वे समझते हो कि अल्लाह की सीमाओं पर क़ायम रह सकते है। और ये अल्लाह कि निर्धारित की हुई सीमाएँ है, जिन्हें वह उन लोगों के लिए बयान कर रहा है जो जानना चाहते हो
\end{hindi}}
\flushright{\begin{Arabic}
\quranayah[2][231]
\end{Arabic}}
\flushleft{\begin{hindi}
और यदि जब तुम स्त्रियों को तलाक़ दे दो और वे अपनी निश्चित अवधि (इद्दत) को पहुँच जाएँ, जो सामान्य नियम के अनुसार उन्हें रोक लो या सामान्य नियम के अनुसार उन्हें विदा कर दो। और तुम उन्हें नुक़सान पहुँचाने के ध्येय से न रोको कि ज़्यादती करो। और जो ऐसा करेगा, तो उसने स्वयं अपने ही ऊपर ज़ुल्म किया। और अल्लाह की आयतों को परिहास का विषय न बनाओ, और अल्लाह की कृपा जो तुम पर हुई है उसे याद रखो और उस किताब और तत्वदर्शिता (हिकमत) को याद रखो जो उसने तुम पर उतारी है, जिसके द्वारा वह तुम्हें नसीहत करता है। और अल्लाह का डर रखो और भली-भाँति जान लो कि अल्लाह हर चीज को जाननेवाला है
\end{hindi}}
\flushright{\begin{Arabic}
\quranayah[2][232]
\end{Arabic}}
\flushleft{\begin{hindi}
और जब तुम स्त्रियों को तलाक़ दे दो और वे अपनी निर्धारित अवधि (इद्दत) को पहुँच जाएँ, तो उन्हें अपने होनेवाले दूसरे पतियों से विवाह करने से न रोको, जबकि वे सामान्य नियम के अनुसार परस्पर रज़ामन्दी से मामला तय करें। यह नसीहत तुममें से उसको की जा रही है जो अल्लाह और अन्तिम दिन पर ईमान रखता है। यही तुम्हारे लिए ज़्यादा बरकतवाला और सुथरा तरीक़ा है। और अल्लाह जानता है, तुम नहीं जानते
\end{hindi}}
\flushright{\begin{Arabic}
\quranayah[2][233]
\end{Arabic}}
\flushleft{\begin{hindi}
और जो कोई पूरी अवधि तक (बच्चे को) दूध पिलवाना चाहे, तो माएँ अपने बच्चों को पूरे दो वर्ष तक दूध पिलाएँ। और वह जिसका बच्चा है, सामान्य नियम के अनुसार उनके खाने और उनके कपड़े का ज़िम्मेदार है। किसी पर बस उसकी अपनी समाई भर ही ज़िम्मेदारी है, न तो कोई माँ अपने बच्चे के कारण (बच्चे के बाप को) नुक़सान पहुँचाए और न बाप अपने बच्चे के कारण (बच्चे की माँ को) नुक़सान पहुँचाए। और इसी प्रकार की ज़िम्मेदारी उसके वारिस पर भी आती है। फिर यदि दोनों पारस्परिक स्वेच्छा और परामर्श से दूध छुड़ाना चाहें तो उनपर कोई गुनाह नहीं। और यदि तुम अपनी संतान को किसी अन्य स्त्री से दूध पिलवाना चाहो तो इसमें भी तुम पर कोई गुनाह नहीं, जबकि तुमने जो कुछ बदले में देने का वादा किया हो, सामान्य नियम के अनुसार उसे चुका दो। और अल्लाह का डर रखो और भली-भाँति जान लो कि जो कुछ तुम करते हो, अल्लाह उसे देख रहा है
\end{hindi}}
\flushright{\begin{Arabic}
\quranayah[2][234]
\end{Arabic}}
\flushleft{\begin{hindi}
और तुममें से जो लोग मर जाएँ और अपने पीछे पत्नियों छोड़ जाएँ, तो वे पत्नियों अपने-आपको चार महीने और दस दिन तक रोके रखें। फिर जब वे अपनी निर्धारित अवधि (इद्दत) को पहुँच जाएँ, तो सामान्य नियम के अनुसार वे अपने लिए जो कुछ करें, उसमें तुमपर कोई गुनाह नहीं। जो कुछ तुम करते हो, अल्लाह उसकी ख़बर रखता है
\end{hindi}}
\flushright{\begin{Arabic}
\quranayah[2][235]
\end{Arabic}}
\flushleft{\begin{hindi}
और इसमें भी तुम पर कोई गुनाह नहीं जो तुम उन औरतों को विवाह के सन्देश सांकेतिक रूप से दो या अपने मन में छिपाए रखो। अल्लाह जानता है कि तुम उन्हें याद करोगे, परन्तु छिपकर उन्हें वचन न देना, सिवाय इसके कि सामान्य नियम के अनुसार कोई बात कह दो। और जब तक निर्धारित अवधि (इद्दत) पूरी न हो जाए, विवाह का नाता जोड़ने का निश्चय न करो। जान रखो कि अल्लाह तुम्हारे मन की बात भी जानता है। अतः उससे सावधान रहो और अल्लाह अत्यन्त क्षमा करनेवाला, सहनशील है
\end{hindi}}
\flushright{\begin{Arabic}
\quranayah[2][236]
\end{Arabic}}
\flushleft{\begin{hindi}
यदि तुम स्त्रियों को इस स्थिति मे तलाक़ दे दो कि यह नौबत पेश न आई हो कि तुमने उन्हें हाथ लगाया हो और उनका कुछ हक़ (मह्रन) निश्चित किया हो, तो तुमपर कोई भार नहीं। हाँ, सामान्य नियम के अनुसार उन्हें कुछ ख़र्च दो - समाई रखनेवाले पर उसकी अपनी हैसियत के अनुसार और तंगदस्त पर उसकी अपनी हैसियत के अनुसार अनिवार्य है - यह अच्छे लोगों पर एक हक़ है
\end{hindi}}
\flushright{\begin{Arabic}
\quranayah[2][237]
\end{Arabic}}
\flushleft{\begin{hindi}
और यदि तुम उन्हें हाथ लगाने से पहले तलाक़ दे दो, किन्तु उसका मह्र- निश्चित कर चुके हो, तो जो मह्रह तुमने निश्चित किया है उसका आधा अदा करना होगा, यह और बात है कि वे स्वयं छोड़ दे या पुरुष जिसके हाथ में विवाह का सूत्र है, वह नर्मी से काम ले (और मह्र पूरा अदा कर दे) । और यह कि तुम नर्मी से काम लो तो यह परहेज़गारी से ज़्यादा क़रीब है और तुम एक-दूसरे को हक़ से बढ़कर देना न भूलो। निश्चय ही अल्लाह उसे देख रहा है, जो तुम करते हो
\end{hindi}}
\flushright{\begin{Arabic}
\quranayah[2][238]
\end{Arabic}}
\flushleft{\begin{hindi}
सदैव नमाज़ो की और अच्छी नमाज़ों की पाबन्दी करो, और अल्लाह के आगे पूरे विनीत और शान्तभाव से खड़े हुआ करो
\end{hindi}}
\flushright{\begin{Arabic}
\quranayah[2][239]
\end{Arabic}}
\flushleft{\begin{hindi}
फिर यदि तुम्हें (शत्रु आदि का) भय हो, तो पैदल या सवार जिस तरह सम्भव हो नमाज़ पढ़ लो। फिर जब निश्चिन्त हो तो अल्लाह को उस प्रकार याद करो जैसाकि उसने तुम्हें सिखाया है, जिसे तुम नहीं जानते थे
\end{hindi}}
\flushright{\begin{Arabic}
\quranayah[2][240]
\end{Arabic}}
\flushleft{\begin{hindi}
और तुममें से जिन लोगों की मृत्यु हो जाए और अपने पीछे पत्नियों छोड़ जाए, अर्थात अपनी पत्नियों के हक़ में यह वसीयत छोड़ जाए कि घर से निकाले बिना एक वर्ष तक उन्हें ख़र्च दिया जाए, तो यदि वे निकल जाएँ तो अपने लिए सामान्य नियम के अनुसार वे जो कुछ भी करें उसमें तुम्हारे लिए कोई दोष नहीं। अल्लाह अत्यन्त प्रभुत्वशाली, तत्वदर्शी है
\end{hindi}}
\flushright{\begin{Arabic}
\quranayah[2][241]
\end{Arabic}}
\flushleft{\begin{hindi}
और तलाक़ पाई हुई स्त्रियों को सामान्य नियम के अनुसार (इद्दत की अवधि में) ख़र्च भी मिलना चाहिए। यह डर रखनेवालो पर एक हक़ है
\end{hindi}}
\flushright{\begin{Arabic}
\quranayah[2][242]
\end{Arabic}}
\flushleft{\begin{hindi}
इस प्रकार अल्लाह तुम्हारे लिए अपनी आयतें खोलकर बयान करता है, ताकि तुम समझ से काम लो
\end{hindi}}
\flushright{\begin{Arabic}
\quranayah[2][243]
\end{Arabic}}
\flushleft{\begin{hindi}
क्या तुमने उन लोगों को नहीं देखा जो हज़ारों की संख्या में होने पर भी मृत्यु के भय से अपने घर-बार छोड़कर निकले थे? तो अल्लाह ने उनसे कहा, "मृत्यु प्राय हो जाओ तुम।" फिर उसने उन्हें जीवन प्रदान किया। अल्लाह तो लोगों के लिए उदार अनुग्राही है, किन्तु अधिकतर लोग कृतज्ञता नहीं दिखलाते
\end{hindi}}
\flushright{\begin{Arabic}
\quranayah[2][244]
\end{Arabic}}
\flushleft{\begin{hindi}
और अल्लाह के मार्ग में युद्ध करो और जान लो कि अल्लाह सब कुछ सुननेवाला, जाननेवाले है
\end{hindi}}
\flushright{\begin{Arabic}
\quranayah[2][245]
\end{Arabic}}
\flushleft{\begin{hindi}
कौन है जो अल्लाह को अच्छा ऋण दे कि अल्लाह उसे उसके लिए कई गुना बढ़ा दे? और अल्लाह ही तंगी भी देता है और कुशादगी भी प्रदान करता है, और उसी की ओर तुम्हें लौटना है
\end{hindi}}
\flushright{\begin{Arabic}
\quranayah[2][246]
\end{Arabic}}
\flushleft{\begin{hindi}
क्या तुमने मूसा के पश्चात इसराईल की सन्तान के सरदारों को नहीं देखा, जब उन्होंने अपने एक नबी से कहा, "हमारे लिए एक सम्राट नियुक्त कर दो ताकि हम अल्लाह के मार्ग में युद्ध करें?" उसने कहा, "यदि तुम्हें लड़ाई का आदेश दिया जाए तो क्या तुम्हारे बारे में यह सम्भावना नहीं है कि तुम न लड़ो?" वे कहने लगे, "हम अल्लाह के मार्ग में क्यों न लड़े, जबकि हम अपने घरों से निकाल दिए गए है और अपने बाल-बच्चों से भी अलग कर दिए गए है?" - फिर जब उनपर युद्ध अनिवार्य कर दिया गया तो उनमें से थोड़े लोगों के सिवा सब फिर गए। और अल्लाह ज़ालिमों को भली-भाँति जानता है। -
\end{hindi}}
\flushright{\begin{Arabic}
\quranayah[2][247]
\end{Arabic}}
\flushleft{\begin{hindi}
उनसे नबी ने उनसे कहा, "अल्लाह ने तुम्हारे लिए तालूत को सम्राट नियुक्त किया है।" बोले, "उसकी बादशाही हम पर कैसे हो सकती है, जबबकि हम उसके मुक़ाबले में बादशाही के ज़्यादा हक़दार है और जबकि उस माल की कुशादगी भी प्राप्त नहीं है?" उसने कहा, "अल्लाह ने तुम्हारे मुक़ाबले में उसको ही चुना है और उसे ज्ञान में और शारीरिक क्षमता में ज़्यादा कुशादगी प्रदान की है। अल्लाह जिसको चाहे अपना राज्य प्रदान करे। और अल्लाह बड़ी समाईवाला, सर्वज्ञ है।"
\end{hindi}}
\flushright{\begin{Arabic}
\quranayah[2][248]
\end{Arabic}}
\flushleft{\begin{hindi}
उनके नबी ने उनसे कहा, "उसकी बादशाही की निशानी यह है कि वह संदूक तुम्हारे पर आ जाएगा, जिसमें तुम्हारे रह की ओर से सकीनत (प्रशान्ति) और मूसा के लोगों और हारून के लोगों की छोड़ी हुई यादगारें हैं, जिसको फ़रिश्ते उठाए हुए होंगे। यदि तुम ईमानवाले हो तो, निस्संदेह इसमें तुम्हारे लिए बड़ी निशानी है।"
\end{hindi}}
\flushright{\begin{Arabic}
\quranayah[2][249]
\end{Arabic}}
\flushleft{\begin{hindi}
फिर तब तालूत सेनाएँ लेकर चला तो उनने कहा, "अल्लाह निश्चित रूप से एक नदी द्वारा तुम्हारी परीक्षा लेनेवाला है। तो जिसने उसका पानी पी लिया, वह मुझमें से नहीं है और जिसने उसको नहीं चखा, वही मुझमें से है। यह और बात है कि कोई अपने हाथ से एक चुल्लू भर ले ले।" फिर उनमें से थोड़े लोगों के सिवा सभी ने उसका पानी पी लिया, फिर जब तालूत और ईमानवाले जो उसके साथ थे नदी पार कर गए तो कहने लगे, "आज हममें जालूत और उसकी सेनाओं का मुक़ाबला करने की शक्ति नहीं हैं।" इस पर लोगों ने, जो समझते थे कि उन्हें अल्लाह से मिलना है, कहा, "कितनी ही बार एक छोटी-सी टुकड़ी ने अल्लाह की अनुज्ञा से एक बड़े गिरोह पर विजय पाई है। अल्लाह तो जमनेवालो के साथ है।"
\end{hindi}}
\flushright{\begin{Arabic}
\quranayah[2][250]
\end{Arabic}}
\flushleft{\begin{hindi}
और जब वे जालूत और उसकी सेनाओं के मुक़ाबले पर आए तो कहा, "ऐ हमारे रब! हमपर धैर्य उडेल दे और हमारे क़दम जमा दे और इनकार करनेवाले लोगों पर हमें विजय प्रदान कर।"
\end{hindi}}
\flushright{\begin{Arabic}
\quranayah[2][251]
\end{Arabic}}
\flushleft{\begin{hindi}
अन्ततः अल्लाह की अनुज्ञा से उन्होंने उनको पराजित कर दिया और दाऊद ने जालूत को क़त्ल कर दिया, और अल्लाह ने उसे राज्य और तत्वदर्शिता (हिकमत) प्रदान की, जो कुछ वह (दाऊद) चाहे, उससे उसको अवगत कराया। और यदि अल्लाह मनुष्यों के एक गिरोह को दूसरे गिरोह के द्वारा हटाता न रहता तो धरती की व्यवस्था बिगड़ जाती, किन्तु अल्लाह संसारवालों के लिए उदार अनुग्राही है
\end{hindi}}
\flushright{\begin{Arabic}
\quranayah[2][252]
\end{Arabic}}
\flushleft{\begin{hindi}
ये अल्लाह की सच्ची आयतें है जो हम तुम्हें (सोद्देश्य) सुना रहे है और निश्चय ही तुम उन लोगों में से हो, जो रसूस बनाकर भेजे गए है
\end{hindi}}
\flushright{\begin{Arabic}
\quranayah[2][253]
\end{Arabic}}
\flushleft{\begin{hindi}
ये रसूल ऐसे हुए है कि इनमें हमने कुछ को कुछ पर श्रेष्ठता प्रदान की। इनमें कुछ से तो अल्लाह ने बातचीत की और इनमें से कुछ को दर्जों की स्पष्ट से उच्चता प्रदान की। और मरयम के बेटे ईसा को हमने खुली निशानियाँ दी और पवित्र आत्मा से उसकी सहायता की। और यदि अल्लाह चाहता तो वे लोग, जो उनके पश्चात हुए, खुली निशानियाँ पा लेने के बाद परस्पर न लड़ते। किन्तु वे विभेद में पड़ गए तो उनमें से कोई तो ईमान लाया और उनमें से किसी ने इनकार की नीति अपनाई। और यदि अल्लाह चाहता तो वे परस्पर न लड़ते, परन्तु अल्लाह जो चाहता है, करता है
\end{hindi}}
\flushright{\begin{Arabic}
\quranayah[2][254]
\end{Arabic}}
\flushleft{\begin{hindi}
ऐ ईमान लानेवालो! हमने जो कुछ तुम्हें प्रदान किया है उसमें से ख़र्च करो, इससे पहले कि वह दिन आ जाए जिसमें न कोई क्रय-विक्रय होगा और न कोई मित्रता होगी और न कोई सिफ़ारिश। ज़ालिम वही है, जिन्होंने इनकार की नीति अपनाई है
\end{hindi}}
\flushright{\begin{Arabic}
\quranayah[2][255]
\end{Arabic}}
\flushleft{\begin{hindi}
अल्लाह कि जिसके सिवा कोई पूज्य-प्रभु नहीं, वह जीवन्त-सत्ता है, सबको सँभालने और क़ायम रखनेवाला है। उसे न ऊँघ लगती है और न निद्रा। उसी का है जो कुछ आकाशों में है और जो कुछ धरती में है। कौन है जो उसके यहाँ उसकी अनुमति के बिना सिफ़ारिश कर सके? वह जानता है जो कुछ उनके आगे है और जो कुछ उनके पीछे है। और वे उसके ज्ञान में से किसी चीज़ पर हावी नहीं हो सकते, सिवाय उसके जो उसने चाहा। उसकी कुर्सी (प्रभुता) आकाशों और धरती को व्याप्त है और उनकी सुरक्षा उसके लिए तनिक भी भारी नहीं और वह उच्च, महान है
\end{hindi}}
\flushright{\begin{Arabic}
\quranayah[2][256]
\end{Arabic}}
\flushleft{\begin{hindi}
धर्म के विषय में कोई ज़बरदस्ती नहीं। सही बात नासमझी की बात से अलग होकर स्पष्ट हो गई है। तो अब जो कोई बढ़े हुए सरकश को ठुकरा दे और अल्लाह पर ईमान लाए, उसने ऐसा मज़बूत सहारा थाम लिया जो कभी टूटनेवाला नहीं। अल्लाह सब कुछ सुनने, जाननेवाला है
\end{hindi}}
\flushright{\begin{Arabic}
\quranayah[2][257]
\end{Arabic}}
\flushleft{\begin{hindi}
जो लोग ईमान लाते है, अल्लाह उनका रक्षक और सहायक है। वह उन्हें अँधेरों से निकालकर प्रकाश की ओर ले जाता है। रहे वे लोग जिन्होंने इनकार किया तो उनके संरक्षक बढ़े हुए सरकश है। वे उन्हें प्रकाश से निकालकर अँधेरों की ओर ले जाते है। वही आग (जहन्नम) में पड़नेवाले है। वे उसी में सदैव रहेंगे
\end{hindi}}
\flushright{\begin{Arabic}
\quranayah[2][258]
\end{Arabic}}
\flushleft{\begin{hindi}
क्या तुमने उनको नहीं देखा, जिसने इबराहीम से उसके 'रब' के सिलसिले में झगड़ा किया था, इस कारण कि अल्लाह ने उसको राज्य दे रखा था? जब इबराहीम ने कहा, "मेरा 'रब' वह है जो जिलाता और मारता है।" उसने कहा, "मैं भी तो जिलाता और मारता हूँ।" इबराहीम ने कहा, "अच्छा तो अल्लाह सूर्य को पूरब से लाता है, तो तू उसे पश्चिम से ले आ।" इसपर वह अधर्मी चकित रह गया। अल्लाह ज़ालिम लोगों को सीधा मार्ग नहीं दिखाता
\end{hindi}}
\flushright{\begin{Arabic}
\quranayah[2][259]
\end{Arabic}}
\flushleft{\begin{hindi}
या उस जैसे (व्यक्ति) को नहीं देखा, जिसका एक ऐसी बस्ती पर से गुज़र हुआ, जो अपनी छतों के बल गिरी हुई थी। उसने कहा, "अल्लाह इसके विनष्ट हो जाने के पश्चात इसे किस प्रकार जीवन प्रदान करेगा?" तो अल्लाह ने उसे सौ वर्ष की मृत्यु दे दी, फिर उसे उठा खड़ा किया। कहा, "तू कितनी अवधि तक इस अवस्था नें रहा।" उसने कहा, "मैं एक या दिन का कुछ हिस्सा रहा।" कहा, "नहीं, बल्कि तू सौ वर्ष रहा है। अब अपने खाने और पीने की चीज़ों को देख ले, उन पर समय का कोई प्रभाव नहीं, और अपने गधे को भी देख, और यह इसलिए कह रहे है ताकि हम तुझे लोगों के लिए एक निशानी बना दें और हड्डियों को देख कि किस प्रकार हम उन्हें उभारते है, फिर, उनपर माँस चढ़ाते है।" तो जब वास्तविकता उस पर प्रकट हो गई तो वह पुकार उठा, " मैं जानता हूँ कि अल्लाह को हर चीज़ की सामर्थ्य प्राप्त है।"
\end{hindi}}
\flushright{\begin{Arabic}
\quranayah[2][260]
\end{Arabic}}
\flushleft{\begin{hindi}
और याद करो जब इबराहीम ने कहा, "ऐ मेरे रब! मुझे दिखा दे, तू मुर्दों को कैसे जीवित करेगा?" कहा," क्या तुझे विश्वास नहीं?" उसने कहा,"क्यों नहीं, किन्तु निवेदन इसलिए है कि मेरा दिल संतुष्ट हो जाए।" कहा, "अच्छा, तो चार पक्षी ले, फिर उन्हें अपने साथ भली-भाँति हिला-मिला से, फिर उनमें से प्रत्येक को एक-एक पर्वत पर रख दे, फिर उनको पुकार, वे तेरे पास लपककर आएँगे। और जान ले कि अल्लाह अत्यन्त प्रभुत्वशाली, तत्वदर्शी है।"
\end{hindi}}
\flushright{\begin{Arabic}
\quranayah[2][261]
\end{Arabic}}
\flushleft{\begin{hindi}
जो लोग अपने माल अल्लाह के मार्ग में ख़र्च करते है, उनकी उपमा ऐसी है, जैसे एक दाना हो, जिससे सात बालें निकलें और प्रत्येक बाल में सौ दाने हो। अल्लाह जिसे चाहता है बढ़ोतरी प्रदान करता है। अल्लाह बड़ी समाईवाला, जाननेवाला है
\end{hindi}}
\flushright{\begin{Arabic}
\quranayah[2][262]
\end{Arabic}}
\flushleft{\begin{hindi}
जो लोग अपने माल अल्लाह के मार्ग में ख़र्च करते है, फिर ख़र्च करके उसका न एहसान जताते है और न दिल दुखाते है, उनका बदला उनके अपने रब के पास है। और न तो उनके लिए कोई भय होगा और न वे दुखी होंगे
\end{hindi}}
\flushright{\begin{Arabic}
\quranayah[2][263]
\end{Arabic}}
\flushleft{\begin{hindi}
एक भली बात कहनी और क्षमा से काम लेना उस सदक़े से अच्छा है, जिसके पीछे दुख हो। और अल्लाह अत्यन्कृत निस्पृह (बेनियाज़), सहनशील है
\end{hindi}}
\flushright{\begin{Arabic}
\quranayah[2][264]
\end{Arabic}}
\flushleft{\begin{hindi}
ऐ ईमानवालो! अपने सदक़ो को एहसान जताकर और दुख देकर उस व्यक्ति की तरह नष्ट न करो जो लोगों को दिखाने के लिए अपना माल ख़र्च करता है और अल्लाह और अंतिम दिन पर ईमान नहीं रखता। तो उसकी हालत उस चट्टान जैसी है जिसपर कुछ मिट्टी पड़ी हुई थी, फिर उस पर ज़ोर की वर्षा हुई और उसे साफ़ चट्टान की दशा में छोड़ गई। ऐसे लोग अपनी कमाई कुछ भी प्राप्त नहीं करते। और अल्लाह इनकार की नीति अपनानेवालों को मार्ग नहीं दिखाता
\end{hindi}}
\flushright{\begin{Arabic}
\quranayah[2][265]
\end{Arabic}}
\flushleft{\begin{hindi}
और जो लोग अपने माल अल्लाह की प्रसन्नता के संसाधनों की तलब में और अपने दिलों को जमाव प्रदान करने के कारण ख़र्च करते है उनकी हालत उस बाग़़ की तरह है जो किसी अच्छी और उर्वर भूमि पर हो। उस पर घोर वर्षा हुई तो उसमें दुगुने फल आए। फिर यदि घोर वर्षा उस पर नहीं हुई, तो फुहार ही पर्याप्त होगी। तुम जो कुछ भी करते हो अल्लाह उसे देख रहा है
\end{hindi}}
\flushright{\begin{Arabic}
\quranayah[2][266]
\end{Arabic}}
\flushleft{\begin{hindi}
क्या तुममें से कोई यह चाहेगा कि उसके पास ख़जूरों और अंगूरों का एक बाग़ हो, जिसके नीचे नहरें बह रही हो, वहाँ उसे हर प्रकार के फल प्राप्त हो और उसका बुढ़ापा आ गया हो और उसके बच्चे अभी कमज़ोर ही हों कि उस बाग़ पर एक आग भरा बगूला आ गया, और वह जलकर रह गया? इस प्रकार अल्लाह तुम्हारे सामने आयतें खोल-खोलकर बयान करता है, ताकि सोच-विचार करो
\end{hindi}}
\flushright{\begin{Arabic}
\quranayah[2][267]
\end{Arabic}}
\flushleft{\begin{hindi}
ऐ ईमान लानेवालो! अपनी कमाई को पाक और अच्छी चीज़ों में से ख़र्च करो और उन चीज़ों में से भी जो हमने धरती से तुम्हारे लिए निकाली है। और देने के लिए उसके ख़राब हिस्से (के देने) का इरादा न करो, जबकि तुम स्वयं उसे कभी न लोगे। यह और बात है कि उसको लेने में देखी-अनदेखी कर जाओ। और जान लो कि अल्लाह निस्पृह, प्रशंसनीय है
\end{hindi}}
\flushright{\begin{Arabic}
\quranayah[2][268]
\end{Arabic}}
\flushleft{\begin{hindi}
शैतान तुम्हें निर्धनता से डराता है और निर्लज्जता के कामों पर उभारता है, जबकि अल्लाह अपनी क्षमा और उदार कृपा का तुम्हें वचन देता है। अल्लाह बड़ी समाईवाला, सर्वज्ञ है
\end{hindi}}
\flushright{\begin{Arabic}
\quranayah[2][269]
\end{Arabic}}
\flushleft{\begin{hindi}
वह जिसे चाहता है तत्वदर्शिता प्रदान करता है और जिसे तत्वदर्शिता प्राप्त हुई उसे बड़ी दौलत मिल गई। किन्तु चेतते वही है जो बुद्धि और समझवाले है
\end{hindi}}
\flushright{\begin{Arabic}
\quranayah[2][270]
\end{Arabic}}
\flushleft{\begin{hindi}
और तुमने जो कुछ भी ख़र्च किया और जो कुछ भी नज़र (मन्नत) की हो, निस्सन्देह अल्लाह उसे भली-भाँति जानता है। और अत्याचारियों का कोई सहायक न होगा
\end{hindi}}
\flushright{\begin{Arabic}
\quranayah[2][271]
\end{Arabic}}
\flushleft{\begin{hindi}
यदि तुम खुले रूप मे सदक़े दो तो यह भी अच्छा है और यदि उनको छिपाकर मुहताजों को दो तो यह तुम्हारे लिए अधिक अच्छा है। और यह तुम्हारे कितने ही गुनाहों को मिटा देगा। और अल्लाह को उसकी पूरी ख़बर है, जो कुछ तुम करते हो
\end{hindi}}
\flushright{\begin{Arabic}
\quranayah[2][272]
\end{Arabic}}
\flushleft{\begin{hindi}
उन्हें मार्ग पर ला देने का दायित्व तुम पर नहीं है, बल्कि अल्लाह ही जिसे चाहता है मार्ग दिखाता है। और जो कुछ भी माल तुम ख़र्च करोगे, वह तुम्हारे अपने ही भले के लिए होगा और तुम अल्लाह के (बताए हुए) उद्देश्य के अतिरिक्त किसी और उद्देश्य से ख़र्च न करो। और जो माल भी तुम्हें तुम ख़र्च करोगे, वह पूरा-पूरा तुम्हें चुका दिया जाएगा और तुम्हारा हक़ न मारा जाएगा
\end{hindi}}
\flushright{\begin{Arabic}
\quranayah[2][273]
\end{Arabic}}
\flushleft{\begin{hindi}
यह उन मुहताजों के लिए है जो अल्लाह के मार्ग में घिर गए कि धरती में (जीविकोपार्जन के लिए) कोई दौड़-धूप नहीं कर सकते। उनके स्वाभिमान के कारण अपरिचित व्यक्ति उन्हें धनवान समझता है। तुम उन्हें उनके लक्षणो से पहचान सकते हो। वे लिपटकर लोगों से नहीं माँगते। जो माल भी तुम ख़र्च करोगे, वह अल्लाह को ज्ञात होगा
\end{hindi}}
\flushright{\begin{Arabic}
\quranayah[2][274]
\end{Arabic}}
\flushleft{\begin{hindi}
जो लोग अपने माल रात-दिन छिपे और खुले ख़र्च करें, उनका बदला तो उनके रब के पास है, और न उन्हें कोई भय है और न वे शोकाकुल होंगे
\end{hindi}}
\flushright{\begin{Arabic}
\quranayah[2][275]
\end{Arabic}}
\flushleft{\begin{hindi}
और लोग ब्याज खाते है, वे बस इस प्रकार उठते है जिस प्रकार वह क्यक्ति उठता है जिसे शैतान ने छूकर बावला कर दिया हो और यह इसलिए कि उनका कहना है, "व्यापार भी तो ब्याज के सदृश है," जबकि अल्लाह ने व्यापार को वैध और ब्याज को अवैध ठहराया है। अतः जिसको उसके रब की ओर से नसीहत पहुँची और वह बाज़ आ गया, तो जो कुछ पहले ले चुका वह उसी का रहा और मामला उसका अल्लाह के हवाले है। और जिसने फिर यही कर्म किया तो ऐसे ही लोग आग (जहन्नम) में पड़नेवाले है। उसमें वे सदैव रहेंगे
\end{hindi}}
\flushright{\begin{Arabic}
\quranayah[2][276]
\end{Arabic}}
\flushleft{\begin{hindi}
अल्लाह ब्याज को घटाता और मिटाता है और सदक़ों को बढ़ाता है। और अल्लाह किसी अकृतज्ञ, हक़ मारनेवाले को पसन्द नहीं करता
\end{hindi}}
\flushright{\begin{Arabic}
\quranayah[2][277]
\end{Arabic}}
\flushleft{\begin{hindi}
निस्संदेह जो लोग ईमान लाए और उन्होंने अच्छे कर्म किए और नमाज़ क़ायम की्य और ज़कात दी, उनके लिए उनका बदला उनके रब के पास है, और उन्हें न कोई भय हो और न वे शोकाकुल होंगे
\end{hindi}}
\flushright{\begin{Arabic}
\quranayah[2][278]
\end{Arabic}}
\flushleft{\begin{hindi}
ऐ ईमान लानेवालो! अल्लाह का डर रखो और जो कुछ ब्याज बाक़ी रह गया है उसे छोड़ दो, यदि तुम ईमानवाले हो
\end{hindi}}
\flushright{\begin{Arabic}
\quranayah[2][279]
\end{Arabic}}
\flushleft{\begin{hindi}
फिर यदि तुमने ऐसा न किया तो अल्लाह और उसके रसूल से युद्ध के लिए ख़बरदार हो जाओ। और यदि तौबा कर लो तो अपना मूलधन लेने का तुम्हें अधिकार है। न तुम अन्याय करो और न तुम्हारे साथ अन्याय किया जाए
\end{hindi}}
\flushright{\begin{Arabic}
\quranayah[2][280]
\end{Arabic}}
\flushleft{\begin{hindi}
और यदि कोई तंगी में हो तो हाथ खुलने तक मुहलत देनी होगी; और सदक़ा कर दो (अर्थात मूलधन भी न लो) तो यह तुम्हारे लिए अधिक उत्तम है, यदि तुम जान सको
\end{hindi}}
\flushright{\begin{Arabic}
\quranayah[2][281]
\end{Arabic}}
\flushleft{\begin{hindi}
और उस दिन का डर रखो जबकि तुम अल्लाह की ओर लौटोगे, फिर प्रत्येक व्यक्ति को जो कुछ उसने कमाया पूरा-पूरा मिल जाएगा और उनके साथ कदापि कोई अन्याय न होगा
\end{hindi}}
\flushright{\begin{Arabic}
\quranayah[2][282]
\end{Arabic}}
\flushleft{\begin{hindi}
ऐ ईमान लानेवालो! जब किसी निश्चित अवधि के लिए आपस में ऋण का लेन-देन करो तो उसे लिख लिया करो और चाहिए कि कोई लिखनेवाला तुम्हारे बीच न्यायपूर्वक (दस्तावेज़) लिख दे। और लिखनेवाला लिखने से इनकार न करे; जिस प्रकार अल्लाह ने उसे सिखाया है, उसी प्रकार वह दूसरों के लिए लिखने के काम आए और बोलकर वह लिखाए जिसके ज़िम्मे हक़ की अदायगी हो। और उसे अल्लाह का, जो उसका रब है, डर रखना चाहिए और उसमें कोई कमी न करनी चाहिए। फिर यदि वह व्यक्ति जिसके ज़िम्मे हक़ की अदायगी हो, कम समझ या कमज़ोर हो या वह बोलकर न लिखा सकता हो तो उसके संरक्षक को चाहिए कि न्यायपूर्वक बोलकर लिखा दे। और अपने पुरुषों में से दो गवाहो को गवाह बना लो और यदि दो पुरुष न हों तो एक पुरुष और दो स्त्रियाँ, जिन्हें तुम गवाह के लिए पसन्द करो, गवाह हो जाएँ (दो स्त्रियाँ इसलिए रखी गई है) ताकि यदि एक भूल जाए तो दूसरी उसे याद दिला दे। और गवाहों को जब बुलाया जाए तो आने से इनकार न करें। मामला चाहे छोटा हो या बड़ा एक निर्धारित अवधि तक के लिए है, तो उसे लिखने में सुस्ती से काम न लो। यह अल्लाह की स्पष्ट से अधिक न्यायसंगत बात है और इससे गवाही भी अधिक ठीक रहती है। और इससे अधि	क संभावना है कि तुम किसी संदेह में नहीं पड़ोगे। हाँ, यदि कोई सौदा नक़द हो, जिसका लेन-देन तुम आपस में कर रहे हो, तो तुम्हारे उसके न लिखने में तुम्हारे लिए कोई दोष नहीं। और जब आपम में क्रय-विक्रय का मामला करो तो उस समय भी गवाह कर लिया करो, और न किसी लिखनेवाले को हानि पहुँचाए जाए और न किसी गवाह को। और यदि ऐसा करोगे तो यह तुम्हारे लिए अवज्ञा की बात होगी। और अल्लाह का डर रखो। अल्लाह तुम्हें शिक्षा दे रहा है। और अल्लाह हर चीज़ को जानता है
\end{hindi}}
\flushright{\begin{Arabic}
\quranayah[2][283]
\end{Arabic}}
\flushleft{\begin{hindi}
और यदि तुम किसी सफ़र में हो और किसी लिखनेवाले को न पा सको, तो गिरवी रखकर मामला करो। फिर यदि तुममें से एक-दूसरे पर भरोसा के, तो जिस पर भरोसा किया है उसे चाहिए कि वह यह सच कर दिखाए कि वह विश्वासपात्र है और अल्लाह का, जो उसका रब है, डर रखे। और गवाही को न छिपाओ। जो उसे छिपाता है तो निश्चय ही उसका दिल गुनाहगार है, और तुम जो कुछ करते हो अल्लाह उसे भली-भाँति जानता है
\end{hindi}}
\flushright{\begin{Arabic}
\quranayah[2][284]
\end{Arabic}}
\flushleft{\begin{hindi}
अल्लाह ही का है जो कुछ आकाशों में है और जो कुछ धरती में है। और जो कुछ तुम्हारे मन है, यदि तुम उसे व्यक्त करो या छिपाओं, अल्लाह तुमसे उसका हिसाब लेगा। फिर वह जिसे चाहे क्षमा कर दे और जिसे चाहे यातना दे। अल्लाह को हर चीज़ की सामर्थ्य प्राप्त है
\end{hindi}}
\flushright{\begin{Arabic}
\quranayah[2][285]
\end{Arabic}}
\flushleft{\begin{hindi}
रसूल उसपर, जो कुछ उसके रब की ओर से उसकी ओर उतरा, ईमान लाया और ईमानवाले भी, प्रत्येक, अल्लाह पर, उसके फ़रिश्तों पर, उसकी किताबों पर और उसके रसूलों पर ईमान लाया। (और उनका कहना यह है,) "हम उसके रसूलों में से किसी को दूसरे रसूलों से अलग नहीं करते।" और उनका कहना है, "हमने सुना और आज्ञाकारी हुए। हमारे रब! हम तेरी क्षमा के इच्छुक है और तेरी ही ओर लौटना है।"
\end{hindi}}
\flushright{\begin{Arabic}
\quranayah[2][286]
\end{Arabic}}
\flushleft{\begin{hindi}
अल्लाह किसी जीव पर बस उसकी सामर्थ्य और समाई के अनुसार ही दायित्व का भार डालता है। उसका है जो उसने कमाया और उसी पर उसका वबाल (आपदा) भी है जो उसने किया। "हमारे रब! यदि हम भूलें या चूक जाएँ तो हमें न पकड़ना। हमारे रब! और हम पर ऐसा बोझ न डाल जैसा तूने हमसे पहले के लोगों पर डाला था। हमारे रब! और हमसे वह बोझ न उठवा, जिसकी हमें शक्ति नहीं। और हमें क्षमा कर और हमें ढाँक ले, और हमपर दया कर। तू ही हमारा संरक्षक है, अतएव इनकार करनेवालों के मुक़ाबले में हमारी सहायता कर।"
\end{hindi}}
\chapter{Al-'Imran (The Family of Amran)}
\begin{Arabic}
\Huge{\centerline{\basmalah}}\end{Arabic}
\flushright{\begin{Arabic}
\quranayah[3][1]
\end{Arabic}}
\flushleft{\begin{hindi}
अलीफ़॰ लाम॰ मीम॰
\end{hindi}}
\flushright{\begin{Arabic}
\quranayah[3][2]
\end{Arabic}}
\flushleft{\begin{hindi}
अल्लाह ही पूज्य हैं, उसके सिवा कोई पूज्य नहीं। वह जीवन्त हैं, सबको सँम्भालने और क़ायम रखनेवाला
\end{hindi}}
\flushright{\begin{Arabic}
\quranayah[3][3]
\end{Arabic}}
\flushleft{\begin{hindi}
उसने तुमपर हक़ के साथ किताब उतारी जो पहले की (किताबों की) पुष्टि करती हैं, और उसने तौरात और इंजील उतारी
\end{hindi}}
\flushright{\begin{Arabic}
\quranayah[3][4]
\end{Arabic}}
\flushleft{\begin{hindi}
इससे पहले लोगों के मार्गदर्शन के लिए और उसने कसौटी भी उतारी। निस्संदेह जिन लोगों ने अल्लाह की आयतों का इनकार किया उनके लिए कठोर यातना हैं और अल्लाह प्रभुत्वशाली भी हैं और (बुराई का) बदला लेनेवाला भी
\end{hindi}}
\flushright{\begin{Arabic}
\quranayah[3][5]
\end{Arabic}}
\flushleft{\begin{hindi}
निस्संदेह अल्लाह से कोई चीज़ न धरती में छिपी हैं और न आकाश में
\end{hindi}}
\flushright{\begin{Arabic}
\quranayah[3][6]
\end{Arabic}}
\flushleft{\begin{hindi}
वही हैं जो गर्भाशयों में, जैसा चाहता हैं, तुम्हारा रूप देता हैं। उस प्रभुत्वशाली, तत्वदर्शी के अतिरिक्त कोई पूज्य-प्रभु नहीं
\end{hindi}}
\flushright{\begin{Arabic}
\quranayah[3][7]
\end{Arabic}}
\flushleft{\begin{hindi}
वही हैं जिसने तुमपर अपनी ओर से किताब उतारी, वे सुदृढ़ आयतें हैं जो किताब का मूल और सारगर्भित रूप हैं और दूसरी उपलक्षित, तो जिन लोगों के दिलों में टेढ़ हैं वे फ़ितना (गुमराही) का तलाश और उसके आशय और परिणाम की चाह में उसका अनुसरण करते हैं जो उपलक्षित हैं। जबकि उनका परिणाम बस अल्लाह ही जानता हैं, और वे जो ज्ञान में पक्के हैं, वे कहते हैं, "हम उसपर ईमान लाए, जो हर एक हमारे रब ही की ओर से हैं।" और चेतते तो केवल वही हैं जो बुद्धि और समझ रखते हैं
\end{hindi}}
\flushright{\begin{Arabic}
\quranayah[3][8]
\end{Arabic}}
\flushleft{\begin{hindi}
हमारे रब! जब तू हमें सीधे मार्ग पर लगा चुका है तो इसके पश्चात हमारे दिलों में टेढ़ न पैदा कर और हमें अपने पास से दयालुता प्रदान कर। निश्चय ही तू बड़ा दाता है
\end{hindi}}
\flushright{\begin{Arabic}
\quranayah[3][9]
\end{Arabic}}
\flushleft{\begin{hindi}
हमारे रब! तू लोगों को एक दिन इकट्ठा करने वाला है, जिसमें कोई संदेह नही। निस्सन्देह अल्लाह अपने वचन के विरुद्ध जाने वाला नही है
\end{hindi}}
\flushright{\begin{Arabic}
\quranayah[3][10]
\end{Arabic}}
\flushleft{\begin{hindi}
जिन लोगों ने इनकार की नीति अपनाई है अल्लाह के मुकाबले में तो न उसके माल उनके कुछ काम आएँगे और न उनकी संतान ही। और वही हैं जो आग (जहन्नम) का ईधन बनकर रहेंगे
\end{hindi}}
\flushright{\begin{Arabic}
\quranayah[3][11]
\end{Arabic}}
\flushleft{\begin{hindi}
जैसे फ़िरऔन के लोगों और उनसे पहले के लोगों का हाल हुआ। उन्होंने हमारी आयतों को झुठलाया तो अल्लाह ने उन्हें उनके गुनाहों पर पकड़ लिया। और अल्लाह कठोर दंड देनेवाला है
\end{hindi}}
\flushright{\begin{Arabic}
\quranayah[3][12]
\end{Arabic}}
\flushleft{\begin{hindi}
इनकार करनेवालों से कह दो, "शीघ्र ही तुम पराभूत होगे और जहन्नम की ओर हाँके जाओगं। और वह क्या ही बुरा ठिकाना है।"
\end{hindi}}
\flushright{\begin{Arabic}
\quranayah[3][13]
\end{Arabic}}
\flushleft{\begin{hindi}
तुम्हारे लिए उन दोनों गिरोहों में एक निशानी है जो (बद्र की) लड़ाई में एक-दूसरे के मुक़ाबिल हुए। एक गिरोह अल्लाह के मार्ग में लड़ रहा था, जबकि दूसरा विधर्मी था। ये अपनी आँखों देख रहे थे कि वे उनसे दुगने है। अल्लाह अपनी सहायता से जिसे चाहता है, शक्ति प्रदान करता है। दृष्टिवान लोगों के लिए इसमें बड़ी शिक्षा-सामग्री है
\end{hindi}}
\flushright{\begin{Arabic}
\quranayah[3][14]
\end{Arabic}}
\flushleft{\begin{hindi}
मनुष्यों को चाहत की चीजों से प्रेम शोभायमान प्रतीत होता है कि वे स्त्रिमयाँ, बेटे, सोने-चाँदी के ढेर और निशान लगे (चुने हुए) घोड़े हैं और चौपाए और खेती। यह सब सांसारिक जीवन की सामग्री है और अल्लाह के पास ही अच्छा ठिकाना है
\end{hindi}}
\flushright{\begin{Arabic}
\quranayah[3][15]
\end{Arabic}}
\flushleft{\begin{hindi}
कहो, "क्या मैं तुम्हें इनसे उत्तम चीज का पता दूँ?" जो लोग अल्लाह का डर रखेंगे उनके लिए उनके रब के पास बाग़ है, जिनके नीचे नहरें बह रहीं होगी। उनमें वे सदैव रहेंगे। वहाँ पाक-साफ़ जोड़े होंगे और अल्लाह की प्रसन्नता प्राप्त होगी। और अल्लाह अपने बन्दों पर नज़र रखता है
\end{hindi}}
\flushright{\begin{Arabic}
\quranayah[3][16]
\end{Arabic}}
\flushleft{\begin{hindi}
ये वे लोग है जो कहते है, "हमारे रब हम ईमान लाए है। अतः हमारे गुनाहों को क्षमा कर दे और हमें आग (जहन्नम) की यातना से बचा ले।"
\end{hindi}}
\flushright{\begin{Arabic}
\quranayah[3][17]
\end{Arabic}}
\flushleft{\begin{hindi}
ये लोग धैर्य से काम लेनेवाले, सत्यवान और अत्यन्त आज्ञाकारी है, ये ((अल्लाह के मार्ग में) खर्च करते और रात की अंतिम घड़ियों में क्षमा की प्रार्थनाएँ करते हैं
\end{hindi}}
\flushright{\begin{Arabic}
\quranayah[3][18]
\end{Arabic}}
\flushleft{\begin{hindi}
अल्लाह ने गवाही दी कि उसके सिवा कोई पूज्य नहीं; और फ़रिश्तों ने और उन लोगों ने भी जो न्याय और संतुलन स्थापित करनेवाली एक सत्ता को जानते है। उस प्रभुत्वशाली, तत्वदर्शी के सिवा कोई पूज्य नहीं
\end{hindi}}
\flushright{\begin{Arabic}
\quranayah[3][19]
\end{Arabic}}
\flushleft{\begin{hindi}
दीन (धर्म) तो अल्लाह की स्पष्ट में इस्लाम ही है। जिन्हें किताब दी गई थी, उन्होंने तो इसमें इसके पश्चात विभेद किया कि ज्ञान उनके पास आ चुका था। ऐसा उन्होंने परस्पर दुराग्रह के कारण किया। जो अल्लाह की आयतों का इनकार करेगा तो अल्लाह भी जल्द हिसाब लेनेवाला है
\end{hindi}}
\flushright{\begin{Arabic}
\quranayah[3][20]
\end{Arabic}}
\flushleft{\begin{hindi}
अब यदि वे तुमसे झगड़े तो कह दो, "मैंने और मेरे अनुयायियों ने तो अपने आपको अल्लाह के हवाले कर दिया हैं।" और जिन्हें किताब मिली थी और जिनके पास किताब नहीं है, उनसे कहो, "क्या तुम भी इस्लाम को अपनाते हो?" यदि वे इस्लाम को अंगीकार कर लें तो सीधा मार्ग पर गए। और यदि मुँह मोड़े तो तुमपर केवल (संदेश) पहुँचा देने की ज़िम्मेदारी है। और अल्लाह स्वयं बन्दों को देख रहा है
\end{hindi}}
\flushright{\begin{Arabic}
\quranayah[3][21]
\end{Arabic}}
\flushleft{\begin{hindi}
जो लोग अल्लाह की आयतों का इनकार करें और नबियों को नाहक क़त्ल करे और उन लोगों का क़ल्त करें जो न्याय के पालन करने को कहें, उनको दुखद यातना की मंगल सूचना दे दो
\end{hindi}}
\flushright{\begin{Arabic}
\quranayah[3][22]
\end{Arabic}}
\flushleft{\begin{hindi}
यही लोग हैं, जिनके कर्म दुनिया और आख़िरत में अकारथ गए और उनका सहायक कोई भी नहीं
\end{hindi}}
\flushright{\begin{Arabic}
\quranayah[3][23]
\end{Arabic}}
\flushleft{\begin{hindi}
क्या तुमने उन लोगों को नहीं देखा जिन्हें ईश-ग्रंथ का एक हिस्सा प्रदान हुआ। उन्हें अल्लाह की किताब की ओर बुलाया जाता है कि वह उनके बीच निर्णय करे, फिर भी उनका एक गिरोह (उसकी) उपेक्षा करते हुए मुँह फेर लेता है?
\end{hindi}}
\flushright{\begin{Arabic}
\quranayah[3][24]
\end{Arabic}}
\flushleft{\begin{hindi}
यह इसलिए कि वे कहते, "आग हमें नहीं छू सकती। हाँ, कुछ गिने-चुने दिनों (के कष्टों) की बात और है।" उनकी मनघड़ंत बातों ने, जो वे घड़ते रहे हैं, उन्हें धोखे में डाल रखा है
\end{hindi}}
\flushright{\begin{Arabic}
\quranayah[3][25]
\end{Arabic}}
\flushleft{\begin{hindi}
फिर क्या हाल होगा, जब हम उन्हें उस दिन इकट्ठा करेंगे, जिसके आने में कोई संदेह नहीं और प्रत्येक व्यक्ति को, जो कुछ उसने कमाया होगा, पूरा-पूरा मिल जाएगा; और उनके साथ अन्याय न होगा
\end{hindi}}
\flushright{\begin{Arabic}
\quranayah[3][26]
\end{Arabic}}
\flushleft{\begin{hindi}
कहो, "ऐ अल्लाह, राज्य के स्वामी! तू जिसे चाहे राज्य दे और जिससे चाहे राज्य छीन ले, और जिसे चाहे इज़्ज़त (प्रभुत्व) प्रदान करे और जिसको चाहे अपमानित कर दे। तेरे ही हाथ में भलाई है। निस्संदेह तुझे हर चीज़ की सामर्थ्य प्राप्त है
\end{hindi}}
\flushright{\begin{Arabic}
\quranayah[3][27]
\end{Arabic}}
\flushleft{\begin{hindi}
"तू रात को दिन में पिरोता है और दिन को रात में पिरोता है। तू निर्जीव से सजीव को निकालता है और सजीव से निर्जीव को निकालता है, बेहिसाब देता है।"
\end{hindi}}
\flushright{\begin{Arabic}
\quranayah[3][28]
\end{Arabic}}
\flushleft{\begin{hindi}
ईमानवालों को चाहिए कि वे ईमानवालों से हटकर इनकारवालों को अपना मित्र (राज़दार) न बनाएँ, और जो ऐसा करेगा, उसका अल्लाह से कोई सम्बन्ध नहीं, क्योंकि उससे सम्बद्ध यही बात है कि तुम उनसे बचो, जिस प्रकार वे तुमसे बचते है। और अल्लाह तुम्हें अपने आपसे डराता है, और अल्लाह ही की ओर लौटना है
\end{hindi}}
\flushright{\begin{Arabic}
\quranayah[3][29]
\end{Arabic}}
\flushleft{\begin{hindi}
कह दो, "यदि तुम अपने दिलों की बात छिपाओ या उसे प्रकट करो, प्रत्येक दशा में अल्लाह उसे जान लेगा। और वह उसे भी जानता है, जो कुछ आकाशों में है और जो कुछ धरती में है। और अल्लाह को हर चीज़ की सामर्थ्य प्राप्त है।"
\end{hindi}}
\flushright{\begin{Arabic}
\quranayah[3][30]
\end{Arabic}}
\flushleft{\begin{hindi}
जिस दिन प्रत्येक व्यक्ति अपनी की हुई भलाई और अपनी की हुई बुराई को सामने मौजूद पाएगा, वह कामना करेगा कि काश! उसके और उस दिन के बीच बहुत दूर का फ़ासला होता। और अल्लाह तुम्हें अपना भय दिलाता है, और वह अपने बन्दों के लिए अत्यन्त करुणामय है
\end{hindi}}
\flushright{\begin{Arabic}
\quranayah[3][31]
\end{Arabic}}
\flushleft{\begin{hindi}
कह दो, "यदि तुम अल्लाह से प्रेम करते हो तो मेरा अनुसरण करो, अल्लाह भी तुमसे प्रेम करेगा और तुम्हारे गुनाहों को क्षमा कर देगा। अल्लाह बड़ा क्षमाशील, दयावान है।"
\end{hindi}}
\flushright{\begin{Arabic}
\quranayah[3][32]
\end{Arabic}}
\flushleft{\begin{hindi}
कह दो, "अल्लाह और रसूल का आज्ञापालन करो।" फिर यदि वे मुँह मोड़े तो अल्लाह भी इनकार करनेवालों से प्रेम नहीं करता
\end{hindi}}
\flushright{\begin{Arabic}
\quranayah[3][33]
\end{Arabic}}
\flushleft{\begin{hindi}
अल्लाह ने आदम, नूह, इबराहीम की सन्तान और इमरान की सन्तान को सारे संसार की अपेक्षा प्राथमिकता देकर चुना
\end{hindi}}
\flushright{\begin{Arabic}
\quranayah[3][34]
\end{Arabic}}
\flushleft{\begin{hindi}
एक नस्त के रूप में, उसमें से एक पीढ़ी, दूसरी पीढ़ी से पैदा हुई। अल्लाह सब कुछ सुनता, जानता है
\end{hindi}}
\flushright{\begin{Arabic}
\quranayah[3][35]
\end{Arabic}}
\flushleft{\begin{hindi}
याद करो जब इमरान की स्त्री ने कहा, "मेरे रब! जो बच्चा मेरे पेट में है उसे मैंने हर चीज़ से छुड़ाकर भेट स्वरूप तुझे अर्पित किया। अतः तू उसे मेरी ओर से स्वीकार कर। निस्संदेह तू सब कुछ सुनता, जानता है।"
\end{hindi}}
\flushright{\begin{Arabic}
\quranayah[3][36]
\end{Arabic}}
\flushleft{\begin{hindi}
फिर जब उसके यहाँ बच्ची पैदा हुई तो उसने कहा, "मेरे रब! मेरे यहाँ तो लड़की पैदा हुई है।" - अल्लाह तो जानता ही था जो कुछ उसके यहाँ पैदा हुआ था। और वह लड़का उस लडकी की तरह नहीं हो सकता - "और मैंने उसका नाम मरयम रखा है और मैं उसे और उसकी सन्तान को तिरस्कृत शैतान (के उपद्रव) से सुरक्षित रखने के लिए तेरी शरण में देती हूँ।"
\end{hindi}}
\flushright{\begin{Arabic}
\quranayah[3][37]
\end{Arabic}}
\flushleft{\begin{hindi}
अतः उसके रब ने उसका अच्छी स्वीकृति के साथ स्वागत किया और उत्तम रूप में उसे परवान चढ़ाया; और ज़करिया को उसका संरक्षक बनाया। जब कभी ज़करिया उसके पास मेहराब (इबादतगाह) में जाता, तो उसके पास कुछ रोज़ी पाता। उसने कहा, "ऐ मरयम! ये चीज़े तुझे कहाँ से मिलती है?" उसने कहा, "यह अल्लाह के पास से है।" निस्संदेह अल्लाह जिसे चाहता है, बेहिसाब देता है
\end{hindi}}
\flushright{\begin{Arabic}
\quranayah[3][38]
\end{Arabic}}
\flushleft{\begin{hindi}
वही ज़करिया ने अपने रब को पुकारा, कहा, "मेरे रब! मुझे तू अपने पास से अच्छी सन्तान (अनुयायी) प्रदान कर। तू ही प्रार्थना का सुननेवाला है।"
\end{hindi}}
\flushright{\begin{Arabic}
\quranayah[3][39]
\end{Arabic}}
\flushleft{\begin{hindi}
तो फ़रिश्तों ने उसे आवाज़ दी, जबकि वह मेहराब में खड़ा नमाज़ पढ़ रहा था, "अल्लाह, तुझे यह्याि की शुभ-सूचना देता है, जो अल्लाह के एक कलिमें की पुष्टि करनेवाला, सरदार, अत्यन्त संयमी और अच्छे लोगो में से एक नबी होगा।"
\end{hindi}}
\flushright{\begin{Arabic}
\quranayah[3][40]
\end{Arabic}}
\flushleft{\begin{hindi}
उसने कहा, "मेरे रब! मेरे यहाँ लड़का कैसे पैदा होगा, जबकि मुझे बुढापा आ गया है और मेरी पत्ऩी बाँझ है?" कहा, "इसी प्रकार अल्लाह जो चाहता है, करता है।"
\end{hindi}}
\flushright{\begin{Arabic}
\quranayah[3][41]
\end{Arabic}}
\flushleft{\begin{hindi}
उसने कहा, "मेरे रब! मेरे लिए कोई आदेश निश्चित कर दे।" कहा, "तुम्हारे लिए आदेश यह है कि तुम लोगों से तीन दिन तक संकेत के सिवा कोई बातचीत न करो। अपने रब को बहुत अधिक याद करो और सायंकाल और प्रातः समय उसकी तसबीह (महिमागान) करते रहो।"
\end{hindi}}
\flushright{\begin{Arabic}
\quranayah[3][42]
\end{Arabic}}
\flushleft{\begin{hindi}
और जब फ़रिश्तों ने कहा, "ऐ मरयम! अल्लाह ने तुझे चुन लिया और तुझे पवित्रता प्रदान की और तुझे संसार की स्त्रियों के मुक़ाबले मं चुन लिया
\end{hindi}}
\flushright{\begin{Arabic}
\quranayah[3][43]
\end{Arabic}}
\flushleft{\begin{hindi}
"ऐ मरयम! पूरी निष्ठा के साथ अपने रब की आज्ञा का पालन करती रह, और सजदा कर और झुकनेवालों के साथ तू भी झूकती रह।"
\end{hindi}}
\flushright{\begin{Arabic}
\quranayah[3][44]
\end{Arabic}}
\flushleft{\begin{hindi}
यह परोक्ष की सूचनाओं में से है, जिसकी वह्य हम तुम्हारी ओर कर रहे है। तुम तो उस समय उनके पास नहीं थे, जब वे अपनी क़लमों को फेंक रहे थ कि उनमें कौन मरयम का संरक्षक बने और न उनके समय थे, जब वे आपस में झगड़ रहे थे
\end{hindi}}
\flushright{\begin{Arabic}
\quranayah[3][45]
\end{Arabic}}
\flushleft{\begin{hindi}
ओर याद करो जब फ़रिश्तों ने कहा, "ऐ मरयम! अल्लाह तुझे अपने एक कलिमे (बात) की शुभ-सूचना देता है, जिसका नाम मसीह, मरयम का बेटा, ईसा होगा। वह दुनिया और आख़िरत मे आबरूवाला होगा और अल्लाह के निकटवर्ती लोगों में से होगा
\end{hindi}}
\flushright{\begin{Arabic}
\quranayah[3][46]
\end{Arabic}}
\flushleft{\begin{hindi}
वह लोगों से पालने में भी बात करेगा और बड़ी आयु को पहुँचकर भी। और वह नेक व्यक्ति होगा। -
\end{hindi}}
\flushright{\begin{Arabic}
\quranayah[3][47]
\end{Arabic}}
\flushleft{\begin{hindi}
वह बोली, "मेरे रब! मेरे यहाँ लड़का कहाँ से होगा, जबकि मुझे किसी आदमी ने छुआ तक नहीं?" कहा, "ऐसा ही होगा, अल्लाह जो चाहता है, पैदा करता है। जब वह किसी कार्य का निर्णय करता है तो उसको बस यही कहता है 'हो जा' तो वह हो जाता है
\end{hindi}}
\flushright{\begin{Arabic}
\quranayah[3][48]
\end{Arabic}}
\flushleft{\begin{hindi}
"और उसको किताब, हिकमत, तौरात और इंजील का भी ज्ञान देगा
\end{hindi}}
\flushright{\begin{Arabic}
\quranayah[3][49]
\end{Arabic}}
\flushleft{\begin{hindi}
"और उसे इसराईल की संतान की ओर रसूल बनाकर भेजेगा। (वह कहेगा) कि मैं तुम्हारे पास तुम्हारे रब की ओर से एक निशाली लेकर आया हूँ कि मैं तुम्हारे लिए मिट्टी से पक्षी के रूप जैसी आकृति बनाता हूँ, फिर उसमें फूँक मारता हूँ, तो वह अल्लाह के आदेश से उड़ने लगती है। और मैं अल्लाह के आदेश से अंधे और कोढ़ी को अच्छा कर देता हूँ और मुर्दे को जीवित कर देता हूँ। और मैं तुम्हें बता देता हूँ जो कुछ तुम खाते हो और जो कुछ अपने घरों में इकट्ठा करके रखते हो। निस्संदेह इसमें तुम्हारे लिए एक निशानी है, यदि तुम माननेवाले हो
\end{hindi}}
\flushright{\begin{Arabic}
\quranayah[3][50]
\end{Arabic}}
\flushleft{\begin{hindi}
"और मैं तौरात की, जो मेरे आगे है, पुष्टि करता हूँ और इसलिए आया हूँ कि तुम्हारे लिए कुछ उन चीज़ों को हलाल कर दूँ जो तुम्हारे लिए हराम थी। और मैं तुम्हारे पास तुम्हारे रब की ओर से एक निशानी लेकर आया हूँ। अतः अल्लाह का डर रखो और मेरी आज्ञा का पालन करो
\end{hindi}}
\flushright{\begin{Arabic}
\quranayah[3][51]
\end{Arabic}}
\flushleft{\begin{hindi}
"निस्संदेह अल्लाह मेरी भी रब है और तुम्हारा रब भी, अतः तुम उसी की बन्दगी करो। यही सीधा मार्ग है।"
\end{hindi}}
\flushright{\begin{Arabic}
\quranayah[3][52]
\end{Arabic}}
\flushleft{\begin{hindi}
फिर जब ईसा को उनके अविश्वास और इनकार का आभास हुआ तो उसने कहा, "कौन अल्लाह की ओर बढ़ने में मेरा सहायक होता है?" हवारियों (साथियों) ने कहा, "हम अल्लाह के सहायक हैं। हम अल्लाह पर ईमान लाए और गवाह रहिए कि हम मुस्लिम है
\end{hindi}}
\flushright{\begin{Arabic}
\quranayah[3][53]
\end{Arabic}}
\flushleft{\begin{hindi}
"हमारे रब! तूने जो कुछ उतारा है, हम उसपर ईमान लाए और इस रसूल का अनुसरण स्वीकार किया। अतः तू हमें गवाही देनेवालों में लिख ले।"
\end{hindi}}
\flushright{\begin{Arabic}
\quranayah[3][54]
\end{Arabic}}
\flushleft{\begin{hindi}
और वे चाल चले तो अल्लाह ने भी उसका तोड़ किया और अल्लाह उत्तम तोड़ करनेवाला है
\end{hindi}}
\flushright{\begin{Arabic}
\quranayah[3][55]
\end{Arabic}}
\flushleft{\begin{hindi}
जब अल्लाह ने कहा, "ऐ ईसा! मैं तुझे अपने क़ब्जे में ले लूँगा और तुझे अपनी ओर उठा लूँगा और अविश्वासियों (की कुचेष्टाओं) से तुझे पाक कर दूँगा और तेरे अनुयायियों को क़ियामत के दिन तक लोगों के ऊपर रखूँगा, जिन्होंने इनकार किया। फिर मेरी ओर तुम्हें लौटना है। फिर मैं तुम्हारे बीच उन चीज़ों का फ़ैसला कर दूँगा, जिनके विषय में तुम विभेद करते रहे हो
\end{hindi}}
\flushright{\begin{Arabic}
\quranayah[3][56]
\end{Arabic}}
\flushleft{\begin{hindi}
"तो जिन लोगों ने इनकार की नीति अपनाई, उन्हें दुनिया और आख़िरत में कड़ी यातना दूँगा। उनका कोई सहायक न होगा।"
\end{hindi}}
\flushright{\begin{Arabic}
\quranayah[3][57]
\end{Arabic}}
\flushleft{\begin{hindi}
रहे वे लोग जो ईमान लाए और उन्होंने अच्छे कर्म किए उन्हें वह उनका पूरा-पूरा बदला देगा। अल्लाह अत्याचारियों से प्रेम नहीं करता
\end{hindi}}
\flushright{\begin{Arabic}
\quranayah[3][58]
\end{Arabic}}
\flushleft{\begin{hindi}
ये आयतें है और हिकमत (तत्वज्ञान) से परिपूर्ण अनुस्मारक, जो हम तुम्हें सुना रहे हैं
\end{hindi}}
\flushright{\begin{Arabic}
\quranayah[3][59]
\end{Arabic}}
\flushleft{\begin{hindi}
निस्संदेह अल्लाह की दृष्टि में ईसा की मिसाल आदम जैसी है कि उसे मिट्टी से बनाया, फिर उससे कहा, "हो जा", तो वह हो जाता है
\end{hindi}}
\flushright{\begin{Arabic}
\quranayah[3][60]
\end{Arabic}}
\flushleft{\begin{hindi}
यह हक़ तुम्हारे रब की ओर से हैं, तो तुम संदेह में न पड़ना
\end{hindi}}
\flushright{\begin{Arabic}
\quranayah[3][61]
\end{Arabic}}
\flushleft{\begin{hindi}
अब इसके पश्चात कि तुम्हारे पास ज्ञान आ चुका है, कोई तुमसे इस विषय में कुतर्क करे तो कह दो, "आओ, हम अपने बेटों को बुला लें और तुम भी अपने बेटों को बुला लो, और हम अपनी स्त्रियों को बुला लें और तुम भी अपनी स्त्रियों को बुला लो, और हम अपने को और तुम अपने को ले आओ, फिर मिलकर प्रार्थना करें और झूठों पर अल्लाह की लानत भेजे।"
\end{hindi}}
\flushright{\begin{Arabic}
\quranayah[3][62]
\end{Arabic}}
\flushleft{\begin{hindi}
निस्संदेह यही सच्चा बयान है और अल्लाह के अतिरिक्त कोई पूज्य नहीं। और अल्लाह ही प्रभुत्वशाली, तत्वदर्शी है
\end{hindi}}
\flushright{\begin{Arabic}
\quranayah[3][63]
\end{Arabic}}
\flushleft{\begin{hindi}
फिर यदि वे लोग मुँह मोड़े तो अल्लाह फ़सादियों को भली-भाँति जानता है
\end{hindi}}
\flushright{\begin{Arabic}
\quranayah[3][64]
\end{Arabic}}
\flushleft{\begin{hindi}
कहो, "ऐ किताबवालो! आओ एक ऐसी बात की ओर जिसे हमारे और तुम्हारे बीच समान मान्यता प्राप्त है; यह कि हम अल्लाह के अतिरिक्त किसी की बन्दगी न करें और न उसके साथ किसी चीज़ को साझी ठहराएँ और न परस्पर हममें से कोई एक-दूसरे को अल्लाह से हटकर रब बनाए।" फिर यदि वे मुँह मोड़े तो कह दो, "गवाह रहो, हम तो मुस्लिम (आज्ञाकारी) है।"
\end{hindi}}
\flushright{\begin{Arabic}
\quranayah[3][65]
\end{Arabic}}
\flushleft{\begin{hindi}
"ऐ किताबवालो! तुम इबराहीम के विषय में हमसे क्यों झगड़ते हो? जबकि तौरात और इंजील तो उसके पश्चात उतारी गई है, तो क्या तुम समझ से काम नहीं लेते?
\end{hindi}}
\flushright{\begin{Arabic}
\quranayah[3][66]
\end{Arabic}}
\flushleft{\begin{hindi}
"ये तुम लोग हो कि उसके विषय में वाद-विवाद कर चुके जिसका तुम्हें कुछ ज्ञान था। अब उसके विषय में क्यों वाद-विवाद करते हो, जिसके विषय में तुम्हें कुछ भी ज्ञान नहीं? अल्लाह जानता है, तुम नहीं जानते"
\end{hindi}}
\flushright{\begin{Arabic}
\quranayah[3][67]
\end{Arabic}}
\flushleft{\begin{hindi}
इबराहीम न यहूदी था और न ईसाई, बल्कि वह तो एक ओर को होकर रहनेवाला मुस्लिम (आज्ञाकारी) था। वह कदापि मुशरिकों में से न था
\end{hindi}}
\flushright{\begin{Arabic}
\quranayah[3][68]
\end{Arabic}}
\flushleft{\begin{hindi}
निस्संदेह इबराहीम से सबसे अधिक निकटता का सम्बन्ध रखनेवाले वे लोग है जिन्होंने उसका अनुसरण किया, और यह नबी और ईमानवाले लोग। और अल्लाह ईमानवालों को समर्थक एवं सहायक है
\end{hindi}}
\flushright{\begin{Arabic}
\quranayah[3][69]
\end{Arabic}}
\flushleft{\begin{hindi}
किताबवालों में से एक गिरोह के लोगों की कामना है कि काश! वे तुम्हें पथभ्रष्ट कर सकें, जबकि वे केवल अपने-आपकों पथभ्रष्ट कर रहे है! किन्तु उन्हें इसका एहसास नहीं
\end{hindi}}
\flushright{\begin{Arabic}
\quranayah[3][70]
\end{Arabic}}
\flushleft{\begin{hindi}
ऐ किताबवालों! तुम अल्लाह की आयतों का इनकार क्यों करते हो, जबकि तुम स्वयं गवाह हो?
\end{hindi}}
\flushright{\begin{Arabic}
\quranayah[3][71]
\end{Arabic}}
\flushleft{\begin{hindi}
ऐ किताबवालो! सत्य को असत्य के साथ क्यों गड्ड-मड्ड करते और जानते-बूझते हुए सत्य को छिपाते हो?
\end{hindi}}
\flushright{\begin{Arabic}
\quranayah[3][72]
\end{Arabic}}
\flushleft{\begin{hindi}
किताबवालों में से एक गिरोह कहता है, "ईमानवालो पर जो कुछ उतरा है, उस पर प्रातःकाल ईमान लाओ और संध्या समय उसका इनकार कर दो, ताकि वे फिर जाएँ
\end{hindi}}
\flushright{\begin{Arabic}
\quranayah[3][73]
\end{Arabic}}
\flushleft{\begin{hindi}
"और तुम अपने धर्म के अनुयायियों के अतिरिक्त किसी पर विश्वास न करो। कह दो, वास्तविक मार्गदर्शन तो अल्लाह का मार्गदर्शन है - कि कहीं जो चीज़ तुम्हें प्राप्त हो जाए, या वे तुम्हारे रब के सामने तुम्हारे ख़िलाफ़ हुज्जत कर सकें।" कह दो, "बढ़-चढ़कर प्रदान करना तो अल्लाह के हाथ में है, जिसे चाहता है प्रदान करता है। और अल्लाह बड़ी समाईवाला, सब कुछ जाननेवाला है
\end{hindi}}
\flushright{\begin{Arabic}
\quranayah[3][74]
\end{Arabic}}
\flushleft{\begin{hindi}
"वह जिसे चाहता है अपनी रहमत (दयालुता) के लिए ख़ास कर लेता है। और अल्लाह बड़ी उदारता दर्शानेवाला है।"
\end{hindi}}
\flushright{\begin{Arabic}
\quranayah[3][75]
\end{Arabic}}
\flushleft{\begin{hindi}
और किताबवालों में कोई तो ऐसा है कि यदि तुम उसके पास धन-दौलच का एक ढेर भी अमानत रख दो तो वह उसे तुम्हें लौटा देगा। और उनमें कोई ऐसा है कि यदि तुम एक दीनार भी उसकी अमानत में रखों, तो जब तक कि तुम उसके सिर पर सवार न हो, वह उसे तुम्हें अदा नहीं करेगा। यह इसलिए कि वे कहते है, "उन लोगों के विषय में जो किताबवाले नहीं हैं हमारी कोई पकड़ नहीं।" और वे जानते-बूझते अल्लाह पर झूठ मढ़ते है
\end{hindi}}
\flushright{\begin{Arabic}
\quranayah[3][76]
\end{Arabic}}
\flushleft{\begin{hindi}
क्यों नहीं, जो कोई अपनी प्रतिज्ञा पूरी करेगा और डर रखेगा, तो अल्लाह भी डर रखनेवालों से प्रेम करता है
\end{hindi}}
\flushright{\begin{Arabic}
\quranayah[3][77]
\end{Arabic}}
\flushleft{\begin{hindi}
रहे वे लोग जो अल्लाह की प्रतिज्ञा और अपनी क़समों का थोड़े मूल्य पर सौदा करते हैं, उनका आख़िरत में कोई हिस्सा नहीं। अल्लाह न तो उनसे बात करेगा और न क़ियामत के दिन उनकी ओर देखेगा, और न ही उन्हें निखारेगा। उनके लिए तो दुखद यातना है
\end{hindi}}
\flushright{\begin{Arabic}
\quranayah[3][78]
\end{Arabic}}
\flushleft{\begin{hindi}
उनमें कुछ लोग ऐसे है जो किताब पढ़ते हुए अपनी ज़बानों का इस प्रकार उलट-फेर करते है कि तुम समझों कि वह किताब ही में से है, जबकि वह किताब में से नहीं होता। और वे कहते है, "यह अल्लाह की ओर से है।" जबकि वह अल्लाह की ओर से नहीं होता। और वे जानते-बूझते झूठ गढ़कर अल्लाह पर थोपते है
\end{hindi}}
\flushright{\begin{Arabic}
\quranayah[3][79]
\end{Arabic}}
\flushleft{\begin{hindi}
किसी मनुष्य के लिए यह सम्भव न था कि अल्लाह उसे किताब और हिकमत (तत्वदर्शिता) और पैग़म्बरी प्रदान करे और वह लोगों से कहने लगे, "तुम अल्लाह को छोड़कर मेरे उपासक बनो।" बल्कि वह तो यही कहेगा कि, "तुम रबवाले बनो, इसलिए कि तुम किताब की शिक्षा देते हो और इसलिए कि तुम स्वयं भी पढ़ते हो।"
\end{hindi}}
\flushright{\begin{Arabic}
\quranayah[3][80]
\end{Arabic}}
\flushleft{\begin{hindi}
और न वह तुम्हें इस बात का हुक्म देगा कि तुम फ़रिश्तों और नबियों को अपना रब बना लो। क्या वह तुम्हें अधर्म का हुक्म देगा, जबकि तुम (उसके) आज्ञाकारी हो?
\end{hindi}}
\flushright{\begin{Arabic}
\quranayah[3][81]
\end{Arabic}}
\flushleft{\begin{hindi}
और याद करो जब अल्लाह ने नबियों के सम्बन्ध में वचन लिया था, "मैंने तुम्हें जो कुछ किताब और हिकमत प्रदान की, इसके पश्चात तुम्हारे पास कोई रसूल उसकी पुष्टि करता हुआ आए जो तुम्हारे पास मौजूद है, तो तुम अवश्य उस पर ईमान लाओगे और निश्चय ही उसकी सहायता करोगे।" कहा, "क्या तुमने इक़रार किया? और इसपर मेरी ओर से डाली हुई जिम्मेदारी को बोझ उठाया?" उन्होंने कहा, "हमने इक़रार किया।" कहा, "अच्छा तो गवाह किया और मैं भी तुम्हारे साथ गवाह हूँ।"
\end{hindi}}
\flushright{\begin{Arabic}
\quranayah[3][82]
\end{Arabic}}
\flushleft{\begin{hindi}
फिर इसके बाद जो फिर गए, तो ऐसे ही लोग अवज्ञाकारी है
\end{hindi}}
\flushright{\begin{Arabic}
\quranayah[3][83]
\end{Arabic}}
\flushleft{\begin{hindi}
अब क्या इन लोगों को अल्लाह के दीन (धर्म) के सिवा किसी और दीन की तलब है, हालाँकि आकाशों और धरती में जो कोई भी है, स्वेच्छापूर्वक या विवश होकर उसी के आगे झुका हुआ है। और उसी की ओर सबको लौटना है?
\end{hindi}}
\flushright{\begin{Arabic}
\quranayah[3][84]
\end{Arabic}}
\flushleft{\begin{hindi}
कहो, "हम तो अल्लाह पर और उस चीज़ पर ईमान लाए जो हम पर उतरी है, और जो इबराहीम, इसमाईल, इसहाक़ और याकूब़ और उनकी सन्तान पर उतरी उसपर भी, और जो मूसा और ईसा और दूसरे नबियो को उनके रब की ओर से प्रदान हुई (उसपर भी हम ईमान रखते है) । हम उनमें से किसी को उस ओर से प्रदान हुई (उसपर भी हम ईमान रखते है) । हम उनमें से किसी को उस सम्बन्ध से अलग नहीं करते जो उनके बीच पाया जाता है, और हम उसी के आज्ञाकारी (मुस्लिम) है।"
\end{hindi}}
\flushright{\begin{Arabic}
\quranayah[3][85]
\end{Arabic}}
\flushleft{\begin{hindi}
जो इस्लाम के अतिरिक्त कोई और दीन (धर्म) तलब करेगा तो उसकी ओर से कुछ भी स्वीकार न किया जाएगा। और आख़िरत में वह घाटा उठानेवालों में से होगा
\end{hindi}}
\flushright{\begin{Arabic}
\quranayah[3][86]
\end{Arabic}}
\flushleft{\begin{hindi}
अल्लाह उन लोगों को कैसे मार्ग दिखाएगा, जिन्होंने अपने ईमान के पश्चात अधर्म और इनकार की नीति अपनाई, जबकि वे स्वयं इस बात की गवाही दे चुके हैं कि यह रसूल सच्चा है और उनके पास स्पष्ट निशानियाँ भी आ चुकी हैं? अल्लाह अत्याचारी लोगों को मार्ग नहीं दिखाया करता
\end{hindi}}
\flushright{\begin{Arabic}
\quranayah[3][87]
\end{Arabic}}
\flushleft{\begin{hindi}
उन लोगों का बदला यही है कि उनपर अल्लाह और फ़रिश्तों और सारे मनुष्यों की लानत है
\end{hindi}}
\flushright{\begin{Arabic}
\quranayah[3][88]
\end{Arabic}}
\flushleft{\begin{hindi}
इसी दशा में वे सदैव रहेंगे, न उनकी यातना हल्की होगी और न उन्हें मुहलत ही दी जाएगी
\end{hindi}}
\flushright{\begin{Arabic}
\quranayah[3][89]
\end{Arabic}}
\flushleft{\begin{hindi}
हाँ, जिन लोगों ने इसके पश्चात तौबा कर ली और अपनी नीति को सुधार लिया तो निस्संदेह अल्लाह बड़ा क्षमाशील, दयावान है
\end{hindi}}
\flushright{\begin{Arabic}
\quranayah[3][90]
\end{Arabic}}
\flushleft{\begin{hindi}
रहे वे लोग जिन्होंने अपने ईमान के पश्चात इनकार किया और अपने इनकार में बढ़ते ही गए, उनकी तौबा कदापि स्वीकार न होगी। वास्तव में वही पथभ्रष्ट हैं
\end{hindi}}
\flushright{\begin{Arabic}
\quranayah[3][91]
\end{Arabic}}
\flushleft{\begin{hindi}
निस्संदेह जिन लोगों ने इनकार किया और इनकार ही की दशा में मरे, तो उनमें किसी से धरती के बराबर सोना भी, यदि उसने प्राण-मुक्ति के लिए दिया हो, कदापि स्वीकार नहीं किया जाएगा। ऐसे लोगों के लिए दुखद यातना है और उनका कोई सहायक न होगा
\end{hindi}}
\flushright{\begin{Arabic}
\quranayah[3][92]
\end{Arabic}}
\flushleft{\begin{hindi}
तुम नेकी और वफ़ादारी के दर्जे को नहीं पहुँच सकते, जब तक कि उन चीज़ो को (अल्लाह के मार्ग में) ख़र्च न करो, जो तुम्हें प्रिय है। और जो चीज़ भी तुम ख़र्च करोगे, निश्चय ही अल्लाह को उसका ज्ञान होगा
\end{hindi}}
\flushright{\begin{Arabic}
\quranayah[3][93]
\end{Arabic}}
\flushleft{\begin{hindi}
खाने की सारी चीज़े इसराईल की संतान के लिए हलाल थी, सिवाय उन चीज़ों के जिन्हें तौरात के उतरने से पहले इसराईल ने स्वयं अपने हराम कर लिया था। कहो, "यदि तुम सच्चे हो तो तौरात लाओ और उसे पढ़ो।"
\end{hindi}}
\flushright{\begin{Arabic}
\quranayah[3][94]
\end{Arabic}}
\flushleft{\begin{hindi}
अब इसके पश्चात भी जो व्यक्ति झूठी बातें अल्लाह से जोड़े, तो ऐसे ही लोग अत्याचारी है
\end{hindi}}
\flushright{\begin{Arabic}
\quranayah[3][95]
\end{Arabic}}
\flushleft{\begin{hindi}
कहो, "अल्लाह ने सच कहा है; अतः इबराहीम के तरीक़े का अनुसरण करो, जो हर ओर से कटकर एक का हो गया था और मुशरिकों में से न था
\end{hindi}}
\flushright{\begin{Arabic}
\quranayah[3][96]
\end{Arabic}}
\flushleft{\begin{hindi}
"निस्ंसदेह इबादत के लिए पहला घर जो 'मानव के लिए' बनाया गया वहीं है जो मक्का में है, बरकतवाला और सर्वथा मार्गदर्शन, संसारवालों के लिए
\end{hindi}}
\flushright{\begin{Arabic}
\quranayah[3][97]
\end{Arabic}}
\flushleft{\begin{hindi}
"उसमें स्पष्ट निशानियाँ है, वह इबराहीम का स्थल है। और जिसने उसमें प्रवेश किया, वह निश्चिन्त हो गया। लोगों पर अल्लाह का हक़ है कि जिसको वहाँ तक पहुँचने की सामर्थ्य प्राप्त हो, वह इस घर का हज करे, और जिसने इनकार किया तो (इस इनकार से अल्लाह का कुछ नहीं बिगड़ता) अल्लाह तो सारे संसार से निरपेक्ष है।"
\end{hindi}}
\flushright{\begin{Arabic}
\quranayah[3][98]
\end{Arabic}}
\flushleft{\begin{hindi}
कहो, "ऐ किताबवालों! तुम अल्लाह की आयतों का इनकार क्यों करते हो, जबकि जो कुछ तुम कर रहे हो, अल्लाह की दृष्टिअ में है?"
\end{hindi}}
\flushright{\begin{Arabic}
\quranayah[3][99]
\end{Arabic}}
\flushleft{\begin{hindi}
कहो, "ऐ किताबवालो! तुम ईमान लानेवालों को अल्लाह के मार्ग से क्यो रोकते हो, तुम्हें उसमें किसी टेढ़ की तलाश रहती है, जबकि तुम भली-भाँति वास्तविकता से अवगत हो और जो कुछ तुम कर रहे हो, अल्लाह उससे बेख़बर नहीं है।"
\end{hindi}}
\flushright{\begin{Arabic}
\quranayah[3][100]
\end{Arabic}}
\flushleft{\begin{hindi}
ऐ ईमान लानेवालो! यदि तुमने उनके किसी गिरोह की बात माल ली, जिन्हें किताब मिली थी, तो वे तुम्हारे ईमान लाने के पश्चात फिर तुम्हें अधर्मी बना देंगे
\end{hindi}}
\flushright{\begin{Arabic}
\quranayah[3][101]
\end{Arabic}}
\flushleft{\begin{hindi}
अब तुम इनकार कैसे कर सकते हो, जबकि तुम्हें अल्लाह की आयतें पढ़कर सुनाई जा रही है और उसका रसूल तुम्हारे बीच मौजूद है? जो कोई अल्लाह को मज़बूती से पकड़ ले, वह सीधे मार्ग पर आ गया
\end{hindi}}
\flushright{\begin{Arabic}
\quranayah[3][102]
\end{Arabic}}
\flushleft{\begin{hindi}
ऐ ईमान लानेवालो! अल्लाह का डर रखो, जैसाकि उसका डर रखने का हक़ है। और तुम्हारी मृत्यु बस इस दशा में आए कि तुम मुस्लिम (आज्ञाकारी) हो
\end{hindi}}
\flushright{\begin{Arabic}
\quranayah[3][103]
\end{Arabic}}
\flushleft{\begin{hindi}
और सब मिलकर अल्लाह की रस्सी को मज़बूती से पकड़ लो और विभेद में न पड़ो। और अल्लाह की उस कृपा को याद करो जो तुमपर हुई। जब तुम आपस में एक-दूसरे के शत्रु थे तो उसने तुम्हारे दिलों को परस्पर जोड़ दिया और तुम उसकी कृपा से भाई-भाई बन गए। तुम आग के एक गड्ढे के किनारे खड़े थे, तो अल्लाह ने उससे तुम्हें बचा लिया। इस प्रकार अल्लाह तुम्हारे लिए अपनी आयते खोल-खोलकर बयान करता है, ताकि तुम मार्ग पा लो
\end{hindi}}
\flushright{\begin{Arabic}
\quranayah[3][104]
\end{Arabic}}
\flushleft{\begin{hindi}
और तुम्हें एक ऐसे समुदाय का रूप धारण कर लेना चाहिए जो नेकी की ओर बुलाए और भलाई का आदेश दे और बुराई से रोके। यही सफलता प्राप्त करनेवाले लोग है
\end{hindi}}
\flushright{\begin{Arabic}
\quranayah[3][105]
\end{Arabic}}
\flushleft{\begin{hindi}
तुम उन लोगों की तरह न हो जाना जो विभेद में पड़ गए, और इसके पश्चात कि उनके पास खुली निशानियाँ आ चुकी थी, वे विभेद में पड़ गए। ये वही लोग है, जिनके लिए बड़ी (घोर) यातना है। (यह यातना उस दिन होगी)
\end{hindi}}
\flushright{\begin{Arabic}
\quranayah[3][106]
\end{Arabic}}
\flushleft{\begin{hindi}
जिस दिन कितने ही चेहरे उज्ज्वल होंगे और कितने ही चेहरे काले पड़ जाएँगे, तो जिनके चेहेर काले पड़ गए होंगे (वे सदा यातना में ग्रस्त रहेंगे। खुली निशानियाँ आने का बाद जिन्होंने विभेद किया) उनसे कहा जाएगा, "क्या तुमने ईमान के पश्चात इनकार की नीति अपनाई? तो लो अब उस इनकार के बदले में जो तुम करते रहे हो, यातना का मज़ा चखो।"
\end{hindi}}
\flushright{\begin{Arabic}
\quranayah[3][107]
\end{Arabic}}
\flushleft{\begin{hindi}
रहे वे लोग जिनके चेहरे उज्ज्वल होंगे, वे अल्लाह की दयालुता की छाया में होंगे। वे उसी में सदैव रहेंगे
\end{hindi}}
\flushright{\begin{Arabic}
\quranayah[3][108]
\end{Arabic}}
\flushleft{\begin{hindi}
ये अल्लाह की आयतें है, जिन्हें हम हक़ के साथ तुम्हें सुना रहे है। अल्लाह संसारवालों पर किसी प्रकार का अत्याचार नहीं करना चाहता
\end{hindi}}
\flushright{\begin{Arabic}
\quranayah[3][109]
\end{Arabic}}
\flushleft{\begin{hindi}
आकाशों और धरती मे जो कुछ है अल्लाह ही का है, और सारे मामले अल्लाह ही की ओर लौटाए जाते है
\end{hindi}}
\flushright{\begin{Arabic}
\quranayah[3][110]
\end{Arabic}}
\flushleft{\begin{hindi}
तुम एक उत्तम समुदाय हो, जो लोगों के समक्ष लाया गया है। तुम नेकी का हुक्म देते हो और बुराई से रोकते हो और अल्लाह पर ईमान रखते हो। और यदि किताबवाले भी ईमान लाते तो उनके लिए यह अच्छा होता। उनमें ईमानवाले भी हैं, किन्तु उनमें अधिकतर लोग अवज्ञाकारी ही हैं
\end{hindi}}
\flushright{\begin{Arabic}
\quranayah[3][111]
\end{Arabic}}
\flushleft{\begin{hindi}
थोड़ा दुख पहुँचाने के अतिरिक्त वे तुम्हारा कुछ भी बिगाड़ नहीं सकते। और यदि वे तुमसे लड़ेंगे, तो तुम्हें पीठ दिखा जाएँगे, फिर उन्हें कोई सहायता भी न मिलेगी
\end{hindi}}
\flushright{\begin{Arabic}
\quranayah[3][112]
\end{Arabic}}
\flushleft{\begin{hindi}
वे जहाँ कहीं भी पाए गए उनपर ज़िल्लत (अपमान) थोप दी गई। किन्तु अल्लाह की रस्सी थामें या लोगों का रस्सी, तो और बात है। वे ल्लाह के प्रकोप के पात्र हुए और उनपर दशाहीनता थोप दी गई। यह इसलिए कि वे अल्लाह की आयतों का इनकार और नबियों को नाहक़ क़त्ल करते रहे है। और यह इसलिए कि उन्होंने अवज्ञा की और सीमोल्लंघन करते रहे
\end{hindi}}
\flushright{\begin{Arabic}
\quranayah[3][113]
\end{Arabic}}
\flushleft{\begin{hindi}
ये सब एक जैसे नहीं है। किताबवालों में से कुछ ऐसे लोग भी है जो सीधे मार्ग पर है और रात की घड़ियों में अल्लाह की आयतें पढ़ते है और वे सजदा करते रहनेवाले है
\end{hindi}}
\flushright{\begin{Arabic}
\quranayah[3][114]
\end{Arabic}}
\flushleft{\begin{hindi}
वे अल्लाह और अन्तिम दिन पर ईमान रखते है और नेकी का हुक्म देते और बुराई से रोकते है और नेक कामों में अग्रसर रहते है, और वे अच्छे लोगों में से है
\end{hindi}}
\flushright{\begin{Arabic}
\quranayah[3][115]
\end{Arabic}}
\flushleft{\begin{hindi}
जो नेकी भी वे करेंगे, उसकी अवमानना न होगी। अल्लाह का डर रखनेवालो से भली-भाँति परिचित है
\end{hindi}}
\flushright{\begin{Arabic}
\quranayah[3][116]
\end{Arabic}}
\flushleft{\begin{hindi}
रहे वे लोग जिन्होंने इनकार किया, तो अल्लाह के मुक़ाबले में न उनके माल कुछ काम आ सकेंगे और न उनकी सन्तान ही। वे तो आग में जानेवाले लोग है, उसी में वे सदैव रहेंगे
\end{hindi}}
\flushright{\begin{Arabic}
\quranayah[3][117]
\end{Arabic}}
\flushleft{\begin{hindi}
इस सांसारिक जीवन के लिए जो कुछ भी वे ख़र्च करते है, उसकी मिसाल उस वायु जैसी है जिसमें पाला हो और वह उन लोगों की खेती पर चल जाए, जिन्होंने अपने ऊपर अत्याचार नहीं किया, अपितु वे तो स्वयं अपने ऊपर अत्याचार कर रहे है
\end{hindi}}
\flushright{\begin{Arabic}
\quranayah[3][118]
\end{Arabic}}
\flushleft{\begin{hindi}
ऐ ईमान लानेवालो! अपनों को छोड़कर दूसरों को अपना अंतरंग मित्र न बनाओ, वे तुम्हें नुक़सान पहुँचाने में कोई कमी नहीं करते। जितनी भी तुम कठिनाई में पड़ो, वही उनको प्रिय है। उनका द्वेष तो उनके मुँह से व्यक्त हो चुका है और जो कुछ उनके सीने छिपाए हुए है, वह तो इससे भी बढ़कर है। यदि तुम बुद्धि से काम लो, तो हमने तुम्हारे लिए निशानियाँ खोलकर बयान कर दी हैं
\end{hindi}}
\flushright{\begin{Arabic}
\quranayah[3][119]
\end{Arabic}}
\flushleft{\begin{hindi}
ये चो तुम हो जो उनसे प्रेम करते हो और वे तुमसे प्रेम नहीं करते, जबकि तुम समस्त किताबों पर ईमान रखते हो। और वे जब तुमसे मिलते है तो कहने को तो कहते है कि "हम ईमान लाए है।" किन्तु जब वे अलग होते है तो तुमपर क्रोध के मारे दाँतों से उँगलियाँ काटने लगते है। कह दो, "तुम अपने क्रोध में आप मरो। निस्संदेह अल्लाह दिलों के भेद को जानता है।"
\end{hindi}}
\flushright{\begin{Arabic}
\quranayah[3][120]
\end{Arabic}}
\flushleft{\begin{hindi}
यदि तुम्हारा कोई भला होता है तो उन्हें बुरा लगता है। परन्तु यदि तुम्हें कोई अप्रिय बात पेश आती है तो उससे वे प्रसन्न हो जाते है। यदि तुमने धैर्य से काम लिया और (अल्लाह का) डर रखा, तो उनकी कोई चाल तुम्हें नुक़सान नहीं पहुँचा सकती। जो कुछ वे कर रहे है, अल्लाह ने उसे अपने धेरे में ले रखा है
\end{hindi}}
\flushright{\begin{Arabic}
\quranayah[3][121]
\end{Arabic}}
\flushleft{\begin{hindi}
याद करो जब तुम सवेरे अपने घर से निकलकर ईमानवालों को युद्ध के मोर्चों पर लगा रहे थे। - अल्लाह तो सब कुछ सुनता, जानता है
\end{hindi}}
\flushright{\begin{Arabic}
\quranayah[3][122]
\end{Arabic}}
\flushleft{\begin{hindi}
जब तुम्हारे दो गिरोहों ने साहस छोड़ देना चाहा, जबकि अल्लाह उनका संरक्षक मौजूद था - और ईमानवालों को तो अल्लाह ही पर भरोसा करना चाहिए
\end{hindi}}
\flushright{\begin{Arabic}
\quranayah[3][123]
\end{Arabic}}
\flushleft{\begin{hindi}
और बद्र में अल्लाह तुम्हारी सहायता कर भी चुका था, जबकि तुम बहुत कमज़ोर हालत में थे। अतः अल्लाह ही का डर रखो, ताकि तुम कृतज्ञ बनो
\end{hindi}}
\flushright{\begin{Arabic}
\quranayah[3][124]
\end{Arabic}}
\flushleft{\begin{hindi}
जब तुम ईमानवालों से कह रहे थे, "क्या यह तुम्हारे लिए काफ़ी नही हैं कि तुम्हारा रब तीन हज़ार फ़रिश्ते उतारकर तुम्हारी सहायता करे?"
\end{hindi}}
\flushright{\begin{Arabic}
\quranayah[3][125]
\end{Arabic}}
\flushleft{\begin{hindi}
हाँ, क्यों नहीं। यदि तुम धैर्य से काम लो और डर रखो, फिर शत्रु सहसा तुमपर चढ़ आएँ, उसी क्षण तुम्हारा रब पाँच हज़ार विध्वंशकारी फ़रिश्तों से तुम्हारी सहायता करेगा
\end{hindi}}
\flushright{\begin{Arabic}
\quranayah[3][126]
\end{Arabic}}
\flushleft{\begin{hindi}
अल्लाह ने तो इसे तुम्हारे लिए बस एक शुभ-सूचना बनाया और इसलिए कि तुम्हारे दिल सन्तुष्ट हो जाएँ - सहायता तो बस अल्लाह ही के यहाँ से आती है जो अत्यन्त प्रभुत्वशाली, तत्वदर्शी है
\end{hindi}}
\flushright{\begin{Arabic}
\quranayah[3][127]
\end{Arabic}}
\flushleft{\begin{hindi}
ताकि इनकार करनेवालों के एक हिस्से को काट डाले या उन्हें बुरी पराजित और अपमानित कर दे कि वे असफल होकर लौटें
\end{hindi}}
\flushright{\begin{Arabic}
\quranayah[3][128]
\end{Arabic}}
\flushleft{\begin{hindi}
तुम्हें इस मामले में कोई अधिकार नहीं - चाहे वह उसकी तौबा क़बूल करे या उन्हें यातना दे, क्योंकि वे अत्याचारी है
\end{hindi}}
\flushright{\begin{Arabic}
\quranayah[3][129]
\end{Arabic}}
\flushleft{\begin{hindi}
आकाशों और धरती में जो कुछ भी है, अल्लाह ही का है। वह जिसे चाहे क्षमा कर दे और जिसे चाहे यातना दे। और अल्लाह अत्यन्त क्षमाशील, दयावान है
\end{hindi}}
\flushright{\begin{Arabic}
\quranayah[3][130]
\end{Arabic}}
\flushleft{\begin{hindi}
ऐ ईमान लानेवालो! बढ़ोत्तरी के ध्येय से ब्याज न खाओ, जो कई गुना अधिक हो सकता है। और अल्लाह का डर रखो, ताकि तुम्हें सफलता प्राप्त हो
\end{hindi}}
\flushright{\begin{Arabic}
\quranayah[3][131]
\end{Arabic}}
\flushleft{\begin{hindi}
और उस आग से बचो जो इनकार करनेवालों के लिए तैयार है
\end{hindi}}
\flushright{\begin{Arabic}
\quranayah[3][132]
\end{Arabic}}
\flushleft{\begin{hindi}
और अल्लाह और रसूल के आज्ञाकारी बनो, ताकि तुमपर दया की जाए
\end{hindi}}
\flushright{\begin{Arabic}
\quranayah[3][133]
\end{Arabic}}
\flushleft{\begin{hindi}
और अपने रब की क्षमा और उस जन्नत की ओर बढ़ो, जिसका विस्तार आकाशों और धरती जैसा है। वह उन लोगों के लिए तैयार है जो डर रखते है
\end{hindi}}
\flushright{\begin{Arabic}
\quranayah[3][134]
\end{Arabic}}
\flushleft{\begin{hindi}
वे लोग जो ख़ुशहाली और तंगी की प्रत्येक अवस्था में ख़र्च करते रहते है और क्रोध को रोकते है और लोगों को क्षमा करते है - और अल्लाह को भी ऐसे लोग प्रिय है, जो अच्छे से अच्छा कर्म करते है
\end{hindi}}
\flushright{\begin{Arabic}
\quranayah[3][135]
\end{Arabic}}
\flushleft{\begin{hindi}
और जिनका हाल यह है कि जब वे कोई खुला गुनाह कर बैठते है या अपने आप पर ज़ुल्म करते है तौ तत्काल अल्लाह उन्हें याद आ जाता है और वे अपने गुनाहों की क्षमा चाहने लगते हैं - और अल्लाह के अतिरिक्त कौन है, जो गुनाहों को क्षमा कर सके? और जानते-बूझते वे अपने किए पर अड़े नहीं रहते
\end{hindi}}
\flushright{\begin{Arabic}
\quranayah[3][136]
\end{Arabic}}
\flushleft{\begin{hindi}
उनका बदला उनके रब की ओर से क्षमादान है और ऐसे बाग़ है जिनके नीचे नहरें बहती होंगी। उनमें वे सदैव रहेंगे। और क्या ही अच्छा बदला है अच्छे कर्म करनेवालों का
\end{hindi}}
\flushright{\begin{Arabic}
\quranayah[3][137]
\end{Arabic}}
\flushleft{\begin{hindi}
तुमसे पहले (धर्मविरोधियों के साथ अल्लाह की) रीति के कितने ही नमूने गुज़र चुके है, तो तुम धरती में चल-फिरकर देखो कि झुठलानेवालों का परिणाम हुआ है
\end{hindi}}
\flushright{\begin{Arabic}
\quranayah[3][138]
\end{Arabic}}
\flushleft{\begin{hindi}
यह लोगों के लिए स्पष्ट बयान और डर रखनेवालों के लिए मार्गदर्शन और उपदेश है
\end{hindi}}
\flushright{\begin{Arabic}
\quranayah[3][139]
\end{Arabic}}
\flushleft{\begin{hindi}
हताश न हो और दुखी न हो, यदि तुम ईमानवाले हो, तो तुम्हीं प्रभावी रहोगे
\end{hindi}}
\flushright{\begin{Arabic}
\quranayah[3][140]
\end{Arabic}}
\flushleft{\begin{hindi}
यदि तुम्हें आघात पहुँचे तो उन लोगों को भी ऐसा ही आघात पहुँच चुका है। ये युद्ध के दिन हैं, जिन्हें हम लोगों के बीच डालते ही रहते है और ऐसा इसलिए हुआ कि अल्लाह ईमानवालों को जान ले और तुममें से कुछ लोगों को गवाह बनाए - और अत्याचारी अल्लाह को प्रिय नहीं है
\end{hindi}}
\flushright{\begin{Arabic}
\quranayah[3][141]
\end{Arabic}}
\flushleft{\begin{hindi}
और ताकि अल्लाह ईमानवालों को निखार दे और इनकार करनेवालों को मिटा दे
\end{hindi}}
\flushright{\begin{Arabic}
\quranayah[3][142]
\end{Arabic}}
\flushleft{\begin{hindi}
क्या तुमने यह समझ रखा है कि जन्नत में यूँ ही प्रवेश करोगे, जबकि अल्लाह ने अभी उन्हें परखा ही नहीं जो तुममें जिहाद (सत्य-मार्ग में जानतोड़ कोशिश) करनेवाले है। - और दृढ़तापूर्वक जमें रहनेवाले है
\end{hindi}}
\flushright{\begin{Arabic}
\quranayah[3][143]
\end{Arabic}}
\flushleft{\begin{hindi}
और तुम तो मृत्यु की कामनाएँ कर रहे थे, जब तक कि वह तुम्हारे सामने नहीं आई थी। लो, अब तो वह तुम्हारे सामने आ गई और तुमने उसे अपनी आँखों से देख लिया
\end{hindi}}
\flushright{\begin{Arabic}
\quranayah[3][144]
\end{Arabic}}
\flushleft{\begin{hindi}
मुहम्मद तो बस एक रसूल है। उनसे पहले भी रसूल गुज़र चुके है। तो क्या यदि उनकी मृत्यु हो जाए या उनकी हत्या कर दी जाए तो तुम उल्टे पाँव फिर जाओगे? जो कोई उल्टे पाँव फिरेगा, वह अल्लाह का कुछ नहीं बिगाडेगा। और कृतज्ञ लोगों को अल्लाह बदला देगा
\end{hindi}}
\flushright{\begin{Arabic}
\quranayah[3][145]
\end{Arabic}}
\flushleft{\begin{hindi}
और अल्लाह की अनुज्ञा के बिना कोई व्यक्ति मर नहीं सकता। हर व्यक्ति एक लिखित निश्चित समय का अनुपालन कर रहा है। और जो कोई दुनिया का बदला चाहेगा, उसे हम इस दुनिया में से देंगे, जो आख़िरत का बदला चाहेगा, उसे हम उसमें से देंगे और जो कृतज्ञता दिखलाएँगे, उन्हें तो हम बदला देंगे ही
\end{hindi}}
\flushright{\begin{Arabic}
\quranayah[3][146]
\end{Arabic}}
\flushleft{\begin{hindi}
कितने ही नबी ऐसे गुज़रे है जिनके साथ होकर बहुत-से ईशभक्तों ने युद्ध किया, तो अल्लाह के मार्ग में जो मुसीबत उन्हें पहुँची उससे वे न तो हताश हुए और न उन्होंने कमज़ोरी दिखाई और न ऐसा हुआ कि वे दबे हो। और अल्लाह दृढ़तापूर्वक जमे रहनेवालों से प्रेम करता है
\end{hindi}}
\flushright{\begin{Arabic}
\quranayah[3][147]
\end{Arabic}}
\flushleft{\begin{hindi}
उन्होंने कुछ नहीं कहा सिवाय इसके कि "ऐ हमारे रब! तू हमारे गुनाहों को और हमारे अपने मामले में जो ज़्यादती हमसे हो गई हो, उसे क्षमा कर दे और हमारे क़दम जमाए रख, और इनकार करनेवाले लोगों के मुक़ाबले में हमारी सहायता कर।"
\end{hindi}}
\flushright{\begin{Arabic}
\quranayah[3][148]
\end{Arabic}}
\flushleft{\begin{hindi}
अतः अल्लाह ने उन्हें दुनिया का भी बदला दिया और आख़िरत का अच्छा बदला भी। और सत्कर्मी लोगों से अल्लाह प्रेम करता है
\end{hindi}}
\flushright{\begin{Arabic}
\quranayah[3][149]
\end{Arabic}}
\flushleft{\begin{hindi}
ऐ ईमान लानेवालो! यदि तुम उन लोगों के कहने पर चलोगे जिन्होंने इनकार का मार्ग अपनाया है, तो वे तुम्हें उल्टे पाँव फेर ले जाएँगे। फिर तुम घाटे में पड़ जाओगे
\end{hindi}}
\flushright{\begin{Arabic}
\quranayah[3][150]
\end{Arabic}}
\flushleft{\begin{hindi}
बल्कि अल्लाह ही तुम्हारा संरक्षक है; और वह सबसे अच्छा सहायक है
\end{hindi}}
\flushright{\begin{Arabic}
\quranayah[3][151]
\end{Arabic}}
\flushleft{\begin{hindi}
हम शीघ्र ही इनकार करनेवालों के दिलों में धाक बिठा देंगे, इसलिए कि उन्होंने ऐसी चीज़ो को अल्लाह का साक्षी ठहराया है जिनसे साथ उसने कोई सनद नहीं उतारी, और उनका ठिकाना आग (जहन्नम) है। और अत्याचारियों का क्या ही बुरा ठिकाना है
\end{hindi}}
\flushright{\begin{Arabic}
\quranayah[3][152]
\end{Arabic}}
\flushleft{\begin{hindi}
और अल्लाह ने तो तुम्हें अपना वादा सच्चा कर दिखाया, जबकि तुम उसकी अनुज्ञा से उन्हें क़त्ल कर रहे थे। यहाँ तक कि जब तुम स्वयं ढीले पड़ गए और काम में झगड़ा डाल दिया और अवज्ञा की, जबकि अल्लाह ने तुम्हें वह चीज़ दिखा दी थी जिसकी तुम्हें चाह थी। तुममें कुछ लोग दुनिया चाहते थे और कुछ आख़िरत के इच्छुक थे। फिर अल्लाह ने तुम्हें उनके मुक़ाबले से हटा दिया, ताकि वह तुम्हारी परीक्षा ले। फिर भी उसने तुम्हें क्षमा कर दिया, क्योंकि अल्लाह ईमानवालों के लिए बड़ा अनुग्राही है
\end{hindi}}
\flushright{\begin{Arabic}
\quranayah[3][153]
\end{Arabic}}
\flushleft{\begin{hindi}
जब तुम लोग दूर भागे चले जा रहे थे और किसी को मुड़कर देखते तक न थे और रसूल तुम्हें पुकार रहा था, जबकि वह तुम्हारी दूसरी टुकड़ी के साथ था (जो भागी नहीं), तो अल्लाह ने तुम्हें शोक पर शोक दिया, ताकि तुम्हारे हाथ से कोई चीज़ निकल जाए या तुमपर कोई मुसीबत आए तो तुम शोकाकुल न हो। और जो कुछ भी तुम करते हो, अल्लाह उसकी भली-भाँति ख़बर रखता है
\end{hindi}}
\flushright{\begin{Arabic}
\quranayah[3][154]
\end{Arabic}}
\flushleft{\begin{hindi}
फिर इस शोक के पश्चात उसने तुमपर एक शान्ति उतारी - एक निद्रा, जो तुममें से कुछ लोगों को घेर रही थी और कुछ लोग ऐसे भी थे जिन्हें अपने प्राणों की चिन्ता थी। वे अल्लाह के विषय में ऐसा ख़याल कर रहे थे, जो सत्य के सर्वथा प्रतिकूल, अज्ञान (काल) का ख़याल था। वे कहते थे, "इन मामलों में क्या हमारा भी कुछ अधिकार है?" कह दो, "मामले तो सबके सब अल्लाह के (हाथ में) हैं।" वे जो कुछ अपने दिलों में छिपाए रखते है, तुमपर ज़ाहिर नहीं करते। कहते है, "यदि इस मामले में हमारा भी कुछ अधिकार होता तो हम यहाँ मारे न जाते।" कह दो, "यदि तुम अपने घरों में भी होते, तो भी जिन लोगों का क़त्ल होना तय था, वे निकलकर अपने अन्तिम शयन-स्थलों कर पहुँचकर रहते।" और यह इसलिए भी था कि जो कुछ तुम्हारे सीनों में है, अल्लाह उसे परख ले और जो कुछ तुम्हारे दिलों में है उसे साफ़ कर दे। और अल्लाह दिलों का हाल भली-भाँति जानता है
\end{hindi}}
\flushright{\begin{Arabic}
\quranayah[3][155]
\end{Arabic}}
\flushleft{\begin{hindi}
तुममें से जो लोग दोनों गिरोहों की मुठभेड़ के दिन पीठ दिखा गए, उन्हें तो शैतान ही ने उनकी कुछ कमाई (कर्म) का कारण विचलित कर दिया था। और अल्लाह तो उन्हें क्षमा कर चुका है। निस्संदेह अल्लाह बड़ा क्षमा करनेवाला, सहनशील है
\end{hindi}}
\flushright{\begin{Arabic}
\quranayah[3][156]
\end{Arabic}}
\flushleft{\begin{hindi}
ऐ ईमान लानेवालो! उन लोगों की तरह न हो जाना जिन्होंने इनकार किया और अपने भाईयों के विषय में, जबकि वे सफ़र में गए हों या युद्ध में हो (और उनकी वहाँ मृत्यु हो जाए तो) कहते है, "यदि वे हमारे पास होते तो न मरते और न क़त्ल होते।" (ऐसी बातें तो इसलिए होती है) ताकि अल्लाह उनको उनके दिलों में घर करनेवाला पछतावा और सन्ताप बना दे। अल्लाह ही जीवन प्रदान करने और मृत्यु देनेवाला है। और तुम जो कुछ भी कर रहे हो वह अल्लाह की स्पष्ट में है
\end{hindi}}
\flushright{\begin{Arabic}
\quranayah[3][157]
\end{Arabic}}
\flushleft{\begin{hindi}
और यदि तुम अल्लाह के मार्ग में मारे गए या मर गए, तो अल्लाह का क्षमादान और उसकी दयालुता तो उससे कहीं उत्तम है, जिसके बटोरने में वे लगे हुए है
\end{hindi}}
\flushright{\begin{Arabic}
\quranayah[3][158]
\end{Arabic}}
\flushleft{\begin{hindi}
हाँ, यदि तुम मर गए या मारे गए, तो प्रत्येक दशा में तुम अल्लाह ही के पास इकट्ठा किए जाओगे
\end{hindi}}
\flushright{\begin{Arabic}
\quranayah[3][159]
\end{Arabic}}
\flushleft{\begin{hindi}
(तुमने तो अपनी दयालुता से उन्हें क्षमा कर दिया) तो अल्लाह की ओर से ही बड़ी दयालुता है जिसके कारण तुम उनके लिए नर्म रहे हो, यदि कहीं तुम स्वभाव के क्रूर और कठोर हृदय होते तो ये सब तुम्हारे पास से छँट जाते। अतः उन्हें क्षमा कर दो और उनके लिए क्षमा की प्रार्थना करो। और मामलों में उनसे परामर्श कर लिया करो। फिर जब तुम्हारे संकल्प किसी सम्मति पर सुदृढ़ हो जाएँ तो अल्लाह पर भरोसा करो। निस्संदेह अल्लाह को वे लोग प्रिय है जो उसपर भरोसा करते है
\end{hindi}}
\flushright{\begin{Arabic}
\quranayah[3][160]
\end{Arabic}}
\flushleft{\begin{hindi}
यदि अल्लाह तुम्हारी सहायता करे, तो कोई तुमपर प्रभावी नहीं हो सकता। और यदि वह तुम्हें छोड़ दे, तो फिर कौन हो जो उसके पश्चात तुम्हारी सहायता कर सके। अतः ईमानवालों को अल्लाह ही पर भरोसा रखना चाहिए
\end{hindi}}
\flushright{\begin{Arabic}
\quranayah[3][161]
\end{Arabic}}
\flushleft{\begin{hindi}
यह किसी नबी के लिए सम्भब नहीं कि वह दिल में कीना-कपट रखे, और जो कोई कीना-कपट रखेगा तो वह क़ियामत के दिन अपने द्वेष समेत हाज़िर होगा। और प्रत्येक व्यक्ति को उसकी कमाई का पूरा-पूरा बदला दे दिया जाएँगा और उनपर कुछ भी ज़ुल्म न होगा
\end{hindi}}
\flushright{\begin{Arabic}
\quranayah[3][162]
\end{Arabic}}
\flushleft{\begin{hindi}
भला क्या जो व्यक्ति अल्लाह की इच्छा पर चले वह उस जैसा हो सकता है जो अल्लाह के प्रकोप का भागी हो चुका हो और जिसका ठिकाना जहन्नम है? और वह क्या ही बुरा ठिकाना है
\end{hindi}}
\flushright{\begin{Arabic}
\quranayah[3][163]
\end{Arabic}}
\flushleft{\begin{hindi}
अल्लाह के यहाँ उनके विभिन्न दर्जे है और जो कुछ वे कर रहे है, अल्लाह की स्पष्ट में है
\end{hindi}}
\flushright{\begin{Arabic}
\quranayah[3][164]
\end{Arabic}}
\flushleft{\begin{hindi}
निस्संदेह अल्लाह ने ईमानवालों पर बड़ा उपकार किया, जबकि स्वयं उन्हीं में से एक ऐसा रसूल उठाया जो उन्हें आयतें सुनाता है और उन्हें निखारता है, और उन्हें किताब और हिक़मत (तत्वदर्शिता) का शिक्षा देता है, अन्यथा इससे पहले वे लोग खुली गुमराही में पड़े हुए थे
\end{hindi}}
\flushright{\begin{Arabic}
\quranayah[3][165]
\end{Arabic}}
\flushleft{\begin{hindi}
यह क्या कि जब तुम्हें एक मुसीबत पहुँची, जिसकी दोगुनी तुमने पहुँचाए, तो तुम कहने लगे कि, "यह कहाँ से आ गई?" कह दो, "यह तो तुम्हारी अपनी ओर से है, अल्लाह को हर चीज़ की सामर्थ्य प्राप्त है।"
\end{hindi}}
\flushright{\begin{Arabic}
\quranayah[3][166]
\end{Arabic}}
\flushleft{\begin{hindi}
और दोनों गिरोह की मुठभेड़ के दिन जो कुछ तुम्हारे सामने आया वह अल्लाह ही की अनुज्ञा से आया और इसलिए कि वह जान ले कि ईमानवाले कौन है
\end{hindi}}
\flushright{\begin{Arabic}
\quranayah[3][167]
\end{Arabic}}
\flushleft{\begin{hindi}
और इसलिए कि वह कपटाचारियों को भी जान ले जिनसे कहा गया कि "आओ, अल्लाह के मार्ग में युद्ध करो या दुश्मनों को हटाओ।" उन्होंने कहा, "यदि हम जानते कि लड़ाई होगी तो हम अवश्य तुम्हारे साथ हो लेते।" उस दिन वे ईमान की अपेक्षा अधर्म के अधिक निकट थे। वे अपने मुँह से वे बातें कहते है, जो उनके दिलों में नहीं होती। और जो कुछ वे छिपाते है, अल्लाह उसे भली-भाँति जानता है
\end{hindi}}
\flushright{\begin{Arabic}
\quranayah[3][168]
\end{Arabic}}
\flushleft{\begin{hindi}
ये वही लोग है जो स्वयं तो बैठे रहे और अपने भाइयों के विषय में कहने लगे, "यदि वे हमारी बात मान लेते तो मारे न जाते।" कह तो, "अच्छा, यदि तुम सच्चे हो, तो अब तुम अपने ऊपर से मृत्यु को टाल देना।"
\end{hindi}}
\flushright{\begin{Arabic}
\quranayah[3][169]
\end{Arabic}}
\flushleft{\begin{hindi}
तुम उन लोगों को जो अल्लाह के मार्ग में मारे गए है, मुर्दा न समझो, बल्कि वे अपने रब के पास जीवित हैं, रोज़ी पा रहे हैं
\end{hindi}}
\flushright{\begin{Arabic}
\quranayah[3][170]
\end{Arabic}}
\flushleft{\begin{hindi}
अल्लाह ने अपनी उदार कृपा से जो कुछ उन्हें प्रदान किया है, वे उसपर बहुत प्रसन्न है और उन लोगों के लिए भी ख़ुश हो रहे है जो उनके पीछे रह गए है, अभी उनसे मिले नहीं है कि उन्हें भी न कोई भय होगा और न वे दुखी होंगे
\end{hindi}}
\flushright{\begin{Arabic}
\quranayah[3][171]
\end{Arabic}}
\flushleft{\begin{hindi}
वे अल्लाह के अनुग्रह और उसकी उदार कृपा से प्रसन्न हो रहे है और इससे कि अल्लाह ईमानवालों का बदला नष्ट नहीं करता
\end{hindi}}
\flushright{\begin{Arabic}
\quranayah[3][172]
\end{Arabic}}
\flushleft{\begin{hindi}
जिन लोगों ने अल्लाह और रसूल की पुकार को स्वीकार किया, इसके पश्चात कि उन्हें आघात पहुँच चुका था। इन सत्कर्मी और (अल्लाह का) डर रखनेवालों के लिए बड़ा प्रतिदान है
\end{hindi}}
\flushright{\begin{Arabic}
\quranayah[3][173]
\end{Arabic}}
\flushleft{\begin{hindi}
ये वही लोग है जिनसे लोगों ने कहा, "तुम्हारे विरुद्ध लोग इकट्ठा हो गए है, अतः उनसे डरो।" तो इस चीज़ ने उनके ईमान को और बढ़ा दिया। और उन्होंने कहा, "हमारे लिए तो बस अल्लाह काफ़ी है और वही सबसे अच्छा कार्य-साधक है।"
\end{hindi}}
\flushright{\begin{Arabic}
\quranayah[3][174]
\end{Arabic}}
\flushleft{\begin{hindi}
तो वे अल्लाह को ओर से प्राप्त होनेवाली नेमत और उदार कृपा के साथ लौटे। उन्हें कोई तकलीफ़ छू भी नहीं सकी और वे अल्लाह की इच्छा पर चले भी, और अल्लाह बड़ी ही उदार कृपावाला है
\end{hindi}}
\flushright{\begin{Arabic}
\quranayah[3][175]
\end{Arabic}}
\flushleft{\begin{hindi}
वह तो शैतान है जो अपने मित्रों को डराता है। अतः तुम उनसे न डरो, बल्कि मुझी से डरो, यदि तुम ईमानवाले हो
\end{hindi}}
\flushright{\begin{Arabic}
\quranayah[3][176]
\end{Arabic}}
\flushleft{\begin{hindi}
जो लोग अधर्म और इनकार में जल्दी दिखाते है, उनके कारण तुम दुखी न हो। वे अल्लाह का कुछ भी नहीं बिगाड़ सकते। अल्लाह चाहता है कि उनके लिए आख़िरत में कोई हिस्सा न रखे, उनके लिए तो बड़ी यातना है
\end{hindi}}
\flushright{\begin{Arabic}
\quranayah[3][177]
\end{Arabic}}
\flushleft{\begin{hindi}
जो लोग ईमान की क़ीमत पर इनकार और अधर्म के ग्राहक हुए, वे अल्लाह का कुछ भी नहीं बिगाड़ सकते, उनके लिए तो दुखद यातना है
\end{hindi}}
\flushright{\begin{Arabic}
\quranayah[3][178]
\end{Arabic}}
\flushleft{\begin{hindi}
और यह ढ़ील जो हम उन्हें दिए जाते है, इसे अधर्मी लोग अपने लिए अच्छा न समझे। यह ढील तो हम उन्हें सिर्फ़ इसलिए दे रहे है कि वे गुनाहों में और अधिक बढ़ जाएँ, और उनके लिए तो अत्यन्त अपमानजनक यातना है
\end{hindi}}
\flushright{\begin{Arabic}
\quranayah[3][179]
\end{Arabic}}
\flushleft{\begin{hindi}
अल्लाह ईमानवालों को इस दशा में नहीं रहने देगा, जिसमें तुम हो। यह तो उस समय तक की बात है जबतक कि वह अपवित्र को पवित्र से पृथक नहीं कर देता। और अल्लाह ऐसा नहीं है कि वह तुम्हें परोक्ष की सूचना दे दे। किन्तु अल्लाह इस काम के लिए जिसको चाहता है चुन लेता है, और वे उसके रसूल होते है। अतः अल्लाह और उसके रसूल पर ईमान लाओ। और यदि तुम ईमान लाओगे और (अल्लाह का) डर रखोगे तो तुमको बड़ा प्रतिदान मिलेगा
\end{hindi}}
\flushright{\begin{Arabic}
\quranayah[3][180]
\end{Arabic}}
\flushleft{\begin{hindi}
जो लोग उस चीज़ में कृपणता से काम लेते है, जो अल्लाह ने अपनी उदार कृपा से उन्हें प्रदान की है, वे यह न समझे कि यह उनके हित में अच्छा है, बल्कि यह उनके लिए बुरा है। जिस चीज़ में उन्होंने कृपणता से काम लिया होगा, वही आगे कियामत के दिन उनके गले का तौक़ बन जाएगा। और ये आकाश और धरती अंत में अल्लाह ही के लिए रह जाएँगे। तुम जो कुछ भी करते हो, अल्लाह उसकी ख़बर रखता है
\end{hindi}}
\flushright{\begin{Arabic}
\quranayah[3][181]
\end{Arabic}}
\flushleft{\begin{hindi}
अल्लाह उन लोगों की बात सुन चुका है जिनका कहना है कि "अल्लाह तो निर्धन है और हम धनवान है।" उनकी बात हम लिख लेंगे और नबियों को जो वे नाहक क़त्ल करते रहे है उसे भी। और हम कहेंगे, "लो, (अब) जलने की यातना का मज़ा चखो।"
\end{hindi}}
\flushright{\begin{Arabic}
\quranayah[3][182]
\end{Arabic}}
\flushleft{\begin{hindi}
यह उसका बदला है जो तुम्हारे हाथों ने आगे भेजा। अल्लाह अपने बन्दों पर तनिक भी ज़ुल्म नहीं करता
\end{hindi}}
\flushright{\begin{Arabic}
\quranayah[3][183]
\end{Arabic}}
\flushleft{\begin{hindi}
ये वही लोग है जिनका कहना है कि "अल्लाह ने हमें ताकीद की है कि हम किसी रसूल पर ईमान न लाएँ, जबतक कि वह हमारे सामने ऐसी क़ुरबानी न पेश करे जिसे आग खा जाए।" कहो, "तुम्हारे पास मुझसे पहले कितने ही रसूल खुली निशानियाँ लेकर आ चुके है, और वे वह चीज़ भी लाए थे जिसके लिए तुम कह रहे हो। फिर यदि तुम सच्चे हो तो तुमने उन्हें क़त्ल क्यों किया?"
\end{hindi}}
\flushright{\begin{Arabic}
\quranayah[3][184]
\end{Arabic}}
\flushleft{\begin{hindi}
फिर यदि वे तुम्हें झुठलाते ही रहें, तो तुमसे पहले भी कितने ही रसूल झुठलाए जा चुके है, जो खुली निशानियाँ, 'ज़बूरें' और प्रकाशमान किताब लेकर आए थे
\end{hindi}}
\flushright{\begin{Arabic}
\quranayah[3][185]
\end{Arabic}}
\flushleft{\begin{hindi}
प्रत्येक जीव मृत्यु का मज़ा चखनेवाला है, और तुम्हें तो क़ियामत के दिन पूरा-पूरा बदला दे दिया जाएगा। अतः जिसे आग (जहन्नम) से हटाकर जन्नत में दाख़िल कर दिया गया, वह सफल रहा। रहा सांसारिक जीवन, तो वह माया-सामग्री के सिवा कुछ भी नहीं
\end{hindi}}
\flushright{\begin{Arabic}
\quranayah[3][186]
\end{Arabic}}
\flushleft{\begin{hindi}
तुम्हारें माल और तुम्हारे प्राण में तुम्हारी परीक्षा होकर रहेगी और तुम्हें उन लोगों से जिन्हें तुमसे पहले किताब प्रदान की गई थी और उन लोगों से जिन्होंने 'शिर्क' किया, बहुत-सी कष्टप्रद बातें सुननी पड़ेगी। परन्तु यदि तुम जमें रहे और (अल्लाह का) डर रखा, तो यह उन कर्मों में से है जो आवश्यक ठहरा दिया गया है
\end{hindi}}
\flushright{\begin{Arabic}
\quranayah[3][187]
\end{Arabic}}
\flushleft{\begin{hindi}
याद करो जब अल्लाह ने उन लोगों से, जिन्हें किताब प्रदान की गई थी, वचन लिया था कि "उसे लोगों के सामने भली-भाँति स्पट् करोगे, उसे छिपाओगे नहीं।" किन्तु उन्होंने उसे पीठ पीछे डाल दिया और तुच्छ मूल्य पर उसका सौदा किया। कितना बुरा सौदा है जो ये कर रहे है
\end{hindi}}
\flushright{\begin{Arabic}
\quranayah[3][188]
\end{Arabic}}
\flushleft{\begin{hindi}
तुम उन्हें कदापि यह न समझना, जो अपने किए पर ख़ुश हो रहे है और जो काम उन्होंने नहीं किए, चाहते है कि उनपर भी उनकी प्रशंसा की जाए - तो तुम उन्हें यह न समझाना कि वे यातना से बच जाएँगे, उनके लिए तो दुखद यातना है
\end{hindi}}
\flushright{\begin{Arabic}
\quranayah[3][189]
\end{Arabic}}
\flushleft{\begin{hindi}
आकाशों और धरती का राज्य अल्लाह ही का है, और अल्लाह को हर चीज़ की सामर्थ्य प्राप्त है
\end{hindi}}
\flushright{\begin{Arabic}
\quranayah[3][190]
\end{Arabic}}
\flushleft{\begin{hindi}
निस्सदेह आकाशों और धरती की रचना में और रात और दिन के आगे पीछे बारी-बारी आने में उन बुद्धिमानों के लिए निशानियाँ है
\end{hindi}}
\flushright{\begin{Arabic}
\quranayah[3][191]
\end{Arabic}}
\flushleft{\begin{hindi}
जो खड़े, बैठे और अपने पहलुओं पर लेटे अल्लाह को याद करते है और आकाशों और धरती की रचना में सोच-विचार करते है। (वे पुकार उठते है,) "हमारे रब! तूने यह सब व्यर्थ नहीं बनाया है। महान है तू, अतः हमें आग की यातना से बचा ले
\end{hindi}}
\flushright{\begin{Arabic}
\quranayah[3][192]
\end{Arabic}}
\flushleft{\begin{hindi}
"हमारे रब, तूने जिसे आग में डाला, उसे रुसवा कर दिया। और ऐसे ज़ालिमों का कोई सहायक न होगा
\end{hindi}}
\flushright{\begin{Arabic}
\quranayah[3][193]
\end{Arabic}}
\flushleft{\begin{hindi}
"हमारे रब! हमने एक पुकारनेवाले को ईमान की ओर बुलाते सुना कि अपने रब पर ईमान लाओ। तो हम ईमान ले आए। हमारे रब! तो अब तू हमारे गुनाहों को क्षमा कर दे और हमारी बुराइयों को हमसे दूर कर दे और हमें नेक और वफ़़ादार लोगों के साथ (दुनिया से) उठा
\end{hindi}}
\flushright{\begin{Arabic}
\quranayah[3][194]
\end{Arabic}}
\flushleft{\begin{hindi}
"हमारे रब! जिस चीज़ का वादा तूने अपने रसूलों के द्वारा किया वह हमें प्रदान कर और क़ियामत के दिन हमें रुसवा न करना। निस्संदेह तू अपने वादे के विरुद्ध जानेवाला नहीं है।"
\end{hindi}}
\flushright{\begin{Arabic}
\quranayah[3][195]
\end{Arabic}}
\flushleft{\begin{hindi}
तो उनके रब ने उनकी पुकार सुन ली कि "मैं तुममें से किसी कर्म करनेवाले के कर्म को अकारथ नहीं करूँगा, चाहे वह पुरुष हो या स्त्री। तुम सब आपस में एक-दूसरे से हो। अतः जिन लोगों ने (अल्लाह के मार्ग में) घरबार छोड़ा और अपने घरों से निकाले गए और मेरे मार्ग में सताए गए, और लड़े और मारे गए, मैं उनसे उनकी बुराइयाँ दूर कर दूँगा और उन्हें ऐसे बाग़ों में प्रवेश कराऊँगा जिनके नीचे नहरें बह रही होंगी।" यह अल्लाह के पास से उनका बदला होगा और सबसे अच्छा बदला अल्लाह ही के पास है
\end{hindi}}
\flushright{\begin{Arabic}
\quranayah[3][196]
\end{Arabic}}
\flushleft{\begin{hindi}
बस्तियों में इनकार करनेवालों की चलत-फिरत तुम्हें किसी धोखे में न डाले
\end{hindi}}
\flushright{\begin{Arabic}
\quranayah[3][197]
\end{Arabic}}
\flushleft{\begin{hindi}
यह तो थोड़ी सुख-सामग्री है फिर तो उनका ठिकाना जहन्नम है, और वह बहुत ही बुरा ठिकाना है
\end{hindi}}
\flushright{\begin{Arabic}
\quranayah[3][198]
\end{Arabic}}
\flushleft{\begin{hindi}
किन्तु जो लोग अपने रब से डरते रहे उनके लिए ऐसे बाग़ होंगे जिनके नीचे नहरें बह रही होंगी। वे उसमें सदैव रहेंगे। यह अल्लाह की ओर से पहला आतिथ्य-सत्कार होगा और जो कुछ अल्लाह के पास है वह नेक और वफ़ादार लोगों के लिए सबसे अच्छा है
\end{hindi}}
\flushright{\begin{Arabic}
\quranayah[3][199]
\end{Arabic}}
\flushleft{\begin{hindi}
और किताबवालों में से कुछ ऐसे भी है, जो इस हाल में कि उनके दिल अल्लाह के आगे झुके हुए होते है, अल्लाह पर ईमान रखते है और उस चीज़ पर भी जो तुम्हारी ओर उतारी गई है, और उस चीज़ पर भी जो स्वयं उनकी ओर उतरी। वे अल्लाह की आयतों का 'तुच्छ मूल्य पर सौदा' नहीं करते, उनके लिए उनके रब के पास उनका प्रतिदान है। अल्लाह हिसाब भी जल्द ही कर देगा
\end{hindi}}
\flushright{\begin{Arabic}
\quranayah[3][200]
\end{Arabic}}
\flushleft{\begin{hindi}
ऐ ईमान लानेवालो! धैर्य से काम लो और (मुक़ाबले में) बढ़-चढ़कर धैर्य दिखाओ और जुटे और डटे रहो और अल्लाह से डरते रहो, ताकि तुम सफल हो सको
\end{hindi}}
\chapter{An-Nisa' (The Women)}
\begin{Arabic}
\Huge{\centerline{\basmalah}}\end{Arabic}
\flushright{\begin{Arabic}
\quranayah[4][1]
\end{Arabic}}
\flushleft{\begin{hindi}
ऐ लोगों! अपने रब का डर रखों, जिसने तुमको एक जीव से पैदा किया और उसी जाति का उसके लिए जोड़ा पैदा किया और उन दोनों से बहुत-से पुरुष और स्त्रियाँ फैला दी। अल्लाह का डर रखो, जिसका वास्ता देकर तुम एक-दूसरे के सामने माँगें रखते हो। और नाते-रिश्तों का भी तुम्हें ख़याल रखना हैं। निश्चय ही अल्लाह तुम्हारी निगरानी कर रहा हैं
\end{hindi}}
\flushright{\begin{Arabic}
\quranayah[4][2]
\end{Arabic}}
\flushleft{\begin{hindi}
और अनाथों को उनका माल दे दो और बुरी चीज़ को अच्छी चीज़ से न बदलो, और न उनके माल को अपने माल के साथ मिलाकर खा जाओ। यह बहुत बड़ा गुनाह हैं
\end{hindi}}
\flushright{\begin{Arabic}
\quranayah[4][3]
\end{Arabic}}
\flushleft{\begin{hindi}
और यदि तुम्हें आशंका हो कि तुम अनाथों (अनाथ लड़कियों) के प्रति न्याय न कर सकोगे तो उनमें से, जो तुम्हें पसन्द हों, दो-दो या तीन-तीन या चार-चार से विवाह कर लो। किन्तु यदि तुम्हें आशंका हो कि तुम उनके साथ एक जैसा व्यवहार न कर सकोंगे, तो फिर एक ही पर बस करो, या उस स्त्री (लौंड़ी) पर जो तुम्हारे क़ब्ज़े में आई हो, उसी पर बस करो। इसमें तुम्हारे न्याय से न हटने की अधिक सम्भावना है
\end{hindi}}
\flushright{\begin{Arabic}
\quranayah[4][4]
\end{Arabic}}
\flushleft{\begin{hindi}
और स्त्रियों को उनके मह्रा ख़ुशी से अदा करो। हाँ, यदि वे अपनी ख़ुशी से उसमें से तुम्हारे लिए छोड़ दे तो उसे तुम अच्छा और पाक समझकर खाओ
\end{hindi}}
\flushright{\begin{Arabic}
\quranayah[4][5]
\end{Arabic}}
\flushleft{\begin{hindi}
और अपने माल, जिसे अल्लाह ने तुम्हारे लिए जीवन-यापन का साधन बनाया है, बेसमझ लोगों को न दो। उन्हें उसमें से खिलाते और पहनाते रहो और उनसे भली बात कहो
\end{hindi}}
\flushright{\begin{Arabic}
\quranayah[4][6]
\end{Arabic}}
\flushleft{\begin{hindi}
और अनाथों को जाँचते रहो, यहाँ तक कि जब वे विवाह की अवस्था को पहुँच जाएँ, तो फिर यदि तुम देखो कि उनमें सूझ-बूझ आ गई है, तो उनके माल उन्हें सौंप दो, और इस भय से कि कहीं वे बड़े न हो जाएँ तुम उनके माल अनुचित रूप से उड़ाकर और जल्दी करके न खाओ। और जो धनवान हो, उसे तो (इस माल से) से बचना ही चाहिए। हाँ, जो निर्धन हो, वह उचित रीति से कुछ खा सकता है। फिर जब उनके माल उन्हें सौंपने लगो, तो उनकी मौजूदगी में गवाह बना लो। हिसाब लेने के लिए अल्लाह काफ़ी है
\end{hindi}}
\flushright{\begin{Arabic}
\quranayah[4][7]
\end{Arabic}}
\flushleft{\begin{hindi}
पुरुषों का उस माल में एक हिस्सा है जो माँ-बाप और नातेदारों ने छोड़ा हो; और स्त्रियों का भी उस माल में एक हिस्सा है जो माल माँ-बाप और नातेदारों ने छोड़ा हो - चाह वह थोड़ा हो या अधिक हो - यह हिस्सा निश्चित किया हुआ है
\end{hindi}}
\flushright{\begin{Arabic}
\quranayah[4][8]
\end{Arabic}}
\flushleft{\begin{hindi}
और जब बाँटने के समय नातेदार और अनाथ और मुहताज उपस्थित हो तो उन्हें भी उसमें से (उनका हिस्सा) दे दो और उनसे भली बात करो
\end{hindi}}
\flushright{\begin{Arabic}
\quranayah[4][9]
\end{Arabic}}
\flushleft{\begin{hindi}
और लोगों को डरना चाहिए कि यदि वे स्वयं अपने पीछे अपने निर्बल बच्चे छोड़ते तो उन्हें उन बच्चों के विषय में कितना भय होता। तो फिर उन्हें अल्लाह से डरना चाहिए और ठीक सीधी बात कहनी चाहिए
\end{hindi}}
\flushright{\begin{Arabic}
\quranayah[4][10]
\end{Arabic}}
\flushleft{\begin{hindi}
जो लोग अनाथों के माल अन्याय के साथ खाते है, वास्तव में वे अपने पेट आग से भरते है, और वे अवश्य भड़कती हुई आग में पड़ेगे
\end{hindi}}
\flushright{\begin{Arabic}
\quranayah[4][11]
\end{Arabic}}
\flushleft{\begin{hindi}
अल्लाह तुम्हारी सन्तान के विषय में तुम्हें आदेश देता है कि दो बेटियों के हिस्से के बराबर एक बेटे का हिस्सा होगा; और यदि दो से अधिक बेटियाँ ही हो तो उनका हिस्सा छोड़ी हुई सम्पत्ति का दो तिहाई है। और यदि वह अकेली हो तो उसके लिए आधा है। और यदि मरनेवाले की सन्तान हो जो उसके माँ-बाप में से प्रत्येक का उसके छोड़े हुए माल का छठा हिस्सा है। और यदि वह निस्संतान हो और उसके माँ-बाप ही उसके वारिस हों, तो उसकी माँ का हिस्सा तिहाई होगा। और यदि उसके भाई भी हो, तो उसका माँ का छठा हिस्सा होगा। ये हिस्से, वसीयत जो वह कर जाए पूरी करने या ऋण चुका देने के पश्चात है। तुम्हारे बाप भी है और तुम्हारे बेटे भी। तुम नहीं जानते कि उनमें से लाभ पहुँचाने की दृष्टि से कौन तुमसे अधिक निकट है। यह हिस्सा अल्लाह का निश्चित किया हुआ है। अल्लाह सब कुछ जानता, समझता है
\end{hindi}}
\flushright{\begin{Arabic}
\quranayah[4][12]
\end{Arabic}}
\flushleft{\begin{hindi}
और तुम्हारी पत्नि यों ने जो कुछ छोड़ा हो, उसमें तुम्हारा आधा है, यदि उनकी सन्तान न हो। लेकिन यदि उनकी सन्तान हो तो वे छोड़े, उसमें तुम्हारा चौथाई होगा, इसके पश्चात कि जो वसीयत वे कर जाएँ वह पूरी कर दी जाए, या जो ऋण (उनपर) हो वह चुका दिया जाए। और जो कुछ तुम छोड़ जाओ, उसमें उनका (पत्ऩियों का) चौथाई हिस्सा होगा, यदि तुम्हारी कोई सन्तान न हो। लेकिन यदि तुम्हारी सन्तान है, तो जो कुछ तुम छोड़ोगे, उसमें से उनका (पत्नियों का) आठवाँ हिस्सा होगा, इसके पश्चात कि जो वसीयत तुमने की हो वह पूरी कर दी जाए, या जो ऋण हो उसे चुका दिया जाए, और यदि किसी पुरुष या स्त्री के न तो कोई सन्तान हो और न उसके माँ-बाप ही जीवित हो और उसके एक भाई या बहन हो तो उन दोनों में से प्रत्येक को छठा हिस्सा होगा। लेकिन यदि वे इससे अधिक हों तो फिर एक तिहाई में वे सब शरीक होंगे, इसके पश्चात कि जो वसीयत उसने की वह पूरी कर दी जाए या जो ऋण (उसपर) हो वह चुका दिया जाए, शर्त यह है कि वह हानिकर न हो। यह अल्लाह की ओर से ताकीदी आदेश है और अल्लाह सब कुछ जाननेवाला, अत्यन्त सहनशील है
\end{hindi}}
\flushright{\begin{Arabic}
\quranayah[4][13]
\end{Arabic}}
\flushleft{\begin{hindi}
ये अल्लाह की निश्चित की हुई सीमाएँ है। जो कोई अल्लाह और उसके रसूल के आदेशों का पालन करेगा, उसे अल्लाह ऐसे बाग़ों में दाख़िल करेगा जिनके नीचे नहरें बह रही होंगी। उनमें वह सदैव रहेगा और यही बड़ी सफलता है
\end{hindi}}
\flushright{\begin{Arabic}
\quranayah[4][14]
\end{Arabic}}
\flushleft{\begin{hindi}
परन्तु जो अल्लाह और उसके रसूल की अवज्ञा करेगा और उसकी सीमाओं का उल्लंघन करेगा उसे अल्लाह आग में डालेगा, जिसमें वह सदैव रहेगा। और उसके लिए अपमानजनक यातना है
\end{hindi}}
\flushright{\begin{Arabic}
\quranayah[4][15]
\end{Arabic}}
\flushleft{\begin{hindi}
और तुम्हारी स्त्रियों में से जो व्यभिचार कर बैठे, उनपर अपने में से चार आदमियों की गवाही लो, फिर यदि वे गवाही दे दें तो उन्हें घरों में बन्द रखो, यहाँ तक कि उनकी मृत्यु आ जाए या अल्लाह उनके लिए कोई रास्ता निकाल दे
\end{hindi}}
\flushright{\begin{Arabic}
\quranayah[4][16]
\end{Arabic}}
\flushleft{\begin{hindi}
और तुममें से जो दो पुरुष यह कर्म करें, उन्हें प्रताड़ित करो, फिर यदि वे तौबा कर ले और अपने आपको सुधार लें, तो उन्हें छोड़ दो। अल्लाह तौबा क़बूल करनेवाला, दयावान है
\end{hindi}}
\flushright{\begin{Arabic}
\quranayah[4][17]
\end{Arabic}}
\flushleft{\begin{hindi}
उन्ही लोगों की तौबा क़बूल करना अल्लाह के ज़िम्मे है जो भावनाओं में बह कर नादानी से कोई बुराई कर बैठे, फिर जल्द ही तौबा कर लें, ऐसे ही लोग है जिनकी तौबा अल्लाह क़बूल करता है। अल्लाह सब कुछ जाननेवाला, तत्वदर्शी है
\end{hindi}}
\flushright{\begin{Arabic}
\quranayah[4][18]
\end{Arabic}}
\flushleft{\begin{hindi}
और ऐसे लोगों की तौबा नहीं जो बुरे काम किए चले जाते है, यहाँ तक कि जब उनमें से किसी की मृत्यु का समय आ जाता है तो कहने लगता है, "अब मैं तौबा करता हूँ।" और इसी प्रकार तौबा उनकी भी नहीं है, जो मरते दम तक इनकार करनेवाले ही रहे। ऐसे लोगों के लिए हमने दुखद यातना तैयार कर रखी है
\end{hindi}}
\flushright{\begin{Arabic}
\quranayah[4][19]
\end{Arabic}}
\flushleft{\begin{hindi}
ऐ ईमान लानेवालो! तुम्हारे लिए वैध नहीं कि स्त्रियों के माल के ज़बरदस्ती वारिस बन बैठो, और न यह वैध है कि उन्हें इसलिए रोको और तंग करो कि जो कुछ तुमने उन्हें दिया है, उसमें से कुछ ले उड़ो। परन्तु यदि वे खुले रूप में अशिष्ट कर्म कर बैठे तो दूसरी बात है। और उनके साथ भले तरीक़े से रहो-सहो। फिर यदि वे तुम्हें पसन्द न हों, तो सम्भव है कि एक चीज़ तुम्हें पसन्द न हो और अल्लाह उसमें बहुत कुछ भलाई रख दे
\end{hindi}}
\flushright{\begin{Arabic}
\quranayah[4][20]
\end{Arabic}}
\flushleft{\begin{hindi}
और यदि तुम एक पत्ऩी की जगह दूसरी पत्ऩी लाना चाहो तो, चाहे तुमने उनमें किसी को ढेरों माल दे दिया हो, उसमें से कुछ मत लेना। क्या तुम उसपर झूठा आरोप लगाकर और खुले रूप में हक़ मारकर उसे लोगे?
\end{hindi}}
\flushright{\begin{Arabic}
\quranayah[4][21]
\end{Arabic}}
\flushleft{\begin{hindi}
और तुम उसे किस तरह ले सकते हो, जबकि तुम एक-दूसरे से मिल चुके हो और वे तुमसे दृढ़ प्रतिज्ञा भी ले चुकी है?
\end{hindi}}
\flushright{\begin{Arabic}
\quranayah[4][22]
\end{Arabic}}
\flushleft{\begin{hindi}
और उन स्त्रियों से विवाह न करो, जिनसे तुम्हारे बाप विवाह कर चुके हों, परन्तु जो पहले हो चुका सो हो चुका। निस्संदेह यह एक अश्लील और अत्यन्त अप्रिय कर्म है, और बुरी रीति है
\end{hindi}}
\flushright{\begin{Arabic}
\quranayah[4][23]
\end{Arabic}}
\flushleft{\begin{hindi}
तुम्हारे लिए हराम है तुम्हारी माएँ, बेटियाँ, बहनें, फूफियाँ, मौसियाँ, भतीतियाँ, भाँजिया, और तुम्हारी वे माएँ जिन्होंने तुम्हें दूध पिलाया हो और दूध के रिश्ते से तुम्हारी बहनें और तुम्हारी सासें और तुम्हारी पत्ऩियों की बेटियाँ जिनसे तुम सम्भोग कर चुक हो। परन्तु यदि सम्भोग नहीं किया है तो इसमें तुमपर कोई गुनाह नहीं - और तुम्हारे उन बेटों की पत्ऩियाँ जो तुमसे पैदा हों और यह भी कि तुम दो बहनों को इकट्ठा करो; जो पहले जो हो चुका सो हो चुका। निश्चय ही अल्लाह अत्यन्त क्षमाशील, दयावान है
\end{hindi}}
\flushright{\begin{Arabic}
\quranayah[4][24]
\end{Arabic}}
\flushleft{\begin{hindi}
और विवाहित स्त्रियाँ भी वर्जित है, सिवाय उनके जो तुम्हारी लौंडी हों। यह अल्लाह ने तुम्हारे लिए अनिवार्य कर दिया है। इनके अतिरिक्त शेष स्त्रियाँ तुम्हारे लिए वैध है कि तुम अपने माल के द्वारा उन्हें प्राप्त करो उनकी पाकदामनी की सुरक्षा के लिए, न कि यह काम स्वच्छन्द काम-तृप्ति के लिए हो। फिर उनसे दाम्पत्य जीवन का आनन्द लो तो उसके बदले उनका निश्चित किया हुए हक़ (मह्रि) अदा करो और यदि हक़ निश्चित हो जाने के पश्चात तुम आपम में अपनी प्रसन्नता से कोई समझौता कर लो, तो इसमें तुम्हारे लिए कोई दोष नहीं। निस्संदेह अल्लाह सब कुछ जाननेवाला, तत्वदर्शी है
\end{hindi}}
\flushright{\begin{Arabic}
\quranayah[4][25]
\end{Arabic}}
\flushleft{\begin{hindi}
और तुममें से जिस किसी की इतनी सामर्थ्य न हो कि पाकदामन, स्वतंत्र, ईमानवाली स्त्रियों से विवाह कर सके, तो तुम्हारी वे ईमानवाली जवान लौडियाँ ही सही जो तुम्हारे क़ब्ज़े में हो। और अल्लाह तुम्हारे ईमान को भली-भाँति जानता है। तुम सब आपस में एक ही हो, तो उनके मालिकों की अनुमति से तुम उनसे विवाह कर लो और सामान्य नियम के अनुसार उन्हें उनका हक़ भी दो। वे पाकदामनी की सुरक्षा करनेवाली हों, स्वच्छन्द काम-तृप्ति न करनेवाली हों और न वे चोरी-छिपे ग़ैरो से प्रेम करनेवाली हों। फिर जब वे विवाहिता बना ली जाएँ और उसके पश्चात कोई अश्लील कर्म कर बैठें, तो जो दंड सम्मानित स्त्रियों के लिए है, उसका आधा उनके लिए होगा। यह तुममें से उस व्यक्ति के लिए है, जिसे ख़राबी में पड़ जाने का भय हो, और यह कि तुम धैर्य से काम लो तो यह तुम्हारे लिए अधिक अच्छा है। निस्संदेह अल्लाह बहुत क्षमाशील, दयावान है
\end{hindi}}
\flushright{\begin{Arabic}
\quranayah[4][26]
\end{Arabic}}
\flushleft{\begin{hindi}
अल्लाह चाहता है कि तुमपर स्पष्ट कर दे और तुम्हें उन लोगों के तरीक़ों पर चलाए, जो तुमसे पहले हुए है और तुमपर दयादृष्टि करे। अल्लाह तो सब कुछ जाननेवाला, तत्वदर्शी है
\end{hindi}}
\flushright{\begin{Arabic}
\quranayah[4][27]
\end{Arabic}}
\flushleft{\begin{hindi}
और अल्लाह चाहता है कि जो तुमपर दयादृष्टि करे, किन्तु जो लोग अपनी तुच्छ इच्छाओं का पालन करते है, वे चाहते है कि तुम राह से हटकर बहुत दूर जा पड़ो
\end{hindi}}
\flushright{\begin{Arabic}
\quranayah[4][28]
\end{Arabic}}
\flushleft{\begin{hindi}
अल्लाह चाहता है कि तुमपर से बोझ हलका कर दे, क्योंकि इनसान निर्बल पैदा किया गया है
\end{hindi}}
\flushright{\begin{Arabic}
\quranayah[4][29]
\end{Arabic}}
\flushleft{\begin{hindi}
ऐ ईमान लानेवालो! आपस में एक-दूसरे के माल ग़लत तरीक़े से न खाओ - यह और बात है कि तुम्हारी आपस में रज़ामन्दी से कोई सौदा हो - और न अपनों की हत्या करो। निस्संदेह अल्लाह तुमपर बहुत दयावान है
\end{hindi}}
\flushright{\begin{Arabic}
\quranayah[4][30]
\end{Arabic}}
\flushleft{\begin{hindi}
और जो कोई ज़ुल्म और ज़्यादती से ऐसा करेगा, तो उसे हम जल्द ही आग में झोंक देंगे, और यह अल्लाह के लिए सरल है
\end{hindi}}
\flushright{\begin{Arabic}
\quranayah[4][31]
\end{Arabic}}
\flushleft{\begin{hindi}
यदि तुम उन बड़े गुनाहों से बचते रहो, जिनसे तुम्हे रोका जा रहा है, तो हम तुम्हारी बुराइयों को तुमसे दूर कर देंगे और तुम्हें प्रतिष्ठित स्थान में प्रवेश कराएँगे
\end{hindi}}
\flushright{\begin{Arabic}
\quranayah[4][32]
\end{Arabic}}
\flushleft{\begin{hindi}
और उसकी कामना न करो जिसमें अल्लाह ने तुमसे किसी को किसी से उच्च रखा है। पुरुषों ने जो कुछ कमाया है, उसके अनुसार उनका हिस्सा है और स्त्रियों ने जो कुछ कमाया है, उसके अनुसार उनका हिस्सा है। अल्लाह से उसका उदार दान चाहो। निस्संदेह अल्लाह को हर चीज़ का ज्ञान है
\end{hindi}}
\flushright{\begin{Arabic}
\quranayah[4][33]
\end{Arabic}}
\flushleft{\begin{hindi}
और प्रत्येक माल के लिए, जो माँ-बाप और नातेदार छोड़ जाएँ, हमने वासिस ठहरा दिए है और जिन लोगों से अपनी क़समों के द्वारा तुम्हारा पक्का मामला हुआ हो, तो उन्हें भी उनका हिस्सा दो। निस्संदेह हर चीज़ अल्लाह के समक्ष है
\end{hindi}}
\flushright{\begin{Arabic}
\quranayah[4][34]
\end{Arabic}}
\flushleft{\begin{hindi}
पति पत्नियों संरक्षक और निगराँ है, क्योंकि अल्लाह ने उनमें से कुछ को कुछ के मुक़ाबले में आगे रहा है, और इसलिए भी कि पतियों ने अपने माल ख़र्च किए है, तो नेक पत्ऩियाँ तो आज्ञापालन करनेवाली होती है और गुप्त बातों की रक्षा करती है, क्योंकि अल्लाह ने उनकी रक्षा की है। और जो पत्नियों ऐसी हो जिनकी सरकशी का तुम्हें भय हो, उन्हें समझाओ और बिस्तरों में उन्हें अकेली छोड़ दो और (अति आवश्यक हो तो) उन्हें मारो भी। फिर यदि वे तुम्हारी बात मानने लगे, तो उनके विरुद्ध कोई रास्ता न ढूढ़ो। अल्लाह सबसे उच्च, सबसे बड़ा है
\end{hindi}}
\flushright{\begin{Arabic}
\quranayah[4][35]
\end{Arabic}}
\flushleft{\begin{hindi}
और यदि तुम्हें पति-पत्नी के बीच बिगाड़ का भय हो, तो एक फ़ैसला करनेवाला पुरुष के लोगों में से और एक फ़ैसला करनेवाला स्त्री के लोगों में से नियुक्त करो, यदि वे दोनों सुधार करना चाहेंगे, तो अल्लाह उनके बीच अनुकूलता पैदा कर देगा। निस्संदेह, अल्लाह सब कुछ जाननेवाला, ख़बर रखनेवाला है
\end{hindi}}
\flushright{\begin{Arabic}
\quranayah[4][36]
\end{Arabic}}
\flushleft{\begin{hindi}
अल्लाह की बन्दगी करो और उसके साथ किसी को साझी न बनाओ और अच्छा व्यवहार करो माँ-बाप के साथ, नातेदारों, अनाथों और मुहताजों के साथ, नातेदार पड़ोसियों के साथ और अपरिचित पड़ोसियों के साथ और साथ रहनेवाले व्यक्ति के साथ और मुसाफ़िर के साथ और उनके साथ भी जो तुम्हारे क़ब्ज़े में हों। अल्लाह ऐसे व्यक्ति को पसन्द नहीं करता, जो इतराता और डींगें मारता हो
\end{hindi}}
\flushright{\begin{Arabic}
\quranayah[4][37]
\end{Arabic}}
\flushleft{\begin{hindi}
वे जो स्वयं कंजूसी करते है और लोगों को भी कंजूसी पर उभारते है और अल्लाह ने अपने उदार दान से जो कुछ उन्हें दे रखा होता है, उसे छिपाते है, जो हमने अकृतज्ञ लोगों के लिए अपमानजनक यातना तैयार कर रखी है
\end{hindi}}
\flushright{\begin{Arabic}
\quranayah[4][38]
\end{Arabic}}
\flushleft{\begin{hindi}
वे जो अपने माल लोगों को दिखाने के लिए ख़र्च करते है, न अल्लाह पर ईमान रखते है, न अन्तिम दिन पर, और जिस किसी का साथी शैतान हुआ, तो वह बहुत ही बुरा साथी है
\end{hindi}}
\flushright{\begin{Arabic}
\quranayah[4][39]
\end{Arabic}}
\flushleft{\begin{hindi}
उनका बिगड़ जाता, यदि वे अल्लाह और अन्तिम दिन पर ईमान लाते और जो कुछ अल्लाह ने उन्हें दिया है, उसमें से ख़र्च करते है? अल्लाह उन्हें भली-भाँति जानता है
\end{hindi}}
\flushright{\begin{Arabic}
\quranayah[4][40]
\end{Arabic}}
\flushleft{\begin{hindi}
निस्संदेह अल्लाह रत्ती-भर भी ज़ुल्म नहीं करता और यदि कोई एक नेकी हो तो वह उसे कई गुना बढ़ा देगा और अपनी ओर से बड़ा बदला देगा
\end{hindi}}
\flushright{\begin{Arabic}
\quranayah[4][41]
\end{Arabic}}
\flushleft{\begin{hindi}
फिर क्या हाल होगा जब हम प्रत्येक समुदाय में से एक गवाह लाएँगे और स्वयं तुम्हें इन लोगों के मुक़ाबले में गवाह बनाकर पेश करेंगे?
\end{hindi}}
\flushright{\begin{Arabic}
\quranayah[4][42]
\end{Arabic}}
\flushleft{\begin{hindi}
उस दिन वे लोग जिन्होंने इनकार किया होगा और रसूल की अवज्ञा की होगी, यही चाहेंगे कि किसी तरह धरती में समोकर उसे बराबर कर दिया जाए। वे अल्लाह से कोई बात भी न छिपा सकेंगे
\end{hindi}}
\flushright{\begin{Arabic}
\quranayah[4][43]
\end{Arabic}}
\flushleft{\begin{hindi}
ऐ ईमान लानेवालो! नशे की दशा में नमाज़ में व्यस्त न हो, जब तक कि तुम यह न जानने लगो कि तुम क्या कह रहे हो। और इसी प्रकार नापाकी की दशा में भी (नमाज़ में व्यस्त न हो), जब तक कि तुम स्नान न कर लो, सिवाय इसके कि तुम सफ़र में हो। और यदि तुम बीमार हो या सफ़र में हो, या तुममें से कोई शौच करके आए या तुमने स्त्रियों को हाथ लगाया हो, फिर तुम्हें पानी न मिले, तो पाक मिट्टी से काम लो और उसपर हाथ मारकर अपने चहरे और हाथों पर मलो। निस्संदेह अल्लाह नर्मी से काम लेनेवाला, अत्यन्त क्षमाशील है
\end{hindi}}
\flushright{\begin{Arabic}
\quranayah[4][44]
\end{Arabic}}
\flushleft{\begin{hindi}
क्या तुमने उन लोगों को नहीं देखा, जिन्हें सौभाग्य प्रदान हुआ था अर्थात किताब दी गई थी? वे पथभ्रष्टता के खरीदार बने हुए है और चाहते है कि तुम भी रास्ते से भटक जाओ
\end{hindi}}
\flushright{\begin{Arabic}
\quranayah[4][45]
\end{Arabic}}
\flushleft{\begin{hindi}
अल्लाह तुम्हारे शत्रुओं को भली-भाँति जानता है। अल्लाह एक संरक्षक के रूप में काफ़ी है और अल्लाह एक सहायक के रूप में भी काफ़ी है
\end{hindi}}
\flushright{\begin{Arabic}
\quranayah[4][46]
\end{Arabic}}
\flushleft{\begin{hindi}
वे लोग जो यहूदी बन गए, वे शब्दों को उनके स्थानों से दूसरी ओर फेर देते है और कहते हैं, "समि'अना व 'असैना" (हमने सुना, लेकिन हम मानते नही); और "इसम'अ ग़ै-र मुसम'इन" (सुनो हालाँकि तुम सुनने के योग्य नहीं हो और "राइना" (हमारी ओर ध्यान दो) - यह वे अपनी ज़बानों को तोड़-मरोड़कर और दीन पर चोटें करते हुए कहते है। और यदि वे कहते, "समिअ'ना व अ-त'अना" (हमने सुना और माना) और "इसम'अ" (सुनो) और "उनज़ुरना" (हमारी ओर निगाह करो) तो यह उनके लिए अच्छा और अधिक ठीक होता। किन्तु उनपर तो उनके इनकार के कारण अल्लाह की फिटकार पड़ी हुई है। फिर वे ईमान थोड़े ही लाते है
\end{hindi}}
\flushright{\begin{Arabic}
\quranayah[4][47]
\end{Arabic}}
\flushleft{\begin{hindi}
ऐ लोगों! जिन्हें किताब दी गई, उस चीज को मानो जो हमने उतारी है, जो उसकी पुष्टि में है, जो स्वयं तुम्हारे पास है, इससे पहले कि हम चेहरों की रूपरेखा को मिटाकर रख दें और उन्हें उनके पीछ की ओर फेर दें या उनपर लानत करें, जिस प्रकार हमने सब्तवालों पर लानत की थी। और अल्लाह का आदेश तो लागू होकर ही रहता है
\end{hindi}}
\flushright{\begin{Arabic}
\quranayah[4][48]
\end{Arabic}}
\flushleft{\begin{hindi}
अल्लाह इसके क्षमा नहीं करेगा कि उसका साझी ठहराया जाए। किन्तु उससे नीचे दर्जे के अपराध को जिसके लिए चाहेगा, क्षमा कर देगा और जिस किसी ने अल्लाह का साझी ठहराया, तो उसने एक बड़ा झूठ घड़ लिया
\end{hindi}}
\flushright{\begin{Arabic}
\quranayah[4][49]
\end{Arabic}}
\flushleft{\begin{hindi}
क्या तुमने उन लोगों को नहीं देखा जो अपने को पूर्ण एवं शिष्ट होने का दावा करते हैं? (कोई यूँ ही शिष्ट नहीं हुआ करता) बल्कि अल्लाह ही जिसे चाहता है, पूर्णता एवं शिष्टता प्रदान करता है। और उनके साथ तनिक भी अत्याचार नहीं किया जाता
\end{hindi}}
\flushright{\begin{Arabic}
\quranayah[4][50]
\end{Arabic}}
\flushleft{\begin{hindi}
देखो तो सही, वे अल्लाह पर कैसा झूठ मढ़ते हैं? खुले गुनाह के लिए तो यही पर्याप्त है
\end{hindi}}
\flushright{\begin{Arabic}
\quranayah[4][51]
\end{Arabic}}
\flushleft{\begin{hindi}
क्या तुमने उन लोगों को नहीं देखा, जिन्हें किताब का एक हिस्सा दिय् गया? वे अवास्तविक चीज़ो और ताग़ूत (बढ़ हुए सरकश) को मानते है। और अधर्मियों के विषय में कहते है, "ये ईमानवालों से बढ़कर मार्ग पर है।"
\end{hindi}}
\flushright{\begin{Arabic}
\quranayah[4][52]
\end{Arabic}}
\flushleft{\begin{hindi}
वही है जिनपर अल्लाह ने लातन की है, और जिसपर अल्लाह लानत कर दे, उसका तुम कोई सहायक कदापि न पाओगे
\end{hindi}}
\flushright{\begin{Arabic}
\quranayah[4][53]
\end{Arabic}}
\flushleft{\begin{hindi}
या बादशाही में इनका कोई हिस्सा है? फिर तो ये लोगों को फूटी कौड़ी तक भी न देते
\end{hindi}}
\flushright{\begin{Arabic}
\quranayah[4][54]
\end{Arabic}}
\flushleft{\begin{hindi}
या ये लोगों से इसलिए ईर्ष्या करते है कि अल्लाह ने उन्हें अपने उदार दान से अनुग्रहित कर दिया? हमने तो इबराहीम के लोगों को किताब और हिकमत (तत्वदर्शिता) दी और उन्हें बड़ा राज्य प्रदान किया
\end{hindi}}
\flushright{\begin{Arabic}
\quranayah[4][55]
\end{Arabic}}
\flushleft{\begin{hindi}
फिर उनमें से कोई उसपर ईमान लाया और उसमें से किसी ने किनारा खीच लिया। और (ऐसे लोगों के लिए) जहन्नम की भड़कती आग ही काफ़ी है
\end{hindi}}
\flushright{\begin{Arabic}
\quranayah[4][56]
\end{Arabic}}
\flushleft{\begin{hindi}
जिन लोगों ने हमारी आयतों का इनकार किया, उन्हें हम जल्द ही आग में झोंकेंगे। जब भी उनकी खालें पक जाएँगी, तो हम उन्हें दूसरी खालों में बदल दिया करेंगे, ताकि वे यातना का मज़ा चखते ही रहें। निस्संदेह अल्लाह प्रभुत्वशाली, तत्वदर्शी है
\end{hindi}}
\flushright{\begin{Arabic}
\quranayah[4][57]
\end{Arabic}}
\flushleft{\begin{hindi}
और जो लोग ईमान लाए और उन्होंने अच्छे कर्म किए, उन्हें हम ऐसे बाग़ो में दाखिल करेंगे, जिनके नीचे नहरें बह रहीं होगी, जहाँ वे सदैव रहेंगे। उनके लिए वहाँ पाक जोड़े होंगे और हम उन्हें घनी छाँव में दाखिल करेंगे
\end{hindi}}
\flushright{\begin{Arabic}
\quranayah[4][58]
\end{Arabic}}
\flushleft{\begin{hindi}
अल्लाह तुम्हें आदेश देता है कि अमानतों को उनके हक़दारों तक पहुँचा दिया करो। और जब लोगों के बीच फ़ैसला करो, तो न्यायपूर्वक फ़ैसला करो। अल्लाह तुम्हें कितनी अच्छी नसीहत करता है। निस्सदेह, अल्लाह सब कुछ सुनता, देखता है
\end{hindi}}
\flushright{\begin{Arabic}
\quranayah[4][59]
\end{Arabic}}
\flushleft{\begin{hindi}
ऐ ईमान लानेवालो! अल्लाह की आज्ञा का पालन करो और रसूल का कहना मानो और उनका भी कहना मानो जो तुममें अधिकारी लोग है। फिर यदि तुम्हारे बीच किसी मामले में झगड़ा हो जाए, तो उसे तुम अल्लाह और रसूल की ओर लौटाओ, यदि तुम अल्लाह और अन्तिम दिन पर ईमान रखते हो। यदि उत्तम है और परिणाम की स्पष्ट से भी अच्छा है
\end{hindi}}
\flushright{\begin{Arabic}
\quranayah[4][60]
\end{Arabic}}
\flushleft{\begin{hindi}
क्या तुमने उन लोगों को नहीं देखा, जो दावा तो करते है कि वे उस चीज़ पर ईमान रखते हैं, जो तुम्हारी ओर उतारी गई है और तुमसे पहले उतारी गई है। और चाहते है कि अपना मामला ताग़ूत के पास ले जाकर फ़ैसला कराएँ, जबकि उन्हें हुक्म दिया गया है कि वे उसका इनकार करें? परन्तु शैतान तो उन्हें भटकाकर बहुत दूर डाल देना चाहता है
\end{hindi}}
\flushright{\begin{Arabic}
\quranayah[4][61]
\end{Arabic}}
\flushleft{\begin{hindi}
और जब उनसे कहा जाता है कि आओ उस चीज़ की ओर जो अल्लाह ने उतारी है और आओ रसूल की ओरस तो तुम मुनाफ़िको (कपटाचारियों) को देखते हो कि वे तुमसे कतराकर रह जाते है
\end{hindi}}
\flushright{\begin{Arabic}
\quranayah[4][62]
\end{Arabic}}
\flushleft{\begin{hindi}
फिर कैसी बात होगी कि जब उनकी अपनी करतूतों के कारण उनपर बड़ी मुसीबत आ पडेगी। फिर वे तुम्हारे पास अल्लाह की क़समें खाते हुए आते है कि हम तो केवल भलाई और बनाव चाहते थे?
\end{hindi}}
\flushright{\begin{Arabic}
\quranayah[4][63]
\end{Arabic}}
\flushleft{\begin{hindi}
ये वे लोग है जिनके दिलों की बात अल्लाह भली-भाँति जानता है; तो तुम उन्हें जाने दो और उन्हें समझओ और उनसे उनके विषय में वह बात कहो जो प्रभावकारी हो
\end{hindi}}
\flushright{\begin{Arabic}
\quranayah[4][64]
\end{Arabic}}
\flushleft{\begin{hindi}
हमने जो रसूल भी भेजा, इसलिए भेजा कि अल्लाह की अनुमति से उसकी आज्ञा का पालन किया जाए। और यदि यह उस समय, जबकि इन्होंने स्वयं अपने ऊपर ज़ुल्म किया था, तुम्हारे पास आ जाते और अल्लाह से क्षमा की प्रार्थना करता तो निश्चय ही वे अल्लाह को अत्यन्त क्षमाशील और दयावान पाते
\end{hindi}}
\flushright{\begin{Arabic}
\quranayah[4][65]
\end{Arabic}}
\flushleft{\begin{hindi}
तो तुम्हें तुम्हारे रब की कसम! ये ईमानवाले नहीं हो सकते जब तक कि अपने आपस के झगड़ो में ये तुमसे फ़ैसला न कराएँ। फिर जो फ़ैसला तुम कर दो, उसपर ये अपने दिलों में कोई तंगी न पाएँ और पूरी तरह मान ले
\end{hindi}}
\flushright{\begin{Arabic}
\quranayah[4][66]
\end{Arabic}}
\flushleft{\begin{hindi}
और यदि कहीं हमने उन्हें आदेश दिया होता कि "अपनों को क़त्ल करो या अपने घरों से निकल जाओ।" तो उनमें से थोड़े ही ऐसा करते। हालाँकि जो नसीहत उन्हें दी जाती है, अगर वे उसे व्यवहार में लाते तो यह बात उनके लिए अच्छी होती और ज़्यादा ज़माव पैदा करनेवाली होती
\end{hindi}}
\flushright{\begin{Arabic}
\quranayah[4][67]
\end{Arabic}}
\flushleft{\begin{hindi}
और उस समय हम उन्हें अपनी ओर से निश्चय ही बड़ा बदला प्रदान करते
\end{hindi}}
\flushright{\begin{Arabic}
\quranayah[4][68]
\end{Arabic}}
\flushleft{\begin{hindi}
और उन्हें सीधे मार्ग पर लगा देते
\end{hindi}}
\flushright{\begin{Arabic}
\quranayah[4][69]
\end{Arabic}}
\flushleft{\begin{hindi}
जो अल्लाह और रसूल की आज्ञा का पालन करता है, तो ऐसे ही लोग उन लोगों के साथ है जिनपर अल्लाह की कृपा स्पष्ट रही है - वे नबी, सिद्दीक़, शहीद और अच्छे लोग है। और वे कितने अच्छे साथी है
\end{hindi}}
\flushright{\begin{Arabic}
\quranayah[4][70]
\end{Arabic}}
\flushleft{\begin{hindi}
यह अल्लाह का उदार अनुग्रह है। और काफ़ी है अल्लाह, इस हाल में कि वह भली-भाँति जानता है
\end{hindi}}
\flushright{\begin{Arabic}
\quranayah[4][71]
\end{Arabic}}
\flushleft{\begin{hindi}
ऐ ईमान लानेवालो! अपने बचाव की साम्रगी (हथियार आदि) सँभालो। फिर या तो अलग-अलग टुकड़ियों में निकलो या इकट्ठे होकर निकलो
\end{hindi}}
\flushright{\begin{Arabic}
\quranayah[4][72]
\end{Arabic}}
\flushleft{\begin{hindi}
तुमसे से कोई ऐसा भी है जो ढीला पड़ जाता है, फिर यदि तुमपर कोई मुसीबत आए तो कहने लगता है कि अल्लाह ने मुझपर कृपा की कि मैं इन लोगों के साथ न गया
\end{hindi}}
\flushright{\begin{Arabic}
\quranayah[4][73]
\end{Arabic}}
\flushleft{\begin{hindi}
परन्तु यदि अल्लाह की ओर से तुमपर कोई उदार अनुग्रह हो तो वह इस प्रकार से जैसे तुम्हारे और उनके बीच प्रेम का कोई सम्बन्ध ही नहीं, कहता है, "क्या ही अच्छा होता कि मैं भी उनके साथ होता, तो बड़ी सफलता प्राप्त करता।"
\end{hindi}}
\flushright{\begin{Arabic}
\quranayah[4][74]
\end{Arabic}}
\flushleft{\begin{hindi}
तो जो लोग आख़िरत (परलोक) के बदले सांसारिक जीवन का सौदा करें, तो उन्हें चाहिए कि अल्लाह के मार्ग में लड़े। जो अल्लाह के मार्ग में लड़ेगी, चाहे वह मारा जाए या विजयी रहे, उसे हम शीघ्र ही बड़ा बदला प्रदान करेंगे
\end{hindi}}
\flushright{\begin{Arabic}
\quranayah[4][75]
\end{Arabic}}
\flushleft{\begin{hindi}
तुम्हें क्या हुआ है कि अल्लाह के मार्ग में और उन कमज़ोर पुरुषों, औरतों और बच्चों के लिए युद्ध न करो, जो प्रार्थनाएँ करते है कि "हमारे रब! तू हमें इस बस्ती से निकाल, जिसके लोग अत्याचारी है। और हमारे लिए अपनी ओर से तू कोई समर्थक नियुक्त कर और हमारे लिए अपनी ओर से तू कोई सहायक नियुक्त कर।"
\end{hindi}}
\flushright{\begin{Arabic}
\quranayah[4][76]
\end{Arabic}}
\flushleft{\begin{hindi}
ईमान लानेवाले तो अल्लाह के मार्ग में युद्ध करते है और अधर्मी लोग ताग़ूत (बढ़े हुए सरकश) के मार्ग में युद्ध करते है। अतः तुम शैतान के मित्रों से लड़ो। निश्चय ही, शैतान की चाल बहुत कमज़ोर होती है
\end{hindi}}
\flushright{\begin{Arabic}
\quranayah[4][77]
\end{Arabic}}
\flushleft{\begin{hindi}
क्या तुमने उन लोगों को नहीं देखा जिनसे कहा गया था कि अपने हाथ रोके रखो और नमाज़ क़ायम करो और ज़कात दो? फिर जब उन्हें युद्ध का आदेश दिया गया तो क्या देखते है कि उनमें से कुछ लोगों का हाल यह है कि वे लोगों से ऐसा डरने लगे जैसे अल्लाह का डर हो या यह डर उससे भी बढ़कर हो। कहने लगे, "हमारे रब! तूने हमपर युद्ध क्यों अनिवार्य कर दिया? क्यों न थोड़ी मुहलत हमें और दी?" कह दो, "दुनिया की पूँजी बहुत थोड़ी है, जबकि आख़िरत उस व्यक्ति के अधिक अच्छी है जो अल्लाह का डर रखता हो और तुम्हारे साथ तनिक भी अन्याय न किया जाएगा।
\end{hindi}}
\flushright{\begin{Arabic}
\quranayah[4][78]
\end{Arabic}}
\flushleft{\begin{hindi}
"तुम जहाँ कहीं भी होंगे, मृत्यु तो तुम्हें आकर रहेगी; चाहे तुम मज़बूत बुर्जों (क़िलों) में ही (क्यों न) हो।" यदि उन्हें कोई अच्छी हालत पेश आती है तो कहते है, "यह तो अल्लाह के पास से है।" परन्तु यदि उन्हें कोई बुरी हालत पेश आती है तो कहते है, "यह तुम्हारे कारण है।" कह दो, "हरेक चीज़ अल्लाह के पास से है।" आख़िर इन लोगों को क्या हो गया कि ये ऐसे नहीं लगते कि कोई बात समझ सकें?
\end{hindi}}
\flushright{\begin{Arabic}
\quranayah[4][79]
\end{Arabic}}
\flushleft{\begin{hindi}
तुम्हें जो भी भलाई प्राप्त" होती है, वह अल्लाह को ओर से है और जो बुरी हालत तुम्हें पेश आ जाती है तो वह तुम्हारे अपने ही कारण पेश आती है। हमने तुम्हें लोगों के लिए रसूल बनाकर भेजा है और (इसपर) अल्लाह का गवाह होना काफ़ी है
\end{hindi}}
\flushright{\begin{Arabic}
\quranayah[4][80]
\end{Arabic}}
\flushleft{\begin{hindi}
जिसने रसूल की आज्ञा का पालन किया, उसने अल्लाह की आज्ञा का पालन किया और जिसने मुँह मोड़ा तो हमने तुम्हें ऐसे लोगों पर कोई रखवाला बनाकर तो नहीं भेजा है
\end{hindi}}
\flushright{\begin{Arabic}
\quranayah[4][81]
\end{Arabic}}
\flushleft{\begin{hindi}
और वे दावा तो आज्ञापालन का करते है, परन्तु जब तुम्हारे पास से हटते है तो उनमें एक गिरोह अपने कथन के विपरीत रात में षड्यंत्र करता है । जो कुछ वे षड्यंत्र करते है, अल्लाह उसे लिख रहा है। तो तुम उनसे रुख़ फेर लो और अल्लाह पर भरोसा रखो, और अल्लाह का कार्यसाधक होना काफ़ी है!
\end{hindi}}
\flushright{\begin{Arabic}
\quranayah[4][82]
\end{Arabic}}
\flushleft{\begin{hindi}
क्या वे क़ुरआन में सोच-विचार नहीं करते? यदि यह अल्लाह के अतिरिक्त किसी और की ओर से होता, तो निश्चय ही वे इसमें बहुत-सी बेमेल बातें पाते
\end{hindi}}
\flushright{\begin{Arabic}
\quranayah[4][83]
\end{Arabic}}
\flushleft{\begin{hindi}
जब उनके पास निश्चिन्तता या भय की कोई बात पहुचती है तो उसे फैला देते है, हालाँकि अगर वे उसे रसूल और अपने समुदाय के उतरदायी व्यक्तियों तक पहुँचाते तो उसे वे लोग जान लेते जो उनमें उसकी जाँच कर सकते है। और यदि तुमपर अल्लाह का उदार अनुग्रह और उसकी दयालुता न होती, तो थोड़े लोगों के सिवा तुम सब शैतान के पीछे चलने लग जाते
\end{hindi}}
\flushright{\begin{Arabic}
\quranayah[4][84]
\end{Arabic}}
\flushleft{\begin{hindi}
अतः अल्लाह के मार्ग में युद्ध करो - तुमपर तो बस तुम्हारी अपनी ही ज़िम्मेदारी है - और ईमानवालों की कमज़ोरियो को दूर करो और उन्हें (युद्ध के लिए) उभारो। इसकी बहुत सम्भावना है कि अल्लाह इनकार करनेवालों के ज़ोर को रोक लगा दे। अल्लाह बड़ा ज़ोरवाला और कठोर दंड देनेवाला है
\end{hindi}}
\flushright{\begin{Arabic}
\quranayah[4][85]
\end{Arabic}}
\flushleft{\begin{hindi}
जो कोई अच्छी सिफ़ारिश करेगा, उसे उसके कारण उसका प्रतिदान मिलेगा और जो बुरी सिफ़ारिश करेगा, तो उसके कारँ उसका बोझ उसपर पड़कर रहेगा। अल्लाह को तो हर चीज़ पर क़ाबू हासिल है
\end{hindi}}
\flushright{\begin{Arabic}
\quranayah[4][86]
\end{Arabic}}
\flushleft{\begin{hindi}
और तुम्हें जब सलामती की कोई दुआ दी जाए, तो तुम सलामती की उससे अच्छी दुआ दो या उसी को लौटा दो। निश्चय ही, अल्लाह हर चीज़ का हिसाब रखता है
\end{hindi}}
\flushright{\begin{Arabic}
\quranayah[4][87]
\end{Arabic}}
\flushleft{\begin{hindi}
अल्लाह के सिवा कोई इष्ट -पूज्य नहीं। वह तुम्हें क़ियामत के दिन की ओर ले जाकर इकट्ठा करके रहेगा, जिसके आने में कोई संदेह नहीं, और अल्लाह से बढ़कर सच्ची बात और किसकी हो सकती है
\end{hindi}}
\flushright{\begin{Arabic}
\quranayah[4][88]
\end{Arabic}}
\flushleft{\begin{hindi}
फिर तुम्हें क्या हो गया है कि कपटाचारियों (मुनाफ़िक़ो) के विषय में तुम दो गिरोह हो रहे हो, यद्यपि अल्लाह ने तो उनकी करतूतों के कारण उन्हें उल्टा फेर दिया है? क्या तुम उसे मार्ग पर लाना चाहते हो जिसे अल्लाह ने गुमराह छोड़ दिया है? हालाँकि जिसे अल्लाह मार्ग न दिखाए, उसके लिए तुम कदापि कोई मार्ग नहीं पा सकते
\end{hindi}}
\flushright{\begin{Arabic}
\quranayah[4][89]
\end{Arabic}}
\flushleft{\begin{hindi}
वे तो चाहते है कि जिस प्रकार वे स्वयं अधर्मी है, उसी प्रकार तुम भी अधर्मी बनकर उन जैसे हो जाओ; तो तुम उनमें से अपने मित्र न बनाओ, जब तक कि वे अल्लाह के मार्ग में घरबार न छोड़े। फिर यदि वे इससे पीठ फेरें तो उन्हें पकड़ो, और उन्हें क़त्ल करो जहाँ कही भी उन्हें पाओ - तो उनमें से किसी को न अपना मित्र बनाना और न सहायक -
\end{hindi}}
\flushright{\begin{Arabic}
\quranayah[4][90]
\end{Arabic}}
\flushleft{\begin{hindi}
सिवाय उन लोगों के जो ऐसे लोगों से सम्बन्ध रखते हों, जिनसे तुम्हारे और उनकी बीच कोई समझौता हो या वे तुम्हारे पास इस दशा में आएँ कि उनके दिल इससे तंग हो रहे हों कि वे तुमसे लड़े या अपने लोगों से लड़ाई करें। यदि अल्लाह चाहता तो उन्हें तुमपर क़ाबू दे देता। फिर तो वे तुमसे अवश्य लड़ते; तो यदि वे तुमसे अलग रहें और तुमसे न लड़ें और संधि के लिए तुम्हारी ओर हाथ बढ़ाएँ तो उनके विरुद्ध अल्लाह ने तुम्हारे लिए कोई रास्ता नहीं रखा है
\end{hindi}}
\flushright{\begin{Arabic}
\quranayah[4][91]
\end{Arabic}}
\flushleft{\begin{hindi}
अब तुम कुछ ऐसे लोगों को भी पाओगे, जो चाहते है कि तुम्हारी ओर से निश्चिन्त होकर रहें और अपने लोगों की ओर से भी निश्चिन्त होकर रहे। परन्तु जब भी वे फ़साद और उपद्रव की ओर फेरे गए तो वे उसी में औधे जो गिरे। तो यदि वे तुमसे अलग-थलग न रहें और तुम्हारी ओर सुलह का हाथ न बढ़ाएँ, और अपने हाथ न रोकें, तो तुम उन्हें पकड़ो और क़त्ल करो, जहाँ कहीं भी तुम उन्हें पाओ। उनके विरुद्ध हमने तुम्हें खुला अधिकार दे रखा है
\end{hindi}}
\flushright{\begin{Arabic}
\quranayah[4][92]
\end{Arabic}}
\flushleft{\begin{hindi}
किसी ईमानवाले का यह काम नहीं कि वह किसी ईमानवाले का हत्या करे, भूल-चूक की बात और है। और यदि कोई क्यक्ति यदि ग़लती से किसी ईमानवाले की हत्या कर दे, तो एक मोमिन ग़ुलाम को आज़ाद करना होगा और अर्थदंड उस (मारे गए क्यक्ति) के घरवालों को सौंपा जाए। यह और बात है कि वे अपनी ख़ुशी से छोड़ दें। और यदि वह उन लोगों में से हो, जो तुम्हारे शत्रु हों और वह (मारा जानेवाला) स्वयं मोमिन रहा तो एक मोमिन को ग़ुलामी से आज़ाद करना होगा। और यदि वह उन लोगों में से हो कि तुम्हारे और उनके बीच कोई संधि और समझौता हो, तो अर्थदंड उसके घरवालों को सौंपा जाए और एक मोमिन को ग़ुलामी से आज़ाद करना होगा। लेकिन जो (ग़ुलाम) न पाए तो वह निरन्तर दो मास के रोज़े रखे। यह अल्लाह की ओर से निश्चित किया हुआ उसकी तरफ़ पलट आने का तरीक़ा है। अल्लाह तो सब कुछ जाननेवाला, तत्वदर्शी है
\end{hindi}}
\flushright{\begin{Arabic}
\quranayah[4][93]
\end{Arabic}}
\flushleft{\begin{hindi}
और जो व्यक्ति जान-बूझकर किसी मोमिन की हत्या करे, तो उसका बदला जहन्नम है, जिसमें वह सदा रहेगा; उसपर अल्लाह का प्रकोप और उसकी फिटकार है और उसके लिए अल्लाह ने बड़ी यातना तैयार कर रखी है
\end{hindi}}
\flushright{\begin{Arabic}
\quranayah[4][94]
\end{Arabic}}
\flushleft{\begin{hindi}
ऐ ईमान लानेवालो! जब तुम अल्लाह के मार्ग से निकलो तो अच्छी तरह पता लगा लो और जो तुम्हें सलाम करे, उससे यह न कहो कि तुम ईमान नहीं रखते, और इससे तुम्हारा ध्येय यह हो कि सांसारिक जीवन का माल प्राप्त करो। अल्लाह ने तुमपर उपकार किया, जो अच्छी तरह पता लगा लिया करो। जो कुछ तुम करते हो अल्लाह उसकी पूरी ख़बर रखता है
\end{hindi}}
\flushright{\begin{Arabic}
\quranayah[4][95]
\end{Arabic}}
\flushleft{\begin{hindi}
ईमानवालों में से वे लोग जो बिना किसी कारण के बैठे रहते है और जो अल्लाह के मार्ग में अपने धन और प्राणों के साथ जी-तोड़ कोशिश करते है, दोनों समान नहीं हो सकते। अल्लाह ने बैठे रहनेवालों की अपेक्षा अपने धन और प्राणों से जी-तोड़ कोशिश करनेवालों का दर्जा बढ़ा रखा है। यद्यपि प्रत्यके के लिए अल्लाह ने अच्छे बदले का वचन दिया है। परन्तु अल्लाह ने जी-तोड़ कोशिश करनेवालों का बड़ा बदला रखा है
\end{hindi}}
\flushright{\begin{Arabic}
\quranayah[4][96]
\end{Arabic}}
\flushleft{\begin{hindi}
उसकी ओर से दर्जे है और क्षमा और दयालुता है। और अल्लाह अत्यन्त क्षमाशील, दयावान है
\end{hindi}}
\flushright{\begin{Arabic}
\quranayah[4][97]
\end{Arabic}}
\flushleft{\begin{hindi}
जो लोग अपने-आप पर अत्याचार करते है, जब फ़रिश्ते उस दशा में उनके प्राण ग्रस्त कर लेते है, तो कहते है, "तुम किस दशा में पड़े रहे?" वे कहते है, "हम धरती में निर्बल और बेबस थे।" फ़रिश्ते कहते है, "क्या अल्लाह की धरती विस्तृत न थी कि तुम उसमें घर-बार छोड़कर कहीं ओर चले जाते?" तो ये वही लोग है जिनका ठिकाना जहन्नम है। - और वह बहुत ही बुरा ठिकाना है
\end{hindi}}
\flushright{\begin{Arabic}
\quranayah[4][98]
\end{Arabic}}
\flushleft{\begin{hindi}
सिवाय उन बेबस पुरुषों, स्त्रियों और बच्चों के जिनके बस में कोई उपाय नहीं और न कोई राह पा रहे है;
\end{hindi}}
\flushright{\begin{Arabic}
\quranayah[4][99]
\end{Arabic}}
\flushleft{\begin{hindi}
तो सम्भव है कि अल्लाह ऐसे लोगों को छोड़ दे; क्योंकि अल्लाह छोड़ देनेवाला और बड़ा क्षमाशील है
\end{hindi}}
\flushright{\begin{Arabic}
\quranayah[4][100]
\end{Arabic}}
\flushleft{\begin{hindi}
जो कोई अल्लाह के मार्ग में घरबार छोड़कर निकलेगा, वह धरती में शरण लेने की बहुत जगह और समाई पाएगा, और जो कोई अपने घर में सब कुछ छोड़कर अल्लाह और उसके रसूल की ओर निकले और उसकी मृत्यु हो जाए, तो उसका प्रतिदान अल्लाह के ज़िम्मे हो गया। अल्लाह बहुत क्षमाशील, दयावान है
\end{hindi}}
\flushright{\begin{Arabic}
\quranayah[4][101]
\end{Arabic}}
\flushleft{\begin{hindi}
और जब तुम धरती में यात्रा करो, तो इसमें तुमपर कोई गुनाह नहीं कि नमाज़ को कुछ संक्षिप्त कर दो; यदि तुम्हें इस बात का भय हो कि विधर्मी लोग तुम्हें सताएँगे और कष्ट पहुँचाएँगे। निश्चय ही विधर्मी लोग तुम्हारे खुले शत्रु है
\end{hindi}}
\flushright{\begin{Arabic}
\quranayah[4][102]
\end{Arabic}}
\flushleft{\begin{hindi}
और जब तुम उनके बीच हो और (लड़ाई की दशा में) उन्हें नमाज़ पढ़ाने के लिए खड़े हो, जो चाहिए कि उनमें से एक गिरोह के लोग तुम्हारे साथ खड़े हो जाएँ और अपने हथियार साथ लिए रहें, और फिर जब वे सजदा कर लें तो उन्हें चाहिए कि वे हटकर तुम्हारे पीछे हो जाएँ और दूसरे गिरोंह के लोग, जिन्होंने अभी नमाज़ नही पढ़ी, आएँ और तुम्हारे साथ नमाज़ पढ़े, और उन्हें भी चाहिए कि वे भी अपने बचाव के सामान और हथियार लिए रहें। विधर्मी चाहते ही है कि वे भी अपने हथियारों और सामान से असावधान हो जाओ तो वे तुम पर एक साथ टूट पड़े। यदि वर्षा के कारण तुम्हें तकलीफ़ होती हो या तुम बीमार हो, तो तुम्हारे लिए कोई गुनाह नहीं कि अपने हथियार रख दो, फिर भी अपनी सुरक्षा का सामान लिए रहो। अल्लाह ने विधर्मियों के लिए अपमानजनक यातना तैयार कर रखी है
\end{hindi}}
\flushright{\begin{Arabic}
\quranayah[4][103]
\end{Arabic}}
\flushleft{\begin{hindi}
फिर जब तुम नमाज़ अदा कर चुको तो खड़े, बैठे या लेटे अल्लाह को याद करते रहो। फिर जब तुम्हें इतमीनान हो जाए तो विधिवत रूप से नमाज़ पढ़ो। निस्संदेह ईमानवालों पर समय की पाबन्दी के साथ नमाज़ पढना अनिवार्य है
\end{hindi}}
\flushright{\begin{Arabic}
\quranayah[4][104]
\end{Arabic}}
\flushleft{\begin{hindi}
और उन लोगों का पीछा करने में सुस्ती न दिखाओ। यदि तुम्हें दुख पहुँचता है; तो उन्हें भी दुख पहुँचता है, जिस तरह तुमको दुख पहुँचता है। और तुम अल्लाह से उस चीज़ की आशा करते हो, जिस चीज़ की वे आशा नहीं करते। अल्लाह तो सब कुछ जाननेवाला, तत्वदर्शी है
\end{hindi}}
\flushright{\begin{Arabic}
\quranayah[4][105]
\end{Arabic}}
\flushleft{\begin{hindi}
निस्संदेह हमने यह किताब हक़ के साथ उतारी है, ताकि अल्लाह ने जो कुछ तुम्हें दिखाया है उसके अनुसार लोगों के बीच फ़ैसला करो। और तुम विश्वासघाती लोगों को ओर से झगड़नेवाले न बनो
\end{hindi}}
\flushright{\begin{Arabic}
\quranayah[4][106]
\end{Arabic}}
\flushleft{\begin{hindi}
अल्लाह से क्षमा की प्रार्थना करो। निस्संदेह अल्लाह बहुत क्षमाशील, दयावान है
\end{hindi}}
\flushright{\begin{Arabic}
\quranayah[4][107]
\end{Arabic}}
\flushleft{\begin{hindi}
और तुम उन लोगों की ओर से न झगड़ो जो स्वयं अपनों के साथ विश्वासघात करते है। अल्लाह को ऐसा व्यक्ति प्रिय नहीं है जो विश्वासघाती, हक़ मारनेवाला हो
\end{hindi}}
\flushright{\begin{Arabic}
\quranayah[4][108]
\end{Arabic}}
\flushleft{\begin{hindi}
वे लोगों से तो छिपते है, परन्तु अल्लाह से नहीं छिपते। वह तो (उस समय भी) उनके साथ होता है, जब वे रातों में उस बात की गुप्त-मंत्रणा करते है जो उनकी इच्छा के विरुद्ध होती है। जो कुछ वे करते है, वह अल्लाह (के ज्ञान) से आच्छदित है
\end{hindi}}
\flushright{\begin{Arabic}
\quranayah[4][109]
\end{Arabic}}
\flushleft{\begin{hindi}
हाँ, ये तुम ही हो, जिन्होंने सांसारिक जीवन में उनको ओर से झगड़ लिया, परन्तु क़ियामत के दिन उनकी ओर से अल्लाह से कौन झगड़ेगा या कौन उनका वकील होगा?
\end{hindi}}
\flushright{\begin{Arabic}
\quranayah[4][110]
\end{Arabic}}
\flushleft{\begin{hindi}
और जो कोई बुरा कर्म कर बैठे या अपने-आप पर अत्याचार करे, फिर अल्लाह से क्षमा की प्रार्थना करे, तो अल्लाह को बड़ा क्षमाशील, दयावान पाएगा
\end{hindi}}
\flushright{\begin{Arabic}
\quranayah[4][111]
\end{Arabic}}
\flushleft{\begin{hindi}
और जो व्यक्ति गुनाह कमाए, तो वह अपने ही लिए कमाता है। अल्लाह तो सर्वज्ञ, तत्वदर्शी है
\end{hindi}}
\flushright{\begin{Arabic}
\quranayah[4][112]
\end{Arabic}}
\flushleft{\begin{hindi}
और जो व्यक्ति कोई ग़लती या गुनाह की कमाई करे, फिर उसे किसी निर्दोष पर थोप दे, तो उसने एक बड़े लांछन और खुले गुनाह का बोझ अपने ऊपर ले लिया
\end{hindi}}
\flushright{\begin{Arabic}
\quranayah[4][113]
\end{Arabic}}
\flushleft{\begin{hindi}
यदि तुमपर अल्लाह का उदार अनुग्रह और उसकी दयालुता न होती तो उनमें से कुछ लोग तो यह निश्चय कर ही चुके थे कि तुम्हें राह से भटका दें, हालाँकि वे अपने आप ही को पथभ्रष्टि कर रहे है, और तुम्हें वे कोई हानि नहीं पहुँचा सकते। अल्लाह ने तुमपर किताब और हिकमत (तत्वदर्शिता) उतारी है और उसने तुम्हें वह कुछ सिखाया है जो तुम जानते न थे। अल्लाह का तुमपर बहुत बड़ा अनुग्रह है
\end{hindi}}
\flushright{\begin{Arabic}
\quranayah[4][114]
\end{Arabic}}
\flushleft{\begin{hindi}
उनकी अधिकतर काना-फूसियों में कोई भलाई नहीं होती। हाँ, जो व्यक्ति सदक़ा देने या भलाई करने या लोगों के बीच सुधार के लिए कुछ कहे, तो उसकी बात और है। और जो कोई यह काम अल्लाह की प्रसन्नता प्राप्त करने के लिए करेगा, उसे हम निश्चय ही बड़ा प्रतिदान प्रदान करेंगे
\end{hindi}}
\flushright{\begin{Arabic}
\quranayah[4][115]
\end{Arabic}}
\flushleft{\begin{hindi}
और जो क्यक्ति, इसके पश्चात भी मार्गदर्शन खुलकर उसके सामने आ गया है, रसूल का विरोध करेगा और ईमानवालों के मार्ग के अतिरिक्त किसी और मार्ग पर चलेगा तो उसे हम उसी पर चलने देंगे, जिसको उसने अपनाया होगा और उसे जहन्नम में झोंक देंगे, और वह बहुत ही बुरा ठिकाना है
\end{hindi}}
\flushright{\begin{Arabic}
\quranayah[4][116]
\end{Arabic}}
\flushleft{\begin{hindi}
निस्संदेह अल्लाह इस चीज़ को क्षमा नहीं करेगा कि उसके साथ किसी को शामिल किया जाए। हाँ, इससे नीचे दर्जे के अपराध को, जिसके लिए चाहेगा, क्षमा कर देगा। जो अल्लाह के साथ किसी को साझी ठहराता है, तो वह भटककर बहुत दूर जा पड़ा
\end{hindi}}
\flushright{\begin{Arabic}
\quranayah[4][117]
\end{Arabic}}
\flushleft{\begin{hindi}
वे अल्लाह से हटकर बस कुछ देवियों को पुकारते है। और वे तो बस सरकश शैतान को पुकारते है;
\end{hindi}}
\flushright{\begin{Arabic}
\quranayah[4][118]
\end{Arabic}}
\flushleft{\begin{hindi}
जिसपर अल्लाह की फिटकार है। उसने कहा था, "मैं तेरे बन्दों में से एख निश्चित हिस्सा लेकर रहूँगा
\end{hindi}}
\flushright{\begin{Arabic}
\quranayah[4][119]
\end{Arabic}}
\flushleft{\begin{hindi}
"और उन्हें अवश्य ही भटकाऊँगा और उन्हें कामनाओं में उलझाऊँगा, और उन्हें हुक्म दूँगा तो वे चौपायों के कान फाड़ेगे, और उन्हें मैं सुझाव दूँगा तो वे अल्लाह की संरचना में परिवर्तन करेंगे।" और जिसने अल्लाह से हटकर शैतान को अपना संरक्षक और मित्र बनाया, वह खुले घाटे में पड़ गया
\end{hindi}}
\flushright{\begin{Arabic}
\quranayah[4][120]
\end{Arabic}}
\flushleft{\begin{hindi}
वह उनसे वादा करता है और उन्हें कामनाओं में उलझाए रखता है, हालाँकि शैतान उनसे जो कुछ वादा करता है वह एक धोके के सिवा कुछ भी नहीं होता
\end{hindi}}
\flushright{\begin{Arabic}
\quranayah[4][121]
\end{Arabic}}
\flushleft{\begin{hindi}
वही लोग है जिनका ठिकाना जहन्नम है और वे उससे अलग होने की कोई जगह न पाएँगे
\end{hindi}}
\flushright{\begin{Arabic}
\quranayah[4][122]
\end{Arabic}}
\flushleft{\begin{hindi}
रहे वे लोग जो ईमान लाए और उन्होंने अच्छे कर्म किए, उन्हें हम जल्द ही ऐसे बाग़ों में दाख़िल करेंगे, जिनके नीचे नहरें बह रही होंगी, जहाँ वे सदैव रहेंगे। अल्लाह का वादा सच्चा है, और अल्लाह से बढ़कर बात का सच्चा कौन हो सकता है?
\end{hindi}}
\flushright{\begin{Arabic}
\quranayah[4][123]
\end{Arabic}}
\flushleft{\begin{hindi}
बात न तुम्हारी कामनाओं की है और न किताबवालों की कामनाओं की। जो भी बुराई करेगा उसे उसका फल मिलेगा और वह अल्लाह से हटकर न तो कोई मित्र पाएगा और न ही सहायक
\end{hindi}}
\flushright{\begin{Arabic}
\quranayah[4][124]
\end{Arabic}}
\flushleft{\begin{hindi}
किन्तु जो अच्छे कर्म करेगा, चाहे पुरुष हो या स्त्री, यदि वह ईमानवाला है तो ऐसे लोग जन्नत में दाख़िल होंगे। और उनका हक़ रत्ती भर भी मारा नहीं जाएगा
\end{hindi}}
\flushright{\begin{Arabic}
\quranayah[4][125]
\end{Arabic}}
\flushleft{\begin{hindi}
और दीन (धर्म) की स्पष्ट से उस व्यक्ति से अच्छा कौन हो सकता है, जिसने अपने आपको अल्लाह के आगे झुका दिया और इबराहीम के तरीक़े का अनुसरण करे, जो सबसे कटकर एक का हो गया था? अल्लाह ने इबराहीम को अपना घनिष्ठ मित्र बनाया था
\end{hindi}}
\flushright{\begin{Arabic}
\quranayah[4][126]
\end{Arabic}}
\flushleft{\begin{hindi}
जो कुछ आकाशों में और जो कुछ धरती में है, वह अल्लाह ही का है और अल्लाह हर चीज़ को घेरे हुए है
\end{hindi}}
\flushright{\begin{Arabic}
\quranayah[4][127]
\end{Arabic}}
\flushleft{\begin{hindi}
लोग तुमसे स्त्रियों के विषय में पूछते है, कहो, "अल्लाह तुम्हें उनके विषय में हुक्म देता है और जो आयतें तुमको इस किताब में पढ़कर सुनाई जाती है, वे उन स्त्रियों के, अनाथों के विषय में भी है, जिनके हक़ तुम अदा नहीं करते। और चाहते हो कि तुम उनके साथ विवाह कर लो और कमज़ोर यतीम बच्चों के बारे में भी यही आदेश है। और इस विषय में भी कि तुम अनाथों के विषय में इनसाफ़ पर क़ायम रहो। जो भलाई भी तुम करोगे तो निश्चय ही, अल्लाह उसे भली-भाँति जानता होगा।"
\end{hindi}}
\flushright{\begin{Arabic}
\quranayah[4][128]
\end{Arabic}}
\flushleft{\begin{hindi}
यदि किसी स्त्री को अपने पति की और से दुर्व्यवहार या बेरुख़ी का भय हो, तो इसमें उनके लिए कोई दोष नहीं कि वे दोनों आपस में मेल-मिलाप की कोई राह निकाल ले। और मेल-मिलाव अच्छी चीज़ है। और मन तो लोभ एवं कृपणता के लिए उद्यत रहता है। परन्तु यदि तुम अच्छा व्यवहार करो और (अल्लाह का) भय रखो, तो अल्लाह को निश्चय ही जो कुछ तुम करोगे उसकी खबर रहेगी
\end{hindi}}
\flushright{\begin{Arabic}
\quranayah[4][129]
\end{Arabic}}
\flushleft{\begin{hindi}
और चाहे तुम कितना ही चाहो, तुममें इसकी सामर्थ्य नहीं हो सकती कि औरतों के बीच पूर्ण रूप से न्याय कर सको। तो ऐसा भी न करो कि किसी से पूर्णरूप से फिर जाओ, जिसके परिणामस्वरूप वह ऐसी हो जाए, जैसे उसका पति खो गया हो। परन्तु यदि तुम अपना व्यवहार ठीक रखो और (अल्लाह से) डरते रहो, तो निस्संदेह अल्लाह भी बड़ा क्षमाशील, दयावान है
\end{hindi}}
\flushright{\begin{Arabic}
\quranayah[4][130]
\end{Arabic}}
\flushleft{\begin{hindi}
और यदि दोनों अलग ही हो जाएँ तो अल्लाह अपनी समाई से एक को दूसरे से बेपररवाह कर देगा। अल्लाह बड़ी समाईवाला, तत्वदर्शी है
\end{hindi}}
\flushright{\begin{Arabic}
\quranayah[4][131]
\end{Arabic}}
\flushleft{\begin{hindi}
आकाशों और धरती में जो कुछ है, सब अल्लाह ही का है। तुमसे पहले जिन्हें किताब दी गई थी, उन्हें और तुम्हें भी हमने ताकीद की है कि "अल्लाह का डर रखो।" यदि तुम इनकार करते हो, तो इससे क्या होने का? आकाशों और धरती में जो कुछ है, सब अल्लाह ही का रहेगा। अल्लाह तो निस्पृह, प्रशंसनीय है
\end{hindi}}
\flushright{\begin{Arabic}
\quranayah[4][132]
\end{Arabic}}
\flushleft{\begin{hindi}
हाँ, आकाशों और धरती में जो कुछ है, अल्लाह ही का है और अल्लाह कार्यसाधक की हैसियत से काफ़ी है
\end{hindi}}
\flushright{\begin{Arabic}
\quranayah[4][133]
\end{Arabic}}
\flushleft{\begin{hindi}
ऐ लोगों! यदि वह चाहे तो तुम्हें हटा दे और तुम्हारी जगह दूसरों को ले आए। अल्लाह को इसकी पूरी सामर्थ्य है
\end{hindi}}
\flushright{\begin{Arabic}
\quranayah[4][134]
\end{Arabic}}
\flushleft{\begin{hindi}
जो कोई दुनिया का बदला चाहता है, तो अल्लाह के पास दुनिया का बदला भी है और आख़िरत का भी। अल्लाह सब कुछ सुनता, देखता है
\end{hindi}}
\flushright{\begin{Arabic}
\quranayah[4][135]
\end{Arabic}}
\flushleft{\begin{hindi}
ऐ ईमान लानेवालो! अल्लाह के लिए गवाही देते हुए इनसाफ़ पर मज़बूती के साथ जमे रहो, चाहे वह स्वयं तुम्हारे अपने या माँ-बाप और नातेदारों के विरुद्ध ही क्यों न हो। कोई धनवान हो या निर्धन (जिसके विरुद्ध तुम्हें गवाही देनी पड़े) अल्लाह को उनसे (तुमसे कहीं बढ़कर) निकटता का सम्बन्ध है, तो तुम अपनी इच्छा के अनुपालन में न्याय से न हटो, क्योंकि यदि तुम हेर-फेर करोगे या कतराओगे, तो जो कुछ तुम करते हो अल्लाह को उसकी ख़बर रहेगी
\end{hindi}}
\flushright{\begin{Arabic}
\quranayah[4][136]
\end{Arabic}}
\flushleft{\begin{hindi}
ऐ ईमान लानेवालो! अल्लाह पर ईमान लाओ और उसके रसूल पर और उस किताब पर जो उसने अपने रसूल पर उतारी है और उस किताब पर भी, जिसको वह इसके पहले उतार चुका है। और जिस किसी ने भी अल्लाह और उसके फ़रिश्तों और उसकी किताबों और उसके रसूलों और अन्तिम दिन का इनकार किया, तो वह भटककर बहुत दूर जा पड़ा
\end{hindi}}
\flushright{\begin{Arabic}
\quranayah[4][137]
\end{Arabic}}
\flushleft{\begin{hindi}
रहे वे लोग जो ईमान लाए, फिर इनकार किया; फिर ईमान लाए, फिर इनकार किया; फिर इनकार की दशा में बढते चले गए तो अल्लाह उन्हें कदापि क्षमा नहीं करेगा और न उन्हें राह दिखाएगा
\end{hindi}}
\flushright{\begin{Arabic}
\quranayah[4][138]
\end{Arabic}}
\flushleft{\begin{hindi}
मुनाफ़िको (कपटाचारियों) को मंगल-सूचना दे दो कि उनके लिए दुखद यातना है;
\end{hindi}}
\flushright{\begin{Arabic}
\quranayah[4][139]
\end{Arabic}}
\flushleft{\begin{hindi}
जो ईमानवालों को छोड़कर इनकार करनेवालों को अपना मित्र बनाते है। क्या उन्हें उनके पास प्रतिष्ठा की तलाश है? प्रतिष्ठा तो सारी की सारी अल्लाह ही के लिए है
\end{hindi}}
\flushright{\begin{Arabic}
\quranayah[4][140]
\end{Arabic}}
\flushleft{\begin{hindi}
वह 'किताब' में तुमपर यह हुक्म उतार चुका है कि जब तुम सुनो कि अल्लाह की आयतों का इनकार किया जा रहा है और उसका उपहास किया जा रहा है, तो जब तब वे किसी दूसरी बात में न लगा जाएँ, उनके साथ न बैठो, अन्यथा तुम भी उन्हीं के जैसे होगे; निश्चय ही अल्लाह कपटाचारियों और इनकार करनेवालों - सबको जहन्नम में एकत्र करनेवाला है
\end{hindi}}
\flushright{\begin{Arabic}
\quranayah[4][141]
\end{Arabic}}
\flushleft{\begin{hindi}
जो तुम्हारे मामले में प्रतीक्षा करते है, यदि अल्लाह की ओर से तुम्हारी विजय़ हुई तो कहते है, "क्या हम तुम्हार साथ न थे?" और यदि विधर्मियों के हाथ कुछ लगा तो कहते है, "क्या हमने तुम्हें घेर नहीं लिया था और ईमानवालों से बचाया नहीं?" अतः अल्लाह क़ियामत के दिन तुम्हारे बीच फ़ैसला कर देगा, और अल्लाह विधर्मियों को ईमानवालों के मुक़ाबले में कोई राह नहीं देगा
\end{hindi}}
\flushright{\begin{Arabic}
\quranayah[4][142]
\end{Arabic}}
\flushleft{\begin{hindi}
कपटाचारी अल्लाह के साथ धोखबाज़ी कर रहे है, हालाँकि उसी ने उन्हें धोखे में डाल रखा है। जब वे नमाज़ के लिए खड़े होते है तो कसमसाते हुए, लोगों को दिखाने के लिए खड़े होते है। और अल्लाह को थोड़े ही याद करते है
\end{hindi}}
\flushright{\begin{Arabic}
\quranayah[4][143]
\end{Arabic}}
\flushleft{\begin{hindi}
इसी के बीच डाँवाडोल हो रहे है, न इन (ईमानवालों) की तरफ़ के है, न इन (इनकार करनेवालों) की तरफ़ के। जिसे अल्लाह भटका दे, उसके लिए तो तुम कोई राह नहीं पा सकते
\end{hindi}}
\flushright{\begin{Arabic}
\quranayah[4][144]
\end{Arabic}}
\flushleft{\begin{hindi}
ऐ ईमान लानेवालो! ईमानवालों से हटकर इनकार करनेवालों को अपना मित्र न बनाओ। क्या तुम चाहते हो कि अल्लाह का स्पष्टा तर्क अपने विरुद्ध जुटाओ?
\end{hindi}}
\flushright{\begin{Arabic}
\quranayah[4][145]
\end{Arabic}}
\flushleft{\begin{hindi}
निस्संदेह कपटाचारी आग (जहन्नम) के सबसे निचले खंड में होंगे, और तुम कदापि उनका कोई सहायक न पाओगे
\end{hindi}}
\flushright{\begin{Arabic}
\quranayah[4][146]
\end{Arabic}}
\flushleft{\begin{hindi}
उन लोगों की बात और है जिन्होंने तौबा कर ली और अपने को सुधार लिया और अल्लाह को मज़बूती से पकड़ लिया और अपने दीन (धर्म) में अल्लाह ही के हो रहे। ऐसे लोग ईमानवालों के साथ है और अल्लाह ईमानवालों को शीघ्र ही बड़ा प्रतिदान प्रदान करेगा
\end{hindi}}
\flushright{\begin{Arabic}
\quranayah[4][147]
\end{Arabic}}
\flushleft{\begin{hindi}
अल्लाह को तुम्हें यातना देकर क्या करना है, यदि तुम कृतज्ञता दिखलाओ और ईमान लाओ? अल्लाह गुणग्राहक, सब कुछ जाननेवाला है
\end{hindi}}
\flushright{\begin{Arabic}
\quranayah[4][148]
\end{Arabic}}
\flushleft{\begin{hindi}
अल्लाह बुरी बात खुल्लम-खुल्ला कहने को पसन्द नहीं करता, मगर उसकी बात और है जिसपर ज़ुल्म किया गया हो। अल्लाह सब कुछ सुनता, जानता है
\end{hindi}}
\flushright{\begin{Arabic}
\quranayah[4][149]
\end{Arabic}}
\flushleft{\begin{hindi}
यदि तुम खुले रूप में नेकी और भलाई करो या उसे छिपाओ या किसी बुराई को क्षमा कर दो, तो अल्लाह भी क्षमा करनेवाला, सामर्थ्यवान है
\end{hindi}}
\flushright{\begin{Arabic}
\quranayah[4][150]
\end{Arabic}}
\flushleft{\begin{hindi}
जो लोग अल्लाह और उसके रसूलों का इनकार करते है और चाहते है कि अल्लाह और उसके रसूलों के बीच विच्छेद करें, और कहते है कि "हम कुछ को मानते है और कुछ को नहीं मानते" और इस तरह वे चाहते है कि बीच की कोई राह अपनाएँ;
\end{hindi}}
\flushright{\begin{Arabic}
\quranayah[4][151]
\end{Arabic}}
\flushleft{\begin{hindi}
वही लोग पक्के इनकार करनेवाले है और हमने इनकार करनेवालों के लिए अपमानजनक यातना तैयार कर रखी है
\end{hindi}}
\flushright{\begin{Arabic}
\quranayah[4][152]
\end{Arabic}}
\flushleft{\begin{hindi}
रहे वे लोग जो अल्लाह और उसके रसूलों पर ईमान रखते है और उनमें से किसी को उस सम्बन्ध में पृथक नहीं करते जो उनके बीच पाया जाता है, ऐसे लोगों को अल्लाह शीघ्र ही उनके प्रतिदान प्रदान करेगा। अल्लाह बड़ा क्षमाशील, दयावान है
\end{hindi}}
\flushright{\begin{Arabic}
\quranayah[4][153]
\end{Arabic}}
\flushleft{\begin{hindi}
किताबवालों की तुमसे माँग है कि तुम उनपर आकाश से कोई किताब उतार लाओ, तो वे तो मूसा से इससे भी बड़ी माँग कर चुके है। उन्होंने कहा था, "हमें अल्लाह को प्रत्यक्ष दिखा दो," तो उनके इस अपराध पर बिजली की कड़क ने उन्हें आ दबोचा। फिर वे बछड़े को अपना उपास्य बना बैठे, हालाँकि उनके पास खुली-खुली निशानियाँ आ चुकी थी। फिर हमने उसे भी क्षमा कर दिया और मूसा का स्पष्टा बल एवं प्रभाव प्रदान किया
\end{hindi}}
\flushright{\begin{Arabic}
\quranayah[4][154]
\end{Arabic}}
\flushleft{\begin{hindi}
और उन लोगों से वचन लेने के साथ (तूर) पहाड़ को उनपर उठा दिया और उनसे कहा, "दरवाज़े में सजदा करते हुए प्रवेश करो।" और उनसे कहा, "सब्त (सामूहिक इबादत का दिन) के विषय में ज़्यादती न करना।" और हमने उनसे बहुत-ही दृढ़ वचन लिया था
\end{hindi}}
\flushright{\begin{Arabic}
\quranayah[4][155]
\end{Arabic}}
\flushleft{\begin{hindi}
फिर उनके अपने वचन भंग करने और अल्लाह की आयतों का इनकार करने के कारण और नबियों को नाहक़ क़त्ल करने और उनके यह कहने के कारण कि "हमारे हृदय आवरणों में सुरक्षित है" - नहीं, बल्कि वास्तव में उनके इनकार के कारण अल्लाह ने उनके दिलों पर ठप्पा लगा दिया है। तो ये ईमान थोड़े ही लाते है
\end{hindi}}
\flushright{\begin{Arabic}
\quranayah[4][156]
\end{Arabic}}
\flushleft{\begin{hindi}
और उनके इनकार के कारण और मरयम के ख़िलाफ ऐसी बात करने पर जो एक बड़ा लांछन था -
\end{hindi}}
\flushright{\begin{Arabic}
\quranayah[4][157]
\end{Arabic}}
\flushleft{\begin{hindi}
और उनके इस कथन के कारण कि हमने मरयम के बेटे ईसा मसीह, अल्लाह के रसूल, को क़त्ल कर डाला - हालाँकि न तो इन्होंने उसे क़त्ल किया और न उसे सूली पर चढाया, बल्कि मामला उनके लिए संदिग्ध हो गया। और जो लोग इसमें विभेद कर रहे है, निश्चय ही वे इस मामले में सन्देह में थे। अटकल पर चलने के अतिरिक्त उनके पास कोई ज्ञान न था। निश्चय ही उन्होंने उसे (ईसा को) क़त्ल नहीं किया,
\end{hindi}}
\flushright{\begin{Arabic}
\quranayah[4][158]
\end{Arabic}}
\flushleft{\begin{hindi}
बल्कि उसे अल्लाह ने अपनी ओर उठा लिया। और अल्लाह अत्यन्त प्रभुत्वशाली, तत्वदर्शी है
\end{hindi}}
\flushright{\begin{Arabic}
\quranayah[4][159]
\end{Arabic}}
\flushleft{\begin{hindi}
किताबवालों में से कोई ऐसा न होगा, जो उसकी मृत्यु से पहले उसपर ईमान न ले आए। वह क़ियामत के दिन उनपर गवाह होगा
\end{hindi}}
\flushright{\begin{Arabic}
\quranayah[4][160]
\end{Arabic}}
\flushleft{\begin{hindi}
सारांश यह कि यहूदियों के अत्याचार के कारण हमने बहुत-सी अच्छी पाक चीज़े उनपर हराम कर दी, जो उनके लिए हलाल थी और उनके प्रायः अल्लाह के मार्ग से रोकने के कारण;
\end{hindi}}
\flushright{\begin{Arabic}
\quranayah[4][161]
\end{Arabic}}
\flushleft{\begin{hindi}
और उनके ब्याज लेने के कारण, जबकि उन्हें इससे रोका गया था। और उनके अवैध रूप से लोगों के माल खाने के कारण ऐसा किया गया और हमने उनमें से जिन लोगों ने इनकार किया उनके लिए दुखद यातना तैयार कर रखी है
\end{hindi}}
\flushright{\begin{Arabic}
\quranayah[4][162]
\end{Arabic}}
\flushleft{\begin{hindi}
परन्तु उनमें से जो लोग ज्ञान में पक्के है और ईमानवाले हैं, वे उस पर ईमान रखते है जो तुम्हारी ओर उतारा गया है और जो तुमसे पहले उतारा गया था, और जो विशेष रूप से नमाज़ क़ायम करते है, ज़कात देते और अल्लाह और अन्तिम दिन पर ईमान रखते है। यही लोग है जिन्हें हम शीघ्र ही बड़ी प्रतिदान प्रदान करेंगे
\end{hindi}}
\flushright{\begin{Arabic}
\quranayah[4][163]
\end{Arabic}}
\flushleft{\begin{hindi}
हमने तुम्हारी ओर उसी प्रकार वह्यड की है जिस प्रकार नूह और उसके बाद के नबियों की ओर वह्यु की। और हमने इबराहीम, इसमाईल, इसहाक़ और याक़ूब और उसकी सन्तान और ईसा और अय्यूब और यूनुस और हारून और सुलैमान की ओर भी वह्यक की। और हमने दाउद को ज़बूर प्रदान किया
\end{hindi}}
\flushright{\begin{Arabic}
\quranayah[4][164]
\end{Arabic}}
\flushleft{\begin{hindi}
और कितने ही रसूल हुए जिनका वृतान्त पहले हम तुमसे बयान कर चुके है और कितने ही ऐसे रसूल हुए जिनका वृतान्त हमने तुमसे बयान नहीं किया। और मूसा से अल्लाह ने बातचीत की, जिस प्रकार बातचीत की जाती है
\end{hindi}}
\flushright{\begin{Arabic}
\quranayah[4][165]
\end{Arabic}}
\flushleft{\begin{hindi}
रसूल शुभ समाचार देनेवाले और सचेत करनेवाले बनाकर भेजे गए है, ताकि रसूलों के पश्चात लोगों के पास अल्लाह के मुक़ाबले में (अपने निर्दोष होने का) कोई तर्क न रहे। अल्लाह अत्यन्त प्रभुत्वशाली, तत्वदर्शी है
\end{hindi}}
\flushright{\begin{Arabic}
\quranayah[4][166]
\end{Arabic}}
\flushleft{\begin{hindi}
परन्तु अल्लाह गवाही देता है कि उसके द्वारा जो उसने तुम्हारी ओर उतारा है कि उसे उसने अपने ज्ञान के साथ उतारा है और फ़रिश्ते भी गवाही देते है, यद्यपि अल्लाह का गवाह होना ही काफ़ी है
\end{hindi}}
\flushright{\begin{Arabic}
\quranayah[4][167]
\end{Arabic}}
\flushleft{\begin{hindi}
निश्चय ही, जिन लोगों ने इनकार किया और अल्लाह के मार्ग से रोका, वे भटककर बहुत दूर जा पड़े
\end{hindi}}
\flushright{\begin{Arabic}
\quranayah[4][168]
\end{Arabic}}
\flushleft{\begin{hindi}
जिन लोगों ने इनकार किया और ज़ुल्म पर उतर आए, उन्हें अल्लाह कदापि क्षमा नहीं करेगा और न उन्हें कोई मार्ग दिखाएगा
\end{hindi}}
\flushright{\begin{Arabic}
\quranayah[4][169]
\end{Arabic}}
\flushleft{\begin{hindi}
सिवाय जहन्नम के मार्ग के, जिसमें वे सदैव पड़े रहेंगे। और यह अल्लाह के लिए बहुत-ही सहज बात है
\end{hindi}}
\flushright{\begin{Arabic}
\quranayah[4][170]
\end{Arabic}}
\flushleft{\begin{hindi}
ऐ लोगों! रसूल तुम्हारे पास तुम्हारे रब की ओर से सत्य लेकर आ गया है। अतः तुम उस भलाई को मानो जो तुम्हारे लिए जुटाई गई। और यदि तुम इनकार करते हो तो आकाशों और धरती में जो कुछ है, वह अल्लाह ही का है। और अल्लाह सब कुछ जाननेवाला, तत्वदर्शी है
\end{hindi}}
\flushright{\begin{Arabic}
\quranayah[4][171]
\end{Arabic}}
\flushleft{\begin{hindi}
ऐ किताबवालों! अपने धर्म में हद से आगे न बढ़ो और अल्लाह से जोड़कर सत्य के अतिरिक्त कोई बात न कहो। मरयम का बेटा मसीह-ईसा इसके अतिरिक्त कुछ नहीं कि अल्लाह का रसूल है और उसका एक 'कलिमा' है, जिसे उसने मरमय की ओर भेजा था। और उसकी ओर से एक रूह है। तो तुम अल्लाह पर और उसके रसूलों पर ईमान लाओ और "तीन" न कहो - बाज़ आ जाओ! यह तुम्हारे लिए अच्छा है - अल्लाह तो केवल अकेला पूज्य है। यह उसकी महानता के प्रतिकूल है कि उसका कोई बेटा हो। आकाशों और धरती में जो कुछ है, उसी का है। और अल्लाह कार्यसाधक की हैसियत से काफ़ी है
\end{hindi}}
\flushright{\begin{Arabic}
\quranayah[4][172]
\end{Arabic}}
\flushleft{\begin{hindi}
मसीह ने कदापि अपने लिए बुरा नहीं समझा कि वह अल्लाह का बन्दा हो और न निकटवर्ती फ़रिश्तों ने ही (इसे बुरा समझा) । और जो कोई अल्लाह की बन्दगी को अपने लिए बुरा समझेगा और घमंड करेगा, तो वह (अल्लाह) उन सभी लोगों को अपने पास इकट्ठा करके रहेगा
\end{hindi}}
\flushright{\begin{Arabic}
\quranayah[4][173]
\end{Arabic}}
\flushleft{\begin{hindi}
अतः जो लोग ईमान लाए और उन्होंने अच्छे कर्म किए, जो अल्लाह उन्हें उनका पूरा-पूरा बदला देगा और अपने उदार अनुग्रह से उन्हें और अधिक प्रदान करेगा। और जिन लोगों ने बन्दगी को बुरा समझा और घमंड किया, तो उन्हें वह दुखद यातना देगा। और वे अल्लाह से बच सकने के लिए न अपना कोई निकट का समर्थक पाएँगे और न ही कोई सहायक
\end{hindi}}
\flushright{\begin{Arabic}
\quranayah[4][174]
\end{Arabic}}
\flushleft{\begin{hindi}
ऐ लोगों! तुम्हारे पास तुम्हारे रब की ओर से खुला प्रमाण आ चुका है और हमने तुम्हारी ओर एक स्पष्ट प्रकाश उतारा है
\end{hindi}}
\flushright{\begin{Arabic}
\quranayah[4][175]
\end{Arabic}}
\flushleft{\begin{hindi}
तो रहे वे लोग जो अल्लाह पर ईमान लाए और उसे मज़बूती के साथ पकड़े रहे, उन्हें तो शीघ्र ही अपनी दयालुता और अपने उदार अनुग्रह के क्षेत्र में दाख़िल करेगा और उन्हें अपनी ओर का सीधा मार्ग दिया देगा
\end{hindi}}
\flushright{\begin{Arabic}
\quranayah[4][176]
\end{Arabic}}
\flushleft{\begin{hindi}
वे तुमसे आदेश मालूम करना चाहते है। कह दो, "अल्लाह तुम्हें ऐसे व्यक्ति के विषय में, जिसका कोई वारिस न हो, आदेश देता है - यदि किसी पुरुष की मृत्यु हो जाए जिसकी कोई सन्तान न हो, परन्तु उसकी एक बहन हो, तो जो कुछ उसने छोड़ा है उसका आधा हिस्सा उस बहन का होगा। और भाई बहन का वारिस होगा, यदि उस (बहन) की कोई सन्तान न हो। और यदि (वारिस) दो बहनें हो, तो जो कुछ उसने छोड़ा है, उसमें से उनके लिए दो-तिहाई होगा। और यदि कई भाई-बहन (वारिस) हो तो एक पुरुष को हिस्सा दो स्त्रियों के बराबर होगा।" अल्लाह तुम्हारे लिए आदेशों को स्पष्ट करता है, ताकि तुम न भटको। और अल्लाह को हर चीज का पूरा ज्ञान है
\end{hindi}}
\chapter{Al-Ma'idah (The Food)}
\begin{Arabic}
\Huge{\centerline{\basmalah}}\end{Arabic}
\flushright{\begin{Arabic}
\quranayah[5][1]
\end{Arabic}}
\flushleft{\begin{hindi}
ऐ ईमान लानेवालो! प्रतिबन्धों (प्रतिज्ञाओं, समझौतों आदि) का पूर्ण रूप से पालन करो। तुम्हारे लिए चौपायों की जाति के जानवर हलाल हैं सिवाय उनके जो तुम्हें बताए जा रहें हैं; लेकिन जब तुम इहराम की दशा में हो तो शिकार को हलाल न समझना। निस्संदेह अल्लाह जो चाहते है, आदेश देता है
\end{hindi}}
\flushright{\begin{Arabic}
\quranayah[5][2]
\end{Arabic}}
\flushleft{\begin{hindi}
ऐ ईमान लानेवालो! अल्लाह की निशानियों का अनादर न करो; न आदर के महीनों का, न क़ुरबानी के जानवरों का और न जानवरों का जिनका गरदनों में पट्टे पड़े हो और न उन लोगों का जो अपने रब के अनुग्रह और उसकी प्रसन्नता की चाह में प्रतिष्ठित गृह (काबा) को जाते हो। और जब इहराम की दशा से बाहर हो जाओ तो शिकार करो। और ऐसा न हो कि एक गिरोह की शत्रुता, जिसने तुम्हारे लिए प्रतिष्ठित घर का रास्ता बन्द कर दिया था, तुम्हें इस बात पर उभार दे कि तुम ज़्यादती करने लगो। हक़ अदा करने और ईश-भय के काम में तुम एक-दूसरे का सहयोग करो और हक़ मारने और ज़्यादती के काम में एक-दूसरे का सहयोग न करो। अल्लाह का डर रखो; निश्चय ही अल्लाह बड़ा कठोर दंड देनेवाला है
\end{hindi}}
\flushright{\begin{Arabic}
\quranayah[5][3]
\end{Arabic}}
\flushleft{\begin{hindi}
तुम्हारे लिए हराम हुआ मुर्दार रक्त, सूअर का मांस और वह जानवर जिसपर अल्लाह के अतिरिक्त किसी और का नाम लिया गया हो और वह जो घुटकर या चोट खाकर या ऊँचाई से गिरकर या सींग लगने से मरा हो या जिसे किसी हिंसक पशु ने फाड़ खाया हो - सिवाय उसके जिसे तुमने ज़बह कर लिया हो - और वह किसी थान पर ज़बह कियी गया हो। और यह भी (तुम्हारे लिए हराम हैं) कि तीरो के द्वारा किस्मत मालूम करो। यह आज्ञा का उल्लंघन है - आज इनकार करनेवाले तुम्हारे धर्म की ओर से निराश हो चुके हैं तो तुम उनसे न डरो, बल्कि मुझसे डरो। आज मैंने तुम्हारे धर्म को पूर्ण कर दिया और तुमपर अपनी नेमत पूरी कर दी और मैंने तुम्हारे धर्म के रूप में इस्लाम को पसन्द किया - तो जो कोई भूख से विवश हो जाए, परन्तु गुनाह की ओर उसका झुकाव न हो, तो निश्चय ही अल्लाह अत्यन्त क्षमाशील, दयावान है
\end{hindi}}
\flushright{\begin{Arabic}
\quranayah[5][4]
\end{Arabic}}
\flushleft{\begin{hindi}
वे तुमसे पूछते है कि "उनके लिए क्या हलाल है?" कह दो, "तुम्हारे लिए सारी अच्छी स्वच्छ चीज़ें हलाल है और जिन शिकारी जानवरों को तुमने सधे हुए शिकारी जानवर के रूप में सधा रखा हो - जिनको जैस अल्लाह ने तुम्हें सिखाया हैं, सिखाते हो - वे जिस शिकार को तुम्हारे लिए पकड़े रखे, उसको खाओ और उसपर अल्लाह का नाम लो। और अल्लाह का डर रखो। निश्चय ही अल्लाह जल्द हिसाब लेनेवाला है।"
\end{hindi}}
\flushright{\begin{Arabic}
\quranayah[5][5]
\end{Arabic}}
\flushleft{\begin{hindi}
आज तुम्हारे लिए अच्छी स्वच्छ चीज़ें हलाल कर दी गई और जिन्हें किताब दी गई उनका भोजन भी तुम्हारे लिए हलाल है और तुम्हारा भोजन उनके लिए हलाल है और शरीफ़ और स्वतंत्र ईमानवाली स्त्रियाँ भी जो तुमसे पहले के किताबवालों में से हो, जबकि तुम उनका हक़ (मेहर) देकर उन्हें निकाह में लाओ। न तो यह काम स्वछन्द कामतृप्ति के लिए हो और न चोरी-छिपे याराना करने को। और जिस किसी ने ईमान से इनकार किया, उसका सारा किया-धरा अकारथ गया और वह आख़िरत में भी घाटे में रहेगा
\end{hindi}}
\flushright{\begin{Arabic}
\quranayah[5][6]
\end{Arabic}}
\flushleft{\begin{hindi}
ऐ ईमान लेनेवालो! जब तुम नमाज़ के लिए उठो तो अपने चहरों को और हाथों को कुहनियों तक धो लिया करो और अपने सिरों पर हाथ फेर लो और अपने पैरों को भी टखनों तक धो लो। और यदि नापाक हो तो अच्छी तरह पाक हो जाओ। परन्तु यदि बीमार हो या सफ़र में हो या तुममें से कोई शौच करके आया हो या तुमने स्त्रियों को हाथ लगया हो, फिर पानी न मिले तो पाक मिट्टी से काम लो। उसपर हाथ मारकर अपने मुँह और हाथों पर फेर लो। अल्लाह तुम्हें किसी तंगी में नहीं डालना चाहता। अपितु वह चाहता हैं कि तुम्हें पवित्र करे और अपनी नेमत तुमपर पूरी कर दे, ताकि तुम कृतज्ञ बनो
\end{hindi}}
\flushright{\begin{Arabic}
\quranayah[5][7]
\end{Arabic}}
\flushleft{\begin{hindi}
और अल्लाह के उस अनुग्रह को याद करो जो उसने तुमपर किया हैं और उस प्रतिज्ञा को भी जो उसने तुमसे की है, जबकि तुमने कहा था - "हमने सुना और माना।" अल्लाह जो कुछ सीनों (दिलों) में है, उसे भी जानता हैं
\end{hindi}}
\flushright{\begin{Arabic}
\quranayah[5][8]
\end{Arabic}}
\flushleft{\begin{hindi}
ऐ ईमान लेनेवालो! अल्लाह के लिए खूब उठनेवाले, इनसाफ़ की निगरानी करनेवाले बनो और ऐसा न हो कि किसी गिरोह की शत्रुता तुम्हें इस बात पर उभार दे कि तुम इनसाफ़ करना छोड़ दो। इनसाफ़ करो, यही धर्मपरायणता से अधिक निकट है। अल्लाह का डर रखो, निश्चय ही जो कुछ तुम करते हो, अल्लाह को उसकी ख़बर हैं
\end{hindi}}
\flushright{\begin{Arabic}
\quranayah[5][9]
\end{Arabic}}
\flushleft{\begin{hindi}
जो लोग ईमान लाए और उन्होंन अच्छे कर्म किए उनसे अल्लाह का वादा है कि उनके लिए क्षमा और बड़ा प्रतिदान है
\end{hindi}}
\flushright{\begin{Arabic}
\quranayah[5][10]
\end{Arabic}}
\flushleft{\begin{hindi}
रहे वे लोग जिन्होंने इनकार किया और हमारी आयतों को झुठलाया, वही भड़कती आग में पड़नेवाले है
\end{hindi}}
\flushright{\begin{Arabic}
\quranayah[5][11]
\end{Arabic}}
\flushleft{\begin{hindi}
ऐ ईमान लेनेवालो! अल्लाह के उस अनुग्रह को याद करो जो उसने तुमपर किया है, जबकि कुछ लोगों ने तुम्हारी ओर हाथ बढ़ाने का निश्चय कर लिया था तो उसने उनके हाथ तुमसे रोक दिए। अल्लाह का डर रखो, और ईमानवालों को अल्लाह ही पर भरोसा करना चाहिए
\end{hindi}}
\flushright{\begin{Arabic}
\quranayah[5][12]
\end{Arabic}}
\flushleft{\begin{hindi}
अल्लाह ने इसराईल की सन्तान से वचन लिया था और हमने उनमें से बारह सरदार नियुक्त किए थे। और अल्लाह ने कहा, "मैं तुम्हारे साथ हूँ, यदि तुमने नमाज़ क़ायम रखी, ज़कात देते रहे, मेरे रसूलों पर ईमान लाए और उनकी सहायता की और अल्लाह को अच्छा ऋण दिया तो मैं अवश्य तुम्हारी बुराइयाँ तुमसे दूर कर दूँगा और तुम्हें निश्चय ही ऐसे बाग़ों में दाख़िल करूँगा, जिनके नीचे नहरें बह रही होगी। फिर इसके पश्चात तुमनें से जिनसे इनकार किया, तो वास्तव में वह ठीक और सही रास्ते से भटक गया।"
\end{hindi}}
\flushright{\begin{Arabic}
\quranayah[5][13]
\end{Arabic}}
\flushleft{\begin{hindi}
फिर उनके बार-बार अपने वचन को भंग कर देने के कारण हमने उनपर लानत की और उनके हृदय कठोर कर दिए। वे शब्दों को उनके स्थान से फेरकर कुछ का कुछ कर देते है और जिनके द्वारा उन्हें याद दिलाया गया था, उसका एक बड़ा भाग वे भुला बैठे। और तुम्हें उनके किसी न किसी विश्वासघात का बराबर पता चलता रहेगा। उनमें ऐसा न करनेवाले थोड़े लोग है, तो तुम उन्हें क्षमा कर दो और उन्हें छोड़ो। निश्चय ही अल्लाह को वे लोग प्रिय है जो उत्तमकर्मी है
\end{hindi}}
\flushright{\begin{Arabic}
\quranayah[5][14]
\end{Arabic}}
\flushleft{\begin{hindi}
और हमने उन लोगों से भी दृढ़ वचन लिया था, जिन्होंने कहा था कि हम नसारा (ईसाई) हैं, किन्तु जो कुछ उन्हें जिसके द्वारा याद कराया गया था उसका एक बड़ा भाग भुला बैठे। फिर हमने उनके बीच क़ियामत तक के लिए शत्रुता और द्वेष की आग भड़का दी, और अल्लाह जल्द उन्हें बता देगा, जो कुछ वे बनाते रहे थे
\end{hindi}}
\flushright{\begin{Arabic}
\quranayah[5][15]
\end{Arabic}}
\flushleft{\begin{hindi}
ऐ किताबवालों! हमारा रसूल तुम्हारे पास आ गया है। किताब की जो बातें तुम छिपाते थे, उसमें से बहुत-सी बातें वह तुम्हारे सामने खोल रहा है और बहुत-सी बातों को छोड़ देता है। तुम्हारे पास अल्लाह की ओर से प्रकाश और एक स्पष्ट किताब आ गई है,
\end{hindi}}
\flushright{\begin{Arabic}
\quranayah[5][16]
\end{Arabic}}
\flushleft{\begin{hindi}
जिसके द्वारा अल्लाह उस व्यक्ति को जो उसकी प्रसन्नता का अनुगामी है, सलामती की राहें दिखा रहा है और अपनी अनुज्ञा से ऐसे लोगों को अँधेरों से निकालकर उजाले की ओर ला रहा है और उन्हें सीधे मार्ग पर चला रहा है
\end{hindi}}
\flushright{\begin{Arabic}
\quranayah[5][17]
\end{Arabic}}
\flushleft{\begin{hindi}
निश्चय ही उन लोगों ने इनकार किया, जिन्होंने कहा, "अल्लाह तो वही मरयम का बेटा मसीह है।" कहो, "अल्लाह के आगे किसका कुछ बस चल सकता है, यदि वह मरयम का पुत्र मसीह को और उसकी माँ (मरयम) को और समस्त धरतीवालो को विनष्ट करना चाहे? और अल्लाह ही के लिए है बादशाही आकाशों और धरती की ओर जो कुछ उनके मध्य है उसकी भी। वह जो चाहता है पैदा करता है। और अल्लाह को हर चीज़ की सामर्थ्य प्राप्त है।"
\end{hindi}}
\flushright{\begin{Arabic}
\quranayah[5][18]
\end{Arabic}}
\flushleft{\begin{hindi}
यहूदी और ईसाई कहते है, "हम तो अल्लाह के बेटे और उसके चहेते है।" कहो, "फिर वह तुम्हें तुम्हारे गुनाहों पर दंड क्यों देता है? बात यह नहीं है, बल्कि तुम भी उसके पैदा किए हुए प्राणियों में से एक मनुष्य हो। वह जिसे चाहे क्षमा करे और जिसे चाहे दंड दे।" और अल्लाह ही के लिए है बादशाही आकाशों और धरती को और जो कुछ उनके बीच है वह भी, और जाना भी उसी की ओर है
\end{hindi}}
\flushright{\begin{Arabic}
\quranayah[5][19]
\end{Arabic}}
\flushleft{\begin{hindi}
ऐ किताबवालो! हमारा रसूल ऐसे समय तुम्हारे पास आया है और तुम्हारे लिए (हमारा आदेश) खोल-खोलकर बयान करता है, जबकि रसूलों के आने का सिलसिला एक मुद्दत से बन्द था, ताकि तुम यह न कह सको कि "हमारे पास कोई शुभ-समाचार देनेवाला और सचेत करनेवाला नहीं आया।" तो देखो! अब तुम्हारे पास शुभ-समाचार देनेवाला और सचेत करनेवाला आ गया है। अल्लाह को हर चीज़ की सामर्थ्य प्राप्त है
\end{hindi}}
\flushright{\begin{Arabic}
\quranayah[5][20]
\end{Arabic}}
\flushleft{\begin{hindi}
और याद करो जब मूसा ने अपनी क़ौम के लोगों से कहा था, "ऐ लोगों! अल्लाह की उस नेमत को याद करो जो उसने तुम्हें प्रदान की है। उसनें तुममें नबी पैदा किए और तुम्हें शासक बनाया और तुमको वह कुछ दिया जो संसार में किसी को नहीं दिया था
\end{hindi}}
\flushright{\begin{Arabic}
\quranayah[5][21]
\end{Arabic}}
\flushleft{\begin{hindi}
"ऐ मेरे लोगो! इस पवित्र भूमि में प्रवेश करो, जो अल्लाह ने तुम्हारे लिए लिख दी है। और पीछे न हटो, अन्यथा, घाटे में पड़ जाओगे।"
\end{hindi}}
\flushright{\begin{Arabic}
\quranayah[5][22]
\end{Arabic}}
\flushleft{\begin{hindi}
उन्होंने कहा, "ऐ मूसा! उसमें तो बड़े शक्तिशाली लोग रहते है। हम तो वहाँ कदापि नहीं जा सकते, जब तक कि वे वहाँ से निकल नहीं जाते। हाँ, यदि वे वहाँ से निकल जाएँ, तो हम अवश्य प्रविष्ट हो जाएँगे।"
\end{hindi}}
\flushright{\begin{Arabic}
\quranayah[5][23]
\end{Arabic}}
\flushleft{\begin{hindi}
उन डरनेवालों में से ही दो व्यक्ति ऐसे भी थे जिनपर अल्लाह का अनुग्रह था। उन्होंने कहा, "उन लोगों के मुक़ाबले में दरवाज़े से प्रविष्ट हो जाओ। जब तुम उसमें प्रविष्टि हो जाओगे, तो तुम ही प्रभावी होगे। अल्लाह पर भरोसा रखो, यदि तुम ईमानवाले हो।"
\end{hindi}}
\flushright{\begin{Arabic}
\quranayah[5][24]
\end{Arabic}}
\flushleft{\begin{hindi}
उन्होंने कहा, "ऐ मूसा! जब तक वे लोग वहाँ है, हम तो कदापि नहीं जाएँगे। ऐसा ही है तो जाओ तुम और तुम्हारा रब, और दोनों लड़ो। हम तो यहीं बैठे रहेंगे।"
\end{hindi}}
\flushright{\begin{Arabic}
\quranayah[5][25]
\end{Arabic}}
\flushleft{\begin{hindi}
उसने कहा, "मेरे रब! मेरा स्वयं अपने और अपने भाई के अतिरिक्त किसी पर अधिकार नहीं है। अतः तू हमारे और इन अवज्ञाकारी लोगों के बीच अलगाव पैदा कर दे।"
\end{hindi}}
\flushright{\begin{Arabic}
\quranayah[5][26]
\end{Arabic}}
\flushleft{\begin{hindi}
कहा, "अच्छा तो अब यह भूमि चालीस वर्ष कर इनके लिए वर्जित है। ये धरती में मारे-मारे फिरेंगे तो तुम इन अवज्ञाकारी लोगों के प्रति शोक न करो"
\end{hindi}}
\flushright{\begin{Arabic}
\quranayah[5][27]
\end{Arabic}}
\flushleft{\begin{hindi}
और इन्हें आदम के दो बेटों का सच्चा वृतान्त सुना दो। जब दोनों ने क़ुरबानी की, तो उनमें से एक की क़ुरबानी स्वीकृत हुई और दूसरे की स्वीकृत न हुई। उसने कहा, "मै तुझे अवश्य मार डालूँगा।" दूसरे न कहा, "अल्लाह तो उन्हीं की (क़ुरबानी) स्वीकृत करता है, जो डर रखनेवाले है।
\end{hindi}}
\flushright{\begin{Arabic}
\quranayah[5][28]
\end{Arabic}}
\flushleft{\begin{hindi}
"यदि तू मेरी हत्या करने के लिए मेरी ओर हाथ बढ़ाएगा तो मैं तेरी हत्या करने के लिए तेरी ओर अपना हाथ नहीं बढ़ाऊँगा। मैं तो अल्लाह से डरता हूँ, जो सारे संसार का रब है
\end{hindi}}
\flushright{\begin{Arabic}
\quranayah[5][29]
\end{Arabic}}
\flushleft{\begin{hindi}
"मैं तो चाहता हूँ कि मेरा गुनाह और अपना गुनाह तू ही अपने सिर ले ले, फिर आग (जहन्नम) में पड़नेवालों में से एक हो जाए, और वही अत्याचारियों का बदला है।"
\end{hindi}}
\flushright{\begin{Arabic}
\quranayah[5][30]
\end{Arabic}}
\flushleft{\begin{hindi}
अन्ततः उसके जी ने उस अपने भाई की हत्या के लिए उद्यत कर दिया, तो उसने उसकी हत्या कर डाली और घाटे में पड़ गया
\end{hindi}}
\flushright{\begin{Arabic}
\quranayah[5][31]
\end{Arabic}}
\flushleft{\begin{hindi}
तब अल्लाह ने एक कौआ भेजा जो भूमि कुरेदने लगा, ताकि उसे दिखा दे कि वह अपने भाई के शव को कैसे छिपाए। कहने लगा, "अफ़सोस मुझ पर! क्या मैं इस कौए जैसा भी न हो सका कि अपने भाई का शव छिपा देता?" फिर वह लज्जित हुआ
\end{hindi}}
\flushright{\begin{Arabic}
\quranayah[5][32]
\end{Arabic}}
\flushleft{\begin{hindi}
इसी कारण हमने इसराईल का सन्तान के लिए लिख दिया था कि जिसने किसी व्यक्ति को किसी के ख़ून का बदला लेने या धरती में फ़साद फैलाने के अतिरिक्त किसी और कारण से मार डाला तो मानो उसने सारे ही इनसानों की हत्या कर डाली। और जिसने उसे जीवन प्रदान किया, उसने मानो सारे इनसानों को जीवन दान किया। उसने पास हमारे रसूल स्पष्टि प्रमाण ला चुके हैं, फिर भी उनमें बहुत-से लोग धरती में ज़्यादतियाँ करनेवाले ही हैं
\end{hindi}}
\flushright{\begin{Arabic}
\quranayah[5][33]
\end{Arabic}}
\flushleft{\begin{hindi}
जो लोग अल्लाह और उसके रसूल से लड़ते है और धरती के लिए बिगाड़ पैदा करने के लिए दौड़-धूप करते है, उनका बदला तो बस यही है कि बुरी तरह से क़त्ल किए जाए या सूली पर चढ़ाए जाएँ या उनके हाथ-पाँव विपरीत दिशाओं में काट डाले जाएँ या उन्हें देश से निष्कासित कर दिया जाए। यह अपमान और तिरस्कार उनके लिए दुनिया में है और आख़िरत में उनके लिए बड़ी यातना है
\end{hindi}}
\flushright{\begin{Arabic}
\quranayah[5][34]
\end{Arabic}}
\flushleft{\begin{hindi}
किन्तु जो लोग, इससे पहले कि तुम्हें उनपर अधिकार प्राप्त हो, पलट आएँ (अर्थात तौबा कर लें) तो ऐसी दशा में तुम्हें मालूम होना चाहिए कि अल्लाह बड़ा क्षमाशील, दयावान है
\end{hindi}}
\flushright{\begin{Arabic}
\quranayah[5][35]
\end{Arabic}}
\flushleft{\begin{hindi}
ऐ ईमान लानेवालो! अल्लाह का डर रखो और उसका सामीप्य प्राप्त करो और उसके मार्ग में जी-तोड़ संघर्ष करो, ताकि तुम्हें सफलता प्राप्त हो
\end{hindi}}
\flushright{\begin{Arabic}
\quranayah[5][36]
\end{Arabic}}
\flushleft{\begin{hindi}
जिन लोगों ने इनकार किया यदि उनके पास वह सब कुछ हो जो सारी धरती में है और उतना ही उसके साथ भी हो कि वह उसे देकर क़ियामत के दिन की यातना से बच जाएँ; तब भी उनकी ओर से यह सब दी जानेवाली वस्तुएँ स्वीकार न की जाएँगी। उनके लिए दुखद यातना ही है
\end{hindi}}
\flushright{\begin{Arabic}
\quranayah[5][37]
\end{Arabic}}
\flushleft{\begin{hindi}
वे चाहेंगे कि आग (जहन्नम) से निकल जाएँ, परन्तु वे उससे न निकल सकेंगे। उनके लिए चिरस्थायी यातना है
\end{hindi}}
\flushright{\begin{Arabic}
\quranayah[5][38]
\end{Arabic}}
\flushleft{\begin{hindi}
और चोर चाहे स्त्री हो या पुरुष दोनों के हाथ काट दो। यह उनकी कमाई का बदला है और अल्लाह की ओर से शिक्षाप्रद दंड। अल्लाह प्रभुत्वशाली, तत्वदर्शी है
\end{hindi}}
\flushright{\begin{Arabic}
\quranayah[5][39]
\end{Arabic}}
\flushleft{\begin{hindi}
फिर जो व्यक्ति अत्याचार करने के बाद पलट आए और अपने को सुधार ले, तो निश्चय ही वह अल्लाह की कृपा का पात्र होगा। निस्संदेह, अल्लाह बड़ा क्षमाशील, दयावान है
\end{hindi}}
\flushright{\begin{Arabic}
\quranayah[5][40]
\end{Arabic}}
\flushleft{\begin{hindi}
क्या तुम नहीं जानते कि अल्लाह ही आकाशों और धरती के राज्य का अधिकारी है? वह जिसे चाहे यातना दे और जिसे चाहे क्षमा कर दे। अल्लाह को हर चीज़ की सामर्थ्य प्राप्त है
\end{hindi}}
\flushright{\begin{Arabic}
\quranayah[5][41]
\end{Arabic}}
\flushleft{\begin{hindi}
ऐ रसूल! जो लोग अधर्म के मार्ग में दौड़ते है, उनके कारण तुम दुखी न होना; वे जिन्होंने अपने मुँह से कहा कि "हम ईमान ले आए," किन्तु उनके दिल ईमान नहीं लाए; और वे जो यहूदी हैं, वे झूठ के लिए कान लगाते हैं और उन दूसरे लोगों की भली-भाँति सुनते है, जो तुम्हारे पास नहीं आए, शब्दों को उनका स्थान निश्चित होने के बाद भी उनके स्थान से हटा देते है। कहते है, "यदि तुम्हें यह (आदेश) मिले, तो इसे स्वीकार करना और यदि न मिले तो बचना।" जिसे अल्लाह ही आपदा में डालना चाहे उसके लिए अल्लाह के यहाँ तुम्हारी कुछ भी नहीं चल सकती। ये वही लोग है जिनके दिलों को अल्लाह ने स्वच्छ करना नहीं चाहा। उनके लिए संसार में भी अपमान और तिरस्कार है और आख़िरत में भी बड़ी यातना है
\end{hindi}}
\flushright{\begin{Arabic}
\quranayah[5][42]
\end{Arabic}}
\flushleft{\begin{hindi}
वे झूठ के लिए कान लगाते रहनेवाले और बड़े हराम खानेवाले है। अतः यदि वे तुम्हारे पास आएँ, तो या तुम उनके बीच फ़ैसला कर दो या उन्हें टाल जाओ। यदि तुम उन्हें टाल गए तो वे तुम्हारा कुछ भी नहीं बिगाड़ सकते। परन्तु यदि फ़ैसला करो तो उनके बीच इनसाफ़ के साथ फ़ैसला करो। निश्चय ही अल्लाह इनसाफ़ करनेवालों से प्रेम करता है
\end{hindi}}
\flushright{\begin{Arabic}
\quranayah[5][43]
\end{Arabic}}
\flushleft{\begin{hindi}
वे तुमसे फ़ैसला कराएँगे भी कैसे, जबकि उनके पास तौरात है, जिसमें अल्लाह का हुक्म मौजूद है! फिर इसके पश्चात भी वे मुँह मोड़ते है। वे तो ईमान नहीं रखते
\end{hindi}}
\flushright{\begin{Arabic}
\quranayah[5][44]
\end{Arabic}}
\flushleft{\begin{hindi}
निस्संदेह हमने तौरात उतारी, जिसमें मार्गदर्शन और प्रकाश था। नबी जो आज्ञाकारी थे, उसको यहूदियों के लिए अनिवार्य ठहराते थे कि वे उसका पालन करें और इसी प्रकार अल्लाहवाले और शास्त्रवेत्ता भी। क्योंकि उन्हें अल्लाह की किताब की सुरक्षा का आदेश दिया गया था और वे उसके संरक्षक थे। तो तुम लोगों से न डरो, बल्कि मुझ ही से डरो और मेरी आयतों के बदले थोड़ा मूल्य प्राप्त न करना। जो लोग उस विधान के अनुसार फ़ैसला न करें, जिसे अल्लाह ने उतारा है, तो ऐसे ही लोग विधर्मी है
\end{hindi}}
\flushright{\begin{Arabic}
\quranayah[5][45]
\end{Arabic}}
\flushleft{\begin{hindi}
और हमने उस (तौरात) में उनके लिए लिख दिया था कि जान जान के बराबर है, आँख आँख के बराहर है, नाक नाक के बराबर है, कान कान के बराबर, दाँत दाँत के बराबर और सब आघातों के लिए इसी तरह बराबर का बदला है। तो जो कोई उसे क्षमा कर दे तो यह उसके लिए प्रायश्चित होगा और जो लोग उस विधान के अनुसार फ़ैसला न करें, जिसे अल्लाह ने उतारा है जो ऐसे लोग अत्याचारी है
\end{hindi}}
\flushright{\begin{Arabic}
\quranayah[5][46]
\end{Arabic}}
\flushleft{\begin{hindi}
और उनके पीछ उन्हीं के पद-चिन्हों पर हमने मरयम के बेटे ईसा को भेजा जो पहले से उसके सामने मौजूद किताब 'तौरात' की पुष्टि करनेवाला था। और हमने उसे इनजील प्रदान की, जिसमें मार्गदर्शन और प्रकाश था। और वह अपनी पूर्ववर्ती किताब तौरात की पुष्टि करनेवाली थी, और वह डर रखनेवालों के लिए मार्गदर्शन और नसीहत थी
\end{hindi}}
\flushright{\begin{Arabic}
\quranayah[5][47]
\end{Arabic}}
\flushleft{\begin{hindi}
अतः इनजील वालों को चाहिए कि उस विधान के अनुसार फ़ैसला करें, जो अल्लाह ने उस इनजील में उतारा है। और जो उसके अनुसार फ़ैसला न करें, जो अल्लाह ने उतारा है, तो ऐसे ही लोग उल्लंघनकारी है
\end{hindi}}
\flushright{\begin{Arabic}
\quranayah[5][48]
\end{Arabic}}
\flushleft{\begin{hindi}
और हमने तुम्हारी ओर यह किताब हक़ के साथ उतारी है, जो उस किताब की पुष्टि करती है जो उसके पहले से मौजूद है और उसकी संरक्षक है। अतः लोगों के बीच तुम मामलों में वही फ़ैसला करना जो अल्लाह ने उतारा है और जो सत्य तुम्हारे पास आ चुका है उसे छोड़कर उनकी इच्छाओं का पालन न करना। हमने तुममें से प्रत्येक के लिए एक ही घाट (शरीअत) और एक ही मार्ग निश्चित किया है। यदि अल्लाह चाहता तो तुम सबको एक समुदाय बना देता। परन्तु जो कुछ उसने तुम्हें दिया है, उसमें वह तुम्हारी परीक्षा करना चाहता है। अतः भलाई के कामों में एक-दूसरे से आगे बढ़ो। तुम सबको अल्लाह ही की ओर लौटना है। फिर वह तुम्हें बता देगा, जिसमें तुम विभेद करते रहे हो
\end{hindi}}
\flushright{\begin{Arabic}
\quranayah[5][49]
\end{Arabic}}
\flushleft{\begin{hindi}
और यह कि तुम उनके बीच वही फ़ैसला करो जो अल्लाह ने उतारा है और उनकी इच्छाओं का पालन न करो और उनसे बचते रहो कि कहीं ऐसा न हो कि वे तुम्हें फ़रेब में डालकर जो कुछ अल्लाह ने तुम्हारी ओर उतारा है उसके किसी भाग से वे तुम्हें हटा दें। फिर यदि वे मुँह मोड़े तो जान लो कि अल्लाह ही उनके गुनाहों के कारण उन्हें संकट में डालना चाहता है। निश्चय ही अधिकांश लोग उल्लंघनकारी है
\end{hindi}}
\flushright{\begin{Arabic}
\quranayah[5][50]
\end{Arabic}}
\flushleft{\begin{hindi}
अब क्या वे अज्ञान का फ़ैसला चाहते है? तो विश्वास करनेवाले लोगों के लिए अल्लाह से अच्छा फ़ैसला करनेवाला कौन हो सकता है?
\end{hindi}}
\flushright{\begin{Arabic}
\quranayah[5][51]
\end{Arabic}}
\flushleft{\begin{hindi}
ऐ ईमान लानेवालो! तुम यहूदियों और ईसाइयों को अपना मित्र (राज़दार) न बनाओ। वे (तुम्हारे विरुद्ध) परस्पर एक-दूसरे के मित्र है। तुममें से जो कोई उनको अपना मित्र बनाएगा, वह उन्हीं लोगों में से होगा। निस्संदेह अल्लाह अत्याचारियों को मार्ग नहीं दिखाता
\end{hindi}}
\flushright{\begin{Arabic}
\quranayah[5][52]
\end{Arabic}}
\flushleft{\begin{hindi}
तो तुम देखते हो कि जिन लोगों के दिलों में रोग है, वे उनके यहाँ जाकर उनके बीच दौड़-धूप कर रहे है। वे कहते है, "हमें भय है कि कहीं हम किसी संकट में न ग्रस्त हो जाएँ।" तो सम्भव है कि जल्द ही अल्लाह (तुम्हे) विजय प्रदान करे या उसकी ओर से कोई और बात प्रकट हो। फिर तो ये लोग जो कुछ अपने जी में छिपाए हुए है, उसपर लज्जित होंगे
\end{hindi}}
\flushright{\begin{Arabic}
\quranayah[5][53]
\end{Arabic}}
\flushleft{\begin{hindi}
उस समय ईमानवाले कहेंगे, "क्या ये वही लोग है जो अल्लाह की कड़ी-कड़ी क़समें खाकर विश्वास दिलाते थे कि हम तुम्हारे साथ है?" इनका किया-धरा सब अकारथ गया और ये घाटे में पड़कर रहे
\end{hindi}}
\flushright{\begin{Arabic}
\quranayah[5][54]
\end{Arabic}}
\flushleft{\begin{hindi}
ऐ ईमान लानेवालो! तुममें से जो कोई अपने धर्म से फिरेगा तो अल्लाह जल्द ही ऐसे लोगों को लाएगा जिनसे उसे प्रेम होगा और जो उससे प्रेम करेंगे। वे ईमानवालों के प्रति नरम और अविश्वासियों के प्रति कठोर होंगे। अल्लाह की राह में जी-तोड़ कोशिश करेंगे और किसी भर्त्सना करनेवाले की भर्त्सना से न डरेंगे। यह अल्लाह का उदार अनुग्रह है, जिसे चाहता है प्रदान करता है। अल्लाह बड़ी समाईवाला, सर्वज्ञ है
\end{hindi}}
\flushright{\begin{Arabic}
\quranayah[5][55]
\end{Arabic}}
\flushleft{\begin{hindi}
तुम्हारे मित्र को केवल अल्लाह और उसका रसूल और वे ईमानवाले है; जो विनम्रता के साथ नमाज़ क़ायम करते है और ज़कात देते है
\end{hindi}}
\flushright{\begin{Arabic}
\quranayah[5][56]
\end{Arabic}}
\flushleft{\begin{hindi}
अब जो कोई अल्लाह और उसके रसूल और ईमानवालों को अपना मित्र बनाए, तो निश्चय ही अल्लाह का गिरोह प्रभावी होकर रहेगा
\end{hindi}}
\flushright{\begin{Arabic}
\quranayah[5][57]
\end{Arabic}}
\flushleft{\begin{hindi}
ऐ ईमान लानेवालो! तुमसे पहले जिनको किताब दी गई थी, जिन्होंने तुम्हारे धर्म को हँसी-खेल बना लिया है, उन्हें और इनकार करनेवालों को अपना मित्र न बनाओ। और अल्लाह का डर रखों यदि तुम ईमानवाले हो
\end{hindi}}
\flushright{\begin{Arabic}
\quranayah[5][58]
\end{Arabic}}
\flushleft{\begin{hindi}
जब तुम नमाज़ के लिए पुकारते हो तो वे उसे हँसी और खेल बना लेते है। इसका कारण यह है कि वे बुद्धिहीन लोग है
\end{hindi}}
\flushright{\begin{Arabic}
\quranayah[5][59]
\end{Arabic}}
\flushleft{\begin{hindi}
कहो, "ऐ किताबवालों! क्या इसके सिवा हमारी कोई और बात तुम्हें बुरी लगती है कि हम अल्लाह और उस चीज़ पर ईमान लाए, जो हमारी ओर उतारी गई, और जो पहले उतारी जा चुकी है? और यह कि तुममें से अधिकांश लोग अवज्ञाकारी है।"
\end{hindi}}
\flushright{\begin{Arabic}
\quranayah[5][60]
\end{Arabic}}
\flushleft{\begin{hindi}
कहो, "क्या मैं तुम्हें बताऊँ कि अल्लाह के यहाँ परिणाम की स्पष्ट से इससे भी बुरी नीति क्या है? कौन गिरोह है जिसपर अल्लाह की फिटकार पड़ी और जिसपर अल्लाह का प्रकोप हुआ और जिसमें से उसने बन्दर और सूअर बनाए और जिसने बढ़े हुए फ़सादी (ताग़ूत) की बन्दगी की, वे लोग (तुमसे भी) निकृष्ट दर्जे के थे। और वे (तुमसे भी अधिक) सीधे मार्ग से भटके हुए थे।"
\end{hindi}}
\flushright{\begin{Arabic}
\quranayah[5][61]
\end{Arabic}}
\flushleft{\begin{hindi}
जब वे (यहूदी) तुम लोगों के पास आते है तो कहते है, "हम ईमान ले आए।" हालाँकि वे इनकार के साथ आए थे और उसी के साथ चले गए। अल्लाह भली-भाँति जानता है जो कुछ वे छिपाते है
\end{hindi}}
\flushright{\begin{Arabic}
\quranayah[5][62]
\end{Arabic}}
\flushleft{\begin{hindi}
तुम देखते हो कि उनमें से बहुतेरे लोग हक़ मारने, ज़्यादती करने और हरामख़ोरी में बड़ी तेज़ी दिखाते है। निश्चय ही बहुत ही बुरा है, जो वे कर रहे है
\end{hindi}}
\flushright{\begin{Arabic}
\quranayah[5][63]
\end{Arabic}}
\flushleft{\begin{hindi}
उनके सन्त और धर्मज्ञाता उन्हें गुनाह की बात बकने और हराम खाने से क्यों नहीं रोकते? निश्चय ही बहुत बुरा है जो काम वे कर रहे है
\end{hindi}}
\flushright{\begin{Arabic}
\quranayah[5][64]
\end{Arabic}}
\flushleft{\begin{hindi}
और यहूदी कहते है, "अल्लाह का हाथ बँध गया है।" उन्हीं के हाथ-बँधे है, और फिटकार है उनपर, उस बकबास के कारण जो वे करते है, बल्कि उसके दोनो हाथ तो खुले हुए है। वह जिस तरह चाहता है, ख़र्च करता है। जो कुछ तुम्हारे रब की ओर से तुम्हारी ओर उतारा गया है, उससे अवश्य ही उनके अधिकतर लोगों की सरकशी और इनकार ही में अभिवृद्धि होगी। और हमने उनके बीच क़ियामत तक के लिए शत्रुता और द्वेष डाल दिया है। वे जब भी युद्ध की आग भड़काते है, अल्लाह उसे बुझा देता है। वे धरती में बिगाड़ फैलाने के लिए प्रयास कर रहे है, हालाँकि अल्लाह बिगाड़ फैलानेवालों को पसन्द नहीं करता
\end{hindi}}
\flushright{\begin{Arabic}
\quranayah[5][65]
\end{Arabic}}
\flushleft{\begin{hindi}
और यदि किताबवाले ईमान लाते और (अल्लाह का) डर रखते तो हम उनकी बुराइयाँ उनसे दूर कर देते और उन्हें नेमत भरी जन्नतों में दाख़िल कर देते
\end{hindi}}
\flushright{\begin{Arabic}
\quranayah[5][66]
\end{Arabic}}
\flushleft{\begin{hindi}
और यदि वे तौरात और इनजील को और जो कुछ उनके रब की ओर से उनकी ओर उतारा गया है, उसे क़ायम रखते, तो उन्हें अपने ऊपर से भी खाने को मिलता और अपने पाँव के नीचे से भी। उनमें से एक गिरोह सीधे मार्ग पर चलनेवाला भी है, किन्तु उनमें से अधिकतर ऐसे है कि जो भी करते है बुरा होता है
\end{hindi}}
\flushright{\begin{Arabic}
\quranayah[5][67]
\end{Arabic}}
\flushleft{\begin{hindi}
ऐ रसूल! तुम्हारे रब की ओर से तुम पर जो कुछ उतारा गया है, उसे पहुँचा दो। यदि ऐसा न किया तो तुमने उसका सन्देश नहीं पहुँचाया। अल्लाह तुम्हें लोगों (की बुराइयों) से बचाएगा। निश्चय ही अल्लाह इनकार करनेवाले लोगों को मार्ग नहीं दिखाता
\end{hindi}}
\flushright{\begin{Arabic}
\quranayah[5][68]
\end{Arabic}}
\flushleft{\begin{hindi}
कह दो, ॅ"ऐ किताबवालो! तुम किसी भी चीज़ पर नहीं हो, जब तक कि तौरात और इनजील को और जो कुछ तुम्हारे रब की ओर से तुम्हारी ओर अवतरित हुआ है, उसे क़ायम न रखो।" किन्तु (ऐ नबी!) तुम्हारे रब की ओर से तुम्हारी ओर जो कुछ अवतरित हुआ है, वह अवश्य ही उनमें से बहुतों की सरकशी और इनकार में अभिवृद्धि करनेवाला है। अतः तुम इनकार करनेवाले लोगों की दशा पर दुखी न होना
\end{hindi}}
\flushright{\begin{Arabic}
\quranayah[5][69]
\end{Arabic}}
\flushleft{\begin{hindi}
निस्संदेह वे लोग जो ईमान लाए है और जो यहूदी हुए है और साबई और ईसाई, उनमें से जो कोई भी अल्लाह और अन्तिम दिन पर ईमान लाए और अच्छा कर्म करे तो ऐसे लोगों को न तो कोई डर होगा और न वे शोकाकुल होंगे
\end{hindi}}
\flushright{\begin{Arabic}
\quranayah[5][70]
\end{Arabic}}
\flushleft{\begin{hindi}
हमने इसराईल की सन्तान से दृढ़ वचन लिया और उनकी ओर रसूल भेजे। उनके पास जब भी कोई रसूल वह कुछ लेकर आया जो उन्हें पसन्द न था, तो कितनों को तो उन्होंने झुठलाया और कितनों की हत्या करने लगे
\end{hindi}}
\flushright{\begin{Arabic}
\quranayah[5][71]
\end{Arabic}}
\flushleft{\begin{hindi}
और उन्होंने समझा कि कोई आपदा न आएगी; इसलिए वे अंधे और बहरे बन गए। फिर अल्लाह ने उनपर दयादृष्टि की, फिर भी उनमें से बहुत-से अंधे और बहरे हो गए। अल्लाह देख रहा है, जो कुछ वे करते है
\end{hindi}}
\flushright{\begin{Arabic}
\quranayah[5][72]
\end{Arabic}}
\flushleft{\begin{hindi}
निश्चय ही उन्होंने (सत्य का) इनकार किया, जिन्होंने कहा, "अल्लाह मरयम का बेटा मसीह ही है।" जब मसीह ने कहा था, "ऐ इसराईल की सन्तान! अल्लाह की बन्दगी करो, जो मेरा भी रब है और तुम्हारा भी रब है। जो कोई अल्लाह का साझी ठहराएगा, उसपर तो अल्लाह ने जन्नत हराम कर दी है और उसका ठिकाना आग है। अत्याचारियों को कोई सहायक नहीं।"
\end{hindi}}
\flushright{\begin{Arabic}
\quranayah[5][73]
\end{Arabic}}
\flushleft{\begin{hindi}
निश्चय ही उन्होंने इनकार किया, जिन्होंने कहा, "अल्लाह तीन में का एक है।" हालाँकि अकेले पूज्य के अतिरिक्त कोई पूज्य नहीं। जो कुछ वे कहते है यदि इससे बाज़ न आएँ तो उनमें से जिन्होंने इनकार किया है, उन्हें दुखद यातना पहुँचकर रहेगी
\end{hindi}}
\flushright{\begin{Arabic}
\quranayah[5][74]
\end{Arabic}}
\flushleft{\begin{hindi}
फिर क्या वे लोग अल्लाह की ओर नहीं पलटेंगे और उससे क्षमा याचना नहीं करेंगे, जबकि अल्लाह बड़ा क्षमाशील, दयावान है
\end{hindi}}
\flushright{\begin{Arabic}
\quranayah[5][75]
\end{Arabic}}
\flushleft{\begin{hindi}
मरयम का बेटा मसीह एक रसूल के अतिरिक्त और कुछ भी नहीं। उससे पहले भी बहुत-से रसूल गुज़र चुके हैं। उसकी माण अत्यन्त सत्यवती थी। दोनों ही भोजन करते थे। देखो, हम किस प्रकार उनके सामने निशानियाँ स्पष्ट करते है; फिर देखो, ये किस प्रकार उलटे फिरे जा रहे है!
\end{hindi}}
\flushright{\begin{Arabic}
\quranayah[5][76]
\end{Arabic}}
\flushleft{\begin{hindi}
कह दो, "क्या तुम अल्लाह से हटकर उसकी बन्दगी करते हो जो न तुम्हारी हानि का अधिकारी है, न लाभ का? हालाँकि सुननेवाला, जाननेवाला अल्लाह ही है।"
\end{hindi}}
\flushright{\begin{Arabic}
\quranayah[5][77]
\end{Arabic}}
\flushleft{\begin{hindi}
कह दो, "ऐ किताबवालो! अपने धर्म में नाहक़ हद से आगे न बढ़ो और उन लोगों की इच्छाओं का पालन न करो, जो इससे पहले स्वयं पथभ्रष्ट हुए और बहुतो को पथभ्रष्ट किया और सीधे मार्ग से भटक गए
\end{hindi}}
\flushright{\begin{Arabic}
\quranayah[5][78]
\end{Arabic}}
\flushleft{\begin{hindi}
इसराईल की सन्तान में से जिन लोगों ने इनकार किया, उनपर दाऊद और मरयम के बेटे ईसा की ज़बान से फिटकार पड़ी, क्योंकि उन्होंने अवज्ञा की और वे हद से आगे बढ़े जा रहे थे
\end{hindi}}
\flushright{\begin{Arabic}
\quranayah[5][79]
\end{Arabic}}
\flushleft{\begin{hindi}
जो बुरा काम वे करते थे, उससे वे एक-दूसरे को रोकते न थे। निश्चय ही बहुत ही बुरा था, जो वे कर रहे थे
\end{hindi}}
\flushright{\begin{Arabic}
\quranayah[5][80]
\end{Arabic}}
\flushleft{\begin{hindi}
तुम उनमें से बहुतेरे लोगों को देखते हो जो इनकार करनेवालो से मित्रता रखते है। निश्चय ही बहुत बुरा है, जो उन्होंने अपने आगे रखा है। अल्लाह का उनपर प्रकोप हुआ और यातना में वे सदैव ग्रस्त रहेंगे
\end{hindi}}
\flushright{\begin{Arabic}
\quranayah[5][81]
\end{Arabic}}
\flushleft{\begin{hindi}
और यदि वे अल्लाह और नबी पर और उस चीज़ पर ईमान लाते, जो उसकी ओर अवतरित हुईस तो वे उनको मित्र न बनाते। किन्तु उनमें अधिकतर अवज्ञाकारी है
\end{hindi}}
\flushright{\begin{Arabic}
\quranayah[5][82]
\end{Arabic}}
\flushleft{\begin{hindi}
तुम ईमानवालों का शत्रु सब लोगों से बढ़कर यहूदियों और बहुदेववादियों को पाओगे। और ईमान लानेवालो के लिए मित्रता में सबसे निकट उन लोगों को पाओगे, जिन्होंने कहा कि 'हम नसारा हैं।' यह इस कारण है कि उनमें बहुत-से धर्मज्ञाता और संसार-त्यागी सन्त पाए जाते हैं। और इस कारण कि वे अहंकार नहीं करते
\end{hindi}}
\flushright{\begin{Arabic}
\quranayah[5][83]
\end{Arabic}}
\flushleft{\begin{hindi}
जब वे उसे सुनते है जो रसूल पर अवतरित हुआ तो तुम देखते हो कि उनकी आँखे आँसुओ से छलकने लगती है। इसका कारण यह है कि उन्होंने सत्य को पहचान लिया। वे कहते हैं, "हमारे रब! हम ईमान ले आए। अतएव तू हमारा नाम गवाही देनेवालों में लिख ले
\end{hindi}}
\flushright{\begin{Arabic}
\quranayah[5][84]
\end{Arabic}}
\flushleft{\begin{hindi}
"और हम अल्लाह पर और जो सत्य हमारे पास पहुँचा है उसपर ईमान क्यों न लाएँ, जबकि हमें आशा है कि हमारा रब हमें अच्छे लोगों के साथ (जन्नत में) प्रविष्ट, करेगा।"
\end{hindi}}
\flushright{\begin{Arabic}
\quranayah[5][85]
\end{Arabic}}
\flushleft{\begin{hindi}
फिर अल्लाह ने उनके इस कथन के कारण उन्हें ऐसे बाग़ प्रदान किए, जिनके नीचे नहरें बहती है, जिनमें वे सदैव रहेंगे। और यही सत्कर्मी लोगो का बदला है
\end{hindi}}
\flushright{\begin{Arabic}
\quranayah[5][86]
\end{Arabic}}
\flushleft{\begin{hindi}
रहे वे लोग जिन्होंने इनकार किया और हमारी आयतों को झुठलाया, वे भड़कती आग (में पड़ने) वाले है
\end{hindi}}
\flushright{\begin{Arabic}
\quranayah[5][87]
\end{Arabic}}
\flushleft{\begin{hindi}
ऐ ईमान लानेवालो! जो अच्छी पाक चीज़े अल्लाह ने तुम्हारे लिए हलाल की है, उन्हें हराम न कर लो और हद से आगे न बढ़ो। निश्चय ही अल्लाह को वे लोग प्रिय नहीं है, जो हद से आगे बढ़ते है
\end{hindi}}
\flushright{\begin{Arabic}
\quranayah[5][88]
\end{Arabic}}
\flushleft{\begin{hindi}
जो कुछ अल्लाह ने हलाल और पाक रोज़ी तुम्हें ही है, उसे खाओ और अल्लाह का डर रखो, जिसपर तुम ईमान लाए हो
\end{hindi}}
\flushright{\begin{Arabic}
\quranayah[5][89]
\end{Arabic}}
\flushleft{\begin{hindi}
तुम्हारी उन क़समों पर अल्लाह तुम्हें नहीं पकड़ता जो यूँ ही असावधानी से ज़बान से निकल जाती है। परन्तु जो तुमने पक्की क़समें खाई हों, उनपर वह तुम्हें पकड़ेगा। तो इसका प्रायश्चित दस मुहताजों को औसत दर्जें का खाना खिला देना है, जो तुम अपने बाल-बच्चों को खिलाते हो या फिर उन्हें कपड़े पहनाना या एक ग़ुलाम आज़ाद करना होगा। और जिसे इसकी सामर्थ्य न हो, तो उसे तीन दिन के रोज़े रखने होंगे। यह तुम्हारी क़समों का प्रायश्चित है, जबकि तुम क़सम खा बैठो। तुम अपनी क़समों की हिफ़ाजत किया करो। इस प्रकार अल्लाह अपनी आयतें तुम्हारे सामने खोल-खोलकर बयान करता है, ताकि तुम कृतज्ञता दिखलाओ
\end{hindi}}
\flushright{\begin{Arabic}
\quranayah[5][90]
\end{Arabic}}
\flushleft{\begin{hindi}
ऐ ईमान लानेवालो! ये शराब और जुआ और देवस्थान और पाँसे तो गन्दे शैतानी काम है। अतः तुम इनसे अलग रहो, ताकि तुम सफल हो
\end{hindi}}
\flushright{\begin{Arabic}
\quranayah[5][91]
\end{Arabic}}
\flushleft{\begin{hindi}
शैतान तो बस यही चाहता है कि शराब और जुए के द्वारा तुम्हारे बीच शत्रुता और द्वेष पैदा कर दे और तुम्हें अल्लाह की याद से और नमाज़ से रोक दे, तो क्या तुम बाज़ न आओगे?
\end{hindi}}
\flushright{\begin{Arabic}
\quranayah[5][92]
\end{Arabic}}
\flushleft{\begin{hindi}
अल्लाह की आज्ञा का पालन करो और रसूल की आज्ञा का पालन करो और बचते रहो, किन्तु यदि तुमने मुँह मोड़ा तो जान लो कि हमारे रसूल पर केवल स्पष्ट रूप से (संदेश) पहुँचा देने की ज़िम्मेदारी है
\end{hindi}}
\flushright{\begin{Arabic}
\quranayah[5][93]
\end{Arabic}}
\flushleft{\begin{hindi}
जो लोग ईमान लाए और उन्होंने अच्छे कर्म किए, वे पहले जो कुछ खा-पी चुके उसके लिए उनपर कोई गुनाह नहीं; जबकि वे डर रखें और ईमान पर क़ायम रहें और अच्छे कर्म करें। फिर डर रखें और ईमान लाए, फिर डर रखे और अच्छे से अच्छा कर्म करें। अल्लाह सत्कर्मियों से प्रेम करता है
\end{hindi}}
\flushright{\begin{Arabic}
\quranayah[5][94]
\end{Arabic}}
\flushleft{\begin{hindi}
ऐ ईमान लानेवालो! अल्लाह उस शिकार के द्वारा तुम्हारी अवश्य परीक्षा लेगा जिस तक तुम्हारे हाथ और नेज़े पहुँच सकें, ताकि अल्लाह यह जान ले कि उससे बिन देखे कौन डरता है। फिर इसके पश्चात जिसने ज़्यादती की, उसके लिए दुखद यातना है
\end{hindi}}
\flushright{\begin{Arabic}
\quranayah[5][95]
\end{Arabic}}
\flushleft{\begin{hindi}
ऐ ईमान लानेवालो! इहराम की हालत में तुम शिकार न मारो। तुम में जो कोई जान-बूझकर उसे मारे, तो उसने जो जानवर मारा हो, चौपायों में से उसी जैसा एक जानवर - जिसका फ़ैसला तुम्हारे दो न्यायप्रिय व्यक्ति कर दें - काबा पहुँचाकर क़ुरबान किया जाए, या प्रायश्चित के रूप में मुहताजों को भोजन कराना होगा या उसके बराबर रोज़े रखने होंगे, ताकि वह अपने किए का मज़ा चख ले। जो पहले हो चुका उसे अल्लाह ने क्षमा कर दिया; परन्तु जिस किसी ने फिर ऐसा किया तो अल्लाह उससे बदला लेगा। अल्लाह प्रभुत्वशाली, बदला लेनेवाला है
\end{hindi}}
\flushright{\begin{Arabic}
\quranayah[5][96]
\end{Arabic}}
\flushleft{\begin{hindi}
तुम्हारे लिए जल की शिकार और उसका खाना हलाल है कि तुम उससे फ़ायदा उठाओ और मुसाफ़िर भी। किन्तु थलीय शिकार जब तक तुम इहराम में हो, तुमपर हराम है। और अल्लाह से डरते रहो, जिसकी ओर तुम इकट्ठा होगे
\end{hindi}}
\flushright{\begin{Arabic}
\quranayah[5][97]
\end{Arabic}}
\flushleft{\begin{hindi}
अल्लाह ने आदरणीय घर काबा को लोगों के लिए क़ायम रहने का साधन बनाया और आदरणीय महीनों और क़ुरबानी के जानबरों और उन जानवरों को भी जिनके गले में पट्टे बँधे हो, यह इसलिए कि तुम जान लो कि अल्लाह जानता है जो कुछ आकाशों में है और जो कुछ धरती में है। और यह कि अल्लाह हर चीज़ से अवगत है
\end{hindi}}
\flushright{\begin{Arabic}
\quranayah[5][98]
\end{Arabic}}
\flushleft{\begin{hindi}
जान लो अल्लाह कठोर दड देनेवाला है और यह कि अल्लाह बड़ा क्षमाशील, दयावान है
\end{hindi}}
\flushright{\begin{Arabic}
\quranayah[5][99]
\end{Arabic}}
\flushleft{\begin{hindi}
रसूल पर (सन्देश) पहुँचा देने के अतिरिक्त और कोई ज़िम्मेदारी नहीं। अल्लाह तो जानता है, जो कुछ तुम प्रकट करते हो और जो कुछ तुम छिपाते हो
\end{hindi}}
\flushright{\begin{Arabic}
\quranayah[5][100]
\end{Arabic}}
\flushleft{\begin{hindi}
कह दो, "बुरी चीज़ और अच्छी चीज़ समान नहीं होती, चाहे बुरी चीज़ों की बहुतायत तुम्हें प्रिय ही क्यों न लगे।" अतः ऐ बुद्धि और समझवालों! अल्लाह का डर रखो, ताकि तुम सफल हो सको
\end{hindi}}
\flushright{\begin{Arabic}
\quranayah[5][101]
\end{Arabic}}
\flushleft{\begin{hindi}
ऐ ईमान लानेवालो! ऐसी चीज़ों के विषय में न पूछो कि वे यदि तुम पर स्पष्ट कर दी जाएँ, तो तुम्हें बूरी लगें। यदि तुम उन्हें ऐसे समय में पूछोगे, जबकि क़ुरआन अवतरित हो रहा है, तो वे तुमपर स्पष्ट कर दी जाएँगी। अल्लाह ने उसे क्षमा कर दिया। अल्लाह बहुत क्षमा करनेवाला, सहनशील है
\end{hindi}}
\flushright{\begin{Arabic}
\quranayah[5][102]
\end{Arabic}}
\flushleft{\begin{hindi}
तुमसे पहले कुछ लोग इस तरह के प्रश्न कर चुके हैं, फिर वे उसके कारण इनकार करनेवाले हो गए
\end{hindi}}
\flushright{\begin{Arabic}
\quranayah[5][103]
\end{Arabic}}
\flushleft{\begin{hindi}
अल्लाह ने न कोई 'बहीरा' ठहराया है और न 'सायबा' और न 'वसीला' और न 'हाम', परन्तु इनकार करनेवाले अल्लाह पर झूठ का आरोपण करते है और उनमें अधिकतर बुद्धि से काम नहीं लेते
\end{hindi}}
\flushright{\begin{Arabic}
\quranayah[5][104]
\end{Arabic}}
\flushleft{\begin{hindi}
और जब उनसे कहा जाता है कि उस चीज़ की ओर आओ जो अल्लाह ने अवतरित की है और रसूल की ओर, तो वे कहते है, "हमारे लिए तो वही काफ़ी है, जिस पर हमने अपने बाप-दादा को पाया है।" क्या यद्यपि उनके बापृ-दादा कुछ भी न जानते रहे हों और न सीधे मार्ग पर रहे हो?
\end{hindi}}
\flushright{\begin{Arabic}
\quranayah[5][105]
\end{Arabic}}
\flushleft{\begin{hindi}
ऐ ईमान लानेवालो! तुमपर अपनी चिन्ता अनिवार्य है, जब तुम रास्ते पर हो, तो जो कोई भटक जाए वह तुम्हारा कुछ नहीं बिगाड़ सकता। अल्लाह की ओर तुम सबको लौटकर जाना है। फिर वह तुम्हें बता देगा, जो कुछ तुम करते रहे होगे
\end{hindi}}
\flushright{\begin{Arabic}
\quranayah[5][106]
\end{Arabic}}
\flushleft{\begin{hindi}
ऐ ईमान लानेवालों! जब तुममें से किसी की मृत्यु का समय आ जाए तो वसीयत के समय तुममें से दो न्यायप्रिय व्यक्ति गवाह हों, या तुम्हारे ग़ैर लोगों में से दूसरे दो व्यक्ति गवाह बन जाएँ, यह उस समय कि यदि तुम कहीं सफ़र में गए हो और मृत्यु तुमपर आ पहुँचे। यदि तुम्हें कोई सन्देह हो तो नमाज़ के पश्चात उन दोनों को रोक लो, फिर वे दोनों अल्लाह की क़समें खाएँ कि "हम इसके बदले कोई मूल्य स्वीकार करनेवाले नहीं हैं चाहे कोई नातेदार ही क्यों न हो और न हम अल्लाह की गवाही छिपाते है। निस्सन्देह ऐसा किया तो हम गुनाहगार ठहरेंगे।"
\end{hindi}}
\flushright{\begin{Arabic}
\quranayah[5][107]
\end{Arabic}}
\flushleft{\begin{hindi}
फिर यदि पता चल जाए कि उन दोनों ने हक़ मारकर अपने को गुनाह में डाल लिया है, तो उनकी जगह दूसरे दो व्यक्ति उन लोगों में से खड़े हो जाएँ, जिनका हक़ पिछले दोनों ने मारना चाहा था, फिर वे दोनों अल्लाह की क़समें खाएँ कि "हम दोनों की गवाही उन दोनों की गवाही से अधिक सच्ची है और हमने कोई ज़्यादती नहीं की है। निस्सन्देह हमने ऐसा किया तो अत्याचारियों में से होंगे।"
\end{hindi}}
\flushright{\begin{Arabic}
\quranayah[5][108]
\end{Arabic}}
\flushleft{\begin{hindi}
इसमें इसकी सम्भावना है कि वे ठीक-ठीक गवाही देंगे या डरेंगे कि उनकी क़समों के पश्चात क़समें ली जाएँगी। अल्लाह का डर रखो और सुनो। अल्लाह अवज्ञाकारी लोगों को मार्ग नहीं दिखाता
\end{hindi}}
\flushright{\begin{Arabic}
\quranayah[5][109]
\end{Arabic}}
\flushleft{\begin{hindi}
जिस दिन अल्लाह रसूलों को इकट्ठा करेगा, फिर कहेगा, "तुम्हें क्या जवाब मिला?" वे कहेंगे, "हमें कुछ नहीं मालूम। तू ही छिपी बातों को जानता है।"
\end{hindi}}
\flushright{\begin{Arabic}
\quranayah[5][110]
\end{Arabic}}
\flushleft{\begin{hindi}
जब अल्लाह कहेगा, "ऐ मरयम के बेटे ईसा! मेरे उस अनुग्रह को याद करो जो तुमपर और तुम्हारी माँ पर हुआ है। जब मैंने पवित्र आत्मा से तुम्हें शक्ति प्रदान की; तुम पालने में भी लोगों से बात करते थे और बड़ी अवस्था को पहुँचकर भी। और याद करो, जबकि मैंने तुम्हें किताब और हिकमत और तौरात और इनजील की शिक्षा दी थी। और याद करो जब तुम मेरे आदेश से मिट्टी से पक्षी का प्रारूपण करते थे; फिर उसमें फूँक मारते थे, तो वह मेरे आदेश से उड़नेवाली बन जाती थी। और तुम मेरे आदेश से मुर्दों को जीवित निकाल खड़ा करते थे। और याद करो जबकि मैंने तुमसे इसराइलियों को रोके रखा, जबकि तुम उनके पास खुली-खुली निशानियाँ लेकर पहुँचे थे, तो उनमें से जो इनकार करनेवाले थे, उन्होंने कहा, यह तो बस खुला जादू है।"
\end{hindi}}
\flushright{\begin{Arabic}
\quranayah[5][111]
\end{Arabic}}
\flushleft{\begin{hindi}
और याद करो, जब मैंने हबारियों (साथियों और शागिर्दों) के दिल में डाला कि "मुझपर और मेरे रसूल पर ईमान लाओ," तो उन्होंने कहा, "हम ईमान लाए और तुम गवाह रहो कि हम मुस्लिम है।"
\end{hindi}}
\flushright{\begin{Arabic}
\quranayah[5][112]
\end{Arabic}}
\flushleft{\begin{hindi}
और याद करो जब हवारियों ने कहा, "ऐ मरयम के बेटे ईसा! क्या तुम्हारा रब आकाश से खाने से भरा भाल उतार सकता है?" कहा, "अल्लाह से डरो, यदि तुम ईमानवाले हो।"
\end{hindi}}
\flushright{\begin{Arabic}
\quranayah[5][113]
\end{Arabic}}
\flushleft{\begin{hindi}
वे बोले, "हम चाहते हैं कि उनमें से खाएँ और हमारे हृदय सन्तुष्ट हो और हमें मालूम हो जाए कि तूने हमने सच कहा और हम उसपर गवाह रहें।"
\end{hindi}}
\flushright{\begin{Arabic}
\quranayah[5][114]
\end{Arabic}}
\flushleft{\begin{hindi}
मरयम के बेटे ईसा ने कहा, "ऐ अल्लाह, हमारे रब! हमपर आकाश से खाने से भरा खाल उतार, जो हमारे लिए और हमारे अंगलों और हमारे पिछलों के लिए ख़ुशी का कारण बने और तेरी ओर से एक निशानी हो, और हमें आहार प्रदान कर। तू सबसे अच्छा प्रदान करनेवाला है।"
\end{hindi}}
\flushright{\begin{Arabic}
\quranayah[5][115]
\end{Arabic}}
\flushleft{\begin{hindi}
अल्लाह ने कहा, "मैं उसे तुमपर उतारूँगा, फिर उसके पश्चात तुममें से जो कोई इनकार करेगा तो मैं अवश्य उसे ऐसी यातना दूँगा जो सम्पूर्ण संसार में किसी को न दूँगा।"
\end{hindi}}
\flushright{\begin{Arabic}
\quranayah[5][116]
\end{Arabic}}
\flushleft{\begin{hindi}
और याद करो जब अल्लाह कहेगा, "ऐ मरयम के बेटे ईसा! क्या तुमने लोगों से कहा था कि अल्लाह के अतिरिक्त दो और पूज्य मुझ और मेरी माँ को बना लो?" वह कहेगा, "महिमावान है तू! मुझसे यह नहीं हो सकता कि मैं यह बात कहूँ, जिसका मुझे कोई हक़ नहीं है। यदि मैंने यह कहा होता तो तुझे मालूम होता। तू जानता है, जो कुछ मेरे मन में है। परन्तु मैं नहीं जानता जो कुछ तेरे मन में है। निश्चय ही, तू छिपी बातों का भली-भाँति जाननेवाला है
\end{hindi}}
\flushright{\begin{Arabic}
\quranayah[5][117]
\end{Arabic}}
\flushleft{\begin{hindi}
"मैंने उनसे उसके सिवा और कुछ नहीं कहा, जिसका तूने मुझे आदेश दिया था, यह कि अल्लाह की बन्दगी करो, जो मेरा भी रब है और तुम्हारा भी रब है। और जब तक मैं उनमें रहा उनकी ख़बर रखता था, फिर जब तूने मुझे उठा लिया तो फिर तू ही उनका निरीक्षक था। और तू ही हर चीज़ का साक्षी है
\end{hindi}}
\flushright{\begin{Arabic}
\quranayah[5][118]
\end{Arabic}}
\flushleft{\begin{hindi}
"यदि तू उन्हें यातना दे तो वे तो तेरे ही बन्दे ही है और यदि तू उन्हें क्षमा कर दे, तो निस्सन्देह तू अत्यन्त प्रभुत्वशाली, तत्वदर्शी है।"
\end{hindi}}
\flushright{\begin{Arabic}
\quranayah[5][119]
\end{Arabic}}
\flushleft{\begin{hindi}
अल्लाह कहेगा, "यह वह दिन है कि सच्चों को उनकी सच्चाई लाभ पहुँचाएगी। उनके लिए ऐसे बाग़ है, जिनके नीचे नहेर बह रही होंगी, उनमें वे सदैव रहेंगे। अल्लाह उनसे राज़ी हुआ और वे उससे राज़ी हुए। यही सबसे बड़ी सफलता है।"
\end{hindi}}
\flushright{\begin{Arabic}
\quranayah[5][120]
\end{Arabic}}
\flushleft{\begin{hindi}
आकाशों और धरती और जो कुछ उनके बीच है, सबपर अल्लाह ही की बादशाही है और उसे हर चीज़ की सामर्थ्य प्राप्त है
\end{hindi}}
\chapter{Al-An'am (The Cattle)}
\begin{Arabic}
\Huge{\centerline{\basmalah}}\end{Arabic}
\flushright{\begin{Arabic}
\quranayah[6][1]
\end{Arabic}}
\flushleft{\begin{hindi}
प्रशंसा अल्लाह के लिए है, जिसने आकाशों और धरती को पैदा किया और अँधरों और उजाले का विधान किया; फिर भी इनकार करनेवाले लोग दूसरों को अपने रब के समकक्ष ठहराते है
\end{hindi}}
\flushright{\begin{Arabic}
\quranayah[6][2]
\end{Arabic}}
\flushleft{\begin{hindi}
वही है जिसने तुम्हें मिट्टी से पैदा किया, फिर (जीवन की) एक अवधि निश्चित कर दी और उसके यहाँ (क़ियामत की) एक अवधि और निश्चित है; फिर भी तुम संदेह करते हो!
\end{hindi}}
\flushright{\begin{Arabic}
\quranayah[6][3]
\end{Arabic}}
\flushleft{\begin{hindi}
वही अल्लाह है, आकाशों में भी और धरती में भी। वह तुम्हारी छिपी और तुम्हारी खुली बातों को जानता है, और जो कुछ तुम कमाते हो, वह उससे भी अवगत है
\end{hindi}}
\flushright{\begin{Arabic}
\quranayah[6][4]
\end{Arabic}}
\flushleft{\begin{hindi}
हाल यह है कि उनके रब की निशानियों में से कोई निशानी भी उनके पास ऐसी नहीं आई, जिससे उन्होंने मुँह न मोड़ लिया हो
\end{hindi}}
\flushright{\begin{Arabic}
\quranayah[6][5]
\end{Arabic}}
\flushleft{\begin{hindi}
उन्होंने सत्य को झुठला दिया, जबकि वह उनके पास आया। अतः जिस चीज़ को वे हँसी उड़ाते रहे हैं, जल्द ही उसके सम्बन्ध में उन्हें ख़बरे मिल जाएगी
\end{hindi}}
\flushright{\begin{Arabic}
\quranayah[6][6]
\end{Arabic}}
\flushleft{\begin{hindi}
क्या उन्होंने नहीं देखा कि उनसे पहले कितने ही गिरोहों को हम विनष्ट कर चुके है। उन्हें हमने धरती में ऐसा जमाव प्रदान किया था, जो तुम्हें नहीं प्रदान किया। और उनपर हमने आकाश को ख़ूब बरसता छोड़ दिया और उनके नीचे नहरें बहाई। फिर हमने आकाश को ख़ूब बरसता छोड़ दिया और उनके नीचे नहरें बहाई। फिर हमने उन्हें उनके गुनाहों के कारण विनष्ट़ कर दिया और उनके पश्चात दूसरे गिरोहों को उठाया
\end{hindi}}
\flushright{\begin{Arabic}
\quranayah[6][7]
\end{Arabic}}
\flushleft{\begin{hindi}
और यदि हम तुम्हारे ऊपर काग़ज़ में लिखी-लिखाई किताब भी उतार देते और उसे लोग अपने हाथों से छू भी लेते तब भी, जिन्होंने इनकार किया है, वे यही कहते, "यह तो बस एक खुला जादू हैं।"
\end{hindi}}
\flushright{\begin{Arabic}
\quranayah[6][8]
\end{Arabic}}
\flushleft{\begin{hindi}
उनका तो कहना है, "इस (नबी) पर कोई फ़रिश्ता (खुले रूप में) क्यों नहीं उतारा गया?" हालाँकि यदि हम फ़रिश्ता उतारते तो फ़ैसला हो चुका होता। फिर उन्हें कोई मुहल्लत न मिलती
\end{hindi}}
\flushright{\begin{Arabic}
\quranayah[6][9]
\end{Arabic}}
\flushleft{\begin{hindi}
यह बात भी है कि यदि हम उसे (नबी को) फ़रिश्ता बना देते तो उसे आदमी ही (के रूप का) बनाते। इस प्रकार उन्हें उसी सन्देह में डाल देते, जिस सन्देह में वे इस समय पड़े हुए है
\end{hindi}}
\flushright{\begin{Arabic}
\quranayah[6][10]
\end{Arabic}}
\flushleft{\begin{hindi}
तुमसे पहले कितने ही रसूलों की हँसी उड़ाई जा चुकी है। अन्ततः जिन लोगों ने उनकी हँसी उड़ाई थी, उन्हें उसी न आ घेरा जिस बात पर वे हँसी उड़ाते थे
\end{hindi}}
\flushright{\begin{Arabic}
\quranayah[6][11]
\end{Arabic}}
\flushleft{\begin{hindi}
कहो, "धरती में चल-फिरकर देखो कि झुठलानेवालों का क्या परिणाम हुआ!"
\end{hindi}}
\flushright{\begin{Arabic}
\quranayah[6][12]
\end{Arabic}}
\flushleft{\begin{hindi}
कहो, "आकाशों और धरती में जो कुछ है किसका है?" कह दो, "अल्लाह ही का है।" उसने दयालुता को अपने ऊपर अनिवार्य कर दिया है। निश्चय ही वह तुम्हें क़ियामत के दिन इकट्ठा करेगा, इसमें कोई सन्देह नहीं है। जिन लोगों ने अपने-आपको घाटे में डाला है, वही है जो ईमान नहीं लाते
\end{hindi}}
\flushright{\begin{Arabic}
\quranayah[6][13]
\end{Arabic}}
\flushleft{\begin{hindi}
हाँ, उसी का है जो भी रात में ठहरता है और दिन में (गतिशील होता है), और वह सब कुछ सुनता, जानता है
\end{hindi}}
\flushright{\begin{Arabic}
\quranayah[6][14]
\end{Arabic}}
\flushleft{\begin{hindi}
कहो, "क्या मैं आकाशों और धरती को पैदा करनेवाले अल्लाह के सिवा किसी और को संरक्षक बना लूँ? उसका हाल यह है कि वह खिलाता है और स्वयं नहीं खाता।" कहो, "मुझे आदेश हुआ है कि सबसे पहले मैं उसके आगे झुक जाऊँ। और (यह कि) तुम बहुदेववादियों में कदापि सम्मिलित न होना।"
\end{hindi}}
\flushright{\begin{Arabic}
\quranayah[6][15]
\end{Arabic}}
\flushleft{\begin{hindi}
कहो, "यदि मैं अपने रब की अवज्ञा करूँ, तो उस स्थिति में मुझे एक बड़े (भयानक) दिन की यातना का डर है।"
\end{hindi}}
\flushright{\begin{Arabic}
\quranayah[6][16]
\end{Arabic}}
\flushleft{\begin{hindi}
उस दिन वह जिसपर से टल गई, उसपर अल्लाह ने दया की, और यही स्पष्ट सफलता है
\end{hindi}}
\flushright{\begin{Arabic}
\quranayah[6][17]
\end{Arabic}}
\flushleft{\begin{hindi}
और यदि अल्लाह तुम्हें कोई कष्ट पहुँचाए तो उसके अतिरिक्त उसे कोई दूर करनेवाला नहीं है और यदि वह तुम्हें कोई भलाई पहुँचाए तो उसे हर चीज़ की सामर्थ्य प्राप्त है
\end{hindi}}
\flushright{\begin{Arabic}
\quranayah[6][18]
\end{Arabic}}
\flushleft{\begin{hindi}
उसे अपने बन्दों पर पूर्ण अधिकार प्राप्त है। और वह तत्वदर्शी, ख़बर रखनेवाला है
\end{hindi}}
\flushright{\begin{Arabic}
\quranayah[6][19]
\end{Arabic}}
\flushleft{\begin{hindi}
कहो, "किस चीज़ की गवाही सबसे बड़ी है?" कहो, "मेरे और तुम्हारे बीच अल्लाह गवाह है। और यह क़ुरआन मेरी ओर वह्यी (प्रकाशना) किया गया है, ताकि मैं इसके द्वारा तुम्हें सचेत कर दूँ। और जिस किसी को यह अल्लाह के साथ दूसरे पूज्य भी है?" तुम कह दो, "मैं तो इसकी गवाही नहीं देता।" कह दो, "वह तो बस अकेला पूज्य है। और तुम जो उसका साझी ठहराते हो, उससे मेरा कोई सम्बन्ध नहीं।"
\end{hindi}}
\flushright{\begin{Arabic}
\quranayah[6][20]
\end{Arabic}}
\flushleft{\begin{hindi}
जिन लोगों को हमने किताब दी है, वे उसे इस प्रकार पहचानते है, जिस प्रकार अपने बेटों को पहचानते है। जिन लोगों ने अपने आपको घाटे में डाला है, वही ईमान नहीं लाते
\end{hindi}}
\flushright{\begin{Arabic}
\quranayah[6][21]
\end{Arabic}}
\flushleft{\begin{hindi}
और उससे बढ़कर अत्याचारी कौन होगा, जो अल्लाह पर झूठ गढ़े या उसकी आयतों को झुठलाए। निस्सन्देह अत्याचारी कभी सफल नहीं हो सकते
\end{hindi}}
\flushright{\begin{Arabic}
\quranayah[6][22]
\end{Arabic}}
\flushleft{\begin{hindi}
और उस दिन को याद करो जब हम सबको इकट्ठा करेंगे; फिर बहुदेववादियों से पूछेंगे, "कहाँ है तुम्हारे ठहराए हुए साझीदार, जिनका तुम दावा किया करते थे?"
\end{hindi}}
\flushright{\begin{Arabic}
\quranayah[6][23]
\end{Arabic}}
\flushleft{\begin{hindi}
फिर उनका कोई फ़िला (उपद्रव) शेष न रहेगा। सिवाय इसके कि वे कहेंगे, "अपने रब अल्लाह की सौगन्ध! हम बहुदेववादी न थे।"
\end{hindi}}
\flushright{\begin{Arabic}
\quranayah[6][24]
\end{Arabic}}
\flushleft{\begin{hindi}
देखो, कैसा वे अपने विषय में झूठ बोले। और वह गुम होकर रह गया जो वे घड़ा करते थे
\end{hindi}}
\flushright{\begin{Arabic}
\quranayah[6][25]
\end{Arabic}}
\flushleft{\begin{hindi}
और उनमें कुछ लोग ऐसे है जो तुम्हारी ओर कान लगाते है, हालाँकि हमने तो उनके दिलों पर परदे डाल रखे है कि वे उसे समझ न सकें और उनके कानों में बोझ डाल दिया है। और वे चाहे प्रत्येक निशानी देख लें तब भी उसे मानेंगे नहीं; यहाँ तक कि जब वे तुम्हारे पास आकर तुमसे झगड़ते है, तो अविश्वास की नीति अपनानेवाले कहते है, "यह तो बस पहले को लोगों की गाथाएँ है।"
\end{hindi}}
\flushright{\begin{Arabic}
\quranayah[6][26]
\end{Arabic}}
\flushleft{\begin{hindi}
और वे उससे दूसरों को रोकते है और स्वयं भी उससे दूर रहते है। वे तो बस अपने आपको ही विनष्ट कर रहे है, किन्तु उन्हें इसका एहसास नहीं
\end{hindi}}
\flushright{\begin{Arabic}
\quranayah[6][27]
\end{Arabic}}
\flushleft{\begin{hindi}
और यदि तुम उस समय देख सकते, जब वे आग के निकट खड़े किए जाएँगे और कहेंगे, "काश! क्या ही अच्छा होता कि हम फिर लौटा दिए जाएँ (कि माने) और अपने रब की आयतों को न झुठलाएँ और माननेवालों में हो जाएँ।"
\end{hindi}}
\flushright{\begin{Arabic}
\quranayah[6][28]
\end{Arabic}}
\flushleft{\begin{hindi}
कुछ नहीं, बल्कि जो कुछ वे पहले छिपाया करते थे, वह उनके सामने आ गया। और यदि वे लौटा भी दिए जाएँ, तो फिर वही कुछ करने लगेंगे जिससे उन्हें रोका गया था। निश्चय ही वे झूठे है
\end{hindi}}
\flushright{\begin{Arabic}
\quranayah[6][29]
\end{Arabic}}
\flushleft{\begin{hindi}
और वे कहते है, "जो कुछ है बस यही हमारा सांसारिक जीवन है; हम कोई फिर उठाए जानेवाले नहीं हैं।"
\end{hindi}}
\flushright{\begin{Arabic}
\quranayah[6][30]
\end{Arabic}}
\flushleft{\begin{hindi}
और यदि तुम देख सकते जब वे अपने रब के सामने खड़े किेए जाएँगे! वह कहेगा, "क्या यह यर्थाथ नहीं है?" कहेंगे, "क्यों नही, हमारे रब की क़सम!" वह कहेगा, "अच्छा तो उस इनकार के बदले जो तुम करते रहें हो, यातना का मज़ा चखो।"
\end{hindi}}
\flushright{\begin{Arabic}
\quranayah[6][31]
\end{Arabic}}
\flushleft{\begin{hindi}
वे लोग घाटे में पड़े, जिन्होंने अल्लाह से मिलने को झुठलाया, यहाँ तक कि जब अचानक उनपर वह घड़ी आ जाएगी तो वे कहेंगे, "हाय! अफ़सोस, उस कोताही पर जो इसके विषय में हमसे हुई।" और हाल यह होगा कि वे अपने बोझ अपनी पीठों पर उठाए होंगे। देखो, कितना बुरा बोझ है जो ये उठाए हुए है!
\end{hindi}}
\flushright{\begin{Arabic}
\quranayah[6][32]
\end{Arabic}}
\flushleft{\begin{hindi}
सांसारिक जीवन तो एक खेल और तमाशे (ग़फलत) के अतिरिक्त कुछ भी नहीं है जबकि आख़िरत का घर उन लोगों के लिए अच्छा है, जो डर रखते है। तो क्या तुम बुद्धि से काम नहीं लेते?
\end{hindi}}
\flushright{\begin{Arabic}
\quranayah[6][33]
\end{Arabic}}
\flushleft{\begin{hindi}
हमें मालूम है, जो कुछ वे कहते है उससे तुम्हें दुख पहुँचता है। तो वे वास्तव में तुम्हें नहीं झुठलाते, बल्कि उन अत्याचारियो को तो अल्लाह की आयतों से इनकार है
\end{hindi}}
\flushright{\begin{Arabic}
\quranayah[6][34]
\end{Arabic}}
\flushleft{\begin{hindi}
तुमसे पहले भी बहुत-से रसूल झुठलाए जा चुके है, तो वे अपने झुठलाए जाने और कष्ट पहुँचाए जाने पर धैर्य से काम लेते रहे, यहाँ तक कि उन्हें हमारी सहायता पहुँच गई। कोई नहीं जो अल्लाह की बातों को बदल सके। तुम्हारे पास तो रसूलों की कुछ ख़बरें पहुँच ही चुकी है
\end{hindi}}
\flushright{\begin{Arabic}
\quranayah[6][35]
\end{Arabic}}
\flushleft{\begin{hindi}
और यदि उनकी विमुखता तुम्हारे लिए असहनीय है, तो यदि तुमसे हो सके कि धरती में कोई सुरंग या आकाश में कोई सीढ़ी ढूँढ़ निकालो और उनके पास कोई निशानी ले आओ, तो (ऐसा कर देखो), यदि अल्लाह चाहता तो उन सबको सीधे मार्ग पर इकट्ठा कर देता। अतः तुम उजड्ड और नादान न बनना
\end{hindi}}
\flushright{\begin{Arabic}
\quranayah[6][36]
\end{Arabic}}
\flushleft{\begin{hindi}
मानते हो वही लोग है जो सुनते है, रहे मुर्दे, तो अल्लाह उन्हें (क़ियामत के दिन) उठा खड़ा करेगा; फिर वे उसी के ओर पलटेंगे
\end{hindi}}
\flushright{\begin{Arabic}
\quranayah[6][37]
\end{Arabic}}
\flushleft{\begin{hindi}
वे यह भी कहते है, "उस (नबी) पर उसके रब की ओर से कोई निशानी क्यों नहीं उतारी गई?" कह दो, "अल्लाह को तो इसकी सामर्थ्य प्राप्त है कि कोई निशानी उतार दे; परन्तु उनमें से अधिकतर लोग नहीं जानते।"
\end{hindi}}
\flushright{\begin{Arabic}
\quranayah[6][38]
\end{Arabic}}
\flushleft{\begin{hindi}
धरती में चलने-फिरनेवाला कोई भी प्राणी हो या अपने दो परो से उड़नवाला कोई पक्षी, ये सब तुम्हारी ही तरह के गिरोह है। हमने किताब में कोई भी चीज़ नहीं छोड़ी है। फिर वे अपने रब की ओर इकट्ठे किए जाएँगे
\end{hindi}}
\flushright{\begin{Arabic}
\quranayah[6][39]
\end{Arabic}}
\flushleft{\begin{hindi}
जिन लोगों ने हमारी आयतों को झुठलाया, वे बहरे और गूँगे है, अँधेरों में पड़े हुए हैं। अल्लाह जिसे चाहे भटकने दे और जिसे चाहे सीधे मार्ग पर लगा दे
\end{hindi}}
\flushright{\begin{Arabic}
\quranayah[6][40]
\end{Arabic}}
\flushleft{\begin{hindi}
कहो, "क्या तुमने यह भी सोचा कि यदि तुमपर अल्लाह की यातना आ पड़े या वह घड़ी तुम्हारे सामने आ जाए, तो क्या अल्लाह के सिवा किसी और को पुकारोगे? बोलो, यदि तुम सच्चे हो?
\end{hindi}}
\flushright{\begin{Arabic}
\quranayah[6][41]
\end{Arabic}}
\flushleft{\begin{hindi}
"बल्कि तुम उसी को पुकारते हो - फिर जिसके लिए तुम उसे पुकारते हो, वह चाहता है तो उसे दूर कर देता है - और उन्हें भूल जाते हो जिन्हें साझीदार ठहराते हो।"
\end{hindi}}
\flushright{\begin{Arabic}
\quranayah[6][42]
\end{Arabic}}
\flushleft{\begin{hindi}
तुमसे पहले कितने ही समुदायों की ओर हमने रसूल भेजे कि उन्हें तंगियों और मुसीबतों में डाला, ताकि वे विनम्र हों
\end{hindi}}
\flushright{\begin{Arabic}
\quranayah[6][43]
\end{Arabic}}
\flushleft{\begin{hindi}
जब हमारी ओर से उनपर सख्ती आई तो फिर क्यों न विनम्र हुए? परन्तु उनके हृदय तो कठोर हो गए थे और जो कुछ वे करते थे शैतान ने उसे उनके लिए मोहक बना दिया
\end{hindi}}
\flushright{\begin{Arabic}
\quranayah[6][44]
\end{Arabic}}
\flushleft{\begin{hindi}
फिर जब उसे उन्होंने भुला दिया जो उन्हें याद दिलाई गई थी, तो हमने उनपर हर चीज़ के दरवाज़े खोल दिए; यहाँ तक कि जो कुछ उन्हें मिला था, जब वे उसमें मग्न हो गए तो अचानक हमने उन्हें पकड़ लिया, तो क्या देखते है कि वे बिल्कुल निराश होकर रह गए
\end{hindi}}
\flushright{\begin{Arabic}
\quranayah[6][45]
\end{Arabic}}
\flushleft{\begin{hindi}
इस प्रकार अत्याचारी लोगों की जड़ काटकर रख दी गई। प्रशंसा अल्लाह ही के लिए है, जो सारे संसार का रब है
\end{hindi}}
\flushright{\begin{Arabic}
\quranayah[6][46]
\end{Arabic}}
\flushleft{\begin{hindi}
कहो, "क्या तुमने यह भी सोचा कि यदि अल्लाह तुम्हारे सुनने की और तुम्हारी देखने की शक्ति छीन ले और तुम्हारे दिलों पर ठप्पा लगा दे, तो अल्लाह के सिवा कौन पूज्य है जो तुम्हें ये चीज़े लाकर दे?" देखो, किस प्रकार हम तरह-तरह से अपनी निशानियाँ बयान करते है! फिर भी वे किनारा ही खींचते जाते है
\end{hindi}}
\flushright{\begin{Arabic}
\quranayah[6][47]
\end{Arabic}}
\flushleft{\begin{hindi}
कहो, "क्या तुमने यह भी सोचा कि यदि तुमपर अचानक या प्रत्यक्षतः अल्लाह की यातना आ जाए, तो क्या अत्याचारी लोगों के सिवा कोई और विनष्ट होगा?"
\end{hindi}}
\flushright{\begin{Arabic}
\quranayah[6][48]
\end{Arabic}}
\flushleft{\begin{hindi}
हम रसूलों को केवल शुभ-सूचना देनेवाले और सचेतकर्ता बनाकर भेजते रहे है। फिर जो ईमान लाए और सुधर जाए, तो ऐसे लोगों के लिए न कोई भय है और न वे कभी दुखी होंगे
\end{hindi}}
\flushright{\begin{Arabic}
\quranayah[6][49]
\end{Arabic}}
\flushleft{\begin{hindi}
रहे वे लोग, जिन्होंने हमारी आयतों को झुठलाया, उन्हें यातना पहुँचकर रहेगी, क्योंकि वे अवज्ञा करते रहे है
\end{hindi}}
\flushright{\begin{Arabic}
\quranayah[6][50]
\end{Arabic}}
\flushleft{\begin{hindi}
कह दो, "मैं तुमसे यह नहीं कहता कि मेरे पास अल्लाह के ख़ज़ाने है, और न मैं परोक्ष का ज्ञान रखता हूँ, और न मैं तुमसे कहता हूँ कि मैं कोई फ़रिश्ता हूँ। मैं तो बस उसी का अनुपालन करता हूँ जो मेरी ओर वह्यं की जाती है।" कहो, "क्या अंधा और आँखोंवाला दोनों बराबर हो जाएँगे? क्या तुम सोच-विचार से काम नहीं लेते?"
\end{hindi}}
\flushright{\begin{Arabic}
\quranayah[6][51]
\end{Arabic}}
\flushleft{\begin{hindi}
और तुम इसके द्वारा उन लोगों को सचेत कर दो, जिन्हें इस बात का भय है कि वे अपने रब के पास इस हाल में इकट्ठा किए जाएँगे कि उसके सिवा न तो उसका कोई समर्थक होगा और न कोई सिफ़ारिश करनेवाला, ताकि वे बचें
\end{hindi}}
\flushright{\begin{Arabic}
\quranayah[6][52]
\end{Arabic}}
\flushleft{\begin{hindi}
और जो लोग अपने रब को उसकी ख़ुशी की चाह में प्रातः और सायंकाल पुकारते रहते है, ऐसे लोगों को दूर न करना। उनके हिसाब की तुमपर कुछ भी ज़िम्मेदारी नहीं है और न तुम्हारे हिसाब की उनपर कोई ज़िम्मेदारी है कि तुम उन्हें दूर करो और फिर हो जाओ अत्याचारियों में से
\end{hindi}}
\flushright{\begin{Arabic}
\quranayah[6][53]
\end{Arabic}}
\flushleft{\begin{hindi}
और इसी प्रकार हमने इनमें से एक को दूसरे के द्वारा आज़माइश में डाला, ताकि वे कहें, "क्या यही वे लोग है, जिनपर अल्लाह न हममें से चुनकर एहसान किया है ?" - क्या अल्लाह कृतज्ञ लोगों से भली-भाँति परिचित नहीं है?
\end{hindi}}
\flushright{\begin{Arabic}
\quranayah[6][54]
\end{Arabic}}
\flushleft{\begin{hindi}
और जब तुम्हारे पास वे लोग आएँ, जो हमारी आयतों को मानते है, तो कहो, "सलाम हो तुमपर! तुम्हारे रब ने दयालुता को अपने ऊपर अनिवार्य कर लिया है कि तुममें से जो कोई नासमझी से कोई बुराई कर बैठे, फिर उसके बाद पलट आए और अपना सुधार कर तो यह है वह बड़ा क्षमाशील, दयावान है।"
\end{hindi}}
\flushright{\begin{Arabic}
\quranayah[6][55]
\end{Arabic}}
\flushleft{\begin{hindi}
इसी प्रकार हम अपनी आयतें खोल-खोलकर बयान करते है (ताकि तुम हर ज़रूरी बात जान लो) और इसलिए कि अपराधियों का मार्ग स्पष्ट हो जाए
\end{hindi}}
\flushright{\begin{Arabic}
\quranayah[6][56]
\end{Arabic}}
\flushleft{\begin{hindi}
कह दो, "तुम लोग अल्लाह से हटकर जिन्हें पुकारते हो, उनकी बन्दगी करने से मुझे रोका गया है।" कहो, "मैं तुम्हारी इच्छाओं का अनुपालन नहीं करता, क्योंकि तब तो मैं मार्ग से भटक गया और मार्ग पानेवालों में से न रहा।"
\end{hindi}}
\flushright{\begin{Arabic}
\quranayah[6][57]
\end{Arabic}}
\flushleft{\begin{hindi}
कह दो, "मैं अपने रब की ओर से एक स्पष्ट प्रमाण पर क़ायम हूँ और तुमने उसे झुठला दिया है। जिस चीज़ के लिए तुम जल्दी मचा रहे हो, वह कोई मेरे पास तो नहीं है। निर्णय का सारा अधिकार अल्लाह ही को है, वही सच्ची बात बयान करता है और वही सबसे अच्छा निर्णायक है।"
\end{hindi}}
\flushright{\begin{Arabic}
\quranayah[6][58]
\end{Arabic}}
\flushleft{\begin{hindi}
कह दो, "जिस चीज़ की तुम्हें जल्दी पड़ी हुई है, यदि कहीं वह चीज़ मेरे पास होती तो मेरे और तुम्हारे बीच कभी का फ़ैसला हो चुका होता। और अल्लाह अत्याचारियों को भली-भाती जानता है।"
\end{hindi}}
\flushright{\begin{Arabic}
\quranayah[6][59]
\end{Arabic}}
\flushleft{\begin{hindi}
उसी के पास परोक्ष की कुंजियाँ है, जिन्हें उसके सिवा कोई नहीं जानता। जल और थल में जो कुछ है, उसे वह जानता है। और जो पत्ता भी गिरता है, उसे वह निश्चय ही जानता है। और धरती के अँधेरों में कोई दाना हो और कोई भी आर्द्र (गीली) और शुष्क (सूखी) चीज़ हो, निश्चय ही एक स्पष्ट किताब में मौजूद है
\end{hindi}}
\flushright{\begin{Arabic}
\quranayah[6][60]
\end{Arabic}}
\flushleft{\begin{hindi}
और वही है जो रात को तुम्हें मौत देता है और दिन में जो कुछ तुमने किया उसे जानता है। फिर वह इसलिए तुम्हें उठाता है, ताकि निश्चित अवधि पूरा हो जाए; फिर उसी की ओर तुम्हें लौटना है, फिर वह तुम्हें बता देगा जो कुछ तुम करते रहे हो
\end{hindi}}
\flushright{\begin{Arabic}
\quranayah[6][61]
\end{Arabic}}
\flushleft{\begin{hindi}
और वही अपने बन्दों पर पूरा-पूरा क़ाबू रखनेवाला है और वह तुमपर निगरानी करनेवाले को नियुक्त करके भेजता है, यहाँ तक कि जब तुममें से किसी की मृत्यु आ जाती है, जो हमारे भेजे हुए कार्यकर्त्ता उसे अपने क़ब्ज़े में कर लेते है और वे कोई कोताही नहीं करते
\end{hindi}}
\flushright{\begin{Arabic}
\quranayah[6][62]
\end{Arabic}}
\flushleft{\begin{hindi}
फिर सब अल्लाह की ओर, जो उसका वास्तविक स्वामी है, लौट जाएँगे। जान लो, निर्णय का अधिकार उसी को है और वह बहुत जल्द हिसाब लेनेवाला है
\end{hindi}}
\flushright{\begin{Arabic}
\quranayah[6][63]
\end{Arabic}}
\flushleft{\begin{hindi}
कहो, "कौन है जो थल और जल के अँधेरो से तुम्हे छुटकारा देता है, जिसे तुम गिड़गिड़ाते हुए और चुपके-चुपके पुकारने लगते हो कि यदि हमें इससे बचा लिया तो हम अवश्य की कृतज्ञ हो जाएँगे?"
\end{hindi}}
\flushright{\begin{Arabic}
\quranayah[6][64]
\end{Arabic}}
\flushleft{\begin{hindi}
कहो, "अल्लाह तुम्हें इनसे और हरके बेचैनी और पीड़ा से छुटकारा देता है, लेकिन फिर तुम उसका साझीदार ठहराने लगते हो।"
\end{hindi}}
\flushright{\begin{Arabic}
\quranayah[6][65]
\end{Arabic}}
\flushleft{\begin{hindi}
कहो, "वह इसकी सामर्थ्य रखता है कि तुमपर तुम्हारे ऊपर से या तुम्हारे पैरों के नीचे से कोई यातना भेज दे या तुम्हें टोलियों में बाँटकर परस्पर भिड़ा दे और एक को दूसरे की लड़ाई का मज़ा चखाए।" देखो, हम आयतों को कैसे, तरह-तरह से, बयान करते है, ताकि वे समझे
\end{hindi}}
\flushright{\begin{Arabic}
\quranayah[6][66]
\end{Arabic}}
\flushleft{\begin{hindi}
तुम्हारी क़ौम ने तो उसे झुठला दिया, हालाँकि वह सत्य है। कह दो, मैं "तुमपर कोई संरक्षक नियुक्त नहीं हूँ
\end{hindi}}
\flushright{\begin{Arabic}
\quranayah[6][67]
\end{Arabic}}
\flushleft{\begin{hindi}
"हर ख़बर का एक निश्चित समय है और शीघ्र ही तुम्हें ज्ञात हो जाएगा।"
\end{hindi}}
\flushright{\begin{Arabic}
\quranayah[6][68]
\end{Arabic}}
\flushleft{\begin{hindi}
और जब तुम उन लोगों को देखो, जो हमारी आयतों पर नुक्ताचीनी करने में लगे हुए है, तो उनसे मुँह फेर लो, ताकि वे किसी दूसरी बात में लग जाएँ। और यदि कभी शैतान तुम्हें भुलावे में डाल दे, तो याद आ जाने के बाद उन अत्याचारियों के पास न बैठो
\end{hindi}}
\flushright{\begin{Arabic}
\quranayah[6][69]
\end{Arabic}}
\flushleft{\begin{hindi}
उनके हिसाब के प्रति तो उन लोगो पर कुछ भी ज़िम्मेदारी नहीं, जो डर रखते है। यदि है तो बस याद दिलाने की; ताकि वे डरें
\end{hindi}}
\flushright{\begin{Arabic}
\quranayah[6][70]
\end{Arabic}}
\flushleft{\begin{hindi}
छोड़ो उन लोगों को, जिन्होंने अपने धर्म को खेल और तमाशा बना लिया है और उन्हें सांसारिक जीवन ने धोखे में डाल रखा है। और इसके द्वारा उन्हें नसीहत करते रहो कि कहीं ऐसा न हो कि कोई अपनी कमाई के कारण तबाही में पड़ जाए। अल्लाह से हटकर कोई भी नहीं, जो उसका समर्थक और सिफ़ारिश करनेवाला हो सके और यदि वह छुटकारा पाने के लिए बदले के रूप में हर सम्भव चीज़ देने लगे, तो भी वह उससे न लिया जाए। ऐसे ही लोग है, जो अपनी कमाई के कारण तबाही में पड गए। उनके लिए पीने को खौलता हुआ पानी है और दुखद यातना भी; क्योंकि वे इनकार करते रहे थे
\end{hindi}}
\flushright{\begin{Arabic}
\quranayah[6][71]
\end{Arabic}}
\flushleft{\begin{hindi}
कहो, "क्या हम अल्लाह को छोड़कर उसे पुकारने लग जाएँ जो न तो हमें लाभ पहुँचा सके और न हमें हानि पहुँचा सके और हम उलटे पाँव फिर जाएँ, जबकि अल्लाह ने हमें मार्ग पर लगा दिया है? - उस व्यक्ति की तरह जिसे शैतानों ने धरती पर भटका दिया हो और वह हैरान होकर रह गया हो। उसके कुछ साथी हो, जो उसे मार्ग की ओर बुला रहे हो कि हमारे पास चला आ!" कह दो, "मार्गदर्शन केवल अल्लाह का मार्गदर्शन है और हमें इसी बात का आदेश हुआ है कि हम सारे संसार के स्वामी को समर्पित हो जाएँ।"
\end{hindi}}
\flushright{\begin{Arabic}
\quranayah[6][72]
\end{Arabic}}
\flushleft{\begin{hindi}
और यह कि "नमाज़ क़ायम करो और उसका डर रखो। वही है, जिसके पास तुम इकट्ठे किए जाओगे,
\end{hindi}}
\flushright{\begin{Arabic}
\quranayah[6][73]
\end{Arabic}}
\flushleft{\begin{hindi}
"और वही है जिसने आकाशों और धरती को हक़ के साथ पैदा किया। और जिस समय वह किसी चीज़ को कहे, 'हो जा', तो वह उसी समय वह हो जाती है। उसकी बात सर्वथा सत्य है और जिस दिन 'सूर' (नरसिंघा) में फूँक मारी जाएगी, राज्य उसी का होगा। वह सभी छिपी और खुली चीज़ का जाननेवाला है, और वही तत्वदर्शी, ख़बर रखनेवाला है।"
\end{hindi}}
\flushright{\begin{Arabic}
\quranayah[6][74]
\end{Arabic}}
\flushleft{\begin{hindi}
और याद करो, जब इबराहीम ने अपने बाप आज़र से कहा था, "क्या तुम मूर्तियों को पूज्य बनाते हो? मैं तो तुम्हें और तुम्हारी क़ौम को खुली गुमराही में पड़ा देख रहा हूँ।"
\end{hindi}}
\flushright{\begin{Arabic}
\quranayah[6][75]
\end{Arabic}}
\flushleft{\begin{hindi}
और इस प्रकार हम इबराहीम को आकाशों और धरती का राज्य दिखाने लगे (ताकि उसके ज्ञान का विस्तार हो) और इसलिए कि उसे विश्वास हो
\end{hindi}}
\flushright{\begin{Arabic}
\quranayah[6][76]
\end{Arabic}}
\flushleft{\begin{hindi}
अतएवः जब रात उसपर छा गई, तो उसने एक तारा देखा। उसने कहा, "इसे मेरा रब ठहराते हो!" फिर जब वह छिप गया तो बोला, "छिप जानेवालों से मैं प्रेम नहीं करता।"
\end{hindi}}
\flushright{\begin{Arabic}
\quranayah[6][77]
\end{Arabic}}
\flushleft{\begin{hindi}
फिर जब उसने चाँद को चमकता हुआ देखा, तो कहा, "इसको मेरा रब ठहराते हो!" फिर जब वह छिप गया, तो कहा, "यदि मेरा रब मुझे मार्ग न दिखाता तो मैं भी पथभ्रष्ट! लोगों में सम्मिलित हो जाता।"
\end{hindi}}
\flushright{\begin{Arabic}
\quranayah[6][78]
\end{Arabic}}
\flushleft{\begin{hindi}
फिर जब उसने सूर्य को चमकता हुआ देखा, तो कहा, "इसे मेरा रब ठहराते हो! यह तो बहुत बड़ा है।" फिर जब वह भी छिप गया, तो कहा, "ऐ मेरी क़ौन के लोगो! मैं विरक्त हूँ उनसे जिनको तुम साझी ठहराते हो
\end{hindi}}
\flushright{\begin{Arabic}
\quranayah[6][79]
\end{Arabic}}
\flushleft{\begin{hindi}
"मैंने तो एकाग्र होकर अपना मुख उसकी ओर कर लिया है, जिसने आकाशों और धरती को पैदा किया। और मैं साझी ठहरानेवालों में से नहीं।"
\end{hindi}}
\flushright{\begin{Arabic}
\quranayah[6][80]
\end{Arabic}}
\flushleft{\begin{hindi}
उसकी क़ौम के लोग उससे झगड़ने लगे। उसने कहा, "क्या तुम मुझसे अल्लाह के विषय में झगड़ते हो? जबकि उसने मुझे मार्ग दिखा दिया है। मैं उनसे नहीं डरता, जिन्हें तुम उसका सहभागी ठहराते हो, बल्कि मेरा रब जो कुछ चाहता है वही पूरा होकर रहता है। प्रत्येक वस्तु मेरे रब की ज्ञान-परिधि के भीतर है। फिर क्या तुम चेतोगे नहीं?
\end{hindi}}
\flushright{\begin{Arabic}
\quranayah[6][81]
\end{Arabic}}
\flushleft{\begin{hindi}
"और मैं तुम्हारे ठहराए हुए साझीदारो से कैसे डरूँ, जबकि तुम इस बात से नहीं डरते कि तुमने उसे अल्लाह का सहभागी उस चीज़ को ठहराया है, जिसका उसने तुमपर कोई प्रमाण अवतरित नहीं किया? अब दोनों फ़रीकों में से कौन अधिक निश्चिन्त रहने का अधिकारी है? बताओ यदि तुम जानते हो
\end{hindi}}
\flushright{\begin{Arabic}
\quranayah[6][82]
\end{Arabic}}
\flushleft{\begin{hindi}
"जो लोग ईमान लाए और अपने ईमान में किसी (शिर्क) ज़ुल्म की मिलावट नहीं की, वही लोग है जो भय मुक्त है और वही सीधे मार्ग पर हैं।"
\end{hindi}}
\flushright{\begin{Arabic}
\quranayah[6][83]
\end{Arabic}}
\flushleft{\begin{hindi}
यह है हमारा वह तर्क जो हमने इबराहीम को उसकी अपनी क़ौम के मुक़ाबले में प्रदान किया था। हम जिसे चाहते है दर्जों (श्रेणियों) में ऊँचा कर देते हैं। निस्संदेह तुम्हारा रब तत्वदर्शी, सर्वज्ञ है
\end{hindi}}
\flushright{\begin{Arabic}
\quranayah[6][84]
\end{Arabic}}
\flushleft{\begin{hindi}
और हमने उसे (इबराहीम को) इसहाक़ और याक़ूब दिए; हर एक को मार्ग दिखाया - और नूह को हमने इससे पहले मार्ग दिखाया था, और उसकी सन्तान में दाऊद, सुलैमान, अय्यूब, यूसुफ़, मूसा और हारून को भी - और इस प्रकार हम शुभ-सुन्दर कर्म करनेवालों को बदला देते है -
\end{hindi}}
\flushright{\begin{Arabic}
\quranayah[6][85]
\end{Arabic}}
\flushleft{\begin{hindi}
और ज़करिया, यह्या , ईसा और इलयास को भी (मार्ग दिखलाया) । इनमें का हर एक योग्य और नेक था
\end{hindi}}
\flushright{\begin{Arabic}
\quranayah[6][86]
\end{Arabic}}
\flushleft{\begin{hindi}
और इसमाईल, अलयसअ, यूनुस और लूत को भी। इनमें से हर एक को हमने संसार के मुक़ाबले में श्रेष्ठता प्रदान की
\end{hindi}}
\flushright{\begin{Arabic}
\quranayah[6][87]
\end{Arabic}}
\flushleft{\begin{hindi}
और उनके बाप-दादा और उनकी सन्तान और उनके भाई-बन्धुओं में भी कितने ही लोगों को (मार्ग दिखाया) । और हमने उन्हें चुन लिया और उन्हें सीधे मार्ग की ओर चलाया
\end{hindi}}
\flushright{\begin{Arabic}
\quranayah[6][88]
\end{Arabic}}
\flushleft{\begin{hindi}
यह अल्लाह का मार्गदर्शन है, जिसके द्वारा वह अपने बन्दों में से जिसको चाहता है मार्ग दिखाता है, और यदि उन लोगों ने कहीं अल्लाह का साझी ठहराया होता, तो उनका सब किया-धरा अकारथ हो जाता
\end{hindi}}
\flushright{\begin{Arabic}
\quranayah[6][89]
\end{Arabic}}
\flushleft{\begin{hindi}
वे ऐसे लोग है जिन्हें हमने किताब और निर्णय-शक्ति और पैग़म्बरी प्रदान की थी (उसी प्रकार हमने मुहम्मद को भी किताब, निर्णय-शक्ति और पैग़म्बरी दी है) । फिर यदि ये लोग इसे मारने से इनकार करें, तो अब हमने इसको ऐसे लोगों को सौंपा है जो इसका इनकार नहीं करते
\end{hindi}}
\flushright{\begin{Arabic}
\quranayah[6][90]
\end{Arabic}}
\flushleft{\begin{hindi}
वे (पिछले पैग़म्बर) ऐसे लोग थे, जिन्हें अल्लाह ने मार्ग दिखाया था, तो तुम उन्हीं के मार्ग का अनुसरण करो। कह दो, "मैं तुमसे उसका कोई प्रतिदान नहीं माँगता। वह तो सम्पूर्ण संसार के लिए बस एक प्रबोध है।"
\end{hindi}}
\flushright{\begin{Arabic}
\quranayah[6][91]
\end{Arabic}}
\flushleft{\begin{hindi}
उन्होंने अल्लाह की क़द्र न जानी, जैसी उसकी क़द्र जाननी चाहिए थी, जबकि उन्होंने कहा, "अल्लाह ने किसी मनुष्य पर कुछ अवतरित ही नहीं किया है।" कहो, "फिर यह किताब किसने अवतरित की, जो मूसा लोगों के लिए प्रकाश और मार्गदर्शन के रूप में लाया था, जिसे तुम पन्ना-पन्ना करके रखते हो? उन्हें दिखाते भी हो, परन्तु बहुत-सा छिपा जाते हो। और तुम्हें वह ज्ञान दिया गया, जिसे न तुम जानते थे और न तुम्हारे बाप-दादा ही।" कह दो, "अल्लाह ही ने," फिर उन्हें छोड़ो कि वे अपनी नुक्ताचीनियों से खेलते रहें
\end{hindi}}
\flushright{\begin{Arabic}
\quranayah[6][92]
\end{Arabic}}
\flushleft{\begin{hindi}
यह किताब है जिसे हमने उतारा है; बरकतवाली है; अपने से पहले की पुष्टि में है (ताकि तुम शुभ-सूचना दो) और ताकि तुम केन्द्रीय बस्ती (मक्का) और उसके चतुर्दिक बसनेवाले लोगों को सचेत करो और जो लोग आख़िरत पर ईमान रखते है, वे इसपर भी ईमान लाते है। और वे अपनी नमाज़ की रक्षा करते है
\end{hindi}}
\flushright{\begin{Arabic}
\quranayah[6][93]
\end{Arabic}}
\flushleft{\begin{hindi}
और उस व्यक्ति से बढ़कर अत्याचारी कौन होगा, जो अल्लाह पर मिथ्यारोपण करे या यह कहे कि "मेरी ओर प्रकाशना (वह्य,) की गई है," हालाँकि उसकी ओर भी प्रकाशना न की गई हो। और वह व्यक्ति से (बढ़कर अत्याचारी कौन होगा) जो यह कहे कि "मैं भी ऐसी चीज़ उतार दूँगा, जैसी अल्लाह ने उतारी है।" और यदि तुम देख सकते, तुम अत्याचारी मृत्यु-यातनाओं में होते है और फ़रिश्ते अपने हाथ बढ़ा रहे होते है कि "निकालो अपने प्राण! आज तुम्हें अपमानजनक यातना दी जाएगी, क्योंकि तुम अल्लाह के प्रति झूठ बका करते थे और उसकी आयतों के मुक़ाबले में अकड़ते थे।"
\end{hindi}}
\flushright{\begin{Arabic}
\quranayah[6][94]
\end{Arabic}}
\flushleft{\begin{hindi}
और निश्चय ही तुम उसी प्रकार एक-एक करके हमारे पास आ गए, जिस प्रकार हमने तुम्हें पहली बार पैदा किया था। और जो कुछ हमने तुम्हें दे रखा था, उसे अपने पीछे छोड़ आए और हम तुम्हारे साथ तुम्हारे उन सिफ़ारिशियों को भी नहीं देख रहे हैं, जिनके विषय में तुम दावे से कहते थे, "वे तुम्हारे मामले में शरीक है।" तुम्हारे पारस्परिक सम्बन्ध टूट चुके है और वे सब तुमसे गुम होकर रह गए, जो दावे तुम किया करते थे
\end{hindi}}
\flushright{\begin{Arabic}
\quranayah[6][95]
\end{Arabic}}
\flushleft{\begin{hindi}
निश्चय ही अल्लाह दाने और गुठली को फाड़ निकालता है, सजीव को निर्जीव से निकालता है और निर्जीव को सजीव से निकालनेवाला है। वही अल्लाह है - फिर तुम कहाँ औंधे हुए जाते हो? -
\end{hindi}}
\flushright{\begin{Arabic}
\quranayah[6][96]
\end{Arabic}}
\flushleft{\begin{hindi}
पौ फाड़ता है, और उसी ने रात को आराम के लिए बनाया और सूर्य और चन्द्रमा को (समय के) हिसाब का साधन ठहराया। यह बड़े शक्तिमान, सर्वज्ञ का ठहराया हुआ परिणाम है
\end{hindi}}
\flushright{\begin{Arabic}
\quranayah[6][97]
\end{Arabic}}
\flushleft{\begin{hindi}
और वही है जिसने तुम्हारे लिए तारे बनाए, ताकि तुम उनके द्वारा स्थल और समुद्र के अंधकारों में मार्ग पा सको। जो लोग जानना चाहे उनके लिए हमने निशानियाँ खोल-खोलकर बयान कर दी है
\end{hindi}}
\flushright{\begin{Arabic}
\quranayah[6][98]
\end{Arabic}}
\flushleft{\begin{hindi}
और वही तो है, जिसने तुम्हें अकेली जान पैदा किया। अतः एक अवधि तक ठहरना है और फिर सौंप देना है। उन लोगों के लिए, जो समझे हमने निशानियाँ खोल-खोलकर बयान कर दी है
\end{hindi}}
\flushright{\begin{Arabic}
\quranayah[6][99]
\end{Arabic}}
\flushleft{\begin{hindi}
और वही है जिसने आकाश से पानी बरसाया, फिर हमने उसके द्वारा हर प्रकार की वनस्पति उगाई; फिर उससे हमने हरी-भरी पत्तियाँ निकाली और तने विकसित किए, जिससे हम तले-ऊपर चढे हुए दान निकालते है - और खजूर के गाभे से झुके पड़ते गुच्छे भी - और अंगूर, ज़ैतून और अनार के बाग़ लगाए, जो एक-दूसरे से भिन्न भी होते है। उसके फल को देखा, जब वह फलता है और उसके पकने को भी देखो! निस्संदेह ईमान लानेवाले लोगों को लिए इनमें बड़ी निशानियाँ है
\end{hindi}}
\flushright{\begin{Arabic}
\quranayah[6][100]
\end{Arabic}}
\flushleft{\begin{hindi}
और लोगों ने जिन्नों को अल्लाह का साझी ठहरा रखा है; हालाँकि उन्हें उसी ने पैदा किया है। और बेजाने-बूझे उनके लिए बेटे और बेटियाँ घड़ ली है। यह उसकी महिमा के प्रतिकूल है! यह उन बातों से उच्च है, जो वे बयान करते है!
\end{hindi}}
\flushright{\begin{Arabic}
\quranayah[6][101]
\end{Arabic}}
\flushleft{\begin{hindi}
वह आकाशों और धरती का सर्वप्रथम पैदा करनेवाला है। उसका कोई बेटा कैसे हो सकता है, जबकि उसकी पत्नी ही नहीं? और उसी ने हर चीज़ को पैदा किया है और उसे हर चीज़ का ज्ञान है
\end{hindi}}
\flushright{\begin{Arabic}
\quranayah[6][102]
\end{Arabic}}
\flushleft{\begin{hindi}
वही अल्लाह तुम्हारा रब; उसके सिवा कोई पूज्य नहीं; हर चीज़ का स्रष्टा है; अतः तुम उसी की बन्दगी करो। वही हर चीज़ का ज़िम्मेदार है
\end{hindi}}
\flushright{\begin{Arabic}
\quranayah[6][103]
\end{Arabic}}
\flushleft{\begin{hindi}
निगाहें उसे नहीं पा सकतीं, बल्कि वही निगाहों को पा लेता है। वह अत्यन्त सूक्ष्म (एवं सूक्ष्मदर्शी) ख़बर रखनेवाला है
\end{hindi}}
\flushright{\begin{Arabic}
\quranayah[6][104]
\end{Arabic}}
\flushleft{\begin{hindi}
तुम्हारे पास तुम्हारे रब की ओर से आँख खोल देनेवाले प्रमाण आ चुके है; तो जिस किसी ने देखा, अपना ही भला किया और जो अंधा बना रहा, तो वह अपने ही को हानि पहुँचाएगा। और मैं तुमपर कोई नियुक्त रखवाला नहीं हूँ
\end{hindi}}
\flushright{\begin{Arabic}
\quranayah[6][105]
\end{Arabic}}
\flushleft{\begin{hindi}
और इसी प्रकार हम अपनी आयतें विभिन्न ढंग से बयान करते है (कि वे सुने) और इसलिए कि वे कह लें, "(ऐ मुहम्मद!) तुमनेकहीं से पढ़-पढ़ा लिया है।" और इसलिए भी कि हम उनके लिए जो जानना चाहें, सत्य को स्पष्ट कर दें
\end{hindi}}
\flushright{\begin{Arabic}
\quranayah[6][106]
\end{Arabic}}
\flushleft{\begin{hindi}
तुम्हारे रब की ओर से तुम्हारी तरफ़ जो वह्यो की गई है, उसी का अनुसरण किए जाओ, उसके सिवा कोई पूज्य नहीं और बहुदेववादियों (की कुनीति) पर ध्यान न दो
\end{hindi}}
\flushright{\begin{Arabic}
\quranayah[6][107]
\end{Arabic}}
\flushleft{\begin{hindi}
यदि अल्लाह चाहता तो वे (उसका) साझी न ठहराते। तुम्हें हमने उनपर कोई नियुक्त संरक्षक तो नहीं बनाया है और न तुम उनके कोई ज़िम्मेदार ही हो
\end{hindi}}
\flushright{\begin{Arabic}
\quranayah[6][108]
\end{Arabic}}
\flushleft{\begin{hindi}
अल्लाह के सिवा जिन्हें ये पुकारते है, तुम उनके प्रति अपशब्द का प्रयोग न करो। ऐसा न हो कि वे हद से आगे बढ़कर अज्ञान वश अल्लाह के प्रति अपशब्द का प्रयोग करने लगें। इसी प्रकार हमने हर गिरोह के लिए उसके कर्म को सुहावना बना दिया है। फिर उन्हें अपने रब की ही ओर लौटना है। उस समय वह उन्हें बता देगा. जो कुछ वे करते रहे होंगे
\end{hindi}}
\flushright{\begin{Arabic}
\quranayah[6][109]
\end{Arabic}}
\flushleft{\begin{hindi}
वे लोग तो अल्लाह की कड़ी-कड़ी क़समें खाते है कि यदि उनके पास कोई निशानी आ जाए, तो उसपर वे अवश्य ईमान लाएँगे। कह दो, "निशानियाँ तो अल्लाह ही के पास है।" और तुम्हें क्या पता कि जब वे आ जाएँगी तो भी वे ईमान नहीं लाएँगे
\end{hindi}}
\flushright{\begin{Arabic}
\quranayah[6][110]
\end{Arabic}}
\flushleft{\begin{hindi}
और हम उनके दिलों और निगाहों को फेर देंगे, जिस प्रकार वे पहली बार ईमान नहीं लाए थे। और हम उन्हें छोड़ देंगे कि वे अपनी सरकशी में भटकते रहें
\end{hindi}}
\flushright{\begin{Arabic}
\quranayah[6][111]
\end{Arabic}}
\flushleft{\begin{hindi}
यदि हम उनकी ओर फ़रिश्ते भी उतार देते और मुर्दें भी उनसे बातें करने लगते और प्रत्येक चीज़ उनके सामने लाकर इकट्ठा कर देते, तो भी वे ईमान न लाते, बल्कि अल्लाह ही का चाहा क्रियान्वित है। परन्तु उनमें से अधिकतर लोग अज्ञानता से काम लेते है
\end{hindi}}
\flushright{\begin{Arabic}
\quranayah[6][112]
\end{Arabic}}
\flushleft{\begin{hindi}
और इसी प्रकार हमने मनुष्यों और जिन्नों में से शैतानों को प्रत्येक नबी का शत्रु बनाया, जो चिकनी-चुपड़ी बात एक-दूसरे के मन में डालकर धोखा देते थे - यदि तुम्हारा रब चाहता तो वे ऐसा न कर सकते। अब छोड़ो उन्हें और उनके मिथ्यारोपण को। -
\end{hindi}}
\flushright{\begin{Arabic}
\quranayah[6][113]
\end{Arabic}}
\flushleft{\begin{hindi}
और ताकि जो लोग परलोक को नहीं मानते, उनके दिल उसकी ओर झुकें और ताकि वे उसे पसन्द कर लें, और ताकि जो कमाई उन्हें करनी है कर लें
\end{hindi}}
\flushright{\begin{Arabic}
\quranayah[6][114]
\end{Arabic}}
\flushleft{\begin{hindi}
अब क्या मैं अल्लाह के सिवा कोई और निर्णायक ढूढूँ? हालाँकि वही है जिसने तुम्हारी ओर किताब अवतरित की है, जिसमें बातें खोल-खोलकर बता दी गई है और जिन लोगों को हमने किताब प्रदान की थी, वे भी जानते है कि यह तुम्हारे रब की ओर से हक़ के साथ अवतरित हुई है, तो तुम कदापि सन्देह में न पड़ना
\end{hindi}}
\flushright{\begin{Arabic}
\quranayah[6][115]
\end{Arabic}}
\flushleft{\begin{hindi}
तुम्हारे रब की बात सच्चाई और इनसाफ़ के साथ पूरी हुई, कोई नहीं जो उसकी बातों को बदल सकें, और वह सुनता, जानता है
\end{hindi}}
\flushright{\begin{Arabic}
\quranayah[6][116]
\end{Arabic}}
\flushleft{\begin{hindi}
और धरती में अधिकतर लोग ऐसे है, यदि तुम उनके कहने पर चले तो वे अल्लाह के मार्ग से तुम्हें भटका देंगे। वे तो केवल अटकल के पीछे चलते है और वे निरे अटकल ही दौड़ाते है
\end{hindi}}
\flushright{\begin{Arabic}
\quranayah[6][117]
\end{Arabic}}
\flushleft{\begin{hindi}
निस्संदेह तुम्हारा रब उसे भली-भाँति जानता है, जो उसके मार्ग से भटकता और वह उन्हें भी जानता है, जो सीधे मार्ग पर है
\end{hindi}}
\flushright{\begin{Arabic}
\quranayah[6][118]
\end{Arabic}}
\flushleft{\begin{hindi}
अतः जिसपर अल्लाह का नाम लिया गया हो, उसे खाओ; यदि तुम उसकी आयतों को मानते हो
\end{hindi}}
\flushright{\begin{Arabic}
\quranayah[6][119]
\end{Arabic}}
\flushleft{\begin{hindi}
और क्या आपत्ति है कि तुम उसे न खाओ, जिसपर अल्लाह का नाम लिया गया हो, बल्कि जो कुछ चीज़े उसने तुम्हारे लिए हराम कर दी है, उनको उसने विस्तारपूर्वक तुम्हे बता दिया है। यह और बात है कि उसके लिए कभी तुम्हें विवश होना पड़े। परन्तु अधिकतर लोग तो ज्ञान के बिना केवल अपनी इच्छाओं (ग़लत विचारों) के द्वारा पथभ्रष्टो करते रहते है। निस्सन्देह तुम्हारा रब मर्यादाहीन लोगों को भली-भाँति जानता है
\end{hindi}}
\flushright{\begin{Arabic}
\quranayah[6][120]
\end{Arabic}}
\flushleft{\begin{hindi}
छोड़ो खुले गुनाह को भी और छिपे को भी। निश्चय ही गुनाह कमानेवालों को उसका बदला दिया जाएगा, जिस कमाई में वे लगे रहे होंगे
\end{hindi}}
\flushright{\begin{Arabic}
\quranayah[6][121]
\end{Arabic}}
\flushleft{\begin{hindi}
और उसे न खाओं जिसपर अल्लाह का नाम न लिया गया हो। निश्चय ही वह तो आज्ञा का उल्लंघन है। शैतान तो अपने मित्रों के दिलों में डालते है कि वे तुमसे झगड़े। यदि तुमने उनकी बात मान ली तो निश्चय ही तुम बहुदेववादी होगे
\end{hindi}}
\flushright{\begin{Arabic}
\quranayah[6][122]
\end{Arabic}}
\flushleft{\begin{hindi}
क्या वह व्यक्ति जो पहले मुर्दा था, फिर उसे हमने जीवित किया और उसके लिए एक प्रकाश उपलब्ध किया जिसको लिए हुए वह लोगों के बीच चलता-फिरता है, उस व्यक्ति को तरह हो सकता है जो अँधेरों में पड़ा हुआ हो, उससे कदापि निकलनेवाला न हो? ऐसे ही इनकार करनेवालों के कर्म उनके लिए सुहाबने बनाए गए है
\end{hindi}}
\flushright{\begin{Arabic}
\quranayah[6][123]
\end{Arabic}}
\flushleft{\begin{hindi}
और इसी प्रकार हमने प्रत्येक बस्ती में उसके बड़े-बड़े अपराधियों को लगा दिया है कि ले वहाँ चालें चले। वे अपने ही विरुद्ध चालें चलते है, किन्तु उन्हें इसका एहसास नहीं
\end{hindi}}
\flushright{\begin{Arabic}
\quranayah[6][124]
\end{Arabic}}
\flushleft{\begin{hindi}
और जब उनके पास कोई आयत (निशानी) आता है, तो वे कहते है, "हम कदापि नहीं मानेंगे, जब तक कि वैसी ही चीज़ हमें न दी जाए जो अल्लाह के रसूलों को दी गई हैं।" अल्लाह भली-भाँति उस (के औचित्य) को जानता है, जिसमें वह अपनी पैग़म्बरी रखता है। अपराधियों को शीघ्र ही अल्लाह के यहाँ बड़े अपमान और कठोर यातना का सामना करना पड़ेगा, उस चाल के कारण जो वे चलते रहे है
\end{hindi}}
\flushright{\begin{Arabic}
\quranayah[6][125]
\end{Arabic}}
\flushleft{\begin{hindi}
अतः (वास्तविकता यह है कि) जिसे अल्लाह सीधे मार्ग पर लाना चाहता है, उसका सीना इस्लाम के लिए खोल देता है। और जिसे गुमराही में पड़ा रहने देता चाहता है, उसके सीने को तंग और भिंचा हुआ कर देता है; मानो वह आकाश में चढ़ रहा है। इस तरह अल्लाह उन लोगों पर गन्दगी डाल देता है, जो ईमान नहीं लाते
\end{hindi}}
\flushright{\begin{Arabic}
\quranayah[6][126]
\end{Arabic}}
\flushleft{\begin{hindi}
और यह तुम्हारे रब का रास्ता है, बिल्कुल सीधा। हमने निशानियाँ, ध्यान देनेवालों के लिए खोल-खोलकर बयान कर दी है
\end{hindi}}
\flushright{\begin{Arabic}
\quranayah[6][127]
\end{Arabic}}
\flushleft{\begin{hindi}
उनके लिए उनके रब के यहाँ सलामती का घर है और वह उनका संरक्षक मित्र है, उन कामों के कारण जो वे करते रहे है
\end{hindi}}
\flushright{\begin{Arabic}
\quranayah[6][128]
\end{Arabic}}
\flushleft{\begin{hindi}
और उस दिन को याद करो, जब वह उन सबको घेरकर इकट्ठा करेगा, (कहेगा), "ऐ जिन्नों के गिरोह! तुमने तो मनुष्यों पर ख़ूब हाथ साफ किया।" और मनुष्यों में से जो उनके साथी रहे होंगे, कहेंग, "ऐ हमारे रब! हमने आपस में एक-दूसरे से लाभ उठाया और अपने उस नियत समय को पहुँच गए, जो तूने हमारे लिए ठहराया था।" वह कहेगा, "आग (नरक) तुम्हारा ठिकाना है, उसमें तुम्हें सदैव रहना है।" अल्लाह का चाहा ही क्रियान्वित है। निश्चय ही तुम्हारा रब तत्वदर्शी, सर्वज्ञ है
\end{hindi}}
\flushright{\begin{Arabic}
\quranayah[6][129]
\end{Arabic}}
\flushleft{\begin{hindi}
इसी प्रकार हम अत्याचारियों को एक-दूसरे के लिए (नरक का) साथी बना देंगे, उस कमाई के कारण जो वे करते रहे थे
\end{hindi}}
\flushright{\begin{Arabic}
\quranayah[6][130]
\end{Arabic}}
\flushleft{\begin{hindi}
"ऐ जिन्नों और मनुष्यों के गिरोह! क्या तुम्हारे पास तुम्हीं में से रसूल नहीं आए थे, जो तुम्हें मेरी आयतें सुनाते और इस दिन के पेश आने से तुम्हें डराते थे?" वे कहेंगे, "क्यों नहीं! (रसूल तो आए थे) हम स्वयं अपने विरुद्ध गवाह है।" उन्हें तो सांसारिक जीवन ने धोखे में रखा। मगर अब वे स्वयं अपने विरुद्ध गवाही देने लगे कि वे इनकार करनेवाले थे
\end{hindi}}
\flushright{\begin{Arabic}
\quranayah[6][131]
\end{Arabic}}
\flushleft{\begin{hindi}
यह जान लो कि तुम्हारा रब ज़ुल्म करके बस्तियों को विनष्ट करनेवाला न था, जबकि उनके निवासी बेसुध रहे हों
\end{hindi}}
\flushright{\begin{Arabic}
\quranayah[6][132]
\end{Arabic}}
\flushleft{\begin{hindi}
सभी के दर्जें उनके कर्मों के अनुसार है। और जो कुछ वे करते है, उससे तुम्हारा रब अनभिज्ञ नहीं है
\end{hindi}}
\flushright{\begin{Arabic}
\quranayah[6][133]
\end{Arabic}}
\flushleft{\begin{hindi}
तुम्हारा रब निस्पृह, दयावान है। यदि वह चाहे तो तुम्हें (दुनिया से) ले जाए और तुम्हारे स्थान पर जिसको चाहे तुम्हारे बाद ले आए, जिस प्रकार उसने तुम्हें कुछ और लोगों की सन्तति से उठाया है
\end{hindi}}
\flushright{\begin{Arabic}
\quranayah[6][134]
\end{Arabic}}
\flushleft{\begin{hindi}
जिस चीज़ का तुमसे वादा किया जाता है, उसे अवश्य आना है और तुममें उसे मात करने की सामर्थ्य नहीं
\end{hindi}}
\flushright{\begin{Arabic}
\quranayah[6][135]
\end{Arabic}}
\flushleft{\begin{hindi}
कह दो, "ऐ मेरी क़ौम के लोगो! तुम अपनी जगह कर्म करते रहो, मैं भी अपनी जगह कर्मशील हूँ। शीघ्र ही तुम्हें मालूम हो जाएगा कि घर (लोक-परलोक) का परिणाम किसके हित में होता है। निश्चय ही अत्याचारी सफल नहीं होते।"
\end{hindi}}
\flushright{\begin{Arabic}
\quranayah[6][136]
\end{Arabic}}
\flushleft{\begin{hindi}
उन्होंने अल्लाह के लिए स्वयं उसी की पैदा की हुई खेती और चौपायों में से एक भाग निश्चित किया है और अपने ख़याल से कहते है, "यह किस्सा अल्लाह का है और यह हमारे ठहराए हुए साझीदारों का है।" फिर जो उनके साझीदारों का (हिस्सा) है, वह अल्लाह को नहीं पहुँचता, परन्तु जो अल्लाह का है, वह उनके साझीदारों को पहुँच जाता है। कितना बुरा है, जो फ़ैसला वे करते है!
\end{hindi}}
\flushright{\begin{Arabic}
\quranayah[6][137]
\end{Arabic}}
\flushleft{\begin{hindi}
इसी प्रकार बहुत-से बहुदेववादियों के लिए उनके लिए साझीदारों ने उनकी अपनी सन्तान की हत्या को सुहाना बना दिया है, ताकि उन्हें विनष्ट कर दें और उनके लिए उनके धर्म को संदिग्ध बना दें। यदि अल्लाह चाहता तो वे ऐसा न करते; तो छोड़ दो उन्हें और उनके झूठ घड़ने को
\end{hindi}}
\flushright{\begin{Arabic}
\quranayah[6][138]
\end{Arabic}}
\flushleft{\begin{hindi}
और वे कहते है, "ये जानवर और खेती वर्जित और सुरक्षित है। इन्हें तो केवल वही खा सकता है, जिसे हम चाहें।" - ऐसा वे स्वयं अपने ख़याल से कहते है - और कुछ चौपाए ऐसे है, जिनकी पीठों को (सवारी के लिए) हराम ठहरा लिया है और कुछ जानवर ऐसे है जिनपर अल्लाह का नाम नहीं लेते। यह यह उन्होंने अल्लाह पर झूठ घड़ा है, और वह शीघ्र ही उन्हें उनके झूठ घड़ने का बदला देगा
\end{hindi}}
\flushright{\begin{Arabic}
\quranayah[6][139]
\end{Arabic}}
\flushleft{\begin{hindi}
और वे कहते है, "जो कुछ इन जानवरों के पेट में है वह बिल्कुल हमारे पुरुषों ही के लिए है और वह हमारी पत्नियों के लिए वर्जित है। परन्तु यदि वह मुर्दा हो, तो वे सब उसमें शरीक है।" शीघ्र ही वह उन्हें उनके ऐसा कहने का बदला देगा। निस्संदेह वह तत्वदर्शी, सर्वज्ञ है
\end{hindi}}
\flushright{\begin{Arabic}
\quranayah[6][140]
\end{Arabic}}
\flushleft{\begin{hindi}
वे लोग कुछ जाने-बूझे बिना घाटे में रहे, जिन्होंने मूर्खता के कारण अपनी सन्तान की हत्या की और जो कुछ अल्लाह ने उन्हें प्रदान किया था, उसे अल्लाह पर झूठ घड़कर हराम ठहरा दिया। वास्तव में वे भटक गए और वे सीधा मार्ग पानेवाले न हुए
\end{hindi}}
\flushright{\begin{Arabic}
\quranayah[6][141]
\end{Arabic}}
\flushleft{\begin{hindi}
और वही है जिसने बाग़ पैदा किए; कुछ जालियों पर चढ़ाए जाते है और कुछ नहीं चढ़ाए जाते और खजूर और खेती भी जिनकी पैदावार विभिन्न प्रकार की होती है, और ज़ैतून और अनार जो एक-दूसरे से मिलते-जुलते भी है और नहीं भी मिलते है। जब वह फल दे, तो उसका फल खाओ और उसका हक़ अदा करो जो उस (फ़सल) की कटाई के दिन वाजिब होता है। और हद से आगे न बढ़ो, क्योंकि वह हद से आगे बढ़नेवालों को पसन्द नहीं करता
\end{hindi}}
\flushright{\begin{Arabic}
\quranayah[6][142]
\end{Arabic}}
\flushleft{\begin{hindi}
और चौपायों में से कुछ बोझ उठानेवाले बड़े और कुछ छोटे जानवर पैदा किए। अल्लाह ने जो कुछ तुम्हें दिया है, उसमें से खाओ और शैतान के क़दमों पर न चलो। निश्चय ही वह तुम्हारा खुला हुआ शत्रु है
\end{hindi}}
\flushright{\begin{Arabic}
\quranayah[6][143]
\end{Arabic}}
\flushleft{\begin{hindi}
आठ नर-मादा पैदा किए - दो भेड़ की जाति से और दो बकरी की जाति से - कहो, "क्या उसने दोनों नर हराम किए है या दोनों मादा को? या उसको जो इन दोनों मादा के पेट में हो? किसी ज्ञान के आधार पर मुझे बताओ, यदि तुम सच्चे हो।"
\end{hindi}}
\flushright{\begin{Arabic}
\quranayah[6][144]
\end{Arabic}}
\flushleft{\begin{hindi}
और दो ऊँट की जाति से और दो गाय की जाति से, कहो, "क्या उसने दोनों नर हराम किए है या दोनों मादा को? या उसको जो इन दोनों मादा के पेट में हो? या, तुम उपस्थित थे, जब अल्लाह ने तुम्हें इसका आदेश दिया था? फिर उस व्यक्ति से बढ़कर अत्याचारी कौन होगा जो लोगों को पथभ्रष्ट करने के लिए अज्ञानता-पूर्वक अल्लाह पर झूठ घड़े? निश्चय ही, अल्लाह अत्याचारी लोगों को मार्ग नहीं दिखाता।"
\end{hindi}}
\flushright{\begin{Arabic}
\quranayah[6][145]
\end{Arabic}}
\flushleft{\begin{hindi}
कह दो, "जो कुछ मेरी ओर प्रकाशना की गई है, उसमें तो मैं नहीं पाता कि किसी खानेवाले पर उसका कोई खाना हराम किया गया हो, सिवाय इसके लिए वह मुरदार हो, यह बहता हुआ रक्त हो या ,सुअर का मांस हो - कि वह निश्चय ही नापाक है - या वह चीज़ जो मर्यादा से हटी हुई हो, जिसपर अल्लाह के अतिरिक्त किसी और का नाम लिया गया हो। इसपर भी जो बहुत विवश और लाचार हो जाए; परन्तु वह अवज्ञाकारी न हो और न हद से आगे बढ़नेवाला हो, तो निश्चय ही तुम्हारा रब अत्यन्त क्षमाशील, दयाबान है।"
\end{hindi}}
\flushright{\begin{Arabic}
\quranayah[6][146]
\end{Arabic}}
\flushleft{\begin{hindi}
और उन लोगों के लिए जो यहूदी हुए हमने नाख़ूनवाला जानवर हराम किया और गाय और बकरी में से इन दोनों की चरबियाँ उनके लिए हराम कर दी थीं, सिवाय उस (चर्बी) के जो उन दोनों की पीठों या आँखों से लगी हुई या हड़्डी से मिली हुई हो। यह बात ध्यान में रखो। हमने उन्हें उनकी सरकशी का बदला दिया था और निश्चय ही हम सच्चे है
\end{hindi}}
\flushright{\begin{Arabic}
\quranayah[6][147]
\end{Arabic}}
\flushleft{\begin{hindi}
फिर यदि वे तुम्हें झुठलाएँ तो कह दो, "तुम्हारा रब व्यापक दयालुतावाला है और अपराधियों से उसकी यातना नहीं फिरती।"
\end{hindi}}
\flushright{\begin{Arabic}
\quranayah[6][148]
\end{Arabic}}
\flushleft{\begin{hindi}
बहुदेववादी कहेंगे, "यदि अल्लाह चाहता तो न हम साझीदार ठहराते और न हमारे पूर्वज ही; और न हम किसी चीज़ को (बिना अल्लाह के आदेश के) हराम ठहराते।" ऐसे ही उनसे पहले के लोगों ने भी झुठलाया था, यहाँ तक की उन्हें हमारी यातना का मज़ा चखना पड़ा। कहो, "क्या तुम्हारे पास कोई ज्ञान है कि उसे हमारे पास पेश करो? तुम लोग केवल गुमान पर चलते हो और निरे अटकल से काम लेते हो।"
\end{hindi}}
\flushright{\begin{Arabic}
\quranayah[6][149]
\end{Arabic}}
\flushleft{\begin{hindi}
कह दो, "पूर्ण तर्क तो अल्लाह ही का है। अतः यदि वह चाहता तो तुम सबको सीधा मार्ग दिखा देता।"
\end{hindi}}
\flushright{\begin{Arabic}
\quranayah[6][150]
\end{Arabic}}
\flushleft{\begin{hindi}
कह दो, "अपने उन गवाहों को लाओ, जो इसकी गवाही दें कि अल्लाह ने इसे हराम किया है।" फिर यदि वे गवाही दें तो तुम उनके साथ गवाही न देना, औऱ उन लोगों की इच्छाओं का अनुसरण न करना जिन्होंने हमारी आयतों को झुठलाया और जो आख़िरत को नहीं मानते और (जिनका) हाल यह है कि वे दूसरो को अपने रब के समकक्ष ठहराते है
\end{hindi}}
\flushright{\begin{Arabic}
\quranayah[6][151]
\end{Arabic}}
\flushleft{\begin{hindi}
कह दो, "आओ, मैं तुम्हें सुनाऊँ कि तुम्हारे रब ने तुम्हारे ऊपर क्या पाबन्दियाँ लगाई है: यह कि किसी चीज़ को उसका साझीदार न ठहराओ और माँ-बाप के साथ सद्व्य वहार करो और निर्धनता के कारण अपनी सन्तान की हत्या न करो; हम तुम्हें भी रोज़ी देते है और उन्हें भी। और अश्लील बातों के निकट न जाओ, चाहे वे खुली हुई हों या छिपी हुई हो। और किसी जीव की, जिसे अल्लाह ने आदरणीय ठहराया है, हत्या न करो। यह और बात है कि हक़ के लिए ऐसा करना पड़े। ये बाते है, जिनकी ताकीद उसने तुम्हें की है, शायद कि तुम बुद्धि से काम लो।
\end{hindi}}
\flushright{\begin{Arabic}
\quranayah[6][152]
\end{Arabic}}
\flushleft{\begin{hindi}
"और अनाथ के धन को हाथ न लगाओ, किन्तु ऐसे तरीक़े से जो उत्तम हो, यहाँ तक कि वह अपनी युवावस्था को पहुँच जाए। और इनसाफ़ के साथ पूरा-पूरा नापो और तौलो। हम किसी व्यक्ति पर उसी काम की ज़िम्मेदारी का बोझ डालते हैं जो उसकी सामर्थ्य में हो। और जब बात कहो, तो न्याय की कहो, चाहे मामला अपने नातेदार ही का क्यों न हो, और अल्लाह की प्रतिज्ञा को पूरा करो। ये बातें हैं, जिनकी उसने तुम्हें ताकीद की है। आशा है तुम ध्यान रखोगे
\end{hindi}}
\flushright{\begin{Arabic}
\quranayah[6][153]
\end{Arabic}}
\flushleft{\begin{hindi}
और यह कि यही मेरा सीधा मार्ग है, तो तुम इसी पर चलो और दूसरे मार्गों पर न चलो कि वे तुम्हें उसके मार्ग से हटाकर इधर-उधर कर देंगे। यह वह बात है जिसकी उसने तुम्हें ताकीद की है, ताकि तुम (पथभ्रष्ट ता से) बचो
\end{hindi}}
\flushright{\begin{Arabic}
\quranayah[6][154]
\end{Arabic}}
\flushleft{\begin{hindi}
फिर (देखो) हमने मूसा को किताब दी थी, (धर्म को) पूर्णता प्रदान करने के लिए, जिसे उसने उत्तम रीति से ग्रहण किया था; और हर चीज़ को स्पष्ट‍ रूप से बयान करने, मार्गदर्शन देने और दया करने के लिए, ताकि वे लोग अपने रब से मिलने पर ईमान लाएँ
\end{hindi}}
\flushright{\begin{Arabic}
\quranayah[6][155]
\end{Arabic}}
\flushleft{\begin{hindi}
और यह किताब भी हमने उतारी है, जो बरकतवाली है; तो तुम इसका अनुसरण करो और डर रखो, ताकि तुमपर दया की जाए,
\end{hindi}}
\flushright{\begin{Arabic}
\quranayah[6][156]
\end{Arabic}}
\flushleft{\begin{hindi}
कि कहीं ऐसा न हो कि तुम कहने लगो, "किताब तो केवल हमसे पहले के दो गिरोहों पर उतारी गई थी और हमें तो उनके पढ़ने-पढ़ाने की ख़बर तक न थी।"
\end{hindi}}
\flushright{\begin{Arabic}
\quranayah[6][157]
\end{Arabic}}
\flushleft{\begin{hindi}
या यह कहने लगो, "यदि हमपर किताब उतारी गई होती तो हम उनसे बढकर सीधे मार्ग पर होते।" तो अब तुम्हारे पास रब की ओर से एक स्पष्ट प्रमाण, मार्गदर्शन और दयालुता आ चुकी है। अब उससे बढ़कर अत्याचारी कौन होगा जो अल्लाह की आयतों को झुठलाए और दूसरों को उनसे फेरे? जो लोग हमारी आयतों से रोकते हैं, उन्हें हम इस रोकने के कारण जल्द बुरी यातना देंगे
\end{hindi}}
\flushright{\begin{Arabic}
\quranayah[6][158]
\end{Arabic}}
\flushleft{\begin{hindi}
क्या ये लोग केवल इसी की प्रतीक्षा कर रहे है कि उनके पास फ़रिश्ते आ जाएँ या स्वयं तुम्हारा रब की कोई निशानी आ जाएगी, फिर किसी ऐसे व्यक्ति को उसका ईमान कुछ लाभ न पहुँचाएगा जो पहले ईमान न लाया हो या जिसने अपने ईमान में कोई भलाई न कमाई हो। कह दो, ?"तुम भी प्रतीक्षा करो, हम भी प्रतीक्षा करते है।"
\end{hindi}}
\flushright{\begin{Arabic}
\quranayah[6][159]
\end{Arabic}}
\flushleft{\begin{hindi}
जिन लोगों ने अपने धर्म के टुकड़े-टुकड़े कर दिए और स्वयं गिरोहों में बँट गए, तुम्हारा उनसे कोई सम्बन्ध नहीं। उनका मामला तो बस अल्लाह के हवाले है। फिर वह उन्हें बता देगा जो कुछ वे किया करते थे
\end{hindi}}
\flushright{\begin{Arabic}
\quranayah[6][160]
\end{Arabic}}
\flushleft{\begin{hindi}
जो कोई अच्छा चरित्र लेकर आएगा उसे उसका दस गुना बदला मिलेगा और जो व्यक्ति बुरा चरित्र लेकर आएगा, उसे उसका बस उतना ही बदला मिलेगा, उनके साथ कोई अन्याय न होगा
\end{hindi}}
\flushright{\begin{Arabic}
\quranayah[6][161]
\end{Arabic}}
\flushleft{\begin{hindi}
कहो, "मेरे रब ने मुझे सीधा मार्ग दिखा दिया है, बिल्कुल ठीक धर्म, इबराहीम के पंथ की ओर जो सबसे कटकर एक (अल्लाह) का हो गया था और वह बहुदेववादियों में से न था।"
\end{hindi}}
\flushright{\begin{Arabic}
\quranayah[6][162]
\end{Arabic}}
\flushleft{\begin{hindi}
कहो, "मेरी नमाज़ और मेरी क़ुरबानी और मेरा जीना और मेरा मरना सब अल्लाह के लिए है, जो सारे संसार का रब है
\end{hindi}}
\flushright{\begin{Arabic}
\quranayah[6][163]
\end{Arabic}}
\flushleft{\begin{hindi}
"उसका कोई साझी नहीं है। मुझे तो इसी का आदेश मिला है और सबसे पहला मुस्लिम (आज्ञाकारी) मैं हूँ।"
\end{hindi}}
\flushright{\begin{Arabic}
\quranayah[6][164]
\end{Arabic}}
\flushleft{\begin{hindi}
कहो, "क्या मैं अल्लाह से भिन्न कोई और रब ढूढूँ, जबकि हर चीज़ का रब वही है!" और यह कि प्रत्येक व्यक्ति जो कुछ कमाता है, उसका फल वही भोगेगा; कोई बोझ उठानेवाला किसी दूसरे का बोझ नहीं उठाएगा। फिर तुम्हें अपने रब की ओर लौटकर जाना है। उस समय वह तुम्हें बता देगा, जिसमें परस्पर तुम्हारा मतभेद और झगड़ा था
\end{hindi}}
\flushright{\begin{Arabic}
\quranayah[6][165]
\end{Arabic}}
\flushleft{\begin{hindi}
वही है जिसने तुम्हें धरती में धरती में ख़लीफ़ा (अधिकारी, उत्ताराधिकारी) बनाया और तुममें से कुछ लोगों के दर्जे कुछ लोगों की अपेक्षा ऊँचे रखे, ताकि जो कुछ उसने तुमको दिया है उसमें वह तम्हारी ले। निस्संदेह तुम्हारा रब जल्द सज़ा देनेवाला है। और निश्चय ही वही बड़ा क्षमाशील, दयावान है
\end{hindi}}
\chapter{Al-A'raf (The Elevated Places)}
\begin{Arabic}
\Huge{\centerline{\basmalah}}\end{Arabic}
\flushright{\begin{Arabic}
\quranayah[7][1]
\end{Arabic}}
\flushleft{\begin{hindi}
अलिफ़॰ लाम॰ मीम॰ साद॰
\end{hindi}}
\flushright{\begin{Arabic}
\quranayah[7][2]
\end{Arabic}}
\flushleft{\begin{hindi}
यह एक किताब है, जो तुम्हारी ओर उतारी गई है - अतः इससे तुम्हारे सीने में कोई तंगी न हो - ताकि तुम इसके द्वारा सचेत करो और यह ईमानवालों के लिए एक प्रबोधन है;
\end{hindi}}
\flushright{\begin{Arabic}
\quranayah[7][3]
\end{Arabic}}
\flushleft{\begin{hindi}
जो कुछ तुम्हारे रब की ओर से तुम्हारी ओर अवतरित हुआ है, उस पर चलो और उसे छोड़कर दूसरे संरक्षक मित्रों का अनुसरण न करो। तुम लोग नसीहत थोड़े ही मानते हो
\end{hindi}}
\flushright{\begin{Arabic}
\quranayah[7][4]
\end{Arabic}}
\flushleft{\begin{hindi}
कितनी ही बस्तियाँ थीं, जिन्हें हमने विनष्टम कर दिया। उनपर हमारी यातना रात को सोते समय आ पहुँची या (दिन-दहाड़) आई, जबकि वे दोपहर में विश्राम कर रहे थे
\end{hindi}}
\flushright{\begin{Arabic}
\quranayah[7][5]
\end{Arabic}}
\flushleft{\begin{hindi}
जब उनपर यातना आ गई तो इसके सिवा उनके मुँह से कुछ न निकला कि वे पुकार उठे, "वास्तव में हम अत्याचारी थे।"
\end{hindi}}
\flushright{\begin{Arabic}
\quranayah[7][6]
\end{Arabic}}
\flushleft{\begin{hindi}
अतः हम उन लोगों से अवश्य पूछेंगे, जिनके पास रसूल भेजे गए थे, और हम रसूलों से भी अवश्य पूछेंगे
\end{hindi}}
\flushright{\begin{Arabic}
\quranayah[7][7]
\end{Arabic}}
\flushleft{\begin{hindi}
फिर हम पूरे ज्ञान के साथ उनके सामने सब बयान कर देंगे। हम कही ग़ायब नहीं थे
\end{hindi}}
\flushright{\begin{Arabic}
\quranayah[7][8]
\end{Arabic}}
\flushleft{\begin{hindi}
और बिल्कुल पक्का-सच्चा वज़न उसी दिन होगा। अतः जिनके कर्म वज़न में भारी होंगे, वही सफलता प्राप्त करेंगे
\end{hindi}}
\flushright{\begin{Arabic}
\quranayah[7][9]
\end{Arabic}}
\flushleft{\begin{hindi}
और वे लोग जिनके कर्म वज़न में हलके होंगे, तो वही वे लोग हैं, जिन्होंने अपने आपको घाटे में डाला, क्योंकि वे हमारी आयतों का इनकार औऱ अपने ऊपर अत्याचार करते रहे
\end{hindi}}
\flushright{\begin{Arabic}
\quranayah[7][10]
\end{Arabic}}
\flushleft{\begin{hindi}
और हमने धरती में तुम्हें अधिकार दिया और उसमें तुम्हारे लिए जीवन-सामग्री रखी। तुम कृतज्ञता थोड़े ही दिखाते हो
\end{hindi}}
\flushright{\begin{Arabic}
\quranayah[7][11]
\end{Arabic}}
\flushleft{\begin{hindi}
हमने तुम्हें पैदा करने का निश्चय किया; फिर तुम्हारा रूप बनाया; फिर हमने फ़रिश्तों से कहो, "आदम को सजदा करो।" तो उन्होंने सजदा किया, सिवाय इबलीस के। वह (इबलीस) सदजा करनेवालों में से न हुआ
\end{hindi}}
\flushright{\begin{Arabic}
\quranayah[7][12]
\end{Arabic}}
\flushleft{\begin{hindi}
कहा, "तुझे किसने सजका करने से रोका, जबकि मैंने तुझे आदेश दिया था?" बोला, "मैं उससे अच्छा हूँ। तूने मुझे अग्नि से बनाया और उसे मिट्टी से बनाया।"
\end{hindi}}
\flushright{\begin{Arabic}
\quranayah[7][13]
\end{Arabic}}
\flushleft{\begin{hindi}
कहा, "उतर जा यहाँ से! तुझे कोई हक़ नहीं है कि यहाँ घमंड करे, तो अब निकल जा; निश्चय ही तू अपमानित है।"
\end{hindi}}
\flushright{\begin{Arabic}
\quranayah[7][14]
\end{Arabic}}
\flushleft{\begin{hindi}
बोला, "मुझे एक दिन तक मुहल्लत दे, जबकि लोग उठाए जाएँगे।"
\end{hindi}}
\flushright{\begin{Arabic}
\quranayah[7][15]
\end{Arabic}}
\flushleft{\begin{hindi}
कहा, "निस्संदेह तुझे मुहल्लत है।"
\end{hindi}}
\flushright{\begin{Arabic}
\quranayah[7][16]
\end{Arabic}}
\flushleft{\begin{hindi}
बोला, "अच्छा, इस कारण कि तूने मुझे गुमराही में डाला है, मैं भी तेरे सीधे मार्ग पर उनके लिए घात में अवश्य बैठूँगा
\end{hindi}}
\flushright{\begin{Arabic}
\quranayah[7][17]
\end{Arabic}}
\flushleft{\begin{hindi}
"फिर उनके आगे और उनके पीछे और उनके दाएँ और उनके बाएँ से उनके पास आऊँगा। और तू उनमें अधिकतर को कृतज्ञ न पाएगा।"
\end{hindi}}
\flushright{\begin{Arabic}
\quranayah[7][18]
\end{Arabic}}
\flushleft{\begin{hindi}
कहा, "निकल जा यहाँ से! निन्दित ठुकराया हुआ। उनमें से जिस किसी ने भी तेरा अनुसरण किया, मैं अवश्य तुम सबसे जहन्नम को भर दूँगा।"
\end{hindi}}
\flushright{\begin{Arabic}
\quranayah[7][19]
\end{Arabic}}
\flushleft{\begin{hindi}
और "ऐ आदम! तुम और तुम्हारी पत्नी दोनों जन्नत में रहो-बसो, फिर जहाँ से चाहो खाओ, लेकिन इस वृक्ष के निकट न जाना, अन्यथा अत्याचारियों में से हो जाओगे।"
\end{hindi}}
\flushright{\begin{Arabic}
\quranayah[7][20]
\end{Arabic}}
\flushleft{\begin{hindi}
फिर शैतान ने दोनों को बहकाया, ताकि उनकी शर्मगाहों को, जो उन दोनों से छिपी थीं, उन दोनों के सामने खोल दे। और उसने (इबलीस ने) कहा, "तुम्हारे रब ने तुम दोनों को जो इस वृक्ष से रोका है, तो केवल इसलिए कि ऐसा न हो कि तुम कहीं फ़रिश्ते हो जाओ या कही ऐसा न हो कि तुम्हें अमरता प्राप्त हो जाए।"
\end{hindi}}
\flushright{\begin{Arabic}
\quranayah[7][21]
\end{Arabic}}
\flushleft{\begin{hindi}
और उसने उन दोनों के आगे क़समें खाई कि "निश्चय ही मैं तुम दोनों का हितैषी हूँ।"
\end{hindi}}
\flushright{\begin{Arabic}
\quranayah[7][22]
\end{Arabic}}
\flushleft{\begin{hindi}
इस प्रकार धोखा देकर उसने उन दोनों को झुका लिया। अन्ततः जब उन्होंने उस वृक्ष का स्वाद लिया, तो उनकी शर्मगाहे एक-दूसरे के सामने खुल गए और वे अपने ऊपर बाग़ के पत्ते जोड़-जोड़कर रखने लगे। तब उनके रब ने उन्हें पुकारा, "क्या मैंने तुम दोनों को इस वृक्ष से रोका नहीं था और तुमसे कहा नहीं था कि शैतान तुम्हारा खुला शत्रु है?"
\end{hindi}}
\flushright{\begin{Arabic}
\quranayah[7][23]
\end{Arabic}}
\flushleft{\begin{hindi}
दोनों बोले, "हमारे रब! हमने अपने आप पर अत्याचार किया। अब यदि तूने हमें क्षमा न किया और हम पर दया न दर्शाई, फिर तो हम घाटा उठानेवालों में से होंगे।"
\end{hindi}}
\flushright{\begin{Arabic}
\quranayah[7][24]
\end{Arabic}}
\flushleft{\begin{hindi}
कहा, "उतर जाओ! तुम परस्पर एक-दूसरे के शत्रु हो और एक अवधि कर तुम्हारे लिए धरती में ठिकाना और जीवन-सामग्री है।"
\end{hindi}}
\flushright{\begin{Arabic}
\quranayah[7][25]
\end{Arabic}}
\flushleft{\begin{hindi}
कहा, "वहीं तुम्हें जीना और वहीं तुम्हें मरना है और उसी में से तुमको निकाला जाएगा।"
\end{hindi}}
\flushright{\begin{Arabic}
\quranayah[7][26]
\end{Arabic}}
\flushleft{\begin{hindi}
ऐ आदम की सन्तान! हमने तुम्हारे लिए वस्त्र उतारा है कि तुम्हारी शर्मगाहों को छुपाए और रक्षा और शोभा का साधन हो। और धर्मपरायणता का वस्त्र - वह तो सबसे उत्तम है, यह अल्लाह की निशानियों में से है, ताकि वे ध्यान दें
\end{hindi}}
\flushright{\begin{Arabic}
\quranayah[7][27]
\end{Arabic}}
\flushleft{\begin{hindi}
ऐ आदम की सन्तान! कहीं शैतान तुम्हें बहकावे में न डाल दे, जिस प्रकार उसने तुम्हारे माँ-बाप को जन्नत से निकलवा दिया था; उनके वस्त्र उनपर से उतरवा दिए थे, ताकि उनकी शर्मगाहें एक-दूसरे के सामने खोल दे। निस्सदेह वह और उसका गिरोह उस स्थान से तुम्हें देखता है, जहाँ से तुम उन्हें नहीं देखते। हमने तो शैतानों को उन लोगों का मित्र बना दिया है, जो ईमान नहीं रखते
\end{hindi}}
\flushright{\begin{Arabic}
\quranayah[7][28]
\end{Arabic}}
\flushleft{\begin{hindi}
और उनका हाल यह है कि जब वे लोग कोई अश्लील कर्म करते है तो कहते है कि "हमने अपने बाप-दादा को इसी तरीक़े पर पाया है और अल्लाह ही ने हमें इसका आदेश दिया है।" कह दो, "अल्लाह कभी अश्लील बातों का आदेश नहीं दिया करता। क्या अल्लाह पर थोपकर ऐसी बात कहते हो, जिसका तुम्हें ज्ञान नहीं?"
\end{hindi}}
\flushright{\begin{Arabic}
\quranayah[7][29]
\end{Arabic}}
\flushleft{\begin{hindi}
कह दो, "मेरे रब ने तो न्याय का आदेश दिया है और यह कि इबादत के प्रत्येक अवसर पर अपना रुख़ ठीक रखो और निरे उसी के भक्त एवं आज्ञाकारी बनकर उसे पुकारो। जैसे उसने तुम्हें पहली बार पैदा किया, वैसे ही तुम फिर पैदा होगे।"
\end{hindi}}
\flushright{\begin{Arabic}
\quranayah[7][30]
\end{Arabic}}
\flushleft{\begin{hindi}
एक गिरोह को उसने मार्ग दिखाया। परन्तु दूसरा गिरोह ऐसा है, जिसके लोगों पर गुमराही चिपककर रह गई। निश्चय ही उन्होंने अल्लाह को छोड़कर शैतानों को अपने मित्र बनाए और समझते यह है कि वे सीधे मार्ग पर हैं
\end{hindi}}
\flushright{\begin{Arabic}
\quranayah[7][31]
\end{Arabic}}
\flushleft{\begin{hindi}
ऐ आदम की सन्तान! इबादत के प्रत्येक अवसर पर शोभा धारण करो; खाओ और पियो, परन्तु हद से आगे न बढ़ो। निश्चय ही, वह हद से आगे बदनेवालों को पसन्द नहीं करता
\end{hindi}}
\flushright{\begin{Arabic}
\quranayah[7][32]
\end{Arabic}}
\flushleft{\begin{hindi}
कहो, "अल्लाह की उस शोभा को जिसे उसने अपने बन्दों के लिए उत्पन्न किया है औऱ आजीविका की पवित्र, अच्छी चीज़ो को किसने हराम कर दिया?" कह दो, "यह सांसारिक जीवन में भी ईमानवालों के लिए हैं; क़ियामत के दिन तो ये केवल उन्हीं के लिए होंगी। इसी प्रकार हम आयतों को उन लोगों के लिए सविस्तार बयान करते है, जो जानना चाहे।"
\end{hindi}}
\flushright{\begin{Arabic}
\quranayah[7][33]
\end{Arabic}}
\flushleft{\begin{hindi}
कह दो, "मेरे रब ने केवल अश्लील कर्मों को हराम किया है - जो उनमें से प्रकट हो उन्हें भी और जो छिपे हो उन्हें भी - और हक़ मारना, नाहक़ ज़्यादती और इस बात को कि तुम अल्लाह का साझीदार ठहराओ, जिसके लिए उसने कोई प्रमाण नहीं उतारा और इस बात को भी कि तुम अल्लाह पर थोपकर ऐसी बात कहो जिसका तुम्हें ज्ञान न हो।"
\end{hindi}}
\flushright{\begin{Arabic}
\quranayah[7][34]
\end{Arabic}}
\flushleft{\begin{hindi}
प्रत्येक समुदाय के लिए एक नियत अवधि है। फिर जब उसका नियत समय आ जाता है, तो एक घड़ी भर न पीछे हट सकते है और न आगे बढ़ सकते है
\end{hindi}}
\flushright{\begin{Arabic}
\quranayah[7][35]
\end{Arabic}}
\flushleft{\begin{hindi}
ऐ आदम की सन्तान! यदि तुम्हारे पास तुम्हीं में से कोई रसूल आएँ; तुम्हें मेरी आयतें सुनाएँ, तो जिसने डर रखा और सुधार कर लिया तो ऐसे लोगों के लिए न कोई भय होगा और न वे शोकाकुल होंगे
\end{hindi}}
\flushright{\begin{Arabic}
\quranayah[7][36]
\end{Arabic}}
\flushleft{\begin{hindi}
रहे वे लोग जिन्होंने हमारी आयतों को झुठलाया और उनके मुक़ाबले में अकड़ दिखाई; वही आगवाले हैं, जिसमें वे सदैव रहेंगे
\end{hindi}}
\flushright{\begin{Arabic}
\quranayah[7][37]
\end{Arabic}}
\flushleft{\begin{hindi}
अब उससे बढ़कर अत्याचारी कौन है, जिसने अल्लाह पर मिथ्यारोपण किया या उसकी आयतों को झुठलाया? ऐसे लोगों को उनके लिए लिखा हुआ हिस्सा पहुँचता रहेगा, यहाँ तक कि जब हमारे भेजे हुए (फ़रिश्ते) उनके प्राण ग्रस्त करने के लिए उनके पास आएँगे तो कहेंगे, "कहाँ हैं, वे जिन्हें तुम अल्लाह को छोड़कर पुकारते थे?" कहेंगे, "वे तो हमसे गुम हो गए।" और वे स्वयं अपने विरुद्ध गवाही देंगे कि वास्तव में वे इनकार करनेवाले थे
\end{hindi}}
\flushright{\begin{Arabic}
\quranayah[7][38]
\end{Arabic}}
\flushleft{\begin{hindi}
वह कहेगा, "जिन्न और इनसान के जो गिरोह तुमसे पहले गुज़रे हैं, उन्हीं के साथ सम्मिलित होकर तुम भी आग में प्रवेश करो।" जब भी कोई जमाअत प्रवेश करेगी, तो वह अपनी बहन पर लानत करेगी, यहाँ तक कि जब सब उसमें रल-मिल जाएँगे तो उनमें से बाद में आनेवाले अपने से पहलेवाले के विषय में कहेंगे, "हमारे रब! हमें इन्हीं लोगों ने गुमराह किया था; तो तू इन्हें आग की दोहरी यातना दे।" वह कहेगा, "हरेक के लिए दोहरी ही है। किन्तु तुम नहीं जानते।"
\end{hindi}}
\flushright{\begin{Arabic}
\quranayah[7][39]
\end{Arabic}}
\flushleft{\begin{hindi}
और उनमें से पहले आनेवाले अपने से बाद में आनेवालों से कहेंगे, "फिर हमारे मुक़ावाले में तुम्हें कोई श्रेष्ठता प्राप्त नहीं, तो जैसी कुछ कमाई तुम करते रहे हो, उसके बदले में तुम यातना का मज़ा चखो!"
\end{hindi}}
\flushright{\begin{Arabic}
\quranayah[7][40]
\end{Arabic}}
\flushleft{\begin{hindi}
जिन लोगों ने हमारी आयतों को झुठलाया और उनके मुक़ाबले में अकड़ दिखाई, उनके लिए आकाश के द्वार नहीं खोले जाएँगे और न वे जन्नत में प्रवेश करेंग जब तक कि ऊँट सुई के नाके में से न गुज़र जाए। हम अपराधियों को ऐसा ही बदला देते है
\end{hindi}}
\flushright{\begin{Arabic}
\quranayah[7][41]
\end{Arabic}}
\flushleft{\begin{hindi}
उनके लिए बिछौना जहन्नम का होगा और ओढ़ना भी उसी का। अत्याचारियों को हम ऐसा ही बदला देते है
\end{hindi}}
\flushright{\begin{Arabic}
\quranayah[7][42]
\end{Arabic}}
\flushleft{\begin{hindi}
इसके विपरित जो लोग ईमान लाए और उन्होंने अच्छे कर्म किए - हम किसी पर उसकी सामर्थ्य से बढ़कर बोझ नहीं डालते - वही लोग जन्नतवाले है। वे उसमें सदैव रहेंगे
\end{hindi}}
\flushright{\begin{Arabic}
\quranayah[7][43]
\end{Arabic}}
\flushleft{\begin{hindi}
उनके सीनों में एक-दूसरे के प्रति जो रंजिश होगी, उसे हम दूर कर देंगे; उनके नीचें नहरें बह रही होंगी और वे कहेंगे, "प्रशंसा अल्लाह के लिए है, जिसने इसकी ओर हमारा मार्गदर्शन किया। और यदि अल्लाह हमारा मार्गदर्शन न करतो तो हम कदापि मार्ग नहीं पा सकते थे। हमारे रब के रसूल निस्संदेह सत्य लेकर आए थे।" और उन्हें आवाज़ दी जाएगी, "यह जन्नत है, जिसके तुम वारिस बनाए गए। उन कर्मों के बदले में जो तुम करते रहे थे।"
\end{hindi}}
\flushright{\begin{Arabic}
\quranayah[7][44]
\end{Arabic}}
\flushleft{\begin{hindi}
जन्नतवाले आगवालों को पुकारेंगे, "हमसे हमारे रब ने जो वादा किया था, उसे हमने सच पाया। तो क्या तुमसे तुम्हारे रब ने जो वादा कर रखा था, तुमने भी उसे सच पाया?" वे कहेंगे, "हाँ।" इतने में एक पुकारनेवाला उनके बीच पुकारेगा, "अल्लाह की फिटकार है अत्याचारियों पर।"
\end{hindi}}
\flushright{\begin{Arabic}
\quranayah[7][45]
\end{Arabic}}
\flushleft{\begin{hindi}
जो अल्लाह के मार्ग से रोकते और उसे टेढ़ा करना चाहते है और जो आख़िरत का इनकार करते है,
\end{hindi}}
\flushright{\begin{Arabic}
\quranayah[7][46]
\end{Arabic}}
\flushleft{\begin{hindi}
और इन दोनों के मध्य एक ओट होगी। और ऊँचाइयों पर कुछ लोग होंगे जो प्रत्येक को उसके लक्षणों से पहचानते होंगे, और जन्नतवालों से पुकारकर कहेंगे, "तुम पर सलाम है।" वे अभी जन्नत में प्रविष्ट तो नहीं हुए होंगे, यद्यपि वे आस लगाए होंगे
\end{hindi}}
\flushright{\begin{Arabic}
\quranayah[7][47]
\end{Arabic}}
\flushleft{\begin{hindi}
और जब उनकी निगाहें आगवालों की ओर फिरेंगी, तो कहेंगे, "हमारे रब, हमें अत्याचारी लोगों में न सम्मिलित न करना।"
\end{hindi}}
\flushright{\begin{Arabic}
\quranayah[7][48]
\end{Arabic}}
\flushleft{\begin{hindi}
और ये ऊँचाइयोंवाले कुछ ऐसे लोगों से, जिन्हें ये उनके लक्षणों से पहचानते हैं, कहेंगे, "तुम्हारे जत्थे तो तुम्हारे कुछ काम न आए और न तुम्हारा अकड़ते रहना ही।
\end{hindi}}
\flushright{\begin{Arabic}
\quranayah[7][49]
\end{Arabic}}
\flushleft{\begin{hindi}
"क्या ये वही हैं ना, जिनके विषय में तुम क़समें खाते थे कि अल्लाह उनपर अपनी दया-दृष्टि न करेगा।" "जन्नत में प्रवेश करो, तुम्हारे लिए न कोई भय है और न तुम्हें कोई शोक होगा।"
\end{hindi}}
\flushright{\begin{Arabic}
\quranayah[7][50]
\end{Arabic}}
\flushleft{\begin{hindi}
आगवाले जन्नतवालों को पुकारेंगे कि ,"थोड़ा पानी हमपर बहा दो, या उन चीज़ों में से कुछ दे दो जो अल्लाह ने तुम्हें दी हैं।" वे कहेंगे, "अल्लाह ने तो ये दोनों चीज़ें इनकार करनेवालों के लिए वर्जित कर दी है।"
\end{hindi}}
\flushright{\begin{Arabic}
\quranayah[7][51]
\end{Arabic}}
\flushleft{\begin{hindi}
उनके लिए जिन्होंने अपना धर्म खेल-तमाशा ठहराया और जिन्हें सांसारिक जीवन ने धोखे में डाल दिया, तो आज हम भी उन्हें भुला देंगे, जिस प्रकार वे अपने इस दिन की मुलाक़ात को भूले रहे और हमारी आयतों का इनकार करते रहे
\end{hindi}}
\flushright{\begin{Arabic}
\quranayah[7][52]
\end{Arabic}}
\flushleft{\begin{hindi}
और निश्चय ही हम उनके पास एक ऐसी किताब ले आए है, जिसे हमने ज्ञान के आधार पर विस्तृत किया है, जो ईमान लानेवालों के लिए मार्गदर्शन और दयालुता है
\end{hindi}}
\flushright{\begin{Arabic}
\quranayah[7][53]
\end{Arabic}}
\flushleft{\begin{hindi}
क्या वे लोग केवल इसी की प्रतीक्षा में है कि उसकी वास्तविकता और परिणाम प्रकट हो जाए? जिस दिन उसकी वास्तविकता सामने आ जाएगी, तो वे लोग इससे पहले उसे भूले हुए थे, बोल उठेंगे, "वास्तव में, हमारे रब के रसूल सत्य लेकर आए थे। तो क्या हमारे कुछ सिफ़ारिशी है, जो हमारी सिफ़ारिश कर दें या हमें वापस भेज दिया जाए कि जो कुछ हम करते थे उससे भिन्न कर्म करें?" उन्होंने अपने आपको घाटे में डाल दिया और जो कुछ वे झूठ घढ़ते थे, वे सब उनसे गुम होकर रह गए
\end{hindi}}
\flushright{\begin{Arabic}
\quranayah[7][54]
\end{Arabic}}
\flushleft{\begin{hindi}
निस्संदेह तुम्हारा रब वही अल्लाह है, जिसने आकाशों और धरती को छह दिनों में पैदा किया - फिर राजसिंहासन पर विराजमान हुआ। वह रात को दिन पर ढाँकता है जो तेज़ी से उसका पीछा करने में सक्रिय है। और सूर्य, चन्द्रमा और तारे भी बनाए, इस प्रकार कि वे उसके आदेश से काम में लगे हुए है। सावधान रहो, उसी की सृष्टि है और उसी का आदेश है। अल्लाह सारे संसार का रब, बड़ी बरकतवाला है
\end{hindi}}
\flushright{\begin{Arabic}
\quranayah[7][55]
\end{Arabic}}
\flushleft{\begin{hindi}
अपने रब को गिड़गिड़ाकर और चुपके-चुपके पुकारो। निश्चय ही वह हद से आगे बढ़नेवालों को पसन्द नहीं करता
\end{hindi}}
\flushright{\begin{Arabic}
\quranayah[7][56]
\end{Arabic}}
\flushleft{\begin{hindi}
और धरती में उसके सुधार के पश्चात बिगाड़ न पैदा करो। भय और आशा के साथ उसे पुकारो। निश्चय ही, अल्लाह की दयालुता सत्कर्मी लोगों के निकट है
\end{hindi}}
\flushright{\begin{Arabic}
\quranayah[7][57]
\end{Arabic}}
\flushleft{\begin{hindi}
और वही है जो अपनी दयालुता से पहले शुभ सूचना देने को हवाएँ भेजता है, यहाँ तक कि जब वे बोझल बादल को उठा लेती है तो हम उसे किसी निर्जीव भूमि की ओर चला देते है, फिर उससे पानी बरसाते है, फिर उससे हर तरह के फल निकालते है। इसी प्रकार हम मुर्दों को मृत अवस्था से निकालेगे - ताकि तुम्हें ध्यान हो
\end{hindi}}
\flushright{\begin{Arabic}
\quranayah[7][58]
\end{Arabic}}
\flushleft{\begin{hindi}
और अच्छी भूमि के पेड़-पौधे उसके रब के आदेश से निकलते है और जो भूमि ख़राब हो गई है तो उससे निकम्मी पैदावार के अतिरिक्त कुछ नहीं निकलता। इसी प्रकार हम निशानियों को उन लोगों के लिए तरह-तरह से बयान करते है, जो कृतज्ञता दिखानेवाले है
\end{hindi}}
\flushright{\begin{Arabic}
\quranayah[7][59]
\end{Arabic}}
\flushleft{\begin{hindi}
हमने नूह को उसकी क़ौम के लोगों की ओर भेजा, तो उसने कहा, "ऐ मेरी क़ौम के लोगो! अल्लाह की बन्दगी करो। उसके अतिरिक्त तुम्हारा कोई पूज्य नहीं। मैं तुम्हारे लिए एक बड़े दिन का यातना से डरता हूँ।"
\end{hindi}}
\flushright{\begin{Arabic}
\quranayah[7][60]
\end{Arabic}}
\flushleft{\begin{hindi}
उसकी क़ौम के सरदारों ने कहा, "हम तो तुम्हें खुली गुमराही में पड़ा देख रहे है।"
\end{hindi}}
\flushright{\begin{Arabic}
\quranayah[7][61]
\end{Arabic}}
\flushleft{\begin{hindi}
उसने कहा, "ऐ मेरी क़ौम के लोगों! किसी गुमराही का मुझसे सम्बन्ध नहीं, बल्कि मैं सारे संसार के रब का एक रसूल हूँ। -
\end{hindi}}
\flushright{\begin{Arabic}
\quranayah[7][62]
\end{Arabic}}
\flushleft{\begin{hindi}
"अपने रब के सन्देश पहुँचता हूँ और तुम्हारा हित चाहता हूँ, और मैं अल्लाह की ओर से वह कुछ जानता हूँ, जो तुम नहीं जानते।"
\end{hindi}}
\flushright{\begin{Arabic}
\quranayah[7][63]
\end{Arabic}}
\flushleft{\begin{hindi}
क्या (तुमने मुझे झूठा समझा) और तुम्हें इस पार आश्चर्य हुआ कि तुम्हारे पास तुम्हीं में से एक आदमी के द्वारा तुम्हारे रब की नसीहत आई? ताकि वह तुम्हें सचेत कर दे और ताकि तुम डर रखने लगो और शायद कि तुमपर दया की जाए
\end{hindi}}
\flushright{\begin{Arabic}
\quranayah[7][64]
\end{Arabic}}
\flushleft{\begin{hindi}
किन्तु उन्होंने झुठला दिया। अन्ततः हमने उसे और उन लोगों को जो उसके साथ एक नौका में थे, बचा लिया और जिन लोगों ने हमारी आयतों को ग़लत समझा, उन्हें हमने डूबो दिया। निश्चय ही वे अन्धे लोग थे
\end{hindi}}
\flushright{\begin{Arabic}
\quranayah[7][65]
\end{Arabic}}
\flushleft{\begin{hindi}
और आद की ओर उनके भाई हूद को भेजा। उसने कहा, "ऐ मेरी क़ौम के लोगो! अल्लाह की बन्दगी करो, उसके अतिरिक्त तुम्हारा कोई पूज्य नहीं। तो क्या (इसे सोचकर) तुम डरते नहीं?"
\end{hindi}}
\flushright{\begin{Arabic}
\quranayah[7][66]
\end{Arabic}}
\flushleft{\begin{hindi}
उसकी क़ौम के इनकार करनेवाले सरदारों ने कहा, "वास्तव में, हम तो देखते है कि तुम बुद्धिहीनता में ग्रस्त हो और हम तो तुम्हें झूठा समझते है।"
\end{hindi}}
\flushright{\begin{Arabic}
\quranayah[7][67]
\end{Arabic}}
\flushleft{\begin{hindi}
उसने कहा, "ऐ मेरी क़ौम के लोगो! मैं बुद्धिहीनता में कदापि ग्रस्त नहीं हूँ। परन्तु मैं सारे संसार के रब का रसूल हूँ।-
\end{hindi}}
\flushright{\begin{Arabic}
\quranayah[7][68]
\end{Arabic}}
\flushleft{\begin{hindi}
"तुम्हें अपने रब के संदेश पहुँचता हूँ और मैं तुम्हारा विश्वस्त हितैषी हूँ
\end{hindi}}
\flushright{\begin{Arabic}
\quranayah[7][69]
\end{Arabic}}
\flushleft{\begin{hindi}
"क्या (तुमने मुझे झूठा समझा) और तुम्हें इसपर आश्चर्य हुआ कि तुम्हारे पास तुम्हीं में से एक आदमी के द्वारा तुम्हारे रब की नसीहत आई, ताकि वह तुम्हें सचेत करे? और याद करो, जब उसने नूह की क़ौम के पश्चात तुम्हें उसका उत्तराधिकारी बनाया और शारीरिक दृष्टि से भी तुम्हें अधिक विशालता प्रदान की। अतः अल्लाह की सामर्थ्य के चमत्कारो को याद करो, ताकि तुम्हें सफलता प्राप्त हो।"
\end{hindi}}
\flushright{\begin{Arabic}
\quranayah[7][70]
\end{Arabic}}
\flushleft{\begin{hindi}
वे बोले, "क्या तुम हमारे पास इसलिए आए हो कि अकेले अल्लाह की हम बन्दगी करें और जिनको हमारे बाप-दादा पूजते रहे है, उन्हें छोड़ दें? अच्छा, तो जिसकी तुम हमें धमकी देते हो, उसे हमपर ले आओ, यदि तुम सच्चे हो।"
\end{hindi}}
\flushright{\begin{Arabic}
\quranayah[7][71]
\end{Arabic}}
\flushleft{\begin{hindi}
उसने कहा, "तुम पर तो तुम्हारे रब की ओर से नापाकी थोप दी गई है और प्रकोप टूट पड़ा है। क्या तुम मुझसे उन नामों के लिए झगड़ते हो जो तुमने और तुम्हारे बाप-दादा ने रख छोड़े है, जिनके लिए अल्लाह ने कोई प्रमाण नहीं उतारा? अच्छा, तो तुम भी प्रतीक्षा करो, मैं भी तुम्हारे साथ प्रतीक्षा करता हूँ।"
\end{hindi}}
\flushright{\begin{Arabic}
\quranayah[7][72]
\end{Arabic}}
\flushleft{\begin{hindi}
फिर हमने अपनी दयालुता से उसको और जो लोग उसके साथ थे उन्हें बचा लिया और उन लोगों की जड़ काट दी, जिन्होंने हमारी आयतों को झुठलाया था और ईमानवाले न थे
\end{hindi}}
\flushright{\begin{Arabic}
\quranayah[7][73]
\end{Arabic}}
\flushleft{\begin{hindi}
और समूद की ओर उनके भाई सालेह को भेजा। उसने कहा, "ऐ मेरी क़ौम के लोगो! अल्लाह की बन्दगी करो। उसके अतिरिक्त तुम्हारा कोई पूज्य नहीं। तुम्हारे पास तुम्हारे रब की ओर से एक स्पष्ट प्रमाण आ चुका है। यह अल्लाह की ऊँटनी तुम्हारे लिए एक निशानी है। अतः इसे छोड़ दो कि अल्लाह की धरती में खाए। और तकलीफ़ पहुँचाने के लिए इसे हाथ न लगाना, अन्यथा तुम्हें एक दुखद यातना आ लेगी।-
\end{hindi}}
\flushright{\begin{Arabic}
\quranayah[7][74]
\end{Arabic}}
\flushleft{\begin{hindi}
और याद करो जब अल्लाह ने आद के पश्चात तुम्हें उसका उत्तराधिकारी बनाया और धरती में तुम्हें ठिकाना प्रदान किया। तुम उसके समतल मैदानों में महल बनाते हो और पहाड़ो को काट-छाँट कर भवनों का रूप देते हो। अतः अल्लाह की सामर्थ्य के चमत्कारों को याद करो और धरती में बिगाड़ पैदा करते न फिरो।"
\end{hindi}}
\flushright{\begin{Arabic}
\quranayah[7][75]
\end{Arabic}}
\flushleft{\begin{hindi}
उसकी क़ौम के सरदार, जो बड़े बने हुए थे, उन कमज़ोर लोगों से, जो उनमें ईमान लाए थे, कहने लगे, "क्या तुम जानते हो कि सालेह अपने रब का भेजा हुआ (पैग़म्बर) है?" उन्होंने कहा, "निस्संदेह जिस चीज़ के साथ वह भेजा गया है, हम उसपर ईमान रखते है।"
\end{hindi}}
\flushright{\begin{Arabic}
\quranayah[7][76]
\end{Arabic}}
\flushleft{\begin{hindi}
उन घमंड करनेवालों ने कहा, "जिस चीज़ पर तुम ईमान लाए हो, हम तो उसको नहीं मानते।"
\end{hindi}}
\flushright{\begin{Arabic}
\quranayah[7][77]
\end{Arabic}}
\flushleft{\begin{hindi}
फिर उन्होंने उस ऊँटनी की कूचें काट दीं और अपने रब के आदेश की अवहेलना की और बोले, "ऐ सालेह! हमें तू जिस चीज़ की धमकी देता है, उसे हमपर ले आ, यदि तू वास्तव में रसूलों में से है।"
\end{hindi}}
\flushright{\begin{Arabic}
\quranayah[7][78]
\end{Arabic}}
\flushleft{\begin{hindi}
अन्ततः एक हिला मारनेवाली आपदा ने उन्हें आ लिया और वे अपने घरों में आँधे पड़े रह गए
\end{hindi}}
\flushright{\begin{Arabic}
\quranayah[7][79]
\end{Arabic}}
\flushleft{\begin{hindi}
फिर वह यह कहता हुआ उनके यहाँ से फिरा, "ऐ मेरी क़ौम के लोगों! मैं तो तुम्हें अपने रब का संदेश पहुँचा चुका और मैंने तुम्हारा हित चाहा। परन्तु तुम्हें अपने हितैषी पसन्द ही नहीं आते।"
\end{hindi}}
\flushright{\begin{Arabic}
\quranayah[7][80]
\end{Arabic}}
\flushleft{\begin{hindi}
और हमने लूत को भेजा। जब उसने अपनी क़ौम से कहा, "क्या तुम वह प्रत्यक्ष अश्लील कर्म करते हो, जिसे दुनिया में तुमसे पहले किसी ने नहीं किया?"
\end{hindi}}
\flushright{\begin{Arabic}
\quranayah[7][81]
\end{Arabic}}
\flushleft{\begin{hindi}
तुम स्त्रियों को छोड़कर मर्दों से कामेच्छा पूरी करते हो, बल्कि तुम नितान्त मर्यादाहीन लोग हो
\end{hindi}}
\flushright{\begin{Arabic}
\quranayah[7][82]
\end{Arabic}}
\flushleft{\begin{hindi}
उसकी क़ौम के लोगों का उत्तर इसके अतिरिक्त और कुछ न था कि वे बोले, "निकालो, उन लोगों को अपनी बस्ती से। ये ऐसे लोग है जो बड़े पाक-साफ़ है!"
\end{hindi}}
\flushright{\begin{Arabic}
\quranayah[7][83]
\end{Arabic}}
\flushleft{\begin{hindi}
फिर हमने उसे और उसके लोगों को छुटकारा दिया, सिवाय उसकी स्त्री के कि वह पीछे रह जानेवालों में से थी
\end{hindi}}
\flushright{\begin{Arabic}
\quranayah[7][84]
\end{Arabic}}
\flushleft{\begin{hindi}
और हमने उनपर एक बरसात बरसाई, तो देखो अपराधियों का कैसा परिणाम हुआ
\end{hindi}}
\flushright{\begin{Arabic}
\quranayah[7][85]
\end{Arabic}}
\flushleft{\begin{hindi}
और मदयनवालों की ओर हमने उनके भाई शुऐब को भेजा। उसने कहा, "ऐ मेरी क़ौम के लोगों! अल्लाह की बन्दगी करो। उसके अतिरिक्त तुम्हारा कोई पूज्य नहीं। तुम्हारे पास तुम्हारे रब की ओर से एक स्पष्ट प्रमाण आ चुका है। तो तुम नाप और तौल पूरी-पूरी करो, और लोगों को उनकी चीज़ों में घाटा न दो, और धरती में उसकी सुधार के पश्चात बिगाड़ पैदा न करो। यही तुम्हारे लिए अच्छा है, यदि तुम ईमानवाले हो
\end{hindi}}
\flushright{\begin{Arabic}
\quranayah[7][86]
\end{Arabic}}
\flushleft{\begin{hindi}
"और प्रत्येक मार्ग पर इसलिए न बैठो कि धमकियाँ दो और उस व्यक्ति को अल्लाह के मार्ग से रोकने लगो जो उसपर ईमान रखता हो और न उस मार्ग को टेढ़ा करने में लग जाओ। याद करो, वह समय जब तुम थोड़े थे, फिर उसने तुम्हें अधिक कर दिया। और देखो, बिगाड़ पैदा करनेवालो का कैसा परिणाम हुआ
\end{hindi}}
\flushright{\begin{Arabic}
\quranayah[7][87]
\end{Arabic}}
\flushleft{\begin{hindi}
"और यदि तुममें एक गिरोह ऐसा है, जो उसपर ईमान लाया है, जिसके साथ मैं भेजा गया हूँ और एक गिरोह ऐसा है, जो उसपर ईमान लाया है, जिसके साथ मैं भेजा गया हूँ और एक गिरोह ईमान नहीं लाया, तो धैर्य से काम लो, यहाँ तक कि अल्लाह हमारे बीच फ़ैसला कर दे। और वह सबसे अच्छा फ़ैसला करनेवाला है।"
\end{hindi}}
\flushright{\begin{Arabic}
\quranayah[7][88]
\end{Arabic}}
\flushleft{\begin{hindi}
उनकी क़ौम के सरदारों ने, जो घमंड में पड़े थे, कहा, "ऐ शुऐब! हम तुझे और तेरे साथ उन लोगों को, जो ईमान लाए है, अपनी बस्ती से निकालकर रहेंगे। या फिर तुम हमारे पन्थ में लौट आओ।" उसने कहा, "क्या (तुम यही चाहोगे) यद्यपि यह हमें अप्रिय हो जब भी?
\end{hindi}}
\flushright{\begin{Arabic}
\quranayah[7][89]
\end{Arabic}}
\flushleft{\begin{hindi}
"हम अल्लाह पर झूठ घड़नेवाले ठहरेंगे, यदि तुम्हारे पन्थ में लौट आएँ, इसके बाद कि अल्लाह ने हमें उससे छुटकारा दे दिया है। यह हमसे तो होने का नहीं कि हम उसमें पलट कर जाएँ, बल्कि हमारे रब अल्लाह की इच्छा ही क्रियान्वित है। ज्ञान की स्पष्ट से हमारा रब हर चीज़ को अपने घेरे में लिए हुए है। हमने अल्लाह ही पर भरोसा किया है। हमारे रब, हमारे और हमारी क़ौम के बीच निश्चित अटल फ़ैसला कर दे। और तू सबसे अच्छा फ़ैसला करनेवाला है।"
\end{hindi}}
\flushright{\begin{Arabic}
\quranayah[7][90]
\end{Arabic}}
\flushleft{\begin{hindi}
और क़ौम के सरदार, जिन्होंने इनकार किया था, बोले, "यदि तुम शुऐब के अनुयायी बने तो तुम घाटे में पड़ जाओगे।"
\end{hindi}}
\flushright{\begin{Arabic}
\quranayah[7][91]
\end{Arabic}}
\flushleft{\begin{hindi}
अन्ततः एक दहला देनेवाली आपदा ने उन्हें आ लिया। फिर वे अपने घर में औंधे पड़े रह गए,
\end{hindi}}
\flushright{\begin{Arabic}
\quranayah[7][92]
\end{Arabic}}
\flushleft{\begin{hindi}
शुऐब को झुठलानेवाले, मानो कभी वहाँ बसे ही न थे। शुऐब को झुठलानेवाले ही घाटे में रहे
\end{hindi}}
\flushright{\begin{Arabic}
\quranayah[7][93]
\end{Arabic}}
\flushleft{\begin{hindi}
तब वह उनके यहाँ से यह कहता हुआ फिरा, "ऐ मेरी क़ौम के लोगो! मैंने अपने रब के सन्देश तुम्हें पहुँचा दिए और मैंने तुम्हारा हित चाहा। अब मैं इनकार करनेवाले लोगो पर कैसे अफ़सोस करूँ!"
\end{hindi}}
\flushright{\begin{Arabic}
\quranayah[7][94]
\end{Arabic}}
\flushleft{\begin{hindi}
हमने जिस बस्ती में भी कभी कोई नबी भेजा, तो वहाँ के लोगों को तंगी और मुसीबत में डाला, ताकि वे (हमारे सामने) गिड़गि़ड़ाए
\end{hindi}}
\flushright{\begin{Arabic}
\quranayah[7][95]
\end{Arabic}}
\flushleft{\begin{hindi}
फिर हमने बदहाली को ख़ुशहाली में बदल दिया, यहाँ तक कि वे ख़ूब फले-फूले और कहने लगे, "ये दुख और सुख तो हमारे बाप-दादा को भी पहुँचे हैं।" अनततः जब वे बेखबर थे, हमने अचानक उन्हें पकड़ लिया
\end{hindi}}
\flushright{\begin{Arabic}
\quranayah[7][96]
\end{Arabic}}
\flushleft{\begin{hindi}
यदि बस्तियों के लोग ईमान लाते और डर रखते तो अवश्य ही हम उनपर आकाश और धरती की बरकतें खोल देते, परन्तु उन्होंने तो झुठलाया। तो जो कुछ कमाई वे करते थे, उसके बदले में हमने उन्हें पकड़ लिया
\end{hindi}}
\flushright{\begin{Arabic}
\quranayah[7][97]
\end{Arabic}}
\flushleft{\begin{hindi}
फिर क्या बस्तियों के लोगों को इस और से निश्चिन्त रहने का अवसर मिल सका कि रात में उनपर हमारी यातना आ जाए, जबकि वे सोए हुए हो?
\end{hindi}}
\flushright{\begin{Arabic}
\quranayah[7][98]
\end{Arabic}}
\flushleft{\begin{hindi}
और क्या बस्तियों के लोगो को इस ओर से निश्चिन्त रहने का अवसर मिल सका कि दिन चढ़े उनपर हमारी यातना आ जाए, जबकि वे खेल रहे हों?
\end{hindi}}
\flushright{\begin{Arabic}
\quranayah[7][99]
\end{Arabic}}
\flushleft{\begin{hindi}
आख़िर क्या वे अल्लाह की चाल से निश्चिन्त हो गए थे? तो (समझ लो उन्हें टोटे में पड़ना ही था, क्योंकि) अल्लाह की चाल से तो वही लोग निश्चित होते है, जो टोटे में पड़नेवाले होते है
\end{hindi}}
\flushright{\begin{Arabic}
\quranayah[7][100]
\end{Arabic}}
\flushleft{\begin{hindi}
क्या जो धरती के, उसके पूर्ववासियों के पश्चात उत्तराधिकारी हुए है, उनपर यह तथ्य प्रकट न हुआ कि यदि हम चाहें तो उनके गुनाहों पर उन्हें आ पकड़े? हम तो उनके दिलों पर मुहर लगा रहे हैं, क्योंकि वे कुछ भी नहीं सुनते
\end{hindi}}
\flushright{\begin{Arabic}
\quranayah[7][101]
\end{Arabic}}
\flushleft{\begin{hindi}
ये है वे बस्तियाँ जिनके कुछ वृत्तान्त हम तुमको सुना रहे है। उनके पास उनके रसूल खुली-खुली निशानियाँ लेकर आए परन्तु वे ऐसे न हुए कि ईमान लाते। इसका कारण यह था कि वे पहले से झुठलाते रहे थे। इसी प्रकार अल्लाह इनकार करनेवालों के दिलों पर मुहर लगा देता है
\end{hindi}}
\flushright{\begin{Arabic}
\quranayah[7][102]
\end{Arabic}}
\flushleft{\begin{hindi}
हमने उनके अधिकतर लोगो में प्रतिज्ञा का निर्वाह न पाया, बल्कि उनके बहुतों को हमने उल्लंघनकारी ही पाया
\end{hindi}}
\flushright{\begin{Arabic}
\quranayah[7][103]
\end{Arabic}}
\flushleft{\begin{hindi}
फिर उनके पश्चात हमने मूसा को अपनी निशानियों के साथ फ़िरऔन और उसके सरदारों के पास भेजा, परन्तु उन्होंने इनकार और स्वयं पर अत्याचार किया। तो देखो, इन बिगाड़ पैदा करनेवालों का कैसा परिणाम हुआ!
\end{hindi}}
\flushright{\begin{Arabic}
\quranayah[7][104]
\end{Arabic}}
\flushleft{\begin{hindi}
मूसा ने कहा, "ऐ फ़िरऔन! मैं सारे संसार के रब का रसूल हूँ
\end{hindi}}
\flushright{\begin{Arabic}
\quranayah[7][105]
\end{Arabic}}
\flushleft{\begin{hindi}
"मैं इसका अधिकारी हूँ कि अल्लाह से सम्बद्ध करके सत्य के अतिरिक्त कोई बात न कहूँ। मैं तुम्हारे पास तुम्हारे रब की ओर से स्पष्ट प्रमाण लेकर आ गया हूँ। अतः तुम इसराईल की सन्तान को मेरे साथ जाने दो।"
\end{hindi}}
\flushright{\begin{Arabic}
\quranayah[7][106]
\end{Arabic}}
\flushleft{\begin{hindi}
बोला, "यदि तुम कोई निशानी लेकर आए हो तो उसे पेश करो, यदि तुम सच्चे हो।"
\end{hindi}}
\flushright{\begin{Arabic}
\quranayah[7][107]
\end{Arabic}}
\flushleft{\begin{hindi}
तब उसने अपनी लाठी डाल दी। क्या देखते है कि वह प्रत्यक्ष अजगर है
\end{hindi}}
\flushright{\begin{Arabic}
\quranayah[7][108]
\end{Arabic}}
\flushleft{\begin{hindi}
और उसने अपना हाथ निकाला, तो क्या देखते है कि वह सब देखनेवालों के सामने चमक रहा है
\end{hindi}}
\flushright{\begin{Arabic}
\quranayah[7][109]
\end{Arabic}}
\flushleft{\begin{hindi}
फ़िरऔन की क़ौम के सरदार कहने लगे, "अरे, यह तो बडा कुशल जादूगर है!
\end{hindi}}
\flushright{\begin{Arabic}
\quranayah[7][110]
\end{Arabic}}
\flushleft{\begin{hindi}
"तुम्हें तुम्हारी धरती से निकाल देना चाहता है। तो अब क्या कहते हो?"
\end{hindi}}
\flushright{\begin{Arabic}
\quranayah[7][111]
\end{Arabic}}
\flushleft{\begin{hindi}
उन्होंने कहा, "इसे और इसके भाई को प्रतीक्षा में रखो और नगरों में हरकारे भेज दो,
\end{hindi}}
\flushright{\begin{Arabic}
\quranayah[7][112]
\end{Arabic}}
\flushleft{\begin{hindi}
"कि वे हर कुशल जादूगर को तुम्हारे पास ले आएँ।"
\end{hindi}}
\flushright{\begin{Arabic}
\quranayah[7][113]
\end{Arabic}}
\flushleft{\begin{hindi}
अतएव जादूगर फ़िरऔन के पास आ गए। कहने लगे, "यदि हम विजयी हुए तो अवश्य ही हमें बड़ा बदला मिलेगा?"
\end{hindi}}
\flushright{\begin{Arabic}
\quranayah[7][114]
\end{Arabic}}
\flushleft{\begin{hindi}
उसने कहा, "हाँ, और बेशक तुम (मेरे) क़रीबियों में से हो जाओगे।"
\end{hindi}}
\flushright{\begin{Arabic}
\quranayah[7][115]
\end{Arabic}}
\flushleft{\begin{hindi}
उन्होंने कहा, "ऐ मूसा! या तुम डालो या फिर हम डालते हैं?"
\end{hindi}}
\flushright{\begin{Arabic}
\quranayah[7][116]
\end{Arabic}}
\flushleft{\begin{hindi}
उसने कहा, "तुम ही डालो।" फिर उन्होंने डाला तो लोगो की आँखों पर जादू कर दिया और उन्हें भयभीत कर दिया। उन्होंने एक बहुत बड़े जादू का प्रदर्शन किया
\end{hindi}}
\flushright{\begin{Arabic}
\quranayah[7][117]
\end{Arabic}}
\flushleft{\begin{hindi}
हमने मूसा की ओर प्रकाशना कि कि "अपनी लाठी डाल दे।" फिर क्या देखते है कि वह उनके रचें हुए स्वांग को निगलती जा रही है
\end{hindi}}
\flushright{\begin{Arabic}
\quranayah[7][118]
\end{Arabic}}
\flushleft{\begin{hindi}
इस प्रकार सत्य प्रकट हो गया और जो कुछ वे कर रहे थे, मिथ्या होकर रहा
\end{hindi}}
\flushright{\begin{Arabic}
\quranayah[7][119]
\end{Arabic}}
\flushleft{\begin{hindi}
अतः वे पराभूत हो गए और अपमानित होकर रहे
\end{hindi}}
\flushright{\begin{Arabic}
\quranayah[7][120]
\end{Arabic}}
\flushleft{\begin{hindi}
और जादूगर सहसा सजदे में गिर पड़े
\end{hindi}}
\flushright{\begin{Arabic}
\quranayah[7][121]
\end{Arabic}}
\flushleft{\begin{hindi}
बोले, "हम सारे संसार के रब पर ईमान ले आए;
\end{hindi}}
\flushright{\begin{Arabic}
\quranayah[7][122]
\end{Arabic}}
\flushleft{\begin{hindi}
"मूसा और हारून के रब पर।"
\end{hindi}}
\flushright{\begin{Arabic}
\quranayah[7][123]
\end{Arabic}}
\flushleft{\begin{hindi}
फ़िरऔन बोला, "इससे पहले कि मैं तुम्हें अनुमति दूँ, तं उसपर ईमान ले आए! यह तो एक चाल है, जो तुम लोग नगर में चले हो, ताकि उसके निवासियों को उससे निकाल दो। अच्छा, तो अब तुम्हें जल्द की मालूम हुआ जाता है!
\end{hindi}}
\flushright{\begin{Arabic}
\quranayah[7][124]
\end{Arabic}}
\flushleft{\begin{hindi}
"मैं तुम्हारे हाथ और तुम्हारे पाँव विपरीत दिशाओं से काट दूँगा; फिर तुम सबको सूली पर चढ़ाकर रहूँगा।"
\end{hindi}}
\flushright{\begin{Arabic}
\quranayah[7][125]
\end{Arabic}}
\flushleft{\begin{hindi}
उन्होंने कहा, "हम तो अपने रब ही की और लौटेंगे
\end{hindi}}
\flushright{\begin{Arabic}
\quranayah[7][126]
\end{Arabic}}
\flushleft{\begin{hindi}
"और तू केबल इस क्रोध से हमें कष्ट पहुँचाने के लिए पीछे पड़ गया है कि हम अपने रब की निशानियों पर ईमान ले आए। हमारे रब! हमपर धैर्य उड़ेल दे और हमें इस दशा में उठा कि हम मुस्लिम (आज्ञाकारी) हो।"
\end{hindi}}
\flushright{\begin{Arabic}
\quranayah[7][127]
\end{Arabic}}
\flushleft{\begin{hindi}
फ़िरऔन की क़ौम के सरदार कहने लगे, "क्या तुम मूसा और उसकी क़ौम को ऐसे ही छोड़ दोगे कि वे ज़मीन में बिगाड़ पैदा करें और वे तुम्हें और तुम्हारे उपास्यों को छोड़ बैठे?" उसने कहा, "हम उनके बेटों को बुरी तरह क़त्ल करेंगे और उनकी स्त्रियों को जीवित रखेंगे। निश्चय ही हमें उनपर पूर्ण अधिकार प्राप्त है।"
\end{hindi}}
\flushright{\begin{Arabic}
\quranayah[7][128]
\end{Arabic}}
\flushleft{\begin{hindi}
मूसा ने अपनी क़ौम से कहा, "अल्लाह से सम्बद्ध होकर सहायता प्राप्त करो और धैर्य से काम लो। धरती अल्लाह की है। वह अपने बन्दों में से जिसे चाहता है, उसका वारिस बना देता है। और अंतिम परिणाम तो डर रखनेवालों ही के लिए है।"
\end{hindi}}
\flushright{\begin{Arabic}
\quranayah[7][129]
\end{Arabic}}
\flushleft{\begin{hindi}
उन्होंने कहा, "तुम्हारे आने से पहले भी हम सताए गए और तुम्हारे आने के बाद भी।" उसने कहा, "निकट है कि तुम्हारा रब तुम्हारे शत्रुओं को विनष्ट कर दे और तुम्हें धरती में ख़लीफ़ा बनाए, फिर यह देखे कि तुम कैसे कर्म करते हो।"
\end{hindi}}
\flushright{\begin{Arabic}
\quranayah[7][130]
\end{Arabic}}
\flushleft{\begin{hindi}
और हमने फ़िरऔनियों को कई वर्ष तक अकाल और पैदावार की कमी में ग्रस्त रखा कि वे चेतें
\end{hindi}}
\flushright{\begin{Arabic}
\quranayah[7][131]
\end{Arabic}}
\flushleft{\begin{hindi}
फिर जब उन्हें अच्छी हालत पेश आती है तो कहते है, "यह तो है ही हमारे लिए।" और उन्हें बुरी हालत पेश आए तो वे उसे मूसा और उसके साथियों की नहूसत (अशकुन) ठहराएँ। सुन लो, उसकी नहूसत तो अल्लाह ही के पास है, परन्तु उनमें से अधिकतर लोग जानते नहीं
\end{hindi}}
\flushright{\begin{Arabic}
\quranayah[7][132]
\end{Arabic}}
\flushleft{\begin{hindi}
वे बोले, "तू हमपर जादू करने के लिए चाहे कोई भी निशानी हमारे पास ले आए, हम तुझपर ईमान लानेवाले नहीं।"
\end{hindi}}
\flushright{\begin{Arabic}
\quranayah[7][133]
\end{Arabic}}
\flushleft{\begin{hindi}
अन्ततः हमने उनपर तूफ़ान और टिड्डियों और छोटे कीड़े और मेंढक और रक्त, कितनी ही निशानियाँ अलग-अलग भेजी, किन्तु वे घमंड ही करते रहे। वे थे ही अपराधी लोग
\end{hindi}}
\flushright{\begin{Arabic}
\quranayah[7][134]
\end{Arabic}}
\flushleft{\begin{hindi}
जब कभी उनपर यातना आ पड़ती, कहते है, "ऐ मूसा, हमारे लिए अपने रब से प्रार्थना करो, उस प्रतिज्ञा के आधार पर जो उसने तुमसे कर रखी है। तुमने यदि हमपर से यह यातना हटा दी, तो हम अवश्य ही तुमपर ईमान ले आएँगे और इसराईल की सन्तान को तुम्हारे साथ जाने देंगे।"
\end{hindi}}
\flushright{\begin{Arabic}
\quranayah[7][135]
\end{Arabic}}
\flushleft{\begin{hindi}
किन्तु जब हम उनपर से यातना को एक नियत समय के लिए जिस तक वे पहुँचनेवाले ही थे, हटा लेते तो क्या देखते कि वे वचन-भंग करने लग गए
\end{hindi}}
\flushright{\begin{Arabic}
\quranayah[7][136]
\end{Arabic}}
\flushleft{\begin{hindi}
फिर हमने उनसे बदला लिया और उन्हें गहरे पानी में डूबो दिया, क्योंकि उन्होंने हमारी निशानियों को ग़लत समझा और उनसे ग़ाफिल हो गए
\end{hindi}}
\flushright{\begin{Arabic}
\quranayah[7][137]
\end{Arabic}}
\flushleft{\begin{hindi}
और जो लोग कमज़ोर पाए जाते थे, उन्हें हमने उस भू-भाग के पूरब के हिस्सों और पश्चिम के हिस्सों का उत्तराधिकारी बना दिया, जिसे हमने बरकत दी थी। और तुम्हारे रब का अच्छा वादा इसराईल की सन्तान के हक़ में पूरा हुआ, क्योंकि उन्होंने धैर्य से काम लिया और फ़िरऔन और उसकी क़ौम का वह सब कुछ हमने विनष्ट कर दिया, जिसे वे बनाते और ऊँचा उठाते थे
\end{hindi}}
\flushright{\begin{Arabic}
\quranayah[7][138]
\end{Arabic}}
\flushleft{\begin{hindi}
और इसराईल की सन्तान को हमने सागर से पार करा दिया, फिर वे ऐसे लोगों को पास पहुँचे जो अपनी कुछ मूर्तियों से लगे बैठे थे। कहने लगे, "ऐ मूसा! हमारे लिए भी कोई ऐसा उपास्य ठहरा दे, जैसे इनके उपास्य है।" उसने कहा, "निश्चय ही तुम बड़े ही अज्ञानी लोग हो
\end{hindi}}
\flushright{\begin{Arabic}
\quranayah[7][139]
\end{Arabic}}
\flushleft{\begin{hindi}
“निश्चय ही वह लोग लगे हुए है, बरबाद होकर रहेगा। और जो कुछ ये कर रहे है सर्वथा व्यर्थ है।"
\end{hindi}}
\flushright{\begin{Arabic}
\quranayah[7][140]
\end{Arabic}}
\flushleft{\begin{hindi}
उसने कहा, "क्या मैं अल्लाह के सिवा तुम्हारे लिए कोई और उपास्य ढूढूँ, हालाँकि उसी ने सारे संसारवालों पर तुम्हें श्रेष्ठता प्रदान की?"
\end{hindi}}
\flushright{\begin{Arabic}
\quranayah[7][141]
\end{Arabic}}
\flushleft{\begin{hindi}
और याद करो जब हमने तुम्हें फ़िरऔन के लोगों से छुटकारा दिया जो तुम्हें बुरी यातना में ग्रस्त रखते थे। तुम्हारे बेटों को मार डालते और तुम्हारी स्त्रियों को जीवित रहने देते थे। और वह (छुटकारा दिलाना) तुम्हारे रब की ओर से बड़ा अनुग्रह है
\end{hindi}}
\flushright{\begin{Arabic}
\quranayah[7][142]
\end{Arabic}}
\flushleft{\begin{hindi}
और हमने मूसा से तीस रातों का वादा ठहराया, फिर हमने दस और बढ़ाकर उसे पूरा किया। इसी प्रकार उसके रब की ठहराई हुई अवधि चालीस रातों में पूरी हुई और मूसा ने अपने भाई हारून से कहा, "मेरे पीछे तुम मेरी क़ौम में मेरा प्रतिनिधित्व करना और सुधारना और बिगाड़ पैदा करनेवालों के मार्ग पर न चलना।"
\end{hindi}}
\flushright{\begin{Arabic}
\quranayah[7][143]
\end{Arabic}}
\flushleft{\begin{hindi}
अब मूसा हमारे निश्चित किए हुए समय पर पहुँचा और उसके रब ने उससे बातें की, तो वह करने लगा, "मेरे रब! मुझे देखने की शक्ति प्रदान कर कि मैं तुझे देखूँ।" कहा, "तू मुझे कदापि न देख सकेगा। हाँ, पहाड़ की ओर देख। यदि वह अपने स्थान पर स्थिर पर स्थिर रह जाए तो फिर तू मुझे देख लेगा।" अतएव जब उसका रब पहाड़ पर प्रकट हुआ तो उसे चकनाचूर कर दिया और मूसा मूर्छित होकर गिर पड़ा। फिर जब होश में आया तो कहा, "महिमा है तेरी! मैं तेरे समझ तौबा करता हूँ और सबसे पहला ईमान लानेवाला मैं हूँ।"
\end{hindi}}
\flushright{\begin{Arabic}
\quranayah[7][144]
\end{Arabic}}
\flushleft{\begin{hindi}
उसने कहा, "ऐ मूसा! मैंने दूसरे लोगों के मुक़ाबले में तुझे चुनकर अपने संदेशों और अपनी वाणी से तुझे उपकृत किया। अतः जो कुछ मैं तुझे दूँ उसे ले और कृतज्ञता दिखा।"
\end{hindi}}
\flushright{\begin{Arabic}
\quranayah[7][145]
\end{Arabic}}
\flushleft{\begin{hindi}
और हमने उसके लिए तख़्तियों पर उपदेश के रूप में हर चीज़ और हर चीज़ का विस्तृत वर्णन लिख दिया। अतः उनको मज़बूती से पकड़। उनमें उत्तम बातें है। अपनी क़ौम के लोगों को हुक्म दे कि वे उनको अपनाएँ। मैं शीघ्र ही तुम्हें अवज्ञाकारियों का घर दिखाऊँगा
\end{hindi}}
\flushright{\begin{Arabic}
\quranayah[7][146]
\end{Arabic}}
\flushleft{\begin{hindi}
जो लोग धरती में नाहक़ बड़े बनते है, मैं अपनी निशानियों की ओर से उन्हें फेर दूँगा। यदि वे प्रत्येक निशानी देख ले तब भी वे उस पर ईमान नहीं लाएँगे। यदि वे सीधा मार्ग देख लें तो भी वे उसे अपना मार्ग नहीं बनाएँगे। लेकिन यदि वे पथभ्रष्ट का मार्ग देख लें तो उसे अपना मार्ग ठहरा लेंगे। यह इसलिए की उन्होंने हमारी आयतों को झुठलाया और उनसे ग़ाफ़िल रहे
\end{hindi}}
\flushright{\begin{Arabic}
\quranayah[7][147]
\end{Arabic}}
\flushleft{\begin{hindi}
जिन लोगों ने हमारी आयतों को और आख़िरत के मिलन को झूठा जाना, उनका तो सारा किया-धरा उनकी जान को लागू हुआ। जो कुछ वे करते रहे क्या उसके सिवा वे किसी और चीज़ का बदला पाएँगे?
\end{hindi}}
\flushright{\begin{Arabic}
\quranayah[7][148]
\end{Arabic}}
\flushleft{\begin{hindi}
और मूसा के पीछे उसकी क़ौम ने अपने ज़ेवरों से अपने लिए एक बछड़ा बना दिया, जिसमें से बैल की-सी आवाज़ निकलती थी। क्या उन्होंने देखा नहीं कि वह न तो उनसे बातें करता है और न उन्हें कोई राह दिखाता है? उन्होंने उसे अपना उपास्य बना लिया, औऱ वे बड़े अत्याचारी थे
\end{hindi}}
\flushright{\begin{Arabic}
\quranayah[7][149]
\end{Arabic}}
\flushleft{\begin{hindi}
और जब (चेताबनी से) उन्हें पश्चाताप हुआ और उन्होंने देख लिया कि वास्तव में वे भटक गए हैं तो कहने लगे, "यदि हमारे रब ने हमपर दया न की और उसने हमें क्षमा न किया तो हम घाटे में पड़ जाएँगे!"
\end{hindi}}
\flushright{\begin{Arabic}
\quranayah[7][150]
\end{Arabic}}
\flushleft{\begin{hindi}
और जब मूसा क्रोध और दुख से भरा हुआ अपनी क़ौम की ओर लौटा तो उसने कहा, "तुम लोगों ने मेरे पीछे मेरी जगह बुरा किया। क्या तुम अपने रब के हुक्म से पहले ही जल्दी कर बैठे?" फिर उसने तख़्तियाँ डाल दी और अपने भाई का सिर पकड़कर उसे अपनी ओर खींचने लगा। वह बोला, "ऐ मेरी माँ के बेटे! लोगों ने मुझे कमज़ोर समझ लिया और निकट था कि मुझे मार डालते। अतः शत्रुओं को मुझपर हुलसने का अवसर न दे और अत्याचारी लोगों में मुझे सम्मिलित न कर।"
\end{hindi}}
\flushright{\begin{Arabic}
\quranayah[7][151]
\end{Arabic}}
\flushleft{\begin{hindi}
उसने कहा, "मेरे रब! मुझे और मेरे भाई को क्षमा कर दे और हमें अपनी दयालुता में दाख़िल कर ले। तू तो सबसे बढ़कर दयावान हैं।"
\end{hindi}}
\flushright{\begin{Arabic}
\quranayah[7][152]
\end{Arabic}}
\flushleft{\begin{hindi}
जिन लोगों ने बछड़े को अपना उपास्य बनाया, वे अपने रब की ओर से प्रकोप और सांसारिक जीवन में अपमान से ग्रस्त होकर रहेंगे; और झूठ घड़नेवालों को हम ऐसा ही बदला देते है
\end{hindi}}
\flushright{\begin{Arabic}
\quranayah[7][153]
\end{Arabic}}
\flushleft{\begin{hindi}
रहे वे लोग जिन्होंने बुरे कर्म किए फिर उसके पश्चात तौबा कर ली और ईमान ले आए, तो इसके बाद तो तुम्हारा रब बड़ा ही क्षमाशील, दयाशील है
\end{hindi}}
\flushright{\begin{Arabic}
\quranayah[7][154]
\end{Arabic}}
\flushleft{\begin{hindi}
और जब मूसा का क्रोध शान्त हुआ तो उसने तख़्तियों को उठा लिया। उनके लेख में उन लोगों के लिए मार्गदर्शन और दयालुता थी जो अपने रब से डरते है
\end{hindi}}
\flushright{\begin{Arabic}
\quranayah[7][155]
\end{Arabic}}
\flushleft{\begin{hindi}
मूसा ने अपनी क़ौम के सत्तर आदमियों को हमारे नियत किए हुए समय के लिए चुना। फिर जब उन लोगों को एक भूकम्प ने आ पकड़ा तो उसने कहा, "मेर रब! यदि तू चाहता तो पहले ही इनको और मुझको विनष्ट़ कर देता। जो कुछ हमारे नादानों ने किया है, क्या उसके कारण तू हमें विनष्ट करेगा? यह तो बस तेरी ओर से एक परीक्षा है। इसके द्वारा तू जिसको चाहे पथभ्रष्ट कर दे और जिसे चाहे मार्ग दिखा दे। तू ही हमारा संरक्षक है। अतः तू हमें क्षमा कर दे और हम पर दया कर, और तू ही सबसे बढ़कर क्षमा करनेवाला है
\end{hindi}}
\flushright{\begin{Arabic}
\quranayah[7][156]
\end{Arabic}}
\flushleft{\begin{hindi}
"और हमारे लिए इस संसार में भलाई लिख दे और आख़िरत में भी। हम तेरी ही ओर उन्मुख हुए।" उसने कहा, "अपनी यातना में मैं तो उसी को ग्रस्त करता हूँ, जिसे चाहता हूँ, किन्तु मेरी दयालुता से हर चीज़ आच्छादित है। उसे तो मैं उन लोगों के हक़ में लिखूँगा जो डर रखते और ज़कात देते है और जो हमारी आयतों पर ईमान लाते है
\end{hindi}}
\flushright{\begin{Arabic}
\quranayah[7][157]
\end{Arabic}}
\flushleft{\begin{hindi}
"(तो आज इस दयालुता के अधिकारी वे लोग है) जो उस रसूल, उम्मी नबी का अनुसरण करते है, जिसे वे अपने यहाँ तौरात और इंजील में लिखा पाते है। और जो उन्हें भलाई का हुक्म देता और बुराई से रोकता है। उनके लिए अच्छी-स्वच्छ चीज़ों का हलाल और बुरी-अस्वच्छ चीज़ों का हराम ठहराता है और उनपर से उनके वह बोझ उतारता है, जो अब तक उनपर लदे हुए थे और उन बन्धनों को खोलता है, जिनमें वे जकड़े हुए थे। अतः जो लोग उसपर ईमान लाए, उसका सम्मान किया और उसकी सहायता की और उस प्रकाश के अनुगत हुए, जो उसके साथ अवतरित हुआ है, वही सफलता प्राप्त करनेवाले है।"
\end{hindi}}
\flushright{\begin{Arabic}
\quranayah[7][158]
\end{Arabic}}
\flushleft{\begin{hindi}
कहो, "ऐ लोगो! मैं तुम सबकी ओर उस अल्लाह का रसूल हूँ, जो आकाशों और धरती के राज्य का स्वामी है उसके सिवा कोई पूज्य नहीं, वही जीवन प्रदान करता और वही मृत्यु देता है। अतः जीवन प्रदान करता और वही मृत्यु देता है। अतः अल्लाह और उसके रसूल, उस उम्मी नबी, पर ईमान लाओ जो स्वयं अल्लाह पर और उसके शब्दों (वाणी) पर ईमान रखता है और उनका अनुसरण करो, ताकि तुम मार्ग पा लो।"
\end{hindi}}
\flushright{\begin{Arabic}
\quranayah[7][159]
\end{Arabic}}
\flushleft{\begin{hindi}
मूसा की क़ौम में से एक गिरोह ऐसे लोगों का भी हुआ जो हक़ के अनुसार मार्ग दिखाते और उसी के अनुसार न्याय करते
\end{hindi}}
\flushright{\begin{Arabic}
\quranayah[7][160]
\end{Arabic}}
\flushleft{\begin{hindi}
और हमने उन्हें बारह ख़ानदानों में विभक्त करके अलग-अलग समुदाय बना दिया। जब उसकी क़ौम के लोगों ने पानी माँगा तो हमने मूसा की ओर प्रकाशना की, "अपनी लाठी अमुक चट्टान पर मारो।" अतएव उससे बारह स्रोत फूट निकले और हर गिरोह ने अपना-अपना घाट मालूम कर लिया। और हमने उनपर बादल की छाया की और उन पर 'मन्न' और 'सलवा' उतारा, "हमनें तुम्हें जो अच्छी-स्वच्छ चीज़े प्रदान की है, उन्हें खाओ।" उन्होंने हम पर कोई ज़ुल्म नहीं किया, बल्कि वास्तव में वे स्वयं अपने ऊपर ही ज़ुल्म करते रहे
\end{hindi}}
\flushright{\begin{Arabic}
\quranayah[7][161]
\end{Arabic}}
\flushleft{\begin{hindi}
याद करो जब उनसे कहा गया, "इस बस्ती में रहो-बसो और इसमें जहाँ से चाहो खाओ और कहो - हित्ततुन। और द्वार में सजदा करते हुए प्रवेश करो। हम तुम्हारी ख़ताओं को क्षमा कर देंगे और हम सुकर्मी लोगों को अधिक भी देंगे।"
\end{hindi}}
\flushright{\begin{Arabic}
\quranayah[7][162]
\end{Arabic}}
\flushleft{\begin{hindi}
किन्तु उनमें से जो अत्याचारी थे उन्होंने, जो कुछ उनसे कहा गया था, उसको उससे भिन्न बात से बदल दिया। अतः जो अत्याचार वे कर रहे थे, उसके कारण हमने आकाश से उनपर यातना भेजी
\end{hindi}}
\flushright{\begin{Arabic}
\quranayah[7][163]
\end{Arabic}}
\flushleft{\begin{hindi}
उनसे उस बस्ती के विषय में पूछो जो सागर-तट पर थी। जब वे सब्त के मामले में सीमा का उल्लंघन करते थे, जब उनके सब्त के दिन उनकी मछलियाँ खुले तौर पर पानी के ऊपर आ जाती थी और जो दिन उनके सब्त का न होता तो वे उनके पास न आती थी। इस प्रकार उनके अवज्ञाकारी होने के कारण हम उनको परीक्षा में डाल रहे थे
\end{hindi}}
\flushright{\begin{Arabic}
\quranayah[7][164]
\end{Arabic}}
\flushleft{\begin{hindi}
और जब उनके एक गिरोह ने कहा, "तुम ऐसे लोगों को क्यों नसीहत किए जा रहे हो, जिन्हें अल्लाह विनष्ट करनेवाला है या जिन्हें वह कठोर यातना देनेवाला है?" उन्होंने कहा, "तुम्हारे रब के समक्ष अपने को निरपराध सिद्ध करने के लिए, और कदाचित वे (अवज्ञा से) बचें।"
\end{hindi}}
\flushright{\begin{Arabic}
\quranayah[7][165]
\end{Arabic}}
\flushleft{\begin{hindi}
फिर जब वे उसे भूल गए जो नसीहत उन्हें दी गई थी तो हमने उन लोगों को बचा लिया, जो बुराई से रोकते थे और अत्याचारियों को उनकी अवज्ञा के कारण कठोर यातना में पकड़ लिया
\end{hindi}}
\flushright{\begin{Arabic}
\quranayah[7][166]
\end{Arabic}}
\flushleft{\begin{hindi}
फिर जब वे सरकशी के साथ वही कुछ करते रहे, जिससे उन्हें रोका गया था तो हमने उनसे कहा, "बन्दर हो जाओ, अपमानित और तिरस्कृत!"
\end{hindi}}
\flushright{\begin{Arabic}
\quranayah[7][167]
\end{Arabic}}
\flushleft{\begin{hindi}
और याद करो जब तुम्हारे रब ने ख़बर कर दी थी कि वह क़ियामत के दिन तक उनके विरुद्ध ऐसे लोगों को उठाता रहेगा, जो उन्हें बुरी यातना देंगे। निश्चय ही तुम्हारा रब जल्द सज़ा देता है और वह बड़ा क्षमाशील, दावान भी है
\end{hindi}}
\flushright{\begin{Arabic}
\quranayah[7][168]
\end{Arabic}}
\flushleft{\begin{hindi}
और हमने उन्हें टुकड़े-टुकड़े करके धरती में अनेक गिरोहों में बिखेर दिया। कुछ उनमें से नेक है और कुछ उनमें इससे भिन्न हैं, और हमने उन्हें अच्छी और बुरी परिस्थितियों में डालकर उनकी परीक्षा ली, कदाचित वे पलट आएँ
\end{hindi}}
\flushright{\begin{Arabic}
\quranayah[7][169]
\end{Arabic}}
\flushleft{\begin{hindi}
फिर उनके पीछ ऐसे अयोग्य लोगों ने उनकी जगह ली, जो किताब के उत्ताराधिकारी होकर इसी तुच्छ संसार का सामान समेटते है और कहते है, "हमें अवश्य क्षमा कर दिया जाएगा।" और यदि इस जैसा और सामान भी उनके पास आ जाए तो वे उसे भी ले लेंगे। क्या उनसे किताब का यह वचन नहीं लिया गया था कि वे अल्लाह पर थोपकर हक़ के सिवा कोई और बात न कहें। और जो उसमें है उसे वे स्वयं पढ़ भी चुके है। और आख़िरत का घर तो उन लोगों के लिए उत्तम है, जो डर रखते है। तो क्या तुम बुद्धि से काम नहीं लेते?
\end{hindi}}
\flushright{\begin{Arabic}
\quranayah[7][170]
\end{Arabic}}
\flushleft{\begin{hindi}
और जो लोग किताब को मज़बूती से थामते है और जिन्होंने नमाज़ क़ायम कर रखी है, तो काम को ठीक रखनेवालों के प्रतिदान को हम कभी अकारथ नहीं करते
\end{hindi}}
\flushright{\begin{Arabic}
\quranayah[7][171]
\end{Arabic}}
\flushleft{\begin{hindi}
और याद करो जब हमने पर्वत को हिलाया, जो उनके ऊपर था। मानो वह कोई छत्र हो और वे समझे कि बस वह उनपर गिरा ही चाहता है, "थामो मज़बूती से, जो कुछ हमने दिया है। और जो कुछ उसमें है उसे याद रखो, ताकि तुम बच सको।"
\end{hindi}}
\flushright{\begin{Arabic}
\quranayah[7][172]
\end{Arabic}}
\flushleft{\begin{hindi}
और याद करो जब तुम्हारे रब ने आदम की सन्तान से (अर्थात उनकी पीठों से) उनकी सन्तति निकाली और उन्हें स्वयं उनके ऊपर गवाह बनाया कि "क्या मैं तुम्हारा रब नहीं हूँ?" बोले, "क्यों नहीं, हम गवाह है।" ऐसा इसलिए किया कि तुम क़ियामत के दिन कहीं यह न कहने लगो कि "हमें तो इसकी ख़बर ही न थी।"
\end{hindi}}
\flushright{\begin{Arabic}
\quranayah[7][173]
\end{Arabic}}
\flushleft{\begin{hindi}
या कहो कि "(अल्लाह के साथ) साझी तो पहले हमारे बाप-दादा ने किया। हम तो उसके पश्चात उनकी सन्तति में हुए है। तो क्या तू हमें उसपर विनष्ट करेगा जो कुछ मिथ्याचारियों ने किया है?"
\end{hindi}}
\flushright{\begin{Arabic}
\quranayah[7][174]
\end{Arabic}}
\flushleft{\begin{hindi}
इस प्रकार स्थिति के अनुकूल आयतें प्रस्तुत करते है। और शायद कि वे पलट आएँ
\end{hindi}}
\flushright{\begin{Arabic}
\quranayah[7][175]
\end{Arabic}}
\flushleft{\begin{hindi}
और उन्हें उस व्यक्ति का हाल सुनाओ जिसे हमने अपनी आयतें प्रदान की किन्तु वह उनसे निकल भागा। फिर शैतान ने उसे अपने पीछे लगा लिया। अन्ततः वह पथभ्रष्ट और विनष्ट होकर रहा
\end{hindi}}
\flushright{\begin{Arabic}
\quranayah[7][176]
\end{Arabic}}
\flushleft{\begin{hindi}
यदि हम चाहते तो इन आयतों के द्वारा उसे उच्चता प्रदान करते, किन्तु वह तो धरती के साथ लग गया और अपनी इच्छा के पीछे चला। अतः उसकी मिसाल कुत्ते जैसी है कि यदि तुम उसपर आक्षेप करो तब भी वह ज़बान लटकाए रहे या यदि तुम उसे छोड़ दो तब भी वह ज़बान लटकाए ही रहे। यही मिसाल उन लोगों की है, जिन्होंने हमारी आयतों को झुठलाया, तो तुम वृत्तान्त सुनाते रहो, कदाचित वे सोच-विचार कर सकें
\end{hindi}}
\flushright{\begin{Arabic}
\quranayah[7][177]
\end{Arabic}}
\flushleft{\begin{hindi}
बुरे है मिसाल की दृष्टि से वे लोग, जिन्होंने हमारी आयतों को झुठलाया और वे स्वयं अपने ही ऊपर अत्याचार करते रहे
\end{hindi}}
\flushright{\begin{Arabic}
\quranayah[7][178]
\end{Arabic}}
\flushleft{\begin{hindi}
जिसे अल्लाह मार्ग दिखाए वही सीधा मार्ग पानेवाला है और जिसे वह मार्ग से वंचित रखे, तो ऐसे ही लोग घाटे में पड़नेवाले हैं
\end{hindi}}
\flushright{\begin{Arabic}
\quranayah[7][179]
\end{Arabic}}
\flushleft{\begin{hindi}
निश्चय ही हमने बहुत-से जिन्नों और मनुष्यों को जहन्नम ही के लिए फैला रखा है। उनके पास दिल है जिनसे वे समझते नहीं, उनके पास आँखें है जिनसे वे देखते नहीं; उनके पास कान है जिनसे वे सुनते नहीं। वे पशुओं की तरह है, बल्कि वे उनसे भी अधिक पथभ्रष्ट है। वही लोग है जो ग़फ़लत में पड़े हुए है
\end{hindi}}
\flushright{\begin{Arabic}
\quranayah[7][180]
\end{Arabic}}
\flushleft{\begin{hindi}
अच्छे नाम अल्लाह ही के है। तो तुम उन्हीं के द्वारा उसे पुकारो और उन लोगों को छोड़ो जो उसके नामों के सम्बन्ध में कुटिलता ग्रहण करते है। जो कुछ वे करते है, उसका बदला वे पाकर रहेंगे
\end{hindi}}
\flushright{\begin{Arabic}
\quranayah[7][181]
\end{Arabic}}
\flushleft{\begin{hindi}
हमारे पैदा किए प्राणियों में कुछ लोग ऐसे भी है जो हक़ के अनुसार मार्ग दिखाते और उसी के अनुसार न्याय करते है
\end{hindi}}
\flushright{\begin{Arabic}
\quranayah[7][182]
\end{Arabic}}
\flushleft{\begin{hindi}
रहे वे लोग जिन्होंने हमारी आयतों को झुठलाया, हम उन्हें क्रमशः तबाही की ओर ले जाएँगे, ऐसे तरीक़े से जिसे वे जानते नहीं
\end{hindi}}
\flushright{\begin{Arabic}
\quranayah[7][183]
\end{Arabic}}
\flushleft{\begin{hindi}
मैं तो उन्हें ढील दिए जा रहा हूँ। निश्चय ही मेरी चाल अत्यन्त सुदृढ़ है
\end{hindi}}
\flushright{\begin{Arabic}
\quranayah[7][184]
\end{Arabic}}
\flushleft{\begin{hindi}
क्या उन लोगों ने विचार नहीं किया? उनके साथी को कोई उन्माद नहीं। वह तो बस एक साफ़-साफ़ सचेत करनेवाला है
\end{hindi}}
\flushright{\begin{Arabic}
\quranayah[7][185]
\end{Arabic}}
\flushleft{\begin{hindi}
या क्या उन्होंने आकाशों और धरती के राज्य पर और जो चीज़ भी अल्लाह ने पैदा की है उसपर दृष्टि नहीं डाली, और इस बात पर कि कदाचित उनकी अवधि निकट आ लगी हो? फिर आख़िर इसके बाद अब कौन-सी बात हो सकती है, जिसपर वे ईमान लाएँगे?
\end{hindi}}
\flushright{\begin{Arabic}
\quranayah[7][186]
\end{Arabic}}
\flushleft{\begin{hindi}
जिसे अल्लाह मार्ग से वंचित रखे उसके लिए कोई मार्गदर्शक नहीं। वह तो तो उन्हें उनकी सरकशी ही में भटकता हुआ छोड़ रहा है
\end{hindi}}
\flushright{\begin{Arabic}
\quranayah[7][187]
\end{Arabic}}
\flushleft{\begin{hindi}
तुमसे उस घड़ी (क़ियामत) के विषय में पूछते है कि वह कब आएगी? कह दो, "उसका ज्ञान मेरे रब ही के पास है। अतः वही उसे उसके समय पर प्रकट करेगा। वह आकाशों और धरती में बोझिल हो गई है - बस अचानक ही वह तुमपर आ जाएगी।" वे तुमसे पूछते है मानो तुम उसके विषय में भली-भाँति जानते हो। कह दो, "उसका ज्ञान तो बस अल्लाह ही के पास है - किन्तु अधिकांश लोग नहीं जानते।"
\end{hindi}}
\flushright{\begin{Arabic}
\quranayah[7][188]
\end{Arabic}}
\flushleft{\begin{hindi}
कहो, "मैं अपने लिए न तो लाभ का अधिकार रखता हूँ और न हानि का,बल्कि अल्लाह ही की इच्छा क्रियान्वित है। यदि मुझे परोक्ष (ग़ैब) का ज्ञान होता तो बहुत-सी भलाई समेट लेता और मुझे कभी कोई हानि न पहुँचती। मैं तो बस सचेत करनेवाला हूँ, उन लोगों के लिए जो ईमान लाएँ।"
\end{hindi}}
\flushright{\begin{Arabic}
\quranayah[7][189]
\end{Arabic}}
\flushleft{\begin{hindi}
वही है जिसने तुम्हें अकेली जान पैदा किया और उसी की जाति से उसका जोड़ा बनाया, ताकि उसकी ओर प्रवृत्त होकर शान्ति और चैन प्राप्त करे। फिर जब उसने उसको ढाँक लिया तो उसने एक हल्का-सा बोझ उठा लिया; फिर वह उसे लिए हुए चलती-फिरती रही, फिर जब वह बोझिल हो गई तो दोनों ने अल्लाह - अपने रब को पुकारा, "यदि तूने हमें भला-चंगा बच्चा दिया, तो निश्चय ही हम तेरे कृतज्ञ होंगे।"
\end{hindi}}
\flushright{\begin{Arabic}
\quranayah[7][190]
\end{Arabic}}
\flushleft{\begin{hindi}
किन्तु उसने जब उन्हें भला-चंगा (बच्चा) प्रदान किया तो जो उन्हें प्रदान किया उसमें वे दोनों उसका (अल्लाह का) साझी ठहराने लगे। किन्तु अल्लाह तो उच्च है उससे, जो साझी वे ठहराते है
\end{hindi}}
\flushright{\begin{Arabic}
\quranayah[7][191]
\end{Arabic}}
\flushleft{\begin{hindi}
क्या वे उसको साझी ठहराते है जो कोई चीज़ भी पैदा नहीं करता, बल्कि ऐसे उनके ठहराए हुए साझीदार तो स्वयं पैदा किए जाते हैं
\end{hindi}}
\flushright{\begin{Arabic}
\quranayah[7][192]
\end{Arabic}}
\flushleft{\begin{hindi}
और वे न तो उनकी सहायता करने की सामर्थ्य रखते है और न स्वयं अपनी ही सहायता कर सकते है?
\end{hindi}}
\flushright{\begin{Arabic}
\quranayah[7][193]
\end{Arabic}}
\flushleft{\begin{hindi}
यदि तुम उन्हें सीधे मार्ग की ओर बुलाओ तो वे तुम्हारे पीछे न आएँगे। तुम्हारे लिए बराबर है - उन्हें पुकारो या तुम चुप रहो
\end{hindi}}
\flushright{\begin{Arabic}
\quranayah[7][194]
\end{Arabic}}
\flushleft{\begin{hindi}
तुम अल्लाह को छोड़कर जिन्हें पुकारते हो वे तो तुम्हारे ही जैसे बन्दे है, अतः पुकार लो उनको, यदि तुम सच्चे हो, तो उन्हें चाहिए कि वे तुम्हें उत्तर दे!
\end{hindi}}
\flushright{\begin{Arabic}
\quranayah[7][195]
\end{Arabic}}
\flushleft{\begin{hindi}
क्या उनके पाँव हैं जिनसे वे चलते हों या उनके हाथ हैं जिनसे वे पकड़ते हों या उनके पास आँखें हीं जिनसे वे देखते हों या उनके कान हैं जिनसे वे सुनते हों? कहों, "तुम अपने ठहराए हु सहभागियों को बुला लो, फिर मेरे विरुद्ध चालें न चलो, इस प्रकार कि मुझे मुहलत न दो
\end{hindi}}
\flushright{\begin{Arabic}
\quranayah[7][196]
\end{Arabic}}
\flushleft{\begin{hindi}
निश्चय ही मेरा संरक्षक मित्र अल्लाह है, जिसने यह किताब उतारी और वह अच्छे लोगों का संरक्षण करता है
\end{hindi}}
\flushright{\begin{Arabic}
\quranayah[7][197]
\end{Arabic}}
\flushleft{\begin{hindi}
रहे वे जिन्हें तुम उसको छोड़कर पुकारते हो, वे तो तुम्हारी, सहायता करने की सामर्थ्य रखते है और न स्वयं अपनी ही सहायता कर सकते है
\end{hindi}}
\flushright{\begin{Arabic}
\quranayah[7][198]
\end{Arabic}}
\flushleft{\begin{hindi}
और यदि तुम उन्हें सीधे मार्ग की ओर बुलाओ तो वे न सुनेंगे। वे तुम्हें ऐसे दीख पड़ते हैं जैसे वे तुम्हारी ओर ताक रहे हैं, हालाँकि वे कुछ भी नहीं देखते
\end{hindi}}
\flushright{\begin{Arabic}
\quranayah[7][199]
\end{Arabic}}
\flushleft{\begin{hindi}
क्षमा की नीति अपनाओ और भलाई का हुक्म देते रहो और अज्ञानियों से किनारा खींचो
\end{hindi}}
\flushright{\begin{Arabic}
\quranayah[7][200]
\end{Arabic}}
\flushleft{\begin{hindi}
और यदि शैतान तुम्हें उकसाए तो अल्लाह की शरण माँगो। निश्चय ही, वह सब कुछ सुनता जानता है
\end{hindi}}
\flushright{\begin{Arabic}
\quranayah[7][201]
\end{Arabic}}
\flushleft{\begin{hindi}
जो डर रखते हैं, उन्हें जब शैतान की ओर से कोई ख़याल छू जाता है, तो वे चौंक उठते हैं। फिर वे साफ़ देखने लगते हैं
\end{hindi}}
\flushright{\begin{Arabic}
\quranayah[7][202]
\end{Arabic}}
\flushleft{\begin{hindi}
और उन (शैतानों) के भाई उन्हें गुमराही में खींचे लिए जाते हैं, फिर वे कोई कमी नहीं करते
\end{hindi}}
\flushright{\begin{Arabic}
\quranayah[7][203]
\end{Arabic}}
\flushleft{\begin{hindi}
और जब तुम उनके सामने कोई निशानी नहीं लाते तो वे कहते हैं, "तुम स्वयं कोई निशानी क्यों न छाँट लाए?" कह दो, "मैं तो केवल उसी का अनुसरण करता हूँ जो मेरे रब की ओर से प्रकाशना की जाती है। यह तुम्हारे रब की ओर से अन्तर्दृष्टियों का प्रकाश-पुंज है, और ईमान लानेवालों के लिए मार्गदर्शन और दयालुता है।"
\end{hindi}}
\flushright{\begin{Arabic}
\quranayah[7][204]
\end{Arabic}}
\flushleft{\begin{hindi}
जब क़ुरआन पढ़ा जाए तो उसे ध्यानपूर्वक सुनो और चुप रहो, ताकि तुमपर दया की जाए
\end{hindi}}
\flushright{\begin{Arabic}
\quranayah[7][205]
\end{Arabic}}
\flushleft{\begin{hindi}
अपने रब को अपने मन में प्रातः और संध्या के समयों में विनम्रतापूर्वक, डरते हुए और हल्की आवाज़ के साथ याद किया करो। और उन लोगों में से न हो जाओ जो ग़फ़लत में पड़े हुए है
\end{hindi}}
\flushright{\begin{Arabic}
\quranayah[7][206]
\end{Arabic}}
\flushleft{\begin{hindi}
निस्संदेह जो तुम्हारे रब के पास है, वे उसकी बन्दगी के मुक़ाबले में अहंकार की नीति नहीं अपनाते; वे तो उसकी तसबीह (महिमागान) करते है और उसी को सजदा करते है
\end{hindi}}
\chapter{Al-Anfal (Voluntary Gifts)}
\begin{Arabic}
\Huge{\centerline{\basmalah}}\end{Arabic}
\flushright{\begin{Arabic}
\quranayah[8][1]
\end{Arabic}}
\flushleft{\begin{hindi}
वे तुमसे ग़नीमतों के विषय में पूछते है। कहो, "ग़नीमतें अल्लाह और रसूल की है। अतः अल्लाह का डर रखों और आपस के सम्बन्धों को ठीक रखो। और, अल्लाह और उसके रसूल की आज्ञा का पालन करो, यदि तुम ईमानवाले हो
\end{hindi}}
\flushright{\begin{Arabic}
\quranayah[8][2]
\end{Arabic}}
\flushleft{\begin{hindi}
ईमानवाले तो वही लोग है जिनके दिल उस समय काँप उठे जबकि अल्लाह को याद किया जाए। और जब उनके सामने उसकी आयतें पढ़ी जाएँ तो वे उनके ईमान को और अधिक बढ़ा दें और वे अपने रब पर भरोसा रखते हों
\end{hindi}}
\flushright{\begin{Arabic}
\quranayah[8][3]
\end{Arabic}}
\flushleft{\begin{hindi}
ये वे लोग हैं जो नमाज़ क़ायम करते है और जो कुछ हमने दिया है उसमें से ख़र्च करते हैं
\end{hindi}}
\flushright{\begin{Arabic}
\quranayah[8][4]
\end{Arabic}}
\flushleft{\begin{hindi}
वही लोग वास्तव में ईमानवाले है। उनके लिेए रब के पास बड़े दर्जे है और क्षमा और सम्मानित उत्तम आजीविका भी
\end{hindi}}
\flushright{\begin{Arabic}
\quranayah[8][5]
\end{Arabic}}
\flushleft{\begin{hindi}
(यह बिल्कुल वैसी ही परिस्थित है) जैसे तुम्हारे ने तुम्हें तुम्हारे घर से एक उद्देश्य के साथ निकाला, किन्तु ईमानवालों में से एक गिरोह को यह अप्रिय लगा था
\end{hindi}}
\flushright{\begin{Arabic}
\quranayah[8][6]
\end{Arabic}}
\flushleft{\begin{hindi}
वे सत्य के विषय में उसके स्पष्ट हो जाने के पश्चात तुमसे झगड़ रहे थे। मानो वे आँखों देखी मृत्यु की ओर हाँके जा रहे हों
\end{hindi}}
\flushright{\begin{Arabic}
\quranayah[8][7]
\end{Arabic}}
\flushleft{\begin{hindi}
और याद करो जब अल्लाह तुमसे वादा कर रहा था कि दो गिरोहों में से एक तुम्हारे हाथ आएगा और तुम चाहते थे कि तुम्हें वह हाथ आए, जो निःशस्त्र था, हालाँकि अल्लाह चाहता था कि अपने वचनों से सत्य को सत्य कर दिखाए और इनकार करनेवालों की जड़ काट दे
\end{hindi}}
\flushright{\begin{Arabic}
\quranayah[8][8]
\end{Arabic}}
\flushleft{\begin{hindi}
ताकि सत्य को सत्य कर दिखाए और असत्य को असत्य, चाहे अपराधियों को कितना ही अप्रिय लगे
\end{hindi}}
\flushright{\begin{Arabic}
\quranayah[8][9]
\end{Arabic}}
\flushleft{\begin{hindi}
याद करो जब तुम अपने रब से फ़रियाद कर रहे थे, तो उसने तुम्हारी पुकार सुन ली। (उसने कहा,) "मैं एक हजार फ़रिश्तों से तुम्हारी मदद करूँगा जो तुम्हारे साथी होंगे।"
\end{hindi}}
\flushright{\begin{Arabic}
\quranayah[8][10]
\end{Arabic}}
\flushleft{\begin{hindi}
अल्लाह ने यह केवल इसलिए किया कि यह एक शुभ-सूचना हो और ताकि इससे तुम्हारे हृदय संतुष्ट हो जाएँ। सहायता अल्लाह ही के यहाँ से होती है। निस्संदेह अल्लाह अत्यन्त प्रभुत्वशाली, तत्वदर्शी है
\end{hindi}}
\flushright{\begin{Arabic}
\quranayah[8][11]
\end{Arabic}}
\flushleft{\begin{hindi}
यह करो जबकि वह अपनी ओर से चैन प्रदान कर तुम्हें ऊँघ से ढँक रहा था और वह आकाश से तुमपर पानी बरसा रहा था, ताकि उसके द्वारा तुम्हें अच्छी तरह पाक करे और शैतान की गन्दगी तुमसे दूर करे और तुम्हारे दिलों को मज़बूत करे और उसके द्वारा तुम्हारे क़दमों को जमा दे
\end{hindi}}
\flushright{\begin{Arabic}
\quranayah[8][12]
\end{Arabic}}
\flushleft{\begin{hindi}
याद करो जब तुम्हारा रब फ़रिश्तों की ओर प्रकाशना (वह्य्) कर रहा था कि "मैं तुम्हारे साथ हूँ। अतः तुम ईमानवालों को जमाए रखो। मैं इनकार करनेवालों के दिलों में रोब डाले देता हूँ। तो तुम उनकी गरदनें मारो और उनके पोर-पोर पर चोट लगाओ!"
\end{hindi}}
\flushright{\begin{Arabic}
\quranayah[8][13]
\end{Arabic}}
\flushleft{\begin{hindi}
यह इसलिए कि उन्होंने अल्लाह और उसके रसूल का विरोध किया। और जो कोई अल्लाह और उसके रसूल का विरोध करे (उसे कठोर यातना मिलकर रहेगी) क्योंकि अल्लाह कड़ी यातना देनेवाला है
\end{hindi}}
\flushright{\begin{Arabic}
\quranayah[8][14]
\end{Arabic}}
\flushleft{\begin{hindi}
यह तो तुम चखो! और यह कि इनकार करनेवालों के लिए आग की यातना है
\end{hindi}}
\flushright{\begin{Arabic}
\quranayah[8][15]
\end{Arabic}}
\flushleft{\begin{hindi}
ऐ ईमान लानेवालो! जब एक सेना के रूप में तुम्हारा इनकार करनेवालों से मुक़ाबला हो तो पीठ न फेरो
\end{hindi}}
\flushright{\begin{Arabic}
\quranayah[8][16]
\end{Arabic}}
\flushleft{\begin{hindi}
जिस किसी ने भी उस दिन उनसे अपनी पीठ फेरी - यह और बात है कि युद्ध-चाल के रूप में या दूसरी टुकड़ी से मिलने के लिए ऐसा करे - तो वह अल्लाह के प्रकोप का भागी हुआ और उसका ठिकाना जहन्नम है, और क्या ही बुरा जगह है वह पहुँचने की!
\end{hindi}}
\flushright{\begin{Arabic}
\quranayah[8][17]
\end{Arabic}}
\flushleft{\begin{hindi}
तुमने उसे क़त्ल नहीं किया बल्कि अल्लाह ही ने उन्हें क़त्ल किया और जब तुमने (उनकी ओर मिट्टी और कंकड़) फेंक, तो तुमने नहीं फेंका बल्कि अल्लाह ने फेंका (कि अल्लाह अपनी गुण-गरिमा दिखाए) और ताकि अपनी ओर से ईमानवालों के गुण प्रकट करे। निस्संदेह अल्लाह सुनता, जानता है
\end{hindi}}
\flushright{\begin{Arabic}
\quranayah[8][18]
\end{Arabic}}
\flushleft{\begin{hindi}
यह तो हुआ, और यह (जान लो) कि अल्लाह इनकार करनेवालों की चाल को कमज़ोर कर देनेवाला है
\end{hindi}}
\flushright{\begin{Arabic}
\quranayah[8][19]
\end{Arabic}}
\flushleft{\begin{hindi}
यदि तुम फ़ैसला चाहते हो तो फ़ैसला तुम्हारे सामने आ चुका और यदि बाज़ आ जाओ तो यह तुम्हारे ही लिए अच्छा है। लेकिन यदि तुमने पलटकर फिर वही हरकत की तो हम भी पलटेंगे और तुम्हारा जत्था, चाहे वह कितना ही अधिक हो, तुम्हारे कुछ काम न आ सकेगा। और यह कि अल्लाह मोमिनों के साथ होता है
\end{hindi}}
\flushright{\begin{Arabic}
\quranayah[8][20]
\end{Arabic}}
\flushleft{\begin{hindi}
ऐ ईमान लानेवालो! अल्लाह और उसके रसूल का आज्ञापालन करो और उससे मुँह न फेरो जबकि तुम सुन रहे हो
\end{hindi}}
\flushright{\begin{Arabic}
\quranayah[8][21]
\end{Arabic}}
\flushleft{\begin{hindi}
और उन लोगों की तरह न हो जाना जिन्होंने कहा था, "हमने सुना" हालाँकि वे सुनते नहीं
\end{hindi}}
\flushright{\begin{Arabic}
\quranayah[8][22]
\end{Arabic}}
\flushleft{\begin{hindi}
अल्लाह की स्पष्ट में तो निकृष्ट पशु वे बहरे-गूँगे लोग है, जो बुद्धि से काम नहीं लेते
\end{hindi}}
\flushright{\begin{Arabic}
\quranayah[8][23]
\end{Arabic}}
\flushleft{\begin{hindi}
यदि अल्लाह जानता कि उनमें कुछ भी भलाई है, तो वह उन्हें अवश्य सुनने का सौभाग्य प्रदान करता। और यदि वह उन्हें सुना देता तो भी वे कतराते हुए मुँह फेर लेते
\end{hindi}}
\flushright{\begin{Arabic}
\quranayah[8][24]
\end{Arabic}}
\flushleft{\begin{hindi}
ऐ ईमान लानेवाले! अल्लाह और रसूल की बात मानो, जब वह तुम्हें उस चीज़ की ओर बुलाए जो तुम्हें जीवन प्रदान करनेवाली है, और जान रखो कि अल्लाह आदमी और उसके दिल के बीच आड़े आ जाता है और यह कि वही है जिसकी ओर (पलटकर) तुम एकत्र होगे
\end{hindi}}
\flushright{\begin{Arabic}
\quranayah[8][25]
\end{Arabic}}
\flushleft{\begin{hindi}
बचो उस फ़ितने से जो अपनी लपेट में विशेष रूप से केवल अत्याचारियों को ही नहीं लेगा, जान लो अल्लाह कठोर दंड देनेवाला है
\end{hindi}}
\flushright{\begin{Arabic}
\quranayah[8][26]
\end{Arabic}}
\flushleft{\begin{hindi}
और याद करो जब तुम थोड़े थे, धरती में निर्बल थे, डरे-सहमे रहते कि लोग कहीं तु्म्हें उचक न ले जाएँ, फिर उसने तुम्हें ठिकाना दिया और अपनी सहायता से तुम्हें शक्ति प्रदान की और अच्छी-स्वच्छ चीज़ों की तुम्हें रोजी दी, ताकि तुम कृतज्ञता दिखलाओ
\end{hindi}}
\flushright{\begin{Arabic}
\quranayah[8][27]
\end{Arabic}}
\flushleft{\begin{hindi}
ऐ ईमान लानेवालो! जानते-बुझते तुम अल्लाह और उसके रसूल के साथ विश्वासघात न करना और न अपनी अमानतों में ख़ियानत करना
\end{hindi}}
\flushright{\begin{Arabic}
\quranayah[8][28]
\end{Arabic}}
\flushleft{\begin{hindi}
और जान रखो कि तुम्हारे माल और तुम्हारी संतान परीक्षा-सामग्री हैं और यह कि अल्लाह के पास बड़ा प्रतिदान है
\end{hindi}}
\flushright{\begin{Arabic}
\quranayah[8][29]
\end{Arabic}}
\flushleft{\begin{hindi}
ऐ ईमान लानेवालो! यदि तुम अल्लाह का डर रखोगे तो वह तुम्हें एक विशिष्टता प्रदान करेगा और तुमसे तुम्हारी बुराइयाँ दूर करेगा और तुम्हे क्षमा करेगा। अल्लाह बड़ा अनुग्राहक है
\end{hindi}}
\flushright{\begin{Arabic}
\quranayah[8][30]
\end{Arabic}}
\flushleft{\begin{hindi}
और याद करो जब इनकार करनेवाले तुम्हारे साथ चालें चल रहे थे कि तम्हें क़ैद रखें या तुम्हे क़त्ल कर दें या तुम्हे निकाल बाहर करे। वे अपनी चालें चल रहे थे और अल्लाह भी अपनी चाल चल रहा था। अल्लाह सबसे अच्छी चाल चलता है
\end{hindi}}
\flushright{\begin{Arabic}
\quranayah[8][31]
\end{Arabic}}
\flushleft{\begin{hindi}
जब उनके सामने हमारी आयतें पढ़ी जाती है, तो वे कहते है, "हम सुन चुके। यदि हम चाहें तो ऐसी बातें हम भी बना लें; ये तो बस पहले के लोगों की कहानियाँ हैं।"
\end{hindi}}
\flushright{\begin{Arabic}
\quranayah[8][32]
\end{Arabic}}
\flushleft{\begin{hindi}
और याद करो जब उन्होंने कहा, "ऐ अल्लाह! यदि यही तेरे यहाँ से सत्य हो तो हमपर आकाश से पत्थर बरसा दे, या हम पर कोई दुखद यातना ही ले आ
\end{hindi}}
\flushright{\begin{Arabic}
\quranayah[8][33]
\end{Arabic}}
\flushleft{\begin{hindi}
और अल्लाह ऐसा नहीं कि तुम उनके बीच उपस्थित हो और वह उन्हें यातना देने लग जाए, और न अल्लाह ऐसा है कि वे क्षमा-याचना कर रहे हो और वह उन्हें यातना से ग्रस्त कर दे
\end{hindi}}
\flushright{\begin{Arabic}
\quranayah[8][34]
\end{Arabic}}
\flushleft{\begin{hindi}
किन्तु अब क्या है उनके पास कि अल्लाह उन्हें यातना न दे, जबकि वे 'मस्जिदे हराम' (काबा) से रोकते है, हालाँकि वे उसके कोई व्यवस्थापक नहीं? उसके व्यवस्थापक तो केवल डर रखनेवाले ही है, परन्तु उनके अधिकतर लोग जानते नहीं
\end{hindi}}
\flushright{\begin{Arabic}
\quranayah[8][35]
\end{Arabic}}
\flushleft{\begin{hindi}
उनकी नमाज़़ इस घर (काबा) के पास सीटियाँ बजाने और तालियाँ पीटने के अलावा कुछ भी नहीं होती। तो अब यातना का मज़ा चखो, उस इनकार के बदले में जो तुम करते रहे हो
\end{hindi}}
\flushright{\begin{Arabic}
\quranayah[8][36]
\end{Arabic}}
\flushleft{\begin{hindi}
निश्चय ही इनकार करनेवाले अपने माल अल्लाह के मार्ग से रोकने के लिए ख़र्च करते रहेंगे, फिर यही उनके लिए पश्चाताप बनेगा। फिर वे पराभूत होंगे और इनकार करनेवाले जहन्नम की ओर समेट लाए जाएँगे
\end{hindi}}
\flushright{\begin{Arabic}
\quranayah[8][37]
\end{Arabic}}
\flushleft{\begin{hindi}
ताकि अल्लाह नापाक को पाक से छाँटकर अलग करे और नापाकों को आपस में एक-दूसरे पर रखकर ढेर बनाए, फिर उसे जहन्नम में डाल दे। यही लोग घाटे में पड़नेवाले है
\end{hindi}}
\flushright{\begin{Arabic}
\quranayah[8][38]
\end{Arabic}}
\flushleft{\begin{hindi}
उन इनकार करनेवालो से कह दो कि वे यदि बाज़ आ जाएँ तो जो कुछ हो चुका, उसे क्षमा कर दिया जाएगा, किन्तु यदि वे फिर भी वहीं करेंगे तो पूर्ववर्ती लोगों के सिलसिले में जो रीति अपनाई गई वह सामने से गुज़र चुकी है
\end{hindi}}
\flushright{\begin{Arabic}
\quranayah[8][39]
\end{Arabic}}
\flushleft{\begin{hindi}
उनसे युद्ध करो, यहाँ तक कि फ़ितना बाक़ी न रहे और दीन (धर्म) पूरा का पूरा अल्लाह ही के लिए हो जाए। फिर यदि वे बाज़ आ जाएँ तो अल्लाह उनके कर्म को देख रहा है
\end{hindi}}
\flushright{\begin{Arabic}
\quranayah[8][40]
\end{Arabic}}
\flushleft{\begin{hindi}
किन्तु यदि वे मुँह मोड़े तो जान रखो कि अल्लाह संरक्षक है। क्या ही अच्छा संरक्षक है वह, और क्या ही अच्छा सहायक!
\end{hindi}}
\flushright{\begin{Arabic}
\quranayah[8][41]
\end{Arabic}}
\flushleft{\begin{hindi}
और तुम्हें मालूम हो कि जो कुछ ग़नीमत के रूप में माल तुमने प्राप्त किया है, उसका पाँचवा भाग अल्लाग का, रसूल का, नातेदारों का, अनाथों का, मुहताजों और मुसाफ़िरों का है। यदि तुम अल्लाह पर और उस चीज़ पर ईमान रखते हो, जो हमने अपने बन्दे पर फ़ैसले के दिन उतारी, जिस दिन दोनों सेनाओं में मुठभेड़ हूई, और अल्लाह को हर चीज़ की पूर्ण सामर्थ्य प्राप्त है
\end{hindi}}
\flushright{\begin{Arabic}
\quranayah[8][42]
\end{Arabic}}
\flushleft{\begin{hindi}
याद करो जब तुम घाटी के निकटवर्ती छोर पर थे और वे घाटी के दूरस्थ छोर पर थे और क़ाफ़िला तुमसे नीचे की ओर था। यदि तुम परस्पर समय निश्चित किए होते तो अनिवार्यतः तुम निश्चित समय पर न पहुँचते। किन्तु जो कुछ हुआ वह इसलिए कि अल्लाह उस बात का फ़ैसला कर दे, जिसका पूरा होना निश्चित था, ताकि जिसे विनष्ट होना हो, वह स्पष्ट प्रमाण देखकर ही विनष्ट हो और जिसे जीवित रहना हो वह स्पष्ट़ प्रमाण देखकर जीवित रहे। निस्संदेह अल्लाह भली-भाँति जानता, सुनता है
\end{hindi}}
\flushright{\begin{Arabic}
\quranayah[8][43]
\end{Arabic}}
\flushleft{\begin{hindi}
और याद करो जब अल्लाह उनको तुम्हारे स्वप्न में थोड़ा करके तुम्हें दिखा रहा था और यदि वह उन्हें ज़्यादा करके तुम्हें दिखा देता तो अवश्य ही तुम हिम्मत हार बैठते और असल मामले में झगड़ने लग जाते, किन्तु अल्लाह ने इससे बचा लिया। निश्चय ही वह तो जो कुछ दिलों में होता है उसे भी जानता है
\end{hindi}}
\flushright{\begin{Arabic}
\quranayah[8][44]
\end{Arabic}}
\flushleft{\begin{hindi}
याद करो जब तुम्हारी परस्पर मुठभेड़ हुई तो वह तुम्हारी निगाहों में उन्हें कम करके और तुम्हें उनकी निगाहों में कम करके दिखा रहा था, ताकि अल्लाह उस बात का फ़ैसला कर दे जिसका होना निश्चित था। और सारे मामले अल्लाह ही की ओर पलटते है
\end{hindi}}
\flushright{\begin{Arabic}
\quranayah[8][45]
\end{Arabic}}
\flushleft{\begin{hindi}
ऐ ईमान लानेवालो! जब तुम्हारा किसी गिरोह से मुक़ाबला हो जाए तो जमे रहो और अल्लाह को ज़्यादा याद करो, ताकि तुम्हें सफलता प्राप्त हो
\end{hindi}}
\flushright{\begin{Arabic}
\quranayah[8][46]
\end{Arabic}}
\flushleft{\begin{hindi}
और अल्लाह और उसके रसूल की आज्ञा मानो और आपस में न झगड़ो, अन्यथा हिम्मत हार बैठोगे और तुम्हारी हवा उखड़ जाएगी। और धैर्य से काम लो। निश्चय ही, अल्लाह धैर्यवानों के साथ है
\end{hindi}}
\flushright{\begin{Arabic}
\quranayah[8][47]
\end{Arabic}}
\flushleft{\begin{hindi}
और उन लोगों की तरह न हो जाना जो अपने घरों से इतराते और लोगों को दिखाते निकले थे और वे अल्लाह के मार्ग से रोकते है, हालाँकि जो कुछ वे करते है, अल्लाह उसे अपने घेरे में लिए हुए है
\end{hindi}}
\flushright{\begin{Arabic}
\quranayah[8][48]
\end{Arabic}}
\flushleft{\begin{hindi}
और याद करो जब शैतान ने उनके कर्म उनके लिए सुन्दर बना दिए और कहा, "आज लोगों में से कोई भी तुमपर प्रभावी नहीं हो सकता। मैं तुम्हारे साथ हूँ।" किन्तु जब दोनों गिरोह आमने-सामने हुए तो वह उलटे पाँव फिर गया और कहने लगा, "मेरा तुमसे कोई सम्बन्ध नहीं। मैं वह कुछ देख रहा हूँ, जो तुम्हें नहीं दिखाई देता। मैं अल्लाह से डरता हूँ, और अल्लाह कठोर यातना देनेवाला है।"
\end{hindi}}
\flushright{\begin{Arabic}
\quranayah[8][49]
\end{Arabic}}
\flushleft{\begin{hindi}
याद करो जब कपटाचारी और वे लोग जिनके दिलों में रोग है, कह रहे थे, "इन लोगों को तो इनके धर्म ने धोखे में डाल रखा है।" हालाँकि जो अल्लाह पर भरोसा रखता है, तो निश्चय ही अल्लाह अत्यन्त प्रभुत्वशाली, तत्वदर्शी है
\end{hindi}}
\flushright{\begin{Arabic}
\quranayah[8][50]
\end{Arabic}}
\flushleft{\begin{hindi}
क्या ही अच्छा होता कि तुम देखते जब फ़रिश्ते इनकार करनेवालों के प्राण ग्रस्त करते हैं! वे उनके चहरों और उनकी पीठों पर मारते जाते हैं कि "लो अब जलने की यातना मज़ा चखो।" (तो उनकी दुर्दशा का अन्दाजा कर सकते)
\end{hindi}}
\flushright{\begin{Arabic}
\quranayah[8][51]
\end{Arabic}}
\flushleft{\begin{hindi}
यह तो उसी का बदला है जो तुम्हारे हाथों ने आगे भेजा और यह कि अल्लाह अपने बन्दों पर तनिक भी अत्याचार नहीं करता
\end{hindi}}
\flushright{\begin{Arabic}
\quranayah[8][52]
\end{Arabic}}
\flushleft{\begin{hindi}
इनके साथ वैसा ही मामला पेश आया जैसा फ़िरऔन के लोगों और उनसे पहले के लोगों के साथ पेश आया। उन्होंने अल्लाह की आयतों का इनकार किया तो अल्लाह ने उनके गुनाहों के कारण उन्हें पकड़ लिया। निस्संदेह अल्लाह शक्तिशाली, कठोर यातना देनेवाला है
\end{hindi}}
\flushright{\begin{Arabic}
\quranayah[8][53]
\end{Arabic}}
\flushleft{\begin{hindi}
यह इसलिए हुआ कि अल्लाह उस उदार अनुग्रह (नेमत) को, जो उसने किसी क़ौम पर किया हो, बदलनेवाला नहीं हैं, जब तक कि लोग उस चीज़ को न बदल डालें, जिसका सम्बन्ध स्वयं उनसे है। और यह कि अल्लाह सब कुछ सुनता, जानता है
\end{hindi}}
\flushright{\begin{Arabic}
\quranayah[8][54]
\end{Arabic}}
\flushleft{\begin{hindi}
जैसे फ़िरऔनियों और उनसे पहले के लोगों का हाल हुआ। उन्होंने अपने रब की आयतों को झुठलाया तो हमने उन्हें उनके गुनाहों के बदले में विनष्ट कर दिया और फ़िरऔनियों को डूबो दिया। ये सभी अत्याचारी थे
\end{hindi}}
\flushright{\begin{Arabic}
\quranayah[8][55]
\end{Arabic}}
\flushleft{\begin{hindi}
निश्चय ही, सबसे बुरे प्राणी अल्लाह की स्पष्ट में वे लोग है, जिन्होंने इनकार किया। फिर वे ईमान नहीं लाते
\end{hindi}}
\flushright{\begin{Arabic}
\quranayah[8][56]
\end{Arabic}}
\flushleft{\begin{hindi}
जिनसे तुमने वचन लिया वे फिर हर बार अपने वचन को भंग कर देते है और वे डर नहीं रखते
\end{hindi}}
\flushright{\begin{Arabic}
\quranayah[8][57]
\end{Arabic}}
\flushleft{\begin{hindi}
अतः यदि युद्ध में तुम उनपर क़ाबू पाओ, तो उनके साथ इस तरह पेश आओ कि उनके पीछेवाले भी भाग खड़े हों, ताकि वे शिक्षा ग्रहण करें
\end{hindi}}
\flushright{\begin{Arabic}
\quranayah[8][58]
\end{Arabic}}
\flushleft{\begin{hindi}
और यदि तुम्हें किसी क़ौम से विश्वासघात की आशंका हो, तो तुम भी उसी प्रकार ऐसे लोगों के साथ हुई संधि को खुल्लम-खुल्ला उनके आगे फेंक दो। निश्चय ही अल्लाह को विश्वासघात करनेवाले प्रिय नहीं
\end{hindi}}
\flushright{\begin{Arabic}
\quranayah[8][59]
\end{Arabic}}
\flushleft{\begin{hindi}
इनकार करनेवाले यह न समझे कि वे आगे निकल गए। वे क़ाबू से बाहर नहीं जा सकते
\end{hindi}}
\flushright{\begin{Arabic}
\quranayah[8][60]
\end{Arabic}}
\flushleft{\begin{hindi}
और जो भी तुमसे हो सके, उनके लिए बल और बँधे घोड़े तैयार रखो, ताकि इसके द्वारा अल्लाह के शत्रुओं और अपने शत्रुओं और इनके अतिरिक्त उन दूसरे लोगों को भी भयभीत कर दो जिन्हें तुम नहीं जानते। अल्लाह उनको जानता है और अल्लाह के मार्ग में तुम जो कुछ भी ख़र्च करोगे, वह तुम्हें पूरा-पूरा चुका दिया जाएगा और तुम्हारे साथ कदापि अन्याय न होगा
\end{hindi}}
\flushright{\begin{Arabic}
\quranayah[8][61]
\end{Arabic}}
\flushleft{\begin{hindi}
और यदि वे संधि और सलामती की ओर झुकें तो तुम भी इसके लिए झुक जाओ और अल्लाह पर भरोसा रखो। निस्संदेह, वह सब कुछ सुनता, जानता है
\end{hindi}}
\flushright{\begin{Arabic}
\quranayah[8][62]
\end{Arabic}}
\flushleft{\begin{hindi}
और यदि वे यह चाहें कि तुम्हें धोखा दें तो तुम्हारे लिए अल्लाह काफ़ी है। वही तो है जिसने तुम्हें अपनी सहायता से और मोमिनों के द्वारा शक्ति प्रदान की
\end{hindi}}
\flushright{\begin{Arabic}
\quranayah[8][63]
\end{Arabic}}
\flushleft{\begin{hindi}
और उनके दिल आपस में एक-दूसरे से जोड़ दिए। यदि तुम धरती में जो कुछ है, सब खर्च कर डालते तो भी उनके दिलों को परस्पर जोड़ न सकते, किन्तु अल्लाह ने उन्हें परस्पर जोड़ दिया। निश्चय ही वह अत्यन्त प्रभुत्वशाली, तत्वदर्शी है
\end{hindi}}
\flushright{\begin{Arabic}
\quranayah[8][64]
\end{Arabic}}
\flushleft{\begin{hindi}
ऐ नबी! तुम्हारे लिए अल्लाह और तुम्हारे ईमानवाले अनुयायी ही काफ़ी है
\end{hindi}}
\flushright{\begin{Arabic}
\quranayah[8][65]
\end{Arabic}}
\flushleft{\begin{hindi}
ऐ नबी! मोमिनों को जिहाद पर उभारो। यदि तुम्हारे बीस आदमी जमे होंगे, तो वे दो सौ पर प्रभावी होंगे और यदि तुमसे से ऐसे सौ होंगे तो वे इनकार करनेवालों में से एक हज़ार पर प्रभावी होंगे, क्योंकि वे नासमझ लोग है
\end{hindi}}
\flushright{\begin{Arabic}
\quranayah[8][66]
\end{Arabic}}
\flushleft{\begin{hindi}
अब अल्लाह ने तुम्हारे बोझ हल्का कर दिया और उसे मालूम हुआ कि तुममें कुछ कमज़ोरी है। तो यदि तुम्हारे सौ आदमी जमे रहनेवाले होंगे, तो वे दो सौ पर प्रभावी रहेंगे और यदि तुममें से ऐसे हजार होंगे तो अल्लाह के हुक्म से वे दो हज़ार पर प्रभावी रहेंगे। अल्लाह तो उन्ही लोगों के साथ है जो जमे रहते है
\end{hindi}}
\flushright{\begin{Arabic}
\quranayah[8][67]
\end{Arabic}}
\flushleft{\begin{hindi}
किसी नबी के लिए यह उचित नहीं कि उसके पास क़ैदी हो यहाँ तक की वह धरती में रक्तपात करे। तुम लोग संसार की सामग्री चाहते हो, जबकि अल्लाह आख़िरत चाहता है। अल्लाह अत्यन्त प्रभुत्वशाली, तत्वदर्शी है
\end{hindi}}
\flushright{\begin{Arabic}
\quranayah[8][68]
\end{Arabic}}
\flushleft{\begin{hindi}
यदि अल्लाह का लिखा पहले से मौजूद न होता, तो जो कुछ नीति तुमने अपनाई है उसपर तुम्हें कोई बड़ी यातना आ लेती
\end{hindi}}
\flushright{\begin{Arabic}
\quranayah[8][69]
\end{Arabic}}
\flushleft{\begin{hindi}
अतः जो कुछ ग़नीमत का माल तुमने प्राप्त किया है, उसे वैध-पवित्र समझकर खाओ और अल्लाह का डर रखो। निश्चय ही अल्लाह बड़ा क्षमाशील, अत्यन्त दयावान है
\end{hindi}}
\flushright{\begin{Arabic}
\quranayah[8][70]
\end{Arabic}}
\flushleft{\begin{hindi}
ऐ नबी! जो क़ैदी तुम्हारे क़ब्जें में है, उनसे कह दो, "यदि अल्लाह ने यह जान लिया कि तुम्हारे दिलों में कुछ भलाई है तो वह तुम्हें उससे कहीं उत्तम प्रदान करेगा, जो तुम से छिन गया है और तुम्हें क्षमा कर देगा। और अल्लाह अत्यन्त क्षमाशील, दयावान है।"
\end{hindi}}
\flushright{\begin{Arabic}
\quranayah[8][71]
\end{Arabic}}
\flushleft{\begin{hindi}
किन्तुम यदि वे तुम्हारे साथ विश्वासघात करना चाहेंगे, तो इससे पहले वे अल्लाह के साथ विश्वासघात कर चुके है। तो उसने तुम्हें उनपर अधिकार दे दिया। अल्लाह सब कुछ जाननेवाला, बड़ा तत्वदर्शी है
\end{hindi}}
\flushright{\begin{Arabic}
\quranayah[8][72]
\end{Arabic}}
\flushleft{\begin{hindi}
जो लोग ईमान लाए और उन्होंने हिजरत की और अल्लाह के मार्ग में अपने मालों और अपनी जानों के साथ जिहाद किया और जिन लोगों ने उन्हें शरण दी और सहायता की, वही लोग परस्पर एक-दूसरे के संरक्षक मित्र है। रहे वे लोग जो ईमान लाए, किन्तु उन्होंने हिजरत नहीं की, उनसे तुम्हारा संरक्षण और मित्रता का कोई सम्बन्ध नहीं है, जब तक कि वे हिजरत न करें, किन्तु यदि वे धर्म के मामले में तुमसे सहायता माँगे तो तुमपर अनिवार्य है कि सहायता करो, सिवाय इसके कि सहायता किसी ऐसी क़ौम के मुक़ाबले में हो जिससे तुम्हारी कोई संधि हो। तुम जो कुछ करते हो अल्लाह उसे देखता है
\end{hindi}}
\flushright{\begin{Arabic}
\quranayah[8][73]
\end{Arabic}}
\flushleft{\begin{hindi}
जो इनकार करनेवाले लोग है, वे आपस में एक-दूसरे के मित्र और सहायक है। यदि तुम ऐसा नहीं करोगे तो धरती में फ़ितना और बड़ा फ़साद फैलेगा
\end{hindi}}
\flushright{\begin{Arabic}
\quranayah[8][74]
\end{Arabic}}
\flushleft{\begin{hindi}
और जो लोग ईमान लाए और उन्होंने हिजरत की और अल्लाह के मार्ग में जिहाद किया और जिन लोगों ने उन्हें शरण दी और सहायता की वही सच्चे मोमिन हैं। उनके क्षमा और सम्मानित - उत्तम आजीविका है
\end{hindi}}
\flushright{\begin{Arabic}
\quranayah[8][75]
\end{Arabic}}
\flushleft{\begin{hindi}
और जो लोग बाद में ईमान लाए और उन्होंने हिजरत की और तुम्हारे साथ मिलकर जिहाद किया तो ऐसे लोग भी तुम में ही से हैं। किन्तु अल्लाह की किताब मे ख़ून के रिश्तेदार एक-दूसरे के ज़्यादा हक़दार है। निश्चय ही अल्लाह को हर चीज़ का ज्ञान है
\end{hindi}}
\chapter{Al-Bara'at / At-Taubah(The Immunity)}
\begin{Arabic}
\Huge{\centerline{\basmalah}}\end{Arabic}
\flushright{\begin{Arabic}
\quranayah[9][1]
\end{Arabic}}
\flushleft{\begin{hindi}
मुशरिकों (बहुदेववादियों) से जिनसे तुमने संधि की थी, विरक्ति (की उद्घॊषणा) है अल्लाह और उसके रसूल की ओर से
\end{hindi}}
\flushright{\begin{Arabic}
\quranayah[9][2]
\end{Arabic}}
\flushleft{\begin{hindi}
"अतः इस धरती में चार महीने और चल-फिर लो और यह बात जान लो कि अल्लाह के क़ाबू से बाहर नहीं जा सकते और यह कि अल्लाह इनकार करनेवालों को अपमानित करता है।"
\end{hindi}}
\flushright{\begin{Arabic}
\quranayah[9][3]
\end{Arabic}}
\flushleft{\begin{hindi}
सार्वजनिक उद्घॊषणा है अल्लाह और उसके रसूल की ओर से, बड़े हज के दिन लोगों के लिए, कि "अल्लाह मुशरिकों के प्रति जिम्मेदार से बरी है और उसका रसूल भी। अब यदि तुम तौबा कर लो, तो यह तुम्हारे ही लिए अच्छा है, किन्तु यदि तुम मुह मोड़ते हो, तो जान लो कि तुम अल्लाह के क़ाबू से बाहर नहीं जा सकते।" और इनकार करनेवालों के लिए एक दुखद यातना की शुभ-सूचना दे दो
\end{hindi}}
\flushright{\begin{Arabic}
\quranayah[9][4]
\end{Arabic}}
\flushleft{\begin{hindi}
सिवाय उन मुशरिकों के जिनसे तुमने संधि-समझौते किए, फिर उन्होंने तुम्हारे साथ अपने वचन को पूर्ण करने में कोई कमी नही की और न तुम्हारे विरुद्ध किसी की सहायता ही की, तो उनके साथ उनकी संधि को उन लोगों के निर्धारित समय तक पूरा करो। निश्चय ही अल्लाह को डर रखनेवाले प्रिय है
\end{hindi}}
\flushright{\begin{Arabic}
\quranayah[9][5]
\end{Arabic}}
\flushleft{\begin{hindi}
फिर, जब हराम (प्रतिष्ठित) महीने बीत जाएँ तो मुशरिकों को जहाँ कहीं पाओ क़त्ल करो, उन्हें पकड़ो और उन्हें घेरो और हर घात की जगह उनकी ताक में बैठो। फिर यदि वे तौबा कर लें और नमाज़ क़ायम करें और ज़कात दें तो उनका मार्ग छोड़ दो, निश्चय ही अल्लाह बड़ा क्षमाशील, दयावान है
\end{hindi}}
\flushright{\begin{Arabic}
\quranayah[9][6]
\end{Arabic}}
\flushleft{\begin{hindi}
और यदि मुशरिकों में से कोई तुमसे शरण माँगे, तो तुम उसे शरण दे दो, यहाँ तक कि वह अल्लाह की वाणी सुन ले। फिर उसे उसके सुरक्षित स्थान पर पहुँचा दो, क्योंकि वे ऐसे लोग हैं, जिन्हें ज्ञान नहीं
\end{hindi}}
\flushright{\begin{Arabic}
\quranayah[9][7]
\end{Arabic}}
\flushleft{\begin{hindi}
इन मुशरिकों को किसी संधि की कोई ज़िम्मेदारी अल्लाह और उसके रसूल पर कैसे बाक़ी रह सकती है? - उन लोगों का मामला इससे अलग है, जिनसे तुमने मस्जिदे हराम (काबा) के पास संधि की थी, तो जब तक वे तुम्हारे साथ सीधे रहें, तब तक तुम भी उनके साथ सीधे रहो। निश्चय ही अल्लाह को डर रखनेवाले प्रिय है। -
\end{hindi}}
\flushright{\begin{Arabic}
\quranayah[9][8]
\end{Arabic}}
\flushleft{\begin{hindi}
कैसे बाक़ी रह सकती है? जबकि उनका हाल यह है कि यदि वे तुम्हें दबा पाएँ तो वे न तुम्हारे विषय में किसी नाते-रिश्ते का ख़याल रखें औऱ न किसी अभिवचन का। वे अपने मुँह ही से तुम्हें राज़ी करते है, किन्तु उनके दिल इनकार करते रहते है और उनमें अधिकतर अवज्ञाकारी है
\end{hindi}}
\flushright{\begin{Arabic}
\quranayah[9][9]
\end{Arabic}}
\flushleft{\begin{hindi}
उन्होंने अल्लाह की आयतों के बदले थोड़ा-सा मूल्य स्वीकार किया और इस प्रकार वे उसका मार्ग अपनाने से रूक गए। निश्चय ही बहुत बुरा है, जो कुछ वे कर रहे हैं
\end{hindi}}
\flushright{\begin{Arabic}
\quranayah[9][10]
\end{Arabic}}
\flushleft{\begin{hindi}
किसी मोमिन के बारे में न तो नाते-रिश्ते का ख़याल रखते है और न किसी अभिवचन का। वही लोग है जिन्होंने सीमा का उल्लंघन किया
\end{hindi}}
\flushright{\begin{Arabic}
\quranayah[9][11]
\end{Arabic}}
\flushleft{\begin{hindi}
अतः यदि वे तौबा कर लें और नमाज़ क़ायम करें और ज़कात दें तो वे धर्म के भाई हैं। और हम उन लोगों के लिए आयतें खोल-खोलकर बयान करते हैं, जो जानना चाहें
\end{hindi}}
\flushright{\begin{Arabic}
\quranayah[9][12]
\end{Arabic}}
\flushleft{\begin{hindi}
और यदि अपने अभिवचन के पश्चात वे अपनी क़समॊं कॊ तॊड़ डालॆं और तुम्हारॆ दीन (धर्म) पर चॊटें करनॆ लगॆं, तॊ फिर कुफ़्र (अधर्म) कॆ सरदारों सॆ युद्ध करॊ, उनकी क़समॆं कुछ नहीं, ताकि वॆ बाज़ आ जाऐं।
\end{hindi}}
\flushright{\begin{Arabic}
\quranayah[9][13]
\end{Arabic}}
\flushleft{\begin{hindi}
क्या तुम ऐसॆ लॊगॊं सॆ नहीं लड़ॊगॆ जिन्हॊंनॆ अपनी क़समों को तोड़ डालीं और रसूल को निकाल देना चाहा और वही हैं जिन्होंने तुमसे छेड़ में पहल की? क्या तुम उनसे डरते हो? यदि तुम मोमिन हो तो इसका ज़्यादा हक़दार अल्लाह है कि तुम उससे डरो
\end{hindi}}
\flushright{\begin{Arabic}
\quranayah[9][14]
\end{Arabic}}
\flushleft{\begin{hindi}
उनसे लड़ो। अल्लाह तुम्हारे हाथों से उन्हें यातना देगा और उन्हें अपमानित करेगा और उनके मुक़ाबले में वह तुम्हारी सहायता करेगा। और ईमानवाले लोगों के दिलों का दुखमोचन करेगा;
\end{hindi}}
\flushright{\begin{Arabic}
\quranayah[9][15]
\end{Arabic}}
\flushleft{\begin{hindi}
उनके दिलों का क्रोध मिटाएगा, अल्लाह जिसे चाहेगा, उसपर दया-दृष्टि डालेगा। अल्लाह सर्वज्ञ, तत्वदर्शी है
\end{hindi}}
\flushright{\begin{Arabic}
\quranayah[9][16]
\end{Arabic}}
\flushleft{\begin{hindi}
क्या तुमने यह समझ रखा है कि तुम ऐसे ही छोड़ दिए जाओगे, हालाँकि अल्लाह ने अभी उन लोगों को छाँटा ही नहीं, जिन्होंने तुममें से जिहाद किया और अल्लाह और उसके रसूल और मोमिनों को छोड़कर किसी को घनिष्ठ मित्र नहीं बनाया? तुम जो कुछ भी करते हो, अल्लाह उसकी ख़बर रखता है
\end{hindi}}
\flushright{\begin{Arabic}
\quranayah[9][17]
\end{Arabic}}
\flushleft{\begin{hindi}
यह मुशरिकों का काम नहीं कि वे अल्लाह की मस्जिदों को आबाद करें और उसके प्रबंधक हों, जबकि वे स्वयं अपने विरुद्ध कुफ़्र की गवाही दे रहे है। उन लोगों का सारा किया-धरा अकारथ गया और वे आग में सदैव रहेंगे
\end{hindi}}
\flushright{\begin{Arabic}
\quranayah[9][18]
\end{Arabic}}
\flushleft{\begin{hindi}
अल्लाह की मस्जिदों का प्रबंधक और उसे आबाद करनेवाला वही हो सकता है जो अल्लाह और अंतिम दिन पर ईमान लाया, नमाज़ क़ायम की और ज़कात दी और अल्लाह के सिवा किसी से न डरा। अतः ऐसे ही लोग, आशा है कि सीधा मार्ग पानेवाले होंगे
\end{hindi}}
\flushright{\begin{Arabic}
\quranayah[9][19]
\end{Arabic}}
\flushleft{\begin{hindi}
क्या तुमने हाजियों को पानी पिलाने और मस्जिदे हराम (काबा) के प्रबंध को उस क्यक्ति के काम के बराबर ठहरा लिया है, जो अल्लाह और अंतिम दिन पर ईमान लाया और उसने अल्लाह के मार्ग में संघर्ष किया?अल्लाह की दृष्टि में वे बराबर नहीं। और अल्लाह अत्याचारी लोगों को मार्ग नहीं दिखाता
\end{hindi}}
\flushright{\begin{Arabic}
\quranayah[9][20]
\end{Arabic}}
\flushleft{\begin{hindi}
जो लोग ईमान लाए और उन्होंने हिजरत की और अल्लाह के मार्ग में अपने मालों और अपनी जानों से जिहाद किया, अल्लाह के यहाँ दर्जे में वे बहुत बड़े है और वही सफल है
\end{hindi}}
\flushright{\begin{Arabic}
\quranayah[9][21]
\end{Arabic}}
\flushleft{\begin{hindi}
उन्हें उनका रब अपना दयालुता और प्रसन्नता और ऐसे बाग़ों की शुभ-सूचना देता है, जिनमें उनके लिए स्थायी सुख-सामग्री है
\end{hindi}}
\flushright{\begin{Arabic}
\quranayah[9][22]
\end{Arabic}}
\flushleft{\begin{hindi}
उनमें वे सदैव रहेंगे। निस्संदेह अल्लाह के पास बड़ा बदला है
\end{hindi}}
\flushright{\begin{Arabic}
\quranayah[9][23]
\end{Arabic}}
\flushleft{\begin{hindi}
ऐ ईमान लानेवालो! अपने बाप और अपने भाइयों को अपने मित्र न बनाओ यदि ईमान के मुक़ाबले में कुफ़्र उन्हें प्रिय हो। तुममें से जो कोई उन्हें अपना मित्र बनाएगा, तो ऐसे ही लोग अत्याचारी होंगे
\end{hindi}}
\flushright{\begin{Arabic}
\quranayah[9][24]
\end{Arabic}}
\flushleft{\begin{hindi}
कह दो, "यदि तुम्हारे बाप, तुम्हारे बेटे, तुम्हारे भाई, तुम्हारी पत्नि यों और तुम्हारे रिश्ते-नातेवाले और माल, जो तुमने कमाए है और कारोबार जिसके मन्दा पड़ जाने का तुम्हें भय है और घर जिन्हें तुम पसन्द करते हो, तुम्हे अल्लाह और उसके रसूल और उसके मार्ग में जिहाद करने से अधिक प्रिय है तो प्रतीक्षा करो, यहाँ तक कि अल्लाह अपना फ़ैसला ले आए। औऱ अल्लाह अवज्ञाकारियों को मार्ग नहीं दिखाता।"
\end{hindi}}
\flushright{\begin{Arabic}
\quranayah[9][25]
\end{Arabic}}
\flushleft{\begin{hindi}
अल्लाह बहुत-से अवसरों पर तुम्हारी सहायता कर चुका है और हुनैन (की लड़ाई) के दिन भी, जब तुम अपनी अधिकता पर फूल गए, तो वह तुम्हारे कुछ काम न आई और धरती अपनी विशालता के बावजूद तुम पर तंग हो गई। फिर तुम पीठ फेरकर भाग खड़े हुए
\end{hindi}}
\flushright{\begin{Arabic}
\quranayah[9][26]
\end{Arabic}}
\flushleft{\begin{hindi}
अन्ततः अल्लाह ने अपने रसूल पर और मोमिनों पर अपनी सकीनत (प्रशान्ति) उतारी और ऐसी सेनाएँ उतारी जिनको तुमने नहीं देखा। और इनकार करनेवालों को यातना दी, और यही इनकार करनेवालों का बदला है
\end{hindi}}
\flushright{\begin{Arabic}
\quranayah[9][27]
\end{Arabic}}
\flushleft{\begin{hindi}
फिर इसके बाद अल्लाह जिसको चाहता है उसे तौबा नसीब करता है। अल्लाह बड़ा क्षमाशील, दयावान है
\end{hindi}}
\flushright{\begin{Arabic}
\quranayah[9][28]
\end{Arabic}}
\flushleft{\begin{hindi}
ऐ ईमान लानेवालो! मुशरिक तो बस अपवित्र ही है। अतः इस वर्ष के पश्चात वे मस्जिदे हराम के पास न आएँ। और यदि तुम्हें निर्धनता का भय हो तो आगे यदि अल्लाह चाहेगा तो तुम्हें अपने अनुग्रह से समृद्ध कर देगा। निश्चय ही अल्लाह सब कुछ जाननेवाला, अत्यन्त तत्वदर्शी है
\end{hindi}}
\flushright{\begin{Arabic}
\quranayah[9][29]
\end{Arabic}}
\flushleft{\begin{hindi}
वे किताबवाले जो न अल्लाह पर ईमान रखते है और न अन्तिम दिन पर और न अल्लाह और उसके रसूल के हराम ठहराए हुए को हराम ठहराते है और न सत्यधर्म का अनुपालन करते है, उनसे लड़ो, यहाँ तक कि वे सत्ता से विलग होकर और छोटे (अधीनस्थ) बनकर जिज़्या देने लगे
\end{hindi}}
\flushright{\begin{Arabic}
\quranayah[9][30]
\end{Arabic}}
\flushleft{\begin{hindi}
यहूदी करते है, "उज़ैर अल्लाह का बेटा है।" और ईसाई कहते है, "मसीह अल्लाह का बेटा है।" ये उनकी अपने मुँह की बातें हैं। ये उन लोगों की-सी बातें कर रहे है जो इससे पहले इनकार कर चुके है। अल्लाह की मार इन पर! ये कहाँ से औधे हुए जा रहे हैं!
\end{hindi}}
\flushright{\begin{Arabic}
\quranayah[9][31]
\end{Arabic}}
\flushleft{\begin{hindi}
उन्होंने अल्लाह से हटकर अपने धर्मज्ञाताओं और संसार-त्यागी संतों और मरयम के बेटे ईसा को अपने रब बना लिए है - हालाँकि उन्हें इसके सिवा और कोई आदेश नहीं दिया गया था कि अकेले इष्टि-पूज्य की वे बन्दगी करें, जिसक सिवा कोई और पूज्य नहीं। उसकी महिमा के प्रतिकूल है वह शिर्क जो ये लोग करते है। -
\end{hindi}}
\flushright{\begin{Arabic}
\quranayah[9][32]
\end{Arabic}}
\flushleft{\begin{hindi}
चाहते है कि अल्लाह के प्रकाश को अपने मुँह से बुझा दें, किन्तु अल्लाह अपने प्रकाश को पूर्ण किए बिना नहीं रहेगा, चाहे इनकार करनेवालों को अप्रिय ही लगे
\end{hindi}}
\flushright{\begin{Arabic}
\quranayah[9][33]
\end{Arabic}}
\flushleft{\begin{hindi}
वही है जिसने अपने रसूल को मार्गदर्शन और सत्यधर्म के साथ भेजा ताकि उसे तमाम दीन (धर्म) पर प्रभावी कर दे, चाहे मुशरिकों को बुरा लगे
\end{hindi}}
\flushright{\begin{Arabic}
\quranayah[9][34]
\end{Arabic}}
\flushleft{\begin{hindi}
ऐ ईमान लानेवालो! अवश्य ही बहुत-से धर्मज्ञाता और संसार-त्यागी संत ऐसे है जो लोगो को माल नाहक़ खाते है और अल्लाह के मार्ग से रोकते है, और जो लोग सोना और चाँदी एकत्र करके रखते है और उन्हें अल्लाह के मार्ग में ख़र्च नहीं करते, उन्हें दुखद यातना की शुभ-सूचना दे दो
\end{hindi}}
\flushright{\begin{Arabic}
\quranayah[9][35]
\end{Arabic}}
\flushleft{\begin{hindi}
जिस दिन उनको जहन्नम की आग में तपाया जाएगा फिर उससे उनके ललाटो और उनके पहलुओ और उनकी पीठों को दाग़ा जाएगा (और कहा जाएगा), "यहीं है जो तुमने अपने लिए संचय किया, तो जो कुछ तुम संचित करते रहे हो, उसका मज़ा चखो!"
\end{hindi}}
\flushright{\begin{Arabic}
\quranayah[9][36]
\end{Arabic}}
\flushleft{\begin{hindi}
निस्संदेह महीनों की संख्या - अल्लाह के अध्यादेश में उस दिन से जब उसने आकाशों और धरती को पैदा किया - अल्लाह की दृष्टि में बारह महीने है। उनमें चार आदर के है, यही सीधा दीन (धर्म) है। अतः तुम उन (महीनों) में अपने ऊपर अत्याचार न करो। और मुशरिकों से तुम सबके सब लड़ो, जिस प्रकार वे सब मिलकर तुमसे लड़ते है। और जान लो कि अल्लाह डर रखनेवालों के साथ है
\end{hindi}}
\flushright{\begin{Arabic}
\quranayah[9][37]
\end{Arabic}}
\flushleft{\begin{hindi}
(आदर के महीनों का) हटाना तो बस कुफ़्र में एक बृद्धि है, जिससे इनकार करनेवाले गुमराही में पड़ते है। किसी वर्ष वे उसे हलाल (वैध) ठहरा लेते है और किसी वर्ष उसको हराम ठहरा लेते है, ताकि अल्लाह के आदृत (महीनों) की संख्या पूरी कर लें, और इस प्रकार अल्लाह के हराम किए हुए को वैध ठहरा ले। उनके अपने बुरे कर्म उनके लिए सुहाने हो गए है और अल्लाह इनकार करनेवाले लोगों को सीधा मार्ग नहीं दिखाता
\end{hindi}}
\flushright{\begin{Arabic}
\quranayah[9][38]
\end{Arabic}}
\flushleft{\begin{hindi}
ऐ ईमान लानेवालो! तुम्हें क्या हो गया है कि जब तुमसे कहा जाता है, "अल्लाह के मार्ग में निकलो" तो तुम धरती पर ढहे जाते हो? क्या तुम आख़िरत की अपेक्षा सांसारिक जीवन पर राज़ी हो गए? सांसारिक जीवन की सुख-सामग्री तो आख़िरत के हिसाब में है कुछ थोड़ी ही!
\end{hindi}}
\flushright{\begin{Arabic}
\quranayah[9][39]
\end{Arabic}}
\flushleft{\begin{hindi}
यदि तुम निकालोगे तो वह तुम्हें दुखद यातना देगा और वह तुम्हारी जगह दूसरे गिरोह को ले आएगा और तुम उसका कुछ न बिगाड़ सकोगे। और अल्लाह हर चीज़ की सामर्थ्य रखता है
\end{hindi}}
\flushright{\begin{Arabic}
\quranayah[9][40]
\end{Arabic}}
\flushleft{\begin{hindi}
यदि तुम उसकी सहायता न भी करो तो अल्लाह उसकी सहायता उस समय कर चुका है जब इनकार करनेवालों ने उसे इस स्थिति में निकाला कि वह केवल दो में का दूसरा था, जब वे दोनों गुफ़ा में थे। जबकि वह अपने साथी से कह रहा था, "शोकाकुल न हो। अवश्यमेव अल्लाह हमारे साथ है।" फिर अल्लाह ने उसपर अपनी ओर से सकीनत (प्रशान्ति) उतारी और उसकी सहायता ऐसी सेनाओं से की जिन्हें तुम देख न सके और इनकार करनेवालों का बोल नीचा कर दिया, बोल तो अल्लाह ही का ऊँचा रहता है। अल्लाह अत्यन्त प्रभुत्वशील, तत्वदर्शी है
\end{hindi}}
\flushright{\begin{Arabic}
\quranayah[9][41]
\end{Arabic}}
\flushleft{\begin{hindi}
हलके और बोझिल निकल पड़ो और अल्लाह के मार्ग में अपने मालों और अपनी जानों के साथ जिहाद करो! यही तुम्हारे लिए उत्तम है, यदि तुम जानो
\end{hindi}}
\flushright{\begin{Arabic}
\quranayah[9][42]
\end{Arabic}}
\flushleft{\begin{hindi}
यदि निकट (भविष्य में) ही कुछ मिलनेवाला होता और सफ़र भी हलका होता तो वे अवश्य तुम्हारे पीछे चल पड़ते, किन्तु मार्ग की दूरी उन्हें कठिन और बहुत दीर्घ प्रतीत हुई। अब वे अल्लाह की क़समें खाएँगे कि, "यदि हममें इसकी सामर्थ्य होती तो हम अवश्य तुम्हारे साथ निकलते।" वे अपने आपको तबाही में डाल रहे है और अल्लाह भली-भाँति जानता है कि निश्चय ही वे झूठे है
\end{hindi}}
\flushright{\begin{Arabic}
\quranayah[9][43]
\end{Arabic}}
\flushleft{\begin{hindi}
अल्लाह ने तुम्हे क्षमा कर दिया! तुमने उन्हें क्यों अनुमति दे दी, यहाँ तक कि जो लोग सच्चे है वे तुम्हारे सामने प्रकट हो जाते और झूठों को भी तुम जान लेते?
\end{hindi}}
\flushright{\begin{Arabic}
\quranayah[9][44]
\end{Arabic}}
\flushleft{\begin{hindi}
जो लोग अल्लाह और अंतिम दिन पर ईमान रखते है, वे तुमसे कभी यह नहीं चाहेंगे कि उन्हें अपने मालों और अपनी जानों के साथ जिहाद करने से माफ़ रखा जाए। और अल्लाह डर रखनेवालों को भली-भाँति जानता है
\end{hindi}}
\flushright{\begin{Arabic}
\quranayah[9][45]
\end{Arabic}}
\flushleft{\begin{hindi}
तुमसे छुट्टी तो बस वही लोग माँगते है जो अल्लाह और अन्तिम दिन पर ईमान नहीं रखते, और जिनके दिल सन्देह में पड़े है, तो वे अपने सन्देह ही में डाँवाडोल हो रहे है
\end{hindi}}
\flushright{\begin{Arabic}
\quranayah[9][46]
\end{Arabic}}
\flushleft{\begin{hindi}
यदि वे निकलने का इरादा करते तो इसके लिए कुछ सामग्री जुटाते, किन्तु अल्लाह ने उनके उठने को नापसन्द किया तो उसने उन्हें रोक दिया। उनके कह दिया गया, "बैठनेवालों के साथ बैठ रहो।"
\end{hindi}}
\flushright{\begin{Arabic}
\quranayah[9][47]
\end{Arabic}}
\flushleft{\begin{hindi}
यदि वे तुम्हारे साथ निकलते भी तो तुम्हारे अन्दर ख़राबी के सिवा किसी और चीज़ की अभिवृद्धि नहीं करते। और वे तुम्हारे बीच उपद्रव मचाने के लिए दौड़-धूप करते और तुममें उनकी सुननेवाले है। और अल्लाह अत्याचारियों को भली-भाँति जानता है
\end{hindi}}
\flushright{\begin{Arabic}
\quranayah[9][48]
\end{Arabic}}
\flushleft{\begin{hindi}
उन्होंने तो इससे पहले भी उपद्रव मचाना चाहा था और वे तुम्हारे विरुद्ध घटनाओं और मामलों के उलटने-पलटने में लगे रहे, यहाँ तक कि हक़ आ गया और अल्लाह को आदेश प्रकट होकर रहा, यद्यपि उन्हें अप्रिय ही लगता रहा
\end{hindi}}
\flushright{\begin{Arabic}
\quranayah[9][49]
\end{Arabic}}
\flushleft{\begin{hindi}
उनमें कोई है, जो कहता है, "मुझे इजाज़त दे दीजिए, मुझे फ़ितने में न डालिए।" जान लो कि वे फ़ितने में तो पड़ ही चुके है और निश्चय ही जहन्नम भी इनकार करनेवालों को घेर रही है
\end{hindi}}
\flushright{\begin{Arabic}
\quranayah[9][50]
\end{Arabic}}
\flushleft{\begin{hindi}
यदि तुम्हें कोई अच्छी हालत पेश आती है, तो उन्हें बुरा लगता है औऱ यदि तुम पर कोई मुसीबत आ जाती है, तो वे कहते है, "हमने तो अपना काम पहले ही सँभाल लिया था।" और वे ख़ुश होते हुए पलटते है
\end{hindi}}
\flushright{\begin{Arabic}
\quranayah[9][51]
\end{Arabic}}
\flushleft{\begin{hindi}
कह दो, "हमें कुछ भी पेश नहीं आ सकता सिवाय उसके जो अल्लाह ने लिख दिया है। वही हमारा स्वामी है। और ईमानवालों को अल्लाह ही पर भरोसा करना चाहिए।"
\end{hindi}}
\flushright{\begin{Arabic}
\quranayah[9][52]
\end{Arabic}}
\flushleft{\begin{hindi}
कहो, "तुम हमारे लिए दो भलाईयों में से किसी एक भलाई के सिवा किसकी प्रतीक्षा कर सकते है? जबकि हमें तुम्हारे हक़ में इसी की प्रतिक्षा है कि अल्लाह अपनी ओर से तुम्हें कोई यातना देता है या हमारे हाथों दिलाता है। अच्छा तो तुम भी प्रतीक्षा करो, हम भी तुम्हारे साथ प्रतीक्षा कर रहे है।"
\end{hindi}}
\flushright{\begin{Arabic}
\quranayah[9][53]
\end{Arabic}}
\flushleft{\begin{hindi}
कह दो, "तुम चाहे स्वेच्छापूर्वक ख़र्च करो या अनिच्छापूर्वक, तुमसे कुछ भी स्वीकार न किया जाएगा। निस्संदेह तुम अवज्ञाकारी लोग हो।"
\end{hindi}}
\flushright{\begin{Arabic}
\quranayah[9][54]
\end{Arabic}}
\flushleft{\begin{hindi}
उनके ख़र्च के स्वीकृत होने में इसके अतिरिक्त और कोई चीज़ बाधक नहीं कि उन्होंने अल्लाह औऱ उसके रसूल के साथ कुफ़्र किया। नमाज़ को आते है तो बस हारे जी आते है और ख़र्च करते है, तो अनिच्छापूर्वक ही
\end{hindi}}
\flushright{\begin{Arabic}
\quranayah[9][55]
\end{Arabic}}
\flushleft{\begin{hindi}
अतः उनके माल तुम्हें मोहित न करें और न उनकी सन्तान ही। अल्लाह तो बस यह चाहता है कि उनके द्वारा उन्हें सांसारिक जीवन में यातना दे और उनके प्राण इस दशा में निकलें कि वे इनकार करनेवाले ही रहे
\end{hindi}}
\flushright{\begin{Arabic}
\quranayah[9][56]
\end{Arabic}}
\flushleft{\begin{hindi}
वे अल्लाह की क़समें खाते है कि वे तुम्हीं में से है, हालाँकि वे तुममें से नहीं है, बल्कि वे ऐसे लोग है जो त्रस्त रहते है
\end{hindi}}
\flushright{\begin{Arabic}
\quranayah[9][57]
\end{Arabic}}
\flushleft{\begin{hindi}
यदि वे कोई शरण पा लें या कोई गुफा या घुस बैठने की जगह, तो अवश्य ही वे बगटुट उसकी ओर उल्टे भाग जाएँ
\end{hindi}}
\flushright{\begin{Arabic}
\quranayah[9][58]
\end{Arabic}}
\flushleft{\begin{hindi}
और उनमें से कुछ लोग सदक़ो के विषय में तुम पर चोटे करते है। किन्तु यदि उन्हें उसमें से दे दिया जाए तो प्रसन्न हो जाएँ और यदि उन्हें उसमें से न दिया गया तो क्या देखोगे कि वे क्रोधित होने लगते है
\end{hindi}}
\flushright{\begin{Arabic}
\quranayah[9][59]
\end{Arabic}}
\flushleft{\begin{hindi}
यदि अल्लाह और उसके रसूल ने जो कुछ उन्हें दिया था, उसपर वे राज़ी रहते औऱ कहते कि "हमारे लिए अल्लाह काफ़ी है। अल्लाह हमें जल्द ही अपने अनुग्रह से देगा और उसका रसूल भी। हम तो अल्लाह ही की ओऱ उन्मुख है।" (तो यह उनके लिए अच्छा होता)
\end{hindi}}
\flushright{\begin{Arabic}
\quranayah[9][60]
\end{Arabic}}
\flushleft{\begin{hindi}
सदक़े तो बस ग़रीबों, मुहताजों और उन लोगों के लिए है, जो काम पर नियुक्त हों और उनके लिए जिनके दिलों को आकृष्ट करना औऱ परचाना अभीष्ट हो और गर्दनों को छुड़ाने और क़र्ज़दारों और तावान भरनेवालों की सहायता करने में, अल्लाह के मार्ग में, मुसाफ़िरों की सहायता करने में लगाने के लिए है। यह अल्लाह की ओर से ठहराया हुआ हुक्म है। अल्लाह सब कुछ जाननेवाला, अत्यन्त तत्वदर्शी है
\end{hindi}}
\flushright{\begin{Arabic}
\quranayah[9][61]
\end{Arabic}}
\flushleft{\begin{hindi}
और उनमें कुछ लोग ऐसे हैं, जो नबी को दुख देते है और कहते है, "वह तो निरा कान है!" कह दो, "वह सर्वथा कान तुम्हारी भलाई के लिए है। वह अल्लाह पर ईमान रखता है और ईमानवालों पर भी विश्वास करता है। और उन लोगों के लिए सर्वथा दयालुता है जो तुममें से ईमान लाए है। रहे वे लोग जो अल्लाह के रसूल को दुख देते है, उनके लिए दुखद यातना है।"
\end{hindi}}
\flushright{\begin{Arabic}
\quranayah[9][62]
\end{Arabic}}
\flushleft{\begin{hindi}
वे तुम लोगों के सामने अल्लाह की क़समें खाते है, ताकि तुम्हें राज़ी कर लें, हालाँकि यदि वे मोमिन है तो अल्लाह और उसका रसूल इसके ज़्यादा हक़दार है कि उनको राज़ी करें
\end{hindi}}
\flushright{\begin{Arabic}
\quranayah[9][63]
\end{Arabic}}
\flushleft{\begin{hindi}
क्या उन्हें मालूम नहीं कि जो अल्लाह औऱ उसके रसूल का विरोध करता है, उसके लिए जहन्नम की आग है जिसमें वह सदैव रहेगा। यह बहुत बड़ी रुसवाई है
\end{hindi}}
\flushright{\begin{Arabic}
\quranayah[9][64]
\end{Arabic}}
\flushleft{\begin{hindi}
मुनाफ़िक़ (कपटाचारी) डर रहे है कि कहीं उनके बारे में कोई ऐसी सूरा न अवतरित हो जाए जो वह सब कुछ उनपर खोल दे, जो उनके दिलों में है। कह दो, "मज़ाक़ उड़ा लो, अल्लाह तो उसे प्रकट करके रहेगा, जिसका तुम्हें डर है।"
\end{hindi}}
\flushright{\begin{Arabic}
\quranayah[9][65]
\end{Arabic}}
\flushleft{\begin{hindi}
और यदि उनसे पूछो तो कह देंगे, "हम तो केवल बातें और हँसी-खेल कर रहे थे।" कहो, "क्या अल्लाह, उसकी आयतों और उसके रसूल के साथ हँसी-मज़ाक़ करते थे?
\end{hindi}}
\flushright{\begin{Arabic}
\quranayah[9][66]
\end{Arabic}}
\flushleft{\begin{hindi}
"बहाने न बनाओ, तुमने अपने ईमान के पश्चात इनकार किया। यदि हम तुम्हारे कुछ लोगों को क्षमा भी कर दें तो भी कुछ लोगों को यातना देकर ही रहेंगे, क्योंकि वे अपराधी हैं।"
\end{hindi}}
\flushright{\begin{Arabic}
\quranayah[9][67]
\end{Arabic}}
\flushleft{\begin{hindi}
मुनाफ़िक़ पुरुष और मुनाफ़िक़ स्त्रियाँ सब एक ही थैली के चट्टे-बट्टे हैं। वे बुराई का हुक्म देते है और भलाई से रोकते है और हाथों को बन्द किए रहते है। वे अल्लाह को भूल बैठे तो उसने भी उन्हें भुला दिया। निश्चय ही मुनाफ़िक़ अवज्ञाकारी हैं
\end{hindi}}
\flushright{\begin{Arabic}
\quranayah[9][68]
\end{Arabic}}
\flushleft{\begin{hindi}
अल्लाह ने मुनाफ़िक़ पुरुषों और मुनाफ़िक़ स्त्रियों और इनकार करनेवालों से जहन्नम की आग का वादा किया है, जिसमें वे सदैव ही रहेंगे। वही उनके लिए काफ़ी है और अल्लाह ने उनपर लानत की, और उनके लिए स्थाई यातना है
\end{hindi}}
\flushright{\begin{Arabic}
\quranayah[9][69]
\end{Arabic}}
\flushleft{\begin{hindi}
उन लोगों की तरह, जो तुमसे पहले गुज़र चुके हैं, वे शक्ति में तुमसे बढ़-बढ़कर थे और माल और औलाद में भी बढ़े हुए थे। फिर उन्होंने अपने हिस्से का मज़ा उठाना चाहा और तुमने भी अपने हिस्से का मज़ा उठाना चाहा, जिस प्रकार कि तुमसे पहले के लोगों ने अपने हिस्से का मज़ा उठाना चाहा, और जिस वाद-विवाद में तुम पड़े थे तुम भी वाद-विवाद में पड़ गए। ये वही लोग है जिनका किया-धरा दुनिया और आख़िरत में अकारथ गया, और वही घाटे में है
\end{hindi}}
\flushright{\begin{Arabic}
\quranayah[9][70]
\end{Arabic}}
\flushleft{\begin{hindi}
क्या उन्हें उन लोगों का वृतान्त नहीं पहुँचा जो उनसे पहले गुज़रे - नूह के लोगो का, आद और समूद का, और इबराहीम की क़ौम का और मदयनवालों का और उन बस्तियों का जिन्हें उलट दिया गया? उसके रसूल उनके पास खुली निशानियाँ लेकर आए थे, फिर अल्लाह ऐसा न था कि वह उनपर अत्याचार करता, किन्तु वे स्वयं अपने-आप पर अत्याचार कर रहे थे
\end{hindi}}
\flushright{\begin{Arabic}
\quranayah[9][71]
\end{Arabic}}
\flushleft{\begin{hindi}
रहे मोमिन मर्द औऱ मोमिन औरतें, वे सब परस्पर एक-दूसरे के मित्र है। भलाई का हुक्म देते है और बुराई से रोकते है। नमाज़ क़ायम करते हैं, ज़कात देते है और अल्लाह और उसके रसूल का आज्ञापालन करते हैं। ये वे लोग है, जिनकर शीघ्र ही अल्लाह दया करेगा। निस्सन्देह प्रभुत्वशाली, तत्वदर्शी है
\end{hindi}}
\flushright{\begin{Arabic}
\quranayah[9][72]
\end{Arabic}}
\flushleft{\begin{hindi}
मोमिन मर्दों और मोमिन औरतों से अल्लाह ने ऐसे बाग़ों का वादा किया है जिनके नीचे नहरें बह रही होंगी, जिनमें वे सदैव रहेंगे और सदाबहार बाग़ों में पवित्र निवास गृहों का (भी वादा है) और, अल्लाह की प्रसन्नता और रज़ामन्दी का; जो सबसे बढ़कर है। यही सबसे बड़ी सफलता है
\end{hindi}}
\flushright{\begin{Arabic}
\quranayah[9][73]
\end{Arabic}}
\flushleft{\begin{hindi}
ऐ नबी! इनकार करनेवालों और मुनाफ़िक़ों से जिहाद करो और उनके साथ सख़्ती से पेश आओ। अन्ततः उनका ठिकाना जहन्नम है और वह जा पहुँचने की बहुत बुरी जगह है!
\end{hindi}}
\flushright{\begin{Arabic}
\quranayah[9][74]
\end{Arabic}}
\flushleft{\begin{hindi}
वे अल्लाह की क़समें खाते है कि उन्होंने नहीं कहा, हालाँकि उन्होंने अवश्य ही कुफ़्र की बात कही है और अपने इस्लाम स्वीकार करने के पश्चात इनकार किया, और वह चाहा जो वे न पा सके। उनके प्रतिशोध का कारण तो यह है कि अल्लाह और उसके रसूल ने अपने अनुग्रह से उन्हें समृद्ध कर दिया। अब यदि वे तौबा कर लें तो उन्हीं के लिए अच्छा है और यदि उन्होंने मुँह मोड़ा तो अल्लाह उन्हें दुनिया और आख़िरत में दुखद यातना देगा और धरती में उनका न कोई मित्र होगा और न सहायक
\end{hindi}}
\flushright{\begin{Arabic}
\quranayah[9][75]
\end{Arabic}}
\flushleft{\begin{hindi}
और उनमें से कुछ लोग ऐसे भी है जिन्होने अल्लाह को वचन दिया था कि "यदि उसने हमें अपने अनुग्रह से दिया तो हम अवश्य दान करेंगे और नेक होकर रहेंगे।"
\end{hindi}}
\flushright{\begin{Arabic}
\quranayah[9][76]
\end{Arabic}}
\flushleft{\begin{hindi}
किन्तु जब अल्लाह ने उन्हें अपने अनुग्रह से दिया तो वे उसमें कंजूसी करने लगे और पहलू बचाकर फिर गए
\end{hindi}}
\flushright{\begin{Arabic}
\quranayah[9][77]
\end{Arabic}}
\flushleft{\begin{hindi}
फिर परिणाम यह हुआ कि उसने उनके दिलों में उस दिन तक के लिए कपटाचार डाल दिया, जब वे उससे मिलेंगे, इसलिए कि उन्होंने अल्लाह से जो प्रतिज्ञा की थी उसे भंग कर दिया और इसलिए भी कि वे झूठ बोलते रहे
\end{hindi}}
\flushright{\begin{Arabic}
\quranayah[9][78]
\end{Arabic}}
\flushleft{\begin{hindi}
क्या उन्हें खबर नहीं कि अल्लाह उनका भेद और उनकी कानाफुसियों को अच्छी तरह जानता है और यह कि अल्लाह परोक्ष की सारी बातों को भली-भाँति जानता है
\end{hindi}}
\flushright{\begin{Arabic}
\quranayah[9][79]
\end{Arabic}}
\flushleft{\begin{hindi}
जो लोग स्वेच्छापूर्वक देनेवाले मोमिनों पर उनके सदक़ो (दान) के विषय में चोटें करते है और उन लोगों का उपहास करते है, जिनके पास इसके सिवा कुछ नहीं जो वे मशक़्क़त उठाकर देते है, उन (उपहास करनेवालों) का उपहास अल्लाह ने किया और उनके लिए दुखद यातना है
\end{hindi}}
\flushright{\begin{Arabic}
\quranayah[9][80]
\end{Arabic}}
\flushleft{\begin{hindi}
तुम उनके लिए क्षमा की प्रार्थना करो या उनके लिए क्षमा की प्रार्थना न करो। यदि तुम उनके लिए सत्तर बार भी क्षमा की प्रार्थना करोगे, तो भी अल्लाह उन्हें क्षमा नहीं करेगा, यह इसलिए कि उन्होंने अल्लाह और उसके रसूल के साथ कुफ़्र किया और अल्लाह अवज्ञाकारियों को सीधा मार्ग नहीं दिखाता
\end{hindi}}
\flushright{\begin{Arabic}
\quranayah[9][81]
\end{Arabic}}
\flushleft{\begin{hindi}
पीछे रह जानेवाले अल्लाह के रसूल के पीछे अपने बैठ रहने पर प्रसन्न हुए। उन्हें यह नापसन्द हुआ कि अल्लाह के मार्ग में अपने मालों और अपनी जानों के साथ जिहाद करें। और उन्होंने कहा, "इस गर्मी में न निकलो।" कह दो, "जहन्नम की आग इससे कहीं अधिक गर्म है," यदि वे समझ पाते (तो ऐसा न कहते)
\end{hindi}}
\flushright{\begin{Arabic}
\quranayah[9][82]
\end{Arabic}}
\flushleft{\begin{hindi}
अब चाहिए कि जो कुछ वे कमाते रहे है, उसके बदले में हँसे कम और रोएँ अधिक
\end{hindi}}
\flushright{\begin{Arabic}
\quranayah[9][83]
\end{Arabic}}
\flushleft{\begin{hindi}
अव यदि अल्लाह तुम्हें उनके किसी गिरोह की ओर रुजू कर दे और भविष्य में वे तुमसे साथ निकलने की अनुमति चाहें तो कह देना, "तुम मेरे साथ कभी भी नहीं निकल सकते और न मेरे साथ होकर किसी शत्रु से लड़ सकते हो। तुम पहली बार बैठ रहने पर ही राज़ी हुए, तो अब पीछे रहनेवालों के साथ बैठे रहो।"
\end{hindi}}
\flushright{\begin{Arabic}
\quranayah[9][84]
\end{Arabic}}
\flushleft{\begin{hindi}
औऱ उनमें से जिस किसी व्यक्ति की मृत्यु हो उसकी जनाज़े की नमाज़ कभी न पढ़ना और न कभी उसकी क़ब्र पर खड़े होना। उन्होंने तो अल्लाह और उसके रसूल के साथ कुफ़्र किया और मरे इस दशा में कि अवज्ञाकारी थे
\end{hindi}}
\flushright{\begin{Arabic}
\quranayah[9][85]
\end{Arabic}}
\flushleft{\begin{hindi}
और उनके माल और उनकी औलाद तुम्हें मोहित न करें। अल्लाह तो बस यह चाहता है कि उनके द्वारा उन्हें संसार में यातना दे और उनके प्राण इस दशा में निकलें कि वे काफ़िर हों
\end{hindi}}
\flushright{\begin{Arabic}
\quranayah[9][86]
\end{Arabic}}
\flushleft{\begin{hindi}
और जब कोई सूरा उतरती है कि "अल्लाह पर ईमान लाओ और उसके रसूल के साथ होकर जिहाद करो।" तो उनके सामर्थ्यवान लोग तुमसे छुट्टी माँगने लगते है और कहते है कि "हमें छोड़ दो कि हम बैठनेवालों के साथ रह जाएँ।"
\end{hindi}}
\flushright{\begin{Arabic}
\quranayah[9][87]
\end{Arabic}}
\flushleft{\begin{hindi}
वे इसी पर राज़ी हुए कि पीछे रह जानेवाली स्त्रियों के साथ रह जाएँ और उनके दिलों पर तो मुहर लग गई है, अतः वे समझते नहीं
\end{hindi}}
\flushright{\begin{Arabic}
\quranayah[9][88]
\end{Arabic}}
\flushleft{\begin{hindi}
किन्तु, रसूल और उसके ईमानवाले साथियों ने अपने मालों और अपनी जानों के साथ जिहाद किया, और वही लोग है जिनके लिए भलाइयाँ है और वही लोग है जो सफल है
\end{hindi}}
\flushright{\begin{Arabic}
\quranayah[9][89]
\end{Arabic}}
\flushleft{\begin{hindi}
अल्लाह ने उनके लिए ऐसे बाग़ तैयार कर रखे हैं, जिनके नीचे नहरें बह रह हैं, वे उनमें सदैव रहेंगे। यही बड़ी सफलता है
\end{hindi}}
\flushright{\begin{Arabic}
\quranayah[9][90]
\end{Arabic}}
\flushleft{\begin{hindi}
बहाने करनेवाले बद्दूल भी आए कि उन्हें (बैठे रहने की) छुट्टी मिल जाए। और जो अल्लाह और उसके रसूल से झूठ बोले वे भी बैठे रहे। उनमें से जिन्होंने इनकार किया उन्हें शीघ्र ही एक दुखद यातना पहुँचकर रहेगी
\end{hindi}}
\flushright{\begin{Arabic}
\quranayah[9][91]
\end{Arabic}}
\flushleft{\begin{hindi}
न तो कमज़ोरों के लिए कोई दोष की बात है और न बीमारों के लिए और न उन लोगों के लिए जिन्हें ख़र्च करने के लिए कुछ प्राप्त नहीं, जबकि वे अल्लाह और उसके रसूल के प्रति निष्ठावान हों। उत्तमकारों पर इलज़ाम की कोई गुंजाइश नहीं है। अल्लाह तो बड़ा क्षमाशील, अत्यन्त दयावान है
\end{hindi}}
\flushright{\begin{Arabic}
\quranayah[9][92]
\end{Arabic}}
\flushleft{\begin{hindi}
और न उन लोगों पर आक्षेप करने की कोई गुंजाइश है जिनका हाल यह है कि जब वे तुम्हारे पास आते है, कि तुम उनके लिए सवारी का प्रबन्ध कर दो, तुम कहते हो, "मुझे ऐसा कुछ प्राप्त नहीं जिसपर तुम्हें सवार करूँ।" वे इस दशा में लौटते है कि इस ग़म में उनकी आँखे आँसू बहा रही होती है कि वे अपने पास ख़र्च करने को कुछ नहीं पाते
\end{hindi}}
\flushright{\begin{Arabic}
\quranayah[9][93]
\end{Arabic}}
\flushleft{\begin{hindi}
इल्ज़ाम तो बस उनपर है जो धनवान होते हुए तुमसे छुट्टी माँगते है। वे इसपर राज़ी हुए कि पीछे डाले गए लोगों के साथ रह जाएँ। अल्लाह ने तो उनके दिलों पर मुहर लगा दी है, इसलिए वे जानते नहीं
\end{hindi}}
\flushright{\begin{Arabic}
\quranayah[9][94]
\end{Arabic}}
\flushleft{\begin{hindi}
जब तुम पलटकर उनके पास पहुँचोगे तो वे तुम्हारे सामने बहाने करेंगे। तुम कह देना, "बहाने न बनाओ। हम तु्म्हारी बात कदापि नहीं मानेंगे। हमें अल्लाह ने तुम्हारे वृत्तांत बता दिए है। अभी अल्लाह और उसका रसूल तुम्हारे काम को देखेगा, फिर तुम उसकी ओर लौटोगे, जो छिपे और खुले का ज्ञान रखता है। फिर जो कुछ तुम करते रहे हो वह तुम्हे बता देगा।"
\end{hindi}}
\flushright{\begin{Arabic}
\quranayah[9][95]
\end{Arabic}}
\flushleft{\begin{hindi}
जब तुम पलटकर उनके पास जाओगे तो वे तुम्हारे सामने अल्लाह की क़समें खाएँगे, ताकि तुम उन्हें उनकी हालत पर छोड़ दो। तो तुम उन्हें छोड़ ही दो। निश्चय ही वे गन्दगी है और उनका ठिकाना जहन्नम है। जो कुछ वे कमाते रहे है, यह उसी का बदला है
\end{hindi}}
\flushright{\begin{Arabic}
\quranayah[9][96]
\end{Arabic}}
\flushleft{\begin{hindi}
वे तुम्हारे सामने क़समें खाएँगे ताकि तुम उनसे राज़ी हो जाओ, किन्तु यदि तुम उनसे राज़ी भी हो गए तो अल्लाह ऐसे लोगो से कदापि राज़ी न होगा, जो अवज्ञाकारी है
\end{hindi}}
\flushright{\begin{Arabic}
\quranayah[9][97]
\end{Arabic}}
\flushleft{\begin{hindi}
वे बद्दूग इनकार और कपटाचार में बहुत-ही बढ़े हुए है। और इसी के ज़्यादा योग्य है कि उनकी सीमाओं से अनभिज्ञ रहें, जिसे अल्लाह ने अपने रसूल पर अवतरित किया है। अल्लाह सर्वज्ञ, तत्वदर्शी है
\end{hindi}}
\flushright{\begin{Arabic}
\quranayah[9][98]
\end{Arabic}}
\flushleft{\begin{hindi}
और कुछ बद्दूज ऐसे है कि वे जो कुछ ख़र्च करते है, उसे तावान समझते है और तुम्हारे हक़ मं बुरी गर्दिशों (बुरे दिन) की प्रतीक्षा में हैं, बुरी गर्दिश में तो वही है। अल्लाह सब कुछ सुनता, जानता है
\end{hindi}}
\flushright{\begin{Arabic}
\quranayah[9][99]
\end{Arabic}}
\flushleft{\begin{hindi}
और बद्दु,ओं में ऐसे भी लोग है जो अल्लाह और अन्तिम दिन को मानते है और जो कुछ ख़र्च करते है, उसे अल्लाह के यहाँ निकटताओं का और रसूल की दुआओं को प्राप्त करने का साधन बनाते है। हाँ! निस्संदेह वह उनके हक़ में निकटता ही है। अल्लाह उन्हें शीघ्र ही अपनी दयालुता में दाख़िल करेगा। निश्चय ही अल्लाह अत्यन्त क्षमाशील, दयावान है
\end{hindi}}
\flushright{\begin{Arabic}
\quranayah[9][100]
\end{Arabic}}
\flushleft{\begin{hindi}
सबसे पहले आगे बढ़नेवाले मुहाजिर और अनसार और जिन्होंने भली प्रकार उनका अनुसरण किया, अल्लाह उनसे राज़ी हुआ और वे उससे राज़ी हुए। और उसने उनके लिए ऐसे बाग़ तैयार कर रखे है, जिनके नीचे नहरें बह रही है, वे उनमें सदैव रहेंगे। यही बड़ी सफलता है
\end{hindi}}
\flushright{\begin{Arabic}
\quranayah[9][101]
\end{Arabic}}
\flushleft{\begin{hindi}
और तुम्हारे आस-पास के बद्दुनओं में और मदीनावालों में कुछ ऐसे कपटाचारी है जो कपट-नीति पर जमें हुए है। उनको तुम नहीं जानते, हम उन्हें भली-भाँति जानते है। शीघ्र ही हम उन्हें दो बार यातना देंगे। फिर वे एक बड़ी यातना की ओर लौटाए जाएँगे
\end{hindi}}
\flushright{\begin{Arabic}
\quranayah[9][102]
\end{Arabic}}
\flushleft{\begin{hindi}
और दूसरे कुछ लोग है जिन्होंने अपने गुनाहों का इक़रार किया। उन्होंने मिले-जुले कर्म किए, कुछ अच्छे और कुछ बुरे। आशा है कि अल्लाह की कृपा-स्पष्ट उनपर हो। निस्संदेह अल्लाह अत्यन्त क्षमाशील, दयावान है
\end{hindi}}
\flushright{\begin{Arabic}
\quranayah[9][103]
\end{Arabic}}
\flushleft{\begin{hindi}
तुम उनके माल में से दान लेकर उन्हें शुद्ध करो और उनके द्वारा उन (की आत्मा) को विकसित करो और उनके लिए दुआ करो। निस्संदेह तुम्हारी दुआ उनके लिए सर्वथा परितोष है। अल्लाह सब कुछ सुनता, जानता है
\end{hindi}}
\flushright{\begin{Arabic}
\quranayah[9][104]
\end{Arabic}}
\flushleft{\begin{hindi}
क्या वे जानते नहीं कि अल्लाह ही अपने बन्दों की तौबा क़बूल करता है और सदक़े लेता है और यह कि अल्लाह ही तौबा क़बूल करनेवाला, अत्यन्त दयावान है
\end{hindi}}
\flushright{\begin{Arabic}
\quranayah[9][105]
\end{Arabic}}
\flushleft{\begin{hindi}
कह दो, "कर्म किए जाओ। अभी अल्लाह और उसका रसूल और ईमानवाले तुम्हारे कर्म को देखेंगे। फिर तुम उसकी ओर पलटोगे, जो छिपे और खुले को जानता है। फिर जो कुछ तम करते रहे हो, वह सब तुम्हें बता देगा।"
\end{hindi}}
\flushright{\begin{Arabic}
\quranayah[9][106]
\end{Arabic}}
\flushleft{\begin{hindi}
और कुछ दूसरे लोग भी है जिनका मामला अल्लाह का हुक्म आने तक स्थगित है, चाहे वह उन्हें यातना दे या उनकी तौबा क़बूल करे। अल्लाह सर्वज्ञ, तत्वदर्शी है
\end{hindi}}
\flushright{\begin{Arabic}
\quranayah[9][107]
\end{Arabic}}
\flushleft{\begin{hindi}
और कुछ ऐसे लोग भी हैं , जिन्होंने मस्जिद बनाई इसलिए कि नुक़सान पहुँचाएँ और कुफ़्र करें और इसलिए कि ईमानवालों के बीच फूट डाले और उस व्यक्ति के घात लगाने का ठिकाना बनाएँ, जो इससे पहले अल्लाह और उसके रसूल से लड़ चुका है। वे निश्चय ही क़समें खाएँगे कि "हमने तो बस अच्छा ही चाहा था।" किन्तु अल्लाह गवाही देता है कि वे बिलकुल झूठे है
\end{hindi}}
\flushright{\begin{Arabic}
\quranayah[9][108]
\end{Arabic}}
\flushleft{\begin{hindi}
तुम कभी भी उसमें खड़े न होना। वह मस्जिद जिसकी आधारशिला पहले दिन ही से ईशपरायणता पर रखी गई है, वह इसकी ज़्यादा हक़दार है कि तुम उसमें खड़े हो। उसमें ऐसे लोग पाए जाते हैं, जो अच्छी तरह स्वच्छ रहना पसन्द करते है, और अल्लाह भी पाक-साफ़ रहनेवालों को पसन्द करता है
\end{hindi}}
\flushright{\begin{Arabic}
\quranayah[9][109]
\end{Arabic}}
\flushleft{\begin{hindi}
फिर क्या वह अच्छा है जिसने अपने भवन की आधारशिला अल्लाह के भय और उसकी ख़ुशी पर रखी है या वह, जिसने अपने भवन की आधारशिला किसी खाई के खोखले कगार पर रखी, जो गिरने को है। फिर वह उसे लेकर जहन्नम की आग में जा गिरा? अल्लाह तो अत्याचारी लोगों को सीधा मार्ग नहीं दिखाता
\end{hindi}}
\flushright{\begin{Arabic}
\quranayah[9][110]
\end{Arabic}}
\flushleft{\begin{hindi}
उनका यह भवन जो उन्होंने बनाया है, सदैव उनके दिलों में खटक बनकर रहेगा। हाँ, यदि उनके दिल ही टुकड़े-टुकड़े हो जाएँ तो दूसरी बात है। अल्लाह तो सब कुछ जाननेवाला, अत्यन्त तत्वदर्शी है
\end{hindi}}
\flushright{\begin{Arabic}
\quranayah[9][111]
\end{Arabic}}
\flushleft{\begin{hindi}
निस्संदेह अल्लाह ने ईमानवालों से उनके प्राण और उनके माल इसके बदले में खरीद लिए है कि उनके लिए जन्नत है। वे अल्लाह के मार्ग में लड़ते है, तो वे मारते भी है और मारे भी जाते है। यह उनके ज़िम्मे तौरात, इनजील और क़ुरआन में (किया गया) एक पक्का वादा है। और अल्लाह से बढ़कर अपने वादे को पूरा करनेवाला हो भी कौन सकता है? अतः अपने उस सौदे पर खु़शियाँ मनाओ, जो सौदा तुमने उससे किया है। और यही सबसे बड़ी सफलता है
\end{hindi}}
\flushright{\begin{Arabic}
\quranayah[9][112]
\end{Arabic}}
\flushleft{\begin{hindi}
वे ऐसे हैं, जो तौबा करते हैं, बन्दगी करते है, स्तुति करते हैं, (अल्लाह के मार्ग में) भ्रमण करते हैं, (अल्लाह के आगे) झुकते है, सजदा करते है, भलाई का हुक्म देते है और बुराई से रोकते हैं और अल्लाह की निर्धारित सीमाओं की रक्षा करते हैं -और इन ईमानवालों को शुभ-सूचना दे दो
\end{hindi}}
\flushright{\begin{Arabic}
\quranayah[9][113]
\end{Arabic}}
\flushleft{\begin{hindi}
नबी और ईमान लानेवालों के लिए उचित नहीं कि वे बहुदेववादियों के लिए क्षमा की प्रार्थना करें, यद्यपि वे उसके नातेदार ही क्यों न हो, जबकि उनपर यह बात खुल चुकी है कि वे भड़कती आगवाले हैं
\end{hindi}}
\flushright{\begin{Arabic}
\quranayah[9][114]
\end{Arabic}}
\flushleft{\begin{hindi}
इबराहीम ने अपने बाप के लिए जो क्षमा की प्रार्थना की थी, वह तो केवल एक वादे के कारण की थी, जो वादा वह उससे कर चुका था। फिर जब उसपर यह बात खुल गई कि वह अल्लाह का शत्रु है तो वह उससे विरक्त हो गया। वास्तव में, इबराहीम बड़ा ही कोमल हृदय, अत्यन्त सहनशील था
\end{hindi}}
\flushright{\begin{Arabic}
\quranayah[9][115]
\end{Arabic}}
\flushleft{\begin{hindi}
अल्लाह ऐसा नहीं कि लोगों को पथभ्रष्ट ठहराए, जबकि वह उनको राह पर ला चुका हो, जब तक कि उन्हें साफ़-साफ़ वे बातें बता न दे, जिनसे उन्हें बचना है। निस्संदेह अल्लाह हर चीज़ को भली-भाँति जानता है
\end{hindi}}
\flushright{\begin{Arabic}
\quranayah[9][116]
\end{Arabic}}
\flushleft{\begin{hindi}
आकाशों और धरती का राज्य अल्लाह ही का है, वही जिलाता है और मारता है। अल्लाह से हटकर न तुम्हारा कोई मित्र है और न सहायक
\end{hindi}}
\flushright{\begin{Arabic}
\quranayah[9][117]
\end{Arabic}}
\flushleft{\begin{hindi}
अल्लाह नबी पर मेहरबान हो गया और मुहाजिरों और अनसार पर भी, जिन्होंने तंगी की घड़ी में उसका साथ दिया, इसके पश्चात कि उनमें से एक गिरोह के दिल कुटिलता की ओर झुक गए थे। फिर उसने उनपर दया-दृष्टि दर्शाई। निस्संदेह, वह उनके लिए अत्यन्त करुणामय, दयावान है
\end{hindi}}
\flushright{\begin{Arabic}
\quranayah[9][118]
\end{Arabic}}
\flushleft{\begin{hindi}
और उन तीनों पर भी जो पीछे छोड़ दिए गए थे, यहाँ तक कि जब धरती विशाल होते हुए भी उनपर तंग हो गई और उनके प्राण उनपर दुभर हो गए और उन्होंने समझा कि अल्लाह से बचने के लिए कोई शरण नहीं मिल सकती है तो उसी के यहाँ। फिर उसने उनपर कृपा-दृष्टि की ताकि वे पलट आएँ। निस्संदेह अल्लाह ही तौबा क़बूल करनेवाला, अत्यन्त दयावान है
\end{hindi}}
\flushright{\begin{Arabic}
\quranayah[9][119]
\end{Arabic}}
\flushleft{\begin{hindi}
ऐ ईमान लानेवालो! अल्लाह का डर रखों और सच्चे लोगों के साथ हो जाओ
\end{hindi}}
\flushright{\begin{Arabic}
\quranayah[9][120]
\end{Arabic}}
\flushleft{\begin{hindi}
मदीनावालों और उसके आसपास के बद्दूहओं को ऐसा नहीं चाहिए था कि अल्लाह के रसूल को छोड़कर पीछे रह जाएँ और न यह कि उसकी जान के मुक़ाबले में उन्हें अपनी जान अधिक प्रिय हो, क्योंकि वह अल्लाह के मार्ग में प्यास या थकान या भूख की कोई भी तकलीफ़ उठाएँ या किसी ऐसी जगह क़दम रखें, जिससे काफ़िरों का क्रोध भड़के या जो चरका भी वे शत्रु को लगाएँ, उसपर उनके हक में अनिवार्यतः एक सुकर्म लिख लिया जाता है। निस्संदेह अल्लाह उत्तमकार का कर्मफल अकारथ नहीं जाने देता
\end{hindi}}
\flushright{\begin{Arabic}
\quranayah[9][121]
\end{Arabic}}
\flushleft{\begin{hindi}
और वे थो़ड़ा या ज़्यादा जो कुछ भी ख़र्च करें या (अल्लाह के मार्ग में) कोई घाटी पार करें, उनके हक़ में अनिवार्यतः लिख लिया जाता है, ताकि अल्लाह उन्हें उनके अच्छे कर्मों का बदला प्रदान करे
\end{hindi}}
\flushright{\begin{Arabic}
\quranayah[9][122]
\end{Arabic}}
\flushleft{\begin{hindi}
यह तो नहीं कि ईमानवाले सब के सब निकल खड़े हों, फिर ऐसा क्यों नहीं हुआ कि उनके हर गिरोह में से कुछ लोग निकलते, ताकि वे धर्म में समझ प्राप्ति करते और ताकि वे अपने लोगों को सचेत करते, जब वे उनकी ओर लौटते, ताकि वे (बुरे कर्मों से) बचते?
\end{hindi}}
\flushright{\begin{Arabic}
\quranayah[9][123]
\end{Arabic}}
\flushleft{\begin{hindi}
ऐ ईमान लानेवालो! उन इनकार करनेवालों से लड़ो जो तुम्हारे निकट है और चाहिए कि वे तुममें सख़्ती पाएँ, और जान रखो कि अल्लाह डर रखनेवालों के साथ है
\end{hindi}}
\flushright{\begin{Arabic}
\quranayah[9][124]
\end{Arabic}}
\flushleft{\begin{hindi}
जब भी कोई सूरा अवतरित की गई, तो उनमें से कुछ लोग कहते है, "इसने तुममें से किसके ईमान को बढ़ाया?" हाँ, जो लोग ईमान लाए है इसने उनके ईमान को बढ़ाया है। और वे आनन्द मना रहे है
\end{hindi}}
\flushright{\begin{Arabic}
\quranayah[9][125]
\end{Arabic}}
\flushleft{\begin{hindi}
रहे वे लोग जिनके दिलों में रोग है, उनकी गन्दगी में अभिवृद्धि करते हुए उसने उन्हें उनकी अपनी गन्दगी में और आगे बढ़ा दिया। और वे मरे तो इनकार की दशा ही में
\end{hindi}}
\flushright{\begin{Arabic}
\quranayah[9][126]
\end{Arabic}}
\flushleft{\begin{hindi}
क्या वे देखते नहीं कि प्रत्येक वर्ष वॆ एक या दो बार आज़माईश में डाले जाते है ? फिर भी न तो वे तौबा करते हैं और न चेतते।
\end{hindi}}
\flushright{\begin{Arabic}
\quranayah[9][127]
\end{Arabic}}
\flushleft{\begin{hindi}
और जब कोई सूरा अवतरित होती है, तो वे परस्पर एक-दूसरे को देखने लगते है कि "तुम्हें कोई देख तो नहीं रहा है।" फिर पलट जाते है। अल्लाह ने उनके दिल फेर दिए, क्योंकि वे ऐसे लोग है जो समझते नहीं है
\end{hindi}}
\flushright{\begin{Arabic}
\quranayah[9][128]
\end{Arabic}}
\flushleft{\begin{hindi}
तुम्हारे पास तुम्हीं में से एक रसूल आ गया है। तुम्हारा मुश्किल में पड़ना उसके लिए असह्य है। वह तुम्हारे लिए लालयित है। वह मोमिनों के प्रति अत्यन्त करुणामय, दयावान है
\end{hindi}}
\flushright{\begin{Arabic}
\quranayah[9][129]
\end{Arabic}}
\flushleft{\begin{hindi}
अब यदि वे मुँह मोड़े तो कह दो, "मेरे लिए अल्लाह काफ़ी है, उसके अतिरिक्त कोई पूज्य-प्रभु नहीं! उसी पर मैंने भऱोसा किया और वही बड़े सिंहासन का प्रभु है।"
\end{hindi}}
\chapter{Yunus (Jonah)}
\begin{Arabic}
\Huge{\centerline{\basmalah}}\end{Arabic}
\flushright{\begin{Arabic}
\quranayah[10][1]
\end{Arabic}}
\flushleft{\begin{hindi}
अलिफ़॰ लाम॰ रा॰। ये तत्वदर्शितायुक्त किताब की आयतें हैं
\end{hindi}}
\flushright{\begin{Arabic}
\quranayah[10][2]
\end{Arabic}}
\flushleft{\begin{hindi}
क्या लोगों को इस बात पर आश्चर्य हो रहा है कि हमने उन्ही में से एक आदमी की ओर प्रकाशना की कि लोगों को सचेत कर दो और जो लोग मान लें, उनको शुभ समाचार दे दो कि उनके लिए रब के पास शाश्वत सच्चा उन्नत स्थान है? इनकार करनेवाले कहने लगे, "निस्संदेह यह एक खुला जादूगर है।"
\end{hindi}}
\flushright{\begin{Arabic}
\quranayah[10][3]
\end{Arabic}}
\flushleft{\begin{hindi}
निस्संदेह तुम्हारा रब वही अल्लाह है, जिसने आकाशों और धरती को छः दिनों में पैदा किया, फिर सिंहासन पर विराजमान होकर व्यवस्था चला रहा है। उसकी अनुज्ञा के बिना कोई सिफ़ारिश करनेवाला भी नहीं है। वह अल्लाह है तुम्हारा रब। अतः उसी की बन्दगी करो। तो क्या तुम ध्यान न दोगे?
\end{hindi}}
\flushright{\begin{Arabic}
\quranayah[10][4]
\end{Arabic}}
\flushleft{\begin{hindi}
उसी की ओर तुम सबको लौटना है। यह अल्लाह का पक्का वादा है। निस्संदेह वही पहली बार पैदा करता है। फिर दोबारा पैदा करेगा, ताकि जो लोग ईमान लाए और उन्होंने अच्छे कर्म किए उन्हें न्यायपूर्वक बदला दे। रहे वे लोग जिन्होंने इनकार किया उनके लिए खौलता पेय और दुखद यातना है, उस इनकार के बदले में जो वे करते रहे
\end{hindi}}
\flushright{\begin{Arabic}
\quranayah[10][5]
\end{Arabic}}
\flushleft{\begin{hindi}
वही है जिसने सूर्य को सर्वथा दीप्ति और चन्द्रमा का प्रकाश बनाया औऱ उनके लिए मंज़िलें निश्चित की, ताकि तुम वर्षों की गिनती और हिसाब मालूम कर लिया करो। अल्लाह ने यह सब कुछ सोद्देश्य ही पैदा किया है। वह अपनी निशानियों को उन लोगों के लिए खोल-खोलकर बयान करता है, जो जानना चाहें
\end{hindi}}
\flushright{\begin{Arabic}
\quranayah[10][6]
\end{Arabic}}
\flushleft{\begin{hindi}
निस्संदेह रात और दिन के उलट-फेर में और जो कुछ अल्लाह ने आकाशों और धरती में पैदा किया उसमें डर रखनेवाले लोगों के लिए निशानियाँ है
\end{hindi}}
\flushright{\begin{Arabic}
\quranayah[10][7]
\end{Arabic}}
\flushleft{\begin{hindi}
रहे वे लोग जो हमसे मिलने की आशा नहीं रखते और सांसारिक जीवन ही पर निहाल हो गए है और उसी पर संतुष्ट हो बैठे, और जो हमारी निशानियों की ओर से असावधान है;
\end{hindi}}
\flushright{\begin{Arabic}
\quranayah[10][8]
\end{Arabic}}
\flushleft{\begin{hindi}
ऐसे लोगों का ठिकाना आग है, उसके बदले में जो वे कमाते रहे
\end{hindi}}
\flushright{\begin{Arabic}
\quranayah[10][9]
\end{Arabic}}
\flushleft{\begin{hindi}
रहे वे लोग जो ईमान लाए और उन्होंने अच्छे कर्म किए, उनका रब उनके ईमान के कारण उनका मार्गदर्शन करेगा। उनके नेमत भरी जन्नतों में नहरें बह रही होगी
\end{hindi}}
\flushright{\begin{Arabic}
\quranayah[10][10]
\end{Arabic}}
\flushleft{\begin{hindi}
वहाँ उनकी पुकार यह होगी कि "महिमा है तेरी, ऐ अल्लाह!" और उनका पारस्परिक अभिवादन "सलाम" होगा। और उनकी पुकार का अन्त इसपर होगा कि "प्रशंसा अल्लाह ही के लिए है जो सारे संसार का रब है।"
\end{hindi}}
\flushright{\begin{Arabic}
\quranayah[10][11]
\end{Arabic}}
\flushleft{\begin{hindi}
यदि अल्लाह लोगों के लिए उनके जल्दी मचाने के कारण भलाई की जगह बुराई को शीघ्र घटित कर दे तो उनकी ओर उनकी अवधि पूरी कर दी जाए, किन्तु हम उन लोगों को जो हमसे मिलने की आशा नहीं रखते उनकी अपनी सरकशी में भटकने के लिए छोड़ देते है
\end{hindi}}
\flushright{\begin{Arabic}
\quranayah[10][12]
\end{Arabic}}
\flushleft{\begin{hindi}
मनुष्य को जब कोई तकलीफ़ पहुँचती है, वह लेटे या बैठे या खड़े हमको पुकारने लग जाता है। किन्तु जब हम उसकी तकलीफ़ उससे दूर कर देते है तो वह इस तरह चल देता है मानो कभी कोई तकलीफ़ पहुँचने पर उसने हमें पुकारा ही न था। इसी प्रकार मर्यादाहीन लोगों के लिए जो कुछ वे कर रहे है सुहावना बना दिया गया है
\end{hindi}}
\flushright{\begin{Arabic}
\quranayah[10][13]
\end{Arabic}}
\flushleft{\begin{hindi}
तुमसे पहले कितनी ही नस्लों को, जब उन्होंने अत्याचार किया, हम विनष्ट कर चुके है, हालाँकि उनके रसूल उनके पास खुली निशानियाँ लेकर आए थे। किन्तु वे ऐसे न थे कि उन्हें मानते। अपराधी लोगों को हम इसी प्रकार बदला दिया करते है
\end{hindi}}
\flushright{\begin{Arabic}
\quranayah[10][14]
\end{Arabic}}
\flushleft{\begin{hindi}
फिर उनके पश्चात हमने धरती में उनकी जगह तुम्हें रखा, ताकि हम देखें कि तुम कैसे कर्म करते हो
\end{hindi}}
\flushright{\begin{Arabic}
\quranayah[10][15]
\end{Arabic}}
\flushleft{\begin{hindi}
और जब उनके सामने हमारी खुली हुई आयतें पढ़ी जाती है तो वे लोग, जो हमसे मिलने की आशा नहीं रखते, कहते है, "इसके सिवा कोई और क़ुरआन ले आओ या इसमें कुछ परिवर्तन करो।" कह दो, "मुझसे यह नहीं हो सकता कि मैं अपनी ओर से इसमें कोई परिवर्तन करूँ। मैं तो बस उसका अनुपालन करता हूँ, जो प्रकाशना मेरी ओर अवतरित की जाती है। यदि मैं अपने प्रभु की अवज्ञा करँस तो इसमें मुझे एक बड़े दिन की यातना का भय है।"
\end{hindi}}
\flushright{\begin{Arabic}
\quranayah[10][16]
\end{Arabic}}
\flushleft{\begin{hindi}
कह दो, "यदि अल्लाह चाहता तो मैं तुम्हें यह पढ़कर न सुनाता और न वह तुम्हें इससे अवगत कराता। आख़िर इससे पहले मैं तुम्हारे बीच जीवन की पूरी अवधि व्यतीत कर चुका हूँ। फिर क्या तुम बुद्धि से काम नहीं लेते?"
\end{hindi}}
\flushright{\begin{Arabic}
\quranayah[10][17]
\end{Arabic}}
\flushleft{\begin{hindi}
फिर उस व्यक्ति से बढ़कर अत्याचारी कौन होगा जो अल्लाह पर थोपकर झूठ घड़े या उसकी आयतों को झुठलाए? निस्संदेह अपराधी कभी सफल नहीं होते
\end{hindi}}
\flushright{\begin{Arabic}
\quranayah[10][18]
\end{Arabic}}
\flushleft{\begin{hindi}
वे लोग अल्लाह से हटकर उनको पूजते हैं, जो न उनका कुछ बिगाड़ सकें और न उनका कुछ भला कर सकें। और वे कहते है, "ये अल्लाह के यहाँ हमारे सिफ़ारिशी है।" कह दो, "क्या तुम अल्लाह को उसकी ख़बर देनेवाले? हो, जिसका अस्तित्व न उसे आकाशों में ज्ञात है न धरती में" महिमावान है वह और उसकी उच्चता के प्रतिकूल है वह शिर्क, जो वे कर रहे है
\end{hindi}}
\flushright{\begin{Arabic}
\quranayah[10][19]
\end{Arabic}}
\flushleft{\begin{hindi}
सारे मनुष्य एक ही समुदाय थे। वे तो स्वयं अलग-अलग हो रहे। और यदि तेरे रब की ओर से पहले ही एक बात निश्चित न हो गई होती, तो उनके बीच का फ़ैसला कर दिया जाता जिसमें वे मतभेद कर रहे हैं
\end{hindi}}
\flushright{\begin{Arabic}
\quranayah[10][20]
\end{Arabic}}
\flushleft{\begin{hindi}
वे कहते है, "उस पर उनके रब की ओर से कोई निशानी क्यों नहीं उतरी?" तो कह दो, "परोक्ष तो अल्लाह ही से सम्बन्ध रखता है। अच्छा, प्रतीक्षा करो, मैं भी तुम्हारे साथ प्रतीक्षा करता हूँ।"
\end{hindi}}
\flushright{\begin{Arabic}
\quranayah[10][21]
\end{Arabic}}
\flushleft{\begin{hindi}
जब हम लोगों को उनके किसी तकलीफ़ में पड़ने के पश्चात दयालुता का रसास्वादन कराते है तो वे हमारी आयतों के विषय में चालबाज़ियाँ करने लग जाते है। कह दो, "अल्लाह की चाल ज़्यादा तेज़ है।" निस्संदेह, जो चालबाजियाँ तुम कर रहे हो, हमारे भेजे हुए (फ़रिश्ते) उनको लिखते जा रहे है
\end{hindi}}
\flushright{\begin{Arabic}
\quranayah[10][22]
\end{Arabic}}
\flushleft{\begin{hindi}
वही है जो तुम्हें थल और जल में चलाता है, यहाँ तक कि जब तुम नौका में होते हो और वह लोगो को लिए हुए अच्छी अनुकूल वायु के सहारे चलती है और वे उससे हर्षित होते है कि अकस्मात उनपर प्रचंड वायु का झोंका आता है, हर ओर से लहरें उनपर चली आती है और वे समझ लेते है कि बस अब वे घिर गए, उस समय वे अल्लाह ही को, निरी उसी पर आस्था रखकर पुकारने लगते है, "यदि तूने हमें इससे बचा लिया तो हम अवश्य आभारी होंगे।"
\end{hindi}}
\flushright{\begin{Arabic}
\quranayah[10][23]
\end{Arabic}}
\flushleft{\begin{hindi}
फिर जब वह उनको बचा लेता है, तो क्या देखते है कि वे नाहक़ धरती में सरकशी करने लग जाते है। ऐ लोगों! तुम्हारी सरकशी तुम्हारे अपने ही विरुद्ध है। सांसारिक जीवन का सुख ले लो। फिर तुम्हें हमारी ही ओर लौटकर आना है। फिर हम तुम्हें बता देंगे जो कुछ तुम करते रहे होगे
\end{hindi}}
\flushright{\begin{Arabic}
\quranayah[10][24]
\end{Arabic}}
\flushleft{\begin{hindi}
सांसारिक जीवन की उपमा तो बस ऐसी है जैसे हमने आकाश से पानी बरसाया, तो उसके कारण धरती से उगनेवाली चीज़े, जिनको मनुष्य और चौपाये सभी खाते है, घनी हो गई, यहाँ तक कि धरती ने अपना शृंगार कर लिया और सँवर गई और उसके मालिक समझने लगे कि उन्हें उसपर पूरा अधिकार प्राप्त है कि रात या दिन में हमारा आदेश आ पहुँचा। फिर हमने उसे कटी फ़सल की तरह कर दिया, मानो कल वहाँ कोई आबादी ही न थी। इसी तरह हम उन लोगों के लिए खोल-खोलकर निशानियाँ बयान करते है, जो सोच-विचार से काम लेना चाहें
\end{hindi}}
\flushright{\begin{Arabic}
\quranayah[10][25]
\end{Arabic}}
\flushleft{\begin{hindi}
और अल्लाह तुम्हें सलामती के घर की ओर बुलाता है, और जिसे चाहता है सीधी राह चलाता है;
\end{hindi}}
\flushright{\begin{Arabic}
\quranayah[10][26]
\end{Arabic}}
\flushleft{\begin{hindi}
अच्छे से अच्छा कर्म करनेवालों के लिए अच्छा बदला है और इसके अतिरिक्त और भी। और उनके चहरों पर न तो कलौस छाएगी और न ज़िल्लत। वही जन्नतवाले है; वे उसमें सदैव रहेंगे
\end{hindi}}
\flushright{\begin{Arabic}
\quranayah[10][27]
\end{Arabic}}
\flushleft{\begin{hindi}
रहे वे लोग जिन्होंने बुराइयाँ कमाई, तो एक बुराई का बदला भी उसी जैसा होगा; और ज़िल्लत उनपर छा रही होगी। उन्हें अल्लाह से बचानेवाला कोई न होगा। उनके चहरों पर मानो अँधेरी रात के टुकड़े ओढ़ा दिए गए हों। वही आगवाले हैं, उन्हें उसमें सदैव रहना है
\end{hindi}}
\flushright{\begin{Arabic}
\quranayah[10][28]
\end{Arabic}}
\flushleft{\begin{hindi}
और जिस दिन हम उन सबको इकट्ठा करेंगे, फिर उन लोगों से, जिन्होंने शिर्क किया होगा, कहेंगे, "अपनी जगह ठहरे रहो तुम भी और तुम्हारे साझीदार भी।" फिर हम उनके बीच अलगाव पैदा कर देंगे, और उनके ठहराए हुए साझीदार कहेंगे, "तुम हमारी तो हमारी बन्दगी नहीं करते थे
\end{hindi}}
\flushright{\begin{Arabic}
\quranayah[10][29]
\end{Arabic}}
\flushleft{\begin{hindi}
"हमारे और तुम्हारे बीच अल्लाह ही एक गवाह काफ़ी है। हमें तो तुम्हारी बन्दगी की ख़बर तक न थी।"
\end{hindi}}
\flushright{\begin{Arabic}
\quranayah[10][30]
\end{Arabic}}
\flushleft{\begin{hindi}
वहाँ प्रत्येक व्यक्ति अपने पहले के किए हुए कर्मों को स्वयं जाँच लेगा और वह अल्लाह, अपने वास्तविक स्वामी की ओर फिरेंगे और जो कुछ झूठ वे घड़ते रहे थे, वह सब उनसे गुम होकर रह जाएगा
\end{hindi}}
\flushright{\begin{Arabic}
\quranayah[10][31]
\end{Arabic}}
\flushleft{\begin{hindi}
कहो, "तुम्हें आकाश और धरती से रोज़ी कौन देता है, या ये कान और आँखें किसके अधिकार में है और कौन जीवन्त को निर्जीव से निकालता है और निर्जीव को जीवन्त से निकालता है और कौन यह सारा इन्तिज़ाम चला रहा है?" इसपर वे बोल पड़ेगे, "अल्लाह!" तो कहो, "फिर आख़िर तुम क्यों नहीं डर रखते?"
\end{hindi}}
\flushright{\begin{Arabic}
\quranayah[10][32]
\end{Arabic}}
\flushleft{\begin{hindi}
फिर यही अल्लाह तो है तुम्हारा वास्तविक रब। फिर आख़िर सत्य के पश्चात पथभ्रष्टता के अतिरिक्त और क्या रह जाता है? फिर तुम कहाँ से फिरे जाते हो?
\end{hindi}}
\flushright{\begin{Arabic}
\quranayah[10][33]
\end{Arabic}}
\flushleft{\begin{hindi}
इसी तरह अवज्ञाकारी लोगों के प्रति तुम्हारे रब की बात सच्ची होकर रही कि वे मानेंगे नहीं
\end{hindi}}
\flushright{\begin{Arabic}
\quranayah[10][34]
\end{Arabic}}
\flushleft{\begin{hindi}
कहो, "क्या तुम्हारे ठहराए हुए साझीदारों में कोई है जो सृष्टि का आरम्भ भी करता हो, फिर उसकी पुनरावृत्ति भी करे?" कहो, "अल्लाह ही सृष्टि का आरम्भ करता है और वही उसकी पुनरावृति भी; आख़िर तुम कहाँ औधे हुए जाते हो?"
\end{hindi}}
\flushright{\begin{Arabic}
\quranayah[10][35]
\end{Arabic}}
\flushleft{\begin{hindi}
कहो, "क्या तुम्हारे ठहराए साझीदारों में कोई है जो सत्य की ओर मार्गदर्शन करे?" कहो, "अल्लाह ही सत्य के मार्ग पर चलाता है। फिर जो सत्य की ओर मार्गदर्शन करता हो, वह इसका ज़्यादा हक़दार है कि उसका अनुसरण किया जाए या वह जो स्वयं ही मार्ग न पाए जब तक कि उसे मार्ग न दिखाया जाए? फिर यह तुम्हें क्या हो गया है, तुम कैसे फ़ैसले कर रहे हो?"
\end{hindi}}
\flushright{\begin{Arabic}
\quranayah[10][36]
\end{Arabic}}
\flushleft{\begin{hindi}
और उनमें से अधिकतर तो बस अटकल पर चलते है। निश्चय ही अटकल सत्य को कुछ भी दूर नहीं कर सकती। वे जो कुछ कर रहे हैं अल्लाह उसको भली-भाँति जानता है
\end{hindi}}
\flushright{\begin{Arabic}
\quranayah[10][37]
\end{Arabic}}
\flushleft{\begin{hindi}
यह क़ुरआन ऐसा नहीं है कि अल्लाह से हटकर घड लिया जाए, बल्कि यह तो जिसके समझ है, उसकी पुष्टि में है और किताब का विस्तार है, जिसमें किसी संदेह की गुंजाइश नहीं। यह सारे संसार के रब की ओर से है
\end{hindi}}
\flushright{\begin{Arabic}
\quranayah[10][38]
\end{Arabic}}
\flushleft{\begin{hindi}
(क्या उन्हें कोई खटक है) या वे कहते है, "इस व्यक्ति (पैग़म्बर) ने उसे स्वयं ही घड़ लिया है?" कहो, "यदि तुम सच्चे हो, तो इस जैसी एक सुरा ले आओ और अल्लाह से हटकर उसे बुला लो, जिसपर तुम्हारा बस चले।"
\end{hindi}}
\flushright{\begin{Arabic}
\quranayah[10][39]
\end{Arabic}}
\flushleft{\begin{hindi}
बल्कि बात यह है कि जिस चीज़ के ज्ञान पर वे हावी न हो सके, उसे उन्होंने झुठला दिया और अभी उसका परिणाम उनके सामने नहीं आया। इसी प्रकार उन लोगों ने भी झुठलाया था, जो इनसे पहले थे। फिर देख लो उन अत्याचारियों का कैसा परिणाम हुआ!
\end{hindi}}
\flushright{\begin{Arabic}
\quranayah[10][40]
\end{Arabic}}
\flushleft{\begin{hindi}
उनमें कुछ लोग उसपर ईमान रखनेवाले है और उनमें कुछ लोग उसपर ईमान लानेवाले नहीं है। और तुम्हारा रब बिगाड़ पैदा करनेवालों को भली-भाँति जानता है
\end{hindi}}
\flushright{\begin{Arabic}
\quranayah[10][41]
\end{Arabic}}
\flushleft{\begin{hindi}
और यदि वे तुझे झुठलाएँ तो कह दो, "मेरा कर्म मेरे लिए है और तुम्हारा कर्म तुम्हारे लिए। जो कुछ मैं करता हूँ उसकी ज़िम्मेदारी से तुम बरी हो और जो कुछ तुम करते हो उसकी ज़िम्मेदारी से मैं बरी हूँ।"
\end{hindi}}
\flushright{\begin{Arabic}
\quranayah[10][42]
\end{Arabic}}
\flushleft{\begin{hindi}
और उनमें बहुत-से ऐसे लोग है जो तेरी ओर कान लगाते है। किन्तु क्या तू बहरों को सुनाएगा, चाहे वे समझ न रखते हों?
\end{hindi}}
\flushright{\begin{Arabic}
\quranayah[10][43]
\end{Arabic}}
\flushleft{\begin{hindi}
और कुछ उनमें ऐसे हैं, जो तेरी ओर ताकते हैं, किन्तु क्या तू अंधों का मार्ग दिखाएगा, चाहे उन्हें कुछ सूझता न हो?
\end{hindi}}
\flushright{\begin{Arabic}
\quranayah[10][44]
\end{Arabic}}
\flushleft{\begin{hindi}
अल्लाह तो लोगों पर तनिक भी अत्याचार नहीं करता, किन्तु लोग स्वयं ही अपने ऊपर अत्याचार करते है
\end{hindi}}
\flushright{\begin{Arabic}
\quranayah[10][45]
\end{Arabic}}
\flushleft{\begin{hindi}
जिस दिन वह उनको इकट्ठा करेगा तो ऐसा जान पड़ेगा जैसे वे दिन की एक घड़ी भर ठहरे थे। वे परस्पर एक-दूसरे को पहचानेंगे। वे लोग घाटे में पड़ गए, जिन्होंने अल्लाह से मिलने को झुठलाया और वे मार्ग न पा सके
\end{hindi}}
\flushright{\begin{Arabic}
\quranayah[10][46]
\end{Arabic}}
\flushleft{\begin{hindi}
जिस चीज़ का हम उनसे वादा करते है उसमें से कुछ चाहे तुझे दिखा दें या हम तुझे (इससे पहले) उठा लें, उन्हें तो हमारी ओर लौटकर आना ही है। फिर जो कुछ वे कर रहे है उसपर अल्लाह गवाह है
\end{hindi}}
\flushright{\begin{Arabic}
\quranayah[10][47]
\end{Arabic}}
\flushleft{\begin{hindi}
प्रत्येक समुदाय के लिए एक रसूल है। फिर जब उनके पास उनका रसूल आ जाता है तो उनके बीच न्यायपूर्वक फ़ैसला कर दिया जाता है। उनपर कुछ भी अत्याचार नहीं किया जाता
\end{hindi}}
\flushright{\begin{Arabic}
\quranayah[10][48]
\end{Arabic}}
\flushleft{\begin{hindi}
वे कहते है, "यदि तुम सच्चे हो तो यह वादा कब पूरा होगा?"
\end{hindi}}
\flushright{\begin{Arabic}
\quranayah[10][49]
\end{Arabic}}
\flushleft{\begin{hindi}
कहो, "मुझे अपने लिए न तो किसी हानि का अधिकार प्राप्त है और न लाभ का, बल्कि अल्लाह जो चाहता है वही होता है। हर समुदाय के लिए एक नियत समय है, जब उनका नियत समय आ जाता है तो वे न घड़ी भर पीछे हट सकते है और न आगे बढ़ सकते है।"
\end{hindi}}
\flushright{\begin{Arabic}
\quranayah[10][50]
\end{Arabic}}
\flushleft{\begin{hindi}
कहो, "क्या तुमने यह भी सोचा कि यदि तुमपर उसकी यातना रातों रात या दिन को आ जाए तो (क्या तुम उसे टाल सकोगे?) वह आख़िर कौन-सी चीज़ होगी जिसके लिए अपराधियों को जल्दी पड़ी हुई है?
\end{hindi}}
\flushright{\begin{Arabic}
\quranayah[10][51]
\end{Arabic}}
\flushleft{\begin{hindi}
क्या फिर जब वह घटित हो जाएगी तब तुम उसे मानोगे? - क्या अब! इसी के लिए तो तुम जल्दी मचा रहे थे!"
\end{hindi}}
\flushright{\begin{Arabic}
\quranayah[10][52]
\end{Arabic}}
\flushleft{\begin{hindi}
"फिर अत्याचारी लोगों से कहा जाएगा, "स्थायी यातना का मज़ा चख़ो! जो कुछ तुम कमाते रहे हो, उसके सिवा तुम्हें और क्या बदला दिया जा सकता है?"
\end{hindi}}
\flushright{\begin{Arabic}
\quranayah[10][53]
\end{Arabic}}
\flushleft{\begin{hindi}
वे तुम से चाहते है कि उन्हें ख़बर दो कि "क्या वह वास्तव में सत्य है?" कह दो, "हाँ, मेरे रब की क़सम! वह बिल्कुल सत्य है और तुम क़ाबू से बाहर निकल जानेवाले नहीं हो।"
\end{hindi}}
\flushright{\begin{Arabic}
\quranayah[10][54]
\end{Arabic}}
\flushleft{\begin{hindi}
यदि प्रत्येक अत्याचारी व्यक्ति के पास वह सब कुछ हो जो धरती में है, तो वह अर्थदंड के रूप में उसे दे डाले। जब वे यातना को देखेंगे तो मन ही मन में पछताएँगे। उनके बीच न्यायपूर्वक फ़ैसला कर दिया जाएगा और उनपर कोई अत्याचार न होगा
\end{hindi}}
\flushright{\begin{Arabic}
\quranayah[10][55]
\end{Arabic}}
\flushleft{\begin{hindi}
सुन लो, जो कुछ आकाशों और धरती में है, अल्लाह ही का है। जान लो, निस्संदेह अल्लाह का वादा सच्चा है, किन्तु उनमें अधिकतर लोग जानते नहीं
\end{hindi}}
\flushright{\begin{Arabic}
\quranayah[10][56]
\end{Arabic}}
\flushleft{\begin{hindi}
वही जिलाता है और मारता है और उसी की ओर तुम लौटाए जा रहे हो
\end{hindi}}
\flushright{\begin{Arabic}
\quranayah[10][57]
\end{Arabic}}
\flushleft{\begin{hindi}
ऐ लोगो! तुम्हारे पास तुम्हारे रब की ओर से उपदेश औऱ जो कुछ सीनों में (रोग) है, उसके लिए रोगमुक्ति और मोमिनों के लिए मार्गदर्शन और दयालुता आ चुकी है
\end{hindi}}
\flushright{\begin{Arabic}
\quranayah[10][58]
\end{Arabic}}
\flushleft{\begin{hindi}
कह दो, "यह अल्लाह के अनुग्रह और उसकी दया से है, अतः इस पर प्रसन्न होना चाहिए। यह उन सब चीज़ों से उत्तम है, जिनको वे इकट्ठा करने में लगे हुए है।"
\end{hindi}}
\flushright{\begin{Arabic}
\quranayah[10][59]
\end{Arabic}}
\flushleft{\begin{hindi}
कह दो, "क्या तुम लोगों ने यह भी देखा कि जो रोज़ी अल्लाह ने तुम्हारे लिए उतारी है उसमें से तुमने स्वयं ही कुछ को हराम और हलाल ठहरा लिया?" कहो, "क्या अल्लाह ने तुम्हें इसकी अनुमति दी है या तुम अल्लाह पर झूठ घड़कर थोप रहे हो?"
\end{hindi}}
\flushright{\begin{Arabic}
\quranayah[10][60]
\end{Arabic}}
\flushleft{\begin{hindi}
जो लोग झूठ घड़कर उसे अल्लाह पर थोंपते है, उन्होंने क़ियामत के दिन के विषय में क्या समझ रखा है? अल्लाह तो लोगों के लिए बड़ा अनुग्रहवाला है, किन्तु उनमें अधिकतर कृतज्ञता नहीं दिखलाते
\end{hindi}}
\flushright{\begin{Arabic}
\quranayah[10][61]
\end{Arabic}}
\flushleft{\begin{hindi}
तुम जिस दशा में भी होते हो और क़ुरआन से जो कुछ भी पढ़ते हो और तुम लोग जो काम भी करते हो हम तुम्हें देख रहे होते है, जब तुम उसमें लगे होते हो। और तुम्हारे रब से कण भर भी कोई चीज़ छिपी नहीं है, न धरती में न आकाश में और न उससे छोटी और न बड़ी कोई त चीज़ ऐसी है जो एक स्पष्ट किताब में मौजूद न हो
\end{hindi}}
\flushright{\begin{Arabic}
\quranayah[10][62]
\end{Arabic}}
\flushleft{\begin{hindi}
सुन लो, अल्लाह के मित्रों को न तो कोई डर है और न वे शोकाकुल ही होंगे
\end{hindi}}
\flushright{\begin{Arabic}
\quranayah[10][63]
\end{Arabic}}
\flushleft{\begin{hindi}
ये वे लोग है जो ईमान लाए और डर कर रहे
\end{hindi}}
\flushright{\begin{Arabic}
\quranayah[10][64]
\end{Arabic}}
\flushleft{\begin{hindi}
उनके लिए सांसारिक जीवन में भी शुभ-सूचना है और आख़िरत में भी - अल्लाह के शब्द बदलते नहीं - यही बड़ी सफलता है
\end{hindi}}
\flushright{\begin{Arabic}
\quranayah[10][65]
\end{Arabic}}
\flushleft{\begin{hindi}
उनकी बात तुम्हें दुखी न करे, सारा प्रभुत्व अल्लाह ही के लिए है, वह सुनता, जानता है
\end{hindi}}
\flushright{\begin{Arabic}
\quranayah[10][66]
\end{Arabic}}
\flushleft{\begin{hindi}
जान रखो! जो कोई भी आकाशों में है और जो कोई धरती में है, अल्लाह ही का है। जो लोग अल्लाह को छोड़कर दूसरे साझीदारों को पुकारते है, वे आखिर किसका अनुसरण करते है? वे तो केवल अटकल पर चलते है और वे निरे अटकले दौड़ाते है
\end{hindi}}
\flushright{\begin{Arabic}
\quranayah[10][67]
\end{Arabic}}
\flushleft{\begin{hindi}
वही है जिसने तुम्हारे लिए रात बनाई ताकि तुम उसमें चैन पाओ और दिन को प्रकाशमान बनाया (ताकि तुम उसमें दौड़-धूप कर सको); निस्संदेह इसमें उन लोगों के लिए निशानियाँ है, जो सुनते है
\end{hindi}}
\flushright{\begin{Arabic}
\quranayah[10][68]
\end{Arabic}}
\flushleft{\begin{hindi}
वे कहते है, "अल्लाह औलाद रखता है।" महान और उच्च है वह! वह निरपेक्ष है, आकाशों और धरती में जो कुछ है उसी का है। तुम्हारे पास इसका कोई प्रमाण नहीं। क्या तुम अल्लाह से जोड़कर वह बाते कहते हो, जिसका तुम्हे ज्ञान नही?
\end{hindi}}
\flushright{\begin{Arabic}
\quranayah[10][69]
\end{Arabic}}
\flushleft{\begin{hindi}
कह दो, "जो लोग अल्लाह पर थोपकर झूठ घड़ते है, वे सफल नहीं होते।"
\end{hindi}}
\flushright{\begin{Arabic}
\quranayah[10][70]
\end{Arabic}}
\flushleft{\begin{hindi}
यह तो सांसारिक सुख है। फिर हमारी ओर ही उन्हें लौटना है, फिर जो इनकार वे करते रहे होगे उसके बदले में हम उन्हें कठोर यातना का मज़ा चखाएँगे
\end{hindi}}
\flushright{\begin{Arabic}
\quranayah[10][71]
\end{Arabic}}
\flushleft{\begin{hindi}
उन्हें नूह का वृत्तान्त सुनाओ। जब उसने अपनी क़ौम से कहा, "ऐ मेरी क़ौम के लोगो! यदि मेरा खड़ा होना और अल्लाह की आयतों के द्वारा नसीहत करना तुम्हें भारी हो गया है तो मेरा भरोसा अल्लाह पर है। तु अपना मामला ठहरा लो और अपने ठहराए हुए साझीदारों को भी साथ ले लो, फिर तुम्हारा मामला तुम पर कुछ संदिग्ध न रहे; फिर मेरे साथ जो कुछ करना है, कर डालों और मुझे मुहलत न दो।"
\end{hindi}}
\flushright{\begin{Arabic}
\quranayah[10][72]
\end{Arabic}}
\flushleft{\begin{hindi}
फिर यदि तुम मुँह फेरोगे तो मैंने तुमसे कोई बदला नहीं माँगा। मेरा बदला (पारिश्रामिक) बस अल्लाह के ज़िम्मे है, और आदेश मुझे मुस्लिम (आज्ञाकारी) होने का हुआ है
\end{hindi}}
\flushright{\begin{Arabic}
\quranayah[10][73]
\end{Arabic}}
\flushleft{\begin{hindi}
किन्तु उन्होंने झूठला दिया, तो हमने उसे और उन लोगों को, जो उनके साथ नौका में थे, बचा लिया और उन्हें उतराधिकारी बनाया, और उन लोगो को डूबो दिया, जिन्होंने हमारी आयतों को झुठलाया था। अतः देख लो, जिन्हें सचेत किया गया था उनका क्या परिणाम हुआ!
\end{hindi}}
\flushright{\begin{Arabic}
\quranayah[10][74]
\end{Arabic}}
\flushleft{\begin{hindi}
फिर उसके बाद कितने ही रसूल हमने उनकी क़ौम की ओर भेजे और वे उनके पास स्पष्ट निशानियां लेकर आए, किन्तु वे ऐसे न थे कि जिसको पहले झुठला चुके हॊं, उसे मानते। इसी तरह अतिक्रमणकारियों कॆ दिलों पर हम मुहर लगा देते हैं
\end{hindi}}
\flushright{\begin{Arabic}
\quranayah[10][75]
\end{Arabic}}
\flushleft{\begin{hindi}
फिर उनके बाद हमने मूसा और हारून को अपनी आयतों के साथ फ़िरऔन और उसके सरदारों के पास भेजा। किन्तु उन्होंने घमंड किया, वे थे ही अपराधी लोग
\end{hindi}}
\flushright{\begin{Arabic}
\quranayah[10][76]
\end{Arabic}}
\flushleft{\begin{hindi}
अतः जब हमारी ओर से सत्य उनके सामने आया तो वे कहने लगे, "यह तो खुला जादू है।"
\end{hindi}}
\flushright{\begin{Arabic}
\quranayah[10][77]
\end{Arabic}}
\flushleft{\begin{hindi}
मूसा ने कहा, "क्या तुम सत्य के विषय में ऐसा कहते हो, जबकि यह तुम्हारे सामने आ गया है? क्या यह कोई जादू है? जादूगर तो सफल नहीं हुआ करते।"
\end{hindi}}
\flushright{\begin{Arabic}
\quranayah[10][78]
\end{Arabic}}
\flushleft{\begin{hindi}
उन्होंने कहा, "क्या तू हमारे पास इसलिए आया है कि हमें उस चीज़ से फेर दे जिसपर हमने अपना बाप-दादा का पाया है और धरती में तुम दोनों की बड़ाई स्थापित हो जाए? हम तो तुम्हें माननेवाले नहीं।"
\end{hindi}}
\flushright{\begin{Arabic}
\quranayah[10][79]
\end{Arabic}}
\flushleft{\begin{hindi}
फ़िरऔन ने कहा, "हर कुशल जादूगर को मेरे पास लाओ।"
\end{hindi}}
\flushright{\begin{Arabic}
\quranayah[10][80]
\end{Arabic}}
\flushleft{\begin{hindi}
फिर जब जादूगर आ गए तो मूसा ने उनसे कहा, "जो कुछ तुम डालते हो, डालो।"
\end{hindi}}
\flushright{\begin{Arabic}
\quranayah[10][81]
\end{Arabic}}
\flushleft{\begin{hindi}
फिर जब उन्होंने डाला तो मूसा ने कहा, "तुम जो कुछ लाए हो, जादू है। अल्लाह अभी उसे मटियामेट किए देता है। निस्संदेह अल्लाह बिगाड़ पैदा करनेवालों के कर्म को फलीभूत नहीं होने देता
\end{hindi}}
\flushright{\begin{Arabic}
\quranayah[10][82]
\end{Arabic}}
\flushleft{\begin{hindi}
"अल्लाह अपने शब्दों से सत्य को सत्य कर दिखाता है, चाहे अपराधी नापसन्द ही करें।"
\end{hindi}}
\flushright{\begin{Arabic}
\quranayah[10][83]
\end{Arabic}}
\flushleft{\begin{hindi}
फिर मूसा की बात उसकी क़ौम की संतति में से बस कुछ ही लोगों ने मानी; फ़िरऔन और उनके सरदारों के भय से कि कहीं उन्हें किसी फ़ितने में न डाल दें। फ़िरऔन था भी धरती में बहुत सिर उठाए हुए, औऱ निश्चय ही वह हद से आगे बढ़ गया था
\end{hindi}}
\flushright{\begin{Arabic}
\quranayah[10][84]
\end{Arabic}}
\flushleft{\begin{hindi}
मूसा ने कहा, "ऐ मेरी क़ौम के लोगो! यदि तुम अल्लाह पर ईमान रखते हो तो उसपर भरोसा करो, यदि तुम आज्ञाकारी हो।"
\end{hindi}}
\flushright{\begin{Arabic}
\quranayah[10][85]
\end{Arabic}}
\flushleft{\begin{hindi}
इसपर वे बोले, "हमने अल्लाह पर भरोसा किया। ऐ हमारे रब! तू हमें अत्याचारी लोगों के हाथों आज़माइश में न डाल
\end{hindi}}
\flushright{\begin{Arabic}
\quranayah[10][86]
\end{Arabic}}
\flushleft{\begin{hindi}
"और अपनी दयालुता से हमें इनकार करनेवालों से छुटकारा दिया।"
\end{hindi}}
\flushright{\begin{Arabic}
\quranayah[10][87]
\end{Arabic}}
\flushleft{\begin{hindi}
हमने मूसा और उसके भाई की ओर प्रकाशना की कि "तुम दोनों अपने लोगों के लिए मिस्र में कुछ घर निश्चित कर लो औऱ अपने घरों को क़िबला बना लो। और नमाज़ क़ायम करो और ईमानवालों को शुभसूचना दे दो।"
\end{hindi}}
\flushright{\begin{Arabic}
\quranayah[10][88]
\end{Arabic}}
\flushleft{\begin{hindi}
मूसा ने कहा, "हमारे रब! तूने फ़िरऔन और उसके सरदारों को सांसारिक जीवन में शोभा-सामग्री और धन दिए है, हमारे रब, इसलिए कि वे तेरे मार्ग से भटकाएँ! हमारे रब, उनके धन नष्ट कर दे और उनके हृदय कठोर कर दे कि वे ईमान न लाएँ, ताकि वे दुखद यातना देख लें।"
\end{hindi}}
\flushright{\begin{Arabic}
\quranayah[10][89]
\end{Arabic}}
\flushleft{\begin{hindi}
कहा, "तुम दोनों की प्रार्थना स्वीकृत हो चुकी। अतः तुम दोनों जमें रहो और उन लोगों के मार्ग पर कदापि न चलना, जो जानते नहीं।"
\end{hindi}}
\flushright{\begin{Arabic}
\quranayah[10][90]
\end{Arabic}}
\flushleft{\begin{hindi}
और हमने इसराईलियों को समुद्र पार करा दिया। फिर फ़िरऔन और उसकी सेनाओं ने सरकशी और ज़्यादती के साथ उनका पीछा किया, यहाँ तक कि जब वह डूबने लगा तो पुकार उठा, "मैं ईमान ले आया कि उसके सिव कोई पूज्य-प्रभु नही, जिस पर इसराईल की सन्तान ईमान लाई। अब मैं आज्ञाकारी हूँ।"
\end{hindi}}
\flushright{\begin{Arabic}
\quranayah[10][91]
\end{Arabic}}
\flushleft{\begin{hindi}
"क्या अब? हालाँकि इससे पहले तुने अवज्ञा की और बिगाड़ पैदा करनेवालों में से था
\end{hindi}}
\flushright{\begin{Arabic}
\quranayah[10][92]
\end{Arabic}}
\flushleft{\begin{hindi}
"अतः आज हम तेरे शरीर को बचा लेगें, ताकि तू अपने बादवालों के लिए एक निशानी हो जाए। निश्चय ही, बहुत-से लोग हमारी निशानियों के प्रति असावधान ही रहते है।"
\end{hindi}}
\flushright{\begin{Arabic}
\quranayah[10][93]
\end{Arabic}}
\flushleft{\begin{hindi}
और हमने इसराईल की सन्तान को अच्छा, सम्मानित ठिकाना दिया औ उन्हें अच्छी आजीविका प्रदान की। फिर उन्होंने उस समय विभेद किया, जबकि ज्ञान उनके पास आ चुका था। निश्चय ही तुम्हारा रब क़ियामत के दिन उनके बीच उस चीज़ का फ़ैसला कर देगा, जिसमें वे विभेद करते रहे है
\end{hindi}}
\flushright{\begin{Arabic}
\quranayah[10][94]
\end{Arabic}}
\flushleft{\begin{hindi}
अतः यदि तुम्हें उस चीज़ के बारे में कोई संदेह हो, जो हमने तुम्हारी ओर अवतरित की है, तो उनसे पूछ लो जो तुमसे पहले से किताब पढ़ रहे है। तुम्हारे पास तो तुम्हारे रब की ओर से सत्य आ चुका। अतः तुम कदापि सन्देह करनेवाले न हो
\end{hindi}}
\flushright{\begin{Arabic}
\quranayah[10][95]
\end{Arabic}}
\flushleft{\begin{hindi}
और न उन लोगों में सम्मिलित होना जिन्होंन अल्लाह की आयतों को झुठलाया, अन्यथा तुम घाटे में पड़कर रहोगे
\end{hindi}}
\flushright{\begin{Arabic}
\quranayah[10][96]
\end{Arabic}}
\flushleft{\begin{hindi}
निस्संदेह जिन लोगों के विषय में तुम्हारे रब की बात सच्ची होकर रही वे ईमान नहीं लाएँगे,
\end{hindi}}
\flushright{\begin{Arabic}
\quranayah[10][97]
\end{Arabic}}
\flushleft{\begin{hindi}
जब तक के वे दुखद यातना न देख लें, चाहे प्रत्येक निशानी उनके पास आ जाए
\end{hindi}}
\flushright{\begin{Arabic}
\quranayah[10][98]
\end{Arabic}}
\flushleft{\begin{hindi}
फिर ऐसी कोई बस्ती क्यों न हुई कि वह ईमान लाती और उसका ईमान उसके लिए लाभप्रद सिद्ध होता? हाँ, यूनुस की क़ौम के लोग इसके लिए अपवाद है। जब वे ईमान लाए तो हमने सांसारिक जीवन में अपमानजनक यातना को उनपर से टाल दिया और उन्हें एक अवधि तक सुखोपभोग का अवसर प्रदान किया
\end{hindi}}
\flushright{\begin{Arabic}
\quranayah[10][99]
\end{Arabic}}
\flushleft{\begin{hindi}
यदि तुम्हारा रब चाहता तो धरती में जितने लोग है वे सब के सब ईमान ले आते, फिर क्या तुम लोगों को विवश करोगे कि वे मोमिन हो जाएँ?
\end{hindi}}
\flushright{\begin{Arabic}
\quranayah[10][100]
\end{Arabic}}
\flushleft{\begin{hindi}
हालाँकि किसी व्यक्ति के लिए यह सम्भव नहीं कि अल्लाह की अनुज्ञा के बिना कोई क्यक्ति ईमान लाए। वह तो उन लोगों पर गन्दगी डाल देता है, जो बुद्धि से काम नहीं लेते
\end{hindi}}
\flushright{\begin{Arabic}
\quranayah[10][101]
\end{Arabic}}
\flushleft{\begin{hindi}
कहो, "देख लो, आकाशों और धरती में क्या कुछ है!" किन्तु निशानियाँ और चेतावनियाँ उन लोगों के कुछ काम नहीं आती, जो ईमान न लाना चाहें
\end{hindi}}
\flushright{\begin{Arabic}
\quranayah[10][102]
\end{Arabic}}
\flushleft{\begin{hindi}
अतः वे तो उस तरह के दिन की प्रतीक्षा कर रहे हैं, जिस तरह के दिन वे लोग देख चुके है जो उनसे पहले गुज़रे है। कह दो, "अच्छा, प्रतीक्षा करो, मैं भी तुम्हारे साथ प्रतीक्षा करता हूँ।"
\end{hindi}}
\flushright{\begin{Arabic}
\quranayah[10][103]
\end{Arabic}}
\flushleft{\begin{hindi}
फिर हम अपने रसूलों और उन लोगों को बचा लेते रहे हैं, जो ईमान ले आए। ऐसी ही हमारी रीति है, हमपर यह हक़ है कि ईमानवालों को बचा लें
\end{hindi}}
\flushright{\begin{Arabic}
\quranayah[10][104]
\end{Arabic}}
\flushleft{\begin{hindi}
कह दो, "ऐ लोगों! यदि तुम मेरे धर्म के विषय में किसी सन्देह में हो तो मैं तो उनकी बन्दगी नहीं करता जिनकी तुम अल्लाह से हटकर बन्दगी करते हो, बल्कि मैं उस अल्लाह की बन्दगी करता हूँ जो तुम्हें मृत्यु देता है। और मुझे आदेश है कि मैं ईमानवालों में से होऊँ
\end{hindi}}
\flushright{\begin{Arabic}
\quranayah[10][105]
\end{Arabic}}
\flushleft{\begin{hindi}
और यह कि हर ओर से एकाग्र होकर अपना रुख़ इस धर्म की ओर कर लो और मुशरिक़ों में कदापि सम्मिलित न हो,
\end{hindi}}
\flushright{\begin{Arabic}
\quranayah[10][106]
\end{Arabic}}
\flushleft{\begin{hindi}
और अल्लाह से हटकर उसे न पुकारो जो न तुम्हें लाभ पहुँचाए और न तुम्हें हानि पहुँचा सके और न तुम्हारा बुरा कर सके, क्योंकि यदि तुमने ऐसा किया तो उस समय तुम अत्याचारी होगे
\end{hindi}}
\flushright{\begin{Arabic}
\quranayah[10][107]
\end{Arabic}}
\flushleft{\begin{hindi}
यदि अल्लाह तुम्हें किसी तकलीफ़ में डाल दे तो उसके सिवा कोई उसे दूर करनेवाला नहीं। और यदि वह तुम्हारे लिए किसी भलाई का इरादा कर ले तो कोई उसके अनुग्रह को फेरनेवाला भी नहीं। वह इसे अपने बन्दों में से जिस तक चाहता है, पहुँचाता है और वह अत्यन्त क्षमाशील, दयावान है।"
\end{hindi}}
\flushright{\begin{Arabic}
\quranayah[10][108]
\end{Arabic}}
\flushleft{\begin{hindi}
कह दो, "ऐ लोगों! तुम्हारे पास तुम्हारे रब की ओर से सत्य आ चुका है। अब जो कोई मार्ग पर आएगा, तो वह अपने ही लिए मार्ग पर आएगा, और जो कोई पथभ्रष्ट होगा तो वह अपने ही बुरे के लिए पथभ्रष्टि होगा। मैं तुम्हारे ऊपर कोई हवालेदार तो हूँ नहीं।"
\end{hindi}}
\flushright{\begin{Arabic}
\quranayah[10][109]
\end{Arabic}}
\flushleft{\begin{hindi}
जो कुछ तुमपर प्रकाशना की जा रही है, उसका अनुसरण करो और धैर्य से काम लो, यहाँ तक कि अल्लाह फ़ैसला कर दे, और वह सबसे अच्छा फ़ैसला करनेवाला है
\end{hindi}}
\chapter{Hud (Hud)}
\begin{Arabic}
\Huge{\centerline{\basmalah}}\end{Arabic}
\flushright{\begin{Arabic}
\quranayah[11][1]
\end{Arabic}}
\flushleft{\begin{hindi}
अलिफ़॰ लाम॰ रा॰। यह एक किताब है जिसकी आयतें पक्की है, फिर सविस्तार बयान हुई हैं; उसकी ओर से जो अत्यन्त तत्वदर्शी, पूरी ख़बर रखनेवाला है
\end{hindi}}
\flushright{\begin{Arabic}
\quranayah[11][2]
\end{Arabic}}
\flushleft{\begin{hindi}
कि "तुम अल्लाह के सिवा किसी की बन्दगी न करो। मैं तो उसकी ओर से तुम्हें सचेत करनेवाला और शुभ सूचना देनेवाला हूँ।"
\end{hindi}}
\flushright{\begin{Arabic}
\quranayah[11][3]
\end{Arabic}}
\flushleft{\begin{hindi}
और यह कि "अपने रब से क्षमा माँगो, फिर उसकी ओर पलट आओ। वह तुम्हें एक निश्चित अवधि तक सुखोपभोग की उत्तम सामग्री प्रदान करेगा। और बढ़-बढ़कर कर्म करनेवालों पर वह तदधिक अपना अनुग्रह करेगा, किन्तु यदि तुम मुँह फेरते हो तो निश्चय ही मुझे तुम्हारे विषय में एक बड़े दिन की यातना का भय है
\end{hindi}}
\flushright{\begin{Arabic}
\quranayah[11][4]
\end{Arabic}}
\flushleft{\begin{hindi}
तुम्हें अल्लाह ही की ओर पलटना है, और उसे हर चीज़ की सामर्थ्य प्राप्त है।"
\end{hindi}}
\flushright{\begin{Arabic}
\quranayah[11][5]
\end{Arabic}}
\flushleft{\begin{hindi}
देखो! ये अपने सीनों को मोड़ते है, चाहिए कि उससे छिपें। देखों! जब ये अपने कपड़ों से स्वयं को ढाँकते है, वह जानता है जो कुछ वे छिपाते है और जो कुछ वे प्रकट करते है। निस्संदेह वह सीनों तक की बात को जानता है
\end{hindi}}
\flushright{\begin{Arabic}
\quranayah[11][6]
\end{Arabic}}
\flushleft{\begin{hindi}
धरती में चलने-फिरनेवाला जो प्राणी भी है उसकी रोज़ी अल्लाह के ज़िम्मे है। वह जानता है जहाँ उसे ठहरना है और जहाँ उसे सौपा जाना है। सब कुछ एक स्पष्ट किताब में मौजूद है
\end{hindi}}
\flushright{\begin{Arabic}
\quranayah[11][7]
\end{Arabic}}
\flushleft{\begin{hindi}
वही है जिसने आकाशों और धरती को छः दिनों में पैदा किया - उसका सिंहासन पानी पर था - ताकि वह तुम्हारी परीक्षा ले कि तुममें कर्म की स्पष्ट से कौन सबसे अच्छा है। और यदि तुम कहो कि "मरने के पश्चात तुम अवश्य उठोगे।" तो जिन्हें इनकार है, वे कहने लगेंगे, "यह तो खुला जादू है।"
\end{hindi}}
\flushright{\begin{Arabic}
\quranayah[11][8]
\end{Arabic}}
\flushleft{\begin{hindi}
यदि हम एक निश्चित अवधि तक के लिए उनसे यातना को टाले रखें, तो वे कहने लगेंगे, "आख़िर किस चीज़ ने उसे रोक रखा है?" सुन लो! जिन दिन वह उनपर आ जाएगी तो फिर वह उनपर से टाली नहीं जाएगी। और वही चीज़ उन्हें घेर लेगी जिसका वे उपहास करते है
\end{hindi}}
\flushright{\begin{Arabic}
\quranayah[11][9]
\end{Arabic}}
\flushleft{\begin{hindi}
यदि हम मनुष्य को अपनी दयालुता का रसास्वादन कराकर फिर उसको छीन लॆं, तॊ (वह दयालुता कॆ लिए याचना नहीं करता) निश्चय ही वह निराशावादी, कृतघ्न है
\end{hindi}}
\flushright{\begin{Arabic}
\quranayah[11][10]
\end{Arabic}}
\flushleft{\begin{hindi}
और यदि हम इसके पश्चात कि उसे तकलीफ़ पहुँची हो, उसे नेमत का रसास्वादन कराते है तो वह कहने लगता है, "मेरे तो सारे दुख दूर हो गए।" वह तो फूला नहीं समाता, डींगे मारने लगता है
\end{hindi}}
\flushright{\begin{Arabic}
\quranayah[11][11]
\end{Arabic}}
\flushleft{\begin{hindi}
उनकी बात दूसरी है जिन्होंने धैर्य से काम लिया और सत्कर्म किए। वही है जिनके लिए क्षमा और बड़ा प्रतिदान है
\end{hindi}}
\flushright{\begin{Arabic}
\quranayah[11][12]
\end{Arabic}}
\flushleft{\begin{hindi}
तो शायद तुम उसमें से कुछ छोड़ बैठोगे, जो तुम्हारी ओर प्रकाशना रूप में भेजी जा रही है। और तुम इस बात पर तंगदिल हो रहे हो कि वे कहते है, "उसपर कोई ख़ज़ाना क्यों नहीं उतरा या उसके साथ कोई फ़रिश्ता क्यों नहीं आया?" तुम तो केवल सचेत करनेवाले हो। हर चीज़ अल्लाह ही के हवाले है
\end{hindi}}
\flushright{\begin{Arabic}
\quranayah[11][13]
\end{Arabic}}
\flushleft{\begin{hindi}
(उन्हें कोई शंका है) या वे कहते है कि "उसने इसे स्वयं घड़ लिया है?" कह दो, "अच्छा, यदि तुम सच्चे हो तो इस जैसी घड़ी हुई दस सूरतें ले आओ और अल्लाह से हटकर जिस किसी को बुला सकते हो बुला लो।"
\end{hindi}}
\flushright{\begin{Arabic}
\quranayah[11][14]
\end{Arabic}}
\flushleft{\begin{hindi}
फिर यदि वे तुम्हारी बातें न मानें तो जान लो, यह अल्लाह के ज्ञान ही के साथ अवतरित हुआ है। और यह कि उसके सिवा कोई पूज्य-प्रभु नहीं। तो अब क्या तुम मुस्लिम (आज्ञाकारी) होते हो?
\end{hindi}}
\flushright{\begin{Arabic}
\quranayah[11][15]
\end{Arabic}}
\flushleft{\begin{hindi}
जो व्यक्ति सांसारिक जीवन और उसकी शोभा का इच्छुक हो तो ऐसे लोगों को उनके कर्मों का पूरा-पूरा बदला हम यहीं दे देते है और इसमें उनका कोई हक़ नहीं मारा जाता
\end{hindi}}
\flushright{\begin{Arabic}
\quranayah[11][16]
\end{Arabic}}
\flushleft{\begin{hindi}
यही वे लोग है जिनके लिए आख़िरत में आग के सिवा और कुछ भी नहीं। उन्होंने जो कुछ बनाया, वह सब वहाँ उनकी जान को लागू हुआ और उनका सारा किया-धरा मिथ्या होकर रहा
\end{hindi}}
\flushright{\begin{Arabic}
\quranayah[11][17]
\end{Arabic}}
\flushleft{\begin{hindi}
फिर क्या वह व्यक्ति जो अपने रब के एक स्पष्ट प्रमाण पर है और स्वयं उसके रूप में भी एक गवाह उसके साथ-साथ रहता है - और इससे पहले मूसा की किताब भी एक मार्गदर्शक और दयालुता के रूप में उपस्थित रही है- (और वह जो प्रकाश एवं मार्गदर्शन से वंचित है, दोनों बराबर हो सकते है) ऐसे ही लोग उसपर ईमान लाते है, किन्तु इन गिरोहों में से जो उसका इनकार करेगा तो उसके लिए जिस जगह का वादा है, वह तो आग है। अतः तुम्हें इसके विषय में कोई सन्देह न हो। यह तुम्हारे रब की ओर से सत्य है, किन्तु अधिकतर लोग मानते नहीं
\end{hindi}}
\flushright{\begin{Arabic}
\quranayah[11][18]
\end{Arabic}}
\flushleft{\begin{hindi}
उस व्यक्ति से बढ़कर अत्याचारी कौन होगा जो अल्लाह पर थोपकर झूठ घड़े। ऐसे लोग अपने रब के सामने पेश होंगे और गवाही देनेवाले कहेंगे, "यही लोग है जिन्होंने अपने रब पर झूठ घड़ा।" सुन लो! ऐसे अत्याचारियों पर अल्लाह की लानत है
\end{hindi}}
\flushright{\begin{Arabic}
\quranayah[11][19]
\end{Arabic}}
\flushleft{\begin{hindi}
जो अल्लाह के मार्ग से रोकते है और उसमें टेढ़ पैदा करना चाहते है; और वही आख़िरत का इनकार करते है
\end{hindi}}
\flushright{\begin{Arabic}
\quranayah[11][20]
\end{Arabic}}
\flushleft{\begin{hindi}
वे धरती में क़ाबू से बाहर नहीं जा सकते और न अल्लाह से हटकर उनका कोई समर्थक ही है। उन्हें दोहरी यातना दी जाएगी। वे न सुन ही सकते थे और न देख ही सकते थे
\end{hindi}}
\flushright{\begin{Arabic}
\quranayah[11][21]
\end{Arabic}}
\flushleft{\begin{hindi}
ये वही लोग है जिन्होंने अपने आपको घाटे में डाला और जो कुछ वे घड़ा करते थे, वह सब उनमें गुम होकर रह गया
\end{hindi}}
\flushright{\begin{Arabic}
\quranayah[11][22]
\end{Arabic}}
\flushleft{\begin{hindi}
निश्चय ही वही आख़िरत में सबसे बढ़कर घाटे में रहेंगे
\end{hindi}}
\flushright{\begin{Arabic}
\quranayah[11][23]
\end{Arabic}}
\flushleft{\begin{hindi}
रहे वे लोग जो ईमान लाए और उन्होंने अच्छे कर्म किए और अपने रब की ओर झूक पड़े वही जन्नतवाले है, उसमें वे सदैव रहेंगे
\end{hindi}}
\flushright{\begin{Arabic}
\quranayah[11][24]
\end{Arabic}}
\flushleft{\begin{hindi}
दोनों पक्षों की उपमा ऐसी है जैसे एक अन्धा और बहरा हो और एक देखने और सुननेवाला। क्या इन दोनों की दशा समान हो सकती है? तो क्या तुम होश से काम नहीं लेते?
\end{hindi}}
\flushright{\begin{Arabic}
\quranayah[11][25]
\end{Arabic}}
\flushleft{\begin{hindi}
हमने नूह को उसकी क़ौम की ओर भेजा। (उसने कहा,) "मैं तुम्हें साफ़-साफ़ चेतावनी देता हूँ
\end{hindi}}
\flushright{\begin{Arabic}
\quranayah[11][26]
\end{Arabic}}
\flushleft{\begin{hindi}
यह कि तुम अल्लाह के सिवा किसी की बन्दगी न करो। मुझे तुम्हारे विषय में एक दुखद दिन की यातना का भय है।"
\end{hindi}}
\flushright{\begin{Arabic}
\quranayah[11][27]
\end{Arabic}}
\flushleft{\begin{hindi}
इसपर उसकी क़ौम के सरदार, जिन्होंने इनकार किया था, कहने लगे, "हमारी दृष्टि में तो तुम हमारे ही जैसे आदमी हो और हम देखते है कि बस कुछ ऐसे लोग ही तुम्हारे अनुयायी है जो पहली स्पष्ट में हमारे यहाँ के नीच है। हम अपने मुक़ाबले में तुममें कोई बड़ाई नहीं देखते, बल्कि हम तो तुम्हें झूठा समझते है।"
\end{hindi}}
\flushright{\begin{Arabic}
\quranayah[11][28]
\end{Arabic}}
\flushleft{\begin{hindi}
उसने कहा, "ऐ मेरी क़ौम के लोगो! तुम्हारा क्या विचार है? यदि मैं अपने रब के एक स्पष्ट प्रमाण पर हूँ और उसने मुझे अपने पास से दयालुता भी प्रदान की है, फिर वह तुम्हें न सूझे तो क्या हम हठात उसे तुमपर चिपका दें, जबकि वह तुम्हें अप्रिय है?
\end{hindi}}
\flushright{\begin{Arabic}
\quranayah[11][29]
\end{Arabic}}
\flushleft{\begin{hindi}
और ऐ मेरी क़ौम के लोगो! मैं इस काम पर कोई धन नहीं माँगता। मेरा पारिश्रमिक तो बस अल्लाह के ज़िम्मे है। मैं ईमान लानेवालो को दूर करनेवाला भी नहीं। उन्हें तो अपने रब से मिलना ही है, किन्तु मैं तुम्हें देख रहा हूँ कि तुम अज्ञानी लोग हो
\end{hindi}}
\flushright{\begin{Arabic}
\quranayah[11][30]
\end{Arabic}}
\flushleft{\begin{hindi}
और ऐ मेरी क़ौम के लोगो! यदि मैं उन्हें धुत्कार दूँ तो अल्लाह के मुक़ाबले में कौन मेरी सहायता कर सकता है? फिर क्या तुम होश से काम नहीं लेते?
\end{hindi}}
\flushright{\begin{Arabic}
\quranayah[11][31]
\end{Arabic}}
\flushleft{\begin{hindi}
और मैं तुमसे यह नहीं कहता कि मेरे पास अल्लाह के ख़जाने है और न मुझे परोक्ष का ज्ञान है और न मैं यह कहता हूँ कि मैं कोई फ़रिश्ता हूँ और न उन लोगों के विषय में, जो तुम्हारी दृष्टि में तुच्छ है, मैं यह कहता हूँ कि अल्लाह उन्हें कोई भलाई न देगा। जो कुछ उनके जी में है, अल्लाह उसे भली-भाँति जानता है। (यदि मैं ऐसा कहूँ) तब तो मैं अवश्य ही ज़ालिमों में से हूँगा।"
\end{hindi}}
\flushright{\begin{Arabic}
\quranayah[11][32]
\end{Arabic}}
\flushleft{\begin{hindi}
उन्होंने कहा, "ऐ नूह! तुम हमसे झगड़ चुके और बहुत झगड़ चुके। यदि तुम सच्चे हो तो जिसकी तुम हमें धमकी देते हो, अब उसे हम पर ले ही आओ।"
\end{hindi}}
\flushright{\begin{Arabic}
\quranayah[11][33]
\end{Arabic}}
\flushleft{\begin{hindi}
उसने कहा, "वह तो अल्लाह ही यदि चाहेगा तो तुमपर लाएगा और तुम क़ाबू से बाहर नहीं जा सकते
\end{hindi}}
\flushright{\begin{Arabic}
\quranayah[11][34]
\end{Arabic}}
\flushleft{\begin{hindi}
अब जबकि अल्लाह ही ने तुम्हें विनष्ट करने का निश्चय कर लिया हो, तो यदि मैं तुम्हारा भला भी चाहूँ, तो मेरा भला चाहना तुम्हें कुछ भी लाभ नहीं पहुँचा सकता। वही तुम्हारा रब है और उसी की ओर तुम्हें पलटना भी है।"
\end{hindi}}
\flushright{\begin{Arabic}
\quranayah[11][35]
\end{Arabic}}
\flushleft{\begin{hindi}
(क्या उन्हें कोई खटक है) या वे कहते है, "उसने स्वयं इसे घड़ लिया है?" कह दो, "यदि मैंने इसे घड़ लिया है तो मेरे अपराध का दायित्व मुझपर ही है। और जो अपराध तुम कर रहे हो मैं उसके दायित्व से मुक्त हूँ।"
\end{hindi}}
\flushright{\begin{Arabic}
\quranayah[11][36]
\end{Arabic}}
\flushleft{\begin{hindi}
नूह की ओर प्रकाशना की गई कि "जो लोग ईमान ला चुके है, उनके सिवा अब तुम्हारी क़ौम में कोई ईमान लानेवाला नहीं। अतः जो कुछ वे कर रहे है उसपर तुम दुखी न हो
\end{hindi}}
\flushright{\begin{Arabic}
\quranayah[11][37]
\end{Arabic}}
\flushleft{\begin{hindi}
तुम हमारे समक्ष और हमारी प्रकाशना के अनुसार नाव बनाओ और अत्याचारियों के विषय में मुझसे बात न करो। निश्चय ही वे डूबकर रहेंगे।"
\end{hindi}}
\flushright{\begin{Arabic}
\quranayah[11][38]
\end{Arabic}}
\flushleft{\begin{hindi}
जब नाव बनाने लगता है। उसकी क़ौम के सरदार जब भी उसके पास से गुज़रते तो उसका उपहास करते। उसने कहा, "यदि तुम हमारा उपहास करते हो तो हम भी तुम्हारा उपहास करेंगे, जैसे तुम हमारा उपहास करते हो
\end{hindi}}
\flushright{\begin{Arabic}
\quranayah[11][39]
\end{Arabic}}
\flushleft{\begin{hindi}
अब शीघ्र ही तुम जान लोगे कि कौन है जिसपर ऐसी यातना आती है, जो उसे अपमानित कर देगी और जिसपर ऐसी स्थाई यातना टूट पड़ती है
\end{hindi}}
\flushright{\begin{Arabic}
\quranayah[11][40]
\end{Arabic}}
\flushleft{\begin{hindi}
यहाँ तक कि जब हमारा आदेश आ गया और तंदूर उबल पड़ा तो हमने कहा, "हर जाति में से दो-दो के जोड़े चढ़ा लो और अपने घरवालों को भी - सिवाय ऐसे व्यक्ति के जिसके बारे में बात तय पा चुकी है - और जो ईमान लाया हो उसे भी।" किन्तु उसके साथ जो ईमान लाए थे वे थोड़े ही थे
\end{hindi}}
\flushright{\begin{Arabic}
\quranayah[11][41]
\end{Arabic}}
\flushleft{\begin{hindi}
उसने कहा, "उसमें सवार हो जाओ। अल्लाह के नाम से इसका चलना भी है और इसका ठहरना भी। निस्संदेह मेरा रब अत्यन्त क्षमाशील, दयावान है।"
\end{hindi}}
\flushright{\begin{Arabic}
\quranayah[11][42]
\end{Arabic}}
\flushleft{\begin{hindi}
और वह (नाव) उन्हें लिए हुए पहाड़ों जैसी ऊँची लहर के बीच चल रही थी। नूह ने अपने बेटे को, जो उससे अलग था, पुकारा, "ऐ मेरे बेटे! हमारे साथ सवार हो जा। तू इनकार करनेवालों के साथ न रह।"
\end{hindi}}
\flushright{\begin{Arabic}
\quranayah[11][43]
\end{Arabic}}
\flushleft{\begin{hindi}
उसने कहा, "मैं किसी पहाड़ से जा लगूँगा, जो मुझे पानी से बचा लेगा।" कहा, "आज अल्लाह के आदेश (फ़ैसले) से कोई बचानेवाला नहीं है सिवाय उसके जिसपर वह दया करे।" इतने में दोनों के बीच लहर आ पड़ी और डूबनेवालों के साथ वह भी डूब गया
\end{hindi}}
\flushright{\begin{Arabic}
\quranayah[11][44]
\end{Arabic}}
\flushleft{\begin{hindi}
और कहा गया, "ऐ धरती! अपना पानी निगल जा और ऐ आकाश! तू थम जा।" अतएव पानी तह में बैठ गया और फ़ैसला चुका दिया गया और वह (नाव) जूदी पर्वत पर टिक गई औऱ कह दिया गया, "फिटकार हो अत्याचारी लोगों पर!"
\end{hindi}}
\flushright{\begin{Arabic}
\quranayah[11][45]
\end{Arabic}}
\flushleft{\begin{hindi}
नूह ने अपने रब को पुकारा और कहा, "मेरे रब! मेरा बेटा मेरे घरवालों में से है और निस्संदेह तेरा वादा सच्चा है और तू सबसे बड़ा हाकिम भी है।"
\end{hindi}}
\flushright{\begin{Arabic}
\quranayah[11][46]
\end{Arabic}}
\flushleft{\begin{hindi}
कहा, "ऐ नूह! वह तेरे घरवालों में से नहीं, वह तो सर्वथा एक बिगड़ा काम है। अतः जिसका तुझे ज्ञान नहीं, उसके विषय में मुझसे न पूछ, तेरे नादान हो जाने की आशंका से मैं तुझे नसीहत करता हूँ।"
\end{hindi}}
\flushright{\begin{Arabic}
\quranayah[11][47]
\end{Arabic}}
\flushleft{\begin{hindi}
उसने कहा, "मेरे रब! मैं इससे तेरी पनाह माँगता हूँ कि तुझसे उस चीज़ का सवाल करूँ जिसका मुझे कोई ज्ञान न हो। अब यदि तूने मुझे क्षमा न किया और मुझपर दया न की, तो मैं घाटे में पड़कर रहूँगा।"
\end{hindi}}
\flushright{\begin{Arabic}
\quranayah[11][48]
\end{Arabic}}
\flushleft{\begin{hindi}
कहा गया, "ऐ नूह! हमारी ओर से सलामती और उन बरकतों के साथ उतर, जो तुझपर और उन गिरोहों पर होगी, जो तेरे साथवालों में से होंगे। कुछ गिरोह ऐसे भी होंगे जिन्हें हम थोड़े दिनों का सुखोपभोग कराएँगे। फिर उन्हें हमारी ओर से दुखद यातना आ पहुँचेगी।"
\end{hindi}}
\flushright{\begin{Arabic}
\quranayah[11][49]
\end{Arabic}}
\flushleft{\begin{hindi}
ये परोक्ष की ख़बरें हैं जिनकी हम तुम्हारी ओर प्रकाशना कर रहे है। इससे पहले तो न तुम्हें इनकी ख़बर थी और न तुम्हारी क़ौम को। अतः धैर्य से काम लो। निस्संदेह अन्तिम परिणाम डर रखनेवालो के पक्ष में है
\end{hindi}}
\flushright{\begin{Arabic}
\quranayah[11][50]
\end{Arabic}}
\flushleft{\begin{hindi}
और 'आद' की ओर उनके भाई 'हूद' को भेजा। उसने कहा, "ऐ मेरी क़ौम के लोगो! अल्लाह की बन्दगी करो। उसके सिवा तुम्हारा कोई पूज्य प्रभु नहीं। तुमने तो बस झूठ घड़ रखा हैं
\end{hindi}}
\flushright{\begin{Arabic}
\quranayah[11][51]
\end{Arabic}}
\flushleft{\begin{hindi}
ऐ मेरी क़ौम के लोगो! मैं इसपर तुमसे कोई पारिश्रमिक नहीं माँगता। मेरा पारिश्रमिक तो बस उसके ज़िम्मे है जिसने मुझे पैदा किया। फिर क्या तुम बुद्धि से काम नहीं लेते?
\end{hindi}}
\flushright{\begin{Arabic}
\quranayah[11][52]
\end{Arabic}}
\flushleft{\begin{hindi}
ऐ मेरी क़ौम के लोगो! अपने रब से क्षमा याचना करो, फिर उसकी ओर पलट आओ। वह तुमपर आकाश को ख़ूब बरसता छोड़ेगा और तुममें शक्ति पर शक्ति की अभिवृद्धि करेगा। तुम अपराधी बनकर मुँह न फेरो।"
\end{hindi}}
\flushright{\begin{Arabic}
\quranayah[11][53]
\end{Arabic}}
\flushleft{\begin{hindi}
उन्होंने कहा, "ऐ हूद! तू हमारे पास कोई स्पष्ट प्रमाण लेकर नहीं आया है। तेरे कहने से हम अपने इष्ट -पूज्यों को नहीं छोड़ सकते और न हम तुझपर ईमान लानेवाले है
\end{hindi}}
\flushright{\begin{Arabic}
\quranayah[11][54]
\end{Arabic}}
\flushleft{\begin{hindi}
हम तो केवल यही कहते है कि हमारे इष्ट-पूज्यों में से किसी की तुझपर मार पड़ गई है।" उसने कहा, "मैं तो अल्लाह को गवाह बनाता हूँ और तुम भी गवाह रहो कि उनसे मेरा कोई सम्बन्ध नहीं,
\end{hindi}}
\flushright{\begin{Arabic}
\quranayah[11][55]
\end{Arabic}}
\flushleft{\begin{hindi}
जिनको तुम साझी ठहराकर उसके सिवा पूज्य मानते हो। अतः तुम सब मिलकर मेरे साथ दाँव-घात लगाकर देखो और मुझे मुहलत न दो
\end{hindi}}
\flushright{\begin{Arabic}
\quranayah[11][56]
\end{Arabic}}
\flushleft{\begin{hindi}
मेरा भरोसा तो अल्लाह, अपने रब और तुम्हारे रब, पर है। चलने-फिरनेवाला जो प्राणी भी है, उसकी चोटी तो उसी के हाथ में है। निस्संदेह मेरा रब सीधे मार्ग पर है
\end{hindi}}
\flushright{\begin{Arabic}
\quranayah[11][57]
\end{Arabic}}
\flushleft{\begin{hindi}
किन्तु यदि तुम मुँह मोड़ते हो तो जो कुछ देकर मुझे तुम्हारी ओर भेजा गया था, वह तो मैं तुम्हें पहुँचा ही चुका। मेरा रब तुम्हारे स्थान पर दूसरी किसी क़ौम को लाएगा और तुम उसका कुछ न बिगाड़ सकोगे। निस्संदेह मेरा रब हर चीज़ की देख-भाल कर रहा है।"
\end{hindi}}
\flushright{\begin{Arabic}
\quranayah[11][58]
\end{Arabic}}
\flushleft{\begin{hindi}
और जब हमारा आदेश आ पहुँचा तो हमने हूद और उसके साथ के ईमान लानेवालों को अपनी दयालुता से बचा लिया। और एक कठोर यातना से हमने उन्हें छुटकारा दिया
\end{hindi}}
\flushright{\begin{Arabic}
\quranayah[11][59]
\end{Arabic}}
\flushleft{\begin{hindi}
ये आद है, जिन्होंने अपने रब की आयतों का इनकार किया; उसके रसूलों की अवज्ञा की और हर सरकश विरोधी के पीछे चलते रहे
\end{hindi}}
\flushright{\begin{Arabic}
\quranayah[11][60]
\end{Arabic}}
\flushleft{\begin{hindi}
इस संसार में भी लानत ने उनका पीछा किया और क़ियामत के दिन भी, "सुन लो! निस्संदेह आद ने अपने रब के साथ कुफ़्र किया। सुनो! विनष्ट हो आद, हूद की क़ौम।"
\end{hindi}}
\flushright{\begin{Arabic}
\quranayah[11][61]
\end{Arabic}}
\flushleft{\begin{hindi}
समूद को और उसके भाई सालेह को भेजा। उसने कहा, "ऐ मेरी क़़ौम के लोगों! अल्लाह की बन्दगी करो। उसके सिवा तुम्हारा कोई अन्य पूज्य-प्रभु नहीं। उसी ने तुम्हें धरती से पैदा किया और उसमें तुम्हें बसायाय़ अतः उससे क्षमा माँगो; फिर उसकी ओर पलट आओ। निस्संदेह मेरा रब निकट है, प्रार्थनाओं को स्वीकार करनेवाला भी।"
\end{hindi}}
\flushright{\begin{Arabic}
\quranayah[11][62]
\end{Arabic}}
\flushleft{\begin{hindi}
उन्होंने कहा, "ऐ सालेह! इससे पहले तू हमारे बीच ऐसा व्यक्ति था जिससे बड़ी आशाएँ थीं। क्या तू हमें उनको पूजने से रोकता है जिनकी पूजा हमारे बाप-दादा करते रहे है? जिनकी ओर तू हमें बुला रहा है उसके विषय में तो हमें संदेह है जो हमें दुविधा में डाले हुए है।"
\end{hindi}}
\flushright{\begin{Arabic}
\quranayah[11][63]
\end{Arabic}}
\flushleft{\begin{hindi}
उसने कहा, "ऐ मेरी क़ौम के लोगों! क्या तुमने सोचा? यदि मैं अपने रब के एक स्पष्ट प्रमाण पर हूँ और उसने मुझे अपनी ओर से दयालुता प्रदान की है, तो यदि मैं उसकी अवज्ञा करूँ तो अल्लाह के मुक़ाबले में कौन मेरी सहायता करेगा? तुम तो और अधिक घाटे में डाल देने के अतिरिक्त मेरे हक़ में और कोई अभिवृद्धि नहीं करोगे
\end{hindi}}
\flushright{\begin{Arabic}
\quranayah[11][64]
\end{Arabic}}
\flushleft{\begin{hindi}
ऐ मेरी क़ौम के लोगो! यह अल्लाह की ऊँटनी तुम्हारे लिए एक निशानी है। इसे छोड़ दो कि अल्लाह की धरती में खाए और इसे तकलीफ़ देने के लिए हाथ न लगाना अन्यथा समीपस्थ यातना तुम्हें आ लेगी।"
\end{hindi}}
\flushright{\begin{Arabic}
\quranayah[11][65]
\end{Arabic}}
\flushleft{\begin{hindi}
किन्तु उन्होंने उसकी कूंचे काट डाली। इसपर उसने कहा, "अपने घरों में तीन दिन और मज़े ले लो। यह ऐसा वादा है, जो झूठा सिद्ध न होगा।"
\end{hindi}}
\flushright{\begin{Arabic}
\quranayah[11][66]
\end{Arabic}}
\flushleft{\begin{hindi}
फिर जब हमारा आदेश आ पहुँचा, तो हमने अपनी दयालुता से सालेह को और उसके साथ के ईमान लानेवालों को बचा लिया, और उस दिन के अपमान से उन्हें सुरक्षित रखा। वास्तव में, तुम्हारा रब बड़ा शक्तिवान, प्रभुत्वशाली है
\end{hindi}}
\flushright{\begin{Arabic}
\quranayah[11][67]
\end{Arabic}}
\flushleft{\begin{hindi}
और अत्याचार करनेवालों को एक भयंकर चिंघार ने आ लिया और वे अपने घरों में औंधे पड़े रह गए,
\end{hindi}}
\flushright{\begin{Arabic}
\quranayah[11][68]
\end{Arabic}}
\flushleft{\begin{hindi}
मानो वे वहाँ कभी बसे ही न थे। "सुनो! समूद ने अपने रब के साथ कुफ़्र किया। सुन लो! फिटकार हो समूद पर!"
\end{hindi}}
\flushright{\begin{Arabic}
\quranayah[11][69]
\end{Arabic}}
\flushleft{\begin{hindi}
और हमारे भेजे हुए (फ़रिश्ते) इबराहीम के पास शुभ सूचना लेकर पहुँचे। उन्होंने कहा, "सलाम हो!" उसने भी कहा, "सलाम हो।" फिर उसने कुछ विलम्भ न किया, एक भुना हुआ बछड़ा ले आया
\end{hindi}}
\flushright{\begin{Arabic}
\quranayah[11][70]
\end{Arabic}}
\flushleft{\begin{hindi}
किन्तु जब देखा कि उनके हाथ उसकी ओर नहीं बढ़ रहे है तो उसने उन्हें अजनबी समझा और दिल में उनसे डरा। वे बोले, "डरो नहीं, हम तो लूत की क़ौम की ओर से भेजे गए है।"
\end{hindi}}
\flushright{\begin{Arabic}
\quranayah[11][71]
\end{Arabic}}
\flushleft{\begin{hindi}
उसकी स्त्री भी खड़ी थी। वह इसपर हँस पड़ी। फिर हमने उसको इसहाक़ और इसहाक़ के बाद याक़ूब की शुभ सूचना दी
\end{hindi}}
\flushright{\begin{Arabic}
\quranayah[11][72]
\end{Arabic}}
\flushleft{\begin{hindi}
वह बोली, "हाय मेरा हतभाग्य! क्या मैं बच्चे को जन्म दूँगी, जबकि मैं वृद्धा और ये मेरे पति है बूढें? यह तो बड़ी ही अद्भुपत बात है!"
\end{hindi}}
\flushright{\begin{Arabic}
\quranayah[11][73]
\end{Arabic}}
\flushleft{\begin{hindi}
वे बोले, "क्या अल्लाह के आदेश पर तुम आश्चर्य करती हो? घरवालो! तुम लोगों पर तो अल्लाह की दयालुता और उसकी बरकतें है। वह निश्चय ही प्रशंसनीय, गौरववाला है।"
\end{hindi}}
\flushright{\begin{Arabic}
\quranayah[11][74]
\end{Arabic}}
\flushleft{\begin{hindi}
फिर जब इबराहीम की घबराहट दूर हो गई और उसे शुभ सूचना भी मिली तो वह लूत की क़ौम के विषय में हम से झगड़ने लगा
\end{hindi}}
\flushright{\begin{Arabic}
\quranayah[11][75]
\end{Arabic}}
\flushleft{\begin{hindi}
निस्संदेह इबराहीम बड़ा ही सहनशील, कोमल हृदय, हमारी ओर रुजू (प्रवृत्त) होनेवाला था
\end{hindi}}
\flushright{\begin{Arabic}
\quranayah[11][76]
\end{Arabic}}
\flushleft{\begin{hindi}
"ऐ ईबराहीम! इसे छोड़ दो। तुम्हारे रब का आदेश आ चुका है और निश्चय ही उनपर न टलनेवाली यातना आनेवाली है।"
\end{hindi}}
\flushright{\begin{Arabic}
\quranayah[11][77]
\end{Arabic}}
\flushleft{\begin{hindi}
और जब हमारे दूत लूत के पास पहुँचे तो वह उनके कारण अप्रसन्न हुआ और उनके मामले में दिल तंग पाया। कहने लगा, "यह तो बड़ा ही कठिन दिन है।"
\end{hindi}}
\flushright{\begin{Arabic}
\quranayah[11][78]
\end{Arabic}}
\flushleft{\begin{hindi}
उसकी क़ौम के लोग दौड़ते हुए उसके पास आ पहुँचे। वे पहले से ही दुष्कर्म किया करते थे। उसने कहा, "ऐ मेरी क़ौम के लोगो! ये मेरी (क़ौम की) बेटियाँ (विधिवत विवाह के लिए) मौजूड है। ये तुम्हारे लिए अधिक पवित्र है। अतः अल्लाह का डर रखो और मेरे अतिथियों के विषय में मुझे अपमानित न करो। क्या तुममें एक भी अच्छी समझ का आदमी नहीं?"
\end{hindi}}
\flushright{\begin{Arabic}
\quranayah[11][79]
\end{Arabic}}
\flushleft{\begin{hindi}
उन्होंने कहा, "तुझे तो मालूम है कि तेरी बेटियों से हमें कोई मतलब नहीं। और हम जो चाहते है, उसे तू भली-भाँति जानता है।"
\end{hindi}}
\flushright{\begin{Arabic}
\quranayah[11][80]
\end{Arabic}}
\flushleft{\begin{hindi}
उसने कहा, "क्या ही अच्छा होता मुझमें तुमसे मुक़ाबले की शक्ति होती या मैं किसी सुदृढ़ आश्रय की शरण ही ले सकता।"
\end{hindi}}
\flushright{\begin{Arabic}
\quranayah[11][81]
\end{Arabic}}
\flushleft{\begin{hindi}
उन्होंने कहा, "ऐ लूत! हम तुम्हारे रब के भेजे हुए है। वे तुम तक कदापि नहीं पहुँच सकते। अतः तुम रात के किसी हिस्सेमें अपने घरवालों को लेकर निकल जाओ और तुममें से कोई पीछे पलटकर न देखे। हाँ, तुम्हारी स्त्री का मामला और है। उनपर भी वही कुछ बीतनेवाला है, जो उनपर बीतेगा। निर्धारित समय उनके लिए प्रातःकाल का है। तो क्या प्रातःकाल निकट नहीं?"
\end{hindi}}
\flushright{\begin{Arabic}
\quranayah[11][82]
\end{Arabic}}
\flushleft{\begin{hindi}
फिर जब हमारा आदेश आ पहुँचा तो हमने उसको तलपट कर दिया और उसपर ककरीले पत्थर ताबड़-तोड़ बरसाए,
\end{hindi}}
\flushright{\begin{Arabic}
\quranayah[11][83]
\end{Arabic}}
\flushleft{\begin{hindi}
जो तुम्हारे रब के यहाँ चिन्हित थे। और वे अत्याचारियों से कुछ दूर भी नहीं
\end{hindi}}
\flushright{\begin{Arabic}
\quranayah[11][84]
\end{Arabic}}
\flushleft{\begin{hindi}
मदयन की ओर उनके भाई शुऐब को भेजा। उसने कहा, "ऐ मेरी क़ौम के लोगो! अल्लाह की बन्दही करो, उनके सिवा तुम्हारा कोई पूज्य-प्रभु नहीं। और नाप और तौल में कमी न करो। मैं तो तुम्हें अच्छी दशा में देख रहा हूँ, किन्तु मुझे तुम्हारे विषय में एक घेर लेनेवाले दिन की यातना का भय है
\end{hindi}}
\flushright{\begin{Arabic}
\quranayah[11][85]
\end{Arabic}}
\flushleft{\begin{hindi}
ऐ मेरी क़ौम के लोगो! इनसाफ़ के साथ नाप और तौल को पूरा रखो। और लोगों को उनकी चीज़ों में घाटा न दो और धरती में बिगाड़ पैदा करनेवाले बनकर अपने मुँह को कुलषित न करो
\end{hindi}}
\flushright{\begin{Arabic}
\quranayah[11][86]
\end{Arabic}}
\flushleft{\begin{hindi}
यदि तुम मोमिन हो तो जो अल्लाह के पास शेष रहता है वही तुम्हारे लिए उत्तम है। मैं तुम्हारे ऊपर कोई नियुक्त रखवाला नहीं हूँ।"
\end{hindi}}
\flushright{\begin{Arabic}
\quranayah[11][87]
\end{Arabic}}
\flushleft{\begin{hindi}
वे बोले, "ऐ शुऐब! क्या तेरी नमाज़ तुझे यही सिखाती है कि उन्हें हम छोड़ दें जिन्हें हमारे बाप-दादा पूजते आए है या यह कि हम अपने माल का उपभोग अपनी इच्छानुसार न करें? बस एक तू ही तो बड़ा सहनशील, समझदार रह गया है!"
\end{hindi}}
\flushright{\begin{Arabic}
\quranayah[11][88]
\end{Arabic}}
\flushleft{\begin{hindi}
उसने कहा, "ऐ मेरी क़ौम के लोगो! तुम्हारा क्या विचार है? यदि मैं अपने रब के एक स्पष्ट प्रमाण पर हूँ और उसने मुझे अपनी ओर से अच्छी आजीविका भी प्रदान की (तो झुठलाना मेरे लिए कितना हानिकारक होगा!) और मैं नहीं चाहता कि जिन बातों से मैं तुम्हें रोकता हूँ स्वयं स्वयं तुम्हारे विपरीत उनको करने लगूँ। मैं तो अपने बस भर केवल सुधार चाहता हूँ। मेरा काम बनना तो अल्लाह ही की सहायता से सम्भव है। उसी पर मेरा भरोसा है और उसी की ओर मैं रुजू करता हूँ
\end{hindi}}
\flushright{\begin{Arabic}
\quranayah[11][89]
\end{Arabic}}
\flushleft{\begin{hindi}
ऐ मेरी क़ौम के लोगो! मेरे प्रति तुम्हारा विरोध कहीं तुम्हें उस अपराध पर न उभारे कि तुमपर वही बीते जो नूह की क़ौम या हूद की क़ौम या सालेह की क़ौम पर बीत चुका है, और लूत की क़ौम तो तुमसे कुछ दूर भी नहीं।
\end{hindi}}
\flushright{\begin{Arabic}
\quranayah[11][90]
\end{Arabic}}
\flushleft{\begin{hindi}
अपने रब से क्षमा माँगो और फिर उसकी ओर पलट आओ। मेरा रब तो बड़ा दयावन्त, बहुत प्रेम करनेवाला हैं।"
\end{hindi}}
\flushright{\begin{Arabic}
\quranayah[11][91]
\end{Arabic}}
\flushleft{\begin{hindi}
उन्होंने कहा, "ऐ शुऐब! तेरी बहुत-सी बातों को समझने में तो हम असमर्थ है। और हम तो तुझे देखते है कि तू हमारे मध्य अत्यन्त निर्बल है। यदि तेरे भाई-बन्धु न होते तो हम पत्थर मार-मारकर कभी का तुझे समाप्त कर चुके होते। तू इतने बल-बूतेवाला तो नहीं कि हमपर भारी हो।"
\end{hindi}}
\flushright{\begin{Arabic}
\quranayah[11][92]
\end{Arabic}}
\flushleft{\begin{hindi}
उसने कहा, "ऐ मेरी क़ौम के लोगो! क्या मेरे भाई-बन्धु तुमपर अल्लाह से भी ज़्यादा भारी है कि तुमने उसे अपने पीछे डाल दिया? तुम जो कुछ भी करते हो निश्चय ही मेरे रब ने उसे अपने घेरे में ले रखा है
\end{hindi}}
\flushright{\begin{Arabic}
\quranayah[11][93]
\end{Arabic}}
\flushleft{\begin{hindi}
ऐ मेरी क़ौम के लोगो! तुम अपनी जगह कर्म करते रहो, मैं भी कर रहा हूँ। शीघ्र ही तुमको ज्ञात हो जाएगा कि किसपर वह यातना आती है, जो उसे अपमानित करके रहेगी, और कौन है जो झूठा है! प्रतीक्षा करो, मैं भी तुम्हारे साथ प्रतीक्षा कर रहा हूँ।"
\end{hindi}}
\flushright{\begin{Arabic}
\quranayah[11][94]
\end{Arabic}}
\flushleft{\begin{hindi}
अन्ततः जब हमारा आदेश आ पहुँचा तो हमने अपनी दयालुता से शुऐब और उसके साथ के ईमान लानेवालों को बचा लिया। और अत्याचार करनेवालों को एक प्रचंड चिंघार ने आ लिया और वे अपने घरों में औंधे पड़े रह गए,
\end{hindi}}
\flushright{\begin{Arabic}
\quranayah[11][95]
\end{Arabic}}
\flushleft{\begin{hindi}
मानो वे वहाँ कभी बसे ही न थे। "सुन लो! फिटकार है मदयनवालों पर, जैसे समूद पर फिटकार हुई!"
\end{hindi}}
\flushright{\begin{Arabic}
\quranayah[11][96]
\end{Arabic}}
\flushleft{\begin{hindi}
और हमने मूसा को अपनी निशानियाँ और स्पष्ट प्रमाण के साथ
\end{hindi}}
\flushright{\begin{Arabic}
\quranayah[11][97]
\end{Arabic}}
\flushleft{\begin{hindi}
फ़िरऔन और उसके सरदारों के पास भेजा, किन्तु वे फ़िरऔन ही के कहने पर चले, हालाँकि फ़िरऔन की बात कोई ठीक बात न थी।
\end{hindi}}
\flushright{\begin{Arabic}
\quranayah[11][98]
\end{Arabic}}
\flushleft{\begin{hindi}
क़ियामत के दिन वह अपनी क़ौम के लोगों के आगे होगा - और उसने उन्हें आग में जा उतारा, और बहुत ही बुरा घाट है वह उतरने का!
\end{hindi}}
\flushright{\begin{Arabic}
\quranayah[11][99]
\end{Arabic}}
\flushleft{\begin{hindi}
यहाँ भी लानत ने उनका पीछा किया और क़ियामत के दिन भी - बहुत ही बुरा पुरस्कार है यह जो किसी को दिया जाए!
\end{hindi}}
\flushright{\begin{Arabic}
\quranayah[11][100]
\end{Arabic}}
\flushleft{\begin{hindi}
ये बस्तियों के कुछ वृत्तान्त हैं, जो हम तुम्हें सुना रहे है। इनमें कुछ तो खड़ी है और कुछ की फ़सल कट चुकी है
\end{hindi}}
\flushright{\begin{Arabic}
\quranayah[11][101]
\end{Arabic}}
\flushleft{\begin{hindi}
हमने उनपर अत्याचार नहीं किया, बल्कि उन्होंने स्वयं अपने आप पर अत्याचार किया। फिर जब तेरे रब का आदेश आ गया तो उसके वे पूज्य, जिन्हें वे अल्लाह से हटकर पुकारा करते थे, उनके कुछ भी काम न आ सके। उन्होंने विनाश के अतिरिक्त उनके लिए किसी और चीज़ में अभिवृद्धि नहीं की
\end{hindi}}
\flushright{\begin{Arabic}
\quranayah[11][102]
\end{Arabic}}
\flushleft{\begin{hindi}
तेरे रब की पकड़ ऐसी ही होती है, जब वह किसी ज़ालिम बस्ती को पकड़ता है। निस्संदेह उसकी पकड़ बड़ी दुखद, अत्यन्त कठोर होती है
\end{hindi}}
\flushright{\begin{Arabic}
\quranayah[11][103]
\end{Arabic}}
\flushleft{\begin{hindi}
निश्चय ही इसमें उस व्यक्ति के लिए एक निशानी है जो आख़िरत की यातना से डरता हो। वह एक ऐसा दिन होगा, जिसमें सारे ही लोग एकत्र किए जाएँगे और वह एक ऐसा दिन होगा, जिसमें सब कुछ आँखों के सामने होगा,
\end{hindi}}
\flushright{\begin{Arabic}
\quranayah[11][104]
\end{Arabic}}
\flushleft{\begin{hindi}
हम उसे केवल थोड़ी अवधि के लिए ही लग रहे है;
\end{hindi}}
\flushright{\begin{Arabic}
\quranayah[11][105]
\end{Arabic}}
\flushleft{\begin{hindi}
जिस दिन वह आएगा, तो उसकी अनुमति के बिना कोई व्यक्ति बात तक न कर सकेगा। फिर (मानवों में) कोई तो उनमें अभागा होगा और कोई भाग्यशाली
\end{hindi}}
\flushright{\begin{Arabic}
\quranayah[11][106]
\end{Arabic}}
\flushleft{\begin{hindi}
तो जो अभागे होंगे, वे आग में होंगे; जहाँ उन्हें आर्तनाद करना और फुँकार मारना है
\end{hindi}}
\flushright{\begin{Arabic}
\quranayah[11][107]
\end{Arabic}}
\flushleft{\begin{hindi}
वहाँ वे सदैव रहेंगे, जब तक आकाश और धरती स्थिर रहें, बात यह है कि तुम्हारे रब की इच्छा ही चलेगी। तुम्हारा रब जो चाहे करे
\end{hindi}}
\flushright{\begin{Arabic}
\quranayah[11][108]
\end{Arabic}}
\flushleft{\begin{hindi}
रहे वे जो भाग्यशाली होंगे तो वे जन्नत में होंगे, जहाँ वे सदैव रहेंगे जब तक आकाश और धरती स्थिर रहें। बात यह है कि तुम्हारे रब की इच्छा ही चलेगी। यह एक ऐसा उपहार है, जिसका सिलसिला कभी न टूटेगा
\end{hindi}}
\flushright{\begin{Arabic}
\quranayah[11][109]
\end{Arabic}}
\flushleft{\begin{hindi}
अतः जिनको ये पूज रहे है, उनके विषय में तुझे कोई संदेह न हो। ये तो बस उसी तरह पूजा किए जा रहे है, जिस तरह इससे पहले इनके बाप-दादा पूजा करते रहे हैं। हम तो इन्हें इनका हिस्सा बिना किसी कमी के पूरा-पूरा देनेवाले हैं
\end{hindi}}
\flushright{\begin{Arabic}
\quranayah[11][110]
\end{Arabic}}
\flushleft{\begin{hindi}
हम मूसा को भी किताब दे चुके है। फिर उसमें भी विभेद किया गया था। यदि तुम्हारे रब की ओर से एक बात पहले ही निश्चित न कर दी गई होती तो उनके बीच कभी का फ़ैसला कर दिया गया होता। ये उसकी ओर से असमंजस में डाल देनेवाले संदेह में पड़े हुए है
\end{hindi}}
\flushright{\begin{Arabic}
\quranayah[11][111]
\end{Arabic}}
\flushleft{\begin{hindi}
निश्चय ही समय आने पर एक-एक को, जितने भी है उनको तुम्हारा रब उनका किया पूरा-पूरा देकर रहेगा। वे जो कुछ कर रहे हैं, निस्संदेह उसे उसकी पूरी ख़बर है
\end{hindi}}
\flushright{\begin{Arabic}
\quranayah[11][112]
\end{Arabic}}
\flushleft{\begin{hindi}
अतः जैसा तुम्हें आदेश हुआ है, जमें रहो और तुम्हारे साथ के तौबा करनेवाले भी जमें रहें, और सीमोल्लंघन न करना। जो कुछ भी तुम करते हो, निश्चय ही वह उसे देख रहा है
\end{hindi}}
\flushright{\begin{Arabic}
\quranayah[11][113]
\end{Arabic}}
\flushleft{\begin{hindi}
उन लोगों की ओर तनिक भी न झुकना, जिन्होंने अत्याचार की नीति अपनाई हैं, अन्यथा आग तुम्हें आ लिपटेगी - और अल्लाह से हटकर तुम्हारा कोई संरक्षक मित्र नहीं - फिर तुम्हें कोई सहायता भी न मिलेगी
\end{hindi}}
\flushright{\begin{Arabic}
\quranayah[11][114]
\end{Arabic}}
\flushleft{\begin{hindi}
और नमाज़ क़ायम करो दिन के दोनों सिरों पर और रात के कुछ हिस्से में। निस्संदेह नेकियाँ बुराइयों को दूर कर देती है। यह याद रखनेवालों के लिए एक अनुस्मरण है
\end{hindi}}
\flushright{\begin{Arabic}
\quranayah[11][115]
\end{Arabic}}
\flushleft{\begin{hindi}
और धैर्य से काम लो, इसलिए कि अल्लाह सुकर्मियों को बदला अकारथ नहीं करता;
\end{hindi}}
\flushright{\begin{Arabic}
\quranayah[11][116]
\end{Arabic}}
\flushleft{\begin{hindi}
फिर तुमसे पहले जो नस्लें गुज़र चुकी है उनमें ऐसे भले-समझदार क्यों न हुए जो धरती में बिगाड़ से रोकते, उन थोड़े-से लोगों के सिवा जिनको उनमें से हमने बचा लिया। अत्याचारी लोग तो उसी सुख-सामग्री के पीछे पड़े रहे, जिसमें वे रखे गए थे। वे तो थे ही अपराधी
\end{hindi}}
\flushright{\begin{Arabic}
\quranayah[11][117]
\end{Arabic}}
\flushleft{\begin{hindi}
तुम्हारा रब तो ऐसा नहीं है कि बस्तियों को अकारण विनष्ट कर दे, जबकि वहाँ के निवासी बनाव और सुधार में लगे हों
\end{hindi}}
\flushright{\begin{Arabic}
\quranayah[11][118]
\end{Arabic}}
\flushleft{\begin{hindi}
और यदि तुम्हारा रब चाहता तो वह सारे मनुष्यों को एक समुदाय बना देता, किन्तु अब तो वे सदैव विभेद करते ही रहेंगे,
\end{hindi}}
\flushright{\begin{Arabic}
\quranayah[11][119]
\end{Arabic}}
\flushleft{\begin{hindi}
सिवाय उनके जिनपर तुम्हारा रब दया करे और इसी के लिए उसने उन्हें पैदा किया है, और तुम्हारे रब की यह बात पूरी होकर रही कि "मैं जहन्नम को अपराधी जिन्नों और मनुष्यों सबसे भरकर रहूँगा।"
\end{hindi}}
\flushright{\begin{Arabic}
\quranayah[11][120]
\end{Arabic}}
\flushleft{\begin{hindi}
रसूलों के वृत्तान्तों में से हर वह कथा जो हम तुम्हें सुनाते है उसके द्वारा हम तुम्हारे हृदय को सुदृढ़ करते हैं। और इसमें तुम्हारे पास सत्य आ गया है और मोमिनों के लिए उपदेश और अनुस्मरण भी
\end{hindi}}
\flushright{\begin{Arabic}
\quranayah[11][121]
\end{Arabic}}
\flushleft{\begin{hindi}
जो लोग ईमान नहीं ला रहे हैं उनसे कह दो, "तुम अपनी जगह कर्म किए जाओ, हम भी कर्म कर रहे है
\end{hindi}}
\flushright{\begin{Arabic}
\quranayah[11][122]
\end{Arabic}}
\flushleft{\begin{hindi}
तुम भी प्रतीक्षा करो, हम भी प्रतीक्षा कर रहे है।"
\end{hindi}}
\flushright{\begin{Arabic}
\quranayah[11][123]
\end{Arabic}}
\flushleft{\begin{hindi}
अल्लाह ही का है जो कुछ आकाशों और धरती में छिपा है, और हर मामला उसी की ओर पलटता है। अतः उसी की बन्दगी करो और उसी पर भरोसा रखो। जो कुछ तुम करते हो, उससे तुम्हारा रब बेख़बर नहीं है
\end{hindi}}
\chapter{Yusuf (Joseph)}
\begin{Arabic}
\Huge{\centerline{\basmalah}}\end{Arabic}
\flushright{\begin{Arabic}
\quranayah[12][1]
\end{Arabic}}
\flushleft{\begin{hindi}
अलिफ़॰ लाम॰ रा॰। ये स्पष्ट किताब की आयतें हैं
\end{hindi}}
\flushright{\begin{Arabic}
\quranayah[12][2]
\end{Arabic}}
\flushleft{\begin{hindi}
हमने इसे अरबी क़ुरआन के रूप में उतारा है, ताकि तुम समझो
\end{hindi}}
\flushright{\begin{Arabic}
\quranayah[12][3]
\end{Arabic}}
\flushleft{\begin{hindi}
इस क़ुरआन की तुम्हारी ओर प्रकाशना करके इसके द्वारा हम तुम्हें एक बहुत ही अच्छा बयान सुनाते है, यद्यपि इससे पहले तुम बेख़बर थे
\end{hindi}}
\flushright{\begin{Arabic}
\quranayah[12][4]
\end{Arabic}}
\flushleft{\begin{hindi}
जब यूसुफ़ ने अपने बाप से कहा, "ऐ मेरे बाप! मैंने स्वप्न में ग्यारह सितारे देखे और सूर्य और चाँद। मैंने उन्हें देखा कि वे मुझे सजदा कर रहे है।"
\end{hindi}}
\flushright{\begin{Arabic}
\quranayah[12][5]
\end{Arabic}}
\flushleft{\begin{hindi}
उसने कहा, "ऐ मेरे बेटे! अपना स्वप्न अपने भाइयों को मत बताना, अन्यथा वे तेरे विरुद्ध कोई चाल चलेंगे। शैतान तो मनुष्य का खुला हुआ शत्रु है
\end{hindi}}
\flushright{\begin{Arabic}
\quranayah[12][6]
\end{Arabic}}
\flushleft{\begin{hindi}
और ऐसा ही होगा, तेरा रब तुझे चुन लेगा और तुझे बातों की तथ्य तक पहुँचना सिखाएगा और अपना अनुग्रह तुझपर और याकूब के घरवालों पर उसी प्रकार पूरा करेगा, जिस प्रकार इससे पहले वह तेरे पूर्वज इबराहीम और इसहाक़ पर पूरा कर चुका है। निस्संदेह तेरा रब सर्वज्ञ, तत्वदर्शी है।"
\end{hindi}}
\flushright{\begin{Arabic}
\quranayah[12][7]
\end{Arabic}}
\flushleft{\begin{hindi}
निश्चय ही यूसुफ़ और उनके भाइयों में सवाल करनेवालों के लिए निशानियाँ है
\end{hindi}}
\flushright{\begin{Arabic}
\quranayah[12][8]
\end{Arabic}}
\flushleft{\begin{hindi}
जबकि उन्होंने कहा, "यूसुफ़ और उसका भाई हमारे बाप को हमसे अधिक प्रिय है, हालाँकि हम एक पूरा जत्था है। वासत्व में हमारे बाप स्पष्ट तः बहक गए है
\end{hindi}}
\flushright{\begin{Arabic}
\quranayah[12][9]
\end{Arabic}}
\flushleft{\begin{hindi}
यूसुफ़ को मार डालो या उसे किसी भूभाग में फेंक आओ, ताकि तुम्हारे बाप का ध्यान केवल तुम्हारी ही ओर हो जाए। इसके पश्चात तुम फिर नेक बन जाना।"
\end{hindi}}
\flushright{\begin{Arabic}
\quranayah[12][10]
\end{Arabic}}
\flushleft{\begin{hindi}
उनमें से एक बोलनेवाला बोल पड़ा, "यूसुफ़ की हत्या न करो, यदि तुम्हें कुछ करना ही है तो उसे किसी कुएँ की तह में डाल दो। कोई राहगीर उसे उठा लेगा।"
\end{hindi}}
\flushright{\begin{Arabic}
\quranayah[12][11]
\end{Arabic}}
\flushleft{\begin{hindi}
उन्होंने कहा, "ऐ हमारे बाप! आपको क्या हो गया है कि यूसुफ़ के मामले में आप हमपर भरोसा नहीं करते, हालाँकि हम तो उसके हितैषी है?
\end{hindi}}
\flushright{\begin{Arabic}
\quranayah[12][12]
\end{Arabic}}
\flushleft{\begin{hindi}
हमारे साथ कल उसे भेज दीजिए कि वह कुछ चर-चुग और खेल ले। उसकी रक्षा के लिए तो हम हैं ही।"
\end{hindi}}
\flushright{\begin{Arabic}
\quranayah[12][13]
\end{Arabic}}
\flushleft{\begin{hindi}
उसने कहा, यह बात कि तुम उसे ले जाओ, मुझे दुखी कर देती है। कहीं ऐसा न हो कि तुम उसका ध्यान न रख सको और भेड़िया उसे खा जाए।"
\end{hindi}}
\flushright{\begin{Arabic}
\quranayah[12][14]
\end{Arabic}}
\flushleft{\begin{hindi}
वे बोले, "हमारे एक जत्थे के होते हुए भी यदि उसे भेड़िए ने खा लिया, तब तो निश्चय ही हम सब कुछ गँवा बैठे।"
\end{hindi}}
\flushright{\begin{Arabic}
\quranayah[12][15]
\end{Arabic}}
\flushleft{\begin{hindi}
फिर जब वे उसे ले गए और सभी इस बात पर सहमत हो गए कि उसे एक कुएँ की गहराई में डाल दें (तो उन्होंने वह किया जो करना चाहते थे), और हमने उसकी ओर प्रकाशना का, "तू उन्हें उनके इस कर्म से अवगत कराएगा और वे जानते न होंगे।"
\end{hindi}}
\flushright{\begin{Arabic}
\quranayah[12][16]
\end{Arabic}}
\flushleft{\begin{hindi}
कुछ रात बीते वे रोते हुए अपने बाप के पास आए
\end{hindi}}
\flushright{\begin{Arabic}
\quranayah[12][17]
\end{Arabic}}
\flushleft{\begin{hindi}
कहने लगे, "ऐ मेरे बाप! हम परस्पर दौड़ में मुक़ाबला करते हुए दूर चले गए और यूसफ़ को हमने अपने सामान के साथ छोड़ दिया था कि इतने में भेड़िया उसे खा गया। आप तो हमपर विश्वास करेंगे नहीं, यद्यपि हम सच्चे है।"
\end{hindi}}
\flushright{\begin{Arabic}
\quranayah[12][18]
\end{Arabic}}
\flushleft{\begin{hindi}
वे उसके कुर्ते पर झूठमूठ का ख़ून लगा लाए थे। उसने कहा, "नहीं, बल्कि तुम्हारे जी ने बहकाकर तुम्हारे लिए एक बात बना दी है। अब धैर्य से काम लेना ही उत्तम है! जो बात तुम बता रहे हो उसमें अल्लाह ही सहायक हो सकता है।"
\end{hindi}}
\flushright{\begin{Arabic}
\quranayah[12][19]
\end{Arabic}}
\flushleft{\begin{hindi}
एक क़ाफ़िला आया। फिर उसने पनिहारा को भेजा। उसने अपना डोल ज्यों ही डाला तो पुकार उठा, "अरे! कितनी ख़ुशी की बात है। यह तो एक लड़का है।" उन्होंने उसे व्यापार का माल समझकर छुपा लिया। किन्तु जो कुछ वे कर रहे थे, अल्लाह तो उसे जानता ही था
\end{hindi}}
\flushright{\begin{Arabic}
\quranayah[12][20]
\end{Arabic}}
\flushleft{\begin{hindi}
उन्होंने उसे सस्ते दाम, गिनती के कुछ दिरहमों में बेच दिया, क्योंकि वे उसके मामलें में बेपरवाह थे
\end{hindi}}
\flushright{\begin{Arabic}
\quranayah[12][21]
\end{Arabic}}
\flushleft{\begin{hindi}
मिस्र के जिस व्यक्ति ने उसे ख़रीदा, उसने अपनी स्त्री से कहा, "इसको अच्छी तरह रखना। बहुत सम्भव है कि यह हमारे काम आए या हम इसे बेटा बना लें।" इस प्रकार हमने उस भूभाग में यूसुफ़ के क़दम जमाने की राह निकाली (ताकि उसे प्रतिष्ठा प्रदान करें) और ताकि मामलों और बातों के परिणाम से हम उसे अवगत कराएँ। अल्लाह तो अपना काम करके रहता है, किन्तु अधिकतर लोग जानते नहीं
\end{hindi}}
\flushright{\begin{Arabic}
\quranayah[12][22]
\end{Arabic}}
\flushleft{\begin{hindi}
और जब वह अपनी जवानी को पहुँचा तो हमने उसे निर्णय-शक्ति और ज्ञान प्रदान किया। उत्तमकार लोगों को हम इसी प्रकार बदला देते है
\end{hindi}}
\flushright{\begin{Arabic}
\quranayah[12][23]
\end{Arabic}}
\flushleft{\begin{hindi}
जिस स्त्री के घर में वह रहता था, वह उस पर डोरे डालने लगी। उसने दरवाज़े बन्द कर दिए और कहने लगी, "लो, आ जाओ!" उसने कहा, "अल्लाह की पनाह! मेरे रब ने मुझे अच्छा स्थान दिया है। अत्याचारी कभी सफल नहीं होते।"
\end{hindi}}
\flushright{\begin{Arabic}
\quranayah[12][24]
\end{Arabic}}
\flushleft{\begin{hindi}
उसने उसका इरादा कर लिया था। यदि वह अपने रब का स्पष्ट॥ प्रमाण न देख लेता तो वह भी उसका इरादा कर लेता। ऐसा इसलिए हुआ ताकि हम बुराई और अश्लीलता को उससे दूर रखें। निस्संदेह वह हमारे चुने हुए बन्दों में से था
\end{hindi}}
\flushright{\begin{Arabic}
\quranayah[12][25]
\end{Arabic}}
\flushleft{\begin{hindi}
वे दोनों दरवाज़े की ओर झपटे और उस स्त्री ने उसका कुर्ता पीछे से फाड़ डाला। दरवाज़े पर दोनों ने उस स्त्री के पति को उपस्थित पाया। वह बोली, "जो कोई तुम्हारी घरवाली के साथ बुरा इरादा करे, उसका बदला इसके सिवा और क्या होगा कि उसे बन्दी बनाया जाए या फिर कोई दुखद यातना दी जाए?"
\end{hindi}}
\flushright{\begin{Arabic}
\quranayah[12][26]
\end{Arabic}}
\flushleft{\begin{hindi}
उसने कहा, "यही मुझपर डोरे डाल रही थी।" उस स्त्री के लोगों में से एक गवाह ने गवाही दी, "यदि इसका कुर्ता आगे से फटा है तो यह सच्ची है और यह झूठा है,
\end{hindi}}
\flushright{\begin{Arabic}
\quranayah[12][27]
\end{Arabic}}
\flushleft{\begin{hindi}
और यदि उसका कुर्ता पीछे से फटा है तो यह झूठी है और यह सच्चा है।"
\end{hindi}}
\flushright{\begin{Arabic}
\quranayah[12][28]
\end{Arabic}}
\flushleft{\begin{hindi}
फिर जब देखा कि उसका कुर्ता पीछे से फटा है तो उसने कहा, "यह तुम स्त्रियों की चाल है। निश्चय ही तुम्हारी चाल बड़े ग़ज़ब की होती है
\end{hindi}}
\flushright{\begin{Arabic}
\quranayah[12][29]
\end{Arabic}}
\flushleft{\begin{hindi}
यूसुफ़! इस मामले को जाने दे और स्त्री तू अपने गुनाह की माफ़ी माँग। निस्संदेह ख़ता तेरी ही है।"
\end{hindi}}
\flushright{\begin{Arabic}
\quranayah[12][30]
\end{Arabic}}
\flushleft{\begin{hindi}
नगर की स्त्रियाँ कहने लगी, "अज़ीज़ की पत्नी अपने नवयुवक ग़ुलाम पर डोरे डालना चाहती है। वह प्रेम-प्रेरणा से उसके मन में घर कर गया है। हम तो उसे देख रहे हैं कि वह खुली ग़लती में पड़ गई है।"
\end{hindi}}
\flushright{\begin{Arabic}
\quranayah[12][31]
\end{Arabic}}
\flushleft{\begin{hindi}
उसने जब उनकी मक्करी की बातें सुनी तो उन्हें बुला भेजा और उनमें से हरेक के लिए आसन सुसज्जित किया और उनमें से हरेक को एक छुरी दी। उसने (यूसुफ़ से) कहा, "इनके सामने आ जाओ।" फिर जब स्त्रियों ने देखा तो वे उसकी बड़ाई से दंग रह गई। उन्होंने अपने हाथ घायल कर लिए और कहने लगी, "अल्लाह की पनाह! यह कोई मनुष्य नहीं। यह तो कोई प्रतिष्ठित फ़रिश्ता है।‍"
\end{hindi}}
\flushright{\begin{Arabic}
\quranayah[12][32]
\end{Arabic}}
\flushleft{\begin{hindi}
वह बोली, "यह वही है जिसके विषय में तुम मुझे मलामत कर रही थीं। हाँ, मैंने इसे रिझाना चाहा था, किन्तु यह बचा रहा। और यदि इसने न किया जो मैं इससे कहती तो यह अवश्य क़ैद किया जाएगा और अपमानित होगा।"
\end{hindi}}
\flushright{\begin{Arabic}
\quranayah[12][33]
\end{Arabic}}
\flushleft{\begin{hindi}
उसने कहा, "ऐ मेरे रब! जिसकी ओर ये सब मुझे बुला रही हैं, उससे अधिक तो मुझे क़ैद ही पसन्द है यदि तूने उनके दाँव-घात को मुझसे न टाला तो मैं उनकी और झुक जाऊँगा और निरे आवेग के वशीभूत हो जाऊँगा।"
\end{hindi}}
\flushright{\begin{Arabic}
\quranayah[12][34]
\end{Arabic}}
\flushleft{\begin{hindi}
अतः उसने रब ने उसकी सुन ली और उसकी ओर से उन स्त्रियों के दाँव-घात को टाल दिया। निस्संदेह वह सब कुछ सुनता, जानता है
\end{hindi}}
\flushright{\begin{Arabic}
\quranayah[12][35]
\end{Arabic}}
\flushleft{\begin{hindi}
फिर उन्हें, इसके पश्चात कि वे निशानियाँ देख चुके थे, यह सूझा कि उसे एक अवधि के लिए क़ैद कर दें
\end{hindi}}
\flushright{\begin{Arabic}
\quranayah[12][36]
\end{Arabic}}
\flushleft{\begin{hindi}
कारागार में दो नव युवकों ने भी उसके साथ प्रवेश किया। उनमें से एक ने कहा, "मैंने स्वप्न देखा है कि मैं शराब निचोड़ रहा हूँ।" दूसरे ने कहा, "मैंने देखा कि मैं अपने सिर पर रोटियाँ उठाए हुए हूँ, जिनको पक्षी खा रहे है। हमें इसका अर्थ बता दीजिए। हमें तो आप बहुत ही नेक नज़र आते है।"
\end{hindi}}
\flushright{\begin{Arabic}
\quranayah[12][37]
\end{Arabic}}
\flushleft{\begin{hindi}
उसने कहा, "जो भोजन तुम्हें मिला करता है वह तुम्हारे पास नहीं आ पाएगा, उसके तुम्हारे पास आने से पहले ही मैं तुम्हें इसका अर्थ बता दूँगा। यह उन बातों में से है, जो मेरे रब ने मुझे सिखाई है। मैं तो उन लोगों का तरीक़ा छोड़कर, जो अल्लाह को नहीं मानते और जो आख़िरत (परलोक) का इनकार करते हैं,
\end{hindi}}
\flushright{\begin{Arabic}
\quranayah[12][38]
\end{Arabic}}
\flushleft{\begin{hindi}
अपने पूर्वज इबराहीम, इसहाक़ और याक़ूब का तरीक़ा अपनाया है। इमसे यह नहीं हो सकता कि हम अल्लाह के साथ किसी चीज़ को साझी ठहराएँ। यह हमपर और लोगों पर अल्लाह का अनुग्रह है। किन्तु अधिकतर लोग आभार नहीं प्रकट करते
\end{hindi}}
\flushright{\begin{Arabic}
\quranayah[12][39]
\end{Arabic}}
\flushleft{\begin{hindi}
ऐ कारागर के मेरे साथियों! क्या अलग-अलग बहुत-से रह अच्छे है या अकेला अल्लाह जिसका प्रभुत्व सबपर है?
\end{hindi}}
\flushright{\begin{Arabic}
\quranayah[12][40]
\end{Arabic}}
\flushleft{\begin{hindi}
तुम उसके सिवा जिनकी भी बन्दगी करते हो वे तो बस निरे नाम हैं जो तुमने रख छोड़े है और तुम्हारे बाप-दादा ने। उनके लिए अल्लाह ने कोई प्रमाण नहीं उतारा। सत्ता और अधिकार तो बस अल्लाह का है। उसने आदेश दिया है कि उसके सिवा किसी की बन्दगी न करो। यही सीधा, सच्चा दीन (धर्म) हैं, किन्तु अधिकतर लोग नहीं जानते
\end{hindi}}
\flushright{\begin{Arabic}
\quranayah[12][41]
\end{Arabic}}
\flushleft{\begin{hindi}
ऐ कारागार के मेरे दोनों साथियों! तुममें से एक तो अपने स्वामी को मद्यपान कराएगा; रहा दूसरा तो उसे सूली पर चढ़ाया जाएगा और पक्षी उसका सिर खाएँगे। फ़ैसला हो चुका उस बात का जिसके विषय में तुम मुझसे पूछ रहे हो।"
\end{hindi}}
\flushright{\begin{Arabic}
\quranayah[12][42]
\end{Arabic}}
\flushleft{\begin{hindi}
उन दोनों में से जिसके विषय में उसने समझा था कि वह रिहा हो जाएगा, उससे कहा, "अपने स्वामी से मेरी चर्चा करना।" किन्तु शैतान ने अपने स्वामी से उसकी चर्चा करना भुलवा दिया। अतः वह (यूसुफ़) कई वर्ष तक कारागार ही में रहा
\end{hindi}}
\flushright{\begin{Arabic}
\quranayah[12][43]
\end{Arabic}}
\flushleft{\begin{hindi}
फिर ऐसा हुआ कि सम्राट ने कहा, "मैं एक स्वप्न देखा कि सात मोटी गायों को सात दुबली गायें खा रही है और सात बालें हरी है और दूसरी (सात सूखी) । ऐ सरदारों! यदि तुम स्वप्न का अर्थ बताते हो, तो मुझे मेरे इस स्वप्न के सम्बन्ध में बताओ।"
\end{hindi}}
\flushright{\begin{Arabic}
\quranayah[12][44]
\end{Arabic}}
\flushleft{\begin{hindi}
उन्होंने कहा, "ये तो सम्भ्रमित स्वप्न है। हम ऐसे स्वप्न का अर्थ नहीं जानते।"
\end{hindi}}
\flushright{\begin{Arabic}
\quranayah[12][45]
\end{Arabic}}
\flushleft{\begin{hindi}
इतने में दोनों में सो जो रिहा हो गया था और एक अर्से के बाद उसे याद आया तो वह बोला, "मैं इसका अर्थ तुम्हें बताता हूँ। ज़रा मुझे (यूसुफ़ के पास) भेज दीजिए।"
\end{hindi}}
\flushright{\begin{Arabic}
\quranayah[12][46]
\end{Arabic}}
\flushleft{\begin{hindi}
"यूसुफ़, ऐ सत्यवान! हमें इसका अर्थ बता कि सात मोटी गायें है, जिन्हें सात दुबली गायें खा रही है और सात हरी बालें है और दूसरी (सात) सूखी, ताकि मैं लोगों के पास लौटकर जाऊँ कि वे जान लें।"
\end{hindi}}
\flushright{\begin{Arabic}
\quranayah[12][47]
\end{Arabic}}
\flushleft{\begin{hindi}
उसने कहा, "सात वर्ष तक तुम व्यवहारतः खेती करते रहोगे। फिर तुम जो फ़सल काटो तो थोड़े हिस्से के सिवा जो तुम्हारे खाने के काम आए शेष को उसकी बाली में रहने देना
\end{hindi}}
\flushright{\begin{Arabic}
\quranayah[12][48]
\end{Arabic}}
\flushleft{\begin{hindi}
फिर उसके पश्चात सात कठिन वर्ष आएँगे जो वे सब खा जाएँगे जो तुमने उनके लिए पहले से इकट्ठा कर रखा होगा, सिवाय उस थोड़े-से हिस्से के जो तुम सुरक्षित कर लोगे
\end{hindi}}
\flushright{\begin{Arabic}
\quranayah[12][49]
\end{Arabic}}
\flushleft{\begin{hindi}
फिर उसके पश्चात एक वर्ष ऐसा आएगा, जिसमें वर्षा द्वारा लोगों की फ़रियाद सुन ली जाएगी और उसमें वे रस निचोड़ेगे।"
\end{hindi}}
\flushright{\begin{Arabic}
\quranayah[12][50]
\end{Arabic}}
\flushleft{\begin{hindi}
सम्राट ने कहा, "उसे मेरे पास ले आओ।" किन्तु जब दूत उसके पास पहुँचा तो उसने कहा, "अपने स्वामी के पास वापस जाओ और उससे पूछो कि उन स्त्रियों का क्या मामला है, जिन्होंने अपने हाथ घायल कर लिए थे। निस्संदेह मेरा रब उनकी मक्कारी को भली-भाँति जानता है।"
\end{hindi}}
\flushright{\begin{Arabic}
\quranayah[12][51]
\end{Arabic}}
\flushleft{\begin{hindi}
उसने कहा, "तुम स्त्रियों का क्या हाल था, जब तुमने यूसुफ़ को रिझाने की चेष्टा की थी?" उन्होंने कहा, "पाक है अल्लाह! हम उसमें कोई बुराई नहीं जानते है।" अज़ीज़ की स्त्री बोल उठी, "अब तो सत्य प्रकट हो गया है। मैंने ही उसे रिझाना चाहा था। वह तो बिलकुल सच्चा है।"
\end{hindi}}
\flushright{\begin{Arabic}
\quranayah[12][52]
\end{Arabic}}
\flushleft{\begin{hindi}
"यह इसलिए कि वह जान ले कि मैंने गुप्त॥ रूप से उसके साथ विश्वासघात नहीं किया और यह कि अल्लाह विश्वासघातियों का चाल को चलने नहीं देता
\end{hindi}}
\flushright{\begin{Arabic}
\quranayah[12][53]
\end{Arabic}}
\flushleft{\begin{hindi}
मैं यह नहीं कहता कि मैं बुरी हूँ - जी तो बुराई पर उभारता ही है - यदि मेरा रब ही दया करे तो बात और है। निश्चय ही मेरा रब बहुत क्षमाशील, दयावान है।"
\end{hindi}}
\flushright{\begin{Arabic}
\quranayah[12][54]
\end{Arabic}}
\flushleft{\begin{hindi}
सम्राट ने कहा, "उसे मेरे पास ले आओ! मैं उसे अपने लिए ख़ास कर लूँगा।" जब उसने उससे बात-चीक करी तो उसने कहा, "निस्संदेह आज तुम हमारे यहाँ विश्व सनीय अधिकार प्राप्त व्यक्ति हो।"
\end{hindi}}
\flushright{\begin{Arabic}
\quranayah[12][55]
\end{Arabic}}
\flushleft{\begin{hindi}
उसने कहा, "इस भू-भाग के ख़जानों पर मुझे नियुक्त कर दीजिए। निश्चय ही मैं रक्षक और ज्ञानवान हूँ।"
\end{hindi}}
\flushright{\begin{Arabic}
\quranayah[12][56]
\end{Arabic}}
\flushleft{\begin{hindi}
इस प्रकार हमने यूसुफ़ को उस भू-भाग में अधिकार प्रदान किया कि वह उसमें जहाँ चाहे अपनी जगह बनाए। हम जिसे चाहते हैं उसे अपनी दया का पात्र बनाते है। उत्तमकारों का बदला हम अकारथ नहीं जाने देते
\end{hindi}}
\flushright{\begin{Arabic}
\quranayah[12][57]
\end{Arabic}}
\flushleft{\begin{hindi}
और ईमान लानेवालों और डर रखनेवालों के लिए आख़िरत का बदला इससे कहीं उत्तम है
\end{hindi}}
\flushright{\begin{Arabic}
\quranayah[12][58]
\end{Arabic}}
\flushleft{\begin{hindi}
फिर ऐसा हुआ कि यूसुफ़ के भाई आए और उसके सामने उपस्थित हुए, उसने तो उन्हें पहचान लिया, किन्तु वे उससे अपरिचित रहे
\end{hindi}}
\flushright{\begin{Arabic}
\quranayah[12][59]
\end{Arabic}}
\flushleft{\begin{hindi}
जब उसने उनके लिए उनका सामान तैयार करा दिया तो कहा, "बाप की ओर सो तुम्हारा भाई है, उसे मेरे पास लाना। क्या देखते नहीं कि मैं पूरी माप से देता हूँ और मैं अच्छा आतिशेय भी हूँ?"
\end{hindi}}
\flushright{\begin{Arabic}
\quranayah[12][60]
\end{Arabic}}
\flushleft{\begin{hindi}
किन्तु यदि तुम उसे मेरे पास न लाए तो फिर तुम्हारे लिए मेरे यहाँ कोई माप (ग़ल्ला) नहीं और तुम मेरे पास आना भी मत।"
\end{hindi}}
\flushright{\begin{Arabic}
\quranayah[12][61]
\end{Arabic}}
\flushleft{\begin{hindi}
वे बोले, "हम उसके लिए उसके बाप को राज़ी करने की कोशिश करेंगे और हम यह काम अवश्य करेंगे।"
\end{hindi}}
\flushright{\begin{Arabic}
\quranayah[12][62]
\end{Arabic}}
\flushleft{\begin{hindi}
उसने अपने सेवकों से कहा, "इनका दिया हुआ माल इनके सामान में रख दो कि जब ये अपने घरवालों की ओर लौटें तो इसे पहचान लें, ताकि ये फिर लौटकर आएँ।"
\end{hindi}}
\flushright{\begin{Arabic}
\quranayah[12][63]
\end{Arabic}}
\flushleft{\begin{hindi}
फिर जब वे अपने बाप के पास लौटकर गए तो कहा, "ऐ मेरे बाप! (अनाज की) माप हमसे रोक दी गई है। अतः हमारे भाई को हमारे साथ भेज दीजिए, ताकि हम माप भर लाएँ; और हम उसकी रक्षा के लिए तो मौजूद ही हैं।"
\end{hindi}}
\flushright{\begin{Arabic}
\quranayah[12][64]
\end{Arabic}}
\flushleft{\begin{hindi}
उसने कहा, "क्या मैं उसके मामले में तुमपर वैसा ही भरोसा करूँ जैसा इससे पहले उसके भाई के मामले में तुमपर भरोसा कर चुका हूँ? हाँ, अल्लाह ही सबसे अच्छ रक्षक है और वह सबसे बढ़कर दयावान है।"
\end{hindi}}
\flushright{\begin{Arabic}
\quranayah[12][65]
\end{Arabic}}
\flushleft{\begin{hindi}
जब उन्होंने अपना सामान खोला, तो उन्होंने अपने माल अपनी ओर वापस किया हुआ पाया। वे बोले, "ऐ मेरे बाप, हमें और क्या चाहिए! यह हमारा माल भी तो हमें लौटा दिया गया है। अब हम अपने घरवालों के लिए खाद्य-सामग्री लाएँगे और अपने भाई की रक्षा भी करेंगे। और एक ऊँट के बोझभर और अधिक लेंगे। इतना माप (ग़ल्ला) मिल जाना तो बिलकुल आसान है।"
\end{hindi}}
\flushright{\begin{Arabic}
\quranayah[12][66]
\end{Arabic}}
\flushleft{\begin{hindi}
उसने कहा, "मैं उसे तुम्हारे साथ कदापि नहीं भेज सकता। जब तक कि तुम अल्लाह को गवाह बनाकर मुझे पक्का वचन न दो कि तुम उसे मेरे पास अवश्य लाओगे, यह और बात है कि तुम घिर जाओ।" फिर जब उन्होंने उसे अपना वचन दे दिया तो उसने कहा, "हम जो कुछ कर रहे है वह अल्लाह के हवाले है।"
\end{hindi}}
\flushright{\begin{Arabic}
\quranayah[12][67]
\end{Arabic}}
\flushleft{\begin{hindi}
उसने यह भी कहा, "ऐ मेरे बेटो! एक द्वार से प्रवेश न करना, बल्कि विभिन्न द्वारों से प्रवेश करना यद्यपि मैं अल्लाह के मुक़ाबले में तुम्हारे काम नहीं आ सकता आदेश तो बस अल्लाह ही का चलता है। उसी पर मैंने भरोसा किया और भरोसा करनेवालों को उसी पर भरोसा करना चाहिए।"
\end{hindi}}
\flushright{\begin{Arabic}
\quranayah[12][68]
\end{Arabic}}
\flushleft{\begin{hindi}
और जब उन्होंने प्रवेश किया जिस तरह से उनके बाप ने उन्हें आदेश दिया था - अल्लाह की ओर से होनेवाली किसी चीज़ को वह उनसे हटा नहीं सकता था। बस याक़ूब के जी की एक इच्छा थी, जो उसने पूरी कर ली। और निस्संदेह वह ज्ञानवान था, क्योंकि हमने उसे ज्ञान प्रदान किया था; किन्तु अधिकतर लोग जानते नहीं -
\end{hindi}}
\flushright{\begin{Arabic}
\quranayah[12][69]
\end{Arabic}}
\flushleft{\begin{hindi}
और जब उन्होंने यूसुफ़ के यहाँ प्रवेश किया तो उसने अपने भाई को अपने पास जगह दी और कहा, "मैं तेरा भाई हूँ। जो कुछ ये करते रहे हैं, अब तू उसपर दुखी न हो।"
\end{hindi}}
\flushright{\begin{Arabic}
\quranayah[12][70]
\end{Arabic}}
\flushleft{\begin{hindi}
फिर जब उनका सामान तैयार कर दिया तो अपने भाई के सामान में पानी पीने का प्याला रख दिया। फिर एक पुकारनेवाले ने पुकारकर कहा, "ऐ क़ाफ़िलेवालो! निश्चय ही तुम चोर हो।"
\end{hindi}}
\flushright{\begin{Arabic}
\quranayah[12][71]
\end{Arabic}}
\flushleft{\begin{hindi}
वे उनकी ओर रुख़ करते हुए बोले, "तुम्हारी क्या चीज़ खो गई है?"
\end{hindi}}
\flushright{\begin{Arabic}
\quranayah[12][72]
\end{Arabic}}
\flushleft{\begin{hindi}
उन्होंने कहा, "शाही पैमाना हमें नहीं मिल रहा है। जो व्यक्ति उसे ला दे उसको एक ऊँट का बोझभर ग़ल्ला इनाम मिलेगा। मैं इसकी ज़िम्मेदारी लेता हूँ।"
\end{hindi}}
\flushright{\begin{Arabic}
\quranayah[12][73]
\end{Arabic}}
\flushleft{\begin{hindi}
वे कहने लगे, "अल्लाह की क़सम! तुम लोग जानते ही हो कि हम इस भू-भाग में बिगाड़ पैदा करने नहीं आए है और न हम चोर है।"
\end{hindi}}
\flushright{\begin{Arabic}
\quranayah[12][74]
\end{Arabic}}
\flushleft{\begin{hindi}
उन्होंने कहा, "यदि तुम झूठे सिद्ध हुए तो फिर उसका दंड क्या है?"
\end{hindi}}
\flushright{\begin{Arabic}
\quranayah[12][75]
\end{Arabic}}
\flushleft{\begin{hindi}
वे बोले, "उसका दंड यह है कि जिसके सामान में वह मिले वही उसका बदला ठहराया जाए। हम अत्याचारियों को ऐसा ही दंड देते है।"
\end{hindi}}
\flushright{\begin{Arabic}
\quranayah[12][76]
\end{Arabic}}
\flushleft{\begin{hindi}
फिर उसके भाई की खुरजी से पहले उनकी ख़ुरजियाँ देखनी शुरू की; फिर उसके भाई की ख़ुरजी से उसे बरामद कर लिया। इस प्रकार हमने यूसुफ़ का उपाय किया। वह शाही क़ानून के अनुसार अपने भाई को प्राप्त नहीं कर सकता था। बल्कि अल्लाह ही की इच्छा लागू है। हम जिसको चाहे उसके दर्जे ऊँचे कर दें। और प्रत्येक ज्ञानवान से ऊपर एक ज्ञानवान मौजूद है
\end{hindi}}
\flushright{\begin{Arabic}
\quranayah[12][77]
\end{Arabic}}
\flushleft{\begin{hindi}
उन्होंने कहा, "यदि यह चोरी करता है तो चोरी तो इससे पहले इसका एक भाई भी कर चुका है।" किन्तु यूसुफ़ ने इसे अपने जी ही में रखा और उनपर प्रकट नहीं किया। उसने कहा, "मक़ाम की दृष्टि से तुम अत्यन्त बुरे हो। जो कुछ तुम बताते हो, अल्लाह को उसका पूरा ज्ञान है।"
\end{hindi}}
\flushright{\begin{Arabic}
\quranayah[12][78]
\end{Arabic}}
\flushleft{\begin{hindi}
उन्होंने कहा, "ऐ अज़ीज़! इसका बाप बहुत ही बूढ़ा है। इसलिए इसके स्थान पर हममें से किसी को रख लीजिए। हमारी स्पष्ट में तो आप बड़े ही सुकर्मी है।"
\end{hindi}}
\flushright{\begin{Arabic}
\quranayah[12][79]
\end{Arabic}}
\flushleft{\begin{hindi}
उसने कहा, "इस बात से अल्लाह पनाह में रखे कि जिसके पास हमने अपना माल पाया है, उसे छोड़कर हम किसी दूसरे को रखें। फिर तो हम अत्याचारी ठहगेंगे।"
\end{hindi}}
\flushright{\begin{Arabic}
\quranayah[12][80]
\end{Arabic}}
\flushleft{\begin{hindi}
तो जब से वे उससे निराश हो गए तो परामर्श करने के लिए अलग जा बैठे। उनमें जो बड़ा था, वह कहने लगा, "क्या तुम जानते नहीं कि तुम्हारा बाप अल्लाह के नाम पर तुमसे वचन ले चुका है और उसको जो इससे पहले यूसुफ़ के मामले में तुमसे क़सूर हो चुका है? मैं तो इस भू-भाग से कदापि टलने का नहीं जब तक कि मेरे बाप मुझे अनुमति न दें या अल्लाह ही मेरे हक़ में कोई फ़ैसला कर दे। और वही सबसे अच्छा फ़ैसला करनेवाला है
\end{hindi}}
\flushright{\begin{Arabic}
\quranayah[12][81]
\end{Arabic}}
\flushleft{\begin{hindi}
तुम अपने बाप के पास लौटकर जाओ और कहो, "ऐ हमारे बाप! आपके बेटे ने चोरी की है। हमने तो वही कहा जो हमें मालूम हो सका, परोक्ष तो हमारी दृष्टि में था नहीं
\end{hindi}}
\flushright{\begin{Arabic}
\quranayah[12][82]
\end{Arabic}}
\flushleft{\begin{hindi}
आप उस बस्ती से पूछ लीजिए जहाँ हम थे और उस क़ाफ़िलें से भी जिसके साथ होकर हम आए। निस्संदेह हम बिलकुल सच्चे है।"
\end{hindi}}
\flushright{\begin{Arabic}
\quranayah[12][83]
\end{Arabic}}
\flushleft{\begin{hindi}
उसने कहा, "नहीं, बल्कि तुम्हारे जी ही ने तुम्हे पट्टी पढ़ाकर एक बात बना दी है। अब धैर्य से काम लेना ही उत्तम है! बहुत सम्भव है कि अल्लाह उन सबको मेरे पास ले आए। वह तो सर्वज्ञ, अत्यन्त तत्वदर्शी है।"
\end{hindi}}
\flushright{\begin{Arabic}
\quranayah[12][84]
\end{Arabic}}
\flushleft{\begin{hindi}
उसने उनकी ओर से मुख फेर लिया और कहने लगा, "हाय अफ़सोस, यूसुफ़ की जुदाई पर!" और ग़म के मारे उसकी आँखें सफ़ेद पड़ गई और वह घुटा जा रहा था
\end{hindi}}
\flushright{\begin{Arabic}
\quranayah[12][85]
\end{Arabic}}
\flushleft{\begin{hindi}
उन्होंने कहा, "अल्लाह की क़सम! आप तो यूसुफ़ ही की याद में लगे रहेंगे, यहाँ तक कि घुलकर रहेंगे या प्राण ही त्याग देंगे।"
\end{hindi}}
\flushright{\begin{Arabic}
\quranayah[12][86]
\end{Arabic}}
\flushleft{\begin{hindi}
उसने कहा, "मैं तो अपनी परेशानी और अपने ग़म की शिकायत अल्लाह ही से करता हूँ और अल्लाह की ओर से जो मैं जानता हूँ, तुम नही जानते
\end{hindi}}
\flushright{\begin{Arabic}
\quranayah[12][87]
\end{Arabic}}
\flushleft{\begin{hindi}
ऐ मेरे बेटों! जाओ और यूसुफ़ और उसके भाई की टोह लगाओ और अल्लाह की सदयता से निराश न हो। अल्लाह की सदयता से तो केवल कुफ़्र करनेवाले ही निराश होते है।"
\end{hindi}}
\flushright{\begin{Arabic}
\quranayah[12][88]
\end{Arabic}}
\flushleft{\begin{hindi}
फिर जब वे उसके पास उपस्थित हुए तो कहा, "ऐ अज़ीज़! हमें और हमारे घरवालों को बहुत तकलीफ़ पहुँची हैं और हम कुछ तुच्छ-सी पूँजी लेकर आए है, किन्तु आप हमें पूरी-पूरी माप प्रदान करें। और हमें दान दें। निश्चय ही दान करनेवालों को बदला अल्लाह देता है।"
\end{hindi}}
\flushright{\begin{Arabic}
\quranayah[12][89]
\end{Arabic}}
\flushleft{\begin{hindi}
उसने कहा, "क्या तुम्हें यह भी मालूम है कि जब तुम आवेग के वशीभूत थे तो यूसुफ़ और उसके भाई के साथ तुमने क्या किया था?"
\end{hindi}}
\flushright{\begin{Arabic}
\quranayah[12][90]
\end{Arabic}}
\flushleft{\begin{hindi}
वे बोल पड़े, "क्या यूसुफ़ आप ही है?" उसने कहा, "मैं ही यूसुफ़ हूँ और यह मेरा भाई है। अल्लाह ने हमपर उपकार किया है। सच तो यह है कि जो कोई डर रखे और धैर्य से काम ले तो अल्लाह भी उत्तमकारों का बदला अकारथ नहीं करता।"
\end{hindi}}
\flushright{\begin{Arabic}
\quranayah[12][91]
\end{Arabic}}
\flushleft{\begin{hindi}
उन्होंने कहा, "अल्लाह की क़सम! आपको अल्लाह ने हमारे मुक़ाबले में पसन्द किया और निश्चय ही चूक तो हमसे हुई।"
\end{hindi}}
\flushright{\begin{Arabic}
\quranayah[12][92]
\end{Arabic}}
\flushleft{\begin{hindi}
उसने कहा, "आज तुमपर कोई आरोप नहीं। अल्लाह तुम्हें क्षमा करे। वह सबसे बढ़कर दयावान है।
\end{hindi}}
\flushright{\begin{Arabic}
\quranayah[12][93]
\end{Arabic}}
\flushleft{\begin{hindi}
मेरा यह कुर्ता ले जाओ और इसे मेरे बाप के मुख पर डाल दो। उनकी नेत्र-ज्योति लौट आएगी, फिर अपने सब घरवालों को मेरे यहाँ ले आओ।"
\end{hindi}}
\flushright{\begin{Arabic}
\quranayah[12][94]
\end{Arabic}}
\flushleft{\begin{hindi}
इधर जब क़ाफ़िला चला तो उनके बाप ने कहा, "यदि तुम मुझे बहकी बातें करनेवाला न समझो तो मुझे तो यूसुफ़ की महक आ रही है।"
\end{hindi}}
\flushright{\begin{Arabic}
\quranayah[12][95]
\end{Arabic}}
\flushleft{\begin{hindi}
वे बोले, "अल्लाह की क़सम! आप तो अभी तक अपनी उसी पुरानी भ्रांति में पड़े हुए है।"
\end{hindi}}
\flushright{\begin{Arabic}
\quranayah[12][96]
\end{Arabic}}
\flushleft{\begin{hindi}
फिर जब शुभ सूचना देनेवाला आया तो उसने उस (कुर्ते) को उसके मुँह पर डाल दिया और तत्क्षण उसकी नेत्र-ज्योति लौट आई। उसने कहा, "क्या मैंने तुमसे कहा नहीं था कि अल्लाह की ओर से जो मैं जानता हूँ, तुम नहीं जानते।"
\end{hindi}}
\flushright{\begin{Arabic}
\quranayah[12][97]
\end{Arabic}}
\flushleft{\begin{hindi}
वे बोले, "ऐ मेरे बाप! आप हमारे गुनाहों की क्षमा के लिए प्रार्थना करें। वास्तव में चूक हमसे ही हुई।"
\end{hindi}}
\flushright{\begin{Arabic}
\quranayah[12][98]
\end{Arabic}}
\flushleft{\begin{hindi}
उसने कहा, "मैं अपने रब से तुम्हारे लिए प्रार्थना करूँगा। वह बहुत क्षमाशील, दयावान है।"
\end{hindi}}
\flushright{\begin{Arabic}
\quranayah[12][99]
\end{Arabic}}
\flushleft{\begin{hindi}
फिर जब वे यूसुफ़ के पास पहुँचे तो उसने अपने माँ-बाप को ख़ास अपने पास जगह दी औऱ कहा, "तुम सब नगर में प्रवेश करो। अल्लाह ने चाहा तो यह प्रवेश निश्चिन्तता के साथ होगा।"
\end{hindi}}
\flushright{\begin{Arabic}
\quranayah[12][100]
\end{Arabic}}
\flushleft{\begin{hindi}
उसने अपने माँ-बाप को ऊँची जगह सिंहासन पर बिठाया और सब उसके आगे सजदे मे गिर पड़े। इस अवसर पर उसने कहा, "ऐ मेरे बाप! यह मेरे विगत स्वप्न का साकार रूप है। इसे मेरे रब ने सच बना दिया। और उसने मुझपर उपकार किया जब मुझे क़ैदख़ाने से निकाला और आप भाइयों के बीच फ़साद डलवा दिया था। निस्संदेह मेरा रब जो चाहता है उसके लिए सूक्ष्म उपाय करता है। वास्तव में वही सर्वज्ञ, अत्यन्त तत्वदर्शी है
\end{hindi}}
\flushright{\begin{Arabic}
\quranayah[12][101]
\end{Arabic}}
\flushleft{\begin{hindi}
मेरे रब! तुने मुझे राज्य प्रदान किया और मुझे घटनाओं और बातों के निष्कर्ष तक पहुँचना सिखाया। आकाश और धरती के पैदा करनेवाले! दुनिया और आख़िरत में तू ही मेरा संरक्षक मित्र है। तू मुझे इस दशा से उठा कि मैं मुस्लिम (आज्ञाकारी) हूँ और मुझे अच्छे लोगों के साथ मिला।"
\end{hindi}}
\flushright{\begin{Arabic}
\quranayah[12][102]
\end{Arabic}}
\flushleft{\begin{hindi}
ये परोक्ष की ख़बरे हैं जिनकी हम तुम्हारी ओर प्रकाशना कर रहे है। तुम तो उनके पास नहीं थे, जब उन्होंने अपने मामले को पक्का करके षड्यंत्र किया था
\end{hindi}}
\flushright{\begin{Arabic}
\quranayah[12][103]
\end{Arabic}}
\flushleft{\begin{hindi}
किन्तु चाहे तुम कितना ही चाहो, अधिकतर लोग तो मानेंगे नहीं
\end{hindi}}
\flushright{\begin{Arabic}
\quranayah[12][104]
\end{Arabic}}
\flushleft{\begin{hindi}
तुम उनसे इसका कोई बदला भी नहीं माँगते। यह तो सारे संसार के लिए बस एक अनुस्मरण है
\end{hindi}}
\flushright{\begin{Arabic}
\quranayah[12][105]
\end{Arabic}}
\flushleft{\begin{hindi}
आकाशों और धरती में कितनी ही निशानियाँ हैं, जिनपर से वे इस तरह गुज़र जाते है कि उनकी ओर वे ध्यान ही नहीं देते
\end{hindi}}
\flushright{\begin{Arabic}
\quranayah[12][106]
\end{Arabic}}
\flushleft{\begin{hindi}
इनमें अधिकतर लोग अल्लाह को मानते भी है तो इस तरह कि वे साझी भी ठहराते है
\end{hindi}}
\flushright{\begin{Arabic}
\quranayah[12][107]
\end{Arabic}}
\flushleft{\begin{hindi}
क्या वे इस बात से निश्चिन्त है कि अल्लाह की कोई यातना उन्हें ढँक ले या सहसा वह घड़ी ही उनपर आ जाए, जबकि वे बिलकुल बेख़बर हों?
\end{hindi}}
\flushright{\begin{Arabic}
\quranayah[12][108]
\end{Arabic}}
\flushleft{\begin{hindi}
कह दो, "यही मेरा मार्ग है। मैं अल्लाह की ओर बुलाता हूँ। मैं स्वयं भी पूर्ण प्रकाश में हूँ और मेरे अनुयायी भी - महिमावान है अल्लाह! ृृ- और मैं कदापि बहुदेववादी नहीं।"
\end{hindi}}
\flushright{\begin{Arabic}
\quranayah[12][109]
\end{Arabic}}
\flushleft{\begin{hindi}
तुमसे पहले भी हमने जिनको रसूल बनाकर भेजा, वे सब बस्तियों के रहनेवाले पुरुष ही थे। हम उनकी ओर प्रकाशना करते रहे - फिर क्या वे धरती में चले-फिरे नहीं कि देखते कि उनका कैसा परिणाम हुआ, जो उनसे पहले गुज़रे है? निश्चय ही आख़िरत का घर ही डर रखनेवालों के लिए सर्वोत्तम है। तो क्या तुम समझते नहीं? -
\end{hindi}}
\flushright{\begin{Arabic}
\quranayah[12][110]
\end{Arabic}}
\flushleft{\begin{hindi}
यहाँ तक कि जब वे रसूल निराश होने लगे और वे समझने लगे कि उनसे झूठ कहा गया था कि सहसा उन्हें हमारी सहायता पहुँच गई। फिर हमने जिसे चाहा बचा लिया। किन्तु अपराधी लोगों पर से तो हमारी यातना टलती ही नहीं
\end{hindi}}
\flushright{\begin{Arabic}
\quranayah[12][111]
\end{Arabic}}
\flushleft{\begin{hindi}
निश्चय ही उनकी कथाओं में बुद्धि और समझ रखनेवालों के लिए एक शिक्षाप्रद सामग्री है। यह कोई घड़ी हुई बात नहीं है, बल्कि यह अपने से पूर्व की पुष्टि में है, और हर चीज़ का विस्तार और ईमान लानेवाले लोगों के लिए मार्ग-दर्शन और दयालुता है
\end{hindi}}
\chapter{Ar-Ra'd (The Thunder)}
\begin{Arabic}
\Huge{\centerline{\basmalah}}\end{Arabic}
\flushright{\begin{Arabic}
\quranayah[13][1]
\end{Arabic}}
\flushleft{\begin{hindi}
अलिफ़॰ लाम॰ मीम॰ रा॰। ये किताब की आयतें है औऱ जो कुछ तुम्हारे रब की ओर से तुम्हारी ओर अवतरित हुआ है, वह सत्य है, किन्तु अधिकतर लोग मान नहीं रहे है
\end{hindi}}
\flushright{\begin{Arabic}
\quranayah[13][2]
\end{Arabic}}
\flushleft{\begin{hindi}
अल्लाह वह है जिसने आकाशों को बिना सहारे के ऊँचा बनाया जैसा कि तुम उन्हें देखते हो। फिर वह सिंहासन पर आसीन हुआ। उसने सूर्य और चन्द्रमा को काम पर लगाया। हरेक एक नियत समय तक के लिए चला जा रहा है। वह सारे काम का विधान कर रहा है; वह निशानियाँ खोल-खोलकर बयान करता है, ताकि तुम्हें अपने रब से मिलने का विश्वास हो
\end{hindi}}
\flushright{\begin{Arabic}
\quranayah[13][3]
\end{Arabic}}
\flushleft{\begin{hindi}
और वही है जिसने धरती को फैलाया और उसमें जमे हुए पर्वत और नदियाँ बनाई और हरेक पैदावार की दो-दो क़िस्में बनाई। वही रात से दिन को छिपा देता है। निश्चय ही इसमें उन लोगों के लिए निशानियाँ है जो सोच-विचार से काम लेते है
\end{hindi}}
\flushright{\begin{Arabic}
\quranayah[13][4]
\end{Arabic}}
\flushleft{\begin{hindi}
और धरती में पास-पास भूभाग पाए जाते है जो परस्पर मिले हुए है, और अंगूरों के बाग़ है और खेतियाँ है औऱ खजूर के पेड़ है, इकहरे भी और दोहरे भी। सबको एक ही पानी से सिंचित करता है, फिर भी हम पैदावार और स्वाद में किसी को किसी के मुक़ाबले में बढ़ा देते है। निश्चय ही इसमें उन लोगों के लिए निशानियाँ हैं, जो बुद्धि से काम लेते है
\end{hindi}}
\flushright{\begin{Arabic}
\quranayah[13][5]
\end{Arabic}}
\flushleft{\begin{hindi}
अब यदि तुम्हें आश्चर्य ही करना है तो आश्चर्य की बात तो उनका यह कहना है कि ,"क्या जब हम मिट्टी हो जाएँगे तो क्या हम नए सिरे से पैदा भी होंगे?" वही हैं जिन्होंने अपने रब के साथ इनकार की नीति अपनाई और वही है, जिनकी गर्दनों मे तौक़ पड़े हुए है और वही आग (में पड़ने) वाले है जिसमें उन्हें सदैव रहना है
\end{hindi}}
\flushright{\begin{Arabic}
\quranayah[13][6]
\end{Arabic}}
\flushleft{\begin{hindi}
वे भलाई से पहले बुराई के लिए तुमसे जल्दी मचा रहे हैं, हालाँकि उनसे पहले कितनी ही शिक्षाप्रद मिसालें गुज़र चुकी है। किन्तु तुम्हारा रब लोगों को उनके अत्याचार के बावजूद क्षमा कर देता है और वास्तव में तुम्हारा रब दंड देने में भी बहुत कठोर है
\end{hindi}}
\flushright{\begin{Arabic}
\quranayah[13][7]
\end{Arabic}}
\flushleft{\begin{hindi}
जिन्होंने इनकार किया, वे कहते हैं, "उसपर उसके रब की ओर से कोई निशानी क्यों नहीं अवतरित हुई?" तुम तो बस एक चेतावनी देनेवाले हो और हर क़ौम के लिए एक मार्गदर्शक हुआ है
\end{hindi}}
\flushright{\begin{Arabic}
\quranayah[13][8]
\end{Arabic}}
\flushleft{\begin{hindi}
किसी भी स्त्री-जाति को जो भी गर्भ रहता है अल्लाह उसे जान रहा होता है और उसे भी जो गर्भाशय में कमी-बेशी होती है। और उसके यहाँ हरेक चीज़ का एक निश्चित अन्दाज़ा है
\end{hindi}}
\flushright{\begin{Arabic}
\quranayah[13][9]
\end{Arabic}}
\flushleft{\begin{hindi}
वह परोक्ष और प्रत्यक्ष का ज्ञाता है, महान है, अत्यन्त उच्च है
\end{hindi}}
\flushright{\begin{Arabic}
\quranayah[13][10]
\end{Arabic}}
\flushleft{\begin{hindi}
तुममें से कोई चुपके से बात करे और जो कोई ज़ोर से और जो कोई रात में छिपता हो और जो दिन में चलता-फिरता दीख पड़ता हो उसके लिए सब बराबर है
\end{hindi}}
\flushright{\begin{Arabic}
\quranayah[13][11]
\end{Arabic}}
\flushleft{\begin{hindi}
उसके रक्षक (पहरेदार) उसके अपने आगे और पीछे लगे होते हैं जो अल्लाह के आदेश से उसकी रक्षा करते है। किसी क़ौम के लोगों को जो कुछ प्राप्त होता है अल्लाह उसे बदलता नहीं, जब तक कि वे स्वयं अपने आपको न बदल डालें। और जब अल्लाह किसी क़ौम का अनिष्ट चाहता है तो फिर वह उससे टल नहीं सकता, और उससे हटकर उनका कोई समर्थक और संरक्षक भी नहीं
\end{hindi}}
\flushright{\begin{Arabic}
\quranayah[13][12]
\end{Arabic}}
\flushleft{\begin{hindi}
वही है जो भय और आशा के निमित्त तुम्हें बिजली की चमक दिखाता है और बोझिल बादलों को उठाता है
\end{hindi}}
\flushright{\begin{Arabic}
\quranayah[13][13]
\end{Arabic}}
\flushleft{\begin{hindi}
बादल की गरज उसका गुणगान करती है और उसके भय से काँपते हुए फ़रिश्ते भी। वही कड़कती बिजलियाँ भेजता है, फिर जिसपर चाहता है उन्हें गिरा देता है, जबकि वे अल्लाह के विषय में झगड़ रहे होते है। निश्चय ही उसकी चाल बड़ी सख़्त है
\end{hindi}}
\flushright{\begin{Arabic}
\quranayah[13][14]
\end{Arabic}}
\flushleft{\begin{hindi}
उसी के लिए सच्ची पुकार है। उससे हटकर जिनको वे पुकारते है, वे उनकी पुकार का कुछ भी उत्तर नहीं देते। बस यह ऐसा ही होता है जैसे कोई अपने दोनों हाथ पानी की ओर इसलिए फैलाए कि वह उसके मुँह में पहुँच जाए, हालाँकि वह उसतक पहुँचनेवाला नहीं। कुफ़्र करनेवालों की पुकार तो बस भटकने ही के लिए होती है
\end{hindi}}
\flushright{\begin{Arabic}
\quranayah[13][15]
\end{Arabic}}
\flushleft{\begin{hindi}
आकाशों और धरती में जो भी है स्वेच्छापूर्वक या विवशतापूर्वक अल्लाह ही को सजदा कर रहे है और उनकी परछाइयाँ भी प्रातः और संध्या समय
\end{hindi}}
\flushright{\begin{Arabic}
\quranayah[13][16]
\end{Arabic}}
\flushleft{\begin{hindi}
कहो, "आकाशों और धरती का रब कौन है?" कहो, "अल्लाह" कह दो, "फिर क्या तुमने उससे हटकर दूसरों को अपना संरक्षक बना रखा है, जिन्हें स्वयं अपने भी किसी लाभ का न अधिकार प्राप्त है और न किसी हानि का?" कहो, "क्या अंधा और आँखोंवाला दोनों बराबर होते है? या बराबर होते हो अँधरे और प्रकाश? या जिनको अल्लाह का सहभागी ठहराया है, उन्होंने भी कुछ पैदा किया है, जैसा कि उसने पैदा किया है, जिसके कारण सृष्टि का मामला इनके लिए गडुमडु हो गया है?" कहो, "हर चीज़ को पैदा करनेवाला अल्लाह है और वह अकेला है, सब पर प्रभावी!"
\end{hindi}}
\flushright{\begin{Arabic}
\quranayah[13][17]
\end{Arabic}}
\flushleft{\begin{hindi}
उसने आकाश से पानी उतारा तो नदी-नाले अपनी-अपनी समाई के अनुसार बह निकले। फिर पानी के बहाव ने उभरे हुए झाग को उठा लिया और उसमें से भी, जिसे वे ज़ेवर या दूसरे सामान बनाने के लिए आग में तपाते हैं, ऐसा ही झाग उठता है। इस प्रकार अल्लाह सत्य और असत्य की मिसाल बयान करता है। फिर जो झाग है वह तो सूखकर नष्ट हो जाता है और जो कुछ लोगों को लाभ पहुँचानेवाला होता है, वह धरती में ठहर जाता है। इसी प्रकार अल्लाह दृष्टांत प्रस्तुत करता है
\end{hindi}}
\flushright{\begin{Arabic}
\quranayah[13][18]
\end{Arabic}}
\flushleft{\begin{hindi}
जिन लोगों ने अपने रब का आमंत्रण स्वीकार कर लिया, उनके लिए अच्छा पुरस्कार है। रहे वे लोग जिन्होंने उसे स्वीकार नहीं किया यदि उनके पास वह सब कुछ हो जो धरती में हैं, बल्कि उसके साथ उतना और भी हो तो अपनी मुक्ति के लिए वे सब दे डालें। वही हैं, जिनका बुरा हिसाब होगा। उनका ठिकाना जहन्नम है और वह अत्यन्त बुरा विश्राम-स्थल है
\end{hindi}}
\flushright{\begin{Arabic}
\quranayah[13][19]
\end{Arabic}}
\flushleft{\begin{hindi}
भला वह व्यक्ति जो जानता है कि जो कुछ तुम पर उतरा है तुम्हारे रब की ओर से सत्य है, कभी उस जैसा हो सकता है जो अंधा है? परन्तु समझते तो वही है जो बुद्धि और समझ रखते है,
\end{hindi}}
\flushright{\begin{Arabic}
\quranayah[13][20]
\end{Arabic}}
\flushleft{\begin{hindi}
जो अल्लाह के साथ की हुई प्रतिज्ञा को पूरा करते है औऱ अभिवचन को तोड़ते नहीं,
\end{hindi}}
\flushright{\begin{Arabic}
\quranayah[13][21]
\end{Arabic}}
\flushleft{\begin{hindi}
और जो ऐसे हैं कि अल्लाह नॆ जिसे जोड़ने का आदेश दिया है उसे जोड़ते हैं और अपनॆ रब से डरते रहते हैं और बुरॆ हिसाब का उन्हॆं डर लगा रहता है
\end{hindi}}
\flushright{\begin{Arabic}
\quranayah[13][22]
\end{Arabic}}
\flushleft{\begin{hindi}
और जिन लोगों ने अपने रब की प्रसन्नता की चाह में धैर्य से काम लिया और नमाज़ क़ायम की और जो कुछ हमने उन्हें दिया है, उसमें से खुले और छिपे ख़र्च किया, और भलाई के द्वारा बुराई को दूर करते है। वही लोग है जिनके लिए आख़िरत के घर का अच्छा परिणाम है,
\end{hindi}}
\flushright{\begin{Arabic}
\quranayah[13][23]
\end{Arabic}}
\flushleft{\begin{hindi}
अर्थात सदैव रहने के बाग़ है जिनमें वे प्रवेश करेंगे और उनके बाप-दादा और उनकी पत्नियों और उनकी सन्तानों में से जो नेक होंगे वे भी और हर दरवाज़े से फ़रिश्ते उनके पास पहुँचेंगे
\end{hindi}}
\flushright{\begin{Arabic}
\quranayah[13][24]
\end{Arabic}}
\flushleft{\begin{hindi}
(वे कहेंगे) "तुमपर सलाम है उसके बदले में जो तुमने धैर्य से काम लिया।" अतः क्या ही अच्छा परिणाम है आख़िरत के घर का!
\end{hindi}}
\flushright{\begin{Arabic}
\quranayah[13][25]
\end{Arabic}}
\flushleft{\begin{hindi}
रहे वे लोग जो अल्लाह की प्रतिज्ञा को उसे दृढ़ करने के पश्चात तोड़ डालते है और अल्लाह ने जिसे जोड़ने का आदेश दिया है, उसे काटते है और धरती में बिगाड़ पैदा करते है। वहीं है जिनके लिए फिटकार है और जिनके लिए आख़िरत का बुरा घर है
\end{hindi}}
\flushright{\begin{Arabic}
\quranayah[13][26]
\end{Arabic}}
\flushleft{\begin{hindi}
अल्लाह जिसको चाहता है प्रचुर फैली हुई रोज़ी प्रदान करता है औऱ इसी प्रकार नपी-तुली भी। और वे सांसारिक जीवन में मग्न हैं, हालाँकि सांसारिक जीवन आख़िरत के मुक़ाबले में तो बस अल्प सुख-सामग्री है
\end{hindi}}
\flushright{\begin{Arabic}
\quranayah[13][27]
\end{Arabic}}
\flushleft{\begin{hindi}
जिन लोगों ने इनकार किया वे कहते है, "उसपर उसके रब की ओर से कोई निशानी क्यों नहीं उतरी?" कहो, "अल्लाह जिसे चाहता है पथभ्रष्ट कर देता है। अपनी ओर से वह मार्गदर्शन उसी का करता है जो रुजू होता है।"
\end{hindi}}
\flushright{\begin{Arabic}
\quranayah[13][28]
\end{Arabic}}
\flushleft{\begin{hindi}
ऐसे ही लोग है जो ईमान लाए और जिनके दिलों को अल्लाह के स्मरण से आराम और चैन मिलता है। सुन लो, अल्लाह के स्मरण से ही दिलों को संतोष प्राप्त हुआ करता है
\end{hindi}}
\flushright{\begin{Arabic}
\quranayah[13][29]
\end{Arabic}}
\flushleft{\begin{hindi}
जो लोग ईमान लाए और उन्होंने अच्छे कर्म किए उनके लिए सुख-सौभाग्य है और लौटने का अच्छा ठिकाना है
\end{hindi}}
\flushright{\begin{Arabic}
\quranayah[13][30]
\end{Arabic}}
\flushleft{\begin{hindi}
अतएव हमने तुम्हें एक ऐसे समुदाय में भेजा है जिससे पहले कितने ही समुदाय गुज़र चुके है, ताकि हमने तुम्हारी ओर जो प्रकाशना की है, उसे उनको सुना दो, यद्यपि वे रहमान के साथ इनकार की नीति अपनाए हुए है। कह दो, "वही मेरा रब है। उसके सिवा कोई पूज्य-प्रभु नहीं। उसी पर मेरा भरोसा है और उसी की ओर मुझे पलटकर जाना है।"
\end{hindi}}
\flushright{\begin{Arabic}
\quranayah[13][31]
\end{Arabic}}
\flushleft{\begin{hindi}
और यदि कोई ऐसा क़ुरआन होता जिसके द्वारा पहाड़ चलने लगते या उससे धरती खंड-खंड हो जाती या उसके द्वारा मुर्दे बोलने लगते (तब भी वे लोग ईमान न लाते) । नहीं, बल्कि बात यह है कि सारे काम अल्लाह ही के अधिकार में है। फिर क्या जो लोग ईमान लाए है वे यह जानकर निराश नहीं हुए कि यदि अल्लाह चाहता तो सारे ही मनुष्यों को सीधे मार्ग पर लगा देता? और इनकार करनेवालों पर तो उनकी करतूतों के बदले में कोई न कोई आपदा निरंतर आती ही रहेगी, या उनके घर के निकट ही कहीं उतरती रहेगी, यहाँ तक कि अल्लाह का वादा आ पूरा होगा। निस्संदेह अल्लाह अपने वादे के विरुद्ध नहीं जाता।"
\end{hindi}}
\flushright{\begin{Arabic}
\quranayah[13][32]
\end{Arabic}}
\flushleft{\begin{hindi}
तुमसे पहले भी कितने ही रसूलों का उपहास किया जा चुका है, किन्तु मैंने इनकार करनेवालों को मुहलत दी। फिर अंततः मैंने उन्हें पकड़ लिया, फिर कैसी रही मेरी सज़ा?
\end{hindi}}
\flushright{\begin{Arabic}
\quranayah[13][33]
\end{Arabic}}
\flushleft{\begin{hindi}
भला वह (अल्लाह) जो प्रत्येक व्यक्ति के सिर पर, उसकी कमाई पर निगाह रखते हुए खड़ा है (उसके समान कोई दूसरा हो सकता है)? फिर भी लोगों ने अल्लाह के सहभागी-ठहरा रखे है। कहो, "तनिक उनके नाम तो लो! (क्या तुम्हारे पास उनके पक्ष में कोई प्रमाण है?) या ऐसा है कि तुम उसे ऐसी बात की ख़बर दे रहे हो, जिसके अस्तित्व की उसे धरती भर में ख़बर नहीं? या यूँ ही यह एक ऊपरी बात ही बात है?" नहीं, बल्कि इनकार करनेवालों को उनकी मक्कारी ही सुहावनी लगती है और वे मार्ग से रुक गए है। जिसे अल्लाह ही गुमराही में छोड़ दे, उसे कोई मार्ग पर लानेवाला नहीं
\end{hindi}}
\flushright{\begin{Arabic}
\quranayah[13][34]
\end{Arabic}}
\flushleft{\begin{hindi}
उनके लिए सांसारिक जीवन में भी यातना, तो वह अत्यन्त कठोर है। औऱ कोई भी तो नहीं जो उन्हें अल्लाह से बचानेवाला हो
\end{hindi}}
\flushright{\begin{Arabic}
\quranayah[13][35]
\end{Arabic}}
\flushleft{\begin{hindi}
डर रखनेवालों के लिए जिस जन्नत का वादा है उसकी शान यह है कि उसके नीचे नहरें बह रही है, उसके फल शाश्वत है और इसी प्रकार उसकी छाया भी। यह परिणाम है उनका जो डर रखते है, जबकि इनकार करनेवालों का परिणाम आग है
\end{hindi}}
\flushright{\begin{Arabic}
\quranayah[13][36]
\end{Arabic}}
\flushleft{\begin{hindi}
जिन लोगों को हमने किताब प्रदान की है वे उससे, जो तुम्हारी ओर उतारा है, हर्षित होते है और विभिन्न गिरोहों के कुछ लोग ऐसे भी है जो उसकी कुछ बातों का इनकार करते है। कह दो, "मुझे पर बस यह आदेश हुआ है कि मैं अल्लाह की बन्दगी करूँ और उसका सहभागी न ठहराऊँ। मैं उसी की ओर बुलाता हूँ और उसी की ओर मुझे लौटकर जाना है।"
\end{hindi}}
\flushright{\begin{Arabic}
\quranayah[13][37]
\end{Arabic}}
\flushleft{\begin{hindi}
और इसी प्रकार हमने इस (क़ुरआन) को एक अरबी फ़रमान के रूप में उतारा है। अब यदि तुम उस ज्ञान के पश्चात भी, जो तुम्हारे पास आ चुका है, उनकी इच्छाओं के पीछे चले तो अल्लाह के मुक़ाबले में न तो तुम्हारा कोई सहायक मित्र होगा और न कोई बचानेवाला
\end{hindi}}
\flushright{\begin{Arabic}
\quranayah[13][38]
\end{Arabic}}
\flushleft{\begin{hindi}
तुमसे पहले भी हम, कितने ही रसूल भेज चुके है और हमने उन्हें पत्नियों और बच्चे भी दिए थे, और किसी रसूल को यह अधिकार नहीं था कि वह अल्लाह की अनुमति के बिना कोई निशानी स्वयं ला लेता। हर चीज़ के एक समय जो अटल लिखित है
\end{hindi}}
\flushright{\begin{Arabic}
\quranayah[13][39]
\end{Arabic}}
\flushleft{\begin{hindi}
अल्लाह जो कुछ चाहता है मिटा देता है। इसी तरह वह क़ायम भी रखता है। मूल किताब तो स्वयं उसी के पास है
\end{hindi}}
\flushright{\begin{Arabic}
\quranayah[13][40]
\end{Arabic}}
\flushleft{\begin{hindi}
हम जो वादा उनसे कर रहे है चाहे उसमें से कुछ हम तुम्हें दिखा दें, या तुम्हें उठा लें। तुम्हारा दायित्व तो बस सन्देश का पहुँचा देना ही है, हिसाब लेना तो हमारे ज़िम्मे है
\end{hindi}}
\flushright{\begin{Arabic}
\quranayah[13][41]
\end{Arabic}}
\flushleft{\begin{hindi}
क्या उन्होंने देखा नहीं कि हम धरती पर चले आ रहे है, उसे उसके किनारों से घटाते हुए? अल्लाह ही फ़ैसला करता है। कोई नहीं जो उसके फ़ैसले को पीछे डाल सके। वह हिसाब भी जल्द लेता है
\end{hindi}}
\flushright{\begin{Arabic}
\quranayah[13][42]
\end{Arabic}}
\flushleft{\begin{hindi}
उनसे पहले जो लोग गुज़रे है, वे भी चालें चल चुके है, किन्तु वास्तविक चाल तो पूरी की पूरी अल्लाह ही के हाथ में है। प्रत्येक व्यक्ति जो कमाई कर रहा है उसे वह जानता है। इनकार करनेवालों को शीघ्र ही ज्ञात हो जाएगा कि परलोक-गृह के शुभ परिणाम के अधिकारी कौन है
\end{hindi}}
\flushright{\begin{Arabic}
\quranayah[13][43]
\end{Arabic}}
\flushleft{\begin{hindi}
जिन लोगों ने इनकार की नीति अपनाई, वे कहते है, "तुम कोई रसूल नहीं हो।" कह दो, "मेरे और तुम्हारे बीच अल्लाह की और जिस किसी के पास किताब का ज्ञान है उसकी, गवाही काफ़ी है।"
\end{hindi}}
\chapter{Ibrahim (Abraham)}
\begin{Arabic}
\Huge{\centerline{\basmalah}}\end{Arabic}
\flushright{\begin{Arabic}
\quranayah[14][1]
\end{Arabic}}
\flushleft{\begin{hindi}
अलिफ़॰ लाम॰ रा॰। यह एक किताब है जिसे हमने तुम्हारी ओर अवतरित की है, ताकि तुम मनुष्यों को अँधेरों से निकालकर प्रकाश की ओर ले आओ, उनके रब की अनुमति से प्रभुत्वशाली, प्रशंस्य सत्ता, उस अल्लाह के मार्ग की ओर
\end{hindi}}
\flushright{\begin{Arabic}
\quranayah[14][2]
\end{Arabic}}
\flushleft{\begin{hindi}
जिसका वह सब है जो कुछ आकाशों में है और जो कुछ धरती में है। इनकार करनेवालों के लिए तो एक कठोर यातना के कारण बड़ी तबाही है
\end{hindi}}
\flushright{\begin{Arabic}
\quranayah[14][3]
\end{Arabic}}
\flushleft{\begin{hindi}
जो आख़िरत की अपेक्षा सांसारिक जीवन को प्राथमिकता देते है और अल्लाह के मार्ग से रोकते है और उसमें टेढ़ पैदा करना चाहते है, वही परले दरजे की गुमराही में पड़े है
\end{hindi}}
\flushright{\begin{Arabic}
\quranayah[14][4]
\end{Arabic}}
\flushleft{\begin{hindi}
हमने जो रसूल भी भेजा, उसकी अपनी क़ौम की भाषा के साथ ही भेजा, ताकि वह उनके लिए अच्छी तरह खोलकर बयान कर दे। फिर अल्लाह जिसे चाहता है पथभ्रष्ट रहने देता है और जिसे चाहता है सीधे मार्ग पर लगा देता है। वह है भी प्रभुत्वशाली, अत्यन्त तत्वदर्शी
\end{hindi}}
\flushright{\begin{Arabic}
\quranayah[14][5]
\end{Arabic}}
\flushleft{\begin{hindi}
हमने मूसा को अपनी निशानियों के साथ भेजा था कि "अपनी क़ौम के लोगों को अँधेरों से प्रकाश की ओर निकाल ला और उन्हें अल्लाह के दिवस याद दिला।" निश्चय ही इसमें प्रत्येक धैर्यवान, कृतज्ञ व्यक्ति के लिए कितनी ही निशानियाँ है
\end{hindi}}
\flushright{\begin{Arabic}
\quranayah[14][6]
\end{Arabic}}
\flushleft{\begin{hindi}
जब मूसा ने अपनी क़ौम के लोगों से कहा, "अल्लाह ही उस कृपादृष्टि को याद करो, जो तुमपर हुई। जब उसने तुम्हें फ़िरऔनियों से छुटकारा दिलाया जो तुम्हें बुरी यातना दे रहे थे, तुम्हारे बेटों का वध कर डालते थे और तुम्हारी औरतों को जीवित रखते थे, किन्तु इसमें तुम्हारे रब की ओर से बड़ी कृपा हुई।"
\end{hindi}}
\flushright{\begin{Arabic}
\quranayah[14][7]
\end{Arabic}}
\flushleft{\begin{hindi}
जब तुम्हारे रब ने सचेत कर दिया था कि 'यदि तुम कृतज्ञ हुए तो मैं तुम्हें और अधिक दूँगा, परन्तु यदि तुम अकृतज्ञ सिद्ध हुए तो निश्चय ही मेरी यातना भी अत्यन्त कठोर है।'
\end{hindi}}
\flushright{\begin{Arabic}
\quranayah[14][8]
\end{Arabic}}
\flushleft{\begin{hindi}
और मूसा ने भी कहा था, "यदि तुम और वे जो भी धरती में हैं सब के सब अकृतज्ञ हो जाओ तो अल्लाह तो बड़ा निरपेक्ष, प्रशंस्य है।"
\end{hindi}}
\flushright{\begin{Arabic}
\quranayah[14][9]
\end{Arabic}}
\flushleft{\begin{hindi}
क्या तुम्हें उन लोगों की खबर नहीं पहुँची जो तुमसे पहले गुज़रे हैं, नूह की क़ौम और आद और समूद और वे लोग जो उनके पश्चात हुए जिनको अल्लाह के अतिरिक्त कोई नहीं जानता? उनके पास उनके रसूल स्पष्टि प्रमाण लेकर आए थे, किन्तु उन्होंने उनके मुँह पर अपने हाथ रख दिए और कहने लगे, "जो कुछ देकर तुम्हें भेजा गया है, हम उसका इनकार करते है और जिसकी ओर तुम हमें बुला रहे हो, उसके विषय में तो हम अत्यन्त दुविधाजनक संदेह में ग्रस्त है।"
\end{hindi}}
\flushright{\begin{Arabic}
\quranayah[14][10]
\end{Arabic}}
\flushleft{\begin{hindi}
उनके रसूलों ने कहो, "क्या अल्लाह के विषय में संदेह है, जो आकाशों और धरती का रचयिता है? वह तो तुम्हें इसलिए बुला रहा है, ताकि तुम्हारे गुनाहों को क्षमा कर दे और तुम्हें एक नियत समय तक मुहल्ल दे।" उन्होंने कहा, "तुम तो बस हमारे ही जैसे एक मनुष्य हो, चाहते हो कि हमें उनसे रोक दो जिनकी पूजा हमारे बाप-दादा करते आए है। अच्छा, तो अब हमारे सामने कोई स्पष्ट प्रमाण ले आओ।"
\end{hindi}}
\flushright{\begin{Arabic}
\quranayah[14][11]
\end{Arabic}}
\flushleft{\begin{hindi}
उनके रसूलों ने उनसे कहा, "हम तो वास्तव में बस तुम्हारे ही जैसे मनुष्य है, किन्तु अल्लाह अपने बन्दों में से जिनपर चाहता है एहसान करता है और यह हमारा काम नहीं कि तुम्हारे सामने कोई प्रमाण ले आएँ। यह तो बस अल्लाह के आदेश के पश्चात ही सम्भव है; और अल्लाह ही पर ईमानवालों को भरोसा करना चाहिए
\end{hindi}}
\flushright{\begin{Arabic}
\quranayah[14][12]
\end{Arabic}}
\flushleft{\begin{hindi}
आख़िर हमें क्या हुआ है कि हम अल्लाह पर भरोसा न करें, जबकि उसने हमें हमारे मार्ग दिखाए है? तुम हमें जो तकलीफ़ पहुँचा रहे हो उसके मुक़ाबले में हम धैर्य से काम लेंगे। भरोसा करनेवालों को तो अल्लाह ही पर भरोसा करना चाहिए।"
\end{hindi}}
\flushright{\begin{Arabic}
\quranayah[14][13]
\end{Arabic}}
\flushleft{\begin{hindi}
अन्ततः इनकार करनेवालों ने अपने रसूलों से कहा, "हम तुम्हें अपने भू-भाग से निकालकर रहेंगे, या तो तुम्हें हमारे पंथ में लौट आना होगा।" तब उनके रब ने उनकी ओर प्रकाशना की, "हम अत्याचारियों को विनष्ट करके रहेंगे
\end{hindi}}
\flushright{\begin{Arabic}
\quranayah[14][14]
\end{Arabic}}
\flushleft{\begin{hindi}
और उनके पश्चात तुम्हें इस धरती में बसाएँगे। यह उसके लिए है, जिसे मेरे समक्ष खड़े होने का भय हो और जो मेरी चेतावनी से डरे।"
\end{hindi}}
\flushright{\begin{Arabic}
\quranayah[14][15]
\end{Arabic}}
\flushleft{\begin{hindi}
उन्होंने फ़ैसला चाहा और प्रत्येक सरकश-दुराग्रही असफल होकर रहा
\end{hindi}}
\flushright{\begin{Arabic}
\quranayah[14][16]
\end{Arabic}}
\flushleft{\begin{hindi}
वह जहन्नम से घिरा है और पीने को उसे कचलोहू का पानी दिया जाएगा,
\end{hindi}}
\flushright{\begin{Arabic}
\quranayah[14][17]
\end{Arabic}}
\flushleft{\begin{hindi}
जिसे वह कठिनाई से घूँट-घूँट करके पिएगा और ऐसा नहीं लगेगा कि वह आसानी से उसे उतार सकता है, और मृत्यु उसपर हर ओर से चली आती होगी, फिर भी वह मरेगा नहीं। और उसके सामने कठोर यातना होगी
\end{hindi}}
\flushright{\begin{Arabic}
\quranayah[14][18]
\end{Arabic}}
\flushleft{\begin{hindi}
जिन लोगों ने अपने रब का इनकार किया उनकी मिसाल यह है कि उनके कर्म जैसे राख हों जिसपर आँधी के दिन प्रचंड हवा का झोंका चले। कुछ भी उन्हें अपनी कमाई में से हाथ न आ सकेगा। यही परले दर्जे की तबाही और गुमराही है
\end{hindi}}
\flushright{\begin{Arabic}
\quranayah[14][19]
\end{Arabic}}
\flushleft{\begin{hindi}
क्या तुमने देखा नहीं कि अल्लाह ने आकाशों और धरती को सोद्देश्य पैदा किया? यदि वह चाहे तो तुम सबको ले जाए और एक नवीन सृष्टा जनसमूह ले आए
\end{hindi}}
\flushright{\begin{Arabic}
\quranayah[14][20]
\end{Arabic}}
\flushleft{\begin{hindi}
और यह अल्लाह के लिए कुछ भी कठिन नहीं है
\end{hindi}}
\flushright{\begin{Arabic}
\quranayah[14][21]
\end{Arabic}}
\flushleft{\begin{hindi}
सबके सब अल्लाह के सामने खुलकर आ जाएँगे तो कमज़ोर लोग, उन लोगों से जो बड़े बने हुए थे, कहेंगे, "हम तो तुम्हारे पीछे चलते थे। तो क्या तुम अल्लाह की यातना में से कुछ हमपर टाल सकते हो? वे कहेंगे, "यदि अल्लाह हमें मार्ग दिखाता तो हम तुम्हें भी दिखाते। अब यदि हम व्याकुल हों या धैर्य से काम लें, हमारे लिए बराबर है। हमारे लिए बचने का कोई उपाय नहीं।"
\end{hindi}}
\flushright{\begin{Arabic}
\quranayah[14][22]
\end{Arabic}}
\flushleft{\begin{hindi}
जब मामले का फ़ैसला हो चुकेगा तब शैतान कहेगा, "अल्लाह ने तो तुमसे सच्चा वादा किया था और मैंने भी तुमसे वादा किया था, फिर मैंने तो तुमसे सत्य के प्रतिकूल कहा था। और मेरा तो तुमपर कोई अधिकार नहीं था, सिवाय इसके कि मैंने मान ली; बल्कि अपने आप ही को मलामत करो, न मैं तुम्हारी फ़रियाद सुन सकता हूँ और न तुम मेरी फ़रियाद सुन सकते हो। पहले जो तुमने सहभागी ठहराया था, मैं उससे विरक्त हूँ।" निश्चय ही अत्याचारियों के लिए दुखदायिनी यातना है
\end{hindi}}
\flushright{\begin{Arabic}
\quranayah[14][23]
\end{Arabic}}
\flushleft{\begin{hindi}
इसके विपरीत जो लोग ईमान लाए और उन्होंने अच्छे कर्म किए वे ऐसे बाग़ों में प्रवेश करेंगे जिनके नीचे नहरें बह रही होंगी। उनमें वे अपने रब की अनुमति से सदैव रहेंगे। वहाँ उनका अभिवादन 'सलाम' से होगा
\end{hindi}}
\flushright{\begin{Arabic}
\quranayah[14][24]
\end{Arabic}}
\flushleft{\begin{hindi}
क्या तुमने देखा नहीं कि अल्लाह ने कैसी मिसाल पेश की? अच्छी उत्तम बात एक अच्छे शुभ वृक्ष के सदृश है, जिसकी जड़ गहरी जमी हुई हो और उसकी शाखाएँ आकाश में पहुँची हुई हों;
\end{hindi}}
\flushright{\begin{Arabic}
\quranayah[14][25]
\end{Arabic}}
\flushleft{\begin{hindi}
अपने रब की अनुमति से वह हर समय अपना फल दे रहा हो। अल्लाह तो लोगों के लिए मिशालें पेश करता है, ताकि वे जाग्रत हों
\end{hindi}}
\flushright{\begin{Arabic}
\quranayah[14][26]
\end{Arabic}}
\flushleft{\begin{hindi}
और अशुभ एंव अशुद्ध बात की मिसाल एक अशुभ वृक्ष के सदृश है, जिसे धरती के ऊपर ही से उखाड़ लिया जाए और उसे कुछ भी स्थिरता प्राप्त न हो
\end{hindi}}
\flushright{\begin{Arabic}
\quranayah[14][27]
\end{Arabic}}
\flushleft{\begin{hindi}
ईमान लानेवालों को अल्लाह सुदृढ़ बात के द्वारा सांसारिक जीवन में भी परलोक में भी सुदृढ़ता प्रदान करता है और अत्याचारियों को अल्लाह विचलित कर देता है। और अल्लाह जो चाहता है, करता है
\end{hindi}}
\flushright{\begin{Arabic}
\quranayah[14][28]
\end{Arabic}}
\flushleft{\begin{hindi}
क्या तुमने उन लोगों को नहीं देखा जिन्होंने अल्लाह की नेमत को कुफ़्र से बदल डाला औऱ अपनी क़ौम को विनाश-गृह में उतार दिया;
\end{hindi}}
\flushright{\begin{Arabic}
\quranayah[14][29]
\end{Arabic}}
\flushleft{\begin{hindi}
जहन्नम में, जिसमें वे झोंके जाएँगे और वह अत्यन्त बुरा ठिकाना है!
\end{hindi}}
\flushright{\begin{Arabic}
\quranayah[14][30]
\end{Arabic}}
\flushleft{\begin{hindi}
और उन्होंने अल्लाह के प्रतिद्वन्दी बना दिए, ताकि परिणामस्वरूप वे उन्हें उसके मार्ग से भटका दें। कह दो, "थोड़े दिन मज़े ले लो। अन्ततः तुम्हें आग ही की ओर जाना है।"
\end{hindi}}
\flushright{\begin{Arabic}
\quranayah[14][31]
\end{Arabic}}
\flushleft{\begin{hindi}
मेरे जो बन्दे ईमान लाए है उनसे कह दो कि वे नमाज़ की पाबन्दी करें और हमने उन्हें जो कुछ दिया है उसमें से छुपे और खुले ख़र्च करें, इससे पहले कि वह दिन आ जाए जिनमें न कोई क्रय-विक्रय होगा और न मैत्री
\end{hindi}}
\flushright{\begin{Arabic}
\quranayah[14][32]
\end{Arabic}}
\flushleft{\begin{hindi}
वह अल्लाह ही है जिसने आकाशों और धरती की सृष्टि की और आकाश से पानी उतारा, फिर वह उसके द्वारा कितने ही पैदावार और फल तुम्हारी आजीविका के रूप में सामने लाया। और नौका को तुम्हारे काम में लगाया, ताकि समुद्र में उसके आदेश से चले और नदियों को भी तुम्हें लाभ पहुँचाने में लगाया
\end{hindi}}
\flushright{\begin{Arabic}
\quranayah[14][33]
\end{Arabic}}
\flushleft{\begin{hindi}
और सूर्य और चन्द्रमा को तुम्हारे लिए कार्यरत किया और एक नियत विधान के अधीन निरंतर गतिशील है। और रात औऱ दिन को भी तुम्हें लाभ पहुँचाने में लगा रखा है
\end{hindi}}
\flushright{\begin{Arabic}
\quranayah[14][34]
\end{Arabic}}
\flushleft{\begin{hindi}
और हर उस चीज़ में से तुम्हें दिया जो तुमने उससे माँगा यदि तुम अल्लाह की नेमतों की गणना नहीं कर सकते। वास्तव में मनुष्य ही बड़ा ही अन्यायी, कृतघ्न है
\end{hindi}}
\flushright{\begin{Arabic}
\quranayah[14][35]
\end{Arabic}}
\flushleft{\begin{hindi}
याद करो जब इबराहीम ने कहा था, "मेरे रब! इस भूभाग (मक्का) को शान्तिमय बना दे और मुझे और मेरी सन्तान को इससे बचा कि हम मूर्तियों को पूजने लग जाए
\end{hindi}}
\flushright{\begin{Arabic}
\quranayah[14][36]
\end{Arabic}}
\flushleft{\begin{hindi}
मेरे रब! इन्होंने (इन मूर्तियों नॆ) बहुत से लोगों को पथभ्रष्ट किया है। अतः जिस किसी ने मॆरा अनुसरण किया वह मेरा है और जिस ने मेरी अवज्ञा की तो निश्चय ही तू बड़ा क्षमाशील, अत्यन्त दयावान है
\end{hindi}}
\flushright{\begin{Arabic}
\quranayah[14][37]
\end{Arabic}}
\flushleft{\begin{hindi}
मेरे रब! मैंने एक ऐसी घाटी में जहाँ कृषि-योग्य भूमि नहीं अपनी सन्तान के एक हिस्से को तेरे प्रतिष्ठित घर (काबा) के निकट बसा दिया है। हमारे रब! ताकि वे नमाज़ क़ायम करें। अतः तू लोगों के दिलों को उनकी ओर झुका दे और उन्हें फलों और पैदावार की आजीविका प्रदान कर, ताकि वे कृतज्ञ बने
\end{hindi}}
\flushright{\begin{Arabic}
\quranayah[14][38]
\end{Arabic}}
\flushleft{\begin{hindi}
हमारे रब! तू जानता ही है जो कुछ हम छिपाते है और जो कुछ प्रकट करते है। अल्लाह से तो कोई चीज़ न धरती में छिपी है और न आकाश में
\end{hindi}}
\flushright{\begin{Arabic}
\quranayah[14][39]
\end{Arabic}}
\flushleft{\begin{hindi}
सारी प्रशंसा है उस अल्लाह की जिसने बुढ़ापे के होते हुए भी मुझे इसमाईल और इसहाक़ दिए। निस्संदेह मेरा रब प्रार्थना अवश्य सुनता है
\end{hindi}}
\flushright{\begin{Arabic}
\quranayah[14][40]
\end{Arabic}}
\flushleft{\begin{hindi}
मेरे रब! मुझे और मेरी सन्तान को नमाज़ क़ायम करनेवाला बना। हमारे रब! और हमारी प्रार्थना स्वीकार कर
\end{hindi}}
\flushright{\begin{Arabic}
\quranayah[14][41]
\end{Arabic}}
\flushleft{\begin{hindi}
हमारे रब! मुझे और मेरे माँ-बाप को और मोमिनों को उस दिन क्षमाकर देना, जिस दिन हिसाब का मामला पेश आएगा।"
\end{hindi}}
\flushright{\begin{Arabic}
\quranayah[14][42]
\end{Arabic}}
\flushleft{\begin{hindi}
अब ये अत्याचारी जो कुछ कर रहे है, उससे अल्लाह को असावधान न समझो। वह तो इन्हें बस उस दिन तक के लिए टाल रहा है जबकि आँखे फटी की फटी रह जाएँगी,
\end{hindi}}
\flushright{\begin{Arabic}
\quranayah[14][43]
\end{Arabic}}
\flushleft{\begin{hindi}
अपने सिर उठाए भागे चले जा रहे होंगे; उनकी निगाह स्वयं उनकी अपनी ओर भी न फिरेगी और उनके दिल उड़े जा रहे होंगे
\end{hindi}}
\flushright{\begin{Arabic}
\quranayah[14][44]
\end{Arabic}}
\flushleft{\begin{hindi}
लोगों को उस दिन से डराओ, जब यातना उन्हें आ लेगी। उस समय अत्याचारी लोग कहेंगे, "हमारे रब! हमें थोड़ी-सी मुहलत दे दे। हम तेरे आमंत्रण को स्वीकार करेंगे और रसूलों का अनुसरण करेंगे।" कहा जाएगा, "क्या तुम इससे पहले क़समें नहीं खाया करते थे कि हमारा तो पतन ही न होगा?"
\end{hindi}}
\flushright{\begin{Arabic}
\quranayah[14][45]
\end{Arabic}}
\flushleft{\begin{hindi}
तुम लोगों की बस्तियों में रह-बस चुके थे, जिन्होंने अपने ऊपर अत्याचार किया था और तुमपर अच्छी तरह स्पष्ट हो चुका था कि उनके साथ हमने कैसा मामला किया और हमने तुम्हारे लिए कितनी ही मिशालें बयान की थी।"
\end{hindi}}
\flushright{\begin{Arabic}
\quranayah[14][46]
\end{Arabic}}
\flushleft{\begin{hindi}
वे अपनी चाल चल चुक हैं। अल्लाह के पास भी उनके लिए चाल मौजूद थी, यद्यपि उनकी चाल ऐसी ही क्यों न रही हो जिससे पर्वत भी अपने स्थान से टल जाएँ
\end{hindi}}
\flushright{\begin{Arabic}
\quranayah[14][47]
\end{Arabic}}
\flushleft{\begin{hindi}
अतः यह न समझना कि अल्लाह अपने रसूलों से किए हुए अपने वादे के विरुद्ध जाएगा। अल्लाह तो अपार शक्तिवाला, प्रतिशोधक है
\end{hindi}}
\flushright{\begin{Arabic}
\quranayah[14][48]
\end{Arabic}}
\flushleft{\begin{hindi}
जिस दिन यह धरती दूसरी धरती से बदल दी जाएगी और आकाश भी। और वे सब के सब अल्लाह के सामने खुलकर आ जाएँगे, जो अकेला है, सबपर जिसका आधिपत्य है
\end{hindi}}
\flushright{\begin{Arabic}
\quranayah[14][49]
\end{Arabic}}
\flushleft{\begin{hindi}
और उस दिन तुम अपराधियों को देखोगे कि ज़ंजीरों में जकड़े हुए है
\end{hindi}}
\flushright{\begin{Arabic}
\quranayah[14][50]
\end{Arabic}}
\flushleft{\begin{hindi}
उनके परिधान तारकोल के होंगे और आग उनके चहरों पर छा रही होगी,
\end{hindi}}
\flushright{\begin{Arabic}
\quranayah[14][51]
\end{Arabic}}
\flushleft{\begin{hindi}
ताकि अल्लाह प्रत्येक जीव को उसकी कमाई का बदला दे। निश्चय ही अल्लाह जल्द हिसाब लेनेवाला है
\end{hindi}}
\flushright{\begin{Arabic}
\quranayah[14][52]
\end{Arabic}}
\flushleft{\begin{hindi}
यह लोगों को सन्देश पहुँचा देना है (ताकि वे इसे ध्यानपूर्वक सुनें) और ताकि उन्हें इसके द्वारा सावधान कर दिया जाए और ताकि वे जान लें कि वही अकेला पूज्य है और ताकि वे सचेत हो जाएँ, तो बुद्धि और समझ रखते है
\end{hindi}}
\chapter{Al-Hijr (The Rock)}
\begin{Arabic}
\Huge{\centerline{\basmalah}}\end{Arabic}
\flushright{\begin{Arabic}
\quranayah[15][1]
\end{Arabic}}
\flushleft{\begin{hindi}
अलिफ़॰ लाम॰ रा॰। यह किताब अर्थात स्पष्ट क़ुरआन की आयतें हैं
\end{hindi}}
\flushright{\begin{Arabic}
\quranayah[15][2]
\end{Arabic}}
\flushleft{\begin{hindi}
ऐसे समय आएँगे जब इनकार करनेवाले कामना करेंगे कि क्या ही अच्छा होता कि हम मुस्लिम (आज्ञाकारी) होते!
\end{hindi}}
\flushright{\begin{Arabic}
\quranayah[15][3]
\end{Arabic}}
\flushleft{\begin{hindi}
छोड़ो उन्हें खाएँ और मज़े उड़ाएँ और (लम्बी) आशा उन्हें भुलावे में डाले रखे। उन्हें जल्द ही मालूम हो जाएगा!
\end{hindi}}
\flushright{\begin{Arabic}
\quranayah[15][4]
\end{Arabic}}
\flushleft{\begin{hindi}
हमने जिस बस्ती को भी विनष्ट किया है, उसके लिए अनिवार्यतः एक निश्चित फ़ैसला रहा है!
\end{hindi}}
\flushright{\begin{Arabic}
\quranayah[15][5]
\end{Arabic}}
\flushleft{\begin{hindi}
किसी समुदाय के लोग न अपने निश्चि‍त समय से आगे बढ़ सकते है और न वे पीछे रह सकते है
\end{hindi}}
\flushright{\begin{Arabic}
\quranayah[15][6]
\end{Arabic}}
\flushleft{\begin{hindi}
वे कहते है, "ऐ व्यक्ति, जिसपर अनुस्मरण अवतरित हुआ, तुम निश्चय ही दीवाने हो!
\end{hindi}}
\flushright{\begin{Arabic}
\quranayah[15][7]
\end{Arabic}}
\flushleft{\begin{hindi}
यदि तुम सच्चे हो तो हमारे समक्ष फ़रिश्तों को क्यों नहीं ले आते?"
\end{hindi}}
\flushright{\begin{Arabic}
\quranayah[15][8]
\end{Arabic}}
\flushleft{\begin{hindi}
फ़रिश्तों को हम केवल सत्य के प्रयोजन हेतु उतारते है और उस समय लोगों को मुहलत नहीं मिलेगी
\end{hindi}}
\flushright{\begin{Arabic}
\quranayah[15][9]
\end{Arabic}}
\flushleft{\begin{hindi}
यह अनुसरण निश्चय ही हमने अवतरित किया है और हम स्वयं इसके रक्षक हैं
\end{hindi}}
\flushright{\begin{Arabic}
\quranayah[15][10]
\end{Arabic}}
\flushleft{\begin{hindi}
तुमसे पहले कितने ही विगत गिरोंहों में हम रसूल भेज चुके है
\end{hindi}}
\flushright{\begin{Arabic}
\quranayah[15][11]
\end{Arabic}}
\flushleft{\begin{hindi}
कोई भी रसूल उनके पास ऐसा नहीं आया, जिसका उन्होंने उपहास न किया हो
\end{hindi}}
\flushright{\begin{Arabic}
\quranayah[15][12]
\end{Arabic}}
\flushleft{\begin{hindi}
इसी तरह हम अपराधियों के दिलों में इसे उतारते है
\end{hindi}}
\flushright{\begin{Arabic}
\quranayah[15][13]
\end{Arabic}}
\flushleft{\begin{hindi}
वे इसे मानेंगे नहीं। पहले के लोगों की मिसालें गुज़र चुकी हैं
\end{hindi}}
\flushright{\begin{Arabic}
\quranayah[15][14]
\end{Arabic}}
\flushleft{\begin{hindi}
यदि हम उनपर आकाश से कोई द्वार खोल दें और वे दिन-दहाड़े उसमें चढ़ने भी लगें,
\end{hindi}}
\flushright{\begin{Arabic}
\quranayah[15][15]
\end{Arabic}}
\flushleft{\begin{hindi}
फिर भी वे यही कहेंगे, "हमारी आँखें मदमाती हैं, बल्कि हम लोगों पर जादू कर दिया गया है!"
\end{hindi}}
\flushright{\begin{Arabic}
\quranayah[15][16]
\end{Arabic}}
\flushleft{\begin{hindi}
हमने आकाश में बुर्ज (तारा-समूह) बनाए और हमने उसे देखनेवालों के लिए सुसज्जित भी किया
\end{hindi}}
\flushright{\begin{Arabic}
\quranayah[15][17]
\end{Arabic}}
\flushleft{\begin{hindi}
और हर फिटकारे हुए शैतान से उसे सुरक्षित रखा -
\end{hindi}}
\flushright{\begin{Arabic}
\quranayah[15][18]
\end{Arabic}}
\flushleft{\begin{hindi}
यह और बात है कि किसी ने चोरी-छिपे कुछ सुनगुन ले लिया तो एक प्रत्यक्ष अग्निशिखा ने भी झपटकर उसका पीछा किया -
\end{hindi}}
\flushright{\begin{Arabic}
\quranayah[15][19]
\end{Arabic}}
\flushleft{\begin{hindi}
और हमने धरती को फैलाया और उसमें अटल पहाड़ डाल दिए और उसमें हर चीज़ नपे-तुले अन्दाज़ में उगाई
\end{hindi}}
\flushright{\begin{Arabic}
\quranayah[15][20]
\end{Arabic}}
\flushleft{\begin{hindi}
और उसमें तुम्हारे गुज़र-बसर के सामान निर्मित किए, और उनको भी जिनको रोज़ी देनेवाले तुम नहीं हो
\end{hindi}}
\flushright{\begin{Arabic}
\quranayah[15][21]
\end{Arabic}}
\flushleft{\begin{hindi}
कोई भी चीज़ तो ऐसी नहीं है जिसके भंडार हमारे पास न हों, फिर भी हम उसे एक ज्ञात (निश्चिंत) मात्रा के साथ उतारते है
\end{hindi}}
\flushright{\begin{Arabic}
\quranayah[15][22]
\end{Arabic}}
\flushleft{\begin{hindi}
हम ही वर्षा लानेवाली हवाओं को भेजते है। फिर आकाश से पानी बरसाते है और उससे तुम्हें सिंचित करते है। उसके ख़जानादार तुम नहीं हो
\end{hindi}}
\flushright{\begin{Arabic}
\quranayah[15][23]
\end{Arabic}}
\flushleft{\begin{hindi}
हम ही जीवन और मृत्यु देते है और हम ही उत्तराधिकारी रह जाते है
\end{hindi}}
\flushright{\begin{Arabic}
\quranayah[15][24]
\end{Arabic}}
\flushleft{\begin{hindi}
हम तुम्हारे पहले के लोगों को भी जानते है और बाद के आनेवालों को भी हम जानते है
\end{hindi}}
\flushright{\begin{Arabic}
\quranayah[15][25]
\end{Arabic}}
\flushleft{\begin{hindi}
तुम्हारा रब ही है, जो उन्हें इकट्ठा करेगा। निस्संदेह वह तत्वदर्शी, सर्वज्ञ है
\end{hindi}}
\flushright{\begin{Arabic}
\quranayah[15][26]
\end{Arabic}}
\flushleft{\begin{hindi}
हमने मनुष्य को सड़े हुए गारे की खनखनाती हुई मिट्टी से बनाया है,
\end{hindi}}
\flushright{\begin{Arabic}
\quranayah[15][27]
\end{Arabic}}
\flushleft{\begin{hindi}
और उससे पहले हम जिन्नों को लू रूपी अग्नि से पैदा कर चुके थे
\end{hindi}}
\flushright{\begin{Arabic}
\quranayah[15][28]
\end{Arabic}}
\flushleft{\begin{hindi}
याद करो जब तुम्हारे रब ने फ़रिश्तों से कहा, "मैं सड़े हुए गारे की खनखनाती हुई मिट्टी से एक मनुष्य पैदा करनेवाला हूँ
\end{hindi}}
\flushright{\begin{Arabic}
\quranayah[15][29]
\end{Arabic}}
\flushleft{\begin{hindi}
तो जब मैं उसे पूरा बना चुकूँ और उसमें अपनी रूह फूँक दूँ तो तुम उसके आगे सजदे में गिर जाना!"
\end{hindi}}
\flushright{\begin{Arabic}
\quranayah[15][30]
\end{Arabic}}
\flushleft{\begin{hindi}
अतएव सब के सब फ़रिश्तो ने सजदा किया,
\end{hindi}}
\flushright{\begin{Arabic}
\quranayah[15][31]
\end{Arabic}}
\flushleft{\begin{hindi}
सिवाय इबलीस के। उसने सजदा करनेवालों के साथ शामिल होने से इनकार कर दिया
\end{hindi}}
\flushright{\begin{Arabic}
\quranayah[15][32]
\end{Arabic}}
\flushleft{\begin{hindi}
कहा, "ऐ इबलीस! तुझे क्या हुआ कि तू सजदा करनेवालों में शामिल नहीं हुआ?"
\end{hindi}}
\flushright{\begin{Arabic}
\quranayah[15][33]
\end{Arabic}}
\flushleft{\begin{hindi}
उसने कहा, "मैं ऐसा नहीं हूँ कि मैं उस मनुष्य को सजदा करूँ जिसको तू ने सड़े हुए गारे की खनखनाती हुए मिट्टी से बनाया।"
\end{hindi}}
\flushright{\begin{Arabic}
\quranayah[15][34]
\end{Arabic}}
\flushleft{\begin{hindi}
कहा, "अच्छा, तू निकल जा यहाँ से, क्योंकि तुझपर फिटकार है!
\end{hindi}}
\flushright{\begin{Arabic}
\quranayah[15][35]
\end{Arabic}}
\flushleft{\begin{hindi}
निश्चय ही बदले के दिन तक तुझ पर धिक्कार है।"
\end{hindi}}
\flushright{\begin{Arabic}
\quranayah[15][36]
\end{Arabic}}
\flushleft{\begin{hindi}
उसने कहा, "मेरे रब! फिर तू मुझे उस दिन तक के लिए मुहलत दे, जबकि सब उठाए जाएँगे।"
\end{hindi}}
\flushright{\begin{Arabic}
\quranayah[15][37]
\end{Arabic}}
\flushleft{\begin{hindi}
कहा, "अच्छा, तुझे मुहलत है,
\end{hindi}}
\flushright{\begin{Arabic}
\quranayah[15][38]
\end{Arabic}}
\flushleft{\begin{hindi}
उस दिन तक के लिए जिसका समय ज्ञात एवं नियत है।"
\end{hindi}}
\flushright{\begin{Arabic}
\quranayah[15][39]
\end{Arabic}}
\flushleft{\begin{hindi}
उसने कहा, "मेरे रब! इसलिए कि तूने मुझे सीधे मार्ग से विचलित कर दिया है, अतः मैं भी धरती में उनके लिए मनमोहकता पैदा करूँगा और उन सबको बहकाकर रहूँगा,
\end{hindi}}
\flushright{\begin{Arabic}
\quranayah[15][40]
\end{Arabic}}
\flushleft{\begin{hindi}
सिवाय उनके जो तेरे चुने हुए बन्दे होंगे।"
\end{hindi}}
\flushright{\begin{Arabic}
\quranayah[15][41]
\end{Arabic}}
\flushleft{\begin{hindi}
कहा, "मुझ तक पहुँचने का यही सीधा मार्ग है,
\end{hindi}}
\flushright{\begin{Arabic}
\quranayah[15][42]
\end{Arabic}}
\flushleft{\begin{hindi}
मेरे बन्दों पर तो तेरा कुछ ज़ोर न चलेगा, सिवाय उन बहके हुए लोगों को जो तेरे पीछे हो लें
\end{hindi}}
\flushright{\begin{Arabic}
\quranayah[15][43]
\end{Arabic}}
\flushleft{\begin{hindi}
निश्चय ही जहन्नम ही का ऐसे समस्त लोगों से वादा है
\end{hindi}}
\flushright{\begin{Arabic}
\quranayah[15][44]
\end{Arabic}}
\flushleft{\begin{hindi}
उसके सात द्वार है। प्रत्येक द्वार के लिए एक ख़ास हिस्सा होगा।"
\end{hindi}}
\flushright{\begin{Arabic}
\quranayah[15][45]
\end{Arabic}}
\flushleft{\begin{hindi}
निस्संदेह डर रखनेवाले बाग़ों और स्रोतों में होंगे,
\end{hindi}}
\flushright{\begin{Arabic}
\quranayah[15][46]
\end{Arabic}}
\flushleft{\begin{hindi}
"प्रवेश करो इनमें निर्भयतापूर्वक सलामती के साथ!"
\end{hindi}}
\flushright{\begin{Arabic}
\quranayah[15][47]
\end{Arabic}}
\flushleft{\begin{hindi}
उनके सीनों में जो मन-मुटाव होगा उसे हम दूर कर देंगे। वे भाई-भाई बनकर आमने-सामने तख़्तों पर होंगे
\end{hindi}}
\flushright{\begin{Arabic}
\quranayah[15][48]
\end{Arabic}}
\flushleft{\begin{hindi}
उन्हें वहाँ न तो कोई थकान और तकलीफ़ पहुँचेगी औऱ न वे वहाँ से कभी निकाले ही जाएँगे
\end{hindi}}
\flushright{\begin{Arabic}
\quranayah[15][49]
\end{Arabic}}
\flushleft{\begin{hindi}
मेरे बन्दों को सूचित कर दो कि मैं अत्यन्त क्षमाशील, दयावान हूँ;
\end{hindi}}
\flushright{\begin{Arabic}
\quranayah[15][50]
\end{Arabic}}
\flushleft{\begin{hindi}
और यह कि मेरी यातना भी अत्यन्त दुखदायिनी यातना है
\end{hindi}}
\flushright{\begin{Arabic}
\quranayah[15][51]
\end{Arabic}}
\flushleft{\begin{hindi}
और उन्हें इबराहीम के अतिथियों का वृत्तान्त सुनाओ,
\end{hindi}}
\flushright{\begin{Arabic}
\quranayah[15][52]
\end{Arabic}}
\flushleft{\begin{hindi}
जब वे उसके यहाँ आए और उन्होंने सलाम किया तो उसने कहा, "हमें तो तुमसे डर लग रहा है।"
\end{hindi}}
\flushright{\begin{Arabic}
\quranayah[15][53]
\end{Arabic}}
\flushleft{\begin{hindi}
वे बोले, "डरो नहीं, हम तुम्हें एक ज्ञानवान पुत्र की शुभ सूचना देते है।"
\end{hindi}}
\flushright{\begin{Arabic}
\quranayah[15][54]
\end{Arabic}}
\flushleft{\begin{hindi}
उसने कहा, "क्या तुम मुझे शुभ सूचना दे रहे हो, इस अवस्था में कि मेरा बुढापा आ गया है? तो अब मुझे किस बात की शुभ सूचना दे रहे हो?"
\end{hindi}}
\flushright{\begin{Arabic}
\quranayah[15][55]
\end{Arabic}}
\flushleft{\begin{hindi}
उन्होंने कहा, "हम तुम्हें सच्ची शुभ सूचना दे रहे हैं, तो तुम निराश न हो"
\end{hindi}}
\flushright{\begin{Arabic}
\quranayah[15][56]
\end{Arabic}}
\flushleft{\begin{hindi}
उसने कहा, "अपने रब की दयालुता से पथभ्रष्टों के सिवा और कौन निराश होगा?"
\end{hindi}}
\flushright{\begin{Arabic}
\quranayah[15][57]
\end{Arabic}}
\flushleft{\begin{hindi}
उसने कहा, "ऐ दूतो, तुम किस अभियान पर आए हो?"
\end{hindi}}
\flushright{\begin{Arabic}
\quranayah[15][58]
\end{Arabic}}
\flushleft{\begin{hindi}
वे बोले, "हम तो एक अपराधी क़ौम की ओर भेजे गए है,
\end{hindi}}
\flushright{\begin{Arabic}
\quranayah[15][59]
\end{Arabic}}
\flushleft{\begin{hindi}
सिवाय लूत के घरवालों के। उन सबको तो हम बचा लेंगे,
\end{hindi}}
\flushright{\begin{Arabic}
\quranayah[15][60]
\end{Arabic}}
\flushleft{\begin{hindi}
सिवाय उसकी पत्नी के - हमने निश्चित कर दिया है, वह तो पीछे रह जानेवालों में रहेंगी।"
\end{hindi}}
\flushright{\begin{Arabic}
\quranayah[15][61]
\end{Arabic}}
\flushleft{\begin{hindi}
फिर जब ये दूत लूत के यहाँ पहुँचे,
\end{hindi}}
\flushright{\begin{Arabic}
\quranayah[15][62]
\end{Arabic}}
\flushleft{\begin{hindi}
तो उसने कहा, "तुम तो अपरिचित लोग हो।"
\end{hindi}}
\flushright{\begin{Arabic}
\quranayah[15][63]
\end{Arabic}}
\flushleft{\begin{hindi}
उन्होंने कहा, "नहीं, बल्कि हम तो तुम्हारे पास वही चीज़ लेकर आए है, जिसके विषय में वे सन्देह कर रहे थे
\end{hindi}}
\flushright{\begin{Arabic}
\quranayah[15][64]
\end{Arabic}}
\flushleft{\begin{hindi}
और हम तुम्हारे पास यक़ीनी चीज़ लेकर आए है, और हम बिलकुल सच कह रहे है
\end{hindi}}
\flushright{\begin{Arabic}
\quranayah[15][65]
\end{Arabic}}
\flushleft{\begin{hindi}
अतएव अब तुम अपने घरवालों को लेकर रात्रि के किसी हिस्से में निकल जाओ, और स्वयं उन सबके पीछे-पीछे चलो। और तुममें से कोई भी पीछे मुड़कर न देखे। बस चले जाओ, जिधर का तुम्हे आदेश है।"
\end{hindi}}
\flushright{\begin{Arabic}
\quranayah[15][66]
\end{Arabic}}
\flushleft{\begin{hindi}
हमने उसे अपना यह फ़ैसला पहुँचा दिया कि प्रातः होते-होते उनकी जड़ कट चुकी होगी
\end{hindi}}
\flushright{\begin{Arabic}
\quranayah[15][67]
\end{Arabic}}
\flushleft{\begin{hindi}
इतने में नगर के लोग ख़ुश-ख़ुश आ पहुँचे
\end{hindi}}
\flushright{\begin{Arabic}
\quranayah[15][68]
\end{Arabic}}
\flushleft{\begin{hindi}
उसने कहा, "ये मेरे अतिथि है। मेरी फ़ज़ीहत मत करना,
\end{hindi}}
\flushright{\begin{Arabic}
\quranayah[15][69]
\end{Arabic}}
\flushleft{\begin{hindi}
अल्लाह का डर ऱखो, मुझे रुसवा न करो।"
\end{hindi}}
\flushright{\begin{Arabic}
\quranayah[15][70]
\end{Arabic}}
\flushleft{\begin{hindi}
उन्होंने कहा, "क्या हमने तुम्हें दुनिया भर के लोगों का ज़िम्मा लेने से रोका नहीं था?"
\end{hindi}}
\flushright{\begin{Arabic}
\quranayah[15][71]
\end{Arabic}}
\flushleft{\begin{hindi}
उसने कहा, "तुमको यदि कुछ करना है, तो ये मेरी (क़ौम की) बेटियाँ (विधितः विवाह के लिए) मौजूद है।"
\end{hindi}}
\flushright{\begin{Arabic}
\quranayah[15][72]
\end{Arabic}}
\flushleft{\begin{hindi}
तुम्हारे जीवन की सौगन्ध, वे अपनी मस्ती में खोए हुए थे,
\end{hindi}}
\flushright{\begin{Arabic}
\quranayah[15][73]
\end{Arabic}}
\flushleft{\begin{hindi}
अन्ततः पौ फटते-फटते एक भयंकर आवाज़ ने उन्हें आ लिया,
\end{hindi}}
\flushright{\begin{Arabic}
\quranayah[15][74]
\end{Arabic}}
\flushleft{\begin{hindi}
और हमने उस बस्ती को तलपट कर दिया, और उनपर कंकरीले पत्थर बरसाए
\end{hindi}}
\flushright{\begin{Arabic}
\quranayah[15][75]
\end{Arabic}}
\flushleft{\begin{hindi}
निश्चय ही इसमें भापनेवालों के लिए निशानियाँ है
\end{hindi}}
\flushright{\begin{Arabic}
\quranayah[15][76]
\end{Arabic}}
\flushleft{\begin{hindi}
और वह (बस्ती) सार्वजनिक मार्ग पर है
\end{hindi}}
\flushright{\begin{Arabic}
\quranayah[15][77]
\end{Arabic}}
\flushleft{\begin{hindi}
निश्चय ही इसमें मोमिनों के लिए एक बड़ी निशानी है
\end{hindi}}
\flushright{\begin{Arabic}
\quranayah[15][78]
\end{Arabic}}
\flushleft{\begin{hindi}
और निश्चय ही ऐसा वाले भी अत्याचारी थे,
\end{hindi}}
\flushright{\begin{Arabic}
\quranayah[15][79]
\end{Arabic}}
\flushleft{\begin{hindi}
फिर हमने उनसे भी बदला लिया, और ये दोनों (भू-भाग) खुले मार्ग पर स्थित है
\end{hindi}}
\flushright{\begin{Arabic}
\quranayah[15][80]
\end{Arabic}}
\flushleft{\begin{hindi}
हिज्रवाले भी रसूलों को झुठला चुके है
\end{hindi}}
\flushright{\begin{Arabic}
\quranayah[15][81]
\end{Arabic}}
\flushleft{\begin{hindi}
हमने तो उन्हें अपनी निशानियाँ प्रदान की थी, परन्तु वे उनकी उपेक्षा ही करते रहे
\end{hindi}}
\flushright{\begin{Arabic}
\quranayah[15][82]
\end{Arabic}}
\flushleft{\begin{hindi}
वे बड़ी बेफ़िक्री से पहाड़ो को काट-काटकर घर बनाते थे
\end{hindi}}
\flushright{\begin{Arabic}
\quranayah[15][83]
\end{Arabic}}
\flushleft{\begin{hindi}
अन्ततः एक भयानक आवाज़ ने प्रातः होते- होते उन्हें आ लिया
\end{hindi}}
\flushright{\begin{Arabic}
\quranayah[15][84]
\end{Arabic}}
\flushleft{\begin{hindi}
फिर जो कुछ वे कमाते रहे, वह उनके कुछ काम न आ सका
\end{hindi}}
\flushright{\begin{Arabic}
\quranayah[15][85]
\end{Arabic}}
\flushleft{\begin{hindi}
हमने तो आकाशों और धरती को और जो कुछ उनके मध्य है, सोद्देश्य पैदा किया है, और वह क़ियामत की घड़ी तो अनिवार्यतः आनेवाली है। अतः तुम भली प्रकार दरगुज़र (क्षमा) से काम लो
\end{hindi}}
\flushright{\begin{Arabic}
\quranayah[15][86]
\end{Arabic}}
\flushleft{\begin{hindi}
निश्चय ही तुम्हारा रब ही बड़ा पैदा करनेवाला, सब कुछ जाननेवाला है
\end{hindi}}
\flushright{\begin{Arabic}
\quranayah[15][87]
\end{Arabic}}
\flushleft{\begin{hindi}
हमने तुम्हें सात 'मसानी' का समूह यानी महान क़ुरआन दिया-
\end{hindi}}
\flushright{\begin{Arabic}
\quranayah[15][88]
\end{Arabic}}
\flushleft{\begin{hindi}
जो कुछ सुख-सामग्री हमने उनमें से विभिन्न प्रकार के लोगों को दी है, तुम उसपर अपनी आँखें न पसारो और न उनपर दुखी हो, तुम तो अपनी भुजाएँ मोमिनों के लिए झुकाए रखो,
\end{hindi}}
\flushright{\begin{Arabic}
\quranayah[15][89]
\end{Arabic}}
\flushleft{\begin{hindi}
और कह दो, "मैं तो साफ़-साफ़ चेतावनी देनेवाला हूँ।"
\end{hindi}}
\flushright{\begin{Arabic}
\quranayah[15][90]
\end{Arabic}}
\flushleft{\begin{hindi}
जिस प्रकार हमने हिस्सा-बख़रा करनेवालों पर उतारा था,
\end{hindi}}
\flushright{\begin{Arabic}
\quranayah[15][91]
\end{Arabic}}
\flushleft{\begin{hindi}
जिन्होंने (अपने) क़ुरआन को टुकड़े-टुकड़े कर डाला
\end{hindi}}
\flushright{\begin{Arabic}
\quranayah[15][92]
\end{Arabic}}
\flushleft{\begin{hindi}
अब तुम्हारे रब की क़सम! हम अवश्य ही उन सबसे उसके विषय में पूछेंगे
\end{hindi}}
\flushright{\begin{Arabic}
\quranayah[15][93]
\end{Arabic}}
\flushleft{\begin{hindi}
जो कुछ वे करते रहे।
\end{hindi}}
\flushright{\begin{Arabic}
\quranayah[15][94]
\end{Arabic}}
\flushleft{\begin{hindi}
अतः तु्म्हें जिस चीज़ का आदेश हुआ है, उसे हाँक-पुकारकर बयान कर दो, और मुशरिको की ओर ध्यान न दो
\end{hindi}}
\flushright{\begin{Arabic}
\quranayah[15][95]
\end{Arabic}}
\flushleft{\begin{hindi}
उपहास करनेवालों के लिए हम तुम्हारी ओर से काफ़ी है
\end{hindi}}
\flushright{\begin{Arabic}
\quranayah[15][96]
\end{Arabic}}
\flushleft{\begin{hindi}
जो अल्लाह के साथ दूसरों को पूज्य-प्रभु ठहराते है, तो शीघ्र ही उन्हें मालूम हो जाएगा!
\end{hindi}}
\flushright{\begin{Arabic}
\quranayah[15][97]
\end{Arabic}}
\flushleft{\begin{hindi}
हम जानते है कि वे जो कुछ कहते है, उससे तुम्हारा दिल तंग होता है
\end{hindi}}
\flushright{\begin{Arabic}
\quranayah[15][98]
\end{Arabic}}
\flushleft{\begin{hindi}
तो तुम अपने रब का गुणगान करो और सजदा करनेवालों में सम्मिलित रहो
\end{hindi}}
\flushright{\begin{Arabic}
\quranayah[15][99]
\end{Arabic}}
\flushleft{\begin{hindi}
और अपने रब की बन्दगी में लगे रहो, यहाँ तक कि जो यक़ीनी है, वह तुम्हारे सामने आ जाए
\end{hindi}}
\chapter{An-Nahl (The Bee)}
\begin{Arabic}
\Huge{\centerline{\basmalah}}\end{Arabic}
\flushright{\begin{Arabic}
\quranayah[16][1]
\end{Arabic}}
\flushleft{\begin{hindi}
आ गया आदेश अल्लाह का, तो अब उसके लिए जल्दी न मचाओ। वह महान और उच्च है उस शिर्क से जो व कर रहे है
\end{hindi}}
\flushright{\begin{Arabic}
\quranayah[16][2]
\end{Arabic}}
\flushleft{\begin{hindi}
वह फ़रिश्तों को अपने हुक्म की रूह (वह्यल) के साथ अपने जिस बन्दे पर चाहता है उतारता है कि "सचेत कर दो, मेरे सिवा कोई पूज्य-प्रभु नहीं। अतः तुम मेरा ही डर रखो।"
\end{hindi}}
\flushright{\begin{Arabic}
\quranayah[16][3]
\end{Arabic}}
\flushleft{\begin{hindi}
उसने आकाशों और धरती को सोद्देश्य पैदा किया। वह अत्यन्त उच्च है उस शिर्क से जो वे कर रहे है
\end{hindi}}
\flushright{\begin{Arabic}
\quranayah[16][4]
\end{Arabic}}
\flushleft{\begin{hindi}
उसने मनुष्यों को एक बूँद से पैदा किया। फिर क्या देखते है कि वह खुला झगड़नेवाला बन गया!
\end{hindi}}
\flushright{\begin{Arabic}
\quranayah[16][5]
\end{Arabic}}
\flushleft{\begin{hindi}
रहे पशु, उन्हें भी उसी ने पैदा किया, जिसमें तुम्हारे लिए ऊष्मा प्राप्त करने का सामान भी है और हैं अन्य कितने ही लाभ। उनमें से कुछ को तुम खाते भी हो
\end{hindi}}
\flushright{\begin{Arabic}
\quranayah[16][6]
\end{Arabic}}
\flushleft{\begin{hindi}
उनमें तुम्हारे लिए सौन्दर्य भी है, जबकि तुम सायंकाल उन्हें लाते और जबकि तुम उन्हें चराने ले जाते हो
\end{hindi}}
\flushright{\begin{Arabic}
\quranayah[16][7]
\end{Arabic}}
\flushleft{\begin{hindi}
वे तुम्हारे बोझ ढोकर ऐसे भूभाग तक ले जाते हैं, जहाँ तुम जी-तोड़ परिश्रम के बिना नहीं पहुँच सकते थे। निस्संदेह तुम्हारा रब बड़ा ही करुणामय, दयावान है
\end{hindi}}
\flushright{\begin{Arabic}
\quranayah[16][8]
\end{Arabic}}
\flushleft{\begin{hindi}
और घोड़े और खच्चर और गधे भी पैदा किए, ताकि तुम उनपर सवार हो और शोभा का कारण भी। और वह उसे भी पैदा करता है, जिसे तुम नहीं जानते
\end{hindi}}
\flushright{\begin{Arabic}
\quranayah[16][9]
\end{Arabic}}
\flushleft{\begin{hindi}
अल्लाह के लिए ज़रूरी है उचित एवं अनुकूल मार्ग दिखाना और कुछ मार्ग टेढ़े भी है। यदि वह चाहता तो तुम सबको अवश्य सीधा मार्ग दिखा देता
\end{hindi}}
\flushright{\begin{Arabic}
\quranayah[16][10]
\end{Arabic}}
\flushleft{\begin{hindi}
वही है जिसने आकाश से तुम्हारे लिए पानी उतारा, जिसे तुम पीते हो और उसी से पेड़ और वनस्पतियाँ भी उगती है, जिनमें तुम जानवरों को चराते हो
\end{hindi}}
\flushright{\begin{Arabic}
\quranayah[16][11]
\end{Arabic}}
\flushleft{\begin{hindi}
और उसी से वह तुम्हारे लिए खेतियाँ उगाता है और ज़ैतून, खजूर, अंगूर और हर प्रकार के फल पैदा करता है। निश्चय ही सोच-विचार करनेवालों के लिए इसमें एक निशानी है
\end{hindi}}
\flushright{\begin{Arabic}
\quranayah[16][12]
\end{Arabic}}
\flushleft{\begin{hindi}
और उसने तुम्हारे लिए रात और दिन को और सूर्य और चन्द्रमा को कार्यरत कर रखा है। और तारे भी उसी की आज्ञा से कार्यरत है - निश्चय ही इसमें उन लोगों के लिए निशानियाँ है जो बुद्धि से काम लेते है-
\end{hindi}}
\flushright{\begin{Arabic}
\quranayah[16][13]
\end{Arabic}}
\flushleft{\begin{hindi}
और धरती में तुम्हारे लिए जो रंग-बिरंग की चीज़े बिखेर रखी है, उसमें भी उन लोगों के लिए बड़ी निशानी है जो शिक्षा लेनेवाले है
\end{hindi}}
\flushright{\begin{Arabic}
\quranayah[16][14]
\end{Arabic}}
\flushleft{\begin{hindi}
वही तो है जिसने समुद्र को वश में किया है, ताकि तुम उससे ताज़ा मांस लेकर खाओ और उससे आभूषण निकालो, जिसे तुम पहनते हो। तुम देखते ही हो कि नौकाएँ उसको चीरती हुई चलती हैं (ताकि तुम सफ़र कर सको) और ताकि तुम उसका अनुग्रह तलाश करो और ताकि तुम कृतज्ञता दिखलाओ
\end{hindi}}
\flushright{\begin{Arabic}
\quranayah[16][15]
\end{Arabic}}
\flushleft{\begin{hindi}
और उसने धरती में अटल पहाड़ डाल दिए, कि वह तुम्हें लेकर झुक न पड़े। और नदियाँ बनाई और प्राकृतिक मार्ग बनाए, ताकि तुम मार्ग पा सको
\end{hindi}}
\flushright{\begin{Arabic}
\quranayah[16][16]
\end{Arabic}}
\flushleft{\begin{hindi}
और मार्ग चिन्ह भी बनाए और तारों के द्वारा भी लोग मार्ग पर लेते है
\end{hindi}}
\flushright{\begin{Arabic}
\quranayah[16][17]
\end{Arabic}}
\flushleft{\begin{hindi}
फिर क्या जो पैदा करता है वह उस जैसा हो सकता है, जो पैदा नहीं करता? फिर क्या तुम्हें होश नहीं होता?
\end{hindi}}
\flushright{\begin{Arabic}
\quranayah[16][18]
\end{Arabic}}
\flushleft{\begin{hindi}
और यदि तुम अल्लाह की नेमतों (कृपादानों) को गिनना चाहो तो उन्हें पूर्ण-रूप से गिन नहीं सकते। निस्संदेह अल्लाह बड़ा क्षमाशील, अत्यन्त दयावान है
\end{hindi}}
\flushright{\begin{Arabic}
\quranayah[16][19]
\end{Arabic}}
\flushleft{\begin{hindi}
और अल्लाह जानता है जो कुछ तुम छिपाते हो और जो कुछ प्रकट करते हो
\end{hindi}}
\flushright{\begin{Arabic}
\quranayah[16][20]
\end{Arabic}}
\flushleft{\begin{hindi}
और जिन्हें वे अल्लाह से हटकर पुकारते है वे किसी चीज़ को भी पैदा नहीं करते, बल्कि वे स्वयं पैदा किए जाते है
\end{hindi}}
\flushright{\begin{Arabic}
\quranayah[16][21]
\end{Arabic}}
\flushleft{\begin{hindi}
मृत है, जिनमें प्राण नहीं। उन्हें मालूम नहीं कि वे कब उठाए जाएँगे
\end{hindi}}
\flushright{\begin{Arabic}
\quranayah[16][22]
\end{Arabic}}
\flushleft{\begin{hindi}
तुम्हारा पूज्य-प्रभु अकेला प्रभु-पूज्य है। किन्तु जो आख़िरत में विश्वास नहीं रखते, उनके दिलों को इनकार है। वे अपने आपको बड़ा समझ रहे है
\end{hindi}}
\flushright{\begin{Arabic}
\quranayah[16][23]
\end{Arabic}}
\flushleft{\begin{hindi}
निश्चय ही अल्लाह भली-भाँति जानता है, जो कुछ वे छिपाते है और जो कुछ प्रकट करते है। उसे ऐसे लोग प्रिय नहीं, जो अपने आपको बड़ा समझते हो
\end{hindi}}
\flushright{\begin{Arabic}
\quranayah[16][24]
\end{Arabic}}
\flushleft{\begin{hindi}
और जब उनसे कहा जाता है कि "तुम्हारे रब ने क्या अवतरित किया है?" कहते है, "वे तो पहले लोगों की कहानियाँ है।"
\end{hindi}}
\flushright{\begin{Arabic}
\quranayah[16][25]
\end{Arabic}}
\flushleft{\begin{hindi}
इसका परिणाम यह होगा कि वे क़ियामत के दिन अपने बोझ भी पूरे उठाएँगे और उनके बोझ में से भी जिन्हें वे अज्ञानता के कारण पथभ्रष्ट कर रहे है। सुन लो, बहुत ही बुरा है वह बोझ जो वे उठा रहे है!
\end{hindi}}
\flushright{\begin{Arabic}
\quranayah[16][26]
\end{Arabic}}
\flushleft{\begin{hindi}
जो उनसे पहले गुज़र है वे भी मक्कारियाँ कर चुके है। फिर अल्लाह उनके भवन पर नीवों की ओर से आया और छत उनपर उनके ऊपर से आ गिरी और ऐसे रुख़ से उनपर यातना आई जिसका उन्हें एहसास तक न था
\end{hindi}}
\flushright{\begin{Arabic}
\quranayah[16][27]
\end{Arabic}}
\flushleft{\begin{hindi}
फिर क़ियामत के दिन अल्लाह उन्हें अपमानित करेगा और कहेगा, "कहाँ है मेरे वे साझीदार, जिनके लिए तुम लड़ते-झगड़ते थे?" जिन्हें ज्ञान प्राप्त था वे कहेंगे, "निश्चय ही आज रुसवाई और ख़राबी है इनकार करनेवालों के लिए।"
\end{hindi}}
\flushright{\begin{Arabic}
\quranayah[16][28]
\end{Arabic}}
\flushleft{\begin{hindi}
जिनकी रूहों को फ़रिश्ते इस दशा में ग्रस्त करते है कि वे अपने आप पर अत्याचार कर रहे होते है, तब आज्ञाकारी एवं वशीभूत होकर आ झुकते है कि "हम तो कोई बुराई नहीं करते थे।" "नहीं, बल्कि अल्लाह भली-भाँति जानता है जो कुछ तुम करते रहे हो
\end{hindi}}
\flushright{\begin{Arabic}
\quranayah[16][29]
\end{Arabic}}
\flushleft{\begin{hindi}
तो अब जहन्नम के द्वारों में, उसमें सदैव रहने के लिए प्रवेश करो। अतः निश्चय ही बहुत ही बुरा ठिकाना है यह अहंकारियों का।"
\end{hindi}}
\flushright{\begin{Arabic}
\quranayah[16][30]
\end{Arabic}}
\flushleft{\begin{hindi}
दूसरी ओर जो डर रखनेवाले है उनसे कहा जाता है, "तुम्हारे रब ने क्या अवतरित किया?" वे कहते है, "जो सबसे उत्तम है।" जिन लोगों ने भलाई की उनकी इस दुनिया में भी अच्छी हालत है और आख़िरत का घर तो अच्छा है ही। और क्या ही अच्छा घर है डर रखनेवालों का!
\end{hindi}}
\flushright{\begin{Arabic}
\quranayah[16][31]
\end{Arabic}}
\flushleft{\begin{hindi}
सदैव रहने के बाग़ जिनमें वे प्रवेश करेंगे, उनके नीचे नहरें बह रहीं होंगी, उनके लिए वहाँ वह सब कुछ संचित होगा, जो वे चाहे। अल्लाह डर रखनेवालों को ऐसा ही प्रतिदान प्रदान करता है
\end{hindi}}
\flushright{\begin{Arabic}
\quranayah[16][32]
\end{Arabic}}
\flushleft{\begin{hindi}
जिनकी रूहों को फ़रिश्ते इस दशा में ग्रस्त करते है कि वे पाक और नेक होते है, वे कहते है, "तुम पर सलाम हो! प्रवेश करो जन्नत में उसके बदले में जो कुछ तुम करते रहे हो।"
\end{hindi}}
\flushright{\begin{Arabic}
\quranayah[16][33]
\end{Arabic}}
\flushleft{\begin{hindi}
क्या अब वे इसी की प्रतीक्षा कर रहे है कि फ़रिश्ते उनके पास आ पहुँचे या तेरे रब का आदेश ही आ जाए? ऐसा ही उन लोगो ने भी किया, जो इनसे पहले थे। अल्लाह ने उनपर अत्याचार नहीं किया, किन्तु वे स्वयं अपने ऊपर अत्याचार करते रहे
\end{hindi}}
\flushright{\begin{Arabic}
\quranayah[16][34]
\end{Arabic}}
\flushleft{\begin{hindi}
अन्ततः उनकी करतूतों की बुराइयाँ उनपर आ पड़ी, और जिसका उपहास वे कहते थे, उसी ने उन्हें आ घेरा
\end{hindi}}
\flushright{\begin{Arabic}
\quranayah[16][35]
\end{Arabic}}
\flushleft{\begin{hindi}
शिर्क करनेवालों का कहना है, "यदि अल्लाह चाहता तो उससे हटकर किसी चीज़ की न हम बन्दगी करते और न हमारे बाप-दादा ही और न हम उसके बिना किसी चीज़ को अवैध ठहराते।" उनसे पहले के लोगों ने भी ऐसा ही किया। तो क्या साफ़-साफ़ सन्देश पहुँचा देने के सिवा रसूलों पर कोई और भी ज़िम्मेदारी है?
\end{hindi}}
\flushright{\begin{Arabic}
\quranayah[16][36]
\end{Arabic}}
\flushleft{\begin{hindi}
हमने हर समुदाय में कोई न कोई रसूल भेजा कि "अल्लाह की बन्दगी करो और ताग़ूत से बचो।" फिर उनमें से किसी को तो अल्लाह ने सीधे मार्ग पर लगाया और उनमें से किसी पर पथभ्रष्ट सिद्ध होकर रही। फिर तनिक धरती में चल-फिरकर तो देखो कि झुठलानेवालों का कैसा परिणाम हुआ
\end{hindi}}
\flushright{\begin{Arabic}
\quranayah[16][37]
\end{Arabic}}
\flushleft{\begin{hindi}
यद्यपि इस बात का कि वे राह पर आ जाएँ तुम्हें लालच ही क्यों न हो, किन्तु अल्लाह जिसे भटका देता है, उसे वह मार्ग नहीं दिखाया करता और ऐसे लोगों का कोई सहायक भी नहीं होता
\end{hindi}}
\flushright{\begin{Arabic}
\quranayah[16][38]
\end{Arabic}}
\flushleft{\begin{hindi}
उन्होंने अल्लाह की कड़ी-कड़ी क़समें खाकर कहा, "जो मर जाता है उसे अल्लाह नहीं उठाएगा।" क्यों नहीं? यह तो एक वादा है, जिसे पूरा करना उसके लिए अनिवार्य है - किन्तु अधिकतर लोग जानते नहीं। -
\end{hindi}}
\flushright{\begin{Arabic}
\quranayah[16][39]
\end{Arabic}}
\flushleft{\begin{hindi}
ताकि वह उनपर उसको स्पष्ट। कर दे, जिसके विषय में वे विभेद करते है और इसलिए भी कि इनकार करनेवाले जान लें कि वे झूठे थे
\end{hindi}}
\flushright{\begin{Arabic}
\quranayah[16][40]
\end{Arabic}}
\flushleft{\begin{hindi}
किसी चीज़ के लिए जब हम उसका इरादा करते है तो हमारा कहना बस यही होता है कि उससे कहते है, "हो जा!" और वह हो जाती है
\end{hindi}}
\flushright{\begin{Arabic}
\quranayah[16][41]
\end{Arabic}}
\flushleft{\begin{hindi}
और जिन लोगों ने इसके पश्चात कि उनपर ज़ुल्म ढाया गया था अल्लाह के लिए घर-बार छोड़ा उन्हें हम दुनिया में भी अच्छा ठिकाना देंगे और आख़िरत का प्रतिदान तो बहुत बड़ा है। क्या ही अच्छा होता कि वे जानते
\end{hindi}}
\flushright{\begin{Arabic}
\quranayah[16][42]
\end{Arabic}}
\flushleft{\begin{hindi}
ये वे लोग है जो जमे रहे और वे अपने रब पर भरोसा रखते है
\end{hindi}}
\flushright{\begin{Arabic}
\quranayah[16][43]
\end{Arabic}}
\flushleft{\begin{hindi}
हमने तुमसे पहले भी पुरुषों ही को रसूल बनाकर भेजा था - जिनकी ओर हम प्रकाशना करते रहे है। यदि तुम नहीं जानते तो अनुस्मृतिवालों से पूछ लो
\end{hindi}}
\flushright{\begin{Arabic}
\quranayah[16][44]
\end{Arabic}}
\flushleft{\begin{hindi}
स्पष्ट प्रमाणों और ज़बूरों (किताबों) के साथ। और अब यह अनुस्मृति तुम्हारी ओर हमने अवतरित की, ताकि तुम लोगों के समक्ष खोल-खोलकर बयान कर दो जो कुछ उनकी ओर उतारा गया है और ताकि वे सोच-विचार करें
\end{hindi}}
\flushright{\begin{Arabic}
\quranayah[16][45]
\end{Arabic}}
\flushleft{\begin{hindi}
फिर क्या वे लोग जो ऐसी बुरी-बुरी चालें चल रहे है, इस बात से निश्चिन्त हो गए है कि अल्लाह उन्हें धरती में धँसा दे या ऐसे मौके से उनपर यातना आ जाए जिसका उन्हें एहसास तक न हो?
\end{hindi}}
\flushright{\begin{Arabic}
\quranayah[16][46]
\end{Arabic}}
\flushleft{\begin{hindi}
या उन्हें चलते-फिरते ही पकड़ ले, वे क़ाबू से बाहर निकल जानेवाले तो है नहीं?
\end{hindi}}
\flushright{\begin{Arabic}
\quranayah[16][47]
\end{Arabic}}
\flushleft{\begin{hindi}
क्या अल्लाह की पैदा की हुई किसी चीज़ को उन्होंने देखा नहीं कि किस प्रकार उसकी परछाइयाँ अल्लाह को सजदा करती और विनम्रता दिखाती हुई दाएँ और बाएँ ढलती है?
\end{hindi}}
\flushright{\begin{Arabic}
\quranayah[16][48]
\end{Arabic}}
\flushleft{\begin{hindi}
या वह उन्हें त्रस्त अवस्था में पकड़ ले? किन्तु तुम्हारा रब तो बड़ा ही करुणामय, दयावान है
\end{hindi}}
\flushright{\begin{Arabic}
\quranayah[16][49]
\end{Arabic}}
\flushleft{\begin{hindi}
और आकाशों और धरती में जितने भी जीवधारी है वे सब अल्लाह ही को सजदा करते है और फ़रिश्ते भी और वे घमंड बिलकुल नहीं करते
\end{hindi}}
\flushright{\begin{Arabic}
\quranayah[16][50]
\end{Arabic}}
\flushleft{\begin{hindi}
अपने ऊपर से अपने रब का डर रखते है और जो उन्हें आदेश होता है, वही करते है
\end{hindi}}
\flushright{\begin{Arabic}
\quranayah[16][51]
\end{Arabic}}
\flushleft{\begin{hindi}
अल्लाह का फ़रमान है, "दो-दो पूज्य-प्रभु न बनाओ, वह तो बस अकेला पूज्य-प्रभु है। अतः मुझी से डरो।"
\end{hindi}}
\flushright{\begin{Arabic}
\quranayah[16][52]
\end{Arabic}}
\flushleft{\begin{hindi}
जो कुछ आकाशों और धरती में है सब उसी का है। उसी का दीन (धर्म) स्थायी और अनिवार्य है। फिर क्या अल्लाह के सिवा तुम किसी और का डर रखोगे?
\end{hindi}}
\flushright{\begin{Arabic}
\quranayah[16][53]
\end{Arabic}}
\flushleft{\begin{hindi}
तुम्हारे पास जो भी नेमत है वह अल्लाह ही की ओर से है। फिर जब तुम्हे कोई तकलीफ़ पहुँचती है, तो तुम उसी से फ़रियाद करते हो
\end{hindi}}
\flushright{\begin{Arabic}
\quranayah[16][54]
\end{Arabic}}
\flushleft{\begin{hindi}
फिर जब वह उस तकलीफ़ को तुमसे टाल देता है, तो क्या देखते है कि तुममें से कुछ लोग अपने रब के साथ साझीदार ठहराने लगते है,
\end{hindi}}
\flushright{\begin{Arabic}
\quranayah[16][55]
\end{Arabic}}
\flushleft{\begin{hindi}
कि परिणामस्वरूप जो कुछ हमने उन्हें दिया है उसके प्रति कृतघ्नता दिखलाएँ। अच्छा, कुछ मज़े ले लो, शीघ्र ही तुम्हें मालूम हो जाएगा
\end{hindi}}
\flushright{\begin{Arabic}
\quranayah[16][56]
\end{Arabic}}
\flushleft{\begin{hindi}
हमने उन्हें जो आजीविका प्रदान की है उसमें वे उसका हिस्सा लगाते है जिन्हें वे जानते भी नहीं। अल्लाह की सौगंध! तुम जो झूठ घड़ते हो उसके विषय में तुमसे अवश्य पूछा जाएगा
\end{hindi}}
\flushright{\begin{Arabic}
\quranayah[16][57]
\end{Arabic}}
\flushleft{\begin{hindi}
और वे अल्लाह के लिए बेटियाँ ठहराते है - महान और उच्च है वह - और अपने लिए वह, जो वे चाहें
\end{hindi}}
\flushright{\begin{Arabic}
\quranayah[16][58]
\end{Arabic}}
\flushleft{\begin{hindi}
और जब उनमें से किसी को बेटी की शुभ सूचना मिलती है तो उसके चहरे पर कलौंस छा जाती है और वह घुटा-घुटा रहता है
\end{hindi}}
\flushright{\begin{Arabic}
\quranayah[16][59]
\end{Arabic}}
\flushleft{\begin{hindi}
जो शुभ सूचना उसे दी गई वह (उसकी दृष्टि में) ऐसी बुराई की बात हुई जो उसके कारण वह लोगों से छिपता फिरता है कि अपमान सहन करके उसे रहने दे या उसे मिट्टी में दबा दे। देखो, कितना बुरा फ़ैसला है जो वे करते है!
\end{hindi}}
\flushright{\begin{Arabic}
\quranayah[16][60]
\end{Arabic}}
\flushleft{\begin{hindi}
जो लोग आख़िरत को नहीं मानते बुरी मिसाल है उनकी। रहा अल्लाह, तो उसकी मिसाल अत्यन्त उच्च है। वह तो प्रभुत्वशाली, तत्वदर्शी है
\end{hindi}}
\flushright{\begin{Arabic}
\quranayah[16][61]
\end{Arabic}}
\flushleft{\begin{hindi}
यदि अल्लाह लोगों को उनके अत्याचार पर पकड़ने ही लग जाता तो धरती पर किसी जीवधारी को न छोड़ता, किन्तु वह उन्हें एक निश्चित समय तक टाले जाता है। फिर जब उनका नियत समय आ जाता है तो वे न तो एक घड़ी पीछे हट सकते है और न आगे बढ़ सकते है
\end{hindi}}
\flushright{\begin{Arabic}
\quranayah[16][62]
\end{Arabic}}
\flushleft{\begin{hindi}
वे अल्लाह के लिए वह कुछ ठहराते है, जिसे ख़ुद अपने लिए नापसन्द करते है और उनकी ज़बाने झूठ कहती है कि उनके लिए अच्छा परिणाम है। निस्संदेह उनके लिए आग है और वे उसी में पड़े छोड़ दिए जाएँगे
\end{hindi}}
\flushright{\begin{Arabic}
\quranayah[16][63]
\end{Arabic}}
\flushleft{\begin{hindi}
अल्लाह की सौगंध! हम तुमसे पहले भी कितने समुदायों की ओर रसूल भेज चुके है, किन्तु शैतान ने उनकी करतूतों को उनके लिए सुहावना बना दिया। तो वही आज भी उनका संरक्षक है। उनके लिए तो एक दुखद यातना है
\end{hindi}}
\flushright{\begin{Arabic}
\quranayah[16][64]
\end{Arabic}}
\flushleft{\begin{hindi}
हमने यह किताब तुमपर इसीलिए अवतरित की है कि जिसमें वे विभेद कर रहे है उसे तुम उनपर स्पष्टा कर दो और यह मार्गदर्शन और दयालुता है उन लोगों के लिए जो ईमान लाएँ
\end{hindi}}
\flushright{\begin{Arabic}
\quranayah[16][65]
\end{Arabic}}
\flushleft{\begin{hindi}
और अल्लाह ही ने आकाश से पानी बरसाया। फिर उसके द्वारा धरती को उसके मृत हो जाने के पश्चात जीवित किया। निश्चय ही इसमें उन लोगों के लिए बड़ी निशानी है जो सुनते है
\end{hindi}}
\flushright{\begin{Arabic}
\quranayah[16][66]
\end{Arabic}}
\flushleft{\begin{hindi}
और तुम्हारे लिए चौपायों में से एक बड़ी शिक्षा-सामग्री है, जो कुछ उनके पेटों में है उसमें से गोबर और रक्त से मध्य से हम तुम्हे विशुद्ध दूध पिलाते है, जो पीनेवालों के लिए अत्यन्त प्रिय है,
\end{hindi}}
\flushright{\begin{Arabic}
\quranayah[16][67]
\end{Arabic}}
\flushleft{\begin{hindi}
और खजूरों और अंगूरों के फलों से भी, जिससे तुम मादक चीज़ भी तैयार कर लेते हो और अच्छी रोज़ी भी। निश्चय ही इसमें बुद्धि से काम लेनेवाले लोगों के लिए एक बड़ी निशानी है
\end{hindi}}
\flushright{\begin{Arabic}
\quranayah[16][68]
\end{Arabic}}
\flushleft{\begin{hindi}
और तुम्हारे रब ने मुधमक्खी के जी में यह बात डाल दी कि "पहाड़ों में और वृक्षों में और लोगों के बनाए हुए छत्रों में घर बना
\end{hindi}}
\flushright{\begin{Arabic}
\quranayah[16][69]
\end{Arabic}}
\flushleft{\begin{hindi}
फिर हर प्रकार के फल-फूलों से ख़ुराक ले और अपने रब के समतम मार्गों पर चलती रह।" उसके पेट से विभिन्न रंग का एक पेय निकलता है, जिसमें लोगों के लिए आरोग्य है। निश्चय ही सोच-विचार करनेवाले लोगों के लिए इसमें एक बड़ी निशानी है
\end{hindi}}
\flushright{\begin{Arabic}
\quranayah[16][70]
\end{Arabic}}
\flushleft{\begin{hindi}
अल्लाह ने तुम्हें पैदा किया। फिर वह तुम्हारी आत्माओं को ग्रस्त कर लेता है और तुममें से कोई (बुढापे की) निकृष्ट तम अवस्था की ओर फिर जाता है, कि (परिणामस्वरूप) जानने के पश्चात फिर वह कुछ न जाने। निस्संदेह अल्लाह सर्वज्ञ, बड़ा सामर्थ्यवान है
\end{hindi}}
\flushright{\begin{Arabic}
\quranayah[16][71]
\end{Arabic}}
\flushleft{\begin{hindi}
और अल्लाह ने तुममें से किसी को किसी पर रोज़ी में बड़ाई दी है। किन्तु जिनको बड़ाई दी गई है वे ऐसे नहीं है कि अपनी रोज़ी उनकी ओर फेर दिया करते हों, जो उनके क़ब्ज़े में है कि वे सब इसमें बराबर हो जाएँ। फिर क्या अल्लाह के अनुग्रह का उन्हें इनकार है?
\end{hindi}}
\flushright{\begin{Arabic}
\quranayah[16][72]
\end{Arabic}}
\flushleft{\begin{hindi}
और अल्लाह ही ने तुम्हारे लिए तुम्हारी सहजाति पत्नियों बनाई और तुम्हारी पत्नियों से तुम्हारे लिए पुत्र और पौत्र पैदा किए और तुम्हे अच्छी पाक चीज़ों की रोज़ी प्रदान की; तो क्या वे मिथ्या को मानते है और अल्लाह के अनुग्रह ही का उन्हें इनकार है?
\end{hindi}}
\flushright{\begin{Arabic}
\quranayah[16][73]
\end{Arabic}}
\flushleft{\begin{hindi}
और अल्लाह से हटकर उन्हें पूजते है, जिन्हें आकाशों और धरती से रोज़ी प्रदान करने का कुछ भी अधिकार प्राप्त नहीं है और न उन्हें कोई सामर्थ्य ही प्राप्त है
\end{hindi}}
\flushright{\begin{Arabic}
\quranayah[16][74]
\end{Arabic}}
\flushleft{\begin{hindi}
अतः अल्लाह के लिए मिसालें न घड़ो। जानता अल्लाह है, तुम नहीं जानते
\end{hindi}}
\flushright{\begin{Arabic}
\quranayah[16][75]
\end{Arabic}}
\flushleft{\begin{hindi}
अल्लाह ने एक मिसाल पेश की है: एक ग़ुलाम है, जिसपर दूसरे का अधिकार है, उसे किसी चीज़ पर अधिकार प्राप्त नहीं। इसके विपरीत एक वह व्यक्ति है, जिसे हमने अपनी ओर से अच्छी रोज़ी प्रदान की है, फिर वह उसमें से खुले और छिपे ख़र्च करता है। तो क्या वे परस्पर समान है? प्रशंसा अल्लाह के लिए है! किन्तु उनमें अधिकतर लोग जानते नहीं
\end{hindi}}
\flushright{\begin{Arabic}
\quranayah[16][76]
\end{Arabic}}
\flushleft{\begin{hindi}
अल्लाह ने एक और मिसाल पेश की है: दो व्यक्ति है। उनमें से एक गूँगा है। किसी चीज़ पर उसे अधिकार प्राप्त नहीं। वह अपने स्वामी पर एक बोझ है - उसे वह जहाँ भेजता है, कुछ भला करके नहीं लाता। क्या वह और जो न्याय का आदेश देता है और स्वयं भी सीधे मार्ग पर है वह, समान हो सकते है?
\end{hindi}}
\flushright{\begin{Arabic}
\quranayah[16][77]
\end{Arabic}}
\flushleft{\begin{hindi}
आकाशों और धरती के रहस्यों का सम्बन्ध अल्लाह ही से है। और उस क़ियामत की घड़ी का मामला तो बस ऐसा है जैसे आँखों का झपकना या वह इससे भी अधिक निकट है। निश्चय ही अल्लाह को हर चीज़ की सामर्थ्य प्राप्ती है
\end{hindi}}
\flushright{\begin{Arabic}
\quranayah[16][78]
\end{Arabic}}
\flushleft{\begin{hindi}
अल्लाह ने तुम्हें तुम्हारी माँओ के पेट से इस दशा में निकाला कि तुम कुछ जानते न थे। उसने तुम्हें कान, आँखें और दिल दिए, ताकि तुम कृतज्ञता दिखलाओ
\end{hindi}}
\flushright{\begin{Arabic}
\quranayah[16][79]
\end{Arabic}}
\flushleft{\begin{hindi}
क्या उन्होंने पक्षियों को नभ मंडल में वशीभूत नहीं देखा? उन्हें तो बस अल्लाह ही थामें हुए होता है। निश्चय ही इसमें उन लोगों के लिए कितनी ही निशानियाँ है जो ईमान लाएँ
\end{hindi}}
\flushright{\begin{Arabic}
\quranayah[16][80]
\end{Arabic}}
\flushleft{\begin{hindi}
और अल्लाह ने तुम्हारे घरों को तुम्हारे लिए टिकने की जगह बनाया है और जानवरों की खालों से भी तुम्हारे लिए घर बनाए - जिन्हें तुम अपनी यात्रा के दिन और अपने ठहरने के दिन हल्का-फुलका पाते हो - और एक अवधि के लिए उनके ऊन, उनके लोमचर्म और उनके बालों से कितने ही सामान और बरतने की चीज़े बनाई
\end{hindi}}
\flushright{\begin{Arabic}
\quranayah[16][81]
\end{Arabic}}
\flushleft{\begin{hindi}
और अल्लाह ने तुम्हारे लिए अपनी पैदा की हुई चीज़ों से छाँवों का प्रबन्ध किया और पहाड़ो में तुम्हारे लिए छिपने के स्थान बनाए और तुम्हें लिबास दिए जो गर्मी से बचाते है और कुछ अन्य वस्त्र भी दिए जो तुम्हारी लड़ाई में तुम्हारे लिए बचाव का काम करते है। इस प्रकार वह तुमपर अपनी नेमत पूरी करता है, ताकि तुम आज्ञाकारी बनो
\end{hindi}}
\flushright{\begin{Arabic}
\quranayah[16][82]
\end{Arabic}}
\flushleft{\begin{hindi}
फिर यदि वे मुँह मोड़ते है तो तुम्हारा दायित्व तो केवल साफ़-साफ़ सन्देश पहुँचा देना है
\end{hindi}}
\flushright{\begin{Arabic}
\quranayah[16][83]
\end{Arabic}}
\flushleft{\begin{hindi}
वे अल्लाह की नेमत को पहचानते है, फिर उसका इनकार करते है और उनमें अधिकतर तो अकृतज्ञ है
\end{hindi}}
\flushright{\begin{Arabic}
\quranayah[16][84]
\end{Arabic}}
\flushleft{\begin{hindi}
याद करो जिस दिन हम हर समुदाय में से एक गवाह खड़ा करेंगे, फिर जिन्होंने इनकार किया होगा उन्हें कोई अनुमति प्राप्त न होगी। और न उन्हें इसका अवसर ही दिया जाएगा वे उसे राज़ी कर लें
\end{hindi}}
\flushright{\begin{Arabic}
\quranayah[16][85]
\end{Arabic}}
\flushleft{\begin{hindi}
और जब वे लोग जिन्होंने अत्याचार किया, यातना देख लेंगे तो न वह उनके लिए हलकी की जाएगी और न उन्हें मुहलत ही मिलेगी
\end{hindi}}
\flushright{\begin{Arabic}
\quranayah[16][86]
\end{Arabic}}
\flushleft{\begin{hindi}
और जब वे लोग जिन्होंने शिर्क किया अपने ठहराए हुए साझीदारों को देखेंगे तो कहेंगे, "हमारे रब! यही हमारे वे साझीदार है जिन्हें हम तुझसे हटकर पुकारते थे।" इसपर वे उनकी ओर बात फेंक मारेंगे कि "तुम बिलकुल झूठे हो।"
\end{hindi}}
\flushright{\begin{Arabic}
\quranayah[16][87]
\end{Arabic}}
\flushleft{\begin{hindi}
उस दिन वे अल्लाह के आगे आज्ञाकारी एवं वशीभूत होकर आ पड़ेगे। और जो कुछ वे घड़ा करते थे वह सब उनसे खोकर रह जाएगा
\end{hindi}}
\flushright{\begin{Arabic}
\quranayah[16][88]
\end{Arabic}}
\flushleft{\begin{hindi}
जिन लोगों ने इनकार किया और अल्लाह के मार्ग से रोका उनके लिए हम यातना पर यातना बढ़ाते रहेंगे, उस बिगाड़ के बदले में जो वे पैदा करते रहे
\end{hindi}}
\flushright{\begin{Arabic}
\quranayah[16][89]
\end{Arabic}}
\flushleft{\begin{hindi}
और उस समय को याद करो जब हम हर समुदाय में स्वयं उसके अपने लोगों में से एक गवाह उनपर नियुक्त करके भेज रहे थे और (इसी रीति के अनुसार) तुम्हें इन लोगों पर गवाह नियुक्त करके लाए। हमने तुमपर किताब अवतरित की हर चीज़ को खोलकर बयान करने के लिए और मुस्लिम (आज्ञाकारियों) के लिए मार्गदर्शन, दयालुता और शुभ सूचना के रूप में
\end{hindi}}
\flushright{\begin{Arabic}
\quranayah[16][90]
\end{Arabic}}
\flushleft{\begin{hindi}
निश्चय ही अल्लाह न्याय का और भलाई का और नातेदारों को (उनके हक़) देने का आदेश देता है और अश्लीलता, बुराई और सरकशी से रोकता है। वह तुम्हें नसीहत करता है, ताकि तुम ध्यान दो
\end{hindi}}
\flushright{\begin{Arabic}
\quranayah[16][91]
\end{Arabic}}
\flushleft{\begin{hindi}
अल्लाह के साथ की हुई प्रतिज्ञा को पूरा करो, जबकि तुमने प्रतिज्ञा की हो। और अपनी क़समों को उन्हें सुदृढ़ करने के पश्चात मत तोड़ो, जबकि तुम अपने ऊपर अल्लाह को अपना ज़ामिन बना चुके हो। निश्चय ही अल्लाह जानता है जो कुछ तुम करते हो
\end{hindi}}
\flushright{\begin{Arabic}
\quranayah[16][92]
\end{Arabic}}
\flushleft{\begin{hindi}
तुम उस स्त्री की भाँति न हो जाओ जिसने अपना सूत मेहनत से कातने के पश्चात टुकड़-टुकड़े करके रख दिया। तुम अपनी क़समों को परस्पर हस्तक्षेप करने का बहाना बनाने लगो इस ध्येय से कहीं ऐसा न हो कि एक गिरोह दूसरे गिरोह से बढ़ जाए। बात केवल यह है कि अल्लाह इस प्रतिज्ञा के द्वारा तुम्हारी परीक्षा लेता है और जिस बात में तुम विभेद करते हो उसकी वास्तविकता तो वह क़ियामत के दिन अवश्य ही तुम पर खोल देगा
\end{hindi}}
\flushright{\begin{Arabic}
\quranayah[16][93]
\end{Arabic}}
\flushleft{\begin{hindi}
यदि अल्लाह चाहता तो तुम सबको एक ही समुदाय बना देता, परन्तु वह जिसे चाहता है गुमराही में छोड़ देता है और जिसे चाहता है सीधा मार्ग दिखाता है। तुम जो कुछ भी करते हो उसके विषय में तो तुमसे अवश्य पूछा जाएगा
\end{hindi}}
\flushright{\begin{Arabic}
\quranayah[16][94]
\end{Arabic}}
\flushleft{\begin{hindi}
तुम अपनी क़समों को परस्पर हस्तक्षेप करने का बहाना न बना लेना। कहीं ऐसा न हो कि कोई क़दम जमने के पश्चात उखड़ जाए और अल्लाह के मार्ग से तुम्हारे रोकने के बदले में तुम्हें तकलीफ़ का मज़ा चखना पड़े और तुम एक बड़ी यातना के भागी ठहरो
\end{hindi}}
\flushright{\begin{Arabic}
\quranayah[16][95]
\end{Arabic}}
\flushleft{\begin{hindi}
और तुच्छ मूल्य के लिए अल्लाह की प्रतिज्ञा का सौदा न करो। अल्लाह के पास जो कुछ है वह तुम्हारे लिए अधिक अच्छा है, यदि तुम जानो;
\end{hindi}}
\flushright{\begin{Arabic}
\quranayah[16][96]
\end{Arabic}}
\flushleft{\begin{hindi}
तुम्हारे पास जो कुछ है वह तो समाप्त हो जाएगा, किन्तु अल्लाह के पास जो कुछ है वही बाक़ी रहनेवाला है। जिन लोगों ने धैर्य से काम लिया उन्हें तो, जो उत्तम कर्म वे करते रहे उसके बदले में, हम अवश्य उनका प्रतिदान प्रदान करेंगे
\end{hindi}}
\flushright{\begin{Arabic}
\quranayah[16][97]
\end{Arabic}}
\flushleft{\begin{hindi}
जिस किसी ने भी अच्छा कर्म किया, पुरुष हो या स्त्री, शर्त यह है कि वह ईमान पर हो, तो हम उसे अवश्य पवित्र जीवन-यापन कराएँगे। ऐसे लोग जो अच्छा कर्म करते रहे उसके बदले में हम उन्हें अवश्य उनका प्रतिदान प्रदान करेंगे
\end{hindi}}
\flushright{\begin{Arabic}
\quranayah[16][98]
\end{Arabic}}
\flushleft{\begin{hindi}
अतः जब तुम क़ुरआन पढ़ने लगो तो फिटकारे हुए शैतान से बचने के लिए अल्लाह की पनाह माँग लिया करो
\end{hindi}}
\flushright{\begin{Arabic}
\quranayah[16][99]
\end{Arabic}}
\flushleft{\begin{hindi}
उसका तो उन लोगों पर कोई ज़ोर नहीं चलता जो ईमान लाए और अपने रब पर भरोसा रखते है
\end{hindi}}
\flushright{\begin{Arabic}
\quranayah[16][100]
\end{Arabic}}
\flushleft{\begin{hindi}
उसका ज़ोर तो बस उन्हीं लोगों पर चलता है जो उसे अपना मित्र बनाते है और उस (अल्लाह) के साथ साझी ठहराते है
\end{hindi}}
\flushright{\begin{Arabic}
\quranayah[16][101]
\end{Arabic}}
\flushleft{\begin{hindi}
जब हम किसी आयत की जगह दूसरी आयत बदलकर लाते है - और अल्लाह भली-भाँति जानता है जो कुछ वह अवतरित करता है - तो वे कहते है, "तुम स्वयं ही घड़ लेते हो!" नहीं, बल्कि उनमें से अधिकतर लोग नहीं जानते
\end{hindi}}
\flushright{\begin{Arabic}
\quranayah[16][102]
\end{Arabic}}
\flushleft{\begin{hindi}
कह दो, "इसे ता पवित्र आत्मा ने तुम्हारे रब की ओर क्रमशः सत्य के साथ उतारा है, ताकि ईमान लानेवालों को जमाव प्रदान करे और आज्ञाकारियों के लिए मार्गदर्शन और शुभ सूचना हो
\end{hindi}}
\flushright{\begin{Arabic}
\quranayah[16][103]
\end{Arabic}}
\flushleft{\begin{hindi}
हमें मालूम है कि वे कहते है, "उसको तो बस एक आदमी सिखाता पढ़ाता है।" हालाँकि जिसकी ओर वे संकेत करते है उसकी भाषा विदेशी है और यह स्पष्ट अरबी भाषा है
\end{hindi}}
\flushright{\begin{Arabic}
\quranayah[16][104]
\end{Arabic}}
\flushleft{\begin{hindi}
सच्ची बात यह है कि जो लोग अल्लाह की आयतों को नहीं मानते, अल्लाह उनका मार्गदर्शन नहीं करता। उनके लिए तो एक दुखद यातना है
\end{hindi}}
\flushright{\begin{Arabic}
\quranayah[16][105]
\end{Arabic}}
\flushleft{\begin{hindi}
झूठ तो बस वही लोग घड़ते है जो अल्लाह की आयतों को मानते नहीं और वही है जो झूठे है
\end{hindi}}
\flushright{\begin{Arabic}
\quranayah[16][106]
\end{Arabic}}
\flushleft{\begin{hindi}
जिस किसी ने अपने ईमान के पश्चात अल्लाह के साथ कुफ़्र किया -सिवाय उसके जो इसके लिए विवश कर दिया गया हो और दिल उसका ईमान पर सन्तुष्ट हो - बल्कि वह जिसने सीना कुफ़्र के लिए खोल दिया हो, तो ऐसे लोगो पर अल्लाह का प्रकोप है और उनके लिए बड़ी यातना है
\end{hindi}}
\flushright{\begin{Arabic}
\quranayah[16][107]
\end{Arabic}}
\flushleft{\begin{hindi}
यह इसलिए कि उन्होंने आख़िरत की अपेक्षा सांसारिक जीवन को पसन्द किया और यह कि अल्लाह कुफ़्र करनेवालो लोगों का मार्गदर्शन नहीं करता
\end{hindi}}
\flushright{\begin{Arabic}
\quranayah[16][108]
\end{Arabic}}
\flushleft{\begin{hindi}
वही लोग है जिनके दिलों और जिनके कानों और जिनकी आँखों पर अल्लाह ने मुहर लगा दी है; और वही है जो ग़फ़लत में पड़े हुए है
\end{hindi}}
\flushright{\begin{Arabic}
\quranayah[16][109]
\end{Arabic}}
\flushleft{\begin{hindi}
निश्चय ही आख़िरत में वही घाटे में रहेंगे
\end{hindi}}
\flushright{\begin{Arabic}
\quranayah[16][110]
\end{Arabic}}
\flushleft{\begin{hindi}
फिर तुम्हारा रब उन लोगों के लिए जिन्होंने इसके उपरान्त कि वे आज़माइश में पड़ चुके थे घर-बार छोड़ा, फिर जिहाद (संघर्ष) किया और जमे रहे तो इन बातों के पश्चात तो निश्चय ही तुम्हारा रब बड़ा क्षमाशील, अत्यन्त दयावान है
\end{hindi}}
\flushright{\begin{Arabic}
\quranayah[16][111]
\end{Arabic}}
\flushleft{\begin{hindi}
जिस दिन प्रत्येक व्यक्ति अपनी और प्रत्येक व्यक्ति को जो कुछ उसने किया होगा, उसका पूरा-पूरा बदला चुका दिया जाएगा, और उनपर कुछ भी अत्याचार न होगा
\end{hindi}}
\flushright{\begin{Arabic}
\quranayah[16][112]
\end{Arabic}}
\flushleft{\begin{hindi}
अल्लाह ने एक मिसाल बयान की है: एक बस्ती थी जो निश्चिन्त और सन्तुष्ट थी। हर जगह से उसकी रोज़ी प्रचुरता के साथ चली आ रही थी कि वह अल्लाह की नेमतों के प्रति अकृतज्ञता दिखाने लगी। तब अल्लाह ने उसके निवासियों को उनकी करतूतों के बदले में भूख का मज़ा चख़ाया और भय का वस्त्र पहनाया
\end{hindi}}
\flushright{\begin{Arabic}
\quranayah[16][113]
\end{Arabic}}
\flushleft{\begin{hindi}
उनके पास उन्हीं में से एक रसूल आया। किन्तु उन्होंने उसे झुठला दिया। अन्ततः यातना ने उन्हें इस दशा में आ लिया कि वे अत्याचारी थे
\end{hindi}}
\flushright{\begin{Arabic}
\quranayah[16][114]
\end{Arabic}}
\flushleft{\begin{hindi}
अतः जो कुछ अल्लाह ने तुम्हें हलाल-पाक रोज़ी दी है उसे खाओ और अल्लाह की नेमत के प्रति कृतज्ञता दिखाओ, यदि तुम उसी को स्वामी मानते हो
\end{hindi}}
\flushright{\begin{Arabic}
\quranayah[16][115]
\end{Arabic}}
\flushleft{\begin{hindi}
उसने तो तुमपर केवल मुर्दार, रक्त, सुअर का मांस और जिसपर अल्लाह के सिवा किसी और का नाम लिया गया हो, हराम ठहराया है। फिर यदि कोई इस प्रकार विवश हो जाए कि न तो उसकी ललक हो और न वह हद से आगे बढ़नेवाला हो तो निश्चय ही अल्लाह बड़ा क्षमाशील, दयावान है
\end{hindi}}
\flushright{\begin{Arabic}
\quranayah[16][116]
\end{Arabic}}
\flushleft{\begin{hindi}
और अपनी ज़बानों के बयान किए हुए झूठ के आधार पर यह न कहा करो, "यह हलाल है और यह हराम है," ताकि इस तरह अल्लाह पर झूठ आरोपित करो। जो लोग अल्लाह से सम्बद्ध करके झूठ घड़ते है, वे कदापि सफल होनेवाले नहीं
\end{hindi}}
\flushright{\begin{Arabic}
\quranayah[16][117]
\end{Arabic}}
\flushleft{\begin{hindi}
यह उपभोग थोड़ा है, उनके लिए वास्तव में तो दुखद यातना है
\end{hindi}}
\flushright{\begin{Arabic}
\quranayah[16][118]
\end{Arabic}}
\flushleft{\begin{hindi}
जो यहूदी है उनपर हम पहले वे चीज़े हराम कर चुके है जिनका उल्लेख हमने तुमसे किया। उनपर तो अत्याचार हमने नहीं किया, बल्कि वे स्वयं ही अपने ऊपर अत्याचार करते रहे
\end{hindi}}
\flushright{\begin{Arabic}
\quranayah[16][119]
\end{Arabic}}
\flushleft{\begin{hindi}
फिर तुम्हारा रब उनके लिए जिन्होंने अज्ञानवश बुरा कर्म किया, फिर इसके बाद तौबा करके सुधार कर लिया, तो निश्चय ही तुम्हारा रब इसके पश्चात बड़ा क्षमाशील, अत्यन्त दयावान है
\end{hindi}}
\flushright{\begin{Arabic}
\quranayah[16][120]
\end{Arabic}}
\flushleft{\begin{hindi}
निश्चय ही इबराहीम की स्थिति एक समुदाय की थी। वह अल्लाह का आज्ञाकारी और उसकी ओर एकाग्र था। वह कोई बहुदेववादी न था
\end{hindi}}
\flushright{\begin{Arabic}
\quranayah[16][121]
\end{Arabic}}
\flushleft{\begin{hindi}
वह उसके (अल्लाह के) उदार अनुग्रहों के प्रति कृतज्ञता दिखलानेवाला था। अल्लाह ने उसे चुन लिया और उसे सीधे मार्ग पर चलाया
\end{hindi}}
\flushright{\begin{Arabic}
\quranayah[16][122]
\end{Arabic}}
\flushleft{\begin{hindi}
और हमने उसे दुनिया में भी भलाई दी और आख़िरत में भी वह अच्छे पूर्णकाम लोगों मे से होगा
\end{hindi}}
\flushright{\begin{Arabic}
\quranayah[16][123]
\end{Arabic}}
\flushleft{\begin{hindi}
फिर अब हमने तुम्हारी ओर प्रकाशना की, "इबराहीम के तरीक़े पर चलो, जो बिलकुल एक ओर का हो गया था और बहुदेववादियों में से न था।"
\end{hindi}}
\flushright{\begin{Arabic}
\quranayah[16][124]
\end{Arabic}}
\flushleft{\begin{hindi}
'सब्त' तो केवल उन लोगों पर लागू हुआ था जिन्होंने उसके विषय में विभेद किया था। निश्चय ही तुम्हारा रब उनके बीच क़ियामत के दिन उसका फ़ैसला कर देगा, जिसमें वे विभेद करते रहे है
\end{hindi}}
\flushright{\begin{Arabic}
\quranayah[16][125]
\end{Arabic}}
\flushleft{\begin{hindi}
अपने रब के मार्ग की ओर तत्वदर्शिता और सदुपदेश के साथ बुलाओ और उनसे ऐसे ढंग से वाद विवाद करो जो उत्तम हो। तुम्हारा रब उसे भली-भाँति जानता है जो उसके मार्ग से भटक गया और वह उन्हें भी भली-भाँति जानता है जो मार्ग पर है
\end{hindi}}
\flushright{\begin{Arabic}
\quranayah[16][126]
\end{Arabic}}
\flushleft{\begin{hindi}
और यदि तुम बदला लो तो उतना ही जितना तुम्हें कष्ट पहुँचा हो, किन्तु यदि तुम सब्र करो तो निश्चय ही यह सब्र करनेवालों के लिए ज़्यादा अच्छा है
\end{hindi}}
\flushright{\begin{Arabic}
\quranayah[16][127]
\end{Arabic}}
\flushleft{\begin{hindi}
सब्र से काम लो - और तुम्हारा सब्र अल्लाह ही से सम्बद्ध है - और उन पर दुखी न हो और न उससे दिल तंग हो जो चालें वे चलते है
\end{hindi}}
\flushright{\begin{Arabic}
\quranayah[16][128]
\end{Arabic}}
\flushleft{\begin{hindi}
निश्चय ही, अल्लाह उनके साथ है जो डर रखते है और जो उत्तमकार है
\end{hindi}}
\chapter{Bani Isra'il (The Israelites)}
\begin{Arabic}
\Huge{\centerline{\basmalah}}\end{Arabic}
\flushright{\begin{Arabic}
\quranayah[17][1]
\end{Arabic}}
\flushleft{\begin{hindi}
क्या ही महिमावान है वह जो रातों-रात अपने बन्दे (मुहम्मद) को प्रतिष्ठित मस्जिद (काबा) से दूरवर्ती मस्जिद (अक़्सा) तक ले गया, जिसके चतुर्दिक को हमने बरकत दी, ताकि हम उसे अपनी कुछ निशानियाँ दिखाएँ। निस्संदेह वही सब कुछ सुनता, देखता है
\end{hindi}}
\flushright{\begin{Arabic}
\quranayah[17][2]
\end{Arabic}}
\flushleft{\begin{hindi}
हमने मूसा को किताब दी थी और उसे इसराईल की सन्तान के लिए मार्गदर्शन बनाया था कि "हमारे सिवा किसी को कार्य-साधक न ठहराना।"
\end{hindi}}
\flushright{\begin{Arabic}
\quranayah[17][3]
\end{Arabic}}
\flushleft{\begin{hindi}
ऐ उनकी सन्तान, जिन्हें हमने नूह के साथ (नौका में) सवार किया था! निश्चय ही वह एक कृतज्ञ बन्दा था
\end{hindi}}
\flushright{\begin{Arabic}
\quranayah[17][4]
\end{Arabic}}
\flushleft{\begin{hindi}
और हमने किताब में इसराईल की सन्तान को इस फ़ैसले की ख़बर दे दी थी, "तुम धरती में अवश्य दो बार बड़ा फ़साद मचाओगे और बड़ी सरकशी दिखाओगे।"
\end{hindi}}
\flushright{\begin{Arabic}
\quranayah[17][5]
\end{Arabic}}
\flushleft{\begin{hindi}
फिर जब उन दोनों में से पहले वादे का मौक़ा आ गया तो हमने तुम्हारे मुक़ाबले में अपने ऐसे बन्दों को उठाया जो युद्ध में बड़े बलशाली थे। तो वे बस्तियों में घुसकर हर ओर फैल गए और यह वादा पूरा होना ही था
\end{hindi}}
\flushright{\begin{Arabic}
\quranayah[17][6]
\end{Arabic}}
\flushleft{\begin{hindi}
फिर हमने तुम्हारी बारी उनपर लौटाई कि उनपर प्रभावी हो सको। और धनों और पुत्रों से तुम्हारी सहायता की और तुम्हें बहुसंख्यक लोगों का एक जत्था बनाया
\end{hindi}}
\flushright{\begin{Arabic}
\quranayah[17][7]
\end{Arabic}}
\flushleft{\begin{hindi}
"यदि तुमने भलाई की तो अपने ही लिए भलाई की और यदि तुमने बुराई की तो अपने ही लिए की।" फिर जब दूसरे वादे का मौक़ा आ गया (तो हमने तुम्हारे मुक़ाबले में ऐसे प्रबल को उठाया) कि वे तुम्हारे चेहरे बिगाड़ दें और मस्जिद (बैतुलमक़दिस) में घुसे थे और ताकि जिस चीज़ पर भी उनका ज़ोर चले विनष्टि कर डालें
\end{hindi}}
\flushright{\begin{Arabic}
\quranayah[17][8]
\end{Arabic}}
\flushleft{\begin{hindi}
हो सकता है तुम्हारा रब तुमपर दया करे, किन्तु यदि तुम फिर उसी पूर्व नीति की ओर पलटे तो हम भी पलटेंगे, और हमने जहन्नम को इनकार करनेवालों के लिए कारागार बना रखा है
\end{hindi}}
\flushright{\begin{Arabic}
\quranayah[17][9]
\end{Arabic}}
\flushleft{\begin{hindi}
वास्तव में यह क़ुरआन वह मार्ग दिखाता है जो सबसे सीधा है और उन मोमिमों को, जो अच्छे कर्म करते है, शूभ सूचना देता है कि उनके लिए बड़ा बदला है
\end{hindi}}
\flushright{\begin{Arabic}
\quranayah[17][10]
\end{Arabic}}
\flushleft{\begin{hindi}
और यह कि जो आख़िरत को नहीं मानते उनके लिए हमने दुखद यातना तैयार कर रखी है
\end{hindi}}
\flushright{\begin{Arabic}
\quranayah[17][11]
\end{Arabic}}
\flushleft{\begin{hindi}
मनुष्य उस प्रकार बुराई माँगता है जिस प्रकार उसकी प्रार्थना भलाई के लिए होनी चाहिए। मनुष्य है ही बड़ा उतावला!
\end{hindi}}
\flushright{\begin{Arabic}
\quranayah[17][12]
\end{Arabic}}
\flushleft{\begin{hindi}
हमने रात और दिन को दो निशानियाँ बनाई है। फिर रात की निशानी को हमने मिटी हुई (प्रकाशहीन) बनाया और दिन की निशानी को हमने प्रकाशमान बनाया, ताकि तुम अपने रब का अनुग्रह (रोज़ी) ढूँढो और ताकि तुम वर्षो की गणना और हिसाब मालूम कर सको, और हर चीज़ को हमने अलग-अलग स्पष्ट कर रखा है
\end{hindi}}
\flushright{\begin{Arabic}
\quranayah[17][13]
\end{Arabic}}
\flushleft{\begin{hindi}
हमने प्रत्येक मनुष्य का शकुन-अपशकुन उसकी अपनी गरदन से बाँध दिया है और क़ियामत के दिन हम उसके लिए एक किताब निकालेंगे, जिसको वह खुला हुआ पाएगा
\end{hindi}}
\flushright{\begin{Arabic}
\quranayah[17][14]
\end{Arabic}}
\flushleft{\begin{hindi}
"पढ़ ले अपनी किताब (कर्मपत्र)! आज तू स्वयं ही अपना हिसाब लेने के लिए काफ़ी है।"
\end{hindi}}
\flushright{\begin{Arabic}
\quranayah[17][15]
\end{Arabic}}
\flushleft{\begin{hindi}
जो कोई सीधा मार्ग अपनाए तो उसने अपने ही लिए सीधा मार्ग अपनाया और जो पथभ्रष्टो हुआ, तो वह अपने ही बुरे के लिए भटका। और कोई भी बोझ उठानेवाला किसी दूसरे का बोझ नहीं उठाएगा। और हम लोगों को यातना नहीं देते जब तक कोई रसूल न भेज दें
\end{hindi}}
\flushright{\begin{Arabic}
\quranayah[17][16]
\end{Arabic}}
\flushleft{\begin{hindi}
और जब हम किसी बस्ती को विनष्ट करने का इरादा कर लेते है तो उसके सुखभोगी लोगों को आदेश देते है तो (आदेश मानने के बजाए) वे वहाँ अवज्ञा करने लग जाते है, तब उनपर बात पूरी हो जाती है, फिर हम उन्हें बिलकुल उखाड़ फेकते है
\end{hindi}}
\flushright{\begin{Arabic}
\quranayah[17][17]
\end{Arabic}}
\flushleft{\begin{hindi}
हमने नूह के पश्चात कितनी ही नस्लों को विनष्ट कर दिया। तुम्हारा रब अपने बन्दों के गुनाहों की ख़बर रखने, देखने के लिए काफ़ी है
\end{hindi}}
\flushright{\begin{Arabic}
\quranayah[17][18]
\end{Arabic}}
\flushleft{\begin{hindi}
जो कोई शीघ्र प्राप्त, होनेवाली को चाहता है उसके लिए हम उसी में जो कुछ किसी के लिए चाहते है शीघ्र प्रदान कर देते है। फिर उसके लिए हमने जहन्नम तैयार कर रखा है जिसमें वह अपयशग्रस्त और ठुकराया हुआ प्रवेश करेगा
\end{hindi}}
\flushright{\begin{Arabic}
\quranayah[17][19]
\end{Arabic}}
\flushleft{\begin{hindi}
और जो आख़िरत चाहता हो और उसके लिए ऐसा प्रयास भी करे जैसा कि उसके लिए प्रयास करना चाहिए और वह हो मोमिन, तो ऐसे ही लोग है जिनके प्रयास की क़द्र की जाएगी
\end{hindi}}
\flushright{\begin{Arabic}
\quranayah[17][20]
\end{Arabic}}
\flushleft{\begin{hindi}
इन्हें भी और इनको भी, प्रत्येक को हम तुम्हारे रब की देन में से सहायता पहुँचाए जा रहे है, और तुम्हारे रब की देन बन्द नहीं है
\end{hindi}}
\flushright{\begin{Arabic}
\quranayah[17][21]
\end{Arabic}}
\flushleft{\begin{hindi}
देखो, कैसे हमने उनके कुछ लोगों को कुछ के मुक़ाबले में आगे रखा है! और आख़िरत दर्जों की दृष्टि से सबसे बढ़कर है और श्रेष्ठ़ता की दृष्टि से भी वह सबसे बढ़-चढ़कर है
\end{hindi}}
\flushright{\begin{Arabic}
\quranayah[17][22]
\end{Arabic}}
\flushleft{\begin{hindi}
अल्लाह के साथ कोई दूसरा पूज्य-प्रभु न बनाओ अन्यथा निन्दित और असहाय होकर बैठे रह जाओगे
\end{hindi}}
\flushright{\begin{Arabic}
\quranayah[17][23]
\end{Arabic}}
\flushleft{\begin{hindi}
तुम्हारे रब ने फ़ैसला कर दिया है कि उसके सिवा किसी की बन्दगी न करो और माँ-बाप के साथ अच्छा व्यवहार करो। यदि उनमें से कोई एक या दोनों ही तुम्हारे सामने बुढ़ापे को पहुँच जाएँ तो उन्हें 'उँह' तक न कहो और न उन्हें झिझको, बल्कि उनसे शिष्टतापूर्वक बात करो
\end{hindi}}
\flushright{\begin{Arabic}
\quranayah[17][24]
\end{Arabic}}
\flushleft{\begin{hindi}
और उनके आगे दयालुता से नम्रता की भुजाएँ बिछाए रखो और कहो, "मेरे रब! जिस प्रकार उन्होंने बालकाल में मुझे पाला है, तू भी उनपर दया कर।"
\end{hindi}}
\flushright{\begin{Arabic}
\quranayah[17][25]
\end{Arabic}}
\flushleft{\begin{hindi}
जो कुछ तुम्हारे जी में है उसे तुम्हारा रब भली-भाँति जानता है। यदि तुम सुयोग्य और अच्छे हुए तो निश्चय ही वह भी ऐसे रुजू करनेवालों के लिए बड़ा क्षमाशील है
\end{hindi}}
\flushright{\begin{Arabic}
\quranayah[17][26]
\end{Arabic}}
\flushleft{\begin{hindi}
और नातेदार को उसका हक़ दो मुहताज और मुसाफ़िर को भी - और फुज़ूलख़र्ची न करो
\end{hindi}}
\flushright{\begin{Arabic}
\quranayah[17][27]
\end{Arabic}}
\flushleft{\begin{hindi}
निश्चय ही फ़ु़ज़ूलख़र्ची करनेवाले शैतान के भाई है और शैतान अपने रब का बड़ा ही कृतघ्न है। -
\end{hindi}}
\flushright{\begin{Arabic}
\quranayah[17][28]
\end{Arabic}}
\flushleft{\begin{hindi}
किन्तु यदि तुम्हें अपने रब की दयालुता की खोज में, जिसकी तुम आशा रखते हो, उनसे कतराना भी पड़े, तो इस दशा में तुम उनसें नर्म बात करो
\end{hindi}}
\flushright{\begin{Arabic}
\quranayah[17][29]
\end{Arabic}}
\flushleft{\begin{hindi}
और अपना हाथ न तो अपनी गरदन से बाँधे रखो और न उसे बिलकुल खुला छोड़ दो कि निन्दित और असहाय होकर बैठ जाओ
\end{hindi}}
\flushright{\begin{Arabic}
\quranayah[17][30]
\end{Arabic}}
\flushleft{\begin{hindi}
तुम्हारा रब जिसको चाहता है प्रचुर और फैली हुई रोज़ी प्रदान करता है और इसी प्रकार नपी-तुली भी। निस्संदेह वह अपने बन्दों की ख़बर और उनपर नज़र रखता है
\end{hindi}}
\flushright{\begin{Arabic}
\quranayah[17][31]
\end{Arabic}}
\flushleft{\begin{hindi}
और निर्धनता के भय से अपनी सन्तान की हत्या न करो, हम उन्हें भी रोज़ी देंगे और तुम्हें भी। वास्तव में उनकी हत्या बहुत ही बड़ा अपराध है
\end{hindi}}
\flushright{\begin{Arabic}
\quranayah[17][32]
\end{Arabic}}
\flushleft{\begin{hindi}
और व्यभिचार के निकट न जाओ। वह एक अश्लील कर्म और बुरा मार्ग है
\end{hindi}}
\flushright{\begin{Arabic}
\quranayah[17][33]
\end{Arabic}}
\flushleft{\begin{hindi}
किसी जीव की हत्या न करो, जिसे (मारना) अल्लाह ने हराम ठहराया है। यह और बात है कि हक़ (न्याय) का तक़ाज़ा यही हो। और जिसकी अन्यायपूर्वक हत्या की गई हो, उसके उत्तराधिकारी को हमने अधिकार दिया है (कि वह हत्यारे से बदला ले सकता है), किन्तु वह हत्या के विषय में सीमा का उल्लंघन न करे। निश्चय ही उसकी सहायता की जाएगी
\end{hindi}}
\flushright{\begin{Arabic}
\quranayah[17][34]
\end{Arabic}}
\flushleft{\begin{hindi}
और अनाथ के माल को हाथ में लगाओ सिवाय उत्तम रीति के, यहाँ तक कि वह अपनी युवा अवस्था को पहुँच जाए, और प्रतिज्ञा पूरी करो। प्रतिज्ञा के विषय में अवश्य पूछा जाएगा
\end{hindi}}
\flushright{\begin{Arabic}
\quranayah[17][35]
\end{Arabic}}
\flushleft{\begin{hindi}
और जब नापकर दो तो, नाप पूरी रखो। और ठीक तराज़ू से तौलो, यही उत्तम और परिणाम की दृष्टि से भी अधिक अच्छा है
\end{hindi}}
\flushright{\begin{Arabic}
\quranayah[17][36]
\end{Arabic}}
\flushleft{\begin{hindi}
और जिस चीज़ का तुम्हें ज्ञान न हो उसके पीछे न लगो। निस्संदेह कान और आँख और दिल इनमें से प्रत्येक के विषय में पूछा जाएगा
\end{hindi}}
\flushright{\begin{Arabic}
\quranayah[17][37]
\end{Arabic}}
\flushleft{\begin{hindi}
और धरती में अकड़कर न चलो, न तो तुम धरती को फाड़ सकते हो और न लम्बे होकर पहाड़ो को पहुँच सकते हो
\end{hindi}}
\flushright{\begin{Arabic}
\quranayah[17][38]
\end{Arabic}}
\flushleft{\begin{hindi}
इनमें से प्रत्येक की बुराई तुम्हारे रब की स्पष्ट में अप्रिय ही है
\end{hindi}}
\flushright{\begin{Arabic}
\quranayah[17][39]
\end{Arabic}}
\flushleft{\begin{hindi}
ये तत्वदर्शिता की वे बातें है, जिनकी प्रकाशना तुम्हारे रब ने तुम्हारी ओर की है। और देखो, अल्लाह के साथ कोई दूसरा पूज्य-प्रभु न घड़ना, अन्यथा जहन्नम में डाल दिए जाओगे निन्दित, ठुकराए हुए!
\end{hindi}}
\flushright{\begin{Arabic}
\quranayah[17][40]
\end{Arabic}}
\flushleft{\begin{hindi}
क्या तुम्हारे रब ने तुम्हें तो बेटों के लिए ख़ास किया और स्वयं अपने लिए फ़रिश्तों को बेटियाँ बनाया? बहुत भारी बात है जो तुम कह रहे हो!
\end{hindi}}
\flushright{\begin{Arabic}
\quranayah[17][41]
\end{Arabic}}
\flushleft{\begin{hindi}
हमने इस क़ुरआन में विभिन्न ढंग से बात का स्पष्टीकरण किया कि वे चेतें, किन्तु इसमें उनकी नफ़रत ही बढ़ती है
\end{hindi}}
\flushright{\begin{Arabic}
\quranayah[17][42]
\end{Arabic}}
\flushleft{\begin{hindi}
कह दो, "यदि उसके साथ अन्य भी पूज्य-प्रभु होते, जैसा कि ये कहते हैं, तब तो वे सिंहासनवाले (के पद) तक पहुँचने का कोई मार्ग अवश्य तलाश करते"
\end{hindi}}
\flushright{\begin{Arabic}
\quranayah[17][43]
\end{Arabic}}
\flushleft{\begin{hindi}
महिमावान है वह! और बहुत उच्च है उन बातों से जो वे कहते है!
\end{hindi}}
\flushright{\begin{Arabic}
\quranayah[17][44]
\end{Arabic}}
\flushleft{\begin{hindi}
सातों आकाश और धरती और जो कोई भी उनमें है सब उसकी तसबीह (महिमागान) करते है और ऐसी कोई चीज़ नहीं जो उसका गुणगान न करती हो। किन्तु तुम उनकी तसबीह को समझते नहीं। निश्चय ही वह अत्यन्त सहनशील, क्षमावान है
\end{hindi}}
\flushright{\begin{Arabic}
\quranayah[17][45]
\end{Arabic}}
\flushleft{\begin{hindi}
जब तुम क़ुरआन पढ़ते हो तो हम तुम्हारे और उन लोगों के बीच, जो आख़िरत को नहीं मानते एक अदृश्य पर्दे की आड़ कर देते है
\end{hindi}}
\flushright{\begin{Arabic}
\quranayah[17][46]
\end{Arabic}}
\flushleft{\begin{hindi}
और उनके दिलों पर भी परदे डाल देते है कि वे समझ न सकें। और उनके कानों में बोझ (कि वे सुन न सकें) । और जब तुम क़ुरआन के माध्यम से अपने रब का वर्णन उसे अकेला बताते हुए करते हो तो वे नफ़रत से अपनी पीठ फेरकर चल देते है
\end{hindi}}
\flushright{\begin{Arabic}
\quranayah[17][47]
\end{Arabic}}
\flushleft{\begin{hindi}
जब वे तुम्हारी ओर कान लगाते हैं तो हम भली-भाँति जानते है कि उनके कान लगाने का प्रयोजन क्या है और उसे भी जब वे आपस में कानाफूसियाँ करते है, जब वे ज़ालिम कहते है, "तुम लोग तो बस उस आदमी के पीछे चलते हो जो पक्का जादूगर है।"
\end{hindi}}
\flushright{\begin{Arabic}
\quranayah[17][48]
\end{Arabic}}
\flushleft{\begin{hindi}
देखो, वे कैसी मिसालें तुमपर चस्पाँ करते है! वे तो भटक गए है, अब कोई मार्ग नहीं पा सकते!
\end{hindi}}
\flushright{\begin{Arabic}
\quranayah[17][49]
\end{Arabic}}
\flushleft{\begin{hindi}
वे कहते है, "क्या जब हम हड्डियाँ और चूर्ण-विचूर्ण होकर रह जाएँगे, तो क्या हम फिर नए बनकर उठेंगे?"
\end{hindi}}
\flushright{\begin{Arabic}
\quranayah[17][50]
\end{Arabic}}
\flushleft{\begin{hindi}
कह दो, "तुम पत्थर या लोहो हो जाओ,
\end{hindi}}
\flushright{\begin{Arabic}
\quranayah[17][51]
\end{Arabic}}
\flushleft{\begin{hindi}
या कोई और चीज़ जो तुम्हारे जी में अत्यन्त विकट हो।" तब वे कहेंगे, "कौन हमें पलटाकर लाएगा?" कह दो, "वहीं जिसने तुम्हें पहली बार पैदा किया।" तब वे तुम्हारे आगे अपने सिरों को हिला-हिलाकर कहेंगे, "अच्छा तो वह कह होगा?" कह दो, "कदाचित कि वह निकट ही हो।"
\end{hindi}}
\flushright{\begin{Arabic}
\quranayah[17][52]
\end{Arabic}}
\flushleft{\begin{hindi}
जिस दिन वह तुम्हें पुकारेगा, तो तुम उसकी प्रशंसा करते हुए उसकी आज्ञा को स्वीकार करोगे और समझोगे कि तुम बस थोड़ी ही देर ठहरे रहे हो
\end{hindi}}
\flushright{\begin{Arabic}
\quranayah[17][53]
\end{Arabic}}
\flushleft{\begin{hindi}
मेरे बन्दों से कह दो कि "बात वहीं कहें जो उत्तम हो। शैतान तो उनके बीच उकसाकर फ़साद डालता रहता है। निस्संदेह शैतान मनुष्य का प्रत्यक्ष शत्रु है।"
\end{hindi}}
\flushright{\begin{Arabic}
\quranayah[17][54]
\end{Arabic}}
\flushleft{\begin{hindi}
तुम्हारा रब तुमसे भली-भाँति परिचित है। वह चाहे तो तुमपर दया करे या चाहे तो तुम्हें यातना दे। हमने तुम्हें उनकी ज़िम्मेदारी लेनेवाला कोई क्यक्ति बनाकर नहीं भेजा है (कि उन्हें अनिवार्यतः संमार्ग पर ला ही दो)
\end{hindi}}
\flushright{\begin{Arabic}
\quranayah[17][55]
\end{Arabic}}
\flushleft{\begin{hindi}
तुम्हारा रब उससे भी भली-भाँति परिचित है जो कोई आकाशों और धरती में है, और हमने कुछ नबियों को कुछ की अपेक्षा श्रेष्ठता दी और हमने ही दाऊद को ज़बूर प्रदान की थी
\end{hindi}}
\flushright{\begin{Arabic}
\quranayah[17][56]
\end{Arabic}}
\flushleft{\begin{hindi}
कह दो, "तुम उससे इतर जिनको भी पूज्य-प्रभु समझते हो उन्हें पुकार कर देखो। वे न तुमसे कोई कष्ट दूर करने का अधिकार रखते है और न उसे बदलने का।"
\end{hindi}}
\flushright{\begin{Arabic}
\quranayah[17][57]
\end{Arabic}}
\flushleft{\begin{hindi}
जिनको ये लोग पुकारते है वे तो स्वयं अपने रब का सामीप्य ढूँढते है कि कौन उनमें से सबसे अधिक निकटता प्राप्त कर ले। और वे उसकी दयालुता की आशा रखते है और उसकी यातना से डरते रहते है। तुम्हारे रब की यातना तो है ही डरने की चीज़!
\end{hindi}}
\flushright{\begin{Arabic}
\quranayah[17][58]
\end{Arabic}}
\flushleft{\begin{hindi}
कोई भी (अवज्ञाकारी) बस्ती ऐसी नहीं जिसे हम क़ियामत के दिन से पहले विनष्टअ न कर दें या उसे कठोर यातना न दे। यह बात किताब में लिखी जा चुकी है
\end{hindi}}
\flushright{\begin{Arabic}
\quranayah[17][59]
\end{Arabic}}
\flushleft{\begin{hindi}
हमें निशानियाँ (देकर नबी को) भेजने से इसके सिवा किसी चीज़ ने नहीं रोका कि पहले के लोग उनको झुठला चुके है। और (उदाहरणार्थ) हमने समूद को स्पष्ट प्रमाण के रूप में ऊँटनी दी, किन्तु उन्होंने ग़लत नीति अपनाकर स्वयं ही अपनी जानों पर ज़ुल्म किया। हम निशानियाँ तो डराने ही के लिए भेजते है
\end{hindi}}
\flushright{\begin{Arabic}
\quranayah[17][60]
\end{Arabic}}
\flushleft{\begin{hindi}
जब हमने तुमसे कहा, "तुम्हारे रब ने लोगों को अपने घेरे में ले रखा है और जो अलौकिक दर्शन हमने तुम्हें कराया उसे तो हमने लोगों के लिए केवल एक आज़माइश बना दिया और उस वृक्ष को भी जिसे क़ुरआन में तिरस्कृत ठहराया गया है। हम उन्हें डराते है, किन्तु यह चीज़ उनकी बढ़ी हुई सरकशी ही को बढ़ा रही है।"
\end{hindi}}
\flushright{\begin{Arabic}
\quranayah[17][61]
\end{Arabic}}
\flushleft{\begin{hindi}
याद करो जब हमने फ़रिश्तों से कहा, "आदम को सजदा करो तो इबलीस को छोड़कर सबने सजदा किया।" उसने कहा, "क्या मैं उसे सजदा करूँ, जिसे तूने मिट्टी से बनाया है?"
\end{hindi}}
\flushright{\begin{Arabic}
\quranayah[17][62]
\end{Arabic}}
\flushleft{\begin{hindi}
कहने लगा, "देख तो सही, उसे जिसको तूने मेरे मुक़ाबले में श्रेष्ठ ता प्रदान की है, यदि तूने मुझे क़ियामत के दिन तक मुहलत दे दी, तो मैं अवश्य ही उसकी सन्तान को वश में करके उसका उन्मूलन कर डालूँगा। केवल थोड़े ही लोग बच सकेंगे।"
\end{hindi}}
\flushright{\begin{Arabic}
\quranayah[17][63]
\end{Arabic}}
\flushleft{\begin{hindi}
कहा, "जा, उनमें से जो भी तेरा अनुसरण करेगा, तो तुझ सहित ऐसे सभी लोगों का भरपूर बदला जहन्नम है
\end{hindi}}
\flushright{\begin{Arabic}
\quranayah[17][64]
\end{Arabic}}
\flushleft{\begin{hindi}
उनमें सो जिस किसी पर तेरा बस चले उसके क़दम अपनी आवाज़ से उखाड़ दे। और उनपर अपने सवार और अपने प्यादे (पैदल सेना) चढ़ा ला। और माल और सन्तान में भी उनके साथ साझा लगा। और उनसे वादे कर!" - किन्तु शैतान उनसे जो वादे करता है वह एक धोखे के सिवा और कुछ भी नहीं होता। -
\end{hindi}}
\flushright{\begin{Arabic}
\quranayah[17][65]
\end{Arabic}}
\flushleft{\begin{hindi}
"निश्चय ही जो मेरे (सच्चे) बन्दे है उनपर तेरा कुछ भी ज़ोर नहीं चल सकता।" तुम्हारा रब इसके लिए काफ़ी है कि अपना मामला उसी को सौंप दिया जाए
\end{hindi}}
\flushright{\begin{Arabic}
\quranayah[17][66]
\end{Arabic}}
\flushleft{\begin{hindi}
तुम्हारा रब तो वह है जो तुम्हारे लिए समुद्र में नौका चलाता है, ताकि तुम उसका अनुग्रह (आजीविका) तलाश करो। वह तुम्हारे हाल पर अत्यन्त दयावान है
\end{hindi}}
\flushright{\begin{Arabic}
\quranayah[17][67]
\end{Arabic}}
\flushleft{\begin{hindi}
जब समुद्र में तुम पर कोई आपदा आती है तो उसके सिवा वे सब जिन्हें तुम पुकारते हो, गुम होकर रह जाते है, किन्तु फिर जब वह तुम्हें बचाकर थल पर पहुँचा देता है तो तुम उससे मुँह मोड़ जाते हो। मानव बड़ा ही अकृतज्ञ है
\end{hindi}}
\flushright{\begin{Arabic}
\quranayah[17][68]
\end{Arabic}}
\flushleft{\begin{hindi}
क्या तुम इससे निश्चिन्त हो कि वह कभी थल की ओर ले जाकर तुम्हें धँसा दे या तुमपर पथराव करनेवाली आँधी भेज दे; फिर अपना कोई कार्यसाधक न पाओ?
\end{hindi}}
\flushright{\begin{Arabic}
\quranayah[17][69]
\end{Arabic}}
\flushleft{\begin{hindi}
या तुम इससे निश्चिन्त हो कि वह फिर तुम्हें उसमें दोबारा ले जाए और तुमपर प्रचंड तूफ़ानी हवा भेज दे और तुम्हें तुम्हारे इनकार के बदले में डूबो दे। फिर तुम किसी को ऐसा न पाओ जो तुम्हारे लिए इसपर हमारा पीछा करनेवाला हो?
\end{hindi}}
\flushright{\begin{Arabic}
\quranayah[17][70]
\end{Arabic}}
\flushleft{\begin{hindi}
हमने आदम की सन्तान को श्रेष्ठता प्रदान की और उन्हें थल औऱ जल में सवारी दी और अच्छी-पाक चीज़ों की उन्हें रोज़ी दी और अपने पैदा किए हुए बहुत-से प्राणियों की अपेक्षा उन्हें श्रेष्ठता प्रदान की
\end{hindi}}
\flushright{\begin{Arabic}
\quranayah[17][71]
\end{Arabic}}
\flushleft{\begin{hindi}
(उस दिन से डरो) जिस दिन हम मानव के प्रत्येक गिरोह को उसके अपने नायक के साथ बुलाएँगे। फिर जिसे उसका कर्मपत्र उसके दाहिने हाथ में दिया गया, तो ऐसे लोग अपना कर्मपत्र पढ़ेंगे और उनके साथ तनिक भी अन्याय न होगा
\end{hindi}}
\flushright{\begin{Arabic}
\quranayah[17][72]
\end{Arabic}}
\flushleft{\begin{hindi}
और जो यहाँ अंधा होकर रहा वह आख़िरत में भी अंधा ही रहेगा, बल्कि वह मार्ग से और भी अधिक दूर पड़ा होगा
\end{hindi}}
\flushright{\begin{Arabic}
\quranayah[17][73]
\end{Arabic}}
\flushleft{\begin{hindi}
और वे लगते थे कि तुम्हें फ़िले में डालकर उस चीज़ से हटा देने को है जिसकी प्रकाशना हमने तुम्हारी ओर की है, ताकि तुम उससे भिन्न चीज़ घड़कर हमपर थोपो, और तब वे तुम्हें अपना घनिष्ठ मित्र बना लेते
\end{hindi}}
\flushright{\begin{Arabic}
\quranayah[17][74]
\end{Arabic}}
\flushleft{\begin{hindi}
यदि हम तुम्हें जमाव प्रदान न करते तो तुम उनकी ओर थोड़ा झुकने के निकट जा पहुँचते
\end{hindi}}
\flushright{\begin{Arabic}
\quranayah[17][75]
\end{Arabic}}
\flushleft{\begin{hindi}
उस समय हम तुम्हें जीवन में भी दोहरा मज़ा चखाते और मृत्यु के पश्चात भी दोहरा मज़ा चखाते। फिर तुम हमारे मुक़ाबले में अपना कोई सहायक न पाते
\end{hindi}}
\flushright{\begin{Arabic}
\quranayah[17][76]
\end{Arabic}}
\flushleft{\begin{hindi}
और निश्चय ही उन्होंने चाल चली कि इस भूभाग से तुम्हारे क़दम उखाड़ दें, ताकि तुम्हें यहाँ से निकालकर ही रहे। और ऐसा हुआ तो तुम्हारे पीछे ये भी रह थोड़े ही पाएँगे
\end{hindi}}
\flushright{\begin{Arabic}
\quranayah[17][77]
\end{Arabic}}
\flushleft{\begin{hindi}
यही कार्य-प्रणाली हमारे उन रसूलों के विषय में भी रही है, जिन्हें हमने तुमसे पहले भेजा था और तुम हमारी कार्य-प्रणाली में कोई अन्तर न पाओगे
\end{hindi}}
\flushright{\begin{Arabic}
\quranayah[17][78]
\end{Arabic}}
\flushleft{\begin{hindi}
नमाज़ क़ायम करो सूर्य के ढलने से लेकर रात के छा जाने तक और फ़ज्र (प्रभात) के क़ुरआन (अर्थात फ़ज्र की नमाज़ः के पाबन्द रहो। निश्चय ही फ़ज्र का क़ुरआन पढ़ना हुज़ूरी की चीज़ है
\end{hindi}}
\flushright{\begin{Arabic}
\quranayah[17][79]
\end{Arabic}}
\flushleft{\begin{hindi}
और रात के कुछ हिस्से में उस (क़ुरआन) के द्वारा जागरण किया करो, यह तुम्हारे लिए तद्अधिक (नफ़्ल) है। आशा है कि तुम्हारा रब तुम्हें उठाए ऐसा उठाना जो प्रशंसित हो
\end{hindi}}
\flushright{\begin{Arabic}
\quranayah[17][80]
\end{Arabic}}
\flushleft{\begin{hindi}
और कहो, "मेरे रब! तू मुझे ख़ूबी के साथ दाख़िल कर और ख़ूबी के साथ निकाल, और अपनी ओर से मुझे सहायक शक्ति प्रदान कर।"
\end{hindi}}
\flushright{\begin{Arabic}
\quranayah[17][81]
\end{Arabic}}
\flushleft{\begin{hindi}
कह दो, "सत्य आ गया और असत्य मिट गया; असत्य तो मिट जानेवाला ही होता है।"
\end{hindi}}
\flushright{\begin{Arabic}
\quranayah[17][82]
\end{Arabic}}
\flushleft{\begin{hindi}
हम क़ुरआन में से जो उतारते है वह मोमिनों के लिए शिफ़ा (आरोग्य) और दयालुता है, किन्तु ज़ालिमों के लिए तो वह बस घाटे ही में अभिवृद्धि करता है
\end{hindi}}
\flushright{\begin{Arabic}
\quranayah[17][83]
\end{Arabic}}
\flushleft{\begin{hindi}
मानव पर जब हम सुखद कृपा करते है तो वह मुँह फेरता और अपना पहलू बचाता है। किन्तु जब उसे तकलीफ़ पहुँचती है, तो वह निराश होने लगता है
\end{hindi}}
\flushright{\begin{Arabic}
\quranayah[17][84]
\end{Arabic}}
\flushleft{\begin{hindi}
कह दो, "हर एक अपने ढब पर काम कर रहा है, तो अब तुम्हारा रब ही भली-भाँति जानता है कि कौन अधिक सीधे मार्ग पर है।"
\end{hindi}}
\flushright{\begin{Arabic}
\quranayah[17][85]
\end{Arabic}}
\flushleft{\begin{hindi}
वे तुमसे रूह के विषय में पूछते है। कह दो, "रूह का संबंध तो मेरे रब के आदेश से है, किन्तु ज्ञान तुम्हें मिला थोड़ा ही है।"
\end{hindi}}
\flushright{\begin{Arabic}
\quranayah[17][86]
\end{Arabic}}
\flushleft{\begin{hindi}
यदि हम चाहें तो वह सब छीन लें जो हमने तुम्हारी ओर प्रकाशना की है, फिर इसके लिए हमारे मुक़ाबले में अपना कोई समर्थक न पाओगे
\end{hindi}}
\flushright{\begin{Arabic}
\quranayah[17][87]
\end{Arabic}}
\flushleft{\begin{hindi}
यह तो बस तुम्हारे रब की दयालुता है। वास्तविकता यह है कि उसका तुमपर बड़ा अनुग्रह है
\end{hindi}}
\flushright{\begin{Arabic}
\quranayah[17][88]
\end{Arabic}}
\flushleft{\begin{hindi}
कह दो, "यदि मनुष्य और जिन्न इसके लिए इकट्ठे हो जाएँ कि क़ुरआन जैसी कोई चीज़ लाएँ, तो वे इस जैसी कोई चीज़ न ला सकेंगे, चाहे वे आपस में एक-दूसरे के सहायक ही क्यों न हों।"
\end{hindi}}
\flushright{\begin{Arabic}
\quranayah[17][89]
\end{Arabic}}
\flushleft{\begin{hindi}
हमने इस क़ुरआन में लोगों के लिए प्रत्येक तत्वदर्शिता की बात फेर-फेरकर बयान की, फिर भी अधिकतर लोगों के लिए इनकार के सिवा हर चीज़ अस्वीकार्य ही रही
\end{hindi}}
\flushright{\begin{Arabic}
\quranayah[17][90]
\end{Arabic}}
\flushleft{\begin{hindi}
और उन्होंने कहा, "हम तुम्हारी बात नहीं मानेंगे, जब तक कि तुम हमारे लिए धरती से एक स्रोत प्रवाहित न कर दो,
\end{hindi}}
\flushright{\begin{Arabic}
\quranayah[17][91]
\end{Arabic}}
\flushleft{\begin{hindi}
या फिर तुम्हारे लिए खजूरों और अंगूरों का एक बाग़ हो और तुम उसके बीच बहती नहरें निकाल दो,
\end{hindi}}
\flushright{\begin{Arabic}
\quranayah[17][92]
\end{Arabic}}
\flushleft{\begin{hindi}
या आकाश को टुकड़े-टुकड़े करके हम पर गिरा दो जैसा कि तुम्हारा दावा है, या अल्लाह और फ़रिश्तों ही को हमारे समझ ले आओ,
\end{hindi}}
\flushright{\begin{Arabic}
\quranayah[17][93]
\end{Arabic}}
\flushleft{\begin{hindi}
या तुम्हारे लिए स्वर्ण-निर्मित एक घर हो जाए या तुम आकाश में चढ़ जाओ, और हम तुम्हारे चढ़ने को भी कदापि न मानेंगे, जब तक कि तुम हम पर एक किताब न उतार लाओ, जिसे हम पढ़ सकें।" कह दो, "महिमावान है मेरा रब! क्या मैं एक संदेश लानेवाला मनुष्य के सिवा कुछ और भी हूँ?"
\end{hindi}}
\flushright{\begin{Arabic}
\quranayah[17][94]
\end{Arabic}}
\flushleft{\begin{hindi}
लोगों को जबकि उनके पास मार्गदर्शन आया तो उनको ईमान लाने से केवल यही चीज़ रुकावट बनी कि वे कहने लगे, "क्या अल्लाह ने एक मनुष्य को रसूल बनाकर भेज दिया?"
\end{hindi}}
\flushright{\begin{Arabic}
\quranayah[17][95]
\end{Arabic}}
\flushleft{\begin{hindi}
कह दो, "यदि धरती में फ़रिश्ते आबाद होकर चलते-फिरते होते तो हम उनके लिए अवश्य आकाश से किसी फ़रिश्ते ही को रसूल बनाकर भेजते।"
\end{hindi}}
\flushright{\begin{Arabic}
\quranayah[17][96]
\end{Arabic}}
\flushleft{\begin{hindi}
कह दो, "मेरे और तुम्हारे बीच अल्लाह ही एक गवाह काफ़ी है। निश्चय ही वह अपने बन्दों की पूरी ख़बर रखनेवाला, देखनेवाला है।"
\end{hindi}}
\flushright{\begin{Arabic}
\quranayah[17][97]
\end{Arabic}}
\flushleft{\begin{hindi}
जिसे अल्लाह ही मार्ग दिखाए वही मार्ग पानेवाला है और वह जिसे पथभ्रष्ट होने दे, तो ऐसे लोगों के लिए उससे इतर तुम सहायक न पाओगे। क़ियामत के दिन हम उन्हें औंधे मुँह इस दशा में इकट्ठा करेंगे कि वे अंधे गूँगे और बहरे होंगे। उनका ठिकाना जहन्नम है। जब भी उसकी आग धीमी पड़ने लगेगी तो हम उसे उनके लिए भड़का देंगे
\end{hindi}}
\flushright{\begin{Arabic}
\quranayah[17][98]
\end{Arabic}}
\flushleft{\begin{hindi}
यही उनका बदला है, इसलिए कि उन्होंने हमारी आयतों का इनकार किया और कहा, "क्या जब हम केवल हड्डियाँ और चूर्ण-विचूर्ण होकर रह जाएँगे, तो क्या हमें नए सिरे से पैदा करके उठा खड़ा किया जाएगा?"
\end{hindi}}
\flushright{\begin{Arabic}
\quranayah[17][99]
\end{Arabic}}
\flushleft{\begin{hindi}
क्या उन्हें यह न सूझा कि जिस अल्लाह ने आकाशों और धरती को पैदा किया है उसे उन जैसों को भी पैदा करने की सामर्थ्य प्राप्त है? उसने तो उनके लिए एक समय निर्धारित कर रखा है, जिसमें कोई सन्देह नहीं है। फिर भी ज़ालिमों के लिए इनकार के सिवा हर चीज़ अस्वीकार्य ही रही
\end{hindi}}
\flushright{\begin{Arabic}
\quranayah[17][100]
\end{Arabic}}
\flushleft{\begin{hindi}
कहो, "यदि कहीं मेरे रब की दयालुता के ख़ज़ाने तुम्हारे अधिकार में होते हो ख़र्च हो जाने के भय से तुम रोके ही रखते। वास्तव में इनसान तो दिल का बड़ा ही तंग है
\end{hindi}}
\flushright{\begin{Arabic}
\quranayah[17][101]
\end{Arabic}}
\flushleft{\begin{hindi}
हमने मूसा को नौ खुली निशानियाँ प्रदान की थी। अब इसराईल की सन्तान से पूछ लो कि जब वह उनके पास आया और फ़िरऔन ने उससे कहा, "ऐ मूसा! मैं तो तुम्हें बड़ा जादूगर समझता हूँ।"
\end{hindi}}
\flushright{\begin{Arabic}
\quranayah[17][102]
\end{Arabic}}
\flushleft{\begin{hindi}
उसने कहा, "तू भली-भाँति जानता हैं कि आकाशों और धऱती के रब के सिवा किसी और ने इन (निशानियों) को स्पष्ट प्रमाण बनाकर नहीं उतारा है। और ऐ फ़िरऔन! मैं तो समझता हूँ कि तू विनष्ट होने को है।"
\end{hindi}}
\flushright{\begin{Arabic}
\quranayah[17][103]
\end{Arabic}}
\flushleft{\begin{hindi}
अन्ततः उसने चाहा कि उनको उस भूभाग से उखाड़ फेंके, किन्तु हमने उसे और जो उसके साथ थे सभी को डूबो दिया
\end{hindi}}
\flushright{\begin{Arabic}
\quranayah[17][104]
\end{Arabic}}
\flushleft{\begin{hindi}
और हमने उसके बाद इसराईल की सन्तान से कहा, "तुम इस भूभाग में बसो। फिर जब आख़िरत का वादा आ पूरा होगा, तो हम तुम सबको इकट्ठा ला उपस्थित करेंगे।"
\end{hindi}}
\flushright{\begin{Arabic}
\quranayah[17][105]
\end{Arabic}}
\flushleft{\begin{hindi}
सत्य के साथ हमने उसे अवतरित किया और सत्य के साथ वह अवतरित भी हुआ। और तुम्हें तो हमने केवल शुभ सूचना देनेवाला और सावधान करनेवाला बनाकर भेजा है
\end{hindi}}
\flushright{\begin{Arabic}
\quranayah[17][106]
\end{Arabic}}
\flushleft{\begin{hindi}
और क़ुरआन को हमने थोड़ा-थोड़ा करके इसलिए अवतरित किया, ताकि तुम ठहर-ठहरकर उसे लोगो को सुनाओ, और हमने उसे उत्तम रीति से क्रमशः उतारा है
\end{hindi}}
\flushright{\begin{Arabic}
\quranayah[17][107]
\end{Arabic}}
\flushleft{\begin{hindi}
कह दो, "तुम उसे मानो या न मानो, जिन लोगों को इससे पहले ज्ञान दिया गया है, उन्हें जब वह पढ़कर सुनाया जाता है, तो वे ठोड़ियों के बल सजदे में गिर पड़ते है
\end{hindi}}
\flushright{\begin{Arabic}
\quranayah[17][108]
\end{Arabic}}
\flushleft{\begin{hindi}
और कहते है, "महान और उच्च है हमारा रब! हमारे रब का वादा तो पूरा होकर ही रहता है।"
\end{hindi}}
\flushright{\begin{Arabic}
\quranayah[17][109]
\end{Arabic}}
\flushleft{\begin{hindi}
और वे रोते हुए ठोड़ियों के बल गिर जाते है और वह (क़ुरआन) उनकी विनम्रता को और बढ़ा देता है
\end{hindi}}
\flushright{\begin{Arabic}
\quranayah[17][110]
\end{Arabic}}
\flushleft{\begin{hindi}
कह दो, "तुम अल्लाह को पुकारो या रहमान को पुकारो या जिस नाम से भी पुकारो, उसके लिए सब अच्छे ही नाम है।" और अपनी नमाज़ न बहुत ऊँची आवाज़ से पढ़ो और न उसे बहुत चुपके से पढ़ो, बल्कि इन दोनों के बीच मध्य मार्ग अपनाओ
\end{hindi}}
\flushright{\begin{Arabic}
\quranayah[17][111]
\end{Arabic}}
\flushleft{\begin{hindi}
और कहो, "प्रशंसा अल्लाह के लिए है, जिसने न तो अपना कोई बेटा बनाया और न बादशाही में उसका कोई सहभागी है और न ऐसा ही है कि वह दीन-हीन हो जिसके कारण बचाव के लिए उसका कोई सहायक मित्र हो।" और बड़ाई बयान करो उसकी, पूर्ण बड़ाई
\end{hindi}}
\chapter{Al-Kahf (The Cave)}
\begin{Arabic}
\Huge{\centerline{\basmalah}}\end{Arabic}
\flushright{\begin{Arabic}
\quranayah[18][1]
\end{Arabic}}
\flushleft{\begin{hindi}
प्रशंसा अल्लाह के लिए है जिसने अपने बन्दे पर यह किताब अवतरित की और उसमें (अर्थात उस बन्दे में) कोई टेढ़ नहीं रखी,
\end{hindi}}
\flushright{\begin{Arabic}
\quranayah[18][2]
\end{Arabic}}
\flushleft{\begin{hindi}
ठीक और दूरुस्त, ताकि एक कठोर आपदा से सावधान कर दे जो उसकी और से आ पड़ेगी। और मोमिनों को, जो अच्छे कर्म करते है, शुभ सूचना दे दे कि उनके लिए अच्छा बदला है;
\end{hindi}}
\flushright{\begin{Arabic}
\quranayah[18][3]
\end{Arabic}}
\flushleft{\begin{hindi}
जिसमें वे सदैव रहेंगे
\end{hindi}}
\flushright{\begin{Arabic}
\quranayah[18][4]
\end{Arabic}}
\flushleft{\begin{hindi}
और उनको सावधान कर दे, जो कहते है, "अल्लाह सन्तानवाला है।"
\end{hindi}}
\flushright{\begin{Arabic}
\quranayah[18][5]
\end{Arabic}}
\flushleft{\begin{hindi}
इसका न उन्हें कोई ज्ञान है और न उनके बाप-दादा ही को था। बड़ी बात है जो उनके मुँह से निकलती है। वे केवल झूठ बोलते है
\end{hindi}}
\flushright{\begin{Arabic}
\quranayah[18][6]
\end{Arabic}}
\flushleft{\begin{hindi}
अच्छा, शायद उनके पीछे, यदि उन्होंने यह बात न मानी तो तुम अफ़सोस के मारे अपने प्राण ही खो दोगे!
\end{hindi}}
\flushright{\begin{Arabic}
\quranayah[18][7]
\end{Arabic}}
\flushleft{\begin{hindi}
धरती पर जो कुछ है उसे तो हमने उसकी शोभा बनाई है, ताकि हम उनकी परीक्षा लें कि उनमें कर्म की दृष्टि से कौन उत्तम है
\end{hindi}}
\flushright{\begin{Arabic}
\quranayah[18][8]
\end{Arabic}}
\flushleft{\begin{hindi}
और जो कुछ उसपर है उसे तो हम एक चटियल मैदान बना देनेवाले है
\end{hindi}}
\flushright{\begin{Arabic}
\quranayah[18][9]
\end{Arabic}}
\flushleft{\begin{hindi}
क्या तुम समझते हो कि गुफा और रक़ीमवाले हमारी अद्भु त निशानियों में से थे?
\end{hindi}}
\flushright{\begin{Arabic}
\quranayah[18][10]
\end{Arabic}}
\flushleft{\begin{hindi}
जब उन नवयुवकों ने गुफ़ा में जाकर शरण ली तो कहा, "हमारे रब! हमें अपने यहाँ से दयालुता प्रदान कर और हमारे लिए हमारे अपने मामले को ठीक कर दे।"
\end{hindi}}
\flushright{\begin{Arabic}
\quranayah[18][11]
\end{Arabic}}
\flushleft{\begin{hindi}
फिर हमने उस गुफा में कई वर्षो के लिए उनके कानों पर परदा डाल दिया
\end{hindi}}
\flushright{\begin{Arabic}
\quranayah[18][12]
\end{Arabic}}
\flushleft{\begin{hindi}
फिर हमने उन्हें भेजा, ताकि मालूम करें कि दोनों गिरोहों में से किसने याद रखा है कि कितनी अवधि तक वे रहे
\end{hindi}}
\flushright{\begin{Arabic}
\quranayah[18][13]
\end{Arabic}}
\flushleft{\begin{hindi}
हम तुन्हें ठीक-ठीक उनका वृत्तान्त सुनाते है। वे कुछ नवयुवक थे जो अपने रब पर ईमान लाए थे, और हमने उन्हें मार्गदर्शन में बढ़ोत्तरी प्रदान की
\end{hindi}}
\flushright{\begin{Arabic}
\quranayah[18][14]
\end{Arabic}}
\flushleft{\begin{hindi}
और हमने उनके दिलों को सुदृढ़ कर दिया। जब वे उठे तो उन्होंने कहा, "हमारा रब तो वही है जो आकाशों और धरती का रब है। हम उससे इतर किसी अन्य पूज्य को कदापि न पुकारेंगे। यदि हमने ऐसा किया तब तो हमारी बात हक़ से बहुत हटी हुई होगी
\end{hindi}}
\flushright{\begin{Arabic}
\quranayah[18][15]
\end{Arabic}}
\flushleft{\begin{hindi}
ये हमारी क़ौम के लोग है, जिन्होंने उससे इतर कुछ अन्य पूज्य-प्रभु बना लिए है। आख़िर ये उनके हक़ में कोई स्पष्ट, प्रमाण क्यों नहीं लाते! भला उससे बढ़कर ज़ालिम कौन होगा जो झूठ घड़कर अल्लाह पर थोपे?
\end{hindi}}
\flushright{\begin{Arabic}
\quranayah[18][16]
\end{Arabic}}
\flushleft{\begin{hindi}
और जबकि इनसे तुम अलग हो गए हो और उनसे भी जिनको अल्लाह के सिवा ये पूजते है, तो गुफा में चलकर शरण लो। तुम्हारा रब तुम्हारे लिए अपनी दयालुता का दामन फैला देगा और तुम्हारे लिए तुम्हारे अपने काम से सम्बन्ध में सुगमता का उपकरण उपलब्ध कराएगा।"
\end{hindi}}
\flushright{\begin{Arabic}
\quranayah[18][17]
\end{Arabic}}
\flushleft{\begin{hindi}
और तुम सूर्य को उसके उदित होते समय देखते तो दिखाई देता कि वह उनकी गुफा से दाहिनी ओर को बचकर निकल जाता है और जब अस्त होता है तो उनकी बाई ओर से कतराकर निकल जाता है। और वे है कि उस (गुफा) के एक विस्तृत स्थान में हैं। यह अल्लाह की निशानियों में से है। जिसे अल्लाह मार्ग दिखाए, वही मार्ग पानेवाला है और जिसे वह भटकता छोड़ दे उसका तुम कोई सहायक मार्गदर्शक कदापि न पाओगे
\end{hindi}}
\flushright{\begin{Arabic}
\quranayah[18][18]
\end{Arabic}}
\flushleft{\begin{hindi}
और तुम समझते कि वे जाग रहे है, हालाँकि वे सोए हुए होते। हम उन्हें दाएँ और बाएँ फेरते और उनका कुत्ता ड्योढ़ी पर अपनी दोनों भुजाएँ फैलाए हुए होता। यदि तुम उन्हें कहीं झाँककर देखते तो उनके पास से उलटे पाँव भाग खड़े होते और तुममें उसका भय समा जाता
\end{hindi}}
\flushright{\begin{Arabic}
\quranayah[18][19]
\end{Arabic}}
\flushleft{\begin{hindi}
और इसी तरह हमने उन्हें उठा खड़ा किया कि वे आपस में पूछताछ करें। उनमें एक कहनेवाले ने कहा, "तुम कितना ठहरे रहे?" वे बोले, "हम यही कोई एक दिन या एक दिन से भी कम ठहरें होंगे।" उन्होंने कहा, "जितना तुम यहाँ ठहरे हो उसे तुम्हारा रब ही भली-भाँति जानता है। अब अपने में से किसी को यह चाँदी का सिक्का देकर नगर की ओर भेजो। फिर वह देख ले कि उसमें सबसे अच्छा खाना किस जगह मिलता है। तो उसमें से वह तुम्हारे लिए कुछ खाने को ले आए और चाहिए की वह नरमी और होशियारी से काम ले और किसी को तुम्हारी ख़बर न होने दे
\end{hindi}}
\flushright{\begin{Arabic}
\quranayah[18][20]
\end{Arabic}}
\flushleft{\begin{hindi}
यदि वे कहीं तुम्हारी ख़बर पा जाएँगे तो पथराव करके तुम्हें मार डालेंगे या तुम्हें अपने पंथ में लौटा ले जाएँगे और तब तो तुम कभी भी सफल न पो सकोगे।"
\end{hindi}}
\flushright{\begin{Arabic}
\quranayah[18][21]
\end{Arabic}}
\flushleft{\begin{hindi}
इस तरह हमने लोगों को उनकी सूचना दे दी, ताकि वे जान लें कि अल्लाह का वादा सच्चा है और यह कि क़ियामत की घड़ी में कोई सन्देह नहीं है। वह समय भी उल्लेखनीय है जब वे आपस में उनके मामले में छीन-झपट कर रहे थे। फिर उन्होंने कहा, "उनपर एक भवन बना दे। उनका रब उन्हें भली-भाँति जानता है।" और जो लोग उनके मामले में प्रभावी रहे उन्होंने कहा, "हम तो उनपर अवश्य एक उपासना गृह बनाएँगे।"
\end{hindi}}
\flushright{\begin{Arabic}
\quranayah[18][22]
\end{Arabic}}
\flushleft{\begin{hindi}
अब वे कहेंगे, "वे तीन थे और उनमें चौथा कुत्ता था।" और वे यह भी कहेंगे, "वे पाँच थे और उनमें छठा उनका कुत्ता था।" यह बिना निशाना देखे पत्थर चलाना है। और वे यह भी कहेंगे, "वे सात थे और उनमें आठवाँ उनका कुत्ता था।" कह दो, "मेरा रब उनकी संख्या को भली-भाँति जानता है।" उनको तो थोड़े ही जानते है। तुम ज़ाहिरी बात के सिवा उनके सम्बन्ध में न झगड़ो और न उनमें से किसी से उनके विषय में कुछ पूछो
\end{hindi}}
\flushright{\begin{Arabic}
\quranayah[18][23]
\end{Arabic}}
\flushleft{\begin{hindi}
और न किसी चीज़ के विषय में कभी यह कहो, "मैं कल इसे कर दूँगा।"
\end{hindi}}
\flushright{\begin{Arabic}
\quranayah[18][24]
\end{Arabic}}
\flushleft{\begin{hindi}
बल्कि अल्लाह की इच्छा ही लागू होती है। और जब तुम भूल जाओ तो अपने रब को याद कर लो और कहो, "आशा है कि मेरा रब इससे भी क़रीब सही बात ही ओर मार्गदर्शन कर दे।"
\end{hindi}}
\flushright{\begin{Arabic}
\quranayah[18][25]
\end{Arabic}}
\flushleft{\begin{hindi}
और वे अपनी गुफा में तीन सौ वर्ष रहे और नौ वर्ष उससे अधिक
\end{hindi}}
\flushright{\begin{Arabic}
\quranayah[18][26]
\end{Arabic}}
\flushleft{\begin{hindi}
कह दो, "अल्लाह भली-भाँति जानता है जितना वे ठहरे।" आकाशों और धरती की छिपी बात का सम्बन्ध उसी से है। वह क्या ही देखनेवाला और सुननेवाला है! उससे इतर न तो उनका कोई संरक्षक है और न वह अपने प्रभुत्व और सत्ता में किसी को साझीदार बनाता है
\end{hindi}}
\flushright{\begin{Arabic}
\quranayah[18][27]
\end{Arabic}}
\flushleft{\begin{hindi}
अपने रब की क़िताब, जो कुछ तुम्हारी ओर प्रकाशना (वह्यस) हुई, पढ़ो। कोई नहीं जो उनके बोलो को बदलनेवाला हो और न तुम उससे हटकर क शरण लेने की जगह पाओगे
\end{hindi}}
\flushright{\begin{Arabic}
\quranayah[18][28]
\end{Arabic}}
\flushleft{\begin{hindi}
अपने आपको उन लोगों के साथ थाम रखो, जो प्रातःकाल और सायंकाल अपने रब को उसकी प्रसन्नता चाहते हुए पुकारते है और सांसारिक जीवन की शोभा की चाह में तुम्हारी आँखें उनसे न फिरें। और ऐसे व्यक्ति की बात न मानना जिसके दिल को हमने अपनी याद से ग़ाफ़िल पाया है और वह अपनी इच्छा और वासना के पीछे लगा हुआ है और उसका मामला हद से आगे बढ़ गया है
\end{hindi}}
\flushright{\begin{Arabic}
\quranayah[18][29]
\end{Arabic}}
\flushleft{\begin{hindi}
कह दो, "वह सत्य है तुम्हारे रब की ओर से। तो अब जो कोई चाहे माने और जो चाहे इनकार कर दे।" हमने तो अत्याचारियों के लिए आग तैयार कर रखी है, जिसकी क़नातों ने उन्हें घेर लिया है। यदि वे फ़रियाद करेंगे तो फ़रियाद के प्रत्युत्तर में उन्हें ऐसा पानी मिलेगा जो तेल की तलछट जैसा होगा; वह उनके मुँह भून डालेगा। बहुत ही बुरा है वह पेय और बहुत ही बुरा है वह विश्रामस्थल!
\end{hindi}}
\flushright{\begin{Arabic}
\quranayah[18][30]
\end{Arabic}}
\flushleft{\begin{hindi}
रहे वे लोग जो ईमान लाए और उन्होंने अच्छे कर्म किए, तो निश्चय ही किसी ऐसे व्यक्ति का प्रतिदान जिसने अच्छे कर्म किया हो, हम अकारथ नहीं करते
\end{hindi}}
\flushright{\begin{Arabic}
\quranayah[18][31]
\end{Arabic}}
\flushleft{\begin{hindi}
ऐसे ही लोगों के लिए सदाबहार बाग़ है। उनके नीचे नहरें बह रही होंगी। वहाँ उन्हें सोने के कंगन पहनाए जाएँगे और वे हरे पतले और गाढ़े रेशमी कपड़े पहनेंगे और ऊँचे तख़्तों पर तकिया लगाए होंगे। क्या ही अच्छा बदला है और क्या ही अच्छा विश्रामस्थल!
\end{hindi}}
\flushright{\begin{Arabic}
\quranayah[18][32]
\end{Arabic}}
\flushleft{\begin{hindi}
उनके समक्ष एक उपमा प्रस्तुत करो, दो व्यक्ति है। उनमें से एक को हमने अंगूरों के दो बाग़ दिए और उनके चारों ओर हमने खजूरो के वृक्षो की बाड़ लगाई और उन दोनों के बीच हमने खेती-बाड़ी रखी
\end{hindi}}
\flushright{\begin{Arabic}
\quranayah[18][33]
\end{Arabic}}
\flushleft{\begin{hindi}
दोनों में से प्रत्येक बाग़ अपने फल लाया और इसमें कोई कमी नहीं की। और उन दोनों के बीच हमने एक नहर भी प्रवाहित कर दी
\end{hindi}}
\flushright{\begin{Arabic}
\quranayah[18][34]
\end{Arabic}}
\flushleft{\begin{hindi}
उसे ख़ूब फल और पैदावार प्राप्त हुई। इसपर वह अपने साथी से, जबकि वह उससे बातचीत कर रहा था, कहने लगा, "मैं तुझसे माल और दौलत में बढ़कर हूँ और मुझे जनशक्ति भी अधिक प्राप्त है।"
\end{hindi}}
\flushright{\begin{Arabic}
\quranayah[18][35]
\end{Arabic}}
\flushleft{\begin{hindi}
वह अपने हकड में ज़ालिम बनकर बाग़ में प्रविष्ट हुआ। कहने लगा, "मैं ऐसा नहीं समझता कि वह कभी विनष्ट होगा
\end{hindi}}
\flushright{\begin{Arabic}
\quranayah[18][36]
\end{Arabic}}
\flushleft{\begin{hindi}
और मैं नहीं समझता कि वह (क़ियामत की) घड़ी कभी आएगी। और यदि मैं वास्तव में अपने रब के पास पलटा भी तो निश्चय ही पलटने की जगह इससे भी उत्तम पाऊँगा।"
\end{hindi}}
\flushright{\begin{Arabic}
\quranayah[18][37]
\end{Arabic}}
\flushleft{\begin{hindi}
उसके साथी ने उससे बातचीत करते हुए कहा, "क्या तू उस सत्ता के साथ कुफ़्र करता है जिसने तुझे मिट्टी से, फिर वीर्य से पैदा किया, फिर तुझे एक पूरा आदमी बनाया?
\end{hindi}}
\flushright{\begin{Arabic}
\quranayah[18][38]
\end{Arabic}}
\flushleft{\begin{hindi}
लेकिन मेरा रब तो वही अल्लाह है और मैं किसी को अपने रब के साथ साझीदार नहीं बनाता
\end{hindi}}
\flushright{\begin{Arabic}
\quranayah[18][39]
\end{Arabic}}
\flushleft{\begin{hindi}
और ऐसा क्यों न हुआ कि जब तूने अपने बाग़ में प्रवेश किया तो कहता, 'जो अल्लाह चाहे, बिना अल्लाह के कोई शक्ति नहीं?' यदि तू देखता है कि मैं धन और संतति में तुझसे कम हूँ,
\end{hindi}}
\flushright{\begin{Arabic}
\quranayah[18][40]
\end{Arabic}}
\flushleft{\begin{hindi}
तो आशा है कि मेरा रब मुझे तेरे बाग़ से अच्छा प्रदान करें और तेरे इस बाग़ पर आकाश से कोई क़ुर्क़ी (आपदा) भेज दे। फिर वह साफ़ मैदान होकर रह जाए
\end{hindi}}
\flushright{\begin{Arabic}
\quranayah[18][41]
\end{Arabic}}
\flushleft{\begin{hindi}
या उसका पानी बिलकुल नीचे उतर जाए। फिर तू उसे ढूँढ़कर न ला सके।"
\end{hindi}}
\flushright{\begin{Arabic}
\quranayah[18][42]
\end{Arabic}}
\flushleft{\begin{hindi}
हुआ भी यही कि उसका सारा फल घिराव में आ गया। उसने उसमें जो कुछ लागत लगाई थी, उसपर वह अपनी हथेलियों को नचाता रह गया. और स्थिति यह थी कि बाग़ अपनी टट्टियों पर हा पड़ा था और वह कह रहा था, "क्या ही अच्छा होता कि मैंने अपने रब के साथ किसी को साझीदार न बनाया होता!"
\end{hindi}}
\flushright{\begin{Arabic}
\quranayah[18][43]
\end{Arabic}}
\flushleft{\begin{hindi}
उसका कोई जत्था न हुआ जो उसके और अल्लाह के बीच पड़कर उसकी सहायता करता और न उसे स्वयं बदला लेने की सामर्थ्य प्राप्त थी
\end{hindi}}
\flushright{\begin{Arabic}
\quranayah[18][44]
\end{Arabic}}
\flushleft{\begin{hindi}
ऐसे अवसर पर काम बनाने का सारा अधिकार परम सत्य अल्लाह ही को प्राप्त है। वही बदला देने में सबसे अच्छा है और वही अच्छा परिणाम दिखाने की स्पष्ट से भी सर्वोत्तम है
\end{hindi}}
\flushright{\begin{Arabic}
\quranayah[18][45]
\end{Arabic}}
\flushleft{\begin{hindi}
और उनके समक्ष सांसारिक जीवन की उपमा प्रस्तुत करो, यह ऐसी है जैसे पानी हो, जिसे हमने आकाश से उतारा तो उससे धरती की पौध घनी होकर परस्पर गुँथ गई। फिर वह चूरा-चूरा होकर रह गई, जिसे हवाएँ उड़ाए लिए फिरती है। अल्लाह को तो हर चीज़ की सामर्थ्य प्राप्त है
\end{hindi}}
\flushright{\begin{Arabic}
\quranayah[18][46]
\end{Arabic}}
\flushleft{\begin{hindi}
माल और बेटे तो केवल सांसारिक जीवन की शोभा है, जबकि बाक़ी रहनेवाली नेकियाँ ही तुम्हारे रब के यहाँ परिणाम की दृष्टि से भी उत्तम है और आशा की दृष्टि से भी वही उत्तम है
\end{hindi}}
\flushright{\begin{Arabic}
\quranayah[18][47]
\end{Arabic}}
\flushleft{\begin{hindi}
जिस दिन हम पहाड़ों को चलाएँगे और तुम धरती को बिलकुल नग्न देखोगे और हम उन्हें इकट्ठा करेंगे तो उनमें से किसी एक को भी न छोड़ेंगे
\end{hindi}}
\flushright{\begin{Arabic}
\quranayah[18][48]
\end{Arabic}}
\flushleft{\begin{hindi}
वे तुम्हारे रब के सामने पंक्तिबद्ध उपस्थित किए जाएँगे - "तुम हमारे सामने आ पहुँचे, जैसा हमने तुम्हें पहली बार पैदा किया था। नहीं, बल्कि तुम्हारा तो यह दावा था कि हम तुम्हारे लिए वादा किया हुआ कोई समय लाएँगे ही नहीं।"
\end{hindi}}
\flushright{\begin{Arabic}
\quranayah[18][49]
\end{Arabic}}
\flushleft{\begin{hindi}
किताब (कर्मपत्रिका) रखी जाएगी तो अपराधियों को देखोंगे कि जो कुछ उसमें होगा उससे डर रहे है और कह रहे है, "हाय, हमारा दुर्भाग्य! यह कैसी किताब है कि यह न कोई छोटी बात छोड़ती है न बड़ी, बल्कि सभी को इसने अपने अन्दर समाहित कर रखा है।" जो कुछ उन्होंने किया होगा सब मौजूद पाएँगे। तुम्हारा रब किसी पर ज़ुल्म न करेगा
\end{hindi}}
\flushright{\begin{Arabic}
\quranayah[18][50]
\end{Arabic}}
\flushleft{\begin{hindi}
याद करो जब हमने फ़रिश्तों से कहा, "आदम को सजदा करो।" तो इबलीस के सिवा सबने सजदा किया। वह जिन्नों में से था। तो उसने अपने रब के आदेश का उल्लंघन किया। अब क्या तुम मुझसे इतर उसे और उसकी सन्तान को संरक्षक मित्र बनाते हो? हालाँकि वे तुम्हारे शत्रु है। क्या ही बुरा विकल्प है, जो ज़ालिमों के हाथ आया!
\end{hindi}}
\flushright{\begin{Arabic}
\quranayah[18][51]
\end{Arabic}}
\flushleft{\begin{hindi}
मैंने न तो आकाशों और धरती को उन्हें दिखाकर पैदा किया और न स्वयं उनको बनाने और पैदा करने के समय ही उन्हें बुलाया। मैं ऐसा नहीं हूँ कि गुमराह करनेवालों को अपनी बाहु-भुजा बनाऊँ
\end{hindi}}
\flushright{\begin{Arabic}
\quranayah[18][52]
\end{Arabic}}
\flushleft{\begin{hindi}
याद करो जिस दिन वह कहेगा, "बुलाओ मेरे साझीदारों को, जिनके साझीदार होने का तुम्हें दावा था।" तो वे उनको पुकारेंगे, किन्तु वे उन्हें कोई उत्तर न देंगे और हम उनके बीच सामूहिक विनाश-स्थल निर्धारित कर देंगे
\end{hindi}}
\flushright{\begin{Arabic}
\quranayah[18][53]
\end{Arabic}}
\flushleft{\begin{hindi}
अपराधी लोग आग को देखेंगे तो समझ लेंगे कि वे उसमें पड़नेवाले है और उससे बच निकलने की कोई जगह न पाएँगे
\end{hindi}}
\flushright{\begin{Arabic}
\quranayah[18][54]
\end{Arabic}}
\flushleft{\begin{hindi}
हमने लोगों के लिए इस क़ुरआन में हर प्रकार के उत्तम विषयों को तरह-तरह से बयान किया है, किन्तु मनुष्य सबसे बढ़कर झगड़ालू है
\end{hindi}}
\flushright{\begin{Arabic}
\quranayah[18][55]
\end{Arabic}}
\flushleft{\begin{hindi}
आख़िर लोगों को, जबकि उनके पास मार्गदर्शन आ गया, तो इस बात से कि वे ईमान लाते और अपने रब से क्षमा चाहते, इसके सिवा किसी चीज़ ने नहीं रोका कि उनके लिए वही कुछ सामने आए जो पूर्व जनों के सामने आ चुका है, यहाँ तक कि यातना उनके सामने आ खड़ी हो
\end{hindi}}
\flushright{\begin{Arabic}
\quranayah[18][56]
\end{Arabic}}
\flushleft{\begin{hindi}
रसूलों को हम केवल शुभ सूचना देनेवाले और सचेतकर्त्ता बनाकर भेजते है। किन्तु इनकार करनेवाले लोग है कि असत्य के सहारे झगड़ते है, ताकि सत्य को डिगा दें। उन्होंने मेरी आयतों का और जो चेतावनी उन्हें दी गई उसका मज़ाक बना दिया है
\end{hindi}}
\flushright{\begin{Arabic}
\quranayah[18][57]
\end{Arabic}}
\flushleft{\begin{hindi}
उस व्यक्ति से बढ़कर ज़ालिम कौन होगा जिसे उसके रब की आयतों के द्वारा समझाया गया, तो उसने उनसे मुँह फेर लिया और उसे भूल गया, जो सामान उसके हाथ आगे बढ़ा चुके है? निश्चय ही हमने उनके दिलों पर परदे डाल दिए है कि कहीं वे उसे समझ न लें और उनके कानों में बोझ डाल दिया (कि कहीं वे सुन न ले) । यद्यपि तुम उन्हें सीधे मार्ग की ओर बुलाओ, वे कभी भी मार्ग नहीं पा सकते
\end{hindi}}
\flushright{\begin{Arabic}
\quranayah[18][58]
\end{Arabic}}
\flushleft{\begin{hindi}
तुम्हारा रब अत्यन्त क्षमाशील और दयावान है। यदि वह उन्हें उसपर पकड़ता जो कुछ कि उन्होंने कमाया है तो उनपर शीघ्र ही यातना ला देता। नहीं, बल्कि उनके लिए तो वादे का एक समय निशिचत है। उससे हटकर वे बच निकलने का कोई मार्ग न पाएँगे
\end{hindi}}
\flushright{\begin{Arabic}
\quranayah[18][59]
\end{Arabic}}
\flushleft{\begin{hindi}
और ये बस्तियाँ वे है कि जब उन्होंने अत्याचार किया तो हमने उन्हें विनष्ट कर दिया, और हमने उनके विनाश के लिए एक समय निश्चित कर रखा था
\end{hindi}}
\flushright{\begin{Arabic}
\quranayah[18][60]
\end{Arabic}}
\flushleft{\begin{hindi}
याद करो, जब मूसा ने अपने युवक सेवक से कहा, "जब तक कि मैं दो दरियाओं के संगम तक न पहुँच जाऊँ चलना नहीं छोड़ूँगा, चाहे मैं यूँ ही दीर्धकाल तक सफ़र करता रहूँ।"
\end{hindi}}
\flushright{\begin{Arabic}
\quranayah[18][61]
\end{Arabic}}
\flushleft{\begin{hindi}
फिर जब वे दोनों संगम पर पहुँचे तो वे अपनी मछली से ग़ाफ़िल हो गए और उस (मछली) ने दरिया में सुरंह बनाती अपनी राह ली
\end{hindi}}
\flushright{\begin{Arabic}
\quranayah[18][62]
\end{Arabic}}
\flushleft{\begin{hindi}
फिर जब वे वहाँ से आगे बढ़ गए तो उसने अपने सेवक से कहा, "लाओ, हमारा नाश्ता। अपने इस सफ़र में तो हमें बड़ी थकान पहुँची है।"
\end{hindi}}
\flushright{\begin{Arabic}
\quranayah[18][63]
\end{Arabic}}
\flushleft{\begin{hindi}
उसने कहा, "ज़रा देखिए तो सही, जब हम उस चट्टान के पास ठहरे हुए थे तो मैं मछली को भूल ही गया - और शैतान ही ने उसको याद रखने से मुझे ग़ाफ़िल कर दिया - और उसने आश्चर्य रूप से दरिया में अपनी राह ली।"
\end{hindi}}
\flushright{\begin{Arabic}
\quranayah[18][64]
\end{Arabic}}
\flushleft{\begin{hindi}
(मूसा ने) कहा, "यही तो है जिसे हम तलाश कर रहे थे।" फिर वे दोनों अपने पदचिन्हों को देखते हुए वापस हुए
\end{hindi}}
\flushright{\begin{Arabic}
\quranayah[18][65]
\end{Arabic}}
\flushleft{\begin{hindi}
फिर उन्होंने हमारे बन्दों में से एक बन्दे को पाया, जिसे हमने अपने पास से दयालुता प्रदान की थी और जिसे अपने पास से ज्ञान प्रदान किया था
\end{hindi}}
\flushright{\begin{Arabic}
\quranayah[18][66]
\end{Arabic}}
\flushleft{\begin{hindi}
मूसा ने उससे कहा, "क्या मैं आपके पीछे चलूँ, ताकि आप मुझे उस ज्ञान औऱ विवेक की शिक्षा दें, जो आपको दी गई है?"
\end{hindi}}
\flushright{\begin{Arabic}
\quranayah[18][67]
\end{Arabic}}
\flushleft{\begin{hindi}
उसने कहा, "तुम मेरे साथ धैर्य न रख सकोगे,
\end{hindi}}
\flushright{\begin{Arabic}
\quranayah[18][68]
\end{Arabic}}
\flushleft{\begin{hindi}
और जो चीज़ तुम्हारे ज्ञान-परिधि से बाहर हो, उस पर तुम धैर्य कैसे रख सकते हो?"
\end{hindi}}
\flushright{\begin{Arabic}
\quranayah[18][69]
\end{Arabic}}
\flushleft{\begin{hindi}
(मूसा ने) कहा, "यदि अल्लाह ने चाहा तो आप मुझे धैर्यवान पाएँगे। और मैं किसी मामले में भी आपकी अवज्ञा नहीं करूँगा।"
\end{hindi}}
\flushright{\begin{Arabic}
\quranayah[18][70]
\end{Arabic}}
\flushleft{\begin{hindi}
उसने कहा, "अच्छा, यदि तुम मेरे साथ चलते हो तो मुझसे किसी चीज़ के विषय में न पूछना, यहाँ तक कि मैं स्वयं ही तुमसे उसकी चर्चा करूँ।"
\end{hindi}}
\flushright{\begin{Arabic}
\quranayah[18][71]
\end{Arabic}}
\flushleft{\begin{hindi}
अन्ततः दोनों चले, यहाँ तक कि जब नौका में सवार हुए तो उसने उसमें दरार डाल दी। (मूसा ने) कहा, "आपने इसमें दरार डाल दी, ताकि उसके सवारों को डुबो दें? आपने तो एक अनोखी हरकत कर डाली।"
\end{hindi}}
\flushright{\begin{Arabic}
\quranayah[18][72]
\end{Arabic}}
\flushleft{\begin{hindi}
उसने कहा, "क्या मैंने कहा नहीं था कि तुम मेरे साथ धैर्य न रख सकोंगे?"
\end{hindi}}
\flushright{\begin{Arabic}
\quranayah[18][73]
\end{Arabic}}
\flushleft{\begin{hindi}
कहा, "जो भूल-चूक मुझसे हो गई उसपर मुझे न पकड़िए और मेरे मामलें में मुझे तंगी में न डालिए।"
\end{hindi}}
\flushright{\begin{Arabic}
\quranayah[18][74]
\end{Arabic}}
\flushleft{\begin{hindi}
फिर वे दोनों चले, यहाँ तक कि जब वे एक लड़के से मिले तो उसने उसे मार डाला। कहा, "क्या आपने एक अच्छी-भली जान की हत्या कर दी, बिना इसके कि किसी की हत्या का बदला लेना अभीष्ट हो? यह तो आपने बहुत ही बुरा किया!"
\end{hindi}}
\flushright{\begin{Arabic}
\quranayah[18][75]
\end{Arabic}}
\flushleft{\begin{hindi}
उसने कहा, "क्या मैंने तुमसे कहा नहीं था कि तुम मेरे साथ धैर्य न रख सकोगे?"
\end{hindi}}
\flushright{\begin{Arabic}
\quranayah[18][76]
\end{Arabic}}
\flushleft{\begin{hindi}
कहा, "इसके बाद यदि मैं आपसे कुछ पूछूँ तो आप मुझे साथ न रखें। अब तो मेरी ओर से आप पूरी तरह उज़ को पहुँच चुके है।"
\end{hindi}}
\flushright{\begin{Arabic}
\quranayah[18][77]
\end{Arabic}}
\flushleft{\begin{hindi}
फिर वे दोनों चले, यहाँ तक कि जब वे एक बस्तीवालों के पास पहुँचे और उनसे भोजन माँगा, किन्तु उन्होंने उनके आतिथ्य से इनकार कर दिया। फिर वहाँ उन्हें एक दीवार मिली जो गिरा चाहती थी, तो उस व्यक्ति ने उसको खड़ा कर दिया। (मूसा ने) कहा, "यदि आप चाहते तो इसकी कुछ मज़दूरी ले सकते थे।"
\end{hindi}}
\flushright{\begin{Arabic}
\quranayah[18][78]
\end{Arabic}}
\flushleft{\begin{hindi}
उसने कहा, "यह मेरे और तुम्हारे बीच जुदाई का अवसर है। अब मैं तुमको उसकी वास्तविकता बताए दे रहा हूँ, जिसपर तुम धैर्य से काम न ले सके।"
\end{hindi}}
\flushright{\begin{Arabic}
\quranayah[18][79]
\end{Arabic}}
\flushleft{\begin{hindi}
वह जो नौका थी, कुछ निर्धन लोगों की थी जो दरिया में काम करते थे, तो मैंने चाहा कि उसे ऐबदार कर दूँ, क्योंकि आगे उनके परे एक सम्राट था, जो प्रत्येक नौका को ज़बरदस्ती छीन लेता था
\end{hindi}}
\flushright{\begin{Arabic}
\quranayah[18][80]
\end{Arabic}}
\flushleft{\begin{hindi}
और रहा वह लड़का, तो उसके माँ-बाप ईमान पर थे। हमें आशंका हुई कि वह सरकशी और कुफ़्र से उन्हें तंग करेगा
\end{hindi}}
\flushright{\begin{Arabic}
\quranayah[18][81]
\end{Arabic}}
\flushleft{\begin{hindi}
इसलिए हमने चाहा कि उनका रब उन्हें इसके बदले दूसरी संतान दे, जो आत्मविकास में इससे अच्छा हो और दया-करूणा से अधिक निकट हो
\end{hindi}}
\flushright{\begin{Arabic}
\quranayah[18][82]
\end{Arabic}}
\flushleft{\begin{hindi}
और रही यह दीवार तो यह दो अनाथ बालकों की है जो इस नगर में रहते है। और इसके नीचे उनका ख़जाना मौजूद है। और उनका बाप नेक था, इसलिए तुम्हारे रब ने चाहा कि वे अपनी युवावस्था को पहुँच जाएँ और अपना ख़जाना निकाल लें। यह तुम्हारे रब की दयालुता के कारण हुआ। मैंने तो अपने अधिकार से कुछ नहीं किया। यह है वास्तविकता उसकी जिसपर तुम धैर्य न रख सके।"
\end{hindi}}
\flushright{\begin{Arabic}
\quranayah[18][83]
\end{Arabic}}
\flushleft{\begin{hindi}
वे तुमसे ज़ुलक़रनैन के विषय में पूछते हैं। कह दो, "मैं तुम्हें उसका कुछ वृतान्त सुनाता हूँ।"
\end{hindi}}
\flushright{\begin{Arabic}
\quranayah[18][84]
\end{Arabic}}
\flushleft{\begin{hindi}
हमने उसे धरती में सत्ता प्रदान की थी और उसे हर प्रकार के संसाधन दिए थे
\end{hindi}}
\flushright{\begin{Arabic}
\quranayah[18][85]
\end{Arabic}}
\flushleft{\begin{hindi}
अतएव उसने एक अभियान का आयोजन किया
\end{hindi}}
\flushright{\begin{Arabic}
\quranayah[18][86]
\end{Arabic}}
\flushleft{\begin{hindi}
यहाँ तक कि जब वह सूर्यास्त-स्थल तक पहुँचा तो उसे मटमैले काले पानी के एक स्रोत में डूबते हुए पाया और उसके निकट उसे एक क़ौम मिली। हमने कहा, "ऐ ज़ुलक़रनैन! तुझे अधिकार है कि चाहे तकलीफ़ पहुँचाए और चाहे उनके साथ अच्छा व्यवहार करे।"
\end{hindi}}
\flushright{\begin{Arabic}
\quranayah[18][87]
\end{Arabic}}
\flushleft{\begin{hindi}
उसने कहा, "जो कोई ज़ुल्म करेगा उसे तो हम दंड देंगे। फिर वह अपने रब की ओर पलटेगा और वह उसे कठोर यातना देगा
\end{hindi}}
\flushright{\begin{Arabic}
\quranayah[18][88]
\end{Arabic}}
\flushleft{\begin{hindi}
किन्तु जो कोई ईमान लाया औऱ अच्छा कर्म किया, उसके लिए तो अच्छा बदला है और हम उसे अपना सहज एवं मृदुल आदेश देंगे।"
\end{hindi}}
\flushright{\begin{Arabic}
\quranayah[18][89]
\end{Arabic}}
\flushleft{\begin{hindi}
फिर उसने एक और अभियान का आयोजन किया
\end{hindi}}
\flushright{\begin{Arabic}
\quranayah[18][90]
\end{Arabic}}
\flushleft{\begin{hindi}
यहाँ तक कि जब वह सूर्योदय स्थल पर जा पहुँचा तो उसने उसे ऐसे लोगों पर उदित होते पाया जिनके लिए हमने सूर्य के मुक़ाबले में कोई ओट नहीं रखी थी
\end{hindi}}
\flushright{\begin{Arabic}
\quranayah[18][91]
\end{Arabic}}
\flushleft{\begin{hindi}
ऐसा ही हमने किया था और जो कुछ उसके पास था, उसकी हमें पूरी ख़बर थी
\end{hindi}}
\flushright{\begin{Arabic}
\quranayah[18][92]
\end{Arabic}}
\flushleft{\begin{hindi}
उसने फिर एक अभियान का आयोजन किया,
\end{hindi}}
\flushright{\begin{Arabic}
\quranayah[18][93]
\end{Arabic}}
\flushleft{\begin{hindi}
यहाँ तक कि जब वह दो पर्वतों के बीच पहुँचा तो उसे उनके इस किनारे कुछ पहुँचा तो उसे उनके इस किनारे कुछ लोग मिले, जो ऐसा लगाता नहीं था कि कोई बात समझ पाते हों
\end{hindi}}
\flushright{\begin{Arabic}
\quranayah[18][94]
\end{Arabic}}
\flushleft{\begin{hindi}
उन्होंने कहा, "ऐ ज़ुलक़रनैन! याजूज और माजूज इस भूभाग में उत्पात मचाते हैं। क्या हम तुम्हें कोई कर इस बात काम के लिए दें कि तुम हमारे और उनके बीच एक अवरोध निर्मित कर दो?"
\end{hindi}}
\flushright{\begin{Arabic}
\quranayah[18][95]
\end{Arabic}}
\flushleft{\begin{hindi}
उसने कहा, "मेरे रब ने मुझे जो कुछ अधिकार एवं शक्ति दी है वह उत्तम है। तुम तो बस बल में मेरी सहायता करो। मैं तुम्हारे और उनके बीच एक सुदृढ़ दीवार बनाए देता हूँ
\end{hindi}}
\flushright{\begin{Arabic}
\quranayah[18][96]
\end{Arabic}}
\flushleft{\begin{hindi}
मुझे लोहे के टुकड़े ला दो।" यहाँ तक कि जब दोनों पर्वतों के बीच के रिक्त स्थान को पाटकर बराबर कर दिया तो कहा, "धौंको!" यहाँ तक कि जब उसे आग कर दिया तो कहा, "मुझे पिघला हुआ ताँबा ला दो, ताकि मैं उसपर उँड़ेल दूँ।"
\end{hindi}}
\flushright{\begin{Arabic}
\quranayah[18][97]
\end{Arabic}}
\flushleft{\begin{hindi}
तो न तो वे (याजूज, माजूज) उसपर चढ़कर आ सकते थे और न वे उसमें सेंध ही लगा सकते थे
\end{hindi}}
\flushright{\begin{Arabic}
\quranayah[18][98]
\end{Arabic}}
\flushleft{\begin{hindi}
उसने कहा, "यह मेरे रब की दयालुता है, किन्तु जब मेरे रब के वादे का समय आ जाएगा तो वह उसे ढाकर बराबर कर देगा, और मेरे रब का वादा सच्चा है।"
\end{hindi}}
\flushright{\begin{Arabic}
\quranayah[18][99]
\end{Arabic}}
\flushleft{\begin{hindi}
उस दिन हम उन्हें छोड़ देंगे कि वे एक-दूसरे से मौज़ों की तरह परस्पर गुत्मथ-गुत्था हो जाएँगे। और "सूर" फूँका जाएगा। फिर हम उन सबको एक साथ इकट्ठा करेंगे
\end{hindi}}
\flushright{\begin{Arabic}
\quranayah[18][100]
\end{Arabic}}
\flushleft{\begin{hindi}
और उस दिन जहन्नम को इनकार करनेवालों के सामने कर देंगे
\end{hindi}}
\flushright{\begin{Arabic}
\quranayah[18][101]
\end{Arabic}}
\flushleft{\begin{hindi}
जिनके नेत्र मेरी अनुस्मृति की ओर से परदे में थे और जो कुछ सुन भी नहीं सकते थे
\end{hindi}}
\flushright{\begin{Arabic}
\quranayah[18][102]
\end{Arabic}}
\flushleft{\begin{hindi}
तो क्या इनकार करनेवाले इस ख़याल में हैं कि मुझसे हटकर मेरे बन्दों को अपना हिमायती बना लें? हमने ऐसे इनकार करनेवालों के आतिथ्य-सत्कार के लिए जहन्नम तैयार कर रखा है
\end{hindi}}
\flushright{\begin{Arabic}
\quranayah[18][103]
\end{Arabic}}
\flushleft{\begin{hindi}
कहो, "क्या हम तुम्हें उन लोगों की ख़बर दें, जो अपने कर्मों की स्पष्ट से सबसे बढ़कर घाटा उठानेवाले हैं?
\end{hindi}}
\flushright{\begin{Arabic}
\quranayah[18][104]
\end{Arabic}}
\flushleft{\begin{hindi}
यो वे लोग है जिनका प्रयास सांसारिक जीवन में अकारथ गया और वे यही समझते है कि वे बहुत अच्छा कर्म कर रहे है
\end{hindi}}
\flushright{\begin{Arabic}
\quranayah[18][105]
\end{Arabic}}
\flushleft{\begin{hindi}
यही वे लोग है जिन्होंने अपने रब की आयतों का और उससे मिलन का इनकार किया। अतः उनके कर्म जान को लागू हुए, तो हम क़ियामत के दिन उन्हें कोई वज़न न देंगे
\end{hindi}}
\flushright{\begin{Arabic}
\quranayah[18][106]
\end{Arabic}}
\flushleft{\begin{hindi}
उनका बदला वही जहन्नम है, इसलिए कि उन्होंने कुफ़्र की नीति अपनाई और मेरी आयतों और मेरे रसूलों का उपहास किया
\end{hindi}}
\flushright{\begin{Arabic}
\quranayah[18][107]
\end{Arabic}}
\flushleft{\begin{hindi}
निश्चय ही जो लोग ईमान लाए और उन्होंने अच्छे कर्म किए उनके आतिथ्य के लिए फ़िरदौस के बाग़ होंगे,
\end{hindi}}
\flushright{\begin{Arabic}
\quranayah[18][108]
\end{Arabic}}
\flushleft{\begin{hindi}
जिनमें वे सदैव रहेंगे, वहाँ से हटना न चाहेंगे।"
\end{hindi}}
\flushright{\begin{Arabic}
\quranayah[18][109]
\end{Arabic}}
\flushleft{\begin{hindi}
कहो, "यदि समुद्र मेरे रब के बोल को लिखने के लिए रोशनाई हो जाए तो इससे पहले कि मेरे रब के बोल समाप्त हों, समुद्र ही समाप्त हो जाएगा। यद्यपि हम उसके सदृश्य एक और भी समुद्र उसके साथ ला मिलाएँ।"
\end{hindi}}
\flushright{\begin{Arabic}
\quranayah[18][110]
\end{Arabic}}
\flushleft{\begin{hindi}
कह दो, "मैं तो केवल तुम्हीं जैसा मनुष्य हूँ। मेरी ओर प्रकाशना की जाती है कि तुम्हारा पूज्य-प्रभु बस अकेला पूज्य-प्रभु है। अतः जो कोई अपने रब से मिलन की आशा रखता हो, उसे चाहिए कि अच्छा कर्म करे और अपने रब की बन्दगी में किसी को साझी न बनाए।"
\end{hindi}}
\chapter{Maryam (Mary)}
\begin{Arabic}
\Huge{\centerline{\basmalah}}\end{Arabic}
\flushright{\begin{Arabic}
\quranayah[19][1]
\end{Arabic}}
\flushleft{\begin{hindi}
काफ॰ हा॰ या॰ ऐन॰ साद॰
\end{hindi}}
\flushright{\begin{Arabic}
\quranayah[19][2]
\end{Arabic}}
\flushleft{\begin{hindi}
वर्णन है तेरे रब की दयालुता का, जो उसने अपने बन्दे ज़करीया पर दर्शाई,
\end{hindi}}
\flushright{\begin{Arabic}
\quranayah[19][3]
\end{Arabic}}
\flushleft{\begin{hindi}
जबकि उसने अपने रब को चुपके से पुकारा
\end{hindi}}
\flushright{\begin{Arabic}
\quranayah[19][4]
\end{Arabic}}
\flushleft{\begin{hindi}
उसने कहा, "मेरे रब! मेरी हड्डियाँ कमज़ोर हो गई और सिर बुढापे से भड़क उठा। और मेरे रब! तुझे पुकारकर मैं कभी बेनसीब नहीं रहा
\end{hindi}}
\flushright{\begin{Arabic}
\quranayah[19][5]
\end{Arabic}}
\flushleft{\begin{hindi}
मुझे अपने पीछे अपने भाई-बन्धुओं की ओर से भय है और मेरी पत्नी बाँझ है। अतः तू मुझे अपने पास से एक उत्ताराधिकारी प्रदान कर,
\end{hindi}}
\flushright{\begin{Arabic}
\quranayah[19][6]
\end{Arabic}}
\flushleft{\begin{hindi}
जो मेरा भी उत्तराधिकारी हो और याक़ूब के वशंज का भी उत्तराधिकारी हो। और उसे मेरे रब! वांछनीय बना।"
\end{hindi}}
\flushright{\begin{Arabic}
\quranayah[19][7]
\end{Arabic}}
\flushleft{\begin{hindi}
(उत्तर मिला,) "ऐ ज़करीया! हम तुझे एक लड़के की शुभ सूचना देते है, जिसका नाम यह्यार होगा। हमने उससे पहले किसी को उसके जैसा नहीं बनाया।"
\end{hindi}}
\flushright{\begin{Arabic}
\quranayah[19][8]
\end{Arabic}}
\flushleft{\begin{hindi}
उसने कहा, "मेरे रब! मेरे लड़का कहाँ से होगा, जबकि मेरी पत्नी बाँझ है और मैं बुढ़ापे की अन्तिम अवस्था को पहुँच चुका हूँ?"
\end{hindi}}
\flushright{\begin{Arabic}
\quranayah[19][9]
\end{Arabic}}
\flushleft{\begin{hindi}
कहा, "ऐसा ही होगा। तेरे रब ने कहा कि यह मेरे लिए सरल है। इससे पहले मैं तुझे पैदा कर चुका हूँ, जबकि तू कुछ भी न था।"
\end{hindi}}
\flushright{\begin{Arabic}
\quranayah[19][10]
\end{Arabic}}
\flushleft{\begin{hindi}
उसने कहा, "मेरे रब! मेरे लिए कोई निशानी निश्चित कर दे।" कहा, "तेरी निशानी यह है कि तू भला-चंगा रहकर भी तीन रात (और दिन) लोगों से बात न करे।"
\end{hindi}}
\flushright{\begin{Arabic}
\quranayah[19][11]
\end{Arabic}}
\flushleft{\begin{hindi}
अतः वह मेहराब से निकलकर अपने लोगों के पास आया और उनसे संकेतों में कहा, "प्रातः काल और सन्ध्या समय तसबीह करते रहो।"
\end{hindi}}
\flushright{\begin{Arabic}
\quranayah[19][12]
\end{Arabic}}
\flushleft{\begin{hindi}
"ऐ यह्याऔ! किताब को मज़बूत थाम ले।" हमने उसे बचपन ही में निर्णय-शक्ति प्रदान की,
\end{hindi}}
\flushright{\begin{Arabic}
\quranayah[19][13]
\end{Arabic}}
\flushleft{\begin{hindi}
और अपने पास से नरमी और शौक़ और आत्मविश्वास। और वह बड़ा डरनेवाला था
\end{hindi}}
\flushright{\begin{Arabic}
\quranayah[19][14]
\end{Arabic}}
\flushleft{\begin{hindi}
और अपने माँ-बाप का हक़ पहचानेवाला था। और वह सरकश अवज्ञाकारी न था
\end{hindi}}
\flushright{\begin{Arabic}
\quranayah[19][15]
\end{Arabic}}
\flushleft{\begin{hindi}
"सलाम उस पर, जिस दिन वह पैदा हुआ और जिस दिन उसकी मृत्यु हो और जिस दिन वह जीवित करके उठाया जाए!"
\end{hindi}}
\flushright{\begin{Arabic}
\quranayah[19][16]
\end{Arabic}}
\flushleft{\begin{hindi}
और इस किताब में मरयम की चर्चा करो, जबकि वह अपने घरवालों से अलग होकर एक पूर्वी स्थान पर चली गई
\end{hindi}}
\flushright{\begin{Arabic}
\quranayah[19][17]
\end{Arabic}}
\flushleft{\begin{hindi}
फिर उसने उनसे परदा कर लिया। तब हमने उसके पास अपनी रूह (फ़रिश्तेप) को भेजा और वह उसके सामने एक पूर्ण मनुष्य के रूप में प्रकट हुआ
\end{hindi}}
\flushright{\begin{Arabic}
\quranayah[19][18]
\end{Arabic}}
\flushleft{\begin{hindi}
वह बोल उठी, "मैं तुझसे बचने के लिए रहमान की पनाह माँगती हूँ; यदि तू (अल्लाह का) डर रखनेवाला है (तो यहाँ से हट जाएगा) ।"
\end{hindi}}
\flushright{\begin{Arabic}
\quranayah[19][19]
\end{Arabic}}
\flushleft{\begin{hindi}
उसने कहा, "मैं तो केवल तेरे रब का भेजा हुआ हूँ, ताकि तुझे नेकी और भलाई से बढ़ा हुआ लड़का दूँ।"
\end{hindi}}
\flushright{\begin{Arabic}
\quranayah[19][20]
\end{Arabic}}
\flushleft{\begin{hindi}
वह बोली, "मेरे कहाँ से लड़का होगा, जबकि मुझे किसी आदमी ने छुआ तक नही और न मैं कोई बदचलन हूँ?"
\end{hindi}}
\flushright{\begin{Arabic}
\quranayah[19][21]
\end{Arabic}}
\flushleft{\begin{hindi}
उसने कहा, "ऐसा ही होगा। रब ने कहा है कि यह मेरे लिए सहज है। और ऐसा इसलिए होगा (ताकि हम तुझे) और ताकि हम उसे लोगों के लिए एक निशानी बनाएँ और अपनी ओर से एक दयालुता। यह तो ऐसी बात है जिसका निर्णय हो चुका है।"
\end{hindi}}
\flushright{\begin{Arabic}
\quranayah[19][22]
\end{Arabic}}
\flushleft{\begin{hindi}
फिर उसे उस (बच्चे) का गर्भ रह गया और वह उसे लिए हुए एक दूर के स्थान पर अलग चली गई।
\end{hindi}}
\flushright{\begin{Arabic}
\quranayah[19][23]
\end{Arabic}}
\flushleft{\begin{hindi}
अन्ततः प्रसव पीड़ा उसे एक खजूर के तने के पास ले आई। वह कहने लगी, "क्या ही अच्छा होता कि मैं इससे पहले ही मर जाती और भूली-बिसरी हो गई होती!"
\end{hindi}}
\flushright{\begin{Arabic}
\quranayah[19][24]
\end{Arabic}}
\flushleft{\begin{hindi}
उस समय उसे उसके नीचे से पुकारा, "शोकाकुल न हो। तेरे रब ने तेरे नीचे एक स्रोत प्रवाहित कर रखा है।
\end{hindi}}
\flushright{\begin{Arabic}
\quranayah[19][25]
\end{Arabic}}
\flushleft{\begin{hindi}
तू खजूर के उस वृक्ष के तने को पकड़कर अपनी ओर हिला। तेरे ऊपर ताज़ा पकी-पकी खजूरें टपक पड़ेगी
\end{hindi}}
\flushright{\begin{Arabic}
\quranayah[19][26]
\end{Arabic}}
\flushleft{\begin{hindi}
अतः तू उसे खा और पी और आँखें ठंडी कर। फिर यदि तू किसी आदमी को देखे तो कह देना, मैंने तो रहमान के लिए रोज़े की मन्नत मानी है। इसलिए मैं आज किसी मनुष्य से न बोलूँगी।"
\end{hindi}}
\flushright{\begin{Arabic}
\quranayah[19][27]
\end{Arabic}}
\flushleft{\begin{hindi}
फिर वह उस बच्चे को लिए हुए अपनी क़ौम के लोगों के पास आई। वे बोले, "ऐ मरयम, तूने तो बड़ा ही आश्चर्य का काम कर डाला!
\end{hindi}}
\flushright{\begin{Arabic}
\quranayah[19][28]
\end{Arabic}}
\flushleft{\begin{hindi}
हे हारून की बहन! न तो तेरा बाप ही कोई बुरा आदमी था और न तेरी माँ ही बदचलन थी।"
\end{hindi}}
\flushright{\begin{Arabic}
\quranayah[19][29]
\end{Arabic}}
\flushleft{\begin{hindi}
तब उसने उस (बच्चे) की ओर संकेत किया। वे कहने लगे, "हम उससे कैसे बात करें जो पालने में पड़ा हुआ एक बच्चा है?"
\end{hindi}}
\flushright{\begin{Arabic}
\quranayah[19][30]
\end{Arabic}}
\flushleft{\begin{hindi}
उसने कहा, "मैं अल्लाह का बन्दा हूँ। उसने मुझे किताब दी और मुझे नबी बनाया
\end{hindi}}
\flushright{\begin{Arabic}
\quranayah[19][31]
\end{Arabic}}
\flushleft{\begin{hindi}
और मुझे बरकतवाला किया जहाँ भी मैं रहूँ, और मुझे नमाज़ और ज़कात की ताकीद की, जब तक कि मैं जीवित रहूँ
\end{hindi}}
\flushright{\begin{Arabic}
\quranayah[19][32]
\end{Arabic}}
\flushleft{\begin{hindi}
और अपनी माँ का हक़ अदा करनेवाला बनाया। और उसने मुझे सरकश और बेनसीब नहीं बनाया
\end{hindi}}
\flushright{\begin{Arabic}
\quranayah[19][33]
\end{Arabic}}
\flushleft{\begin{hindi}
सलाम है मुझपर जिस दिन कि मैं पैदा हुआ और जिस दिन कि मैं मरूँ और जिस दिन कि जीवित करके उठाया जाऊँ!"
\end{hindi}}
\flushright{\begin{Arabic}
\quranayah[19][34]
\end{Arabic}}
\flushleft{\begin{hindi}
सच्ची और पक्की बात की स्पष्ट से यह है कि मरयम का बेटा ईसा, जिसके विषय में वे सन्देह में पड़े हुए है
\end{hindi}}
\flushright{\begin{Arabic}
\quranayah[19][35]
\end{Arabic}}
\flushleft{\begin{hindi}
अल्लाह ऐसा नहीं कि वह किसी को अपना बेटा बनाए। महान और उच्च है, वह! जब वह किसी चीज़ का फ़ैसला करता है तो बस उसे कह देता है, "हो जा!" तो वह हो जाती है। -
\end{hindi}}
\flushright{\begin{Arabic}
\quranayah[19][36]
\end{Arabic}}
\flushleft{\begin{hindi}
"और निस्संदेह अल्लाह मेरा रब भी है और तुम्हारा रब भी। अतः तुम उसी की बन्दगी करो यही सीधा मार्ग है।"
\end{hindi}}
\flushright{\begin{Arabic}
\quranayah[19][37]
\end{Arabic}}
\flushleft{\begin{hindi}
किन्तु उनमें कितने ही गिरोहों ने पारस्परिक वैमनस्य के कारण विभेद किया, तो जिन लोगों ने इनकार किया उनके लिए बड़ी तबाही है एक बड़े दिन की उपस्थिति से
\end{hindi}}
\flushright{\begin{Arabic}
\quranayah[19][38]
\end{Arabic}}
\flushleft{\begin{hindi}
भली-भाँति सुननेवाले और भली-भाँति देखनेवाले होंगे, जिस दिन वे हमारे समाने आएँगे! किन्तु आज ये ज़ालिम खुली गुमराही में पड़े हुए है
\end{hindi}}
\flushright{\begin{Arabic}
\quranayah[19][39]
\end{Arabic}}
\flushleft{\begin{hindi}
उन्हें पश्चाताप के दिन से डराओ, जबकि मामले का फ़ैसला कर दिया जाएगा, और उनका हाल यह है कि वे ग़फ़लत में पड़े हुए है और वे ईमान नहीं ला रहे है
\end{hindi}}
\flushright{\begin{Arabic}
\quranayah[19][40]
\end{Arabic}}
\flushleft{\begin{hindi}
धरती और जो भी उसके ऊपर है उसके वारिस हम ही रह जाएँगे और हमारी ही ओर उन्हें लौटना होगा
\end{hindi}}
\flushright{\begin{Arabic}
\quranayah[19][41]
\end{Arabic}}
\flushleft{\begin{hindi}
और इस किताब में इबराहीम की चर्चा करो। निस्संदेह वह एक सत्यवान नबी था
\end{hindi}}
\flushright{\begin{Arabic}
\quranayah[19][42]
\end{Arabic}}
\flushleft{\begin{hindi}
जबकि उसने अपने बाप से कहा, "ऐ मेरे बाप! आप उस चीज़ को क्यों पूजते हो, जो न सुने और न देखे और न आपके कुछ काम आए?
\end{hindi}}
\flushright{\begin{Arabic}
\quranayah[19][43]
\end{Arabic}}
\flushleft{\begin{hindi}
ऐ मेरे बाप! मेरे पास ज्ञान आ गया है जो आपके पास नहीं आया। अतः आप मेरा अनुसरण करें, मैं आपको सीधा मार्ग दिखाऊँगा
\end{hindi}}
\flushright{\begin{Arabic}
\quranayah[19][44]
\end{Arabic}}
\flushleft{\begin{hindi}
ऐ मेरे बाप! शैतान की बन्दगी न कीजिए। शैतान तो रहमान का अवज्ञाकारी है
\end{hindi}}
\flushright{\begin{Arabic}
\quranayah[19][45]
\end{Arabic}}
\flushleft{\begin{hindi}
ऐ मेरे बाप! मैं डरता हूँ कि कहीं आपको रहमान की कोई यातना न आ पकड़े और आप शैतान के साथी होकर रह जाएँ।"
\end{hindi}}
\flushright{\begin{Arabic}
\quranayah[19][46]
\end{Arabic}}
\flushleft{\begin{hindi}
उसने कहा, "ऐ इबराहीम! क्या तू मेरे उपास्यों से फिर गया है? यदि तू बाज़ न आया तो मैं तुझपर पथराव कर दूँगा। तू अलग हो जा मुझसे मुद्दत के लिए!"
\end{hindi}}
\flushright{\begin{Arabic}
\quranayah[19][47]
\end{Arabic}}
\flushleft{\begin{hindi}
कहा, "सलाम है आपको! मैं आपके लिए रब से क्षमा की प्रार्थना करूँगा। वह तो मुझपर बहुत मेहरबान है
\end{hindi}}
\flushright{\begin{Arabic}
\quranayah[19][48]
\end{Arabic}}
\flushleft{\begin{hindi}
मैं आप लोगों को छोड़ता हूँ और उनको भी जिन्हें अल्लाह से हटकर आप लोग पुकारा करते है। मैं तो अपने रब को पुकारूँगा। आशा है कि मैं अपने रब को पुकारकर बेनसीब नहीं रहूँगा।"
\end{hindi}}
\flushright{\begin{Arabic}
\quranayah[19][49]
\end{Arabic}}
\flushleft{\begin{hindi}
फिर जब वह उन लोगों से और जिन्हें वे अल्लाह के सिवा पूजते थे उनसे अलग हो गया, तो हमने उसे इसहाक़ और याक़ूब प्रदान किए और हर एक को हमने नबी बनाया
\end{hindi}}
\flushright{\begin{Arabic}
\quranayah[19][50]
\end{Arabic}}
\flushleft{\begin{hindi}
और उन्हें अपनी दयालुता से हिस्सा दिया। और उन्हें एक सच्ची उच्च ख्याति प्रदान की
\end{hindi}}
\flushright{\begin{Arabic}
\quranayah[19][51]
\end{Arabic}}
\flushleft{\begin{hindi}
और इस किताब में मूसा की चर्चा करो। निस्संदेह वह चुना हुआ था और एक रसूल, नबी था
\end{hindi}}
\flushright{\begin{Arabic}
\quranayah[19][52]
\end{Arabic}}
\flushleft{\begin{hindi}
हमने उसे 'तूर' के मुबारक छोर से पुकारा और रहस्य की बातें करने के लिए हमने उसे समीप किया
\end{hindi}}
\flushright{\begin{Arabic}
\quranayah[19][53]
\end{Arabic}}
\flushleft{\begin{hindi}
और अपनी दयालुता से अपने भाई हारून को नबी बनाकर उसे दिया
\end{hindi}}
\flushright{\begin{Arabic}
\quranayah[19][54]
\end{Arabic}}
\flushleft{\begin{hindi}
और इस किताब में इसमाईल की चर्चा करो। निस्संदेह वह वादे का सच्च, नबी था
\end{hindi}}
\flushright{\begin{Arabic}
\quranayah[19][55]
\end{Arabic}}
\flushleft{\begin{hindi}
और अपने लोगों को नमाज़ और ज़कात का हुक्म देता था। और वह अपने रब के यहाँ प्रीतिकर व्यक्ति था
\end{hindi}}
\flushright{\begin{Arabic}
\quranayah[19][56]
\end{Arabic}}
\flushleft{\begin{hindi}
और इस किताब में इदरीस की भी चर्चा करो। वह अत्यन्त सत्यवान, एक नबी था
\end{hindi}}
\flushright{\begin{Arabic}
\quranayah[19][57]
\end{Arabic}}
\flushleft{\begin{hindi}
हमने उसे उच्च स्थान पर उठाया था
\end{hindi}}
\flushright{\begin{Arabic}
\quranayah[19][58]
\end{Arabic}}
\flushleft{\begin{hindi}
ये वे पैग़म्बर है जो अल्लाह के कृपापात्र हुए, आदम की सन्तान में से और उन लोगों के वंशज में से जिनको हमने नूह के साथ सवार किया, और इबराहीम और इसराईल के वंशज में से और उनमें से जिनको हमने सीधा मार्ग दिखाया और चुन लिया। जब उन्हें रहमान की आयतें सुनाई जातीं तो वे सजदा करते और रोते हुए गिर पड़ते थे
\end{hindi}}
\flushright{\begin{Arabic}
\quranayah[19][59]
\end{Arabic}}
\flushleft{\begin{hindi}
फिर उनके पश्चात ऐसे बुरे लोग उनके उत्तराधिकारी हुए, जिन्होंने नमाज़ को गँवाया और मन की इच्छाओं के पीछे पड़े। अतः जल्द ही वे गुमराही (के परिणाम) से दोचार होंगा
\end{hindi}}
\flushright{\begin{Arabic}
\quranayah[19][60]
\end{Arabic}}
\flushleft{\begin{hindi}
किन्तु जो तौबा करे और ईमान लाए और अच्छा कर्म करे, तो ऐसे लोग जन्नत में प्रवेश करेंगे। उनपर कुछ भी ज़ुल्म न होगा। -
\end{hindi}}
\flushright{\begin{Arabic}
\quranayah[19][61]
\end{Arabic}}
\flushleft{\begin{hindi}
अदन (रहने) के बाग़ जिनका रहमान ने अपने बन्दों से परोक्ष में होते हुए वादा किया है। निश्चय ही उसके वादे पर उपस्थित हाना है। -
\end{hindi}}
\flushright{\begin{Arabic}
\quranayah[19][62]
\end{Arabic}}
\flushleft{\begin{hindi}
वहाँ वे 'सलाम' के सिवा कोई व्यर्थ बात नहीं सुनेंगे। उनकी रोज़ी उन्हें वहाँ प्रातः और सन्ध्या समय प्राप्त होती रहेगी
\end{hindi}}
\flushright{\begin{Arabic}
\quranayah[19][63]
\end{Arabic}}
\flushleft{\begin{hindi}
यह है वह जन्नत जिसका वारिस हम अपने बन्दों में से हर उस व्यक्ति को बनाएँगे, जो डर रखनेवाला हो
\end{hindi}}
\flushright{\begin{Arabic}
\quranayah[19][64]
\end{Arabic}}
\flushleft{\begin{hindi}
हम तुम्हारे रब की आज्ञा के बिना नहीं उतरते। जो कुछ हमारे आगे है और जो कुछ हमारे पीछे है और जो कुछ इसके मध्य है सब उसी का है, और तुम्हारा रब भूलनेवाला नहीं है
\end{hindi}}
\flushright{\begin{Arabic}
\quranayah[19][65]
\end{Arabic}}
\flushleft{\begin{hindi}
आकाशों और धरती का रब है और उसका भी जो इन दोनों के मध्य है। अतः तुम उसी की बन्दगी पर जमे रहो। क्या तुम्हारे ज्ञान में उस जैसा कोई है?
\end{hindi}}
\flushright{\begin{Arabic}
\quranayah[19][66]
\end{Arabic}}
\flushleft{\begin{hindi}
और मनुष्य कहता है, "क्या जब मैं मर गया तो फिर जीवित करके निकाला जाऊँगा?"
\end{hindi}}
\flushright{\begin{Arabic}
\quranayah[19][67]
\end{Arabic}}
\flushleft{\begin{hindi}
क्या मनुष्य याद नहीं करता कि हम उसे इससे पहले पैदा कर चुके है, जबकि वह कुछ भी न था?
\end{hindi}}
\flushright{\begin{Arabic}
\quranayah[19][68]
\end{Arabic}}
\flushleft{\begin{hindi}
अतः तुम्हारे रब की क़सम! हम अवश्य उन्हें और शैतानों को भी इकट्ठा करेंगे। फिर हम उन्हें जहन्नम के चतुर्दिक इस दशा में ला उपस्थित करेंगे कि वे घुटनों के बल झुके होंगे
\end{hindi}}
\flushright{\begin{Arabic}
\quranayah[19][69]
\end{Arabic}}
\flushleft{\begin{hindi}
फिर प्रत्येक गिरोह में से हम अवश्य ही उसे छाँटकर अलग करेंगे जो उनमें से रहमान (कृपाशील प्रभु) के मुक़ाबले में सबसे बढ़कर सरकश रहा होगा
\end{hindi}}
\flushright{\begin{Arabic}
\quranayah[19][70]
\end{Arabic}}
\flushleft{\begin{hindi}
फिर हम उन्हें भली-भाँति जानते है जो उसमें झोंके जाने के सर्वाधिक योग्य है
\end{hindi}}
\flushright{\begin{Arabic}
\quranayah[19][71]
\end{Arabic}}
\flushleft{\begin{hindi}
तुममें से प्रत्येक को उसपर पहुँचना ही है। यह एक निश्चय पाई हुई बात है, जिसे पूरा करना तेरे रब के ज़िम्मे है।
\end{hindi}}
\flushright{\begin{Arabic}
\quranayah[19][72]
\end{Arabic}}
\flushleft{\begin{hindi}
फिर हम डर रखनेवालों को बचा लेंगे और ज़ालिमों को उसमें घुटनों के बल छोड़ देंगे
\end{hindi}}
\flushright{\begin{Arabic}
\quranayah[19][73]
\end{Arabic}}
\flushleft{\begin{hindi}
जब उन्हें हमारी खुली हुई आयतें सुनाई जाती है तो जिन लोगों ने कुफ़्र किया, वे ईमान लानेवालों से कहते हैं, "दोनों गिरोहों में स्थान की स्पष्ट से कौन उत्तम है और कौन मजलिस की दृष्टि से अधिक अच्छा है?"
\end{hindi}}
\flushright{\begin{Arabic}
\quranayah[19][74]
\end{Arabic}}
\flushleft{\begin{hindi}
हालाँकि उनसे पहले हम कितनी ही नसलों को विनष्ट कर चुके है जो सामग्री और बाह्य भव्यता में इनसे कहीं अच्छी थीं!
\end{hindi}}
\flushright{\begin{Arabic}
\quranayah[19][75]
\end{Arabic}}
\flushleft{\begin{hindi}
कह दो, "जो गुमराही में पड़ा हुआ है उसके प्रति तो यही चाहिए कि रहमान उसकी रस्सी ख़ूब ढीली छोड़ दे, यहाँ तक कि जब ऐसे लोग उस चीज़ को देख लेंगे जिसका उनसे वादा किया जाता है - चाहे यातना हो या क़ियामत की घड़ी - तो वे उस समय जान लेंगे कि अपने स्थान की स्पष्ट से कौन निकृष्ट और जत्थे की दृष्टि से अधिक कमजोर है।"
\end{hindi}}
\flushright{\begin{Arabic}
\quranayah[19][76]
\end{Arabic}}
\flushleft{\begin{hindi}
और जिन लोगों ने मार्ग पा लिया है, अल्लाह उनके मार्गदर्शन में अभिवृद्धि प्रदान करता है और शेष रहनेवाली नेकियाँ ही तुम्हारे रब के यहाँ बदले और अन्तिम परिणाम की स्पष्ट से उत्तम है
\end{hindi}}
\flushright{\begin{Arabic}
\quranayah[19][77]
\end{Arabic}}
\flushleft{\begin{hindi}
फिर क्या तुमने उस व्यक्ति को देखा जिसने हमारी आयतों का इनकार किया और कहा, "मुझे तो अवश्य ही धन और सन्तान मिलने को है?"
\end{hindi}}
\flushright{\begin{Arabic}
\quranayah[19][78]
\end{Arabic}}
\flushleft{\begin{hindi}
क्या उसने परोक्ष को झाँककर देख लिया है, या रहमान से कोई वचन ले रखा है?
\end{hindi}}
\flushright{\begin{Arabic}
\quranayah[19][79]
\end{Arabic}}
\flushleft{\begin{hindi}
कदापि नहीं, हम लिखेंगे जो कुछ वह कहता है और उसके लिए हम यातना को दीर्घ करते चले जाएँगे।
\end{hindi}}
\flushright{\begin{Arabic}
\quranayah[19][80]
\end{Arabic}}
\flushleft{\begin{hindi}
औऱ जो कुछ वह बताता है उसके वारिस हम होंगे और वह अकेला ही हमारे पास आएगा
\end{hindi}}
\flushright{\begin{Arabic}
\quranayah[19][81]
\end{Arabic}}
\flushleft{\begin{hindi}
और उन्होंने अल्लाह से इतर अपने कुछ पूज्य-प्रभु बना लिए है, ताकि वे उनके लिए शक्ति का कारण बनें।
\end{hindi}}
\flushright{\begin{Arabic}
\quranayah[19][82]
\end{Arabic}}
\flushleft{\begin{hindi}
कुछ नहीं, ये उनकी बन्दगी का इनकार करेंगे और उनके विरोधी बन जाएँगे। -
\end{hindi}}
\flushright{\begin{Arabic}
\quranayah[19][83]
\end{Arabic}}
\flushleft{\begin{hindi}
क्या तुमने देखा नहीं कि हमने शैतानों को छोड़ रखा है, जो इनकार करनेवालों पर नियुक्त है?
\end{hindi}}
\flushright{\begin{Arabic}
\quranayah[19][84]
\end{Arabic}}
\flushleft{\begin{hindi}
अतः तुम उनके लिए जल्दी न करो। हम तो बस उनके लिए (उनकी बातें) गिन रहे है
\end{hindi}}
\flushright{\begin{Arabic}
\quranayah[19][85]
\end{Arabic}}
\flushleft{\begin{hindi}
याद करो जिस दिन हम डर रखनेवालों के सम्मानित गिरोह के रूप में रहमान के पास इकट्ठा करेंगे।
\end{hindi}}
\flushright{\begin{Arabic}
\quranayah[19][86]
\end{Arabic}}
\flushleft{\begin{hindi}
और अपराधियों को जहन्नम के घाट की ओर प्यासा हाँक ले जाएँगे।
\end{hindi}}
\flushright{\begin{Arabic}
\quranayah[19][87]
\end{Arabic}}
\flushleft{\begin{hindi}
उन्हें सिफ़ारिश का अधिकार प्राप्त न होगा। सिवाय उसके, जिसने रहमान के यहाँ से अनुमोदन प्राप्त कर लिया हो
\end{hindi}}
\flushright{\begin{Arabic}
\quranayah[19][88]
\end{Arabic}}
\flushleft{\begin{hindi}
वे कहते है, "रहमान ने किसी को अपना बेटा बनाया है।"
\end{hindi}}
\flushright{\begin{Arabic}
\quranayah[19][89]
\end{Arabic}}
\flushleft{\begin{hindi}
अत्यन्त भारी बात है, जो तुम घड़ लाए हो!
\end{hindi}}
\flushright{\begin{Arabic}
\quranayah[19][90]
\end{Arabic}}
\flushleft{\begin{hindi}
निकट है कि आकाश इससे फट पड़े और धरती टुकड़े-टुकड़े हो जाए और पहाड़ धमाके के साथ गिर पड़े,
\end{hindi}}
\flushright{\begin{Arabic}
\quranayah[19][91]
\end{Arabic}}
\flushleft{\begin{hindi}
इस बात पर कि उन्होंने रहमान के लिए बेटा होने का दावा किया!
\end{hindi}}
\flushright{\begin{Arabic}
\quranayah[19][92]
\end{Arabic}}
\flushleft{\begin{hindi}
जबकि रहमान की प्रतिष्ठा के प्रतिकूल है कि वह किसी को अपना बेटा बनाए
\end{hindi}}
\flushright{\begin{Arabic}
\quranayah[19][93]
\end{Arabic}}
\flushleft{\begin{hindi}
आकाशों और धरती में जो कोई भी है एक बन्दें के रूप में रहमान के पास आनेवाला है
\end{hindi}}
\flushright{\begin{Arabic}
\quranayah[19][94]
\end{Arabic}}
\flushleft{\begin{hindi}
उसने उनका आकलन कर रखा है और उन्हें अच्छी तरह गिन रखा है
\end{hindi}}
\flushright{\begin{Arabic}
\quranayah[19][95]
\end{Arabic}}
\flushleft{\begin{hindi}
और उनमें से प्रत्येक क़ियामत के दिन उस अकेले (रहमान) के सामने उपस्थित होगा
\end{hindi}}
\flushright{\begin{Arabic}
\quranayah[19][96]
\end{Arabic}}
\flushleft{\begin{hindi}
निस्संदेह जो लोग ईमान लाए और उन्होंने अच्छे कर्म किए शीघ्र ही रहमान उनके लिए प्रेम उत्पन्न कर देगा
\end{hindi}}
\flushright{\begin{Arabic}
\quranayah[19][97]
\end{Arabic}}
\flushleft{\begin{hindi}
अतः हमने इस वाणी को तुम्हारी भाषा में इसी लिए सहज एवं उपयुक्त बनाया है, ताकि तुम इसके द्वारा डर रखनेवालों को शुभ सूचना दो और उन झगड़ालू लोगों को इसके द्वारा डराओ
\end{hindi}}
\flushright{\begin{Arabic}
\quranayah[19][98]
\end{Arabic}}
\flushleft{\begin{hindi}
उनसे पहले कितनी ही नसलों को हम विनष्ट कर चुके है। क्या उनमें किसी की आहट तुम पाते हो या उनकी कोई भनक सुनते हो?
\end{hindi}}
\chapter{Ta Ha (Ta Ha)}
\begin{Arabic}
\Huge{\centerline{\basmalah}}\end{Arabic}
\flushright{\begin{Arabic}
\quranayah[20][1]
\end{Arabic}}
\flushleft{\begin{hindi}
ता॰ हा॰।
\end{hindi}}
\flushright{\begin{Arabic}
\quranayah[20][2]
\end{Arabic}}
\flushleft{\begin{hindi}
हमने तुमपर यह क़ुरआन इसलिए नहीं उतारा कि तुम मशक़्क़त में पड़ जाओ
\end{hindi}}
\flushright{\begin{Arabic}
\quranayah[20][3]
\end{Arabic}}
\flushleft{\begin{hindi}
यह तो बस एक अनुस्मृति है, उसके लिए जो डरे,
\end{hindi}}
\flushright{\begin{Arabic}
\quranayah[20][4]
\end{Arabic}}
\flushleft{\begin{hindi}
भली-भाँति अवतरित हुआ है उस सत्ता की ओर से, जिसने पैदा किया है धरती और उच्च आकाशों को
\end{hindi}}
\flushright{\begin{Arabic}
\quranayah[20][5]
\end{Arabic}}
\flushleft{\begin{hindi}
वह रहमान है, जो राजासन पर विराजमान हुआ
\end{hindi}}
\flushright{\begin{Arabic}
\quranayah[20][6]
\end{Arabic}}
\flushleft{\begin{hindi}
उसी का है जो कुछ आकाशों में है और जो कुछ धरती में है और जो कुछ इन दोनों के मध्य है और जो कुछ आर्द्र मिट्टी के नीचे है
\end{hindi}}
\flushright{\begin{Arabic}
\quranayah[20][7]
\end{Arabic}}
\flushleft{\begin{hindi}
तुम चाहे बात पुकार कर कहो (या चुपके से), वह तो छिपी हुई और अत्यन्त गुप्त बात को भी जानता है
\end{hindi}}
\flushright{\begin{Arabic}
\quranayah[20][8]
\end{Arabic}}
\flushleft{\begin{hindi}
अल्लाह, कि उसके सिवा कोई पूज्य-प्रभू नहीं। उसके नाम बहुत ही अच्छे हैं।
\end{hindi}}
\flushright{\begin{Arabic}
\quranayah[20][9]
\end{Arabic}}
\flushleft{\begin{hindi}
क्या तुम्हें मूसा की भी ख़बर पहुँची?
\end{hindi}}
\flushright{\begin{Arabic}
\quranayah[20][10]
\end{Arabic}}
\flushleft{\begin{hindi}
जबकि उसने एक आग देखी तो उसने अपने घरवालों से कहा, "ठहरो! मैंने एक आग देखी है। शायद कि तुम्हारे लिए उसमें से कोई अंगारा ले आऊँ या उस आग पर मैं मार्ग का पता पा लूँ।"
\end{hindi}}
\flushright{\begin{Arabic}
\quranayah[20][11]
\end{Arabic}}
\flushleft{\begin{hindi}
फिर जब वह वहाँ पहुँचा तो पुकारा गया, "ऐ मूसा!
\end{hindi}}
\flushright{\begin{Arabic}
\quranayah[20][12]
\end{Arabic}}
\flushleft{\begin{hindi}
मैं ही तेरा रब हूँ। अपने जूते उतार दे। तू पवित्र घाटी 'तुवा' में है
\end{hindi}}
\flushright{\begin{Arabic}
\quranayah[20][13]
\end{Arabic}}
\flushleft{\begin{hindi}
मैंने तुझे चुन लिया है। अतः सुन, जो कुछ प्रकाशना की जाती है
\end{hindi}}
\flushright{\begin{Arabic}
\quranayah[20][14]
\end{Arabic}}
\flushleft{\begin{hindi}
निस्संदेह मैं ही अल्लाह हूँ। मेरे सिवा कोई पूज्य-प्रभु नहीं। अतः तू मेरी बन्दगी कर और मेरी याद के लिए नमाज़ क़ायम कर
\end{hindi}}
\flushright{\begin{Arabic}
\quranayah[20][15]
\end{Arabic}}
\flushleft{\begin{hindi}
निश्चय ही वह (क़ियामत की) घड़ी आनेवाली है - शीघ्र ही उसे लाऊँगा, उसे छिपाए रखता हूँ - ताकि प्रत्येक व्यक्ति जो प्रयास वह करता है, उसका बदला पाए
\end{hindi}}
\flushright{\begin{Arabic}
\quranayah[20][16]
\end{Arabic}}
\flushleft{\begin{hindi}
अतः जो कोई उसपर ईमान नहीं लाता और अपनी वासना के पीछे पड़ा है, वह तुझे उससे रोक न दे, अन्यथा तू विनष्ट हो जाएगा
\end{hindi}}
\flushright{\begin{Arabic}
\quranayah[20][17]
\end{Arabic}}
\flushleft{\begin{hindi}
और ऐ मूसा! यह तेरे दाहिने हाथ में क्या है?"
\end{hindi}}
\flushright{\begin{Arabic}
\quranayah[20][18]
\end{Arabic}}
\flushleft{\begin{hindi}
उसने कहा, "यह मेरी लाठी है। मैं इसपर टेक लगाता हूँ और इससे अपनी बकरियों के लिए पत्ते झाड़ता हूँ और इससे मेरी दूसरी ज़रूरतें भी पूरी होती है।"
\end{hindi}}
\flushright{\begin{Arabic}
\quranayah[20][19]
\end{Arabic}}
\flushleft{\begin{hindi}
कहा, "डाल दे उसे, ऐ मूसा!"
\end{hindi}}
\flushright{\begin{Arabic}
\quranayah[20][20]
\end{Arabic}}
\flushleft{\begin{hindi}
अतः उसने डाल दिया। सहसा क्या देखते है कि वह एक साँप है, जो दौड़ रहा है
\end{hindi}}
\flushright{\begin{Arabic}
\quranayah[20][21]
\end{Arabic}}
\flushleft{\begin{hindi}
कहा, "इसे पकड़ ले और डर मत। हम इसे इसकी पहली हालत पर लौटा देंगे
\end{hindi}}
\flushright{\begin{Arabic}
\quranayah[20][22]
\end{Arabic}}
\flushleft{\begin{hindi}
और अपने हाथ अपने बाज़ू की ओर समेट ले। वह बिना किसी ऐब के रौशन दूसरी निशानी के रूप में निकलेगा
\end{hindi}}
\flushright{\begin{Arabic}
\quranayah[20][23]
\end{Arabic}}
\flushleft{\begin{hindi}
इसलिए कि हम तुझे अपनी बड़ी निशानियाँ दिखाएँ
\end{hindi}}
\flushright{\begin{Arabic}
\quranayah[20][24]
\end{Arabic}}
\flushleft{\begin{hindi}
तू फ़िरऔन के पास जा। वह बहुत सरकश हो गया है।"
\end{hindi}}
\flushright{\begin{Arabic}
\quranayah[20][25]
\end{Arabic}}
\flushleft{\begin{hindi}
उसने निवेदन किया, "मेरे रब! मेरा सीना मेरे लिए खोल दे
\end{hindi}}
\flushright{\begin{Arabic}
\quranayah[20][26]
\end{Arabic}}
\flushleft{\begin{hindi}
और मेरे काम को मेरे लिए आसान कर दे
\end{hindi}}
\flushright{\begin{Arabic}
\quranayah[20][27]
\end{Arabic}}
\flushleft{\begin{hindi}
और मेरी ज़बान की गिरह खोल दे।
\end{hindi}}
\flushright{\begin{Arabic}
\quranayah[20][28]
\end{Arabic}}
\flushleft{\begin{hindi}
कि वे मेरी बात समझ सकें
\end{hindi}}
\flushright{\begin{Arabic}
\quranayah[20][29]
\end{Arabic}}
\flushleft{\begin{hindi}
और मेरे लिए अपने घरवालों में से एक सहायक नियुक्त कर दें,
\end{hindi}}
\flushright{\begin{Arabic}
\quranayah[20][30]
\end{Arabic}}
\flushleft{\begin{hindi}
हारून को, जो मेरा भाई है
\end{hindi}}
\flushright{\begin{Arabic}
\quranayah[20][31]
\end{Arabic}}
\flushleft{\begin{hindi}
उसके द्वारा मेरी कमर मज़बूत कर
\end{hindi}}
\flushright{\begin{Arabic}
\quranayah[20][32]
\end{Arabic}}
\flushleft{\begin{hindi}
और उसे मेरे काम में शरीक कर दें,
\end{hindi}}
\flushright{\begin{Arabic}
\quranayah[20][33]
\end{Arabic}}
\flushleft{\begin{hindi}
कि हम अधिक से अधिक तेरी तसबीह करें
\end{hindi}}
\flushright{\begin{Arabic}
\quranayah[20][34]
\end{Arabic}}
\flushleft{\begin{hindi}
और तुझे ख़ूब याद करें
\end{hindi}}
\flushright{\begin{Arabic}
\quranayah[20][35]
\end{Arabic}}
\flushleft{\begin{hindi}
निश्चय ही तू हमें खूब देख रहा है।"
\end{hindi}}
\flushright{\begin{Arabic}
\quranayah[20][36]
\end{Arabic}}
\flushleft{\begin{hindi}
कहा, "दिया गया तुझे जो तूने माँगा, ऐ मूसा!
\end{hindi}}
\flushright{\begin{Arabic}
\quranayah[20][37]
\end{Arabic}}
\flushleft{\begin{hindi}
हम तो तुझपर एक बार और भी उपकार कर चुके है
\end{hindi}}
\flushright{\begin{Arabic}
\quranayah[20][38]
\end{Arabic}}
\flushleft{\begin{hindi}
जब हमने तेरी माँ के दिल में यह बात डाली थी, जो अब प्रकाशना की जा रही है,
\end{hindi}}
\flushright{\begin{Arabic}
\quranayah[20][39]
\end{Arabic}}
\flushleft{\begin{hindi}
कि उसको सन्दूक में रख दे; फिर उसे दरिया में डाल दे; फिर दरिया उसे तट पर डाल दे कि उसे मेरा शत्रु और उसका शत्रु उठा ले। मैंने अपनी ओर से तुझपर अपना प्रेम डाला। (ताकि तू सुरक्षित रहे) और ताकि मेरी आँख के सामने तेरा पालन-पोषण हो
\end{hindi}}
\flushright{\begin{Arabic}
\quranayah[20][40]
\end{Arabic}}
\flushleft{\begin{hindi}
याद कर जबकि तेरी बहन जाती और कहती थी, क्या मैं तुम्हें उसका पता बता दूँ जो इसका पालन-पोषण अपने ज़िम्मे ले ले? इस प्रकार हमने फिर तुझे तेरी माँ के पास पहुँचा दिया, ताकि उसकी आँख ठंड़ी हो और वह शोकाकुल न हो। और हमने तुझे भली-भाँति परखा। फिर तू कई वर्ष मदयन के लोगों में ठहरा रहा। फिर ऐ मूसा! तू ख़ास समय पर आ गया है
\end{hindi}}
\flushright{\begin{Arabic}
\quranayah[20][41]
\end{Arabic}}
\flushleft{\begin{hindi}
हमने तुझे अपने लिए तैयार किया है
\end{hindi}}
\flushright{\begin{Arabic}
\quranayah[20][42]
\end{Arabic}}
\flushleft{\begin{hindi}
जो, तू और तेरी भाई मेरी निशानियो के साथ; और मेरी याद में ढ़ीले मत पड़ना
\end{hindi}}
\flushright{\begin{Arabic}
\quranayah[20][43]
\end{Arabic}}
\flushleft{\begin{hindi}
जाओ दोनों, फ़िरऔन के पास, वह सरकश हो गया है
\end{hindi}}
\flushright{\begin{Arabic}
\quranayah[20][44]
\end{Arabic}}
\flushleft{\begin{hindi}
उससे नर्म बात करना, कदाचित वह ध्यान दे या डरे।"
\end{hindi}}
\flushright{\begin{Arabic}
\quranayah[20][45]
\end{Arabic}}
\flushleft{\begin{hindi}
दोनों ने कहा, "ऐ हमारे रब! हमें इसका भय है कि वह हमपर ज़्यादती करे या सरकशी करने लग जाए।"
\end{hindi}}
\flushright{\begin{Arabic}
\quranayah[20][46]
\end{Arabic}}
\flushleft{\begin{hindi}
कहा, "डरो नहीं, मै तुम्हारे साथ हूँ। सुनता और देखता हूँ
\end{hindi}}
\flushright{\begin{Arabic}
\quranayah[20][47]
\end{Arabic}}
\flushleft{\begin{hindi}
अतः जाओ, उसके पास और कहो, हम तेरे रब के रसूल है। इसराईल की सन्तान को हमारे साथ भेज दे। और उन्हें यातना न दे। हम तेरे पास तेरे रब की निशानी लेकर आए है। और सलामती है उसके लिए जो संमार्ग का अनुसरण करे!
\end{hindi}}
\flushright{\begin{Arabic}
\quranayah[20][48]
\end{Arabic}}
\flushleft{\begin{hindi}
निस्संदेह हमारी ओर प्रकाशना हुई है कि यातना उसके लिए है, जो झुठलाए और मुँह फेरे।"
\end{hindi}}
\flushright{\begin{Arabic}
\quranayah[20][49]
\end{Arabic}}
\flushleft{\begin{hindi}
उसने कहा, "अच्छा, तुम दोनों का रब कौन है, मूसा?"
\end{hindi}}
\flushright{\begin{Arabic}
\quranayah[20][50]
\end{Arabic}}
\flushleft{\begin{hindi}
कहा, "हमारा रब वह है जिसने हर चीज़ को उसकी आकृति दी, फिर तदनुकूव निर्देशन किया।"
\end{hindi}}
\flushright{\begin{Arabic}
\quranayah[20][51]
\end{Arabic}}
\flushleft{\begin{hindi}
उसने कहा, "अच्छा तो उन नस्लों का क्या हाल है, जो पहले थी?"
\end{hindi}}
\flushright{\begin{Arabic}
\quranayah[20][52]
\end{Arabic}}
\flushleft{\begin{hindi}
कहा, "उसका ज्ञान मेरे रब के पास एक किताब में सुरक्षित है। मेरा रब न चूकता है और न भूलता है।"
\end{hindi}}
\flushright{\begin{Arabic}
\quranayah[20][53]
\end{Arabic}}
\flushleft{\begin{hindi}
"वही है जिसने तुम्हारे लिए धरती को पालना (बिछौना) बनाया और उसने तुम्हारे लिए रास्ते निकाले और आकाश से पानी उतारा। फिर हमने उसके द्वारा विभिन्न प्रकार के पेड़-पौधे निकाले
\end{hindi}}
\flushright{\begin{Arabic}
\quranayah[20][54]
\end{Arabic}}
\flushleft{\begin{hindi}
खाओ और अपने चौपायों को भी चराओ! निस्संदेह इसमें बुद्धिमानों के लिए बहुत-सी निशानियाँ है
\end{hindi}}
\flushright{\begin{Arabic}
\quranayah[20][55]
\end{Arabic}}
\flushleft{\begin{hindi}
उसी से हमने तुम्हें पैदा किया और उसी में हम तुम्हें लौटाते है और उसी से तुम्हें दूसरी बार निकालेंगे।"
\end{hindi}}
\flushright{\begin{Arabic}
\quranayah[20][56]
\end{Arabic}}
\flushleft{\begin{hindi}
और हमने फ़िरऔन को अपनी सब निशानियाँ दिखाई, किन्तु उसने झुठलाया और इनकार किया।-
\end{hindi}}
\flushright{\begin{Arabic}
\quranayah[20][57]
\end{Arabic}}
\flushleft{\begin{hindi}
उसने कहा, "ऐ मूसा! क्या तू हमारे पास इसलिए आया है कि अपने जादू से हमको हमारे अपने भूभाग से निकाल दे?
\end{hindi}}
\flushright{\begin{Arabic}
\quranayah[20][58]
\end{Arabic}}
\flushleft{\begin{hindi}
अच्छा, हम भी तेरे पास ऐसा ही जादू लाते है। अब हमारे और अपने बीच एक निश्चित स्थान ठहरा ले, कोई बीच की जगह, न हम इसके विरुद्ध जाएँ और न तू।"
\end{hindi}}
\flushright{\begin{Arabic}
\quranayah[20][59]
\end{Arabic}}
\flushleft{\begin{hindi}
कहा, "उत्सव का दिन तुम्हारे वादे का है और यह कि लोग दिन चढ़े इकट्ठे हो जाएँ।"
\end{hindi}}
\flushright{\begin{Arabic}
\quranayah[20][60]
\end{Arabic}}
\flushleft{\begin{hindi}
तब फ़िरऔन ने पलटकर अपने सारे हथकंडे जुटाए। और आ गया
\end{hindi}}
\flushright{\begin{Arabic}
\quranayah[20][61]
\end{Arabic}}
\flushleft{\begin{hindi}
मूसा ने उन लोगों से कहा, "तबाही है तुम्हारी; झूठ घड़कर अल्लाह पर न थोपो कि वह तुम्हें एक यातना से विनष्ट कर दे और झूठ जिस किसी ने भी घड़कर थोपा, वह असफल रहा।"
\end{hindi}}
\flushright{\begin{Arabic}
\quranayah[20][62]
\end{Arabic}}
\flushleft{\begin{hindi}
इसपर उन्होंने परस्पर बड़ा मतभेद किया औऱ और चुपके-चुपके कानाफूसी की
\end{hindi}}
\flushright{\begin{Arabic}
\quranayah[20][63]
\end{Arabic}}
\flushleft{\begin{hindi}
कहने लगे, "ये दोनों जादूगर है, चाहते है कि अपने जादू से तुम्हें तुम्हारे भूभाग से निकाल बाहर करें। और तुम्हारी उत्तम और उच्च प्रणाली को तहस-नहस करके रख दे।"
\end{hindi}}
\flushright{\begin{Arabic}
\quranayah[20][64]
\end{Arabic}}
\flushleft{\begin{hindi}
अतः तुम सब मिलकर अपना उपाय जुटा लो, फिर पंक्तिबद्ध होकर आओ। आज तो प्रभावी रहा, वही सफल है।"
\end{hindi}}
\flushright{\begin{Arabic}
\quranayah[20][65]
\end{Arabic}}
\flushleft{\begin{hindi}
वे बोले, "ऐ मूसा! या तो तुम फेंको या फिर हम पहले फेंकते हैं।"
\end{hindi}}
\flushright{\begin{Arabic}
\quranayah[20][66]
\end{Arabic}}
\flushleft{\begin{hindi}
कहा, "नहीं, बल्कि तुम्हीं फेंको।" फिर अचानक क्या देखते है कि उनकी रस्सियाँ और लाठियाँ उनके जादू से उनके ख़याल में दौड़ती हुई प्रतीत हुई
\end{hindi}}
\flushright{\begin{Arabic}
\quranayah[20][67]
\end{Arabic}}
\flushleft{\begin{hindi}
और मूसा अपने जी में डरा
\end{hindi}}
\flushright{\begin{Arabic}
\quranayah[20][68]
\end{Arabic}}
\flushleft{\begin{hindi}
हमने कहा, "मत डर! निस्संदेह तू ही प्रभावी रहेगा।
\end{hindi}}
\flushright{\begin{Arabic}
\quranayah[20][69]
\end{Arabic}}
\flushleft{\begin{hindi}
और डाल दे जो तेरे दाहिने हाथ में है। जो कुछ उन्होंने रचा है, वह उसे निगल जाएगा। जो कुछ उन्होंने रचा है, वह तो बस जादूगर का स्वांग है और जादूगर सफल नहीं होता, चाहे वह जैसे भी आए।"
\end{hindi}}
\flushright{\begin{Arabic}
\quranayah[20][70]
\end{Arabic}}
\flushleft{\begin{hindi}
अन्ततः जादूगर सजदे में गिर पड़े, बोले, "हम हारून और मूसा के रब पर ईमान ले आए।"
\end{hindi}}
\flushright{\begin{Arabic}
\quranayah[20][71]
\end{Arabic}}
\flushleft{\begin{hindi}
उसने कहा, "तुमने मान लिया उसको, इससे पहले कि मैं तुम्हें इसकी अनुज्ञा देता? निश्चय ही यह तुम सबका प्रमुख है, जिसने जादू सिखाया है। अच्छा, अब मैं तुम्हारा हाथ और पाँव विपरीत दिशाओं से कटवा दूँगा और खंजूर के तनों पर तुम्हें सूली दे दूँगा। तब तुम्हें अवश्य ही मालूम हो जाएगा कि हममें से किसकी यातना अधिक कठोर और स्थायी है!"
\end{hindi}}
\flushright{\begin{Arabic}
\quranayah[20][72]
\end{Arabic}}
\flushleft{\begin{hindi}
उन्होंने कहा, "जो स्पष्ट निशानियाँ हमारे सामने आ चुकी है उनके मुक़ाबले में सौगंध है उस सत्ता की, जिसने हमें पैदा किया है, हम कदापि तुझे प्राथमिकता नहीं दे सकते। तो जो कुछ तू फ़ैसला करनेवाला है, कर ले। तू बस इसी सांसारिक जीवन का फ़ैसला कर सकता है
\end{hindi}}
\flushright{\begin{Arabic}
\quranayah[20][73]
\end{Arabic}}
\flushleft{\begin{hindi}
हम तो अपने रब पर ईमान ले आए, ताकि वह हमारी खताओं को माफ़ कर दे औऱ इस जादू को भी जिसपर तूने हमें बाध्य किया। अल्लाह की उत्तम और शेष रहनेवाला है।" -
\end{hindi}}
\flushright{\begin{Arabic}
\quranayah[20][74]
\end{Arabic}}
\flushleft{\begin{hindi}
सत्य यह है कि जो कोई अपने रब के पास अपराधी बनकर आया उसके लिए जहन्नम है, जिसमें वह न मरेगा और न जिएगा
\end{hindi}}
\flushright{\begin{Arabic}
\quranayah[20][75]
\end{Arabic}}
\flushleft{\begin{hindi}
और जो कोई उसके पास मोमिन होकर आया, जिसने अच्छे कर्म किए होंगे, तो ऐसे लोगों के लिए तो ऊँचे दर्जें है
\end{hindi}}
\flushright{\begin{Arabic}
\quranayah[20][76]
\end{Arabic}}
\flushleft{\begin{hindi}
अदन के बाग़ है, जिनके नीचें नहरें बहती होंगी। उनमें वे सदैव रहेंगे। यह बदला है उसका जिसने स्वयं को विकसित किया--
\end{hindi}}
\flushright{\begin{Arabic}
\quranayah[20][77]
\end{Arabic}}
\flushleft{\begin{hindi}
और हमने मूसा की ओर प्रकाशना की, "रातों रात मेरे बन्दों को लेकर निकल पड़, और उनके लिए दरिया में सूखा मार्ग निकाल ले। न तो तुझे पीछा किए जाने औऱ न पकड़े जाने का भय हो और न किसी अन्य चीज़ से तुझे डर लगे।"
\end{hindi}}
\flushright{\begin{Arabic}
\quranayah[20][78]
\end{Arabic}}
\flushleft{\begin{hindi}
फ़िरऔन ने अपनी सेना के साथ उनका पीछा किया। अन्ततः पानी उनपर छा गया, जैसाकि उसे उनपर छा जाना था
\end{hindi}}
\flushright{\begin{Arabic}
\quranayah[20][79]
\end{Arabic}}
\flushleft{\begin{hindi}
फ़िरऔन ने अपनी क़ौम को पथभ्रष्ट किया और मार्ग न दिखाया
\end{hindi}}
\flushright{\begin{Arabic}
\quranayah[20][80]
\end{Arabic}}
\flushleft{\begin{hindi}
ऐ ईसराईल की सन्तान! हमने तुम्हें तुम्हारे शत्रु से छुटकारा दिया और तुमसे तूर के दाहिने छोर का वादा किया और तुमपर मग्न और सलवा उतारा,
\end{hindi}}
\flushright{\begin{Arabic}
\quranayah[20][81]
\end{Arabic}}
\flushleft{\begin{hindi}
"खाओ, जो कुछ पाक अच्छी चीज़े हमने तुम्हें प्रदान की है, किन्तु इसमें हद से आगे न बढ़ो कि तुमपर मेरा प्रकोप टूट पड़े और जिस किसी पर मेरा प्रकोप टूटा, वह तो गिरकर ही रहा
\end{hindi}}
\flushright{\begin{Arabic}
\quranayah[20][82]
\end{Arabic}}
\flushleft{\begin{hindi}
और जो तौबा कर ले और ईमान लाए और अच्छा कर्म करे, फिर सीधे मार्ग पर चलता रहे, उसके लिए निश्चय ही मैं अत्यन्त क्षमाशील हूँ।" -
\end{hindi}}
\flushright{\begin{Arabic}
\quranayah[20][83]
\end{Arabic}}
\flushleft{\begin{hindi}
"और अपनी क़ौम को छोड़कर तुझे शीघ्र आने पर किस चीज़ ने उभारा, ऐ मूसा?"
\end{hindi}}
\flushright{\begin{Arabic}
\quranayah[20][84]
\end{Arabic}}
\flushleft{\begin{hindi}
उसने कहा, "वे मेरे पीछे ही और मैं जल्दी बढ़कर आया तेरी ओर, ऐ रब! ताकि तू राज़ी हो जाए।"
\end{hindi}}
\flushright{\begin{Arabic}
\quranayah[20][85]
\end{Arabic}}
\flushleft{\begin{hindi}
कहा, "अच्छा, तो हमने तेरे पीछे तेरी क़ौम के लोगों को आज़माइश में डाल दिया है। और सामरी ने उन्हें पथभ्रष्ट कर डाला।"
\end{hindi}}
\flushright{\begin{Arabic}
\quranayah[20][86]
\end{Arabic}}
\flushleft{\begin{hindi}
तब मूसा अत्यन्त क्रोध और खेद में डूबा हुआ अपनी क़ौम के लोगों की ओर पलटा। कहा, "ऐ मेरी क़ौम के लोगों! क्या तुमसे तुम्हारे रब ने अच्छा वादा नहीं किया था? क्या तुमपर लम्बी मुद्दत गुज़र गई या तुमने यही चाहा कि तुमपर तुम्हारे रब का प्रकोप ही टूटे कि तुमने मेरे वादे के विरुद्ध आचरण किया?"
\end{hindi}}
\flushright{\begin{Arabic}
\quranayah[20][87]
\end{Arabic}}
\flushleft{\begin{hindi}
उन्होंने कहा, "हमने आपसे किए हुए वादे के विरुद्ध अपने अधिकार से कुछ नहीं किया, बल्कि लोगों के ज़ेवरों के बोझ हम उठाए हुए थे, फिर हमने उनको (आग में) फेंक दिया, सामरी ने इसी तरह प्रेरित किया था।"
\end{hindi}}
\flushright{\begin{Arabic}
\quranayah[20][88]
\end{Arabic}}
\flushleft{\begin{hindi}
और उसने उनके लिए एक बछड़ा ढालकर निकाला, एक धड़ जिसकी आवाज़ बैल की थी। फिर उन्होंने कहा, "यही तुम्हारा इष्ट-पूज्य है और मूसा का भी इष्ट -पूज्य है, किन्तु वह भूल गया है।"
\end{hindi}}
\flushright{\begin{Arabic}
\quranayah[20][89]
\end{Arabic}}
\flushleft{\begin{hindi}
क्या वे देखते न थे कि न वह किसी बात का उत्तर देता है और न उसे उनकी हानि का कुछ अधिकार प्राप्त है और न लाभ का?
\end{hindi}}
\flushright{\begin{Arabic}
\quranayah[20][90]
\end{Arabic}}
\flushleft{\begin{hindi}
और हारून इससे पहले उनसे कह भी चुका था कि "मेरी क़ौम के लोगों! तुम इसके कारण बस फ़ितने में पड़ गए हो। तुम्हारा रब तो रहमान है। अतः तुम मेरा अनुसरण करो और मेरी बात मानो।"
\end{hindi}}
\flushright{\begin{Arabic}
\quranayah[20][91]
\end{Arabic}}
\flushleft{\begin{hindi}
उन्होंने कहा, "जब तक मूसा लौटकर हमारे पास न आ जाए, हम तो इससे ही लगे बैठे रहेंगे।"
\end{hindi}}
\flushright{\begin{Arabic}
\quranayah[20][92]
\end{Arabic}}
\flushleft{\begin{hindi}
उसने कहा, "ऐ हारून! जब तुमने देखा कि ये पथभ्रष्ट हो गए है, तो किस चीज़ ने तुम्हें रोका
\end{hindi}}
\flushright{\begin{Arabic}
\quranayah[20][93]
\end{Arabic}}
\flushleft{\begin{hindi}
कि तुमने मेरा अनुसरण न किया? क्या तुमने मेरे आदेश की अवहेलना की?"
\end{hindi}}
\flushright{\begin{Arabic}
\quranayah[20][94]
\end{Arabic}}
\flushleft{\begin{hindi}
उसने कहा, "ऐ मेरी माँ के बेटे! मेरी दाढ़ी न पकड़ और न मेरा सिर! मैं डरा कि तू कहेंगा कि तूने इसराईल की सन्तान में फूट डाल दी और मेरी बात पर ध्यान न दिया।"
\end{hindi}}
\flushright{\begin{Arabic}
\quranayah[20][95]
\end{Arabic}}
\flushleft{\begin{hindi}
(मूसा ने) कहा, "ऐ सामरी! तेरा क्या मामला है?"
\end{hindi}}
\flushright{\begin{Arabic}
\quranayah[20][96]
\end{Arabic}}
\flushleft{\begin{hindi}
उसने कहा, "मुझे उसकी सूझ प्राप्त हुई, जिसकी सूझ उन्हें प्राप्त॥ न हुई। फिर मैंने रसूल के पद-चिन्ह से एक मुट्ठी उठा ली। फिर उसे डाल दिया और मेरे जी ने मुझे कुछ ऐसा ही सुझाया।"
\end{hindi}}
\flushright{\begin{Arabic}
\quranayah[20][97]
\end{Arabic}}
\flushleft{\begin{hindi}
कहा, "अच्छा, तू जा! अब इस जीवन में तेरे लिए यही है कि कहता रहे, कोई छुए नहीं! और निश्चित वादा है, जो तेरे लिए एक निश्चित वादा है, जो तुझपर से कदापि न टलेगा। और देख अपने इष्ट-पूज्य को जिसपर तू रीझा-जमा बैठा था! हम उसे जला डालेंगे, फिर उसे चूर्ण-विचूर्ण करके दरिया में बिखेर देंगे।"
\end{hindi}}
\flushright{\begin{Arabic}
\quranayah[20][98]
\end{Arabic}}
\flushleft{\begin{hindi}
"तुम्हारा पूज्य-प्रभु तो बस वही अल्लाह है, जिसके अतिरिक्त कोई पूज्य-प्रभु नहीं। वह अपने ज्ञान से हर चीज़ पर हावी है।"
\end{hindi}}
\flushright{\begin{Arabic}
\quranayah[20][99]
\end{Arabic}}
\flushleft{\begin{hindi}
इस प्रकार विगत वृत्तांत हम तुम्हें सुनाते है और हमने तुम्हें अपने पास से एक अनुस्मृति प्रदान की है
\end{hindi}}
\flushright{\begin{Arabic}
\quranayah[20][100]
\end{Arabic}}
\flushleft{\begin{hindi}
जिस किसी ने उससे मुँह मोड़ा, वह निश्चय ही क़ियामत के दिन एक बोझ उठाएगा
\end{hindi}}
\flushright{\begin{Arabic}
\quranayah[20][101]
\end{Arabic}}
\flushleft{\begin{hindi}
ऐसे दिन सदैव इसी वबाल में पड़े रहेंगे और क़ियामत के दिन उनके हक़ में यह बहुत ही बुरा बोझ सिद्ध होगा
\end{hindi}}
\flushright{\begin{Arabic}
\quranayah[20][102]
\end{Arabic}}
\flushleft{\begin{hindi}
जिस दिन सूर फूँका जाएगा और हम अपराधियों को उस दिन इस दशा में इकट्ठा करेंगे कि उनकी आँखे नीली पड़ गई होंगी
\end{hindi}}
\flushright{\begin{Arabic}
\quranayah[20][103]
\end{Arabic}}
\flushleft{\begin{hindi}
वे आपस में चुपके-चुपके कहेंगे कि "तुम बस दस ही दिन ठहरे हो।"
\end{hindi}}
\flushright{\begin{Arabic}
\quranayah[20][104]
\end{Arabic}}
\flushleft{\begin{hindi}
हम भली-भाँति जानते है जो कुछ वे कहेंगे, जबकि उनका सबसे अच्छी सम्मतिवाला कहेगा, "तुम तो बस एक ही दिन ठहरे हो।"
\end{hindi}}
\flushright{\begin{Arabic}
\quranayah[20][105]
\end{Arabic}}
\flushleft{\begin{hindi}
वे तुमसे पर्वतों के विषय में पूछते है। कह दो, "मेरा रब उन्हें छूल की तरह उड़ा देगा,
\end{hindi}}
\flushright{\begin{Arabic}
\quranayah[20][106]
\end{Arabic}}
\flushleft{\begin{hindi}
और धरती को एक समतल चटियल मैदान बनाकर छोड़ेगा
\end{hindi}}
\flushright{\begin{Arabic}
\quranayah[20][107]
\end{Arabic}}
\flushleft{\begin{hindi}
तुम उसमें न कोई सिलवट देखोगे और न ऊँच-नीच।"
\end{hindi}}
\flushright{\begin{Arabic}
\quranayah[20][108]
\end{Arabic}}
\flushleft{\begin{hindi}
उस दिन वे पुकारनेवाले के पीछे चल पड़ेंगे और उसके सामने कोई अकड़ न दिखाई जा सकेगी। आवाज़े रहमान के सामने दब जाएँगी। एक हल्की मन्द आवाज़ के अतिरिक्त तुम कुछ न सुनोगे
\end{hindi}}
\flushright{\begin{Arabic}
\quranayah[20][109]
\end{Arabic}}
\flushleft{\begin{hindi}
उस दिन सिफ़ारिश काम न आएगी। यह और बात है कि किसी के लिए रहमान अनुज्ञा दे और उसके लिए बात करने को पसन्द करे
\end{hindi}}
\flushright{\begin{Arabic}
\quranayah[20][110]
\end{Arabic}}
\flushleft{\begin{hindi}
वह जानता है जो कुछ उनके आगे है और जो कुछ उनके पीछे है, किन्तु वे अपने ज्ञान से उसपर हावी नहीं हो सकते
\end{hindi}}
\flushright{\begin{Arabic}
\quranayah[20][111]
\end{Arabic}}
\flushleft{\begin{hindi}
चेहरे उस जीवन्त, शाश्वत सत्ता के आगे झुकें होंगे। असफल हुआ वह जिसने ज़ुल्म का बोझ उठाया
\end{hindi}}
\flushright{\begin{Arabic}
\quranayah[20][112]
\end{Arabic}}
\flushleft{\begin{hindi}
किन्तु जो कोई अच्छे कर्म करे और हो वह मोमिन, तो उसे न तो किसी ज़ुल्म का भय होगा और न हक़ मारे जाने का
\end{hindi}}
\flushright{\begin{Arabic}
\quranayah[20][113]
\end{Arabic}}
\flushleft{\begin{hindi}
और इस प्रकार हमने इसे अरबी क़ुरआन के रूप में अवतरित किया है और हमने इसमें तरह-तरह से चेतावनी दी है, ताकि वे डर रखें या यह उन्हें होश दिलाए
\end{hindi}}
\flushright{\begin{Arabic}
\quranayah[20][114]
\end{Arabic}}
\flushleft{\begin{hindi}
अतः सर्वोच्च है अल्लाह, सच्चा सम्राट! क़ुरआन के (फ़ैसले के) सिलसिले में जल्दी न करो, जब तक कि वह पूरा न हो जाए। तेरी ओर उसकी प्रकाशना हो रही है। और कहो, "मेरे रब, मुझे ज्ञान में अभिवृद्धि प्रदान कर।"
\end{hindi}}
\flushright{\begin{Arabic}
\quranayah[20][115]
\end{Arabic}}
\flushleft{\begin{hindi}
और हमने इससे पहले आदम से वचन लिया था, किन्तु वह भूल गया और हमने उसमें इरादे की मज़बूती न पाई
\end{hindi}}
\flushright{\begin{Arabic}
\quranayah[20][116]
\end{Arabic}}
\flushleft{\begin{hindi}
और जब हमने फ़रिश्तों से कहा, "आदम को सजदा करो।" तो उन्होंने सजदा किया सिवाय इबलीस के, वह इनकार कर बैठा
\end{hindi}}
\flushright{\begin{Arabic}
\quranayah[20][117]
\end{Arabic}}
\flushleft{\begin{hindi}
इसपर हमने कहा, "ऐ आदम! निश्चय ही यह तुम्हारा और तुम्हारी पत्नी का शत्रु है। ऐसा न हो कि तुम दोनों को जन्नत से निकलवा दे और तुम तकलीफ़ में पड़ जाओ
\end{hindi}}
\flushright{\begin{Arabic}
\quranayah[20][118]
\end{Arabic}}
\flushleft{\begin{hindi}
तुम्हारे लिए तो ऐसा है कि न तुम यहाँ भूखे रहोगे और न नंगे
\end{hindi}}
\flushright{\begin{Arabic}
\quranayah[20][119]
\end{Arabic}}
\flushleft{\begin{hindi}
और यह कि न यहाँ प्यासे रहोगे और न धूप की तकलीफ़ उठाओगे।"
\end{hindi}}
\flushright{\begin{Arabic}
\quranayah[20][120]
\end{Arabic}}
\flushleft{\begin{hindi}
फिर शैतान ने उसे उकसाया। कहने लगा, "ऐ आदम! क्या मैं तुझे शाश्वत जीवन के वृक्ष का पता दूँ और ऐसे राज्य का जो कभी जीर्ण न हो?"
\end{hindi}}
\flushright{\begin{Arabic}
\quranayah[20][121]
\end{Arabic}}
\flushleft{\begin{hindi}
अन्ततः उन दोनों ने उसमें से खा लिया, जिसके परिणामस्वरूप उनकी छिपाने की चीज़े उनके आगे खुल गई और वे दोनों अपने ऊपर जन्नत के पत्ते जोड-जोड़कर रखने लगे। और आदम ने अपने रब की अवज्ञा की, तो वह मार्ग से भटक गया
\end{hindi}}
\flushright{\begin{Arabic}
\quranayah[20][122]
\end{Arabic}}
\flushleft{\begin{hindi}
इसके पश्चात उसके रब ने उसे चुन लिया और दोबारा उसकी ओर ध्यान दिया और उसका मार्गदर्शन किया
\end{hindi}}
\flushright{\begin{Arabic}
\quranayah[20][123]
\end{Arabic}}
\flushleft{\begin{hindi}
कहा, "तुम दोनों के दोनों यहाँ से उतरो! तुम्हारे कुछ लोग कुछ के शत्रु होंगे। फिर यदि मेरी ओर से तुम्हें मार्गदर्शन पहुँचे, तो जिस किसी ने मेरे मार्गदर्शन का अनुपालन किया, वह न तो पथभ्रष्ट होगा और न तकलीफ़ में पड़ेगा
\end{hindi}}
\flushright{\begin{Arabic}
\quranayah[20][124]
\end{Arabic}}
\flushleft{\begin{hindi}
और जिस किसी ने मेरी स्मृति से मुँह मोडा़ तो उसका जीवन संकीर्ण होगा और क़ियामत के दिन हम उसे अंधा उठाएँगे।"
\end{hindi}}
\flushright{\begin{Arabic}
\quranayah[20][125]
\end{Arabic}}
\flushleft{\begin{hindi}
वह कहेगा, "ऐ मेरे रब! तूने मुझे अंधा क्यों उठाया, जबकि मैं आँखोंवाला था?"
\end{hindi}}
\flushright{\begin{Arabic}
\quranayah[20][126]
\end{Arabic}}
\flushleft{\begin{hindi}
वह कहेगा, "इसी प्रकार (तू संसार में अंधा रहा था) । तेरे पास मेरी आयतें आई थी, तो तूने उन्हें भूला दिया था। उसी प्रकार आज तुझे भुलाया जा रहा है।"
\end{hindi}}
\flushright{\begin{Arabic}
\quranayah[20][127]
\end{Arabic}}
\flushleft{\begin{hindi}
इसी प्रकार हम उसे बदला देते है जो मर्यादा का उल्लंघन करे और अपने रब की आयतों पर ईमान न लाए। और आख़िरत की यातना तो अत्यन्त कठोर और अधिक स्थायी है
\end{hindi}}
\flushright{\begin{Arabic}
\quranayah[20][128]
\end{Arabic}}
\flushleft{\begin{hindi}
फिर क्या उनको इससे भी मार्ग न मिला कि हम उनसे पहले कितनी ही नस्लों को विनष्ट कर चुके है, जिनकी बस्तियों में वे चलते-फिरते है? निस्संदेह बुद्धिमानों के लिए इसमें बहुत-सी निशानियाँ है
\end{hindi}}
\flushright{\begin{Arabic}
\quranayah[20][129]
\end{Arabic}}
\flushleft{\begin{hindi}
यदि तेरे रब की ओर से पहले ही एक बात निश्चित न हो गई होती और एक अवधि नियत न की जा चुकी होती, तो अवश्य ही उन्हें यातना आ पकड़ती
\end{hindi}}
\flushright{\begin{Arabic}
\quranayah[20][130]
\end{Arabic}}
\flushleft{\begin{hindi}
अतः जो कुछ वे कहते है उसपर धैर्य से काम लो और अपने रब का गुणगान करो, सूर्योदय से पहले और उसके डूबने से पहले, और रात की घड़ियों में भी तसबीह करो, और दिन के किनारों पर भी, ताकि तुम राज़ी हो जाओ
\end{hindi}}
\flushright{\begin{Arabic}
\quranayah[20][131]
\end{Arabic}}
\flushleft{\begin{hindi}
और उसकी ओर आँख उठाकर न देखो, जो कुछ हमने उनमें से विभिन्न लोगों को उपभोग के लिए दे रखा है, ताकि हम उसके द्वारा उन्हें आज़माएँ। वह तो बस सांसारिक जीवन की शोभा है। तुम्हारे रब की रोज़ी उत्तम भी है और स्थायी भी
\end{hindi}}
\flushright{\begin{Arabic}
\quranayah[20][132]
\end{Arabic}}
\flushleft{\begin{hindi}
और अपने लोगों को नमाज़ का आदेश करो और स्वयं भी उसपर जमे रहो। हम तुमसे कोई रोज़ी नहीं माँगते। रोज़ी हम ही तुम्हें देते है, और अच्छा परिणाम तो धर्मपरायणता ही के लिए निश्चित है
\end{hindi}}
\flushright{\begin{Arabic}
\quranayah[20][133]
\end{Arabic}}
\flushleft{\begin{hindi}
और वे कहते है कि "यह अपने रब की ओर से हमारे पास कोई निशानी क्यों नहीं लाता?" क्या उनके पास उसका स्पष्ट प्रमाण नहीं आ गया, जो कुछ कि पहले की पुस्तकों में उल्लिखित है?
\end{hindi}}
\flushright{\begin{Arabic}
\quranayah[20][134]
\end{Arabic}}
\flushleft{\begin{hindi}
यदि हम उसके पहले इन्हें किसी यातना से विनष्ट कर देते तो ये कहते कि "ऐ हमारे रब, तूने हमारे पास कोई रसूल क्यों न भेजा कि इससे पहले कि हम अपमानित और रुसवा होते, तेरी आयतों का अनुपालन करने लगते?"
\end{hindi}}
\flushright{\begin{Arabic}
\quranayah[20][135]
\end{Arabic}}
\flushleft{\begin{hindi}
कह दो, "हर एक प्रतीक्षा में है। अतः अब तुम भी प्रतीक्षा करो। शीघ्र ही तुम जान लोगे कि कौन सीधे मार्गवाला है और किनको मार्गदर्शन प्राप्त है।"
\end{hindi}}
\chapter{Al-Anbiya' (The Prophets)}
\begin{Arabic}
\Huge{\centerline{\basmalah}}\end{Arabic}
\flushright{\begin{Arabic}
\quranayah[21][1]
\end{Arabic}}
\flushleft{\begin{hindi}
निकट आ गया लोगों का हिसाब और वे है कि असावधान कतराते जा रहे है
\end{hindi}}
\flushright{\begin{Arabic}
\quranayah[21][2]
\end{Arabic}}
\flushleft{\begin{hindi}
उनके पास जो ताज़ा अनुस्मृति भी उनके रब की ओर से आती है, उसे वे हँसी-खेल करते हुए ही सुनते है
\end{hindi}}
\flushright{\begin{Arabic}
\quranayah[21][3]
\end{Arabic}}
\flushleft{\begin{hindi}
उनके दिल दिलचस्पियों में खोए हुए होते है। उन्होंने चुपके-चुपके कानाफूसी की - अर्थात अत्याचार की नीति अपनानेवालों ने कि "यह तो बस तुम जैसा ही एक मनुष्य है। फिर क्या तुम देखते-बूझते जादू में फँस जाओगे?"
\end{hindi}}
\flushright{\begin{Arabic}
\quranayah[21][4]
\end{Arabic}}
\flushleft{\begin{hindi}
उसने कहा, "मेरा रब जानता है उस बात को जो आकाश और धरती में हो। और वह भली-भाँति सब कुछ सुनने, जाननेवाला है।"
\end{hindi}}
\flushright{\begin{Arabic}
\quranayah[21][5]
\end{Arabic}}
\flushleft{\begin{hindi}
नहीं, बल्कि वे कहते है, "ये तो संभ्रमित स्वप्नं है, बल्कि उसने इसे स्वयं ही घड़ लिया है, बल्कि वह एक कवि है! उसे तो हमारे पास कोई निशानी लानी चाहिए, जैसे कि (निशानियाँ लेकर) पहले के रसूल भेजे गए थे।"
\end{hindi}}
\flushright{\begin{Arabic}
\quranayah[21][6]
\end{Arabic}}
\flushleft{\begin{hindi}
इनसे पहले कोई बस्ती भी, जिसको हमने विनष्ट किया, ईमान न लाई। फिर क्या ये ईमान लाएँगे?
\end{hindi}}
\flushright{\begin{Arabic}
\quranayah[21][7]
\end{Arabic}}
\flushleft{\begin{hindi}
और तुमसे पहले भी हमने पुरुषों ही को रसूल बनाकर भेजा, जिनकी ओर हम प्रकाशना करते थे। - यदि तुम्हें मालूम न हो तो ज़िक्रवालों (किताबवालों) से पूछ लो। -
\end{hindi}}
\flushright{\begin{Arabic}
\quranayah[21][8]
\end{Arabic}}
\flushleft{\begin{hindi}
उनको हमने कोई ऐसा शरीर नहीं दिया था कि वे भोजन न करते हों और न वे सदैव रहनेवाले ही थे
\end{hindi}}
\flushright{\begin{Arabic}
\quranayah[21][9]
\end{Arabic}}
\flushleft{\begin{hindi}
फिर हमने उनके साथ वादे को सच्चा कर दिखाया और उन्हें हमने छुटकारा दिया, और जिसे हम चाहें उसे छुटकारा मिलता है। और मर्यादाहीनों को हमने विनष्ट कर दिया
\end{hindi}}
\flushright{\begin{Arabic}
\quranayah[21][10]
\end{Arabic}}
\flushleft{\begin{hindi}
लो, हमने तुम्हारी ओर एक किताब अवतरित कर दी है, जिसमें तुम्हारे लिए याददिहानी है। तो क्या तुम बुद्धि से काम नहीं लेते?
\end{hindi}}
\flushright{\begin{Arabic}
\quranayah[21][11]
\end{Arabic}}
\flushleft{\begin{hindi}
कितनी ही बस्तियों को, जो ज़ालिम थीं, हमने तोड़कर रख दिया और उनके बाद हमने दूसरे लोगों को उठाया
\end{hindi}}
\flushright{\begin{Arabic}
\quranayah[21][12]
\end{Arabic}}
\flushleft{\begin{hindi}
फिर जब उन्हें हमारी यातना का आभास हुआ तो लगे वहाँ से भागने
\end{hindi}}
\flushright{\begin{Arabic}
\quranayah[21][13]
\end{Arabic}}
\flushleft{\begin{hindi}
कहा गया, "भागो नहीं! लौट चलो, उसी भोग-विलास की ओर जो तुम्हें प्राप्त था और अपने घरों की ओर ताकि तुमसे पूछा जाए।"
\end{hindi}}
\flushright{\begin{Arabic}
\quranayah[21][14]
\end{Arabic}}
\flushleft{\begin{hindi}
कहने लगे, "हाय हमारा दुर्भाग्य! निस्संदेह हम ज़ालिम थे।"
\end{hindi}}
\flushright{\begin{Arabic}
\quranayah[21][15]
\end{Arabic}}
\flushleft{\begin{hindi}
फिर उनकी निरन्तर यही पुकार रही, यहाँ तक कि हमने उन्हें ऐसा कर दिया जैसे कटी हुई खेती, बुझी हुई आग हो
\end{hindi}}
\flushright{\begin{Arabic}
\quranayah[21][16]
\end{Arabic}}
\flushleft{\begin{hindi}
और हमने आकाश और धरती को और जो कुछ उसके मध्य में है कुछ इस प्रकार नहीं बनाया कि हम कोई खेल करने वाले हो
\end{hindi}}
\flushright{\begin{Arabic}
\quranayah[21][17]
\end{Arabic}}
\flushleft{\begin{hindi}
यदि हम कोई खेल-तमाशा करना चाहते हो अपने ही पास से कर लेते, यदि हम ऐसा करने ही वाले होते
\end{hindi}}
\flushright{\begin{Arabic}
\quranayah[21][18]
\end{Arabic}}
\flushleft{\begin{hindi}
नहीं, बल्कि हम तो असत्य पर सत्य की चोट लगाते है, तो वह उसका सिर तोड़ देता है। फिर क्या देखते है कि वह मिटकर रह जाता है और तुम्हारे लिए तबाही है उन बातों के कारण जो तुम बनाते हो!
\end{hindi}}
\flushright{\begin{Arabic}
\quranayah[21][19]
\end{Arabic}}
\flushleft{\begin{hindi}
और आकाशों और धरती में जो कोई है उसी का है। और जो (फ़रिश्ते) उसके पास है वे न तो अपने को बड़ा समझकर उसकी बन्दगी से मुँह मोड़ते है औऱ न वे थकते है
\end{hindi}}
\flushright{\begin{Arabic}
\quranayah[21][20]
\end{Arabic}}
\flushleft{\begin{hindi}
रात और दिन तसबीह करते रहते है, दम नहीं लेते
\end{hindi}}
\flushright{\begin{Arabic}
\quranayah[21][21]
\end{Arabic}}
\flushleft{\begin{hindi}
(क्या उन्होंने आकाश से कुछ पूज्य बना लिए है)... या उन्होंने धरती से ऐसे इष्ट -पूज्य बना लिए है, जो पुनर्जीवित करते हों?
\end{hindi}}
\flushright{\begin{Arabic}
\quranayah[21][22]
\end{Arabic}}
\flushleft{\begin{hindi}
यदि इन दोनों (आकाश और धरती) में अल्लाह के सिवा दूसरे इष्ट-पूज्य भी होते तो दोनों की व्यवस्था बिगड़ जाती। अतः महान और उच्च है अल्लाह, राजासन का स्वामी, उन बातों से जो ये बयान करते है
\end{hindi}}
\flushright{\begin{Arabic}
\quranayah[21][23]
\end{Arabic}}
\flushleft{\begin{hindi}
जो कुछ वह करता है उससे उसकी कोई पूछ नहीं हो सकती, किन्तु इनसे पूछ होगी
\end{hindi}}
\flushright{\begin{Arabic}
\quranayah[21][24]
\end{Arabic}}
\flushleft{\begin{hindi}
(क्या ये अल्लाह के हक़ को नहीं पहचानते) या उसे छोड़कर इन्होंने दूसरे इष्ट-पूज्य बना लिए है (जिसके लिए इनके पास कुछ प्रमाण है)? कह दो, "लाओ, अपना प्रमाण! यह अनुस्मृति है उनकी जो मेरे साथ है और अनुस्मृति है उनकी जो मुझसे पहले हुए है, किन्तु बात यह है कि इनमें अधिकतर सत्य को जानते नहीं, इसलिए कतरा रहे है
\end{hindi}}
\flushright{\begin{Arabic}
\quranayah[21][25]
\end{Arabic}}
\flushleft{\begin{hindi}
हमने तुमसे पहले जो रसूल भी भेजा, उसकी ओर यही प्रकाशना की कि " "मेरे सिवा कोई पूज्य-प्रभु नहीं। अतः तुम मेरी ही बन्दगी करो।"
\end{hindi}}
\flushright{\begin{Arabic}
\quranayah[21][26]
\end{Arabic}}
\flushleft{\begin{hindi}
और वे कहते है कि "रहमान सन्तान रखता है।" महान हो वह! बल्कि वे तो प्रतिष्ठित बन्दे हैं
\end{hindi}}
\flushright{\begin{Arabic}
\quranayah[21][27]
\end{Arabic}}
\flushleft{\begin{hindi}
उससे आगे बढ़कर नहीं बोलते और उनके आदेश का पालन करते है
\end{hindi}}
\flushright{\begin{Arabic}
\quranayah[21][28]
\end{Arabic}}
\flushleft{\begin{hindi}
वह जानता है जो कुछ उनके आगे है और जो कुछ उनके पीछे है, और वे किसी की सिफ़ारिश नहीं करते सिवाय उसके जिसके लिए अल्लाह पसन्द करे। और वे उसके भय से डरते रहते है
\end{hindi}}
\flushright{\begin{Arabic}
\quranayah[21][29]
\end{Arabic}}
\flushleft{\begin{hindi}
और जो उनमें से यह कहे कि "उनके सिवा मैं भी एक इष्ट -पूज्य हूँ।" तो हम उसे बदले में जहन्नम देंगे। ज़ालिमों को हम ऐसा ही बदला दिया करते है
\end{hindi}}
\flushright{\begin{Arabic}
\quranayah[21][30]
\end{Arabic}}
\flushleft{\begin{hindi}
क्या उन लोगों ने जिन्होंने इनकार किया, देखा नहीं कि ये आकाश और धरती बन्द थे। फिर हमने उन्हें खोल दिया। और हमने पानी से हर जीवित चीज़ बनाई, तो क्या वे मानते नहीं?
\end{hindi}}
\flushright{\begin{Arabic}
\quranayah[21][31]
\end{Arabic}}
\flushleft{\begin{hindi}
और हमने धरती में अटल पहाड़ रख दिए, ताकि कहीं ऐसा न हो कि वह उन्हें लेकर ढुलक जाए और हमने उसमें ऐसे दर्रे बनाए कि रास्तों का काम देते है, ताकि वे मार्ग पाएँ
\end{hindi}}
\flushright{\begin{Arabic}
\quranayah[21][32]
\end{Arabic}}
\flushleft{\begin{hindi}
और हमने आकाश को एक सुरक्षित छत बनाया, किन्तु वे है कि उसकी निशानियों से कतरा जाते है
\end{hindi}}
\flushright{\begin{Arabic}
\quranayah[21][33]
\end{Arabic}}
\flushleft{\begin{hindi}
वही है जिसने रात और दिन बनाए और सूर्य और चन्द्र भी। प्रत्येक अपने-अपने कक्ष में तैर रहा है
\end{hindi}}
\flushright{\begin{Arabic}
\quranayah[21][34]
\end{Arabic}}
\flushleft{\begin{hindi}
हमने तुमसे पहले भी किसी आदमी के लिए अमरता नहीं रखी। फिर क्या यदि तुम मर गए तो वे सदैव रहनेवाले है?
\end{hindi}}
\flushright{\begin{Arabic}
\quranayah[21][35]
\end{Arabic}}
\flushleft{\begin{hindi}
हर जीव को मौत का मज़ा चखना है और हम अच्छी और बुरी परिस्थितियों में डालकर तुम सबकी परीक्षा करते है। अन्ततः तुम्हें हमारी ही ओर पलटकर आना है
\end{hindi}}
\flushright{\begin{Arabic}
\quranayah[21][36]
\end{Arabic}}
\flushleft{\begin{hindi}
जिन लोगों ने इनकार किया वे जब तुम्हें देखते है तो तुम्हारा उपहास ही करते है। (कहते है,) "क्या यही वह व्यक्ति है, जो तुम्हारे इष्ट -पूज्यों की बुराई के साथ चर्चा करता है?" और उनका अपना हाल यह है कि वे रहमान के ज़िक्र (स्मरण) से इनकार करते हैं
\end{hindi}}
\flushright{\begin{Arabic}
\quranayah[21][37]
\end{Arabic}}
\flushleft{\begin{hindi}
मनुष्य उतावला पैदा किया गया है। मैं तुम्हें शीघ्र ही अपनी निशानियाँ दिखाए देता हूँ। अतः तुम मुझसे जल्दी मत मचाओ
\end{hindi}}
\flushright{\begin{Arabic}
\quranayah[21][38]
\end{Arabic}}
\flushleft{\begin{hindi}
वे कहते है कि "यह वादा कब पूरा होगा, यदि तुम सच्चे हो?"
\end{hindi}}
\flushright{\begin{Arabic}
\quranayah[21][39]
\end{Arabic}}
\flushleft{\begin{hindi}
अगर इनकार करनेवालें उस समय को जानते, जबकि वे न तो अपने चहरों की ओर आग को रोक सकेंगे और न अपनी पीठों की ओर से और न उन्हें कोई सहायता ही पहुँच सकेगी तो (यातना की जल्दी न मचाते)
\end{hindi}}
\flushright{\begin{Arabic}
\quranayah[21][40]
\end{Arabic}}
\flushleft{\begin{hindi}
बल्कि वह अचानक उनपर आएगी और उन्हें स्तब्ध कर देगी। फिर न उसे वे फेर सकेंगे और न उन्हें मुहलत ही मिलेगी
\end{hindi}}
\flushright{\begin{Arabic}
\quranayah[21][41]
\end{Arabic}}
\flushleft{\begin{hindi}
तुमसे पहले भी रसूलों की हँसी उड़ाई जा चुकी है, किन्तु उनमें से जिन लोगों ने उनकी हँसी उड़ाई थी उन्हें उसी चीज़ ने आ घेरा, जिसकी वे हँसी उड़ाते थे
\end{hindi}}
\flushright{\begin{Arabic}
\quranayah[21][42]
\end{Arabic}}
\flushleft{\begin{hindi}
कहो कि "कौन रहमान के मुक़ाबले में रात-दिन तुम्हारी रक्षा करेगा? बल्कि बात यह है कि वे अपने रब की याददिहानी से कतरा रहे है
\end{hindi}}
\flushright{\begin{Arabic}
\quranayah[21][43]
\end{Arabic}}
\flushleft{\begin{hindi}
(क्या वे हमें नहीं जानते) या हमसे हटकर उनके और भी इष्ट-पूज्य है, जो उन्हें बचा ले? वे तो स्वयं अपनी ही सहायता नहीं कर सकते है और न हमारे मुक़ाबले में उनका कोई साथ ही दे सकता है
\end{hindi}}
\flushright{\begin{Arabic}
\quranayah[21][44]
\end{Arabic}}
\flushleft{\begin{hindi}
बल्कि बात यह है कि हमने उन्हें और उनके बाप-दादा को सुख-सुविधा प्रदान की, यहाँ तक कि इसी दशा में एक लम्बी मुद्दत उनपर गुज़र गई, तो क्या वे देखते नहीं कि हम इस भूभाग को उसके चतुर्दिक से घटाते हुए बढ़ रहे है? फिर क्या वे अभिमानी रहेंगे?
\end{hindi}}
\flushright{\begin{Arabic}
\quranayah[21][45]
\end{Arabic}}
\flushleft{\begin{hindi}
कह दो, "मैं तो बस प्रकाशना के आधार पर तुम्हें सावधान करता हूँ।" किन्तु बहरे पुकार को नहीं सुनते, जबकि उन्हें सावधान किया जाए
\end{hindi}}
\flushright{\begin{Arabic}
\quranayah[21][46]
\end{Arabic}}
\flushleft{\begin{hindi}
और यदि तुम्हारे रब की यातना का कोई झोंका भी उन्हें छू जाए तो वे कहन लगे, "हाय, हमारा दुर्भाग्य! निस्संदेह हम ज़ालिम थे।"
\end{hindi}}
\flushright{\begin{Arabic}
\quranayah[21][47]
\end{Arabic}}
\flushleft{\begin{hindi}
और हम बज़नी, अच्छे न्यायपूर्ण कामों को क़ियामत के दिन के लिए रख रहे है। फिर किसी व्यक्ति पर कुछ भी ज़ुल्म न होगा, यद्यपि वह (कर्म) राई के दाने के बराबर हो, हम उसे ला उपस्थित करेंगे। और हिसाब करने के लिए हम काफ़ी है
\end{hindi}}
\flushright{\begin{Arabic}
\quranayah[21][48]
\end{Arabic}}
\flushleft{\begin{hindi}
और हम मूसा और हारून को कसौटी और रौशनी और याददिहानी प्रदान कर चुके हैं, उन डर रखनेवालों के लिए
\end{hindi}}
\flushright{\begin{Arabic}
\quranayah[21][49]
\end{Arabic}}
\flushleft{\begin{hindi}
जो परोक्ष में रहते हुए अपने रब से डरते है और उन्हें क़ियामत की घड़ी का भय लगा रहता है
\end{hindi}}
\flushright{\begin{Arabic}
\quranayah[21][50]
\end{Arabic}}
\flushleft{\begin{hindi}
और वह बरकतवाली अनुस्मृति है, जिसको हमने अवतरित किया है। तो क्या तुम्हें इससे इनकार है
\end{hindi}}
\flushright{\begin{Arabic}
\quranayah[21][51]
\end{Arabic}}
\flushleft{\begin{hindi}
और इससे पहले हमने इबराहीम को उसकी हिदायत और समझ दी थी - और हम उसे भली-भाँति जानते थे। -
\end{hindi}}
\flushright{\begin{Arabic}
\quranayah[21][52]
\end{Arabic}}
\flushleft{\begin{hindi}
जब उसने अपने बाप और अपनी क़ौम से कहा, "ये मूर्तियाँ क्या है, जिनसे तुम लगे बैठे हो?"
\end{hindi}}
\flushright{\begin{Arabic}
\quranayah[21][53]
\end{Arabic}}
\flushleft{\begin{hindi}
वे बोले, "हमने अपने बाप-दादा को इन्हीं की पूजा करते पाया है।"
\end{hindi}}
\flushright{\begin{Arabic}
\quranayah[21][54]
\end{Arabic}}
\flushleft{\begin{hindi}
उसने कहा, "तुम भी और तुम्हारे बाप-दादा भी खुली गुमराही में हो।"
\end{hindi}}
\flushright{\begin{Arabic}
\quranayah[21][55]
\end{Arabic}}
\flushleft{\begin{hindi}
उन्होंने कहा, "क्या तू हमारे पास सत्य लेकर आया है या यूँ ही खेल कर रहा है?"
\end{hindi}}
\flushright{\begin{Arabic}
\quranayah[21][56]
\end{Arabic}}
\flushleft{\begin{hindi}
उसने कहा, "नहीं, बल्कि बात यह है कि तुम्हारा रब आकाशों और धरती का रब है, जिसने उनको पैदा किया है और मैं इसपर तुम्हारे सामने गवाही देता हूँ
\end{hindi}}
\flushright{\begin{Arabic}
\quranayah[21][57]
\end{Arabic}}
\flushleft{\begin{hindi}
और अल्लाह की क़सम! इसके पश्चात कि तुम पीठ फेरकर लौटो, मैं तुम्हारी मूर्तियों के साथ अवश्य् एक चाल चलूँगा।"
\end{hindi}}
\flushright{\begin{Arabic}
\quranayah[21][58]
\end{Arabic}}
\flushleft{\begin{hindi}
अतएव उसने उन्हें खंड-खंड कर दिया सिवाय उनकी एक बड़ी के, कदाचित वे उसकी ओर रुजू करें
\end{hindi}}
\flushright{\begin{Arabic}
\quranayah[21][59]
\end{Arabic}}
\flushleft{\begin{hindi}
वे कहने लगे, "किसने हमारे देवताओं के साथ यह हरकत की है? निश्चय ही वह कोई ज़ालिम है।"
\end{hindi}}
\flushright{\begin{Arabic}
\quranayah[21][60]
\end{Arabic}}
\flushleft{\begin{hindi}
(कुछ लोग) बोले, "हमने एक नवयुवक को, जिसे इबराहीम कहते है, उसके विषय में कुछ कहते सुना है।"
\end{hindi}}
\flushright{\begin{Arabic}
\quranayah[21][61]
\end{Arabic}}
\flushleft{\begin{hindi}
उन्होंने कहा, "तो उसे ले आओ लोगों की आँखों के सामने कि वे भी गवाह रहें।"
\end{hindi}}
\flushright{\begin{Arabic}
\quranayah[21][62]
\end{Arabic}}
\flushleft{\begin{hindi}
उन्होंने कहा, "क्या तूने हमारे देवों के साथ यह हरकत की है, ऐ इबराहीम!"
\end{hindi}}
\flushright{\begin{Arabic}
\quranayah[21][63]
\end{Arabic}}
\flushleft{\begin{hindi}
उसने कहा, "नहीं, बल्कि उनके इस बड़े ने की होगी, उन्हीं से पूछ लो, यदि वे बोलते हों।"
\end{hindi}}
\flushright{\begin{Arabic}
\quranayah[21][64]
\end{Arabic}}
\flushleft{\begin{hindi}
तब वे उसकी ओर पलटे और कहने लगे, "वास्तव में, ज़ालिम तो तुम्हीं लोग हो।"
\end{hindi}}
\flushright{\begin{Arabic}
\quranayah[21][65]
\end{Arabic}}
\flushleft{\begin{hindi}
किन्तु फिर वे बिल्कुल औंधे हो रहे। (फिर बोले,) "तुझे तो मालूम है कि ये बोलते नहीं।"
\end{hindi}}
\flushright{\begin{Arabic}
\quranayah[21][66]
\end{Arabic}}
\flushleft{\begin{hindi}
उसने कहा, "फिर क्या तुम अल्लाह से इतर उसे पूजते हो, जो न तुम्हें कुछ लाभ पहुँचा सके और न तुम्हें कोई हानि पहुँचा सके?
\end{hindi}}
\flushright{\begin{Arabic}
\quranayah[21][67]
\end{Arabic}}
\flushleft{\begin{hindi}
धिक्कार है तुमपर, और उनपर भी, जिनको तुम अल्लाह को छोड़कर पूजते हो! तो क्या तुम बुद्धि से काम नहीं लेते?"
\end{hindi}}
\flushright{\begin{Arabic}
\quranayah[21][68]
\end{Arabic}}
\flushleft{\begin{hindi}
उन्होंने कहा, "जला दो उसे, और सहायक हो अपने देवताओं के, यदि तुम्हें कुछ करना है।"
\end{hindi}}
\flushright{\begin{Arabic}
\quranayah[21][69]
\end{Arabic}}
\flushleft{\begin{hindi}
हमने कहा, "ऐ आग! ठंड़ी हो जा और सलामती बन जा इबराहीम पर!"
\end{hindi}}
\flushright{\begin{Arabic}
\quranayah[21][70]
\end{Arabic}}
\flushleft{\begin{hindi}
उन्होंने उसके साथ एक चाल चलनी चाही, किन्तु हमने उन्हीं को घाटे में डाल दिया
\end{hindi}}
\flushright{\begin{Arabic}
\quranayah[21][71]
\end{Arabic}}
\flushleft{\begin{hindi}
और हम उसे और लूत को बचाकर उस भूभाग की ओर निकाल ले गए, जिसमें हमने दुनियावालों के लिए बरकतें रखी थीं
\end{hindi}}
\flushright{\begin{Arabic}
\quranayah[21][72]
\end{Arabic}}
\flushleft{\begin{hindi}
और हमने उसे इसहाक़ प्रदान किया और तदधिक याक़ूब भी। और प्रत्येक को हमने नेक बनाया
\end{hindi}}
\flushright{\begin{Arabic}
\quranayah[21][73]
\end{Arabic}}
\flushleft{\begin{hindi}
और हमने उन्हें नायक बनाया कि वे हमारे आदेश से मार्ग दिखाते थे और हमने उनकी ओर नेक कामों के करने और नमाज़ की पाबन्दी करने और ज़कात देने की प्रकाशना की, और वे हमारी बन्दगी में लगे हुए थे
\end{hindi}}
\flushright{\begin{Arabic}
\quranayah[21][74]
\end{Arabic}}
\flushleft{\begin{hindi}
और रहा लूत तो उसे हमने निर्णय-शक्ति और ज्ञान प्रदान किया और उसे उस बस्ती से छुटकारा दिया जो गन्दे कर्म करती थी। वास्तव में वह बहुत ही बुरी और अवज्ञाकारी क़ौम थी
\end{hindi}}
\flushright{\begin{Arabic}
\quranayah[21][75]
\end{Arabic}}
\flushleft{\begin{hindi}
और उसको हमने अपनी दयालुता में प्रवेश कराया। निस्संदेह वह अच्छे लोगों में से था
\end{hindi}}
\flushright{\begin{Arabic}
\quranayah[21][76]
\end{Arabic}}
\flushleft{\begin{hindi}
और नूह की भी चर्चा करो, जबकि उसने इससे पहले हमें पुकारा था, तो हमने उसकी सुन ली और हमने उसे और उसके लोगों को बड़े क्लेश से छुटकारा दिया
\end{hindi}}
\flushright{\begin{Arabic}
\quranayah[21][77]
\end{Arabic}}
\flushleft{\begin{hindi}
औऱ उस क़ौम के मुक़ाबले में जिसने हमारी आयतों को झुठला दिया था, हमने उसकी सहायता की। वास्तव में वे बुरे लोग थे। अतः हमने उन सबको डूबो दिया
\end{hindi}}
\flushright{\begin{Arabic}
\quranayah[21][78]
\end{Arabic}}
\flushleft{\begin{hindi}
औऱ दाऊद और सुलैमान पर भी हमने कृपा-स्पष्ट की। याद करो जबकि वे दोनों खेती के एक झगड़े का निबटारा कर रहे थे, जब रात को कुछ लोगों की बकरियाँ उसे रौंद गई थीं। और उनका (क़ौम के लोगों का) फ़ैसला हमारे सामने था
\end{hindi}}
\flushright{\begin{Arabic}
\quranayah[21][79]
\end{Arabic}}
\flushleft{\begin{hindi}
तब हमने उसे सुलैमान को समझा दिया और यूँ तो हरेक को हमने निर्णय-शक्ति और ज्ञान प्रदान किया था। और दाऊद के साथ हमने पहाड़ों को वशीभूत कर दिया था, जो तसबीह करते थे, और पक्षियों को भी। और ऐसा करनेवाले हम भी थे
\end{hindi}}
\flushright{\begin{Arabic}
\quranayah[21][80]
\end{Arabic}}
\flushleft{\begin{hindi}
और हमने उसे तुम्हारे लिए एक परिधान (बनाने) की शिल्प-कला भी सिखाई थी, ताकि युद्ध में वह तुम्हारी रक्षा करे। फिर क्या तुम आभार मानते हो?
\end{hindi}}
\flushright{\begin{Arabic}
\quranayah[21][81]
\end{Arabic}}
\flushleft{\begin{hindi}
और सुलैमान के लिए हमने तेज वायु को वशीभूत कर दिया था, जो उसके आदेश से उस भूभाग की ओर चलती थी जिसे हमने बरकत दी थी। हम तो हर चीज़ का ज्ञान रखते है
\end{hindi}}
\flushright{\begin{Arabic}
\quranayah[21][82]
\end{Arabic}}
\flushleft{\begin{hindi}
और कितने ही शैतानों को भी अधीन किया था, जो उसके लिए गोते लगाते और इसके अतिरिक्त दूसरा काम भी करते थे। और हम ही उनको संभालनेवाले थे
\end{hindi}}
\flushright{\begin{Arabic}
\quranayah[21][83]
\end{Arabic}}
\flushleft{\begin{hindi}
और अय्यूब पर भी दया दर्शाई। याद करो जबकि उसने अपने रब को पुकारा कि "मुझे बहुत तकलीफ़ पहुँची है, और तू सबसे बढ़कर दयावान है।"
\end{hindi}}
\flushright{\begin{Arabic}
\quranayah[21][84]
\end{Arabic}}
\flushleft{\begin{hindi}
अतः हमने उसकी सुन ली और जिस तकलीफ़ में वह पड़ा था उसको दूर कर दिया, और हमने उसे उसके परिवार के लोग दिए और उनके साथ उनके जैसे और भी दिए अपने यहाँ दयालुता के रूप में और एक याददिहानी के रूप में बन्दगी करनेवालों के लिए
\end{hindi}}
\flushright{\begin{Arabic}
\quranayah[21][85]
\end{Arabic}}
\flushleft{\begin{hindi}
और इसमाईल और इदरीस और ज़ुलकिफ़्ल पर भी कृपा-स्पष्ट की। इनमें से प्रत्येक धैर्यवानों में से था
\end{hindi}}
\flushright{\begin{Arabic}
\quranayah[21][86]
\end{Arabic}}
\flushleft{\begin{hindi}
औऱ उन्हें हमने अपनी दयालुता में प्रवेश कराया। निस्संदेह वे सब अच्छे लोगों में से थे
\end{hindi}}
\flushright{\begin{Arabic}
\quranayah[21][87]
\end{Arabic}}
\flushleft{\begin{hindi}
और ज़ुन्नून (मछलीवाले) पर भी दया दर्शाई। याद करो जबकि वह अत्यन्त क्रद्ध होकर चल दिया और समझा कि हम उसे तंगी में न डालेंगे। अन्त में उसनें अँधेरों में पुकारा, "तेरे सिवा कोई इष्ट-पूज्य नहीं, महिमावान है तू! निस्संदेह मैं दोषी हूँ।"
\end{hindi}}
\flushright{\begin{Arabic}
\quranayah[21][88]
\end{Arabic}}
\flushleft{\begin{hindi}
तब हमने उसकी प्रार्थना स्वीकार की और उसे ग़म से छुटकारा दिया। इसी प्रकार तो हम मोमिनों को छुटकारा दिया करते है
\end{hindi}}
\flushright{\begin{Arabic}
\quranayah[21][89]
\end{Arabic}}
\flushleft{\begin{hindi}
और ज़करिया पर भी कृपा की। याद करो जबकि उसने अपने रब को पुकारा, "ऐ मेरे रब! मुझे अकेला न छोड़ यूँ, सबसे अच्छा वारिस तो तू ही है।"
\end{hindi}}
\flushright{\begin{Arabic}
\quranayah[21][90]
\end{Arabic}}
\flushleft{\begin{hindi}
अतः हमने उसकी प्रार्थना स्वीकार कर ली और उसे याह्या् प्रदान किया और उसके लिए उसकी पत्नी को स्वस्थ कर दिया। निश्चय ही वे नेकी के कामों में एक-दूसरे के मुक़ाबले में जल्दी करते थे। और हमें ईप्सा (चाह) और भय के साथ पुकारते थे और हमारे आगे दबे रहते थे
\end{hindi}}
\flushright{\begin{Arabic}
\quranayah[21][91]
\end{Arabic}}
\flushleft{\begin{hindi}
और वह नारी जिसने अपने सतीत्व की रक्षा की थी, हमने उसके भीतर अपनी रूह फूँकी और उसे और उसके बेटे को सारे संसार के लिए एक निशानी बना दिया
\end{hindi}}
\flushright{\begin{Arabic}
\quranayah[21][92]
\end{Arabic}}
\flushleft{\begin{hindi}
"निश्चय ही यह तुम्हारा समुदाय एक ही समुदाय है और मैं तुम्हारा रब हूँ। अतः तुम मेरी बन्दगी करो।"
\end{hindi}}
\flushright{\begin{Arabic}
\quranayah[21][93]
\end{Arabic}}
\flushleft{\begin{hindi}
किन्तु उन्होंने आपस में अपने मामलों को टुकड़े-टुकड़े कर डाला। - प्रत्येक को हमारी ओर पलटना है। -
\end{hindi}}
\flushright{\begin{Arabic}
\quranayah[21][94]
\end{Arabic}}
\flushleft{\begin{hindi}
फिर जो अच्छे कर्म करेगा, शर्त या कि वह मोमिन हो, तो उसके प्रयास की उपेक्षा न होगी। हम तो उसके लिए उसे लिख रहे है
\end{hindi}}
\flushright{\begin{Arabic}
\quranayah[21][95]
\end{Arabic}}
\flushleft{\begin{hindi}
और किसी बस्ती के लिए असम्भव है जिसे हमने विनष्ट कर दिया कि उसके लोग (क़ियामत के दिन दंड पाने हेतु) न लौटें
\end{hindi}}
\flushright{\begin{Arabic}
\quranayah[21][96]
\end{Arabic}}
\flushleft{\begin{hindi}
यहाँ तक कि वह समय आ जाए जब याजूज और माजूज खोल दिए जाएँगे। और वे हर ऊँची जगह से निकल पड़ेंगे
\end{hindi}}
\flushright{\begin{Arabic}
\quranayah[21][97]
\end{Arabic}}
\flushleft{\begin{hindi}
और सच्चा वादा निकट आ लगेगा, तो क्या देखेंगे कि उन लोगों की आँखें फटी की फटी रह गई हैं, जिन्होंने इनकार किया था, "हाय, हमारा दुर्भाग्य! हम इसकी ओर से असावधान रहे, बल्कि हम ही अत्याचारी थे।"
\end{hindi}}
\flushright{\begin{Arabic}
\quranayah[21][98]
\end{Arabic}}
\flushleft{\begin{hindi}
"निश्चय ही तुम और वह कुछ जिनको तुम अल्लाह को छोड़कर पूजते हो सब जहन्नम के ईधन हो। तुम उसके घाट उतरोगे।"
\end{hindi}}
\flushright{\begin{Arabic}
\quranayah[21][99]
\end{Arabic}}
\flushleft{\begin{hindi}
यदि वे पूज्य होते, तो उसमें न उतरते। और वे सब उसमें सदैव रहेंगे भी
\end{hindi}}
\flushright{\begin{Arabic}
\quranayah[21][100]
\end{Arabic}}
\flushleft{\begin{hindi}
उनके लिए वहाँ शोर गुल होगा और वे वहाँ कुछ भी नहीं सुन सकेंगे
\end{hindi}}
\flushright{\begin{Arabic}
\quranayah[21][101]
\end{Arabic}}
\flushleft{\begin{hindi}
रहे वे लोग जिनके लिए पहले ही हमारी ओर से अच्छे इनाम का वादा हो चुका है, वे उससे दूर रहेंगे
\end{hindi}}
\flushright{\begin{Arabic}
\quranayah[21][102]
\end{Arabic}}
\flushleft{\begin{hindi}
वे उसकी आहट भी नहीं सुनेंगे और अपनी मनचाही चीज़ों के मध्य सदैव रहेंगे
\end{hindi}}
\flushright{\begin{Arabic}
\quranayah[21][103]
\end{Arabic}}
\flushleft{\begin{hindi}
वह सबसे बड़ी घबराहट उन्हें ग़म में न डालेगी। फ़रिश्ते उनका स्वागत करेगें, "यह तुम्हारा वही दिन है, जिसका तुमसे वादा किया जाता रहा है।"
\end{hindi}}
\flushright{\begin{Arabic}
\quranayah[21][104]
\end{Arabic}}
\flushleft{\begin{hindi}
जिस दिन हम आकाश को लपेट लेंगे, जैसे पंजी में पन्ने लपेटे जाते हैं, जिस प्रकाऱ पहले हमने सृष्टि का आरम्भ किया था उसी प्रकार हम उसकी पुनरावृत्ति करेंगे। यह हमारे ज़िम्मे एक वादा है। निश्चय ही हमें यह करना है
\end{hindi}}
\flushright{\begin{Arabic}
\quranayah[21][105]
\end{Arabic}}
\flushleft{\begin{hindi}
और हम ज़बूर में याददिहानी के पश्चात लिए चुके है कि "धरती के वारिस मेरे अच्छे बन्दें होंगे।"
\end{hindi}}
\flushright{\begin{Arabic}
\quranayah[21][106]
\end{Arabic}}
\flushleft{\begin{hindi}
इसमें बन्दगी करनेवालों लोगों के लिए एक संदेश है
\end{hindi}}
\flushright{\begin{Arabic}
\quranayah[21][107]
\end{Arabic}}
\flushleft{\begin{hindi}
हमने तुम्हें सारे संसार के लिए बस एक सर्वथा दयालुता बनाकर भेजा है
\end{hindi}}
\flushright{\begin{Arabic}
\quranayah[21][108]
\end{Arabic}}
\flushleft{\begin{hindi}
कहो, "मेरे पास को बस यह प्रकाशना की जाती है कि तुम्हारा पूज्य-प्रभु अकेला पूज्य-प्रभु है। फिर क्या तुम आज्ञाकारी होते हो?"
\end{hindi}}
\flushright{\begin{Arabic}
\quranayah[21][109]
\end{Arabic}}
\flushleft{\begin{hindi}
फिर यदि वे मुँह फेरें तो कह दो, "मैंने तुम्हें सामान्य रूप से सावधान कर दिया है। अब मैं यह नहीं जानता कि जिसका तुमसे वादा किया जा रहा है वह निकट है या दूर।"
\end{hindi}}
\flushright{\begin{Arabic}
\quranayah[21][110]
\end{Arabic}}
\flushleft{\begin{hindi}
निश्चय ही वह ऊँची आवाज़ में कही हुई बात को जानता है और उसे भी जानता है जो तुम छिपाते हो
\end{hindi}}
\flushright{\begin{Arabic}
\quranayah[21][111]
\end{Arabic}}
\flushleft{\begin{hindi}
मुझे नहीं मालूम कि कदाचित यह तुम्हारे लिए एक परीक्षा हो और एक नियत समय तक के लिए जीवन-सुख
\end{hindi}}
\flushright{\begin{Arabic}
\quranayah[21][112]
\end{Arabic}}
\flushleft{\begin{hindi}
उसने कहा, "ऐ मेरे रब, सत्य का फ़ैसला कर दे! और हमारा रब रहमान है। उसी से सहायता की प्रार्थना है, उन बातों के मुक़ाबले में जो तुम लोग बयान करते हो।"
\end{hindi}}
\chapter{Al-Hajj (The Pilgrimage)}
\begin{Arabic}
\Huge{\centerline{\basmalah}}\end{Arabic}
\flushright{\begin{Arabic}
\quranayah[22][1]
\end{Arabic}}
\flushleft{\begin{hindi}
ऐ लोगो! अपने रब का डर रखो! निश्चय ही क़ियामत की घड़ी का भूकम्प बड़ी भयानक चीज़ है
\end{hindi}}
\flushright{\begin{Arabic}
\quranayah[22][2]
\end{Arabic}}
\flushleft{\begin{hindi}
जिस जिन तुम उसे देखोगे, हाल यह होगा कि प्रत्येक दूध पिलानेवाली अपने दूध पीते बच्चे को भूल जाएगी और प्रत्येक गर्भवती अपना गर्भभार रख देगी। और लोगों को तुम नशे में देखोगे, हालाँकि वे नशे में न होंगे, बल्कि अल्लाह की यातना है ही बड़ी कठोर चीज़
\end{hindi}}
\flushright{\begin{Arabic}
\quranayah[22][3]
\end{Arabic}}
\flushleft{\begin{hindi}
लोगों में कोई ऐसा भी है, जो ज्ञान के बिना अल्लाह के विषय में झगड़ता है और प्रत्येक सरकश शैतान का अनुसरण करता है
\end{hindi}}
\flushright{\begin{Arabic}
\quranayah[22][4]
\end{Arabic}}
\flushleft{\begin{hindi}
जबकि उसके लिए लिख दिया गया है कि जो उससे मित्रता का सम्बन्ध रखेगा उसे वह पथभ्रष्ट करके रहेगा और उसे दहकती अग्नि की यातना की ओर ले जाएगा
\end{hindi}}
\flushright{\begin{Arabic}
\quranayah[22][5]
\end{Arabic}}
\flushleft{\begin{hindi}
ऐ लोगो! यदि तुम्हें दोबारा जी उठने के विषय में कोई सन्देह हो तो देखो, हमने तुम्हें मिट्टी से पैदा किया, फिर वीर्य से, फिर लोथड़े से, फिर माँस की बोटी से जो बनावट में पूर्ण दशा में भी होती है और अपूर्ण दशा में भी, ताकि हम तुमपर स्पष्ट कर दें और हम जिसे चाहते है एक नियत समय तक गर्भाशयों में ठहराए रखते है। फिर तुम्हें एक बच्चे के रूप में निकाल लाते है। फिर (तुम्हारा पालन-पोषण होता है) ताकि तुम अपनी युवावस्था को प्राप्त हो और तुममें से कोई तो पहले मर जाता है और कोई बुढ़ापे की जीर्ण अवस्था की ओर फेर दिया जाता है जिसके परिणामस्वरूप, जानने के पश्चात वह कुछ भी नहीं जानता। और तुम भूमि को देखते हो कि सूखी पड़ी है। फिर जहाँ हमने उसपर पानी बरसाया कि वह फबक उठी और वह उभर आई और उसने हर प्रकार की शोभायमान चीज़े उगाई
\end{hindi}}
\flushright{\begin{Arabic}
\quranayah[22][6]
\end{Arabic}}
\flushleft{\begin{hindi}
यह इसलिए कि अल्लाह ही सत्य है और वह मुर्दों को जीवित करता है और उसे हर चीज़ की सामर्थ्य प्राप्ती है
\end{hindi}}
\flushright{\begin{Arabic}
\quranayah[22][7]
\end{Arabic}}
\flushleft{\begin{hindi}
और यह कि क़ियामत की घड़ी आनेवाली है, इसमें कोई सन्देह नहीं है। और यह कि अल्लाह उन्हें उठाएगा जो क़ब्रों में है
\end{hindi}}
\flushright{\begin{Arabic}
\quranayah[22][8]
\end{Arabic}}
\flushleft{\begin{hindi}
और लोगों मे कोई ऐसा है जो किसी ज्ञान, मार्गदर्शन और प्रकाशमान किताब के बिना अल्लाह के विषय में (घमंड से) अपने पहलू मोड़ते हुए झगड़ता है,
\end{hindi}}
\flushright{\begin{Arabic}
\quranayah[22][9]
\end{Arabic}}
\flushleft{\begin{hindi}
ताकि अल्लाह के मार्ग से भटका दे। उसके लिए दुनिया में भी रुसवाई है और क़ियामत के दिन हम उसे जलने की यातना का मज़ा चखाएँगे
\end{hindi}}
\flushright{\begin{Arabic}
\quranayah[22][10]
\end{Arabic}}
\flushleft{\begin{hindi}
(कहा जाएगा,) यह उसके कारण है जो तेरे हाथों ने आगे भेजा था और इसलिए कि अल्लाह बन्दों पर तनिक भी ज़ुल्म करनेवाला नहीं
\end{hindi}}
\flushright{\begin{Arabic}
\quranayah[22][11]
\end{Arabic}}
\flushleft{\begin{hindi}
और लोगों में कोई ऐसा है, जो एक किनारे पर रहकर अल्लाह की बन्दगी करता है। यदि उसे लाभ पहुँचा तो उससे सन्तुष्ट हो गया और यदि उसे कोई आज़माइश पेश आ गई तो औंधा होकर पलट गया। दुनिया भी खोई और आख़िरत भी। यही है खुला घाटा
\end{hindi}}
\flushright{\begin{Arabic}
\quranayah[22][12]
\end{Arabic}}
\flushleft{\begin{hindi}
वह अल्लाह को छोड़कर उसे पुकारता है, जो न उसे हानि पहुँचा सके और न उसे लाभ पहुँचा सके। यही हैं परले दर्जे की गुमराही
\end{hindi}}
\flushright{\begin{Arabic}
\quranayah[22][13]
\end{Arabic}}
\flushleft{\begin{hindi}
वह उसको पुकारता है जिससे पहुँचनेवाली हानि उससे अपेक्षित लाभ की अपेक्षा अधिक निकट है। बहुत ही बुरा संरक्षक है वह और बहुत ही बुरा साथी!
\end{hindi}}
\flushright{\begin{Arabic}
\quranayah[22][14]
\end{Arabic}}
\flushleft{\begin{hindi}
निश्चय ही अल्लाह उन लोगों को, जो ईमान लाए और उन्होंने अच्छे कर्म किए, ऐसे बाग़ों में दाखिल करेगा, जिनके नीचे नहरें बह रही होंगी। निस्संदेह अल्लाह जो चाहे करे
\end{hindi}}
\flushright{\begin{Arabic}
\quranayah[22][15]
\end{Arabic}}
\flushleft{\begin{hindi}
जो कोई यह समझता है कि अल्लाह दुनिया औऱ आख़िरत में उसकी (रसूल की) कदापि कोई सहायता न करेगा तो उसे चाहिए कि वह आकाश की ओर एक रस्सी ताने, फिर (अल्लाह की सहायता के सिलसिले को) काट दे। फिर देख ले कि क्या उसका उपाय उस चीज़ को दूर कर सकता है जो उसे क्रोध में डाले हुए है
\end{hindi}}
\flushright{\begin{Arabic}
\quranayah[22][16]
\end{Arabic}}
\flushleft{\begin{hindi}
इसी प्रकार हमने इस (क़ुरआन) को स्पष्ट आयतों के रूप में अवतरित किया। और बात यह है कि अल्लाह जिसे चाहता है मार्ग दिखाता है
\end{hindi}}
\flushright{\begin{Arabic}
\quranayah[22][17]
\end{Arabic}}
\flushleft{\begin{hindi}
जो लोग ईमान लाए और जो यहूदी हुए और साबिई और ईसाई और मजूस और जिन लोगों ने शिर्क किया - इस सबके बीच अल्लाह क़ियामत के दिन फ़ैसला कर देगा। निस्संदेह अल्लाह की दृष्टि में हर चीज़ है
\end{hindi}}
\flushright{\begin{Arabic}
\quranayah[22][18]
\end{Arabic}}
\flushleft{\begin{hindi}
क्या तुमनें देखा नहीं कि अल्लाह ही को सजदा करते है वे सब जो आकाशों में है और जो धरती में है, और सूर्य, चन्द्रमा, तारे पहाड़, वृक्ष, जानवर और बहुत-से मनुष्य? और बहुत-से ऐसे है जिनपर यातना का औचित्य सिद्ध हो चुका है, और जिसे अल्लाह अपमानित करे उस सम्मानित करनेवाला कोई नहीं। निस्संदेह अल्लाह जो चाहे करता है
\end{hindi}}
\flushright{\begin{Arabic}
\quranayah[22][19]
\end{Arabic}}
\flushleft{\begin{hindi}
ये दो विवादी हैं, जो अपने रब के विषय में आपस में झगड़े। अतः जिन लोगों ने कुफ्र किया उनके लिए आग के वस्त्र काटे जा चुके है। उनके सिरों पर खौलता हुआ पानी डाला जाएगा
\end{hindi}}
\flushright{\begin{Arabic}
\quranayah[22][20]
\end{Arabic}}
\flushleft{\begin{hindi}
इससे जो कुछ उनके पेटों में है, वह पिघल जाएगा और खालें भी
\end{hindi}}
\flushright{\begin{Arabic}
\quranayah[22][21]
\end{Arabic}}
\flushleft{\begin{hindi}
और उनके लिए (दंड देने को) लोहे के गुर्ज़ होंगे
\end{hindi}}
\flushright{\begin{Arabic}
\quranayah[22][22]
\end{Arabic}}
\flushleft{\begin{hindi}
जब कभी भी घबराकर उससे निकलना चाहेंगे तो उसी में लौटा दिए जाएँगे और (कहा जाएगा,) "चखो दहकती आग की यातना का मज़ा!"
\end{hindi}}
\flushright{\begin{Arabic}
\quranayah[22][23]
\end{Arabic}}
\flushleft{\begin{hindi}
निस्संदेह अल्लाह उन लोगों को, जो ईमान लाए और उन्होंने अच्छे कर्म किए, ऐसे बाग़ों में दाखिल करेगा जिनके नीचें नहरें बह रही होंगी। वहाँ वे सोने के कंगनों और मोती से आभूषित किए जाएँगे और वहाँ उनका परिधान रेशमी होगा
\end{hindi}}
\flushright{\begin{Arabic}
\quranayah[22][24]
\end{Arabic}}
\flushleft{\begin{hindi}
निर्देशित किया गया उन्हें अच्छे पाक बोल की ओर और उनको प्रशंसित अल्लाह का मार्ग दिखाया गया
\end{hindi}}
\flushright{\begin{Arabic}
\quranayah[22][25]
\end{Arabic}}
\flushleft{\begin{hindi}
जिन लोगों ने इनकार किया और वे अल्लाह के मार्ग से रोकते हैं और प्रतिष्ठित मस्जिद (काबा) से, जिसे हमने सब लोगों के लिए ऐसा बनाया है कि उसमें बराबर है वहाँ का रहनेवाला और बाहर से आया हुआ। और जो व्यक्ति उस (प्रतिष्ठित मस्जिद) में कुटिलता अर्थात ज़ूल्म के साथ कुछ करना चाहेगा, उसे हम दुखद यातना का मज़ा चखाएँगे
\end{hindi}}
\flushright{\begin{Arabic}
\quranayah[22][26]
\end{Arabic}}
\flushleft{\begin{hindi}
याद करो जब कि हमने इबराहीम के लिए अल्लाह के घर को ठिकाना बनाया, इस आदेश के साथ कि "मेरे साथ किसी चीज़ को साझी न ठहराना और मेरे घर को तवाफ़ (परिक्रमा) करनेवालों और खड़े होने और झुकने और सजदा करनेवालों के लिए पाक-साफ़ रखना।"
\end{hindi}}
\flushright{\begin{Arabic}
\quranayah[22][27]
\end{Arabic}}
\flushleft{\begin{hindi}
और लोगों में हज के लिए उद्घोषणा कर दो कि "वे प्रत्येक गहरे मार्ग से, पैदल भी और दुबली-दुबली ऊँटनियों पर, तेरे पास आएँ
\end{hindi}}
\flushright{\begin{Arabic}
\quranayah[22][28]
\end{Arabic}}
\flushleft{\begin{hindi}
ताकि वे उन लाभों को देखें जो वहाँ उनके लिए रखे गए है। और कुछ ज्ञात और निश्चित दिनों में उन चौपाए अर्थात मवेशियों पर अल्लाह का नाम लें, जो उसने उन्हें दिए है। फिर उसमें से स्वयं भी खाओ और तंगहाल मुहताज को भी खिलाओ।"
\end{hindi}}
\flushright{\begin{Arabic}
\quranayah[22][29]
\end{Arabic}}
\flushleft{\begin{hindi}
फिर उन्हें चाहिए कि अपना मैल-कुचैल दूर करें और अपनी मन्नतें पूरी करें और इस पुरातन घर का तवाफ़ (परिक्रमा) करें
\end{hindi}}
\flushright{\begin{Arabic}
\quranayah[22][30]
\end{Arabic}}
\flushleft{\begin{hindi}
इन बातों का ध्यान रखों और जो कोई अल्लाह द्वारा निर्धारित मर्यादाओं का आदर करे, तो यह उसके रब के यहाँ उसी के लिए अच्छा है। और तुम्हारे लिए चौपाए हलाल है, सिवाय उनके जो तुम्हें बताए गए हैं। तो मूर्तियों की गन्दगी से बचो और बचो झूठी बातों से
\end{hindi}}
\flushright{\begin{Arabic}
\quranayah[22][31]
\end{Arabic}}
\flushleft{\begin{hindi}
इस तरह कि अल्लाह ही की ओर के होकर रहो। उसके साथ किसी को साझी न ठहराओ, क्योंकि जो कोई अल्लाह के साथ साझी ठहराता है तो मानो वह आकाश से गिर पड़ा। फिर चाहे उसे पक्षी उचक ले जाएँ या वायु उसे किसी दूरवर्ती स्थान पर फेंक दे
\end{hindi}}
\flushright{\begin{Arabic}
\quranayah[22][32]
\end{Arabic}}
\flushleft{\begin{hindi}
इन बातों का ख़याल रखो। और जो कोई अल्लाह के नाम लगी चीज़ों का आदर करे, तो निस्संदेह वे (चीज़ें) दिलों के तक़वा (धर्मपरायणता) से सम्बन्ध रखती है
\end{hindi}}
\flushright{\begin{Arabic}
\quranayah[22][33]
\end{Arabic}}
\flushleft{\begin{hindi}
उनमें एक निश्चित समय तक तुम्हारे लिए फ़ायदे है। फिर उनको उस पुरातन घर तक (क़ुरबानी के लिए) पहुँचना है
\end{hindi}}
\flushright{\begin{Arabic}
\quranayah[22][34]
\end{Arabic}}
\flushleft{\begin{hindi}
और प्रत्येक समुदाय के लिए हमने क़ुरबानी का विधान किया, ताकि वे उन जानवरों अर्थात मवेशियों पर अल्लाह का नाम लें, जो उसने उन्हें प्रदान किए हैं। अतः तुम्हारा पूज्य-प्रभु अकेला पूज्य-प्रभु है। तो उसी के आज्ञाकारी बनकर रहो और विनम्रता अपनानेवालों को शुभ सूचना दे दो
\end{hindi}}
\flushright{\begin{Arabic}
\quranayah[22][35]
\end{Arabic}}
\flushleft{\begin{hindi}
ये वे लोग है कि जब अल्लाह को याद किया जाता है तो उनके दिल दहल जाते है और जो मुसीबत उनपर आती है उसपर धैर्य से काम लेते है और नमाज़ को क़ायम करते है, और जो कुछ हमने उन्हें दिया है उसमें से ख़र्च करते है
\end{hindi}}
\flushright{\begin{Arabic}
\quranayah[22][36]
\end{Arabic}}
\flushleft{\begin{hindi}
(क़ुरबानी के) ऊँटों को हमने तुम्हारे लिए अल्लाह की निशानियों में से बनाया है। तुम्हारे लिए उनमें भलाई है। अतः खड़ा करके उनपर अल्लाह का नाम लो। फिर जब उनके पहलू भूमि से आ लगें तो उनमें से स्वयं भी खाओ औऱ संतोष से बैठनेवालों को भी खिलाओ और माँगनेवालों को भी। ऐसी ही करो। हमने उनको तुम्हारे लिए वशीभूत कर दिया है, ताकि तुम कृतज्ञता दिखाओ
\end{hindi}}
\flushright{\begin{Arabic}
\quranayah[22][37]
\end{Arabic}}
\flushleft{\begin{hindi}
न उनके माँस अल्लाह को पहुँचते है और न उनके रक्त। किन्तु उसे तुम्हारा तक़वा (धर्मपरायणता) पहुँचता है। इस प्रकार उसने उन्हें तुम्हारे लिए वशीभूत किया है, ताकि तुम अल्लाह की बड़ाई बयान करो, इसपर कि उसने तुम्हारा मार्गदर्शन किया और सुकर्मियों को शुभ सूचना दे दो
\end{hindi}}
\flushright{\begin{Arabic}
\quranayah[22][38]
\end{Arabic}}
\flushleft{\begin{hindi}
निश्चय ही अल्लाह उन लोगों की ओर से प्रतिरक्षा करता है, जो ईमान लाए। निस्संदेह अल्लाह किसी विश्वासघाती, अकृतज्ञ को पसन्द नहीं करता
\end{hindi}}
\flushright{\begin{Arabic}
\quranayah[22][39]
\end{Arabic}}
\flushleft{\begin{hindi}
अनुमति दी गई उन लोगों को जिनके विरुद्ध युद्ध किया जा रहा है, क्योंकि उनपर ज़ुल्म किया गया - और निश्चय ही अल्लाह उनकी सहायता की पूरी सामर्थ्य रखता है। -
\end{hindi}}
\flushright{\begin{Arabic}
\quranayah[22][40]
\end{Arabic}}
\flushleft{\begin{hindi}
ये वे लोग है जो अपने घरों से नाहक़ निकाले गए, केवल इसलिए कि वे कहते है कि "हमारा रब अल्लाह है।" यदि अल्लाह लोगों को एक-दूसरे के द्वारा हटाता न रहता तो मठ और गिरजा और यहूदी प्रार्थना भवन और मस्जिदें, जिनमें अल्लाह का अधिक नाम लिया जाता है, सब ढा दी जातीं। अल्लाह अवश्य उसकी सहायता करेगा, जो उसकी सहायता करेगा - निश्चय ही अल्लाह बड़ा बलवान, प्रभुत्वशाली है
\end{hindi}}
\flushright{\begin{Arabic}
\quranayah[22][41]
\end{Arabic}}
\flushleft{\begin{hindi}
ये वे लोग है कि यदि धरती में हम उन्हें सत्ता प्रदान करें तो वे नमाज़ का आयोजन करेंगे और ज़कात देंगे और भलाई का आदेश करेंगे और बुराई से रोकेंगे। और सब मामलों का अन्तिम परिणाम अल्लाह ही के हाथ में है
\end{hindi}}
\flushright{\begin{Arabic}
\quranayah[22][42]
\end{Arabic}}
\flushleft{\begin{hindi}
यदि वे तुम्हें झुठलाते है तो उनसे पहले नूह की क़ौम, आद और समूद
\end{hindi}}
\flushright{\begin{Arabic}
\quranayah[22][43]
\end{Arabic}}
\flushleft{\begin{hindi}
और इबराहीम की क़ौम और लूत की क़ौम
\end{hindi}}
\flushright{\begin{Arabic}
\quranayah[22][44]
\end{Arabic}}
\flushleft{\begin{hindi}
और मदयनवाले भी झुठला चुके है और मूसा को भी झूठलाया जा चुका है। किन्तु मैंने इनकार करनेवालों को मुहलत दी, फिर उन्हें पकड़ लिया। तो कैसी रही मेरी यंत्रणा!
\end{hindi}}
\flushright{\begin{Arabic}
\quranayah[22][45]
\end{Arabic}}
\flushleft{\begin{hindi}
कितनी ही बस्तियाँ है जिन्हें हमने विनष्ट कर दिया इस दशा में कि वे ज़ालिम थी, तो वे अपनी छतों के बल गिरी पड़ी है। और कितने ही परित्यक्त (उजाड़) कुएँ पड़े है और कितने ही पक्के महल भी!
\end{hindi}}
\flushright{\begin{Arabic}
\quranayah[22][46]
\end{Arabic}}
\flushleft{\begin{hindi}
क्या वे धरती में चले फिरे नहीं है कि उनके दिल होते जिनसे वे समझते या (कम से कम) कान होते जिनसे वे सुनते? बात यह है कि आँखें अंधी नहीं हो जातीं, बल्कि वे दिल अंधे हो जाते है जो सीनों में होते है
\end{hindi}}
\flushright{\begin{Arabic}
\quranayah[22][47]
\end{Arabic}}
\flushleft{\begin{hindi}
और वे तुमसे यातना के लिए जल्दी मचा रहे है! अल्लाह कदापि अपने वादे के विरुद्ध न करेंगा। किन्तु तुम्हारे रब के यहाँ एक दिन, तुम्हारी गणना के अनुसार, एक हजार वर्ष जैसा है
\end{hindi}}
\flushright{\begin{Arabic}
\quranayah[22][48]
\end{Arabic}}
\flushleft{\begin{hindi}
कितनी ही बस्तियाँ है जिनको मैंने मुहलत दी इस दशा में कि वे ज़ालिम थीं। फिर मैंने उन्हें पकड़ लिया और अन्ततः आना तो मेरी ही ओर है
\end{hindi}}
\flushright{\begin{Arabic}
\quranayah[22][49]
\end{Arabic}}
\flushleft{\begin{hindi}
कह दो, "ऐ लोगों! मैं तो तुम्हारे लिए बस एक साफ़-साफ़ सचेत करनेवाला हूँ।"
\end{hindi}}
\flushright{\begin{Arabic}
\quranayah[22][50]
\end{Arabic}}
\flushleft{\begin{hindi}
फिर जो लोग ईमान लाए और उन्होंने अच्छे कर्म किए उनके लिए क्षमादान और सम्मानपूर्वक आजीविका है
\end{hindi}}
\flushright{\begin{Arabic}
\quranayah[22][51]
\end{Arabic}}
\flushleft{\begin{hindi}
किन्तु जिन लोगों ने हमारी आयतों को नीचा दिखाने की कोशिश की, वही भड़कती आगवाले है
\end{hindi}}
\flushright{\begin{Arabic}
\quranayah[22][52]
\end{Arabic}}
\flushleft{\begin{hindi}
तुमसे पहले जो रसूल और नबी भी हमने भेजा, तो जब भी उसने कोई कामना की तो शैतान ने उसकी कामना में विघ्न डालता है, अल्लाह उसे मिटा देता है। फिर अल्लाह अपनी आयतों को सुदृढ़ कर देता है। - अल्लाह सर्वज्ञ, बड़ा तत्वदर्शी है
\end{hindi}}
\flushright{\begin{Arabic}
\quranayah[22][53]
\end{Arabic}}
\flushleft{\begin{hindi}
ताकि शैतान के डाले हुए विघ्न को उन लोगों के लिए आज़माइश बना दे जिनके दिलों में रोग है और जिनके दिल कठोर है। निस्संदेह ज़ालिम परले दर्ज के विरोध में ग्रस्त है। -
\end{hindi}}
\flushright{\begin{Arabic}
\quranayah[22][54]
\end{Arabic}}
\flushleft{\begin{hindi}
और ताकि वे लोग जिन्हें ज्ञान मिला है, जान लें कि यह तुम्हारे रब की ओर से सत्य है। अतः वे इसपर ईमान लाएँ और उसके सामने उनके दिल झुक जाएँ और निश्चय ही अल्लाह ईमान लानेवालों को अवश्य सीधा मार्ग दिखाता है
\end{hindi}}
\flushright{\begin{Arabic}
\quranayah[22][55]
\end{Arabic}}
\flushleft{\begin{hindi}
जिन लोगों ने इनकार किया वे सदैव इसकी ओर से सन्देह में पड़े रहेंगे, यहाँ तक कि क़ियामत की घड़ी अचानक उनपर आ जाए या एक अशुभ दिन की यातना उनपर आ पहुँचे
\end{hindi}}
\flushright{\begin{Arabic}
\quranayah[22][56]
\end{Arabic}}
\flushleft{\begin{hindi}
उस दिन बादशाही अल्लाह ही की होगी। वह उनके बीच फ़ैसला कर देगा। अतः जो लोग ईमान लाए और उन्होंने अच्छे कर्म किए, वे नेमत भरी जन्नतों में होंगे
\end{hindi}}
\flushright{\begin{Arabic}
\quranayah[22][57]
\end{Arabic}}
\flushleft{\begin{hindi}
और जिन लोगों ने इनकार किया और हमारी आयतों को झुठलाया, उनके लिए अपमानजनक यातना है
\end{hindi}}
\flushright{\begin{Arabic}
\quranayah[22][58]
\end{Arabic}}
\flushleft{\begin{hindi}
और जिन लोगों ने अल्लाह के मार्ग में घरबार छोड़ा, फिर मारे गए या मर गए, अल्लाह अवश्य उन्हें अच्छी आजीविका प्रदान करेगा। और निस्संदेह अल्लाह ही उत्तम आजीविका प्रदान करनेवाला है
\end{hindi}}
\flushright{\begin{Arabic}
\quranayah[22][59]
\end{Arabic}}
\flushleft{\begin{hindi}
वह उन्हें ऐसी जगह प्रवेश कराएगा जिससे वे प्रसन्न हो जाएँगे। और निश्चय ही अल्लाह सर्वज्ञ, अत्यन्त सहनशील है
\end{hindi}}
\flushright{\begin{Arabic}
\quranayah[22][60]
\end{Arabic}}
\flushleft{\begin{hindi}
यह बात तो सुन ली। और जो कोई बदला लें, वैसा ही जैसा उसके साथ किया गया और फिर उसपर ज़्यादती की गई, तो अल्लाह अवश्य उसकी सहायता करेगा। निश्चय ही अल्लाह दरगुज़र करनेवाला (छोड़ देनेवाला), बहुत क्षमाशील है
\end{hindi}}
\flushright{\begin{Arabic}
\quranayah[22][61]
\end{Arabic}}
\flushleft{\begin{hindi}
यह इसलिए कि अल्लाह ही है जो रात को दिन में पिरोता हुआ ले आता है और दिन को रात में पिरोता हुआ ले आता है। और यह कि अल्लाह सुनता, देखता है
\end{hindi}}
\flushright{\begin{Arabic}
\quranayah[22][62]
\end{Arabic}}
\flushleft{\begin{hindi}
यह इसलिए कि अल्लाह ही सत्य है और जिसे वे उसको छोड़कर पुकारते है, वे सब असत्य है, और यह कि अल्लाह ही सर्वोच्च, महान है
\end{hindi}}
\flushright{\begin{Arabic}
\quranayah[22][63]
\end{Arabic}}
\flushleft{\begin{hindi}
क्या तुमने देखा नहीं कि अल्लाह आकाश से पानी बरसाता है, तो धरती हरी-भरी हो जाती है? निस्संदेह अल्लाह सूक्ष्मदर्शी, ख़बर रखनेवाला है
\end{hindi}}
\flushright{\begin{Arabic}
\quranayah[22][64]
\end{Arabic}}
\flushleft{\begin{hindi}
उसी का है जो कुछ आकाशों में और जो कुछ धरती में है। निस्संदेह अल्लाह ही निस्पृह प्रशंसनीय है
\end{hindi}}
\flushright{\begin{Arabic}
\quranayah[22][65]
\end{Arabic}}
\flushleft{\begin{hindi}
क्या तुमने देखा नहीं कि धरती में जो कुछ भी है उसे अल्लाह ने तुम्हारे लिए वशीभूत कर रखा है और नौका को भी कि उसके आदेश से दरिया में चलती है, और उसने आकाश को धरती पर गिरने से रोक रखा है। उसकी अनुज्ञा हो तो बात दूसरी है। निस्संदेह अल्लाह लोगों के हक़ में बड़ा करुणाशील, दयावान है
\end{hindi}}
\flushright{\begin{Arabic}
\quranayah[22][66]
\end{Arabic}}
\flushleft{\begin{hindi}
और वही है जिसने तुम्हें जीवन प्रदान किया। फिर वही तुम्हें मृत्यु देता है और फिर वही तुम्हें जीवित करनेवाला है। निस्संदेह मानव बड़ा ही अकृतज्ञ है
\end{hindi}}
\flushright{\begin{Arabic}
\quranayah[22][67]
\end{Arabic}}
\flushleft{\begin{hindi}
प्रत्येक समुदाय के लिए हमने बन्दगी की एक रीति निर्धारित कर दी है, जिसका पालन उसके लोग करते है। अतः इस मामले में वे तुमसे झगड़ने की राह न पाएँ। तुम तो अपने रब की ओर बुलाए जाओ। निस्संदेह तुम सीधे मार्ग पर हो
\end{hindi}}
\flushright{\begin{Arabic}
\quranayah[22][68]
\end{Arabic}}
\flushleft{\begin{hindi}
और यदि वे तुमसे झगड़ा करें तो कह दो कि "तुम जो कुछ करते हो अल्लाह उसे भली-भाँति जानता है
\end{hindi}}
\flushright{\begin{Arabic}
\quranayah[22][69]
\end{Arabic}}
\flushleft{\begin{hindi}
अल्लाह क़ियामत के दिन तुम्हारे बीच उस चीज़ का फ़ैसला कर देगा, जिसमें तुम विभेद करते हो।"
\end{hindi}}
\flushright{\begin{Arabic}
\quranayah[22][70]
\end{Arabic}}
\flushleft{\begin{hindi}
क्या तुम्हें नहीं मालूम कि अल्लाह जानता है जो कुछ आकाश और धरती मैं हैं? निश्चय ही वह (लोगों का कर्म) एक किताब में अंकित है। निस्संदेह वह (फ़ैसला करना) अल्लाह के लिए अत्यन्त सरल है
\end{hindi}}
\flushright{\begin{Arabic}
\quranayah[22][71]
\end{Arabic}}
\flushleft{\begin{hindi}
और वे अल्लाह से इतर उनकी बन्दगी करते है जिनके लिए न तो उसने कोई प्रमाण उतारा और न उन्हें उनके विषय में कोई ज्ञान ही है। और इन ज़ालिमों को कोई सहायक नहीं
\end{hindi}}
\flushright{\begin{Arabic}
\quranayah[22][72]
\end{Arabic}}
\flushleft{\begin{hindi}
और जब उन्हें हमारी स्पष्ट आयतें सुनाई जाती है, तो इनकार करनेवालों के चेहरों पर तुम्हें नागवारी प्रतीत होती है। लगता है कि अभी वे उन लोगों पर टूट पड़ेगे जो उन्हें हमारी आयतें सुनाते है। कह दो, "क्या मैं तुम्हे इससे बुरी चीज़ की ख़बर दूँ? आग है वह - अल्लाह ने इनकार करनेवालों से उसी का वादा कर रखा है - और वह बहुत ही बुरा ठिकाना है।"
\end{hindi}}
\flushright{\begin{Arabic}
\quranayah[22][73]
\end{Arabic}}
\flushleft{\begin{hindi}
ऐ लोगों! एक मिसाल पेश की जाती है। उसे ध्यान से सुनो, अल्लाह से हटकर तुम जिन्हें पुकारते हो वे एक मक्खी भी पैदा नहीं कर सकते। यद्यपि इसके लिए वे सब इकट्ठे हो जाएँ और यदि मक्खी उनसे कोई चीज़ छीन ले जाए तो उससे वे उसको छुड़ा भी नहीं सकते। बेबस और असहाय रहा चाहनेवाला भी (उपासक) और उसका अभीष्ट (उपास्य) भी
\end{hindi}}
\flushright{\begin{Arabic}
\quranayah[22][74]
\end{Arabic}}
\flushleft{\begin{hindi}
उन्होंने अल्लाह की क़द्र ही नहीं पहचानी जैसी कि उसकी क़द्र पहचाननी चाहिए थी। निश्चय ही अल्लाह अत्यन्त बलवान, प्रभुत्वशाली है
\end{hindi}}
\flushright{\begin{Arabic}
\quranayah[22][75]
\end{Arabic}}
\flushleft{\begin{hindi}
अल्लाह फ़रिश्तों में से संदेशवाहक चुनता और मनुष्यों में से भी। निश्चय ही अल्लाह सब कुछ सुनता, देखता है
\end{hindi}}
\flushright{\begin{Arabic}
\quranayah[22][76]
\end{Arabic}}
\flushleft{\begin{hindi}
वह जानता है जो कुछ उनके आगे है और जो कुछ उनके पीछे है। और सारे मामले अल्लाह ही की ओर पलटते है
\end{hindi}}
\flushright{\begin{Arabic}
\quranayah[22][77]
\end{Arabic}}
\flushleft{\begin{hindi}
ऐ ईमान लानेवालो! झुको और सजदा करो और अपने रब की बन्दही करो और भलाई करो, ताकि तुम्हें सफलता प्राप्त हो
\end{hindi}}
\flushright{\begin{Arabic}
\quranayah[22][78]
\end{Arabic}}
\flushleft{\begin{hindi}
और परस्पर मिलकर जिहाद करो अल्लाह के मार्ग में, जैसा कि जिहाद का हक़ है। उसने तुम्हें चुन लिया है - और धर्म के मामले में तुमपर कोई तंगी और कठिनाई नहीं रखी। तुम्हारे बाप इबराहीम के पंथ को तुम्हारे लिए पसन्द किया। उसने इससे पहले तुम्हारा नाम मुस्लिम (आज्ञाकारी) रखा था और इस ध्येय से - ताकि रसूल तुमपर गवाह हो और तुम लोगों पर गवाह हो। अतः नमाज़ का आयोजन करो और ज़कात दो और अल्लाह को मज़बूती से पकड़े रहो। वही तुम्हारा संरक्षक है। तो क्या ही अच्छा संरक्षक है और क्या ही अच्छा सहायक!
\end{hindi}}
\chapter{Al-Mu'minun (The Believers)}
\begin{Arabic}
\Huge{\centerline{\basmalah}}\end{Arabic}
\flushright{\begin{Arabic}
\quranayah[23][1]
\end{Arabic}}
\flushleft{\begin{hindi}
सफल हो गए ईमानवाले,
\end{hindi}}
\flushright{\begin{Arabic}
\quranayah[23][2]
\end{Arabic}}
\flushleft{\begin{hindi}
जो अपनी नमाज़ों में विनम्रता अपनाते है;
\end{hindi}}
\flushright{\begin{Arabic}
\quranayah[23][3]
\end{Arabic}}
\flushleft{\begin{hindi}
और जो व्यर्थ बातों से पहलू बचाते है;
\end{hindi}}
\flushright{\begin{Arabic}
\quranayah[23][4]
\end{Arabic}}
\flushleft{\begin{hindi}
और जो ज़कात अदा करते है;
\end{hindi}}
\flushright{\begin{Arabic}
\quranayah[23][5]
\end{Arabic}}
\flushleft{\begin{hindi}
और जो अपने गुप्तांगों की रक्षा करते है-
\end{hindi}}
\flushright{\begin{Arabic}
\quranayah[23][6]
\end{Arabic}}
\flushleft{\begin{hindi}
सिवाय इस सूरत के कि अपनी पत्नि यों या लौंडियों के पास जाएँ कि इसपर वे निन्दनीय नहीं है
\end{hindi}}
\flushright{\begin{Arabic}
\quranayah[23][7]
\end{Arabic}}
\flushleft{\begin{hindi}
परन्तु जो कोई इसके अतिरिक्त कुछ और चाहे तो ऐसे ही लोग सीमा उल्लंघन करनेवाले है।-
\end{hindi}}
\flushright{\begin{Arabic}
\quranayah[23][8]
\end{Arabic}}
\flushleft{\begin{hindi}
और जो अपनी अमानतों और अपनी प्रतिज्ञा का ध्यान रखते है;
\end{hindi}}
\flushright{\begin{Arabic}
\quranayah[23][9]
\end{Arabic}}
\flushleft{\begin{hindi}
और जो अपनी नमाज़ों की रक्षा करते हैं;
\end{hindi}}
\flushright{\begin{Arabic}
\quranayah[23][10]
\end{Arabic}}
\flushleft{\begin{hindi}
वही वारिस होने वाले है
\end{hindi}}
\flushright{\begin{Arabic}
\quranayah[23][11]
\end{Arabic}}
\flushleft{\begin{hindi}
जो फ़िरदौस की विरासत पाएँगे। वे उसमें सदैव रहेंगे
\end{hindi}}
\flushright{\begin{Arabic}
\quranayah[23][12]
\end{Arabic}}
\flushleft{\begin{hindi}
हमने मनुष्य को मिट्टी के सत से बनाया
\end{hindi}}
\flushright{\begin{Arabic}
\quranayah[23][13]
\end{Arabic}}
\flushleft{\begin{hindi}
फिर हमने उसे एक सुरक्षित ठहरने की जगह टपकी हुई बूँद बनाकर रखा
\end{hindi}}
\flushright{\begin{Arabic}
\quranayah[23][14]
\end{Arabic}}
\flushleft{\begin{hindi}
फिर हमने उस बूँद को लोथड़े का रूप दिया; फिर हमने उस लोथड़े को बोटी का रूप दिया; फिर हमने उन हड्डियों पर मांस चढाया; फिर हमने उसे एक दूसरा ही सर्जन रूप देकर खड़ा किया। अतः बहुत ही बरकतवाला है अल्लाह, सबसे उत्तम स्रष्टा!
\end{hindi}}
\flushright{\begin{Arabic}
\quranayah[23][15]
\end{Arabic}}
\flushleft{\begin{hindi}
फिर तुम अवश्य मरनेवाले हो
\end{hindi}}
\flushright{\begin{Arabic}
\quranayah[23][16]
\end{Arabic}}
\flushleft{\begin{hindi}
फिर क़ियामत के दिन तुम निश्चय ही उठाए जाओगे
\end{hindi}}
\flushright{\begin{Arabic}
\quranayah[23][17]
\end{Arabic}}
\flushleft{\begin{hindi}
और हमने तुम्हारे ऊपर सात रास्ते बनाए है। और हम सृष्टि-कार्य से ग़ाफ़िल नहीं
\end{hindi}}
\flushright{\begin{Arabic}
\quranayah[23][18]
\end{Arabic}}
\flushleft{\begin{hindi}
और हमने आकाश से एक अंदाज़े के साथ पानी उतारा। फिर हमने उसे धरती में ठहरा दिया, और उसे विलुप्त करने की सामर्थ्य भी हमें प्राप्त है
\end{hindi}}
\flushright{\begin{Arabic}
\quranayah[23][19]
\end{Arabic}}
\flushleft{\begin{hindi}
फिर हमने उसके द्वारा तुम्हारे लिए खजूरो और अंगूरों के बाग़ पैदा किए। तुम्हारे लिए उनमें बहुत-से फल है (जिनमें तुम्हारे लिए कितने ही लाभ है) और उनमें से तुम खाते हो
\end{hindi}}
\flushright{\begin{Arabic}
\quranayah[23][20]
\end{Arabic}}
\flushleft{\begin{hindi}
और वह वृक्ष भी जो सैना पर्वत से निकलता है, जो तेल और खानेवालों के लिए सालन लिए हुए उगता है
\end{hindi}}
\flushright{\begin{Arabic}
\quranayah[23][21]
\end{Arabic}}
\flushleft{\begin{hindi}
और निश्चय ही तुम्हारे लिए चौपायों में भी एक शिक्षा है। उनके पेटों में जो कुछ है उसमें से हम तुम्हें पिलाते है। औऱ तुम्हारे लिए उनमें बहुत-से फ़ायदे है और उन्हें तुम खाते भी हो
\end{hindi}}
\flushright{\begin{Arabic}
\quranayah[23][22]
\end{Arabic}}
\flushleft{\begin{hindi}
और उनपर और नौकाओं पर तुम सवार होते हो
\end{hindi}}
\flushright{\begin{Arabic}
\quranayah[23][23]
\end{Arabic}}
\flushleft{\begin{hindi}
हमने नूह को उसकी क़ौम की ओर भेजा तो उसने कहा, "ऐ मेरी क़ौम के लोगो! अल्लाह की बन्दगी करो। उसके सिवा तुम्हारा और कोई इष्ट-पूज्य नहीं है तो क्या तुम डर नहीं रखते?"
\end{hindi}}
\flushright{\begin{Arabic}
\quranayah[23][24]
\end{Arabic}}
\flushleft{\begin{hindi}
इसपर उनकी क़ौम के सरदार, जिन्होंने इनकार किया था, कहने लगे, "यह तो बस तुम्हीं जैसा एक मनुष्य है। चाहता है कि तुमपर श्रेष्ठता प्राप्त करे।""अल्लाह यदि चाहता तो फ़रिश्ते उतार देता। यह बात तो हमने अपने अगले बाप-दादा के समयों से सुनी ही नहीं
\end{hindi}}
\flushright{\begin{Arabic}
\quranayah[23][25]
\end{Arabic}}
\flushleft{\begin{hindi}
यह तो बस एक उन्मादग्रस्त व्यक्ति है। अतः एक समय तक इसकी प्रतीक्षा कर लो।"
\end{hindi}}
\flushright{\begin{Arabic}
\quranayah[23][26]
\end{Arabic}}
\flushleft{\begin{hindi}
उसने कहा, "ऐ मेरे रब! इन्होंने मुझे जो झुठलाया है, इसपर तू मेरी सहायता कर।"
\end{hindi}}
\flushright{\begin{Arabic}
\quranayah[23][27]
\end{Arabic}}
\flushleft{\begin{hindi}
तब हमने उसकी ओर प्रकाशना की कि "हमारी आँखों के सामने और हमारी प्रकाशना के अनुसार नौका बना और फिर जब हमारा आदेश आ जाए और तूफ़ान उमड़ पड़े तो प्रत्येक प्रजाति में से एक-एक जोड़ा उसमें रख ले और अपने लोगों को भी, सिवाय उनके जिनके विरुद्ध पहले फ़ैसला हो चुका है। और अत्याचारियों के विषय में मुझसे बात न करना। वे तो डूबकर रहेंगे
\end{hindi}}
\flushright{\begin{Arabic}
\quranayah[23][28]
\end{Arabic}}
\flushleft{\begin{hindi}
फिर जब तू नौका पर सवार हो जाए और तेरे साथी भी तो कह, प्रशंसा है अल्लाह की, जिसने हमें ज़ालिम लोगों से छुटकारा दिया
\end{hindi}}
\flushright{\begin{Arabic}
\quranayah[23][29]
\end{Arabic}}
\flushleft{\begin{hindi}
और कह, ऐ मेरे रब! मुझे बरकतवाली जगह उतार। और तू सबसे अच्छा मेज़बान है।"
\end{hindi}}
\flushright{\begin{Arabic}
\quranayah[23][30]
\end{Arabic}}
\flushleft{\begin{hindi}
निस्संदेह इसमें कितनी ही निशानियाँ हैं और परीक्षा तो हम करते ही है
\end{hindi}}
\flushright{\begin{Arabic}
\quranayah[23][31]
\end{Arabic}}
\flushleft{\begin{hindi}
फिर उनके पश्चात हमने एक दूसरी नस्ल को उठाया;
\end{hindi}}
\flushright{\begin{Arabic}
\quranayah[23][32]
\end{Arabic}}
\flushleft{\begin{hindi}
और उनमें हमने स्वयं उन्हीं में से एक रसूल भेजा कि "अल्लाह की बन्दगी करो। उसके सिवा तुम्हारा कोई इष्ट-पूज्य नहीं। तो क्या तुम डर नहीं रखते?"
\end{hindi}}
\flushright{\begin{Arabic}
\quranayah[23][33]
\end{Arabic}}
\flushleft{\begin{hindi}
उसकी क़ौम के सरदार, जिन्होंने इनकार किया और आख़िरत के मिलन को झूठलाया और जिन्हें हमने सांसारिक जीवन में सुख प्रदान किया था, कहने लगे, "यह तो बस तुम्हीं जैसा एक मनुष्य है। जो कुछ तुम खाते हो, वही यह भी खाता है और जो कुछ तुम पीते हो, वही यह भी पीता है
\end{hindi}}
\flushright{\begin{Arabic}
\quranayah[23][34]
\end{Arabic}}
\flushleft{\begin{hindi}
यदि तुम अपने ही जैसे एक मनुष्य के आज्ञाकारी हुए तो निश्चय ही तुम घाटे में पड़ गए
\end{hindi}}
\flushright{\begin{Arabic}
\quranayah[23][35]
\end{Arabic}}
\flushleft{\begin{hindi}
क्या यह तुमसे वादा करता है कि जब तुम मरकर मिट्टी और हड़्डियाँ होकर रह जाओगे तो तुम निकाले जाओगे?
\end{hindi}}
\flushright{\begin{Arabic}
\quranayah[23][36]
\end{Arabic}}
\flushleft{\begin{hindi}
दूर की बात है, बहुत दूर की, जिसका तुमसे वादा किया जा रहा है!
\end{hindi}}
\flushright{\begin{Arabic}
\quranayah[23][37]
\end{Arabic}}
\flushleft{\begin{hindi}
वह तो बस हमारा सांसारिक जीवन ही है। (यहीं) हम मरते और जीते है। हम कोई दोबारा उठाए जानेवाले नहीं है
\end{hindi}}
\flushright{\begin{Arabic}
\quranayah[23][38]
\end{Arabic}}
\flushleft{\begin{hindi}
वह तो बस एक ऐसा व्यक्ति है जिसने अल्लाह पर झूठ घड़ा है। हम उसे कदापि माननेवाले नहीं।"
\end{hindi}}
\flushright{\begin{Arabic}
\quranayah[23][39]
\end{Arabic}}
\flushleft{\begin{hindi}
उसने कहा, "ऐ मेरे रब! उन्होंने जो मुझे झुठलाया, उसपर तू मेरी सहायता कर।"
\end{hindi}}
\flushright{\begin{Arabic}
\quranayah[23][40]
\end{Arabic}}
\flushleft{\begin{hindi}
कहा, "शीघ्र ही वे पछताकर रहेंगे।"
\end{hindi}}
\flushright{\begin{Arabic}
\quranayah[23][41]
\end{Arabic}}
\flushleft{\begin{hindi}
फिर घटित होनेवाली बात के अनुसार उन्हें एक प्रचंड आवाज़ ने आ लिया और हमने उन्हें कूड़ा-कर्कट बनाकर रख दिया। अतः फिटकार है, ऐसे अत्याचारी लोगों पर!
\end{hindi}}
\flushright{\begin{Arabic}
\quranayah[23][42]
\end{Arabic}}
\flushleft{\begin{hindi}
फिर हमने उनके पश्चात दूसरी नस्लों को उठाया
\end{hindi}}
\flushright{\begin{Arabic}
\quranayah[23][43]
\end{Arabic}}
\flushleft{\begin{hindi}
कोई समुदाय न तो अपने निर्धारित समय से आगे बढ़ सकता है और न पीछे रह सकता है
\end{hindi}}
\flushright{\begin{Arabic}
\quranayah[23][44]
\end{Arabic}}
\flushleft{\begin{hindi}
फिर हमने निरन्तर अपने रसूल भेजे। जब भी किसी समुदाय के पास उसका रसूल आया, तो उसके लोगों ने उसे झुठला दिया। अतः हम एक दूसरे के पीछे (विनाश के लिए) लगाते चले गए और हमने उन्हें ऐसा कर दिया कि वे कहानियाँ होकर रह गए। फिटकार हो उन लोगों पर जो ईमान न लाएँ
\end{hindi}}
\flushright{\begin{Arabic}
\quranayah[23][45]
\end{Arabic}}
\flushleft{\begin{hindi}
फिर हमने मूसा और उसके भाई हारून को अपनी निशानियों और खुले प्रमाण के साथ फ़िरऔन और उसके सरदारों की ओर भेजा।
\end{hindi}}
\flushright{\begin{Arabic}
\quranayah[23][46]
\end{Arabic}}
\flushleft{\begin{hindi}
किन्तु उन्होंने अहंकार किया। वे थे ही सरकश लोग
\end{hindi}}
\flushright{\begin{Arabic}
\quranayah[23][47]
\end{Arabic}}
\flushleft{\begin{hindi}
तो व कहने लगे, "क्या हम अपने ही जैसे दो मनुष्यों की बात मान लें, जबकि उनकी क़ौम हमारी ग़ुलाम भी है?"
\end{hindi}}
\flushright{\begin{Arabic}
\quranayah[23][48]
\end{Arabic}}
\flushleft{\begin{hindi}
अतः उन्होंने उन दोनों को झुठला दिया और विनष्ट होनेवालों में सम्मिलित होकर रहे
\end{hindi}}
\flushright{\begin{Arabic}
\quranayah[23][49]
\end{Arabic}}
\flushleft{\begin{hindi}
और हमने मूसा को किताब प्रदान की, ताकि वे लोग मार्ग पा सकें
\end{hindi}}
\flushright{\begin{Arabic}
\quranayah[23][50]
\end{Arabic}}
\flushleft{\begin{hindi}
और मरयम के बेटे और उसकी माँ को हमने एक निशानी बनाया। और हमने उन्हें रहने योग्य स्रोतबाली ऊँची जगह शरण दी,
\end{hindi}}
\flushright{\begin{Arabic}
\quranayah[23][51]
\end{Arabic}}
\flushleft{\begin{hindi}
"ऐ पैग़म्बरो! अच्छी पाक चीज़े खाओ और अच्छा कर्म करो। जो कुछ तुम करते हो उसे मैं जानता हूँ
\end{hindi}}
\flushright{\begin{Arabic}
\quranayah[23][52]
\end{Arabic}}
\flushleft{\begin{hindi}
और निश्चय ही यह तुम्हारा समुदाय, एक ही समुदाय है और मैं तुम्हारा रब हूँ। अतः मेरा डर रखो।"
\end{hindi}}
\flushright{\begin{Arabic}
\quranayah[23][53]
\end{Arabic}}
\flushleft{\begin{hindi}
किन्तु उन्होंने स्वयं अपने मामले (धर्म) को परस्पर टुकड़े-टुकड़े कर डाला। हर गिरोह उसी पर खुश है, जो कुछ उसके पास है
\end{hindi}}
\flushright{\begin{Arabic}
\quranayah[23][54]
\end{Arabic}}
\flushleft{\begin{hindi}
अच्छा तो उन्हें उनकी अपनी बेहोशी में डूबे हुए ही एक समय तक छोड़ दो
\end{hindi}}
\flushright{\begin{Arabic}
\quranayah[23][55]
\end{Arabic}}
\flushleft{\begin{hindi}
क्या वे समझते है कि हम जो उनकी धन और सन्तान से सहायता किए जा रहे है,
\end{hindi}}
\flushright{\begin{Arabic}
\quranayah[23][56]
\end{Arabic}}
\flushleft{\begin{hindi}
तो यह उनके भलाइयों में कोई जल्दी कर रहे है?
\end{hindi}}
\flushright{\begin{Arabic}
\quranayah[23][57]
\end{Arabic}}
\flushleft{\begin{hindi}
नहीं, बल्कि उन्हें इसका एहसास नहीं है। निश्चय ही जो लोग अपने रब के भय से काँपते रहते हैं;
\end{hindi}}
\flushright{\begin{Arabic}
\quranayah[23][58]
\end{Arabic}}
\flushleft{\begin{hindi}
और जो लोग अपने रब की आयतों पर ईमान लाते है;
\end{hindi}}
\flushright{\begin{Arabic}
\quranayah[23][59]
\end{Arabic}}
\flushleft{\begin{hindi}
और जो लोग अपने रब के साथ किसी को साझी नहीं ठहराते;
\end{hindi}}
\flushright{\begin{Arabic}
\quranayah[23][60]
\end{Arabic}}
\flushleft{\begin{hindi}
और जो लोग देते है, जो कुछ देते है और हाल यह होता है कि दिल उनके काँप रहे होते है, इसलिए कि उन्हें अपने रब की ओर पलटना है;
\end{hindi}}
\flushright{\begin{Arabic}
\quranayah[23][61]
\end{Arabic}}
\flushleft{\begin{hindi}
यही वे लोग है, जो भलाइयों में जल्दी करते है और यही उनके लिए अग्रसर रहनेवाले है।
\end{hindi}}
\flushright{\begin{Arabic}
\quranayah[23][62]
\end{Arabic}}
\flushleft{\begin{hindi}
हम किसी व्यक्ति पर उसकी समाई (क्षमता) से बढ़कर ज़िम्मेदारी का बोझ नहीं डालते और हमारे पास एक किताब है, जो ठीक-ठीक बोलती है, और उनपर ज़ुल्म नहीं किया जाएगा
\end{hindi}}
\flushright{\begin{Arabic}
\quranayah[23][63]
\end{Arabic}}
\flushleft{\begin{hindi}
बल्कि उनके दिल इसकी (सत्य धर्म की) ओर से हटकर (वसवसों और गफ़लतों आदि के) भँवर में पडे हुए है और उससे (ईमानवालों की नीति से) हटकर उनके कुछ और ही काम है। वे उन्हीं को करते रहेंगे;
\end{hindi}}
\flushright{\begin{Arabic}
\quranayah[23][64]
\end{Arabic}}
\flushleft{\begin{hindi}
यहाँ तक कि जब हम उनके खुशहाल लोगों को यातना में पकड़ेगे तो क्या देखते है कि वे विलाप और फ़रियाद कर रहे है
\end{hindi}}
\flushright{\begin{Arabic}
\quranayah[23][65]
\end{Arabic}}
\flushleft{\begin{hindi}
(कहा जाएगा,) "आज चिल्लाओ मत, तुम्हें हमारी ओर से कोई सहायता मिलनेवाली नहीं
\end{hindi}}
\flushright{\begin{Arabic}
\quranayah[23][66]
\end{Arabic}}
\flushleft{\begin{hindi}
तुम्हें मेरी आयतें सुनाई जाती थीं, तो तुम अपनी एड़ियों के बल फिर जाते थे।
\end{hindi}}
\flushright{\begin{Arabic}
\quranayah[23][67]
\end{Arabic}}
\flushleft{\begin{hindi}
हाल यह था कि इसके कारण स्वयं को बड़ा समझते थे, उसे एक कहानी कहनेवाला ठहराकर छोड़ चलते थे
\end{hindi}}
\flushright{\begin{Arabic}
\quranayah[23][68]
\end{Arabic}}
\flushleft{\begin{hindi}
क्या उन्होंने इस वाणी पर विचार नहीं किया या उनके पास वह चीज़ आ गई जो उनके पहले बाप-दादा के पास न आई थी?
\end{hindi}}
\flushright{\begin{Arabic}
\quranayah[23][69]
\end{Arabic}}
\flushleft{\begin{hindi}
या उन्होंने अपने रसूल को पहचाना नहीं, इसलिए उसका इनकार कर रहे है?
\end{hindi}}
\flushright{\begin{Arabic}
\quranayah[23][70]
\end{Arabic}}
\flushleft{\begin{hindi}
या वे कहते है, "उसे उन्माद हो गया है।" नहीं, बल्कि वह उनके पास सत्य लेकर आया है। किन्तु उनमें अधिकांश को सत्य अप्रिय है
\end{hindi}}
\flushright{\begin{Arabic}
\quranayah[23][71]
\end{Arabic}}
\flushleft{\begin{hindi}
और यदि सत्य कहीं उनकी इच्छाओं के पीछे चलता तो समस्त आकाश और धरती और जो भी उनमें है, सबमें बिगाड़ पैदा हो जाता। नहीं, बल्कि हम उनके पास उनके हिस्से की अनुस्मृति लाए है। किन्तु वे अपनी अनुस्मृति से कतरा रहे है
\end{hindi}}
\flushright{\begin{Arabic}
\quranayah[23][72]
\end{Arabic}}
\flushleft{\begin{hindi}
या तुम उनसे कुथ शुल्क माँग रहे हो? तुम्हारे रब का दिया ही उत्तम है। और वह सबसे अच्छी रोज़ी देनेवाला है
\end{hindi}}
\flushright{\begin{Arabic}
\quranayah[23][73]
\end{Arabic}}
\flushleft{\begin{hindi}
और वास्तव में तुम उन्हें सीधे मार्ग की ओर बुला रहे हो
\end{hindi}}
\flushright{\begin{Arabic}
\quranayah[23][74]
\end{Arabic}}
\flushleft{\begin{hindi}
किन्तु जो लोग आख़िरत पर ईमान नहीं रखते वे इस मार्ग से हटकर चलना चाहते है
\end{hindi}}
\flushright{\begin{Arabic}
\quranayah[23][75]
\end{Arabic}}
\flushleft{\begin{hindi}
यदि हम (किसी आज़माइश में डालने के पश्चात) उनपर दया करते और जिस तकलीफ़ में वे होते उसे दूर कर देते तो भी वे अपनी सरकशी में हठात बहकते रहते
\end{hindi}}
\flushright{\begin{Arabic}
\quranayah[23][76]
\end{Arabic}}
\flushleft{\begin{hindi}
यद्यपि हमने उन्हें यातना में पकड़ा, फिर भी वे अपने रब के आगे न तो झुके और न वे गिड़गिड़ाते ही थे
\end{hindi}}
\flushright{\begin{Arabic}
\quranayah[23][77]
\end{Arabic}}
\flushleft{\begin{hindi}
यहाँ तक कि जब हम उनपर कठोर यातना का द्वार खोल दें तो क्या देखेंगे कि वे उसमें निराश होकर रह गए है
\end{hindi}}
\flushright{\begin{Arabic}
\quranayah[23][78]
\end{Arabic}}
\flushleft{\begin{hindi}
और वही है जिसने तुम्हारे लिए कान और आँखे और दिल बनाए। तुम कृतज्ञता थोड़े ही दिखाते हो!
\end{hindi}}
\flushright{\begin{Arabic}
\quranayah[23][79]
\end{Arabic}}
\flushleft{\begin{hindi}
वही है जिसने तुम्हें धरती में पैदा करके फैलाया और उसी की ओर तुम इकट्ठे होकर जाओगे
\end{hindi}}
\flushright{\begin{Arabic}
\quranayah[23][80]
\end{Arabic}}
\flushleft{\begin{hindi}
और वही है जो जीवन प्रदान करता और मृत्यु देता है और रात और दिन का उलट-फेर उसी के अधिकार में है। फिर क्या तुम बुद्धि से काम नहीं लेते?
\end{hindi}}
\flushright{\begin{Arabic}
\quranayah[23][81]
\end{Arabic}}
\flushleft{\begin{hindi}
नहीं, बल्कि वे लोग वहीं कुछ करते है जो उनके पहले के लोग कह चुके है
\end{hindi}}
\flushright{\begin{Arabic}
\quranayah[23][82]
\end{Arabic}}
\flushleft{\begin{hindi}
उन्होंने कहा, "क्या जब हम मरकर मिट्टी और हड्डियाँ होकर रह जाएँगे , तो क्या हमें दोबारा जीवित करके उठाया जाएगा?
\end{hindi}}
\flushright{\begin{Arabic}
\quranayah[23][83]
\end{Arabic}}
\flushleft{\begin{hindi}
यह वादा तो हमसे और इससे पहले हमारे बाप-दादा से होता आ रहा है। कुछ नहीं, यह तो बस अगलों की कहानियाँ है।"
\end{hindi}}
\flushright{\begin{Arabic}
\quranayah[23][84]
\end{Arabic}}
\flushleft{\begin{hindi}
कहो, "यह धरती और जो भी इसमें आबाद है, वे किसके है, बताओ यदि तुम जानते हो?"
\end{hindi}}
\flushright{\begin{Arabic}
\quranayah[23][85]
\end{Arabic}}
\flushleft{\begin{hindi}
वे बोल पड़ेगे, "अल्लाह के!" कहो, "फिर तुम होश में क्यों नहीं आते?"
\end{hindi}}
\flushright{\begin{Arabic}
\quranayah[23][86]
\end{Arabic}}
\flushleft{\begin{hindi}
कहो, "सातों आकाशों का मालिक और महान राजासन का स्वामीकौन है?"
\end{hindi}}
\flushright{\begin{Arabic}
\quranayah[23][87]
\end{Arabic}}
\flushleft{\begin{hindi}
वे कहेंगे, "सब अल्लाह के है।" कहो, "फिर डर क्यों नहीं रखते?"
\end{hindi}}
\flushright{\begin{Arabic}
\quranayah[23][88]
\end{Arabic}}
\flushleft{\begin{hindi}
कहो, "हर चीज़ की बादशाही किसके हाथ में है, वह जो शरण देता है औऱ जिसके मुक़ाबले में कोई शरण नहीं मिल सकती, बताओ यजि तुम जानते हो?"
\end{hindi}}
\flushright{\begin{Arabic}
\quranayah[23][89]
\end{Arabic}}
\flushleft{\begin{hindi}
वे बोल पड़ेगे, "अल्लाह की।" कहो, "फिर कहाँ से तुमपर जादू चल जाता है?"
\end{hindi}}
\flushright{\begin{Arabic}
\quranayah[23][90]
\end{Arabic}}
\flushleft{\begin{hindi}
नहीं, बल्कि हम उनके पास सत्य लेकर आए है और निश्चय ही वे झूठे है
\end{hindi}}
\flushright{\begin{Arabic}
\quranayah[23][91]
\end{Arabic}}
\flushleft{\begin{hindi}
अल्लाह ने अपना कोई बेटा नहीं बनाया और न उसके साथ कोई अन्य पूज्य-प्रभु है। ऐसा होता तो प्रत्येक पूज्य-प्रभु अपनी सृष्टि को लेकर अलग हो जाता और उनमें से एक-दूसरे पर चढ़ाई कर देता। महान और उच्च है अल्लाह उन बातों से, जो वे बयान करते है;
\end{hindi}}
\flushright{\begin{Arabic}
\quranayah[23][92]
\end{Arabic}}
\flushleft{\begin{hindi}
जाननेवाला है छुपे और खुले का। सो वह उच्चतर है वह शिर्क से जो वे करते है!
\end{hindi}}
\flushright{\begin{Arabic}
\quranayah[23][93]
\end{Arabic}}
\flushleft{\begin{hindi}
कहो, "ऐ मेरे रब! जिस चीज़ का वादा उनसे किया जा रहा है, वह यदि तू मुझे दिखाए
\end{hindi}}
\flushright{\begin{Arabic}
\quranayah[23][94]
\end{Arabic}}
\flushleft{\begin{hindi}
तो मेरे रब! मुझे उन अत्याचारी लोगों में सम्मिलित न करना।"
\end{hindi}}
\flushright{\begin{Arabic}
\quranayah[23][95]
\end{Arabic}}
\flushleft{\begin{hindi}
निश्चय ही हमें इसकी सामर्थ्य प्राप्त है कि हम उनसे जो वादा कर रहे है, वह तुम्हें दिखा दें।
\end{hindi}}
\flushright{\begin{Arabic}
\quranayah[23][96]
\end{Arabic}}
\flushleft{\begin{hindi}
बुराई को उस ढंग से दूर करो, जो सबसे उत्तम हो। हम भली-भाँति जानते है जो कुछ बातें वे बनाते है
\end{hindi}}
\flushright{\begin{Arabic}
\quranayah[23][97]
\end{Arabic}}
\flushleft{\begin{hindi}
और कहो, "ऐ मेरे रब! मैं शैतान की उकसाहटों से तेरी शरण चाहता हूँ
\end{hindi}}
\flushright{\begin{Arabic}
\quranayah[23][98]
\end{Arabic}}
\flushleft{\begin{hindi}
और मेरे रब! मैं इससे भी तेरी शरण चाहता हूँ कि वे मेरे पास आएँ।" -
\end{hindi}}
\flushright{\begin{Arabic}
\quranayah[23][99]
\end{Arabic}}
\flushleft{\begin{hindi}
यहाँ तक कि जब उनमें से किसी की मृत्यु आ गई तो वह कहेगा, "ऐ मेरे रब! मुझे लौटा दे। - ताकि जिस (संसार) को मैं छोड़ आया हूँ
\end{hindi}}
\flushright{\begin{Arabic}
\quranayah[23][100]
\end{Arabic}}
\flushleft{\begin{hindi}
उसमें अच्छा कर्म करूँ।" कुछ नहीं, यह तो बस एक (व्यर्थ) बात है जो वह कहेगा और उनके पीछे से लेकर उस दिन तक एक रोक लगी हुई है, जब वे दोबारा उठाए जाएँगे
\end{hindi}}
\flushright{\begin{Arabic}
\quranayah[23][101]
\end{Arabic}}
\flushleft{\begin{hindi}
फिर जब सूर (नरसिंघा) में फूँक मारी जाएगी तो उस दिन उनके बीच रिश्ते-नाते शेष न रहेंगे, और न वे एक-दूसरे को पूछेंगे
\end{hindi}}
\flushright{\begin{Arabic}
\quranayah[23][102]
\end{Arabic}}
\flushleft{\begin{hindi}
फिर जिनके पलड़े भारी हुए तॊ वही हैं जो सफल।
\end{hindi}}
\flushright{\begin{Arabic}
\quranayah[23][103]
\end{Arabic}}
\flushleft{\begin{hindi}
रहे वे लोग जिनके पलड़े हल्के हुए, तो वही है जिन्होंने अपने आपको घाटे में डाला। वे सदैव जहन्नम में रहेंगे
\end{hindi}}
\flushright{\begin{Arabic}
\quranayah[23][104]
\end{Arabic}}
\flushleft{\begin{hindi}
आग उनके चेहरों को झुलसा देगी और उसमें उनके मुँह विकृत हो रहे होंगे
\end{hindi}}
\flushright{\begin{Arabic}
\quranayah[23][105]
\end{Arabic}}
\flushleft{\begin{hindi}
(कहा जाएगा,) "क्या तुम्हें मेरी आयातें सुनाई नहीं जाती थी, तो तुम उन्हें झुठलाते थे?"
\end{hindi}}
\flushright{\begin{Arabic}
\quranayah[23][106]
\end{Arabic}}
\flushleft{\begin{hindi}
वे कहेंगे, "ऐ हमारे रब! हमारा दुर्भाग्य हमपर प्रभावी हुआ और हम भटके हुए लोग थे
\end{hindi}}
\flushright{\begin{Arabic}
\quranayah[23][107]
\end{Arabic}}
\flushleft{\begin{hindi}
हमारे रब! हमें यहाँ से निकाल दे! फिर हम दोबारा ऐसा करें तो निश्चय ही हम अत्याचारी होंगे।"
\end{hindi}}
\flushright{\begin{Arabic}
\quranayah[23][108]
\end{Arabic}}
\flushleft{\begin{hindi}
वह कहेगा, "फिटकारे हुए तिरस्कृत, इसी में पड़े रहो और मुझसे बात न करो
\end{hindi}}
\flushright{\begin{Arabic}
\quranayah[23][109]
\end{Arabic}}
\flushleft{\begin{hindi}
मेरे बन्दों में कुछ लोग थे, जो कहते थे, हमारे रब! हम ईमान ले आए। अतः तू हमें क्षमा कर दे और हमपर दया कर। तू सबसे अच्छा दया करनेवाला है
\end{hindi}}
\flushright{\begin{Arabic}
\quranayah[23][110]
\end{Arabic}}
\flushleft{\begin{hindi}
तो तुमने उनका उपहास किया, यहाँ तक कि उनके कारण तुम मेरी याद को भुला बैठे और तुम उनपर हँसते रहे
\end{hindi}}
\flushright{\begin{Arabic}
\quranayah[23][111]
\end{Arabic}}
\flushleft{\begin{hindi}
आज मैंने उनके धैर्य का यह बदला प्रदान किया कि वही है जो सफलता को प्राप्त हुए।"
\end{hindi}}
\flushright{\begin{Arabic}
\quranayah[23][112]
\end{Arabic}}
\flushleft{\begin{hindi}
वह कहेगाः “तुम धरती में कितने वर्ष रहे”?
\end{hindi}}
\flushright{\begin{Arabic}
\quranayah[23][113]
\end{Arabic}}
\flushleft{\begin{hindi}
वॆ कहेंगेः , "एक दिन या एक दिन का कुछ भाग। गणना करनेवालों से पूछ लीजिए।?"
\end{hindi}}
\flushright{\begin{Arabic}
\quranayah[23][114]
\end{Arabic}}
\flushleft{\begin{hindi}
वह कहेगा, "तुम ठहरे थोड़े ही, क्या अच्छा होता तुम जानते होते!
\end{hindi}}
\flushright{\begin{Arabic}
\quranayah[23][115]
\end{Arabic}}
\flushleft{\begin{hindi}
तो क्या तुमने यह समझा था कि हमने तुम्हें व्यर्थ पैदा किया है और यह कि तुम्हें हमारी और लौटना नहीं है?"
\end{hindi}}
\flushright{\begin{Arabic}
\quranayah[23][116]
\end{Arabic}}
\flushleft{\begin{hindi}
तो सर्वोच्च है अल्लाह, सच्चा सम्राट! उसके सिवा कोई पूज्य-प्रभु नहीं, स्वामी है महिमाशाली सिंहासन का
\end{hindi}}
\flushright{\begin{Arabic}
\quranayah[23][117]
\end{Arabic}}
\flushleft{\begin{hindi}
और जो कोई अल्लाह के साथ किसी दूसरे पूज्य को पुकारे, जिसके लिए उसके पास कोई प्रमाम नहीं, तो बस उसका हिसाब उसके रब के पास है। निश्चय ही इनकार करनेवाले कभी सफल नहीं होगे
\end{hindi}}
\flushright{\begin{Arabic}
\quranayah[23][118]
\end{Arabic}}
\flushleft{\begin{hindi}
और कहो, "मेरे रब! मुझे क्षमा कर दे और दया कर। तू तो सबसे अच्छा दया करनेवाला है।"
\end{hindi}}
\chapter{An-Nur (The Light)}
\begin{Arabic}
\Huge{\centerline{\basmalah}}\end{Arabic}
\flushright{\begin{Arabic}
\quranayah[24][1]
\end{Arabic}}
\flushleft{\begin{hindi}
यह एक (महत्वपूर्ण) सूरा है, जिसे हमने उतारा है। और इसे हमने अनिवार्य किया है, और इसमें हमने स्पष्ट आयतें (आदेश) अवतरित की है। कदाचित तुम शिक्षा ग्रहण करो
\end{hindi}}
\flushright{\begin{Arabic}
\quranayah[24][2]
\end{Arabic}}
\flushleft{\begin{hindi}
व्यभिचारिणी और व्यभिचारी - इन दोनों में से प्रत्येक को सौ कोड़े मारो और अल्लाह के धर्म (क़ानून) के विषय में तुम्हें उनपर तरस न आए, यदि तुम अल्लाह औऱ अन्तिम दिन को मानते हो। और उन्हें दंड देते समय मोमिनों में से कुछ लोगों को उपस्थित रहना चाहिए
\end{hindi}}
\flushright{\begin{Arabic}
\quranayah[24][3]
\end{Arabic}}
\flushleft{\begin{hindi}
व्यभिचारी किसी व्यभिचारिणी या बहुदेववादी स्त्री से ही निकाह करता है। और (इसी प्रकार) व्यभिचारिणी, किसी व्यभिचारी या बहुदेववादी से ही निकाह करते है। और यह मोमिनों पर हराम है
\end{hindi}}
\flushright{\begin{Arabic}
\quranayah[24][4]
\end{Arabic}}
\flushleft{\begin{hindi}
और जो लोग शरीफ़ और पाकदामन स्त्री पर तोहमत लगाएँ, फिर चार गवाह न लाएँ, उन्हें अस्सी कोड़े मारो और उनकी गवाही कभी भी स्वीकार न करो - वही है जो अवज्ञाकारी है। -
\end{hindi}}
\flushright{\begin{Arabic}
\quranayah[24][5]
\end{Arabic}}
\flushleft{\begin{hindi}
सिवाय उन लोगों के जो इसके पश्चात तौबा कर लें और सुधार कर लें। तो निश्चय ही अल्लाह बहुत क्षमाशील, अत्यन्त दयावान है
\end{hindi}}
\flushright{\begin{Arabic}
\quranayah[24][6]
\end{Arabic}}
\flushleft{\begin{hindi}
औऱ जो लोग अपनी पत्नियों पर दोषारोपण करें औऱ उनके पास स्वयं के सिवा गवाह मौजूद न हों, तो उनमें से एक (अर्थात पति) चार बार अल्लाह की क़सम खाकर यह गवाही दे कि वह बिलकुल सच्चा है
\end{hindi}}
\flushright{\begin{Arabic}
\quranayah[24][7]
\end{Arabic}}
\flushleft{\begin{hindi}
और पाँचवी बार यह गवाही दे कि यदि वह झूठा हो तो उसपर अल्लाह की फिटकार हो
\end{hindi}}
\flushright{\begin{Arabic}
\quranayah[24][8]
\end{Arabic}}
\flushleft{\begin{hindi}
पत्ऩी से भी सज़ा को यह बात टाल सकती है कि वह चार बार अल्लाह की क़सम खाकर गवाही दे कि वह बिलकुल झूठा है
\end{hindi}}
\flushright{\begin{Arabic}
\quranayah[24][9]
\end{Arabic}}
\flushleft{\begin{hindi}
और पाँचवी बार यह कहें कि उसपर (उस स्त्री पर) अल्लाह का प्रकोप हो, यदि वह सच्चा हो
\end{hindi}}
\flushright{\begin{Arabic}
\quranayah[24][10]
\end{Arabic}}
\flushleft{\begin{hindi}
यदि तुम अल्लाह की उदार कृपा और उसकी दया न होती (तो तुम संकट में पड़े जाते), और यह कि अल्लाह बड़ा तौबा क़बूल करनेवाला,अत्यन्त तत्वदर्शी है
\end{hindi}}
\flushright{\begin{Arabic}
\quranayah[24][11]
\end{Arabic}}
\flushleft{\begin{hindi}
जो लोग तोहमत घड़ लाए है वे तुम्हारे ही भीतर की एक टोली है। तुम उसे अपने लिए बुरा मत समझो, बल्कि वह भी तुम्हारे लिए अच्छा ही है। उनमें से प्रत्येक व्यक्ति के लिए उतना ही हिस्सा है जितना गुनाह उसने कमाया, और उनमें से जिस व्यक्ति ने उसकी ज़िम्मेदारी का एक बड़ा हिस्सा अपने सिर लिया उसके लिए बड़ा यातना है
\end{hindi}}
\flushright{\begin{Arabic}
\quranayah[24][12]
\end{Arabic}}
\flushleft{\begin{hindi}
ऐसा क्यों न हुआ कि जब तुम लोगों ने उसे सुना था, तब मोमिन पुरुष और मोमिन स्त्रियाँ अपने आपसे अच्छा गुमान करते और कहते कि "यह तो खुली तोहमत है?"
\end{hindi}}
\flushright{\begin{Arabic}
\quranayah[24][13]
\end{Arabic}}
\flushleft{\begin{hindi}
आख़िर वे इसपर चार गवाह क्यों न लाए? अब जबकि वे गवाह नहीं लाए, तो अल्लाह की स्पष्ट में वही झूठे है
\end{hindi}}
\flushright{\begin{Arabic}
\quranayah[24][14]
\end{Arabic}}
\flushleft{\begin{hindi}
यदि तुमपर दुनिया और आख़िरत में अल्लाह की उदार कृपा और उसकी दयालुता न होती तो जिस बात में तुम पड़ गए उसके कारण तुम्हें एक बड़ी यातना आ लेती
\end{hindi}}
\flushright{\begin{Arabic}
\quranayah[24][15]
\end{Arabic}}
\flushleft{\begin{hindi}
सोचो, जब तुम एक-दूसरे से उस (झूठ) को अपनी ज़बानों पर लेते जा रहे थे और तुम अपने मुँह से वह कुछ कहे जो रहे थे, जिसके विषय में तुम्हें कोई ज्ञान न था और तुम उसे एक साधारण बात समझ रहे थे; हालाँकि अल्लाह के निकट वह एक भारी बात थी
\end{hindi}}
\flushright{\begin{Arabic}
\quranayah[24][16]
\end{Arabic}}
\flushleft{\begin{hindi}
और ऐसा क्यों न हुआ कि जब तुमने उसे सुना था तो कह देते, "हमारे लिए उचित नहीं कि हम ऐसी बात ज़बान पर लाएँ। महान और उच्च है तू (अल्लाह)! यह तो एक बड़ी तोहमत है?"
\end{hindi}}
\flushright{\begin{Arabic}
\quranayah[24][17]
\end{Arabic}}
\flushleft{\begin{hindi}
अल्लाह तुम्हें नसीहत करता है कि फिर कभी ऐसा न करना, यदि तुम मोमिन हो
\end{hindi}}
\flushright{\begin{Arabic}
\quranayah[24][18]
\end{Arabic}}
\flushleft{\begin{hindi}
अल्लाह तो आयतों को तुम्हारे लिए खोल-खोलकर बयान करता है। अल्लाह तो सर्वज्ञ, तत्वदर्शी है
\end{hindi}}
\flushright{\begin{Arabic}
\quranayah[24][19]
\end{Arabic}}
\flushleft{\begin{hindi}
जो लोग चाहते है कि उन लोगों में जो ईमान लाए है, अश्लीहलता फैले, उनके लिए दुनिया और आख़िरत (लोक-परलोक) में दुखद यातना है। और अल्लाह बड़ा करुणामय, अत्यन्त दयावान है
\end{hindi}}
\flushright{\begin{Arabic}
\quranayah[24][20]
\end{Arabic}}
\flushleft{\begin{hindi}
और यदि तुमपर अल्लाह का उदार अनुग्रह और उसकी दयालुता न होती (तॊ अवश्य ही तुमपर यातना आ जाती) और यह कि अल्लाह बड़ा करुणामय, अत्यन्त दयावान है।
\end{hindi}}
\flushright{\begin{Arabic}
\quranayah[24][21]
\end{Arabic}}
\flushleft{\begin{hindi}
ऐ ईमान लानेवालो! शैतान के पद-चिन्हों पर न चलो। जो कोई शैतान के पद-चिन्हों पर चलेगा तो वह तो उसे अश्लीलता औऱ बुराई का आदेश देगा। और यदि अल्लाह का उदार अनुग्रह और उसकी दयालुता तुमपर न होती तो तुममें से कोई भी आत्म-विश्वास को प्राप्त न कर सकता। किन्तु अल्लाह जिसे चाहता है, सँवारता-निखारता है। अल्लाह तो सब कुछ सुनता, जानता है
\end{hindi}}
\flushright{\begin{Arabic}
\quranayah[24][22]
\end{Arabic}}
\flushleft{\begin{hindi}
तुममें जो बड़ाईवाले और सामर्थ्यवान है, वे नातेदारों, मुहताजों और अल्लाह की राह में घरबार छोड़नेवालों को देने से बाज़ रहने की क़सम न खा बैठें। उन्हें चाहिए कि क्षमा कर दें और उनसे दरगुज़र करें। क्या तुम यह नहीं चाहते कि अल्लाह तुम्हें क्षमा करें? अल्लाह बहुत क्षमाशील,अत्यन्त दयावान है
\end{hindi}}
\flushright{\begin{Arabic}
\quranayah[24][23]
\end{Arabic}}
\flushleft{\begin{hindi}
निस्संदेह जो लोग शरीफ़, पाकदामन, भोली-भाली बेख़बर ईमानवाली स्त्रियों पर तोहमत लगाते है उनपर दुनिया और आख़िरत में फिटकार है। और उनके लिए एक बड़ी यातना है
\end{hindi}}
\flushright{\begin{Arabic}
\quranayah[24][24]
\end{Arabic}}
\flushleft{\begin{hindi}
जिस दिन कि उनकी ज़बानें और उनके हाथ और उनके पाँव उनके विरुद्ध उसकी गवाही देंगे, जो कुछ वे करते रहे थे,
\end{hindi}}
\flushright{\begin{Arabic}
\quranayah[24][25]
\end{Arabic}}
\flushleft{\begin{hindi}
उस दिन अल्लाह उन्हें उनका ठीक बदला पूरी तरह दे देगा जिसके वे पात्र है। और वे जान लेंगे कि निस्संदेह अल्लाह ही सत्य है खुला हुआ, प्रकट कर देनेवाला
\end{hindi}}
\flushright{\begin{Arabic}
\quranayah[24][26]
\end{Arabic}}
\flushleft{\begin{hindi}
गन्दी चीज़े गन्दें लोगों के लिए है और गन्दे लोग गन्दी चीज़ों के लिए, और अच्छी चीज़ें अच्छे लोगों के लिए है और अच्छे लोग अच्छी चीज़ों के लिए। वे लोग उन बातों से बरी है, जो वे कह रहे है। उनके लिए क्षमा और सम्मानित आजीविका है
\end{hindi}}
\flushright{\begin{Arabic}
\quranayah[24][27]
\end{Arabic}}
\flushleft{\begin{hindi}
ऐ ईमान लानेवालो! अपने घरों के सिवा दूसरे घऱों में प्रवेश करो, जब तक कि रज़ामन्दी हासिल न कर लो और उन घरवालों को सलाम न कर लो। यही तुम्हारे लिए उत्तम है, कदाचित तुम ध्यान रखो
\end{hindi}}
\flushright{\begin{Arabic}
\quranayah[24][28]
\end{Arabic}}
\flushleft{\begin{hindi}
फिर यदि उनमें किसी को न पाओ, तो उनमें प्रवेश न करो जब तक कि तुम्हें अनुमति प्राप्त न हो। और यदि तुमसे कहा जाए कि वापस हो जाओ तो वापस हो जाओ, यही तुम्हारे लिए अधिक अच्छी बात है। अल्लाह भली-भाँति जानता है जो कुछ तुम करते हो
\end{hindi}}
\flushright{\begin{Arabic}
\quranayah[24][29]
\end{Arabic}}
\flushleft{\begin{hindi}
इसमें तुम्हारे लिए कोई दोष नहीं है कि तुम ऐसे घरों में प्रवेश करो जिनमें कोई न रहता हो, जिनमें तुम्हारे फ़ायदे की कोई चीज़ हो। और अल्लाह जानता है जो कुछ तुम प्रकट करते हो और जो कुछ छिपाते हो
\end{hindi}}
\flushright{\begin{Arabic}
\quranayah[24][30]
\end{Arabic}}
\flushleft{\begin{hindi}
ईमानवाले पुरुषों से कह दो कि अपनी निगाहें बचाकर रखें और अपने गुप्तांगों की रक्षा करें। यही उनके लिए अधिक अच्छी बात है। अल्लाह को उसकी पूरी ख़बर रहती है, जो कुछ वे किया करते है
\end{hindi}}
\flushright{\begin{Arabic}
\quranayah[24][31]
\end{Arabic}}
\flushleft{\begin{hindi}
और ईमानवाली स्त्रियों से कह दो कि वे भी अपनी निगाहें बचाकर रखें और अपने गुप्तांगों की रक्षा करें। और अपने शृंगार प्रकट न करें, सिवाय उसके जो उनमें खुला रहता है। और अपने सीनों (वक्षस्थल) पर अपने दुपट्टे डाल रहें और अपना शृंगार किसी पर ज़ाहिर न करें सिवाय अपने पतियों के या अपने बापों के या अपने पतियों के बापों के या अपने बेटों के या अपने पतियों के बेटों के या अपने भाइयों के या अपने भतीजों के या अपने भांजों के या मेल-जोल की स्त्रियों के या जो उनकी अपनी मिल्कियत में हो उनके, या उन अधीनस्थ पुरुषों के जो उस अवस्था को पार कर चुके हों जिससें स्त्री की ज़रूरत होती है, या उन बच्चों के जो स्त्रियों के परदे की बातों से परिचित न हों। और स्त्रियाँ अपने पाँव धरती पर मारकर न चलें कि अपना जो शृंगार छिपा रखा हो, वह मालूम हो जाए। ऐ ईमानवालो! तुम सब मिलकर अल्लाह से तौबा करो, ताकि तुम्हें सफलता प्राप्त हो
\end{hindi}}
\flushright{\begin{Arabic}
\quranayah[24][32]
\end{Arabic}}
\flushleft{\begin{hindi}
तुममें जो बेजोड़े के हों और तुम्हारे ग़ुलामों और तुम्हारी लौंडियों मे जो नेक और योग्य हों, उनका विवाह कर दो। यदि वे ग़रीब होंगे तो अल्लाह अपने उदार अनुग्रह से उन्हें समृद्ध कर देगा। अल्लाह बड़ी समाईवाला, सर्वज्ञ है
\end{hindi}}
\flushright{\begin{Arabic}
\quranayah[24][33]
\end{Arabic}}
\flushleft{\begin{hindi}
और जो विवाह का अवसर न पा रहे हो उन्हें चाहिए कि पाकदामनी अपनाए रहें, यहाँ तक कि अल्लाह अपने उदार अनुग्रह से उन्हें समृद्ध कर दे। और जिन लोगों पर तुम्हें स्वामित्व का अधिकार प्राप्त हो उनमें से जो लोग लिखा-पढ़ी के इच्छुक हो उनसे लिखा-पढ़ी कर लो, यदि तुम्हें मालूम हो कि उनमें भलाई है। और उन्हें अल्लाह के माल में से दो, जो उसने तुम्हें प्रदान किया है। और अपनी लौंडियों को सांसारिक जीवन-सामग्री की चाह में व्यविचार के लिए बाध्य न करो, जबकि वे पाकदामन रहना भी चाहती हों। औऱ इसके लिए जो कोई उन्हें बाध्य करेगा, तो निश्चय ही अल्लाह उनके बाध्य किए जाने के पश्चात अत्यन्त क्षमाशील, दयावान है
\end{hindi}}
\flushright{\begin{Arabic}
\quranayah[24][34]
\end{Arabic}}
\flushleft{\begin{hindi}
हमने तुम्हारी ओर खुली हुई आयतें उतार दी है और उन लोगों की मिशालें भी पेश कर दी हैं, जो तुमसे पहले गुज़रे है, और डर रखनेवालों के लिए नसीहत भी
\end{hindi}}
\flushright{\begin{Arabic}
\quranayah[24][35]
\end{Arabic}}
\flushleft{\begin{hindi}
अल्लाह आकाशों और धरती का प्रकाश है। (मोमिनों के दिल में) उसके प्रकाश की मिसाल ऐसी है जैसे एक ताक़ है, जिसमें एक चिराग़ है - वह चिराग़ एक फ़ानूस में है। वह फ़ानूस ऐसा है मानो चमकता हुआ कोई तारा है। - वह चिराग़ ज़ैतून के एक बरकतवाले वृक्ष के तेल से जलाया जाता है, जो न पूर्वी है न पश्चिमी। उसका तेल आप है आप भड़का पड़ता है, यद्यपि आग उसे न भी छुए। प्रकाश पर प्रकाश! - अल्लाह जिसे चाहता है अपने प्रकाश के प्राप्त होने का मार्ग दिखा देता है। अल्लाह लोगों के लिए मिशालें प्रस्तुत करता है। अल्लाह तो हर चीज़ जानता है।
\end{hindi}}
\flushright{\begin{Arabic}
\quranayah[24][36]
\end{Arabic}}
\flushleft{\begin{hindi}
उन घरों में जिनको ऊँचा करने और जिनमें अपने नाम के याद करने का अल्लाह ने हुक्म दिया है,
\end{hindi}}
\flushright{\begin{Arabic}
\quranayah[24][37]
\end{Arabic}}
\flushleft{\begin{hindi}
उनमें ऐसे लोग प्रभात काल और संध्या समय उसकी तसबीह करते है जिन्हें अल्लाह की याद और नमाज क़ायम करने और ज़कात देने से न तो व्यापार ग़ाफ़िल करता है और न क्रय-विक्रय। वे उस दिन से डरते रहते है जिसमें दिल और आँखें विकल हो जाएँगी
\end{hindi}}
\flushright{\begin{Arabic}
\quranayah[24][38]
\end{Arabic}}
\flushleft{\begin{hindi}
ताकि अल्लाह उन्हें बदला प्रदान करे। उनके अच्छे से अच्छे कामों का, और अपने उदार अनुग्रह से उन्हें और अधिक प्रदान करें। अल्लाह जिसे चाहता है बेहिसाब देता है
\end{hindi}}
\flushright{\begin{Arabic}
\quranayah[24][39]
\end{Arabic}}
\flushleft{\begin{hindi}
रहे वे लोग जिन्होंने इनकार किया उनके कर्म चटियल मैदान में मरीचिका की तरह है कि प्यासा उसे पानी समझता है, यहाँ तक कि जब वह उसके पास पहुँचा तो उसे कुछ भी न पाया। अलबत्ता अल्लाह ही को उसके पास पाया, जिसने उसका हिसाब पूरा-पूरा चुका दिया। और अल्लाह बहुत जल्द हिसाब करता है
\end{hindi}}
\flushright{\begin{Arabic}
\quranayah[24][40]
\end{Arabic}}
\flushleft{\begin{hindi}
या फिर जैसे एक गहरे समुद्र में अँधेरे, लहर के ऊपर लहर छा रही हैं; उसके ऊपर बादल है, अँधेरे है एक पर एक। जब वह अपना हाथ निकाले तो उसे वह सुझाई देता प्रतीत न हो। जिसे अल्लाह ही प्रकाश न दे फिर उसके लिए कोई प्रकाश नहीं
\end{hindi}}
\flushright{\begin{Arabic}
\quranayah[24][41]
\end{Arabic}}
\flushleft{\begin{hindi}
क्या तुमने नहीं देखा कि जो कोई भी आकाशों और धरती में है, अल्लाह की तसबीह (गुणगान) कर रहा है और पंख पसारे हुए पक्षी भी? हर एक अपनी नमाज़ और तसबीह से परिचित है। अल्लाह भली-भाँति जाना है जो कुछ वे करते है
\end{hindi}}
\flushright{\begin{Arabic}
\quranayah[24][42]
\end{Arabic}}
\flushleft{\begin{hindi}
अल्लाह ही के लिए है आकाशों और धरती का राज्य। और अल्लाह ही की ओर लौटकर जाना है
\end{hindi}}
\flushright{\begin{Arabic}
\quranayah[24][43]
\end{Arabic}}
\flushleft{\begin{hindi}
क्या तुमने देखा नहीं कि अल्लाह बादल को चलाता है। फिर उनको परस्पर मिलाता है। फिर उसे तह पर तह कर देता है। फिर तुम देखते हो कि उसके बीच से मेह बरसता है? और आकाश से- उसमें जो पहाड़ है (बादल जो पहाड़ जैसे प्रतीत होते है उनसे) - ओले बरसाता है। फिर जिस पर चाहता है, उसे हटा देता है। ऐसा प्रतीत होता है कि बिजली की चमक निगाहों को उचक ले जाएगी
\end{hindi}}
\flushright{\begin{Arabic}
\quranayah[24][44]
\end{Arabic}}
\flushleft{\begin{hindi}
अल्लाह ही रात और दिन का उलट-फेर करता है। निश्चय ही आँखें रखनेवालों के लिए इसमें एक शिक्षा है
\end{hindi}}
\flushright{\begin{Arabic}
\quranayah[24][45]
\end{Arabic}}
\flushleft{\begin{hindi}
अल्लाह ने हर जीवधारी को पानी से पैदा किया, तो उनमें से कोई अपने पेट के बल चलता है और कोई उनमें दो टाँगों पर चलता है और कोई चार (टाँगों) पर चलता है। अल्लाह जो चाहता है, पैदा करता है। निस्संदेह अल्लाह को हर चीज़ की सामर्थ्य प्राप्त है
\end{hindi}}
\flushright{\begin{Arabic}
\quranayah[24][46]
\end{Arabic}}
\flushleft{\begin{hindi}
हमने सत्य को प्रकट कर देनेवाली आयतें उतार दी है। आगे अल्लाह जिसे चाहता है सीधे मार्ग की ओर लगा देता है
\end{hindi}}
\flushright{\begin{Arabic}
\quranayah[24][47]
\end{Arabic}}
\flushleft{\begin{hindi}
वे मुनाफ़िक लोग कहते है कि "हम अल्लाह और रसूल पर ईमान लाए और हमने आज्ञापालन स्वीकार किया।" फिर इसके पश्चात उनमें से एक गिरोह मुँह मोड़ जाता है। ऐसे लोग मोमिन नहीं है
\end{hindi}}
\flushright{\begin{Arabic}
\quranayah[24][48]
\end{Arabic}}
\flushleft{\begin{hindi}
जब उन्हें अल्लाह और उसके रसूल की ओर बुलाया जाता है, ताकि वह उनके बीच फ़ैसला करें तो क्या देखते है कि उनमें से एक गिरोह कतरा जाता है;
\end{hindi}}
\flushright{\begin{Arabic}
\quranayah[24][49]
\end{Arabic}}
\flushleft{\begin{hindi}
किन्तु यदि हक़ उन्हें मिलनेवाला हो तो उसकी ओर बड़े आज्ञाकारी बनकर चले आएँ
\end{hindi}}
\flushright{\begin{Arabic}
\quranayah[24][50]
\end{Arabic}}
\flushleft{\begin{hindi}
क्या उनके दिलों में रोग है या वे सन्देह में पड़े हुए है या उनको यह डर है कि अल्लाह औऱ उसका रसूल उनके साथ अन्याय करेंगे? नहीं, बल्कि बात यह है कि वही लोग अत्याचारी हैं
\end{hindi}}
\flushright{\begin{Arabic}
\quranayah[24][51]
\end{Arabic}}
\flushleft{\begin{hindi}
मोमिनों की बात तो बस यह होती है कि जब अल्लाह और उसके रसूल की ओर बुलाए जाएँ, ताकि वह उनके बीच फ़ैसला करे, तो वे कहें, "हमने सुना और आज्ञापालन किया।" और वही सफलता प्राप्त करनेवाले हैं
\end{hindi}}
\flushright{\begin{Arabic}
\quranayah[24][52]
\end{Arabic}}
\flushleft{\begin{hindi}
और जो कोई अल्लाह और उसके रसूल का आज्ञा का पालन करे और अल्लाह से डरे और उसकी सीमाओं का ख़याल रखे, तो ऐसे ही लोग सफल है
\end{hindi}}
\flushright{\begin{Arabic}
\quranayah[24][53]
\end{Arabic}}
\flushleft{\begin{hindi}
वे अल्लाह की कड़ी-कड़ी क़समें खाते है कि यदि तुम उन्हें हुक्म दो तो वे अवश्य निकल खड़े होंगे। कह दो, "क़समें न खाओ। सामान्य नियम के अनुसार आज्ञापालन ही वास्तकिव चीज़ है। तुम जो कुछ करते हो अल्लाह उसकी ख़बर रखता है।"
\end{hindi}}
\flushright{\begin{Arabic}
\quranayah[24][54]
\end{Arabic}}
\flushleft{\begin{hindi}
कहो, "अल्लाह का आज्ञापालन करो और उसके रसूल का कहा मानो। परन्तु यदि तुम मुँह मोड़ते हो तो उसपर तो बस वही ज़िम्मेदारी है जिसका बोझ उसपर डाला गया है, और तुम उसके ज़िम्मेदार हो जिसका बोझ तुमपर डाला गया है। और यदि तुम आज्ञा का पालन करोगे तो मार्ग पा लोगे। और रसूल पर तो बस साफ़-साफ़ (संदेश) पहुँचा देने ही की ज़िम्मेदारी है
\end{hindi}}
\flushright{\begin{Arabic}
\quranayah[24][55]
\end{Arabic}}
\flushleft{\begin{hindi}
अल्लाह ने उन लोगों से जो तुममें से ईमान लाए और उन्होने अच्छे कर्म किए, वादा किया है कि वह उन्हें धरती में अवश्य सत्ताधिकार प्रदान करेगा, जैसे उसने उनसे पहले के लोगों को सत्ताधिकार प्रदान किया था। औऱ उनके लिए अवश्य उनके उस धर्म को जमाव प्रदान करेगा जिसे उसने उनके लिए पसन्द किया है। और निश्चय ही उनके वर्तमान भय के पश्चात उसे उनके लिए शान्ति और निश्चिन्तता में बदल देगा। वे मेरी बन्दगी करते है, मेरे साथ किसी चीज़ को साझी नहीं बनाते। और जो कोई इसके पश्चात इनकार करे, तो ऐसे ही लोग अवज्ञाकारी है
\end{hindi}}
\flushright{\begin{Arabic}
\quranayah[24][56]
\end{Arabic}}
\flushleft{\begin{hindi}
नमाज़ का आयोजन करो और ज़कात दो और रसूल की आज्ञा का पालन करो, ताकि तुमपर दया की जाए
\end{hindi}}
\flushright{\begin{Arabic}
\quranayah[24][57]
\end{Arabic}}
\flushleft{\begin{hindi}
यह कदापि न समझो कि इनकार की नीति अपनानेवाले धरती में क़ाबू से बाहर निकल जानेवाले है। उनका ठिकाना आग है, और वह बहुत ही बुरा ठिकाना है
\end{hindi}}
\flushright{\begin{Arabic}
\quranayah[24][58]
\end{Arabic}}
\flushleft{\begin{hindi}
ऐ ईमान लानेवालो! जो तुम्हारी मिल्कियत में हो और तुममें जो अभी युवावस्था को नहीं पहुँचे है, उनको चाहिए कि तीन समयों में तुमसे अनुमति लेकर तुम्हारे पास आएँ: प्रभात काल की नमाज़ से पहले और जब दोपहर को तुम (आराम के लिए) अपने कपड़े उतार रखते हो और रात्रि की नमाज़ के पश्चात - ये तीन समय तुम्हारे लिए परदे के हैं। इनके पश्चात न तो तुमपर कोई गुनाह है और न उनपर। वे तुम्हारे पास अधिक चक्कर लगाते है। तुम्हारे ही कुछ अंश परस्पर कुछ अंश के पास आकर मिलते है। इस प्रकार अल्लाह तुम्हारे लिए अपनी आयतों को स्पष्टप करता है। अल्लाह भली-भाँति जाननेवाला है, तत्वदर्शी है
\end{hindi}}
\flushright{\begin{Arabic}
\quranayah[24][59]
\end{Arabic}}
\flushleft{\begin{hindi}
और जब तुममें से बच्चे युवावस्था को पहुँच जाएँ तो उन्हें चाहिए कि अनुमति ले लिया करें जैसे उनसे पहले लोग अनुमति लेते रहे है। इस प्रकार अल्लाह तुम्हारे लिए अपनी आयतों को स्पष्ट करता है। अल्लाह भली-भाँति जाननेवाला, तत्वदर्शी है
\end{hindi}}
\flushright{\begin{Arabic}
\quranayah[24][60]
\end{Arabic}}
\flushleft{\begin{hindi}
जो स्त्रियाँ युवावस्था से गुज़रकर बैठ चुकी हों, जिन्हें विवाह की आशा न रह गई हो, उनपर कोई दोष नहीं कि वे अपने कपड़े (चादरें) उतारकर रख दें जबकि वे शृंगार का प्रदर्शन करनेवाली न हों। फिर भी वे इससे बचें तो उनके लिए अधिक अच्छा है। अल्लाह भली-भाँति सुनता, जानता है
\end{hindi}}
\flushright{\begin{Arabic}
\quranayah[24][61]
\end{Arabic}}
\flushleft{\begin{hindi}
न अंधे के लिए कोई हरज है, न लँगड़े के लिए कोई हरज है और न रोगी के लिए कोई हरज है और न तुम्हारे अपने लिए इस बात में कि तुम अपने घरों में खाओ या अपने बापों के घरों से या अपनी माँओ के घरों से या अपने भाइयों के घरों से या अपनी बहनों के घरों से या अपने चाचाओं के घरों से या अपनी फूफियों (बुआओं) के घरों से या अपनी ख़ालाओं के घरों से या जिसकी कुंजियों के मालिक हुए हो या अपने मित्र के यहाँ। इसमें तुम्हारे लिए कोई हरज नहीं कि तुम मिलकर खाओ या अलग-अलग। हाँ, अलबत्ता जब घरों में जाया करो तो अपने लोगों को सलाम किया करो, अभिवादन अल्लाह की ओर से नियत किया हुए, बरकतवाला और अत्याधिक पाक। इस प्रकार अल्लाह तुम्हारे लिए अपनी आयतों को स्पष्ट करता है, ताकि तुम बुद्धि से काम लो
\end{hindi}}
\flushright{\begin{Arabic}
\quranayah[24][62]
\end{Arabic}}
\flushleft{\begin{hindi}
मोमिन तो बस वही है जो अल्लाह और उसके रसूल पर पक्का ईमान रखते है। और जब किसी सामूहिक मामले के लिए उसके साथ हो तो चले न जाएँ जब तक कि उससे अनुमति न प्राप्त कर लें। (ऐ नबी!) जो लोग (आवश्यकता पड़ने पर) तुमसे अनुमति ले लेते है, वही लोग अल्लाह और रसूल पर ईमान रखते है, तो जब वे किसी काम के लिए अनुमति चाहें तो उनमें से जिसको चाहो अनुमति दे दिया करो, और उन लोगों के लिए अल्लाह से क्षमा की प्रार्थना किया करो। निस्संदेह अल्लाह बहुत क्षमाशील, अत्यन्त दयावान है
\end{hindi}}
\flushright{\begin{Arabic}
\quranayah[24][63]
\end{Arabic}}
\flushleft{\begin{hindi}
अपने बीच रसूल के बुलाने को तुम आपस में एक-दूसरे जैसा बुलाना न समझना। अल्लाह उन लोगों को भली-भाँति जानता है जो तुममें से ऐसे है कि (एक-दूसरे की) आड़ लेकर चुपके से खिसक जाते है। अतः उनको, जो उसके आदेश की अवहेलना करते है, डरना चाहिए कि कही ऐसा न हो कि उनपर कोई आज़माइश आ पड़े या उनपर कोई दुखद यातना आ जाए
\end{hindi}}
\flushright{\begin{Arabic}
\quranayah[24][64]
\end{Arabic}}
\flushleft{\begin{hindi}
सुन लो! आकाशों और धरती में जो कुछ भी है, अल्लाह का है। वह जानता है तुम जिस (नीति) पर हो। और जिस दिन वे उसकी ओर पलटेंगे, तो जो कुछ उन्होंने किया होगा, वह उन्हें बता देगा। अल्लाह तो हर चीज़ को जानता है
\end{hindi}}
\chapter{Al-Furqan (The Discrimination)}
\begin{Arabic}
\Huge{\centerline{\basmalah}}\end{Arabic}
\flushright{\begin{Arabic}
\quranayah[25][1]
\end{Arabic}}
\flushleft{\begin{hindi}
बड़ी बरकतवाला है वह जिसने यह फ़ुरक़ान अपने बन्दे पर अवतरित किया, ताकि वह सारे संसार के लिए सावधान करनेवाला हो
\end{hindi}}
\flushright{\begin{Arabic}
\quranayah[25][2]
\end{Arabic}}
\flushleft{\begin{hindi}
वह जिसका राज्य है आकाशों और धरती पर, और उसने न तो किसी को अपना बेटा बनाया और न राज्य में उसका कोई साझी है। उसने हर चीज़ को पैदा किया; फिर उसे ठीक अन्दाजें पर रखा
\end{hindi}}
\flushright{\begin{Arabic}
\quranayah[25][3]
\end{Arabic}}
\flushleft{\begin{hindi}
फिर भी उन्होंने उससे हटकर ऐसे इष्ट -पूज्य बना लिए जो किसी चीज़ को पैदा नहीं करते, बल्कि वे स्वयं पैदा किए जाते है। उन्हें न तो अपनी हानि का अधिकार प्राप्त है और न लाभ का। और न उन्हें मृत्यु का अधिकार प्राप्त है और न जीवन का और न दोबारा जीवित होकर उठने का
\end{hindi}}
\flushright{\begin{Arabic}
\quranayah[25][4]
\end{Arabic}}
\flushleft{\begin{hindi}
जिन लोगों ने इनकार किया उनका कहना है, "यह तो बस मनघड़ंत है जो उसने स्वयं ही घड़ लिया है। और कुछ दूसरे लोगों ने इस काम में उसकी सहायता की है।" वे तो ज़ुल्म और झूठ ही के ध्येय से आए
\end{hindi}}
\flushright{\begin{Arabic}
\quranayah[25][5]
\end{Arabic}}
\flushleft{\begin{hindi}
कहते है, "ये अगलों की कहानियाँ है, जिनको उसने लिख लिया है तो वही उसके पास प्रभात काल और सन्ध्या समय लिखाई जाती है।"
\end{hindi}}
\flushright{\begin{Arabic}
\quranayah[25][6]
\end{Arabic}}
\flushleft{\begin{hindi}
कहो, "उसे अवतरित किया है उसने, जो आकाशों और धरती के रहस्य जानता है। निश्चय ही वह बहुत क्षमाशील, अत्यन्त दयावान है।"
\end{hindi}}
\flushright{\begin{Arabic}
\quranayah[25][7]
\end{Arabic}}
\flushleft{\begin{hindi}
उनका यह भी कहना है, "इस रसूल को क्या हुआ कि यह खाना खाता है और बाज़ारों में चलता-फिरता है? क्यों न इसकी ओर कोई फ़रिश्ता उतरा कि वह इसके साथ रहकर सावधान करता?
\end{hindi}}
\flushright{\begin{Arabic}
\quranayah[25][8]
\end{Arabic}}
\flushleft{\begin{hindi}
या इसकी ओर कोई ख़ज़ाना ही डाल दिया जाता या इसके पास कोई बाग़ होता, जिससे यह खाता।" और इन ज़ालिमों का कहना है, "तुम लोग तो बस एक ऐसे व्यक्ति के पीछे चल रहे हो जो जादू का मारा हुआ है!"
\end{hindi}}
\flushright{\begin{Arabic}
\quranayah[25][9]
\end{Arabic}}
\flushleft{\begin{hindi}
देखों, उन्होंने तुमपर कैसी-कैसी फब्तियाँ कसीं। तो वे बहक गए है। अब उनमें इसकी सामर्थ्य नहीं कि कोई मार्ग पा सकें!
\end{hindi}}
\flushright{\begin{Arabic}
\quranayah[25][10]
\end{Arabic}}
\flushleft{\begin{hindi}
बरकतवाला है वह जो यदि चाहे तो तुम्हारे लिए इससे भी उत्तम प्रदान करे, बहत-से बाग़ जिनके नीचे नहरें बह रही हों, और तुम्हारे लिए बहुत-से महल तैया कर दे
\end{hindi}}
\flushright{\begin{Arabic}
\quranayah[25][11]
\end{Arabic}}
\flushleft{\begin{hindi}
नहीं, बल्कि बात यह है कि वे लोग क़ियामत की घड़ी को झुठला चुके है। और जो उस घड़ी को झुठला दे, उसके लिए दहकती आग तैयार कर रखी है
\end{hindi}}
\flushright{\begin{Arabic}
\quranayah[25][12]
\end{Arabic}}
\flushleft{\begin{hindi}
जब वह उनको दूर से देखेगी तो वे उसके बिफरने और साँस खींचने की आवाज़ें सुनेंगे
\end{hindi}}
\flushright{\begin{Arabic}
\quranayah[25][13]
\end{Arabic}}
\flushleft{\begin{hindi}
और जब वे उसकी किसी तंग जगह जकड़े हुए डाले जाएँगे, तो वहाँ विनाश को पुकारने लगेंगे
\end{hindi}}
\flushright{\begin{Arabic}
\quranayah[25][14]
\end{Arabic}}
\flushleft{\begin{hindi}
(कहा जाएगा,) "आज एक विनाश को मत पुकारो, बल्कि बहुत-से विनाशों को पुकारो!"
\end{hindi}}
\flushright{\begin{Arabic}
\quranayah[25][15]
\end{Arabic}}
\flushleft{\begin{hindi}
कहो, "यह अच्छा है या वह शाश्वत जन्नत, जिसका वादा डर रखनेवालों से किया गया है? यह उनका बदला और अन्तिम मंज़िल होगी।"
\end{hindi}}
\flushright{\begin{Arabic}
\quranayah[25][16]
\end{Arabic}}
\flushleft{\begin{hindi}
उनके लिए उसमें वह सबकुछ होगा, जो वे चाहेंगे। उसमें वे सदैव रहेंगे। यह तुम्हारे रब के ज़िम्मे एक ऐसा वादा है जो प्रार्थनीय है
\end{hindi}}
\flushright{\begin{Arabic}
\quranayah[25][17]
\end{Arabic}}
\flushleft{\begin{hindi}
और जिस दिन उन्हें इकट्ठा किया जाएगा और उनको भी जिन्हें वे अल्लाह को छोड़कर पूजते है, फिर वह कहेगा, "क्या मेरे बन्दों को तुमने पथभ्रष्ट किया था या वे स्वयं मार्ग छोड़ बैठे थे?"
\end{hindi}}
\flushright{\begin{Arabic}
\quranayah[25][18]
\end{Arabic}}
\flushleft{\begin{hindi}
वे कहेंगे, "महान और उच्च है तू! यह हमसे नहीं हो सकता था कि तुझे छोड़कर दूसरे संरक्षक बनाएँ। किन्तु हुआ यह कि तूने उन्हें औऱ उनके बाप-दादा को अत्यधिक सुख-सामग्री दी, यहाँ तक कि वे अनुस्मृति को भुला बैठे और विनष्ट होनेवाले लोग होकर रहे।"
\end{hindi}}
\flushright{\begin{Arabic}
\quranayah[25][19]
\end{Arabic}}
\flushleft{\begin{hindi}
अतः इस प्रकार वे तुम्हें उस बात में, जो तुम कहते हो झूठा ठहराए हुए है। अब न तो तुम यातना को फेर सकते हो और न कोई सहायता ही पा सकते हो। जो कोई तुममें से ज़ुल्म करे उसे हम बड़ी यातना का मज़ा चखाएँगे
\end{hindi}}
\flushright{\begin{Arabic}
\quranayah[25][20]
\end{Arabic}}
\flushleft{\begin{hindi}
और तुमसे पहले हमने जितने रसूल भी भेजे हैं, वे सब खाना खाते और बाज़ारों में चलते-फिरते थे। हमने तो तुम्हें परस्पर एक को दूसरे के लिए आज़माइश बना दिया है, "क्या तुम धैर्य दिखाते हो?" तुम्हारा रब तो सब कुछ देखता है
\end{hindi}}
\flushright{\begin{Arabic}
\quranayah[25][21]
\end{Arabic}}
\flushleft{\begin{hindi}
जिन्हें हमसे मिलने की आशंका नहीं, वे कहते है, "क्यों न फ़रिश्ते हमपर उतरे या फिर हम अपने रब को देखते?" उन्होंने अपने जी में बड़ा घमंज किया और बड़ी सरकशी पर उतर आए
\end{hindi}}
\flushright{\begin{Arabic}
\quranayah[25][22]
\end{Arabic}}
\flushleft{\begin{hindi}
जिस दिन वे फ़रिश्तों को देखेंगे उस दिन अपराधियों के लिए कोई ख़ुशख़बरी न होगी और वे पुकार उठेंगे, "पनाह! पनाह!!"
\end{hindi}}
\flushright{\begin{Arabic}
\quranayah[25][23]
\end{Arabic}}
\flushleft{\begin{hindi}
हम बढ़ेंगे उस कर्म की ओर जो उन्होंने किया होगा और उसे उड़ती धूल कर देंगे
\end{hindi}}
\flushright{\begin{Arabic}
\quranayah[25][24]
\end{Arabic}}
\flushleft{\begin{hindi}
उस दिन जन्नतवाले ठिकाने की दृष्टि से अच्छे होगे और आरामगाह की दृष्टि से भी अच्छे होंगे
\end{hindi}}
\flushright{\begin{Arabic}
\quranayah[25][25]
\end{Arabic}}
\flushleft{\begin{hindi}
उस दिन आकाश एक बादल के साथ फटेगा और फ़रिश्ते भली प्रकार उतारे जाएँगे
\end{hindi}}
\flushright{\begin{Arabic}
\quranayah[25][26]
\end{Arabic}}
\flushleft{\begin{hindi}
उस दिन वास्तविक राज्य रहमान का होगा और वह दिन इनकार करनेवालों के लिए बड़ा ही मुश्किल होगा
\end{hindi}}
\flushright{\begin{Arabic}
\quranayah[25][27]
\end{Arabic}}
\flushleft{\begin{hindi}
उस दिन अत्याचारी अत्याचारी अपने हाथ चबाएगा। कहेंगा, "ऐ काश! मैंने रसूल के साथ मार्ग अपनाया होता!
\end{hindi}}
\flushright{\begin{Arabic}
\quranayah[25][28]
\end{Arabic}}
\flushleft{\begin{hindi}
हाय मेरा दुर्भाग्य! काश, मैंने अमुक व्यक्ति को मित्र न बनाया होता!
\end{hindi}}
\flushright{\begin{Arabic}
\quranayah[25][29]
\end{Arabic}}
\flushleft{\begin{hindi}
उसने मुझे भटकाकर अनुस्मृति से विमुख कर दिया, इसके पश्चात कि वह मेरे पास आ चुकी थी। शैतान तो समय पर मनुष्य का साथ छोड़ ही देता है।"
\end{hindi}}
\flushright{\begin{Arabic}
\quranayah[25][30]
\end{Arabic}}
\flushleft{\begin{hindi}
रसूल कहेगा, "ऐ मेरे रब! निस्संदेह मेरी क़ौम के लोगों ने इस क़ुरआन को व्यर्थ बकवास की चीज़ ठहरा लिया था।"
\end{hindi}}
\flushright{\begin{Arabic}
\quranayah[25][31]
\end{Arabic}}
\flushleft{\begin{hindi}
और इसी तरह हमने अपराधियों में से प्रत्यॆक नबी के लिये शत्रु बनाया। मार्गदर्शन और सहायता कॆ लिए तॊ तुम्हारा रब ही काफ़ी है।
\end{hindi}}
\flushright{\begin{Arabic}
\quranayah[25][32]
\end{Arabic}}
\flushleft{\begin{hindi}
और जिन लोगों ने इनकार किया उनका कहना है कि "उसपर पूरा क़ुरआन एक ही बार में क्यों नहीं उतारा?" ऐसा इसलिए किया गया ताकि हम इसके द्वारा तुम्हारे दिल को मज़बूत रखें और हमने इसे एक उचित क्रम में रखा
\end{hindi}}
\flushright{\begin{Arabic}
\quranayah[25][33]
\end{Arabic}}
\flushleft{\begin{hindi}
और जब कभी भी वे तुम्हारे पास कोई आक्षेप की बात लेकर आएँगे तो हम तुम्हारे पास पक्की-सच्ची चीज़ लेकर आएँगे! इस दशा में कि वह स्पष्टीतकरण की स्पष्ट से उत्तम है
\end{hindi}}
\flushright{\begin{Arabic}
\quranayah[25][34]
\end{Arabic}}
\flushleft{\begin{hindi}
जो लोग औंधे मुँह जहन्नम की ओर ले जाए जाएँगे वही स्थान की दृष्टि से बहुत बुरे है, और मार्ग की दृष्टि से भी बहुत भटके हुए है
\end{hindi}}
\flushright{\begin{Arabic}
\quranayah[25][35]
\end{Arabic}}
\flushleft{\begin{hindi}
हमने मूसा को किताब प्रदान की और उसके भाई हारून को सहायक के रूप में उसके साथ किया
\end{hindi}}
\flushright{\begin{Arabic}
\quranayah[25][36]
\end{Arabic}}
\flushleft{\begin{hindi}
और कहा कि "तुम दोनों उन लोगों के पास जाओ जिन्होंने हमारी आयतों को झुठलाया है।" अन्ततः हमने उन लोगों को विनष्ट करके रख दिया
\end{hindi}}
\flushright{\begin{Arabic}
\quranayah[25][37]
\end{Arabic}}
\flushleft{\begin{hindi}
और नूह की क़ौम को भी, जब उन्होंने रसूलों को झुठलाया तो हमने उन्हें डुबा दिया और लोगों के लिए उन्हें एक निशानी बना दिया, और उन ज़ालिमों के लिए हमने एक दुखद यातना तैयार कर रखी है
\end{hindi}}
\flushright{\begin{Arabic}
\quranayah[25][38]
\end{Arabic}}
\flushleft{\begin{hindi}
और आद और समूद और अर-रस्सवालों और उस बीच की बहुत-सी नस्लों को भी विनष्ट किया।
\end{hindi}}
\flushright{\begin{Arabic}
\quranayah[25][39]
\end{Arabic}}
\flushleft{\begin{hindi}
प्रत्येक के लिए हमने मिसालें बयान कीं। अन्ततः प्रत्येक को हमने पूरी तरह विध्वस्त कर दिया
\end{hindi}}
\flushright{\begin{Arabic}
\quranayah[25][40]
\end{Arabic}}
\flushleft{\begin{hindi}
और उस बस्ती पर से तो वे हो आए है जिसपर बुरी वर्षा बरसी; तो क्या वे उसे देखते नहीं रहे हैं? नहीं, बल्कि वे दोबारा जीवित होकर उठने की आशा ही नहीं रखते रहे है
\end{hindi}}
\flushright{\begin{Arabic}
\quranayah[25][41]
\end{Arabic}}
\flushleft{\begin{hindi}
वे जब भी तुम्हें देखते हैं तो तुम्हारा मज़ाक़ बना लेते हैं कि "क्या यही, जिसे अल्लाह ने रसूल बनाकर भेजा है?
\end{hindi}}
\flushright{\begin{Arabic}
\quranayah[25][42]
\end{Arabic}}
\flushleft{\begin{hindi}
इसने तो हमें भटकाकर हमको हमारे प्रभु-पूज्यों से फेर ही दिया होता, यदि हम उनपर मज़बूती से जम न गए होते।"
\end{hindi}}
\flushright{\begin{Arabic}
\quranayah[25][43]
\end{Arabic}}
\flushleft{\begin{hindi}
क्या तुमने उसको भी देखा, जिसने अपना प्रभु अपनी (तुच्छ) इच्छा को बना रखा है? तो क्या तुम उसका ज़िम्मा ले सकते हो
\end{hindi}}
\flushright{\begin{Arabic}
\quranayah[25][44]
\end{Arabic}}
\flushleft{\begin{hindi}
या तुम समझते हो कि उनमें अधिकतर सुनते और समझते है? वे तो बस चौपायों की तरह हैं, बल्कि उनसे भी अधिक पथभ्रष्ट!
\end{hindi}}
\flushright{\begin{Arabic}
\quranayah[25][45]
\end{Arabic}}
\flushleft{\begin{hindi}
क्या तुमने अपने रब को नहीं देखा कि कैसे फैलाई छाया? यदि चाहता तो उसे स्थिर रखता। फिर हमने सूर्य को उसका पता देनेवाला बनाया,
\end{hindi}}
\flushright{\begin{Arabic}
\quranayah[25][46]
\end{Arabic}}
\flushleft{\begin{hindi}
फिर हम उसको धीरे-धीरे अपनी ओर समेट लेते है
\end{hindi}}
\flushright{\begin{Arabic}
\quranayah[25][47]
\end{Arabic}}
\flushleft{\begin{hindi}
वही है जिसने रात्रि को तुम्हारे लिए वस्त्र और निद्रा को सर्वथा विश्राम एवं शान्ति बनाया और दिन को जी उठने का समय बनाया
\end{hindi}}
\flushright{\begin{Arabic}
\quranayah[25][48]
\end{Arabic}}
\flushleft{\begin{hindi}
और वही है जिसने अपनी दयालुता (वर्षा) के आगे-आगे हवाओं को शुभ सूचना बनाकर भेजता है, और हम ही आकाश से स्वच्छ जल उतारते है
\end{hindi}}
\flushright{\begin{Arabic}
\quranayah[25][49]
\end{Arabic}}
\flushleft{\begin{hindi}
ताकि हम उसके द्वारा निर्जीव भू-भाग को जीवन प्रदान करें और उसे अपने पैदा किए हुए बहुत-से चौपायों और मनुष्यों को पिलाएँ
\end{hindi}}
\flushright{\begin{Arabic}
\quranayah[25][50]
\end{Arabic}}
\flushleft{\begin{hindi}
उसे हमने उनके बीच विभिन्न ढ़ंग से पेश किया है, ताकि वे ध्यान दें। परन्तु अधिकतर लोगों ने इनकार और अकृतज्ञता के अतिरिक्त दूसरी नीति अपनाने से इनकार ही किया
\end{hindi}}
\flushright{\begin{Arabic}
\quranayah[25][51]
\end{Arabic}}
\flushleft{\begin{hindi}
यदि हम चाहते तो हर बस्ती में एक डरानेवाला भेज देते
\end{hindi}}
\flushright{\begin{Arabic}
\quranayah[25][52]
\end{Arabic}}
\flushleft{\begin{hindi}
अतः इनकार करनेवालों की बात न मानता और इस (क़ुरआन) के द्वारा उनसे जिहाद करो, बड़ा जिहाद! (जी तोड़ कोशिश)
\end{hindi}}
\flushright{\begin{Arabic}
\quranayah[25][53]
\end{Arabic}}
\flushleft{\begin{hindi}
वही है जिसने दो समुद्रों को मिलाया। यह स्वादिष्ट और मीठा है और यह खारी और कडुआ। और दोनों के बीच उसने एक परदा डाल दिया है और एक पृथक करनेवाली रोक रख दी है
\end{hindi}}
\flushright{\begin{Arabic}
\quranayah[25][54]
\end{Arabic}}
\flushleft{\begin{hindi}
और वही है जिसने पानी से एक मनुष्य पैदा किया। फिर उसे वंशगत सम्बन्धों और ससुराली रिश्तेवाला बनाया। तुम्हारा रब बड़ा ही सामर्थ्यवान है
\end{hindi}}
\flushright{\begin{Arabic}
\quranayah[25][55]
\end{Arabic}}
\flushleft{\begin{hindi}
अल्लाह से इतर वे उनको पूजते है जो न उन्हें लाभ पहुँचा सकते है और न ही उन्हें हानि पहुँचा सकते है। और ऊपर से यह भी कि इनकार करनेवाला अपने रब का विरोधी और उसके मुक़ाबले में दूसरों का सहायक बना हुआ है
\end{hindi}}
\flushright{\begin{Arabic}
\quranayah[25][56]
\end{Arabic}}
\flushleft{\begin{hindi}
और हमने तो तुमको शुभ-सूचना देनेवाला और सचेतकर्ता बनाकर भेजा है।
\end{hindi}}
\flushright{\begin{Arabic}
\quranayah[25][57]
\end{Arabic}}
\flushleft{\begin{hindi}
कह दो, "मैं इस काम पर तुमसे कोई बदला नहीं माँगता सिवाय इसके कि जो कोई चाहे अपने रब की ओर ले जानेवाला मार्ग अपना ले।"
\end{hindi}}
\flushright{\begin{Arabic}
\quranayah[25][58]
\end{Arabic}}
\flushleft{\begin{hindi}
और उस अल्लाह पर भरोसा करो जो जीवन्त और अमर है और उसका गुणगान करो। वह अपने बन्दों के गुनाहों की ख़बर रखने के लिए काफ़ी है
\end{hindi}}
\flushright{\begin{Arabic}
\quranayah[25][59]
\end{Arabic}}
\flushleft{\begin{hindi}
जिसने आकाशों और धरती को और जो कुछ उन दोनों के बीच है छह दिनों में पैदा किया, फिर सिंहासन पर विराजमान हुआ। रहमान है वह! अतः पूछो उससे जो उसकी ख़बर रखता है
\end{hindi}}
\flushright{\begin{Arabic}
\quranayah[25][60]
\end{Arabic}}
\flushleft{\begin{hindi}
उन लोगों से जब कहा जाता है कि "रहमान को सजदा करो" तो वे कहते है, "और रहमान क्या होता है? क्या जिसे तू हमसे कह दे उसी को हम सजदा करने लगें?" और यह चीज़ उनकी घृणा को और बढ़ा देती है
\end{hindi}}
\flushright{\begin{Arabic}
\quranayah[25][61]
\end{Arabic}}
\flushleft{\begin{hindi}
बड़ी बरकतवाला है वह, जिसने आकाश में बुर्ज (नक्षत्र) बनाए और उसमें एक चिराग़ और एक चमकता चाँद बनाया
\end{hindi}}
\flushright{\begin{Arabic}
\quranayah[25][62]
\end{Arabic}}
\flushleft{\begin{hindi}
और वही है जिसने रात और दिन को एक-दूसरे के पीछे आनेवाला बनाया, उस व्यक्ति के लिए (निशानी) जो चेतना चाहे या कृतज्ञ होना चाहे
\end{hindi}}
\flushright{\begin{Arabic}
\quranayah[25][63]
\end{Arabic}}
\flushleft{\begin{hindi}
रहमान के (प्रिय) बन्दें वहीं है जो धरती पर नम्रतापूर्वक चलते है और जब जाहिल उनके मुँह आएँ तो कह देते है, "तुमको सलाम!"
\end{hindi}}
\flushright{\begin{Arabic}
\quranayah[25][64]
\end{Arabic}}
\flushleft{\begin{hindi}
जो अपने रब के आगे सजदे में और खड़े रातें गुज़ारते है;
\end{hindi}}
\flushright{\begin{Arabic}
\quranayah[25][65]
\end{Arabic}}
\flushleft{\begin{hindi}
जो कहते है कि "ऐ हमारे रब! जहन्नम की यातना को हमसे हटा दे।" निश्चय ही उनकी यातना चिमटकर रहनेवाली है
\end{hindi}}
\flushright{\begin{Arabic}
\quranayah[25][66]
\end{Arabic}}
\flushleft{\begin{hindi}
निश्चय ही वह जगह ठहरने की दृष्टि! से भी बुरी है और स्थान की दृष्टि से भी
\end{hindi}}
\flushright{\begin{Arabic}
\quranayah[25][67]
\end{Arabic}}
\flushleft{\begin{hindi}
जो ख़र्च करते है तो न अपव्यय करते है और न ही तंगी से काम लेते है, बल्कि वे इनके बीच मध्यमार्ग पर रहते है
\end{hindi}}
\flushright{\begin{Arabic}
\quranayah[25][68]
\end{Arabic}}
\flushleft{\begin{hindi}
जो अल्लाह के साथ किसी दूसरे इष्ट-पूज्य को नहीं पुकारते और न नाहक़ किसी जीव को जिस (के क़त्ल) को अल्लाह ने हराम किया है, क़त्ल करते है। और न वे व्यभिचार करते है - जो कोई यह काम करे तो वह गुनाह के वबाल से दोचार होगा
\end{hindi}}
\flushright{\begin{Arabic}
\quranayah[25][69]
\end{Arabic}}
\flushleft{\begin{hindi}
क़ियामत के दिन उसकी यातना बढ़ती चली जाएगी॥ और वह उसी में अपमानित होकर स्थायी रूप से पड़ा रहेगा
\end{hindi}}
\flushright{\begin{Arabic}
\quranayah[25][70]
\end{Arabic}}
\flushleft{\begin{hindi}
सिवाय उसके जो पलट आया और ईमान लाया और अच्छा कर्म किया, तो ऐसे लोगों की बुराइयों को अल्लाह भलाइयों से बदल देगा। और अल्लाह है भी अत्यन्त क्षमाशील, दयावान
\end{hindi}}
\flushright{\begin{Arabic}
\quranayah[25][71]
\end{Arabic}}
\flushleft{\begin{hindi}
और जिसने तौबा की और अच्छा कर्म किया, तो निश्चय ही वह अल्लाह की ओर पलटता है, जैसा कि पलटने का हक़ है
\end{hindi}}
\flushright{\begin{Arabic}
\quranayah[25][72]
\end{Arabic}}
\flushleft{\begin{hindi}
जो किसी झूठ और असत्य में सम्मिलित नहीं होते और जब किसी व्यर्थ के कामों के पास से गुज़रते है, तो श्रेष्ठतापूर्वक गुज़र जाते है,
\end{hindi}}
\flushright{\begin{Arabic}
\quranayah[25][73]
\end{Arabic}}
\flushleft{\begin{hindi}
जो ऐसे हैं कि जब उनके रब की आयतों के द्वारा उन्हें याददिहानी कराई जाती है तो उन (आयतों) पर वे अंधे और बहरे होकर नहीं गिरते।
\end{hindi}}
\flushright{\begin{Arabic}
\quranayah[25][74]
\end{Arabic}}
\flushleft{\begin{hindi}
और जो कहते है, "ऐ हमारे रब! हमें हमारी अपनी पत्नियों और हमारी संतान से आँखों की ठंडक प्रदान कर और हमें डर रखनेवालों का नायक बना दे।"
\end{hindi}}
\flushright{\begin{Arabic}
\quranayah[25][75]
\end{Arabic}}
\flushleft{\begin{hindi}
यही वे लोग है जिन्हें, इसके बदले में कि वे जमे रहे, उच्च भवन प्राप्त होगा, तथा ज़िन्दाबाद और सलाम से उनका वहाँ स्वागत होगा
\end{hindi}}
\flushright{\begin{Arabic}
\quranayah[25][76]
\end{Arabic}}
\flushleft{\begin{hindi}
वहाँ वे सदैव रहेंगे। बहुत ही अच्छी है वह ठहरने की जगह और स्थान;
\end{hindi}}
\flushright{\begin{Arabic}
\quranayah[25][77]
\end{Arabic}}
\flushleft{\begin{hindi}
कह दो, "मेरे रब को तुम्हारी कोई परवाह नहीं अगर तुम (उसको) न पुकारो। अब जबकि तुम झुठला चुके हो, तो शीघ्र ही वह चीज़ चिमट जानेवाली होगी।"
\end{hindi}}
\chapter{Ash-Shu'ara' (The Poets)}
\begin{Arabic}
\Huge{\centerline{\basmalah}}\end{Arabic}
\flushright{\begin{Arabic}
\quranayah[26][1]
\end{Arabic}}
\flushleft{\begin{hindi}
ता॰ सीन॰ मीम॰
\end{hindi}}
\flushright{\begin{Arabic}
\quranayah[26][2]
\end{Arabic}}
\flushleft{\begin{hindi}
ये स्पष्ट किताब की आयतें है
\end{hindi}}
\flushright{\begin{Arabic}
\quranayah[26][3]
\end{Arabic}}
\flushleft{\begin{hindi}
शायद इसपर कि वे ईमान नहीं लाते, तुम अपने प्राण ही खो बैठोगे
\end{hindi}}
\flushright{\begin{Arabic}
\quranayah[26][4]
\end{Arabic}}
\flushleft{\begin{hindi}
यदि हम चाहें तो उनपर आकाश से एक निशानी उतार दें। फिर उनकी गर्दनें उसके आगे झुकी रह जाएँ
\end{hindi}}
\flushright{\begin{Arabic}
\quranayah[26][5]
\end{Arabic}}
\flushleft{\begin{hindi}
उनके पास रहमान की ओर से जो नवीन अनुस्मृति भी आती है, वे उससे मुँह फेर ही लेते है
\end{hindi}}
\flushright{\begin{Arabic}
\quranayah[26][6]
\end{Arabic}}
\flushleft{\begin{hindi}
अब जबकि वे झुठला चुके है, तो शीघ्र ही उन्हें उसकी हक़ीकत मालूम हो जाएगी, जिसका वे मज़ाक़ उड़ाते रहे है
\end{hindi}}
\flushright{\begin{Arabic}
\quranayah[26][7]
\end{Arabic}}
\flushleft{\begin{hindi}
क्या उन्होंने धरती को नहीं देखा कि हमने उसमें कितने ही प्रकार की उमदा चीज़ें पैदा की है?
\end{hindi}}
\flushright{\begin{Arabic}
\quranayah[26][8]
\end{Arabic}}
\flushleft{\begin{hindi}
निश्चय ही इसमें एक बड़ी निशानी है, इसपर भी उनमें से अधिकतर माननेवाले नहीं
\end{hindi}}
\flushright{\begin{Arabic}
\quranayah[26][9]
\end{Arabic}}
\flushleft{\begin{hindi}
और निश्चय ही तुम्हारा रब ही है जो बड़ा प्रभुत्वशाली, अत्यन्त दयावान है
\end{hindi}}
\flushright{\begin{Arabic}
\quranayah[26][10]
\end{Arabic}}
\flushleft{\begin{hindi}
और जबकि तुम्हारे रह ने मूसा को पुकारा कि "ज़ालिम लोगों के पास जा -
\end{hindi}}
\flushright{\begin{Arabic}
\quranayah[26][11]
\end{Arabic}}
\flushleft{\begin{hindi}
फ़िरऔन की क़ौम के पास - क्या वे डर नहीं रखते?"
\end{hindi}}
\flushright{\begin{Arabic}
\quranayah[26][12]
\end{Arabic}}
\flushleft{\begin{hindi}
उसने कहा, "ऐ मेरे रब! मुझे डर है कि वे मुझे झुठला देंगे,
\end{hindi}}
\flushright{\begin{Arabic}
\quranayah[26][13]
\end{Arabic}}
\flushleft{\begin{hindi}
और मेरा सीना घुटता है और मेरी ज़बान नहीं चलती। इसलिए हारून की ओर भी संदेश भेज दे
\end{hindi}}
\flushright{\begin{Arabic}
\quranayah[26][14]
\end{Arabic}}
\flushleft{\begin{hindi}
और मुझपर उनके यहाँ के एक गुनाह का बोझ भी है। इसलिए मैं डरता हूँ कि वे मुझे मार डालेंगे।"
\end{hindi}}
\flushright{\begin{Arabic}
\quranayah[26][15]
\end{Arabic}}
\flushleft{\begin{hindi}
कहा, "कदापि नहीं, तुम दोनों हमारी निशानियाँ लेकर जाओ। हम तुम्हारे साथ है, सुनने को मौजूद है
\end{hindi}}
\flushright{\begin{Arabic}
\quranayah[26][16]
\end{Arabic}}
\flushleft{\begin{hindi}
अतः तुम दोनो फ़िरऔन को पास जाओ और कहो कि हम सारे संसार के रब के भेजे हुए है
\end{hindi}}
\flushright{\begin{Arabic}
\quranayah[26][17]
\end{Arabic}}
\flushleft{\begin{hindi}
कि तू इसराईल की सन्तान को हमारे साथ जाने दे।"
\end{hindi}}
\flushright{\begin{Arabic}
\quranayah[26][18]
\end{Arabic}}
\flushleft{\begin{hindi}
(फ़िरऔन ने) कहा, "क्या हमने तुझे जबकि तू बच्चा था, अपने यहाँ पाला नहीं था? और तू अपनी अवस्था के कई वर्षों तक हमारे साथ रहा,
\end{hindi}}
\flushright{\begin{Arabic}
\quranayah[26][19]
\end{Arabic}}
\flushleft{\begin{hindi}
और तूने अपना वह काम किया, जो किया। तू बड़ा ही कृतघ्न है।"
\end{hindi}}
\flushright{\begin{Arabic}
\quranayah[26][20]
\end{Arabic}}
\flushleft{\begin{hindi}
कहा, ऐसा तो मुझसे उस समय हुआ जबकि मैं चूक गया था
\end{hindi}}
\flushright{\begin{Arabic}
\quranayah[26][21]
\end{Arabic}}
\flushleft{\begin{hindi}
फिर जब मुझे तुम्हारा भय हुआ तो मैं तुम्हारे यहाँ से भाग गया। फिर मेरे रब ने मुझे निर्णय-शक्ति प्रदान की और मुझे रसूलों में सम्मिलित किया
\end{hindi}}
\flushright{\begin{Arabic}
\quranayah[26][22]
\end{Arabic}}
\flushleft{\begin{hindi}
यही वह उदार अनुग्रह है जिसका रहमान तू मुझपर जताता है कि तूने इसराईल की सन्तान को ग़ुलाम बना रखा है।"
\end{hindi}}
\flushright{\begin{Arabic}
\quranayah[26][23]
\end{Arabic}}
\flushleft{\begin{hindi}
फ़िरऔन ने कहा, "और यह सारे संसार का रब क्या होता है?"
\end{hindi}}
\flushright{\begin{Arabic}
\quranayah[26][24]
\end{Arabic}}
\flushleft{\begin{hindi}
उसने कहा, "आकाशों और धरती का रब और जो कुछ इन दोनों का मध्य है उसका भी, यदि तुम्हें यकीन हो।"
\end{hindi}}
\flushright{\begin{Arabic}
\quranayah[26][25]
\end{Arabic}}
\flushleft{\begin{hindi}
उसने अपने आस-पासवालों से कहा, "क्या तुम सुनते नहीं हो?"
\end{hindi}}
\flushright{\begin{Arabic}
\quranayah[26][26]
\end{Arabic}}
\flushleft{\begin{hindi}
कहा, "तुम्हारा रब और तुम्हारे अगले बाप-दादा का रब।"
\end{hindi}}
\flushright{\begin{Arabic}
\quranayah[26][27]
\end{Arabic}}
\flushleft{\begin{hindi}
बोला, "निश्चय ही तुम्हारा यह रसूल, जो तुम्हारी ओर भेजा गया है, बिलकुल ही पागल है।"
\end{hindi}}
\flushright{\begin{Arabic}
\quranayah[26][28]
\end{Arabic}}
\flushleft{\begin{hindi}
उसने कहा, "पूर्व और पश्चिम का रब और जो कुछ उनके बीच है उसका भी, यदि तुम कुछ बुद्धि रखते हो।"
\end{hindi}}
\flushright{\begin{Arabic}
\quranayah[26][29]
\end{Arabic}}
\flushleft{\begin{hindi}
बोला, "यदि तूने मेरे सिवा किसी और को पूज्य एवं प्रभु बनाया, तो मैं तुझे बन्दी बनाकर रहूँगा।"
\end{hindi}}
\flushright{\begin{Arabic}
\quranayah[26][30]
\end{Arabic}}
\flushleft{\begin{hindi}
उसने कहा, "क्या यदि मैं तेरे पास एक स्पष्ट चीज़ ले आऊँ तब भी?"
\end{hindi}}
\flushright{\begin{Arabic}
\quranayah[26][31]
\end{Arabic}}
\flushleft{\begin{hindi}
बोलाः “अच्छा वह ले आ; यदि तू सच्चा है” ।
\end{hindi}}
\flushright{\begin{Arabic}
\quranayah[26][32]
\end{Arabic}}
\flushleft{\begin{hindi}
फिर उसने अपनी लाठी डाल दी, तो अचानक क्या देखते है कि वह एक प्रत्यक्ष अज़गर है
\end{hindi}}
\flushright{\begin{Arabic}
\quranayah[26][33]
\end{Arabic}}
\flushleft{\begin{hindi}
और उसने अपना हाथ बाहर खींचा तो फिर क्या देखते है कि वह देखनेवालों के सामने चमक रहा है
\end{hindi}}
\flushright{\begin{Arabic}
\quranayah[26][34]
\end{Arabic}}
\flushleft{\begin{hindi}
उसने अपने आस-पास के सरदारों से कहा, "निश्चय ही यह एक बड़ा ही प्रवीण जादूगर है
\end{hindi}}
\flushright{\begin{Arabic}
\quranayah[26][35]
\end{Arabic}}
\flushleft{\begin{hindi}
चाहता है कि अपने जादू से तुम्हें तुम्हारी अपनी भूमि से निकाल बाहर करें; तो अब तुम क्या कहते हो?"
\end{hindi}}
\flushright{\begin{Arabic}
\quranayah[26][36]
\end{Arabic}}
\flushleft{\begin{hindi}
उन्होंने कहा, "इसे और इसके भाई को अभी टाले रखिए, और एकत्र करनेवालों को नगरों में भेज दीजिए
\end{hindi}}
\flushright{\begin{Arabic}
\quranayah[26][37]
\end{Arabic}}
\flushleft{\begin{hindi}
कि वे प्रत्येक प्रवीण जादूगर को आपके पास ले आएँ।"
\end{hindi}}
\flushright{\begin{Arabic}
\quranayah[26][38]
\end{Arabic}}
\flushleft{\begin{hindi}
अतएव एक निश्चित दिन के नियत समय पर जादूगर एकत्र कर लिए गए
\end{hindi}}
\flushright{\begin{Arabic}
\quranayah[26][39]
\end{Arabic}}
\flushleft{\begin{hindi}
और लोगों से कहा गया, "क्या तुम भी एकत्र होते हो?"
\end{hindi}}
\flushright{\begin{Arabic}
\quranayah[26][40]
\end{Arabic}}
\flushleft{\begin{hindi}
कदाचित हम जादूगरों ही के अनुयायी रह जाएँ, यदि वे विजयी हुए
\end{hindi}}
\flushright{\begin{Arabic}
\quranayah[26][41]
\end{Arabic}}
\flushleft{\begin{hindi}
फिर जब जादूगर आए तो उन्होंने फ़िरऔन से कहा, "क्या हमारे लिए कोई प्रतिदान भी है, यदि हम प्रभावी रहे?"
\end{hindi}}
\flushright{\begin{Arabic}
\quranayah[26][42]
\end{Arabic}}
\flushleft{\begin{hindi}
उसने कहा, "हाँ, और निश्चित ही तुम तो उस समय निकटतम लोगों में से हो जाओगे।"
\end{hindi}}
\flushright{\begin{Arabic}
\quranayah[26][43]
\end{Arabic}}
\flushleft{\begin{hindi}
मूसा ने उनसे कहा, "डालो, जो कुछ तुम्हें डालना है।"
\end{hindi}}
\flushright{\begin{Arabic}
\quranayah[26][44]
\end{Arabic}}
\flushleft{\begin{hindi}
तब उन्होंने अपनी रस्सियाँ और लाठियाँ डाल दी और बोले, "फ़िरऔन के प्रताप से हम ही विजयी रहेंगे।"
\end{hindi}}
\flushright{\begin{Arabic}
\quranayah[26][45]
\end{Arabic}}
\flushleft{\begin{hindi}
फिर मूसा ने अपनी लाठी फेकी तो क्या देखते है कि वह उसे स्वाँग को, जो वे रचाते है, निगलती जा रही है
\end{hindi}}
\flushright{\begin{Arabic}
\quranayah[26][46]
\end{Arabic}}
\flushleft{\begin{hindi}
इसपर जादूगर सजदे में गिर पड़े
\end{hindi}}
\flushright{\begin{Arabic}
\quranayah[26][47]
\end{Arabic}}
\flushleft{\begin{hindi}
वे बोल उठे, "हम सारे संसार के रब पर ईमान ले आए -
\end{hindi}}
\flushright{\begin{Arabic}
\quranayah[26][48]
\end{Arabic}}
\flushleft{\begin{hindi}
मूसा और हारून के रब पर!"
\end{hindi}}
\flushright{\begin{Arabic}
\quranayah[26][49]
\end{Arabic}}
\flushleft{\begin{hindi}
उसने कहा, "तुमने उसको मान लिया, इससे पहले कि मैं तुम्हें अनुमति देता। निश्चय ही वह तुम सबका प्रमुख है, जिसने तुमको जादू सिखाया है। अच्छा, शीघ्र ही तुम्हें मालूम हुआ जाता है! मैं तुम्हारे हाथ और पाँव विपरीत दिशाओं से कटवा दूँगा और तुम सभी को सूली पर चढ़ा दूँगा।"
\end{hindi}}
\flushright{\begin{Arabic}
\quranayah[26][50]
\end{Arabic}}
\flushleft{\begin{hindi}
उन्होंने कहा, "कुछ हरज नहीं; हम तो अपने रब ही की ओर पलटकर जानेवाले है
\end{hindi}}
\flushright{\begin{Arabic}
\quranayah[26][51]
\end{Arabic}}
\flushleft{\begin{hindi}
हमें तो इसी की लालसा है कि हमारा रब हमारी ख़ताओं को क्षमा कर दें, क्योंकि हम सबसे पहले ईमान लाए।"
\end{hindi}}
\flushright{\begin{Arabic}
\quranayah[26][52]
\end{Arabic}}
\flushleft{\begin{hindi}
हमने मूसा की ओर प्रकाशना की, "मेरे बन्दों को लेकर रातों-रात निकल जा। निश्चय ही तुम्हारा पीछा किया जाएगा।"
\end{hindi}}
\flushright{\begin{Arabic}
\quranayah[26][53]
\end{Arabic}}
\flushleft{\begin{hindi}
इसपर फ़िरऔन ने एकत्र करनेवालों को नगर में भेजा
\end{hindi}}
\flushright{\begin{Arabic}
\quranayah[26][54]
\end{Arabic}}
\flushleft{\begin{hindi}
कि "यह गिरे-पड़े थोड़े लोगों का एक गिरोह है,
\end{hindi}}
\flushright{\begin{Arabic}
\quranayah[26][55]
\end{Arabic}}
\flushleft{\begin{hindi}
और ये हमें क्रुद्ध कर रहे है।
\end{hindi}}
\flushright{\begin{Arabic}
\quranayah[26][56]
\end{Arabic}}
\flushleft{\begin{hindi}
और हम चौकन्ना रहनेवाले लोग है।"
\end{hindi}}
\flushright{\begin{Arabic}
\quranayah[26][57]
\end{Arabic}}
\flushleft{\begin{hindi}
इस प्रकार हम उन्हें बाग़ों और स्रोतों
\end{hindi}}
\flushright{\begin{Arabic}
\quranayah[26][58]
\end{Arabic}}
\flushleft{\begin{hindi}
और ख़जानों और अच्छे स्थान से निकाल लाए
\end{hindi}}
\flushright{\begin{Arabic}
\quranayah[26][59]
\end{Arabic}}
\flushleft{\begin{hindi}
ऐसा ही हम करते है और इनका वारिस हमने इसराईल की सन्तान को बना दिया
\end{hindi}}
\flushright{\begin{Arabic}
\quranayah[26][60]
\end{Arabic}}
\flushleft{\begin{hindi}
सुबह-तड़के उन्होंने उनका पीछा किया
\end{hindi}}
\flushright{\begin{Arabic}
\quranayah[26][61]
\end{Arabic}}
\flushleft{\begin{hindi}
फिर जब दोनों गिरोहों ने एक-दूसरे को देख लिया तो मूसा के साथियों ने कहा, "हम तो पकड़े गए!"
\end{hindi}}
\flushright{\begin{Arabic}
\quranayah[26][62]
\end{Arabic}}
\flushleft{\begin{hindi}
उसने कहा, "कदापि नहीं, मेरे साथ मेरा रब है। वह अवश्य मेरा मार्गदर्शन करेगा।"
\end{hindi}}
\flushright{\begin{Arabic}
\quranayah[26][63]
\end{Arabic}}
\flushleft{\begin{hindi}
तब हमने मूसा की ओर प्रकाशना की, "अपनी लाठी सागर पर मार।"
\end{hindi}}
\flushright{\begin{Arabic}
\quranayah[26][64]
\end{Arabic}}
\flushleft{\begin{hindi}
और हम दूसरों को भी निकट ले आए
\end{hindi}}
\flushright{\begin{Arabic}
\quranayah[26][65]
\end{Arabic}}
\flushleft{\begin{hindi}
हमने मूसा को और उन सबको जो उसके साथ थे, बचा लिया
\end{hindi}}
\flushright{\begin{Arabic}
\quranayah[26][66]
\end{Arabic}}
\flushleft{\begin{hindi}
और दूसरों को डूबो दिया
\end{hindi}}
\flushright{\begin{Arabic}
\quranayah[26][67]
\end{Arabic}}
\flushleft{\begin{hindi}
निस्संदेह इसमें एक बड़ी निशानी है। इसपर भी उनमें से अधिकतर माननेवाले नहीं
\end{hindi}}
\flushright{\begin{Arabic}
\quranayah[26][68]
\end{Arabic}}
\flushleft{\begin{hindi}
और निश्चय ही तुम्हारा रब ही है जो बड़ा प्रभुत्वशाली, अत्यन्त दयावान है
\end{hindi}}
\flushright{\begin{Arabic}
\quranayah[26][69]
\end{Arabic}}
\flushleft{\begin{hindi}
और उन्हें इबराहीम का वृत्तान्त सुनाओ,
\end{hindi}}
\flushright{\begin{Arabic}
\quranayah[26][70]
\end{Arabic}}
\flushleft{\begin{hindi}
जबकि उसने अपने बाप और अपनी क़ौंम के लोगों से कहा, "तुम क्या पूजते हो?"
\end{hindi}}
\flushright{\begin{Arabic}
\quranayah[26][71]
\end{Arabic}}
\flushleft{\begin{hindi}
उन्होंने कहा, "हम बुतों की पूजा करते है, हम तो उन्हीं की सेवा में लगे रहेंगे।"
\end{hindi}}
\flushright{\begin{Arabic}
\quranayah[26][72]
\end{Arabic}}
\flushleft{\begin{hindi}
उसने कहा, "क्या ये तुम्हारी सुनते है, जब तुम पुकारते हो,
\end{hindi}}
\flushright{\begin{Arabic}
\quranayah[26][73]
\end{Arabic}}
\flushleft{\begin{hindi}
या ये तुम्हें कुछ लाभ या हानि पहुँचाते है?"
\end{hindi}}
\flushright{\begin{Arabic}
\quranayah[26][74]
\end{Arabic}}
\flushleft{\begin{hindi}
उन्होंने कहा, "नहीं, बल्कि हमने तो अपने बाप-दादा को ऐसा ही करते पाया है।"
\end{hindi}}
\flushright{\begin{Arabic}
\quranayah[26][75]
\end{Arabic}}
\flushleft{\begin{hindi}
उसने कहा, "क्या तुमने उनपर विचार भी किया कि जिन्हें तुम पूजते हो,
\end{hindi}}
\flushright{\begin{Arabic}
\quranayah[26][76]
\end{Arabic}}
\flushleft{\begin{hindi}
तुम और तुम्हारे पहले के बाप-दादा?
\end{hindi}}
\flushright{\begin{Arabic}
\quranayah[26][77]
\end{Arabic}}
\flushleft{\begin{hindi}
वे सब तो मेरे शत्रु है, सिवाय सारे संसार के रब के,
\end{hindi}}
\flushright{\begin{Arabic}
\quranayah[26][78]
\end{Arabic}}
\flushleft{\begin{hindi}
जिसने मुझे पैदा किया और फिर वही मेरा मार्गदर्शन करता है
\end{hindi}}
\flushright{\begin{Arabic}
\quranayah[26][79]
\end{Arabic}}
\flushleft{\begin{hindi}
और वही है जो मुझे खिलाता और पिलाता है
\end{hindi}}
\flushright{\begin{Arabic}
\quranayah[26][80]
\end{Arabic}}
\flushleft{\begin{hindi}
और जब मैं बीमार होता हूँ, तो वही मुझे अच्छा करता है
\end{hindi}}
\flushright{\begin{Arabic}
\quranayah[26][81]
\end{Arabic}}
\flushleft{\begin{hindi}
और वही है जो मुझे मारेगा, फिर मुझे जीवित करेगा
\end{hindi}}
\flushright{\begin{Arabic}
\quranayah[26][82]
\end{Arabic}}
\flushleft{\begin{hindi}
और वही है जिससे मुझे इसकी आकांक्षा है कि बदला दिए जाने के दिन वह मेरी ख़ता माफ़ कर देगा
\end{hindi}}
\flushright{\begin{Arabic}
\quranayah[26][83]
\end{Arabic}}
\flushleft{\begin{hindi}
ऐ मेरे रब! मुझे निर्णय-शक्ति प्रदान कर और मुझे योग्य लोगों के साथ मिला।
\end{hindi}}
\flushright{\begin{Arabic}
\quranayah[26][84]
\end{Arabic}}
\flushleft{\begin{hindi}
और बाद के आनेवालों में से मुझे सच्ची ख़्याति प्रदान कर
\end{hindi}}
\flushright{\begin{Arabic}
\quranayah[26][85]
\end{Arabic}}
\flushleft{\begin{hindi}
और मुझे नेमत भरी जन्नत के वारिसों में सम्मिलित कर
\end{hindi}}
\flushright{\begin{Arabic}
\quranayah[26][86]
\end{Arabic}}
\flushleft{\begin{hindi}
और मेरे बाप को क्षमा कर दे। निश्चय ही वह पथभ्रष्ट लोगों में से है
\end{hindi}}
\flushright{\begin{Arabic}
\quranayah[26][87]
\end{Arabic}}
\flushleft{\begin{hindi}
और मुझे उस दिन रुसवा न कर, जब लोग जीवित करके उठाए जाएँगे।
\end{hindi}}
\flushright{\begin{Arabic}
\quranayah[26][88]
\end{Arabic}}
\flushleft{\begin{hindi}
जिस दिन न माल काम आएगा और न औलाद,
\end{hindi}}
\flushright{\begin{Arabic}
\quranayah[26][89]
\end{Arabic}}
\flushleft{\begin{hindi}
सिवाय इसके कि कोई भला-चंगा दिल लिए हुए अल्लाह के पास आया हो।"
\end{hindi}}
\flushright{\begin{Arabic}
\quranayah[26][90]
\end{Arabic}}
\flushleft{\begin{hindi}
और डर रखनेवालों के लिए जन्नत निकट लाई जाएगी
\end{hindi}}
\flushright{\begin{Arabic}
\quranayah[26][91]
\end{Arabic}}
\flushleft{\begin{hindi}
और भडकती आग पथभ्रष्टि लोगों के लिए प्रकट कर दी जाएगी
\end{hindi}}
\flushright{\begin{Arabic}
\quranayah[26][92]
\end{Arabic}}
\flushleft{\begin{hindi}
और उनसे कहा जाएगा, "कहाँ है वे जिन्हें तुम अल्लाह को छोड़कर पूजते रहे हो?
\end{hindi}}
\flushright{\begin{Arabic}
\quranayah[26][93]
\end{Arabic}}
\flushleft{\begin{hindi}
क्या वे तुम्हारी कुछ सहायता कर रहे है या अपना ही बचाव कर सकते है?"
\end{hindi}}
\flushright{\begin{Arabic}
\quranayah[26][94]
\end{Arabic}}
\flushleft{\begin{hindi}
फिर वे उसमें औंधे झोक दिए जाएँगे, वे और बहके हुए लोग
\end{hindi}}
\flushright{\begin{Arabic}
\quranayah[26][95]
\end{Arabic}}
\flushleft{\begin{hindi}
और इबलीस की सेनाएँ, सबके सब।
\end{hindi}}
\flushright{\begin{Arabic}
\quranayah[26][96]
\end{Arabic}}
\flushleft{\begin{hindi}
वे वहाँ आपस में झगड़ते हुए कहेंगे,
\end{hindi}}
\flushright{\begin{Arabic}
\quranayah[26][97]
\end{Arabic}}
\flushleft{\begin{hindi}
"अल्लाह की क़सम! निश्चय ही हम खुली गुमराही में थे
\end{hindi}}
\flushright{\begin{Arabic}
\quranayah[26][98]
\end{Arabic}}
\flushleft{\begin{hindi}
जबकि हम तुम्हें सारे संसार के रब के बराबर ठहरा रहे थे
\end{hindi}}
\flushright{\begin{Arabic}
\quranayah[26][99]
\end{Arabic}}
\flushleft{\begin{hindi}
और हमें तो बस उन अपराधियों ने ही पथभ्रष्ट किया
\end{hindi}}
\flushright{\begin{Arabic}
\quranayah[26][100]
\end{Arabic}}
\flushleft{\begin{hindi}
अब न हमारा कोई सिफ़ारिशी है,
\end{hindi}}
\flushright{\begin{Arabic}
\quranayah[26][101]
\end{Arabic}}
\flushleft{\begin{hindi}
और न घनिष्ट मित्र
\end{hindi}}
\flushright{\begin{Arabic}
\quranayah[26][102]
\end{Arabic}}
\flushleft{\begin{hindi}
क्या ही अच्छा होता कि हमें एक बार फिर पलटना होता, तो हम मोमिनों में से हो जाते!"
\end{hindi}}
\flushright{\begin{Arabic}
\quranayah[26][103]
\end{Arabic}}
\flushleft{\begin{hindi}
निश्चय ही इसमें एक बड़ी निशानी है। इसपर भी उनमें से अधिकरतर माननेवाले नहीं
\end{hindi}}
\flushright{\begin{Arabic}
\quranayah[26][104]
\end{Arabic}}
\flushleft{\begin{hindi}
और निस्संदेह तुम्हारा रब ही है जो बड़ा प्रभुत्वशाली, अत्यन्त दयावान है
\end{hindi}}
\flushright{\begin{Arabic}
\quranayah[26][105]
\end{Arabic}}
\flushleft{\begin{hindi}
नूह की क़ौम ने रसूलों को झुठलाया;
\end{hindi}}
\flushright{\begin{Arabic}
\quranayah[26][106]
\end{Arabic}}
\flushleft{\begin{hindi}
जबकि उनसे उनके भाई नूह ने कहा, "क्या तुम डर नहीं रखते?
\end{hindi}}
\flushright{\begin{Arabic}
\quranayah[26][107]
\end{Arabic}}
\flushleft{\begin{hindi}
निस्संदेह मैं तुम्हारे लिए एक अमानतदार रसूल हूँ
\end{hindi}}
\flushright{\begin{Arabic}
\quranayah[26][108]
\end{Arabic}}
\flushleft{\begin{hindi}
अतः अल्लाह का डर रखो और मेरा कहा मानो
\end{hindi}}
\flushright{\begin{Arabic}
\quranayah[26][109]
\end{Arabic}}
\flushleft{\begin{hindi}
मैं इस काम के बदले तुमसे कोई बदला नहीं माँगता। मेरा बदला तो बस सारे संसार के रब के ज़िम्मे है
\end{hindi}}
\flushright{\begin{Arabic}
\quranayah[26][110]
\end{Arabic}}
\flushleft{\begin{hindi}
अतः अल्लाह का डर रखो और मेरी आज्ञा का पालन करो।"
\end{hindi}}
\flushright{\begin{Arabic}
\quranayah[26][111]
\end{Arabic}}
\flushleft{\begin{hindi}
उन्होंने कहा, "क्या हम तेरी बात मान लें, जबकि तेरे पीछे तो अत्यन्त नीच लोग चल रहे है?"
\end{hindi}}
\flushright{\begin{Arabic}
\quranayah[26][112]
\end{Arabic}}
\flushleft{\begin{hindi}
उसने कहा, "मुझे क्या मालूम कि वे क्या करते रहे है?
\end{hindi}}
\flushright{\begin{Arabic}
\quranayah[26][113]
\end{Arabic}}
\flushleft{\begin{hindi}
उनका हिसाब तो बस मेरे रब के ज़िम्मे है। क्या ही अच्छा होता कि तुममें चेतना होती।
\end{hindi}}
\flushright{\begin{Arabic}
\quranayah[26][114]
\end{Arabic}}
\flushleft{\begin{hindi}
और मैं ईमानवालों को धुत्कारनेवाला नहीं हूँ।
\end{hindi}}
\flushright{\begin{Arabic}
\quranayah[26][115]
\end{Arabic}}
\flushleft{\begin{hindi}
मैं तो बस स्पष्ट रूप से एक सावधान करनेवाला हूँ।"
\end{hindi}}
\flushright{\begin{Arabic}
\quranayah[26][116]
\end{Arabic}}
\flushleft{\begin{hindi}
उन्होंने कहा, "यदि तू बाज़ न आया ऐ नूह, तो तू संगसार होकर रहेगा।"
\end{hindi}}
\flushright{\begin{Arabic}
\quranayah[26][117]
\end{Arabic}}
\flushleft{\begin{hindi}
उसने कहा, "ऐ मेरे रब! मेरी क़ौम के लोगों ने तो मुझे झुठला दिया
\end{hindi}}
\flushright{\begin{Arabic}
\quranayah[26][118]
\end{Arabic}}
\flushleft{\begin{hindi}
अब मेरे और उनके बीच दो टूक फ़ैसला कर दे और मुझे और जो ईमानवाले मेरे साथ है, उन्हें बचा ले!"
\end{hindi}}
\flushright{\begin{Arabic}
\quranayah[26][119]
\end{Arabic}}
\flushleft{\begin{hindi}
अतः हमने उसे और जो उसके साथ भरी हुई नौका में थे बचा लिया
\end{hindi}}
\flushright{\begin{Arabic}
\quranayah[26][120]
\end{Arabic}}
\flushleft{\begin{hindi}
और उसके पश्चात शेष लोगों को डूबो दिया
\end{hindi}}
\flushright{\begin{Arabic}
\quranayah[26][121]
\end{Arabic}}
\flushleft{\begin{hindi}
निश्चय ही इसमें एक बड़ी निशानी है। इसपर भी उनमें से अधिकतर माननेवाले नहीं
\end{hindi}}
\flushright{\begin{Arabic}
\quranayah[26][122]
\end{Arabic}}
\flushleft{\begin{hindi}
और निस्संदेह तुम्हारा रब ही है जो बड़ा प्रभुत्वशाली, अत्यन्त दयावान है
\end{hindi}}
\flushright{\begin{Arabic}
\quranayah[26][123]
\end{Arabic}}
\flushleft{\begin{hindi}
आद ने रसूलों को झूठलाया
\end{hindi}}
\flushright{\begin{Arabic}
\quranayah[26][124]
\end{Arabic}}
\flushleft{\begin{hindi}
जबकि उनके भाई हूद ने उनसे कहा, "क्या तुम डर नहीं रखते?
\end{hindi}}
\flushright{\begin{Arabic}
\quranayah[26][125]
\end{Arabic}}
\flushleft{\begin{hindi}
मैं तो तुम्हारे लिए एक अमानतदार रसूल हूँ
\end{hindi}}
\flushright{\begin{Arabic}
\quranayah[26][126]
\end{Arabic}}
\flushleft{\begin{hindi}
अतः तुम अल्लाह का डर रखो और मेरी आज्ञा मानो
\end{hindi}}
\flushright{\begin{Arabic}
\quranayah[26][127]
\end{Arabic}}
\flushleft{\begin{hindi}
मैं इस काम पर तुमसे कोई प्रतिदान नहीं माँगता। मेरा प्रतिदान तो बस सारे संसार के रब के ज़ि्म्मे है।
\end{hindi}}
\flushright{\begin{Arabic}
\quranayah[26][128]
\end{Arabic}}
\flushleft{\begin{hindi}
क्या तुम प्रत्येक उच्च स्थान पर व्यर्थ एक स्मारक का निर्माण करते रहोगे?
\end{hindi}}
\flushright{\begin{Arabic}
\quranayah[26][129]
\end{Arabic}}
\flushleft{\begin{hindi}
और भव्य महल बनाते रहोगे, मानो तुम्हें सदैव रहना है?
\end{hindi}}
\flushright{\begin{Arabic}
\quranayah[26][130]
\end{Arabic}}
\flushleft{\begin{hindi}
और जब किसी पर हाथ डालते हो तो बिलकुल निर्दय अत्याचारी बनकर हाथ डालते हो!
\end{hindi}}
\flushright{\begin{Arabic}
\quranayah[26][131]
\end{Arabic}}
\flushleft{\begin{hindi}
अतः अल्लाह का डर रखो और मेरी आज्ञा का पालन करो
\end{hindi}}
\flushright{\begin{Arabic}
\quranayah[26][132]
\end{Arabic}}
\flushleft{\begin{hindi}
उसका डर रखो जिसने तुम्हें वे चीज़े पहुँचाई जिनको तुम जानते हो
\end{hindi}}
\flushright{\begin{Arabic}
\quranayah[26][133]
\end{Arabic}}
\flushleft{\begin{hindi}
उसने तुम्हारी सहायता की चौपायों और बेटों से,
\end{hindi}}
\flushright{\begin{Arabic}
\quranayah[26][134]
\end{Arabic}}
\flushleft{\begin{hindi}
और बाग़ो और स्रोतो से
\end{hindi}}
\flushright{\begin{Arabic}
\quranayah[26][135]
\end{Arabic}}
\flushleft{\begin{hindi}
निश्चय ही मुझे तुम्हारे बारे में एक बड़े दिन की यातना का भय है।"
\end{hindi}}
\flushright{\begin{Arabic}
\quranayah[26][136]
\end{Arabic}}
\flushleft{\begin{hindi}
उन्होंने कहा, "हमारे लिए बराबर है चाहे तुम नसीहत करो या नसीहत करने वाले न बनो।
\end{hindi}}
\flushright{\begin{Arabic}
\quranayah[26][137]
\end{Arabic}}
\flushleft{\begin{hindi}
यह तो बस पहले लोगों की पुरानी आदत है
\end{hindi}}
\flushright{\begin{Arabic}
\quranayah[26][138]
\end{Arabic}}
\flushleft{\begin{hindi}
और हमें कदापि यातना न दी जाएगी।"
\end{hindi}}
\flushright{\begin{Arabic}
\quranayah[26][139]
\end{Arabic}}
\flushleft{\begin{hindi}
अन्ततः उन्होंने उन्हें झुठला दिया जो हमने उनको विनष्ट कर दिया। बेशक इसमें एक बड़ी निशानी है। इसपर भी उनमें से अधिकतर माननेवाले नहीं
\end{hindi}}
\flushright{\begin{Arabic}
\quranayah[26][140]
\end{Arabic}}
\flushleft{\begin{hindi}
और बेशक तुम्हारा रब ही है, जो बड़ा प्रभुत्वशाली, अत्यन्त दयावान है
\end{hindi}}
\flushright{\begin{Arabic}
\quranayah[26][141]
\end{Arabic}}
\flushleft{\begin{hindi}
समूद ने रसूलों को झुठलाया,
\end{hindi}}
\flushright{\begin{Arabic}
\quranayah[26][142]
\end{Arabic}}
\flushleft{\begin{hindi}
जबकि उसके भाई सालेह ने उससे कहा, "क्या तुम डर नहीं रखते?
\end{hindi}}
\flushright{\begin{Arabic}
\quranayah[26][143]
\end{Arabic}}
\flushleft{\begin{hindi}
निस्संदेह मैं तुम्हारे लिए एक अमानतदार रसूल हूँ
\end{hindi}}
\flushright{\begin{Arabic}
\quranayah[26][144]
\end{Arabic}}
\flushleft{\begin{hindi}
अतः तुम अल्लाह का डर रखो और मेरी बात मानो
\end{hindi}}
\flushright{\begin{Arabic}
\quranayah[26][145]
\end{Arabic}}
\flushleft{\begin{hindi}
मैं इस काम पर तुमसे कोई बदला नहीं माँगता। मेरा बदला तो बस सारे संसार के रब के ज़िम्मे है
\end{hindi}}
\flushright{\begin{Arabic}
\quranayah[26][146]
\end{Arabic}}
\flushleft{\begin{hindi}
क्या तुम यहाँ जो कुछ है उसके बीच, निश्चिन्त छोड़ दिए जाओगे,
\end{hindi}}
\flushright{\begin{Arabic}
\quranayah[26][147]
\end{Arabic}}
\flushleft{\begin{hindi}
बाग़ों और स्रोतों
\end{hindi}}
\flushright{\begin{Arabic}
\quranayah[26][148]
\end{Arabic}}
\flushleft{\begin{hindi}
और खेतों और उन खजूरों में जिनके गुच्छे तरो ताज़ा और गुँथे हुए है?
\end{hindi}}
\flushright{\begin{Arabic}
\quranayah[26][149]
\end{Arabic}}
\flushleft{\begin{hindi}
तुम पहाड़ों को काट-काटकर इतराते हुए घर बनाते हो?
\end{hindi}}
\flushright{\begin{Arabic}
\quranayah[26][150]
\end{Arabic}}
\flushleft{\begin{hindi}
अतः अल्लाह का डर रखो और मेरी आज्ञा का पालन करो
\end{hindi}}
\flushright{\begin{Arabic}
\quranayah[26][151]
\end{Arabic}}
\flushleft{\begin{hindi}
और उन हद से गुज़र जानेवालों की आज्ञा का पालन न करो,
\end{hindi}}
\flushright{\begin{Arabic}
\quranayah[26][152]
\end{Arabic}}
\flushleft{\begin{hindi}
जो धरती में बिगाड़ पैदा करते है, और सुधार का काम नहीं करते।"
\end{hindi}}
\flushright{\begin{Arabic}
\quranayah[26][153]
\end{Arabic}}
\flushleft{\begin{hindi}
उन्होंने कहा, "तू तो बस जादू का मारा हुआ है।
\end{hindi}}
\flushright{\begin{Arabic}
\quranayah[26][154]
\end{Arabic}}
\flushleft{\begin{hindi}
तू बस हमारे ही जैसा एक आदमी है। यदि तू सच्चा है, तो कोई निशानी ले आ।"
\end{hindi}}
\flushright{\begin{Arabic}
\quranayah[26][155]
\end{Arabic}}
\flushleft{\begin{hindi}
उसने कहा, "यह ऊँटनी है। एक दिन पानी पीने की बारी इसकी है और एक नियत दिन की बारी पानी लेने की तुम्हारी है
\end{hindi}}
\flushright{\begin{Arabic}
\quranayah[26][156]
\end{Arabic}}
\flushleft{\begin{hindi}
तकलीफ़ पहुँचाने के लिए इसे हाथ न लगाना, अन्यथा एक बड़े दिन की यातना तुम्हें आ लेगी।"
\end{hindi}}
\flushright{\begin{Arabic}
\quranayah[26][157]
\end{Arabic}}
\flushleft{\begin{hindi}
किन्तु उन्होंने उसकी कूचें काट दी। फिर पछताते रह गए
\end{hindi}}
\flushright{\begin{Arabic}
\quranayah[26][158]
\end{Arabic}}
\flushleft{\begin{hindi}
अन्ततः यातना ने उन्हें आ दबोचा। निश्चय ही इसमें एक बड़ी निशानी है। इसपर भी उनमें से अधिकतर माननेवाले नहीं
\end{hindi}}
\flushright{\begin{Arabic}
\quranayah[26][159]
\end{Arabic}}
\flushleft{\begin{hindi}
और निस्संदेह तुम्हारा रब ही है जो बड़ा प्रभुत्वशाली, अत्यन्त दयाशील है
\end{hindi}}
\flushright{\begin{Arabic}
\quranayah[26][160]
\end{Arabic}}
\flushleft{\begin{hindi}
लूत की क़ौम के लोगों ने रसूलों को झुठलाया;
\end{hindi}}
\flushright{\begin{Arabic}
\quranayah[26][161]
\end{Arabic}}
\flushleft{\begin{hindi}
जबकि उनके भाई लूत ने उनसे कहा, "क्या तुम डर नहीं रखते?
\end{hindi}}
\flushright{\begin{Arabic}
\quranayah[26][162]
\end{Arabic}}
\flushleft{\begin{hindi}
मैं तो तुम्हारे लिए एक अमानतदार रसूल हूँ
\end{hindi}}
\flushright{\begin{Arabic}
\quranayah[26][163]
\end{Arabic}}
\flushleft{\begin{hindi}
अतः अल्लाह का डर रखो और मेरी आज्ञा का पालन करो
\end{hindi}}
\flushright{\begin{Arabic}
\quranayah[26][164]
\end{Arabic}}
\flushleft{\begin{hindi}
मैं इस काम पर तुमसे कोई प्रतिदान नहीं माँगता, मेरा प्रतिदान तो बस सारे संसार के रब के ज़िम्मे है
\end{hindi}}
\flushright{\begin{Arabic}
\quranayah[26][165]
\end{Arabic}}
\flushleft{\begin{hindi}
क्या सारे संसारवालों में से तुम ही ऐसे हो जो पुरुषों के पास जाते हो,
\end{hindi}}
\flushright{\begin{Arabic}
\quranayah[26][166]
\end{Arabic}}
\flushleft{\begin{hindi}
और अपनी पत्नियों को, जिन्हें तुम्हारे रब ने तुम्हारे लिए पैदा किया, छोड़ देते हो? इतना ही नहीं, बल्कि तुम हद से आगे बढ़े हुए लोग हो।"
\end{hindi}}
\flushright{\begin{Arabic}
\quranayah[26][167]
\end{Arabic}}
\flushleft{\begin{hindi}
उन्होंने कहा, "यदि तू बाज़ न आया, ऐ लतू! तो तू अवश्य ही निकाल बाहर किया जाएगा।"
\end{hindi}}
\flushright{\begin{Arabic}
\quranayah[26][168]
\end{Arabic}}
\flushleft{\begin{hindi}
उसने कहा, "मैं तुम्हारे कर्म से अत्यन्त विरक्त हूँ।
\end{hindi}}
\flushright{\begin{Arabic}
\quranayah[26][169]
\end{Arabic}}
\flushleft{\begin{hindi}
ऐ मेरे रब! मुझे और मेरे लोगों को, जो कुछ ये करते है उसके परिणाम से, बचा ले।"
\end{hindi}}
\flushright{\begin{Arabic}
\quranayah[26][170]
\end{Arabic}}
\flushleft{\begin{hindi}
अन्ततः हमने उसे और उसके सारे लोगों को बचा लिया;
\end{hindi}}
\flushright{\begin{Arabic}
\quranayah[26][171]
\end{Arabic}}
\flushleft{\begin{hindi}
सिवाय एक बुढ़िया के जो पीछे रह जानेवालों में थी
\end{hindi}}
\flushright{\begin{Arabic}
\quranayah[26][172]
\end{Arabic}}
\flushleft{\begin{hindi}
फिर शेष दूसरे लोगों को हमने विनष्ट कर दिया।
\end{hindi}}
\flushright{\begin{Arabic}
\quranayah[26][173]
\end{Arabic}}
\flushleft{\begin{hindi}
और हमने उनपर एक बरसात बरसाई। और यह चेताए हुए लोगों की बहुत ही बुरी वर्षा थी
\end{hindi}}
\flushright{\begin{Arabic}
\quranayah[26][174]
\end{Arabic}}
\flushleft{\begin{hindi}
निश्चय ही इसमें एक बड़ी निशानी है। इसपर भी उनमें से अधिकतर माननेवाले नहीं
\end{hindi}}
\flushright{\begin{Arabic}
\quranayah[26][175]
\end{Arabic}}
\flushleft{\begin{hindi}
और निश्चय ही तुम्हारा रब बड़ा प्रभुत्वशाली, अत्यन्त दयावान है
\end{hindi}}
\flushright{\begin{Arabic}
\quranayah[26][176]
\end{Arabic}}
\flushleft{\begin{hindi}
अल-ऐकावालों ने रसूलों को झुठलाया
\end{hindi}}
\flushright{\begin{Arabic}
\quranayah[26][177]
\end{Arabic}}
\flushleft{\begin{hindi}
जबकि शुऐब ने उनसे कहा, "क्या तुम डर नहीं रखते?
\end{hindi}}
\flushright{\begin{Arabic}
\quranayah[26][178]
\end{Arabic}}
\flushleft{\begin{hindi}
मैं तुम्हारे लिए एक अमानतदार रसूल हूँ
\end{hindi}}
\flushright{\begin{Arabic}
\quranayah[26][179]
\end{Arabic}}
\flushleft{\begin{hindi}
अतः अल्लाह का डर रखो और मेरी आज्ञा का पालन करो
\end{hindi}}
\flushright{\begin{Arabic}
\quranayah[26][180]
\end{Arabic}}
\flushleft{\begin{hindi}
मैं इस काम पर तुमसे कोई प्रतिदान नहीं माँगता। मेरा प्रतिदान तो बस सारे संसार के रब के ज़िम्मे है
\end{hindi}}
\flushright{\begin{Arabic}
\quranayah[26][181]
\end{Arabic}}
\flushleft{\begin{hindi}
तुम पूरा-पूरा पैमाना भरो और घाटा न दो
\end{hindi}}
\flushright{\begin{Arabic}
\quranayah[26][182]
\end{Arabic}}
\flushleft{\begin{hindi}
और ठीक तराज़ू से तौलो
\end{hindi}}
\flushright{\begin{Arabic}
\quranayah[26][183]
\end{Arabic}}
\flushleft{\begin{hindi}
और लोगों को उनकी चीज़ों में घाटा न दो और धरती में बिगाड़ और फ़साद मचाते मत फिरो
\end{hindi}}
\flushright{\begin{Arabic}
\quranayah[26][184]
\end{Arabic}}
\flushleft{\begin{hindi}
उसका डर रखो जिसने तुम्हें और पिछली नस्लों को पैदा किया हैं।"
\end{hindi}}
\flushright{\begin{Arabic}
\quranayah[26][185]
\end{Arabic}}
\flushleft{\begin{hindi}
उन्होंने कहा, "तू तो बस जादू का मारा हुआ है
\end{hindi}}
\flushright{\begin{Arabic}
\quranayah[26][186]
\end{Arabic}}
\flushleft{\begin{hindi}
और तू बस हमारे ही जैसा एक आदमी है और हम तो तुझे झूठा समझते है
\end{hindi}}
\flushright{\begin{Arabic}
\quranayah[26][187]
\end{Arabic}}
\flushleft{\begin{hindi}
फिर तू हमपर आकाश को कोई टुकड़ा गिरा दे, यदि तू सच्चा है।"
\end{hindi}}
\flushright{\begin{Arabic}
\quranayah[26][188]
\end{Arabic}}
\flushleft{\begin{hindi}
उसने कहा, " मेरा रब भली-भाँति जानता है जो कुछ तुम कर रहे हो।"
\end{hindi}}
\flushright{\begin{Arabic}
\quranayah[26][189]
\end{Arabic}}
\flushleft{\begin{hindi}
किन्तु उन्होंने उसे झुठला दिया। फिर छायावाले दिन की यातना ने आ लिया। निश्चय ही वह एक बड़े दिन की यातना थी
\end{hindi}}
\flushright{\begin{Arabic}
\quranayah[26][190]
\end{Arabic}}
\flushleft{\begin{hindi}
निस्संदेह इसमें एक बड़ी निशानी है। इसपर भी उनमें से अधिकतर माननेवाले नहीं
\end{hindi}}
\flushright{\begin{Arabic}
\quranayah[26][191]
\end{Arabic}}
\flushleft{\begin{hindi}
और निश्चय ही तुम्हारा रब ही है, जो बड़ा प्रभुत्वशाली, अत्यन्त दयावान है
\end{hindi}}
\flushright{\begin{Arabic}
\quranayah[26][192]
\end{Arabic}}
\flushleft{\begin{hindi}
निश्चय ही यह (क़ुरआन) सारे संसार के रब की अवतरित की हुई चीज़ है
\end{hindi}}
\flushright{\begin{Arabic}
\quranayah[26][193]
\end{Arabic}}
\flushleft{\begin{hindi}
इसको लेकर तुम्हारे हृदय पर एक विश्वसनीय आत्मा उतरी है,
\end{hindi}}
\flushright{\begin{Arabic}
\quranayah[26][194]
\end{Arabic}}
\flushleft{\begin{hindi}
ताकि तुम सावधान करनेवाले हो
\end{hindi}}
\flushright{\begin{Arabic}
\quranayah[26][195]
\end{Arabic}}
\flushleft{\begin{hindi}
स्पष्ट अरबी भाषा में
\end{hindi}}
\flushright{\begin{Arabic}
\quranayah[26][196]
\end{Arabic}}
\flushleft{\begin{hindi}
और निस्संदेह यह पिछले लोगों की किताबों में भी मौजूद है
\end{hindi}}
\flushright{\begin{Arabic}
\quranayah[26][197]
\end{Arabic}}
\flushleft{\begin{hindi}
क्या यह उनके लिए कोई निशानी नहीं है कि इसे बनी इसराईल के विद्वान जानते है?
\end{hindi}}
\flushright{\begin{Arabic}
\quranayah[26][198]
\end{Arabic}}
\flushleft{\begin{hindi}
यदि हम इसे ग़ैर अरबी भाषी पर भी उतारते,
\end{hindi}}
\flushright{\begin{Arabic}
\quranayah[26][199]
\end{Arabic}}
\flushleft{\begin{hindi}
और वह इसे उन्हें पढ़कर सुनाता तब भी वे इसे माननेवाले न होते
\end{hindi}}
\flushright{\begin{Arabic}
\quranayah[26][200]
\end{Arabic}}
\flushleft{\begin{hindi}
इसी प्रकार हमने इसे अपराधियों के दिलों में पैठाया है
\end{hindi}}
\flushright{\begin{Arabic}
\quranayah[26][201]
\end{Arabic}}
\flushleft{\begin{hindi}
वे इसपर ईमान लाने को नहीं, जब तक कि दुखद यातना न देख लें
\end{hindi}}
\flushright{\begin{Arabic}
\quranayah[26][202]
\end{Arabic}}
\flushleft{\begin{hindi}
फिर जब वह अचानक उनपर आ जाएगी और उन्हें ख़बर भी न होगी,
\end{hindi}}
\flushright{\begin{Arabic}
\quranayah[26][203]
\end{Arabic}}
\flushleft{\begin{hindi}
तब वे कहेंगे, "क्या हमें कुछ मुहलत मिल सकती है?"
\end{hindi}}
\flushright{\begin{Arabic}
\quranayah[26][204]
\end{Arabic}}
\flushleft{\begin{hindi}
तो क्या वे लोग हमारी यातना के लिए जल्दी मचा रहे है?
\end{hindi}}
\flushright{\begin{Arabic}
\quranayah[26][205]
\end{Arabic}}
\flushleft{\begin{hindi}
क्या तुमने कुछ विचार किया? यदि हम उन्हें कुछ वर्षों तक सुख भोगने दें;
\end{hindi}}
\flushright{\begin{Arabic}
\quranayah[26][206]
\end{Arabic}}
\flushleft{\begin{hindi}
फिर उनपर वह चीज़ आ जाए, जिससे उन्हें डराया जाता रहा है;
\end{hindi}}
\flushright{\begin{Arabic}
\quranayah[26][207]
\end{Arabic}}
\flushleft{\begin{hindi}
तो जो सुख उन्हें मिला होगा वह उनके कुछ काम न आएगा
\end{hindi}}
\flushright{\begin{Arabic}
\quranayah[26][208]
\end{Arabic}}
\flushleft{\begin{hindi}
हमने किसी बस्ती को भी इसके बिना विनष्ट नहीं किया कि उसके लिए सचेत करनेवाले याददिहानी के लिए मौजूद रहे हैं।
\end{hindi}}
\flushright{\begin{Arabic}
\quranayah[26][209]
\end{Arabic}}
\flushleft{\begin{hindi}
हम कोई ज़ालिम नहीं है
\end{hindi}}
\flushright{\begin{Arabic}
\quranayah[26][210]
\end{Arabic}}
\flushleft{\begin{hindi}
इसे शैतान लेकर नहीं उतरे हैं।
\end{hindi}}
\flushright{\begin{Arabic}
\quranayah[26][211]
\end{Arabic}}
\flushleft{\begin{hindi}
न यह उन्हें फबता ही है और न ये उनके बस का ही है
\end{hindi}}
\flushright{\begin{Arabic}
\quranayah[26][212]
\end{Arabic}}
\flushleft{\begin{hindi}
वे तो इसके सुनने से भी दूर रखे गए है
\end{hindi}}
\flushright{\begin{Arabic}
\quranayah[26][213]
\end{Arabic}}
\flushleft{\begin{hindi}
अतः अल्लाह के साथ दूसरे इष्ट-पूज्य को न पुकारना, अन्यथा तुम्हें भी यातना दी जाएगी
\end{hindi}}
\flushright{\begin{Arabic}
\quranayah[26][214]
\end{Arabic}}
\flushleft{\begin{hindi}
और अपने निकटतम नातेदारों को सचेत करो
\end{hindi}}
\flushright{\begin{Arabic}
\quranayah[26][215]
\end{Arabic}}
\flushleft{\begin{hindi}
और जो ईमानवाले तुम्हारे अनुयायी हो गए है, उनके लिए अपनी भुजाएँ बिछाए रखो
\end{hindi}}
\flushright{\begin{Arabic}
\quranayah[26][216]
\end{Arabic}}
\flushleft{\begin{hindi}
किन्तु यदि वे तुम्हारी अवज्ञा करें तो कह दो, "जो कुछ तुम करते हो, उसकी ज़िम्मेदारी से मं1 बरी हूँ।"
\end{hindi}}
\flushright{\begin{Arabic}
\quranayah[26][217]
\end{Arabic}}
\flushleft{\begin{hindi}
और उस प्रभुत्वशाली और दया करनेवाले पर भरोसा रखो
\end{hindi}}
\flushright{\begin{Arabic}
\quranayah[26][218]
\end{Arabic}}
\flushleft{\begin{hindi}
जो तुम्हें देख रहा होता है, जब तुम खड़े होते हो
\end{hindi}}
\flushright{\begin{Arabic}
\quranayah[26][219]
\end{Arabic}}
\flushleft{\begin{hindi}
और सजदा करनेवालों में तुम्हारे चलत-फिरत को भी वह देखता है
\end{hindi}}
\flushright{\begin{Arabic}
\quranayah[26][220]
\end{Arabic}}
\flushleft{\begin{hindi}
निस्संदेह वह भली-भाँति सुनता-जानता है
\end{hindi}}
\flushright{\begin{Arabic}
\quranayah[26][221]
\end{Arabic}}
\flushleft{\begin{hindi}
क्या मैं तुम्हें बताऊँ कि शैतान किसपर उतरते है?
\end{hindi}}
\flushright{\begin{Arabic}
\quranayah[26][222]
\end{Arabic}}
\flushleft{\begin{hindi}
वे प्रत्येक ढोंग रचनेवाले गुनाहगार पर उतरते है
\end{hindi}}
\flushright{\begin{Arabic}
\quranayah[26][223]
\end{Arabic}}
\flushleft{\begin{hindi}
वे कान लगाते है और उनमें से अधिकतर झूठे होते है
\end{hindi}}
\flushright{\begin{Arabic}
\quranayah[26][224]
\end{Arabic}}
\flushleft{\begin{hindi}
रहे कवि, तो उनके पीछे बहके हुए लोग ही चला करते है।-
\end{hindi}}
\flushright{\begin{Arabic}
\quranayah[26][225]
\end{Arabic}}
\flushleft{\begin{hindi}
क्या तुमने देखा नहीं कि वे हर घाटी में बहके फिरते हैं,
\end{hindi}}
\flushright{\begin{Arabic}
\quranayah[26][226]
\end{Arabic}}
\flushleft{\begin{hindi}
और कहते वह है जो करते नहीं? -
\end{hindi}}
\flushright{\begin{Arabic}
\quranayah[26][227]
\end{Arabic}}
\flushleft{\begin{hindi}
वे नहीं जो ईमान लाए और उन्होंने अच्छे कर्म किए और अल्लाह को अधिक .याद किया। औऱ इसके बाद कि उनपर ज़ुल्म किया गया तो उन्होंने उसका प्रतिकार किया और जिन लोगों ने ज़ुल्म किया, उन्हें जल्द ही मालूम हो जाएगा कि वे किस जगह पलटते हैं
\end{hindi}}
\chapter{An-Naml (The Naml)}
\begin{Arabic}
\Huge{\centerline{\basmalah}}\end{Arabic}
\flushright{\begin{Arabic}
\quranayah[27][1]
\end{Arabic}}
\flushleft{\begin{hindi}
ता॰ सीन॰। ये आयतें है क़ुरआन और एक स्पष्ट किताब की
\end{hindi}}
\flushright{\begin{Arabic}
\quranayah[27][2]
\end{Arabic}}
\flushleft{\begin{hindi}
मार्गदर्शन है और शुभ-सूचना उन ईमानवालों के लिए,
\end{hindi}}
\flushright{\begin{Arabic}
\quranayah[27][3]
\end{Arabic}}
\flushleft{\begin{hindi}
जो नमाज़ का आयोजन करते है और ज़कात देते है और वही है जो आख़िरत पर विश्वास रखते है
\end{hindi}}
\flushright{\begin{Arabic}
\quranayah[27][4]
\end{Arabic}}
\flushleft{\begin{hindi}
रहे वे लोग जो आख़िरत को नहीं मानते, उनके लिए हमने उनकी करतूतों को शोभायमान बना दिया है। अतः वे भटकते फिरते है
\end{hindi}}
\flushright{\begin{Arabic}
\quranayah[27][5]
\end{Arabic}}
\flushleft{\begin{hindi}
वही लोग है, जिनके लिए बुरी यातना है और वही है जो आख़िरत में अत्यन्त घाटे में रहेंगे
\end{hindi}}
\flushright{\begin{Arabic}
\quranayah[27][6]
\end{Arabic}}
\flushleft{\begin{hindi}
निश्चय ही तुम यह क़ुरआन एक बड़े तत्वदर्शी, ज्ञानवान (प्रभु) की ओर से पा रहे हो
\end{hindi}}
\flushright{\begin{Arabic}
\quranayah[27][7]
\end{Arabic}}
\flushleft{\begin{hindi}
याद करो जब मूसा ने अपने घरवालों से कहा कि "मैंने एक आग-सी देखी है। मैं अभी वहाँ से तुम्हारे पास कोई ख़बर लेकर आता हूँ या तुम्हारे पास कोई दहकता अंगार लाता हूँ, ताकि तुम तापो।"
\end{hindi}}
\flushright{\begin{Arabic}
\quranayah[27][8]
\end{Arabic}}
\flushleft{\begin{hindi}
फिर जब वह उसके पास पहुँचा तो उसे आवाज़ आई कि "मुबारक है वह जो इस आग में है और जो इसके आस-पास है। महान और उच्च है अल्लाह, सारे संसार का रब!
\end{hindi}}
\flushright{\begin{Arabic}
\quranayah[27][9]
\end{Arabic}}
\flushleft{\begin{hindi}
ऐ मूसा! वह तो मैं अल्लाह हूँ, अत्यन्त प्रभुत्वशाली, तत्वदर्शी!
\end{hindi}}
\flushright{\begin{Arabic}
\quranayah[27][10]
\end{Arabic}}
\flushleft{\begin{hindi}
तू अपनी लाठी डाल दे।" जब मूसा ने देखा कि वह बल खा रहा है जैसे वह कोई साँप हो, तो वह पीठ फेरकर भागा और पीछे मुड़कर न देखा। "ऐ मूसा! डर मत। निस्संदेह रसूल मेरे पास डरा नहीं करते,
\end{hindi}}
\flushright{\begin{Arabic}
\quranayah[27][11]
\end{Arabic}}
\flushleft{\begin{hindi}
सिवाय उसके जिसने कोई ज़्यादती की हो। फिर बुराई के पश्चात उसे भलाई से बदल दिया, तो मैं भी बड़ा क्षमाशील, अत्यन्त दयावान हूँ
\end{hindi}}
\flushright{\begin{Arabic}
\quranayah[27][12]
\end{Arabic}}
\flushleft{\begin{hindi}
अपना हाथ गिरेबान में डाल। वह बिना किसी ख़राबी के उज्जवल चमकता निकलेगा। ये नौ निशानियों में से है फ़िरऔन और उसकी क़ौम की ओर भेजने के लिए। निश्चय ही वे अवज्ञाकारी लोग है।"
\end{hindi}}
\flushright{\begin{Arabic}
\quranayah[27][13]
\end{Arabic}}
\flushleft{\begin{hindi}
किन्तु जब आँखें खोल देनेवाली हमारी निशानियाँ उनके पास आई तो उन्होंने कहा, "यह तो खुला हुआ जादू है।"
\end{hindi}}
\flushright{\begin{Arabic}
\quranayah[27][14]
\end{Arabic}}
\flushleft{\begin{hindi}
उन्होंने ज़ुल्म और सरकशी से उनका इनकार कर दिया, हालाँकि उनके जी को उनका विश्वास हो चुका था। अब देख लो इन बिगाड़ पैदा करनेवालों का क्या परिणाम हुआ?
\end{hindi}}
\flushright{\begin{Arabic}
\quranayah[27][15]
\end{Arabic}}
\flushleft{\begin{hindi}
हमने दाऊद और सुलैमान को बड़ा ज्ञान प्रदान किया था, (उन्होंने उसके महत्व को जाना) और उन दोनों ने कहा, "सारी प्रशंसा अल्लाह के लिए है, जिसने हमें अपने बहुत-से ईमानवाले बन्दों के मुक़ाबले में श्रेष्ठता प्रदान की।"
\end{hindi}}
\flushright{\begin{Arabic}
\quranayah[27][16]
\end{Arabic}}
\flushleft{\begin{hindi}
दाऊद का उत्तराधिकारी सुलैमान हुआ औऱ उसने कहा, "ऐ लोगों! हमें पक्षियों की बोली सिखाई गई है और हमें हर चीज़ दी गई है। निस्संदेह वह स्पष्ट बड़ाई है।"
\end{hindi}}
\flushright{\begin{Arabic}
\quranayah[27][17]
\end{Arabic}}
\flushleft{\begin{hindi}
सुलैमान के लिए जिन्न और मनुष्य और पक्षियों मे से उसकी सेनाएँ एकत्र कर गई फिर उनकी दर्जाबन्दी की जा रही थी
\end{hindi}}
\flushright{\begin{Arabic}
\quranayah[27][18]
\end{Arabic}}
\flushleft{\begin{hindi}
यहाँ तक कि जब वे चींटियों की घाटी में पहुँचे तो एक चींटी ने कहा, "ऐ चींटियों! अपने घरों में प्रवेश कर जाओ। कहीं सुलैमान और उसकी सेनाएँ तुम्हें कुचल न डालें और उन्हें एहसास भी न हो।"
\end{hindi}}
\flushright{\begin{Arabic}
\quranayah[27][19]
\end{Arabic}}
\flushleft{\begin{hindi}
तो वह उसकी बात पर प्रसन्न होकर मुस्कराया और कहा, "मेरे रब! मुझे संभाले रख कि मैं तेरी उस कृपा पर कृतज्ञता दिखाता रहूँ जो तूने मुझपर और मेरे माँ-बाप पर की है। और यह कि अच्छा कर्म करूँ जो तुझे पसन्द आए और अपनी दयालुता से मुझे अपने अच्छे बन्दों में दाखिल कर।"
\end{hindi}}
\flushright{\begin{Arabic}
\quranayah[27][20]
\end{Arabic}}
\flushleft{\begin{hindi}
उसने पक्षियों की जाँच-पड़ताल की तो कहा, "क्या बात है कि मैं हुदहुद को नहीं देख रहा हूँ, (वह यहीं कहीं है) या ग़ायब हो गया है?
\end{hindi}}
\flushright{\begin{Arabic}
\quranayah[27][21]
\end{Arabic}}
\flushleft{\begin{hindi}
मैं उसे कठोर दंड दूँगा या उसे ज़बह ही कर डालूँगा या फिर वह मेरे सामने कोई स्पष्ट॥ कारण प्रस्तुत करे।"
\end{hindi}}
\flushright{\begin{Arabic}
\quranayah[27][22]
\end{Arabic}}
\flushleft{\begin{hindi}
फिर कुछ अधिक देर नहीं ठहरा कि उसने आकर कहा, "मैंने वह जानकारी प्राप्त की है जो आपको मालूम नहीं है। मैं सबा से आपके पास एक विश्वसनीय सूचना लेकर आया हूँ
\end{hindi}}
\flushright{\begin{Arabic}
\quranayah[27][23]
\end{Arabic}}
\flushleft{\begin{hindi}
मैंने एक स्त्री को उनपर शासन करते पाया है। उसे हर चीज़ प्राप्त है औऱ उसका एक बड़ा सिंहासन है
\end{hindi}}
\flushright{\begin{Arabic}
\quranayah[27][24]
\end{Arabic}}
\flushleft{\begin{hindi}
मैंने उसे और उसकी क़ौम के लोगों को अल्लाह से इतर सूर्य को सजदा करते हुए पाया। शैतान ने उनके कर्मों को उनके लिए शोभायमान बना दिया है और उन्हें मार्ग से रोक दिया है - अतः वे सीधा मार्ग नहीं पा रहे है। -
\end{hindi}}
\flushright{\begin{Arabic}
\quranayah[27][25]
\end{Arabic}}
\flushleft{\begin{hindi}
खि अल्लाह को सजदा न करें जो आकाशों और धरती की छिपी चीज़ें निकालता है, और जानता है जो कुछ भी तुम छिपाते हो और जो कुछ प्रकट करते हो
\end{hindi}}
\flushright{\begin{Arabic}
\quranayah[27][26]
\end{Arabic}}
\flushleft{\begin{hindi}
अल्लाह कि उसके सिवा कोई इष्ट -पूज्य नहीं, वह महान सिंहासन का रब है।"
\end{hindi}}
\flushright{\begin{Arabic}
\quranayah[27][27]
\end{Arabic}}
\flushleft{\begin{hindi}
उसने कहा, "अभी हम देख लेते है कि तूने सच कहा या तू झूठा है
\end{hindi}}
\flushright{\begin{Arabic}
\quranayah[27][28]
\end{Arabic}}
\flushleft{\begin{hindi}
मेरा यह पत्र लेकर जा, और इसे उन लोगों की ओर डाल दे। फिर उनके पास से अलग हटकर देख कि वे क्या प्रतिक्रिया व्यक्त करते है।"
\end{hindi}}
\flushright{\begin{Arabic}
\quranayah[27][29]
\end{Arabic}}
\flushleft{\begin{hindi}
वह बोली, "ऐ सरदारों! मेरी ओर एक प्रतिष्ठित पत्र डाला गया है
\end{hindi}}
\flushright{\begin{Arabic}
\quranayah[27][30]
\end{Arabic}}
\flushleft{\begin{hindi}
वह सुलैमान की ओर से है और वह यह है कि अल्लाह के नाम से जो बड़ा कृपाशील, अत्यन्त दयावान है
\end{hindi}}
\flushright{\begin{Arabic}
\quranayah[27][31]
\end{Arabic}}
\flushleft{\begin{hindi}
यह कि मेरे मुक़ाबले में सरकशी न करो और आज्ञाकारी बनकर मेरे पास आओ।"
\end{hindi}}
\flushright{\begin{Arabic}
\quranayah[27][32]
\end{Arabic}}
\flushleft{\begin{hindi}
उसने कहा, "ऐ सरदारों! मेरे मामलें में मुझे परामर्श दो। मैं किसी मामले का फ़ैसला नहीं करती, जब तक कि तुम मेरे पास मौजूद न हो।"
\end{hindi}}
\flushright{\begin{Arabic}
\quranayah[27][33]
\end{Arabic}}
\flushleft{\begin{hindi}
उन्होंने कहा, "हम शक्तिशाली है और हमें बड़ी युद्ध-क्षमता प्राप्त है। आगे मामले का अधिकार आपको है, अतः आप देख लें कि आपको क्या आदेश देना है।"
\end{hindi}}
\flushright{\begin{Arabic}
\quranayah[27][34]
\end{Arabic}}
\flushleft{\begin{hindi}
उसने कहा, "सम्राट जब किसी बस्ती में प्रवेश करते है, तो उसे ख़राब कर देते है और वहाँ के प्रभावशाली लोगों को अपमानित करके रहते है। और वे ऐसा ही करेंगे
\end{hindi}}
\flushright{\begin{Arabic}
\quranayah[27][35]
\end{Arabic}}
\flushleft{\begin{hindi}
मैं उनके पास एक उपहार भेजती हूँ; फिर देखती हूँ कि दूत क्या उत्तर लेकर लौटते है।"
\end{hindi}}
\flushright{\begin{Arabic}
\quranayah[27][36]
\end{Arabic}}
\flushleft{\begin{hindi}
फिर जब वह सुलैमान के पास पहुँचा तो उसने (सुलैमान ने) कहा, "क्या तुम माल से मेरी सहायता करोगे, तो जो कुछ अल्लाह ने मुझे दिया है वह उससे कहीं उत्तम है, जो उसने तुम्हें दिया है? बल्कि तुम्ही लोग हो जो अपने उपहार से प्रसन्न होते हो!
\end{hindi}}
\flushright{\begin{Arabic}
\quranayah[27][37]
\end{Arabic}}
\flushleft{\begin{hindi}
उनके पास वापस जाओ। हम उनपर ऐसी सेनाएँ लेकर आएँगे, जिनका मुक़ाबला वे न कर सकेंगे और हम उन्हें अपमानित करके वहाँ से निकाल देंगे कि वे पस्त होकर रहेंगे।"
\end{hindi}}
\flushright{\begin{Arabic}
\quranayah[27][38]
\end{Arabic}}
\flushleft{\begin{hindi}
उसने (सुलैमान ने) कहा, "ऐ सरदारो! तुममें कौन उसका सिंहासन लेकर मेरे पास आता है, इससे पहले कि वे लोग आज्ञाकारी होकर मेरे पास आएँ?"
\end{hindi}}
\flushright{\begin{Arabic}
\quranayah[27][39]
\end{Arabic}}
\flushleft{\begin{hindi}
जिन्नों में से एक बलिष्ठ निर्भीक ने कहा, "मैं उसे आपके पास ले आऊँगा। इससे पहले कि आप अपने स्थान से उठे। मुझे इसकी शक्ति प्राप्त है और मैं अमानतदार भी हूँ।"
\end{hindi}}
\flushright{\begin{Arabic}
\quranayah[27][40]
\end{Arabic}}
\flushleft{\begin{hindi}
जिस व्यक्ति के पास किताब का ज्ञान था, उसने कहा, "मैं आपकी पलक झपकने से पहले उसे आपके पास लाए देता हूँ।" फिर जब उसने उसे अपने पास रखा हुआ देखा तो कहा, "यह मेरे रब का उदार अनुग्रह है, ताकि वह मेरी परीक्षा करे कि मैं कृतज्ञता दिखाता हूँ या कृतघ्न बनता हूँ। जो कृतज्ञता दिखलाता है तो वह अपने लिए ही कृतज्ञता दिखलाता है और वह जिसने कृतघ्नता दिखाई, तो मेरा रब निश्चय ही निस्पृह, बड़ा उदार है।"
\end{hindi}}
\flushright{\begin{Arabic}
\quranayah[27][41]
\end{Arabic}}
\flushleft{\begin{hindi}
उसने कहा, "उसके पास उसके सिंहासन का रूप बदल दो। देंखे वह वास्तविकता को पा लेती है या उन लोगों में से होकर रह जाती है, जो वास्तविकता को पा लेती है या उन लोगों में से होकर जाती है, जो वास्तविकता को पा लेती है या उन लोगों में से होकर रह जाती है, जो वास्तविकता को नहीं पाते।"
\end{hindi}}
\flushright{\begin{Arabic}
\quranayah[27][42]
\end{Arabic}}
\flushleft{\begin{hindi}
जब वह आई तो कहा गया, "क्या तुम्हारा सिंहासन ऐसा ही है?" उसने कहा, "यह तो जैसे वही है, और हमें तो इससे पहले ही ज्ञान प्राप्त हो चुका था; और हम आज्ञाकारी हो गए थे।"
\end{hindi}}
\flushright{\begin{Arabic}
\quranayah[27][43]
\end{Arabic}}
\flushleft{\begin{hindi}
अल्लाह से हटकर वह दूसरे को पूजती थी। इसी चीज़ ने उसे रोक रखा था। निस्संदेह वह एक इनकार करनेवाली क़ौम में से थी
\end{hindi}}
\flushright{\begin{Arabic}
\quranayah[27][44]
\end{Arabic}}
\flushleft{\begin{hindi}
उससे कहा गया कि "महल में प्रवेश करो।" तो जब उसने उसे देखा तो उसने उसको गहरा पानी समझा और उसने अपनी दोनों पिंडलियाँ खोल दी। उसने कहा, "यह तो शीशे से निर्मित महल है।" बोली, "ऐ मेरे रब! निश्चय ही मैंने अपने आपपर ज़ुल्म किया। अब मैंने सुलैमान के साथ अपने आपको अल्लाह के समर्पित कर दिया, जो सारे संसार का रब है।"
\end{hindi}}
\flushright{\begin{Arabic}
\quranayah[27][45]
\end{Arabic}}
\flushleft{\begin{hindi}
और समूद की ओर हमने उनके भाई सालेह को भेजा कि "अल्लाह की बन्दगी करो।" तो क्या देखते है कि वे दो गिरोह होकर आपस में झगड़ने लगे
\end{hindi}}
\flushright{\begin{Arabic}
\quranayah[27][46]
\end{Arabic}}
\flushleft{\begin{hindi}
उसने कहा, "ऐ मेरी क़ौम के लोगों, तुम भलाई से पहले बुराई के लिए क्यों जल्दी मचा रहे हो? तुम अल्लाह से क्षमा याचना क्यों नहीं करते? कदाचित तुमपर दया की जाए।"
\end{hindi}}
\flushright{\begin{Arabic}
\quranayah[27][47]
\end{Arabic}}
\flushleft{\begin{hindi}
उन्होंने कहा, "हमने तुम्हें और तुम्हारे साथवालों को अपशकुन पाया है।" उसने कहा, "तुम्हारा शकुन-अपशकुन तो अल्लाह के पास है, बल्कि बात यह है कि तुम लोग आज़माए जा रहे हो।"
\end{hindi}}
\flushright{\begin{Arabic}
\quranayah[27][48]
\end{Arabic}}
\flushleft{\begin{hindi}
नगर में नौ जत्थेदार थे जो धरती में बिगाड़ पैदा करते थे, सुधार का काम नहीं करते थे
\end{hindi}}
\flushright{\begin{Arabic}
\quranayah[27][49]
\end{Arabic}}
\flushleft{\begin{hindi}
वे आपस में अल्लाह की क़समें खाकर बोले, "हम अवश्य उसपर और उसके घरवालों पर रात को छापा मारेंगे। फिर उसके वली (परिजन) से कह देंगे कि हम उसके घरवालों के विनाश के अवसर पर मौजूद न थे। और हम बिलकुल सच्चे है।"
\end{hindi}}
\flushright{\begin{Arabic}
\quranayah[27][50]
\end{Arabic}}
\flushleft{\begin{hindi}
वे एक चाल चले और हमने भी एक चाल चली और उन्हें ख़बर तक न हुई
\end{hindi}}
\flushright{\begin{Arabic}
\quranayah[27][51]
\end{Arabic}}
\flushleft{\begin{hindi}
अब देख लो, उनकी चाल का कैसा परिणाम हुआ! हमने उन्हें और उनकी क़ौम - सबको विनष्ट करके रख दिया
\end{hindi}}
\flushright{\begin{Arabic}
\quranayah[27][52]
\end{Arabic}}
\flushleft{\begin{hindi}
अब ये उनके घर उनके ज़ुल्म के कारण उजडें पड़े हुए है। निश्चय ही इसमें एक बड़ी निशानी है उन लोगों के लिए जो जानना चाहें
\end{hindi}}
\flushright{\begin{Arabic}
\quranayah[27][53]
\end{Arabic}}
\flushleft{\begin{hindi}
और हमने उन लोगों को बचा लिया, जो ईमान लाए और डर रखते थे
\end{hindi}}
\flushright{\begin{Arabic}
\quranayah[27][54]
\end{Arabic}}
\flushleft{\begin{hindi}
और लूत को भी भेजा, जब उसने अपनी क़ौम के लोगों से कहा, "क्या तुम आँखों देखते हुए अश्लील कर्म करते हो?
\end{hindi}}
\flushright{\begin{Arabic}
\quranayah[27][55]
\end{Arabic}}
\flushleft{\begin{hindi}
क्या तुम स्त्रियों को छोड़कर अपनी काम-तृप्ति के लिए पुरुषों के पास जाते हो? बल्कि बात यह है कि तुम बड़े ही जाहिल लोग हो।"
\end{hindi}}
\flushright{\begin{Arabic}
\quranayah[27][56]
\end{Arabic}}
\flushleft{\begin{hindi}
परन्तु उसकी क़ौम के लोगों का उत्तर इसके सिवा कुछ न था कि उन्होंने कहा, "निकाल बाहर करो लूत के घरवालों को अपनी बस्ती से। ये लोग सुथराई को बहुत पसन्द करते है!"
\end{hindi}}
\flushright{\begin{Arabic}
\quranayah[27][57]
\end{Arabic}}
\flushleft{\begin{hindi}
अन्ततः हमने उसे और उसके घरवालों को बचा लिया सिवाय उसकी स्त्री के। उसके लिए हमने नियत कर दिया था कि वह पीछे रह जानेवालों में से होगी
\end{hindi}}
\flushright{\begin{Arabic}
\quranayah[27][58]
\end{Arabic}}
\flushleft{\begin{hindi}
और हमने उनपर एक बरसात बरसाई और वह बहुत ही बुरी बरसात था उन लोगों के हक़ में, जिन्हें सचेत किया जा चुका था
\end{hindi}}
\flushright{\begin{Arabic}
\quranayah[27][59]
\end{Arabic}}
\flushleft{\begin{hindi}
कहो, "प्रशंसा अल्लाह के लिए है और सलाम है उनके उन बन्दों पर जिन्हें उसने चुन लिया। क्या अल्लाह अच्छा है या वे जिन्हें वे साझी ठहरा रहे है?
\end{hindi}}
\flushright{\begin{Arabic}
\quranayah[27][60]
\end{Arabic}}
\flushleft{\begin{hindi}
(तुम्हारे पूज्य अच्छे है) या वह जिसने आकाशों और धरती को पैदा किया और तुम्हारे लिए आकाश से पानी बरसाया; उसके द्वारा हमने रमणीय उद्यान उगाए? तुम्हारे लिए सम्भव न था कि तुम उनके वृक्षों को उगाते। - क्या अल्लाह के साथ कोई और प्रभु-पूज्य है? नहीं, बल्कि वही लोग मार्ग से हटकर चले जा रहे है!
\end{hindi}}
\flushright{\begin{Arabic}
\quranayah[27][61]
\end{Arabic}}
\flushleft{\begin{hindi}
या वह जिसने धरती को ठहरने का स्थान बनाया और उसके बीच-बीच में नदियाँ बहाई और उसके लिए मज़बूत पहाड़ बनाए और दो समुद्रों के बीच एक रोक लगा दी। क्या अल्लाह के साथ कोई और प्रभु पूज्य है? नहीं, उनमें से अधिकतर लोग जानते ही नही!
\end{hindi}}
\flushright{\begin{Arabic}
\quranayah[27][62]
\end{Arabic}}
\flushleft{\begin{hindi}
या वह जो व्यग्र की पुकार सुनता है, जब वह उसे पुकारे और तकलीफ़ दूर कर देता है और तुम्हें धरती में अधिकारी बनाता है? क्या अल्लाह के साथ कोई और पूज्य-प्रभु है? तुम ध्यान थोड़े ही देते हो
\end{hindi}}
\flushright{\begin{Arabic}
\quranayah[27][63]
\end{Arabic}}
\flushleft{\begin{hindi}
या वह जो थल और जल के अँधेरों में तुम्हारा मार्गदर्शन करता है और जो अपनी दयालुता के आगे हवाओं को शुभ-सूचना बनाकर भेजता है? क्या अल्लाह के साथ कोई और प्रभु पूज्य है? उच्च है अल्लाह, उस शिर्क से जो वे करते है
\end{hindi}}
\flushright{\begin{Arabic}
\quranayah[27][64]
\end{Arabic}}
\flushleft{\begin{hindi}
या वह जो सृष्टि का आरम्भ करता है, फिर उसकी पुनरावृत्ति भी करता है, और जो तुमको आकाश और धरती से रोज़ी देता है? क्या अल्लाह के साथ कोई और प्रभ पूज्य है? कहो, "लाओ अपना प्रमाण, यदि तुम सच्चे हो।"
\end{hindi}}
\flushright{\begin{Arabic}
\quranayah[27][65]
\end{Arabic}}
\flushleft{\begin{hindi}
कहो, "आकाशों और धरती में जो भी है, अल्लाह के सिवा किसी को भी परोक्ष का ज्ञान नहीं है। और न उन्हें इसकी चेतना प्राप्त है कि वे कब उठाए जाएँगे।"
\end{hindi}}
\flushright{\begin{Arabic}
\quranayah[27][66]
\end{Arabic}}
\flushleft{\begin{hindi}
बल्कि आख़िरत के विषय में उनका ज्ञान पक्का हो गया है, बल्कि ये उसकी ओर से कुछ संदेह में है, बल्कि वे उससे अंधे है
\end{hindi}}
\flushright{\begin{Arabic}
\quranayah[27][67]
\end{Arabic}}
\flushleft{\begin{hindi}
जिन लोगों ने इनकार किया वे कहते है कि "क्या जब हम मिट्टी हो जाएँगे और हमारे बाप-दादा भी, तो क्या वास्तव में हम (जीवित करके) निकाले जाएँगे?
\end{hindi}}
\flushright{\begin{Arabic}
\quranayah[27][68]
\end{Arabic}}
\flushleft{\begin{hindi}
इसका वादा तो इससे पहले भी किया जा चुका है, हमसे भी और हमारे बाप-दादा से भी। ये तो बस पहले लोगो की कहानियाँ है।"
\end{hindi}}
\flushright{\begin{Arabic}
\quranayah[27][69]
\end{Arabic}}
\flushleft{\begin{hindi}
कहो कि "धरती में चलो-फिरो और देखो कि अपराधियों का कैसा परिणाम हुआ।"
\end{hindi}}
\flushright{\begin{Arabic}
\quranayah[27][70]
\end{Arabic}}
\flushleft{\begin{hindi}
उनके प्रति शोकाकुल न हो और न उस चाल से दिल तंग हो, जो वे चल रहे है।
\end{hindi}}
\flushright{\begin{Arabic}
\quranayah[27][71]
\end{Arabic}}
\flushleft{\begin{hindi}
वे कहते है, "यह वादा कब पूरा होगा, यदि तुम सच्चे हो?"
\end{hindi}}
\flushright{\begin{Arabic}
\quranayah[27][72]
\end{Arabic}}
\flushleft{\begin{hindi}
कहो, "जिसकी तुम जल्दी मचा रहे हो बहुत सम्भव है कि उसका कोई हिस्सा तुम्हारे पीछे ही लगा हो।"
\end{hindi}}
\flushright{\begin{Arabic}
\quranayah[27][73]
\end{Arabic}}
\flushleft{\begin{hindi}
निश्चय ही तुम्हारा रब तो लोगों पर उदार अनुग्रह करनेवाला है, किन्तु उनमें से अधिकतर लोग कृतज्ञता नहीं दिखाते
\end{hindi}}
\flushright{\begin{Arabic}
\quranayah[27][74]
\end{Arabic}}
\flushleft{\begin{hindi}
निश्चय ही तुम्हारा रह भली-भाँति जानता है, जो कुछ उनके सीने छिपाए हुए है और जो कुछ वे प्रकट करते है।
\end{hindi}}
\flushright{\begin{Arabic}
\quranayah[27][75]
\end{Arabic}}
\flushleft{\begin{hindi}
आकाश और धरती में छिपी कोई भी चीज़ ऐसी नहीं जो एक स्पष्ट किताब में मौजूद न हो
\end{hindi}}
\flushright{\begin{Arabic}
\quranayah[27][76]
\end{Arabic}}
\flushleft{\begin{hindi}
निस्संदेह यह क़ुरआन इसराईल की सन्तान को अधिकतर ऐसी बाते खोलकर सुनाता है जिनके विषय में उनसे मतभेद है
\end{hindi}}
\flushright{\begin{Arabic}
\quranayah[27][77]
\end{Arabic}}
\flushleft{\begin{hindi}
और निस्संदह यह तो ईमानवालों के लिए मार्गदर्शन और दयालुता है
\end{hindi}}
\flushright{\begin{Arabic}
\quranayah[27][78]
\end{Arabic}}
\flushleft{\begin{hindi}
निश्चय ही तुम्हारा रब उनके बीच अपने हुक्म से फ़ैसला कर देगा। वह अत्यन्त प्रभुत्वशाली, सर्वज्ञ है
\end{hindi}}
\flushright{\begin{Arabic}
\quranayah[27][79]
\end{Arabic}}
\flushleft{\begin{hindi}
अतः अल्लाह पर भरोसा रखो। निश्चय ही तुम स्पष्ट सत्य पर हो
\end{hindi}}
\flushright{\begin{Arabic}
\quranayah[27][80]
\end{Arabic}}
\flushleft{\begin{hindi}
तुम मुर्दों को नहीं सुना सकते और न बहरों को अपनी पुकार सुना सकते हो, जबकि वे पीठ देकर फिरे भी जा रहें हो।
\end{hindi}}
\flushright{\begin{Arabic}
\quranayah[27][81]
\end{Arabic}}
\flushleft{\begin{hindi}
और न तुम अंधों को उनकी गुमराही से हटाकर राह पर ला सकते हो। तुम तो बस उन्हीं को सुना सकते हो, जो हमारी आयतों पर ईमान लाना चाहें। अतः वही आज्ञाकारी होते है
\end{hindi}}
\flushright{\begin{Arabic}
\quranayah[27][82]
\end{Arabic}}
\flushleft{\begin{hindi}
और जब उनपर बात पूरी हो जाएगी, तो हम उनके लिए धरती का प्राणी सामने लाएँगे जो उनसे बातें करेगा कि "लोग हमारी आयतों पर विश्वास नहीं करते थे"
\end{hindi}}
\flushright{\begin{Arabic}
\quranayah[27][83]
\end{Arabic}}
\flushleft{\begin{hindi}
और जिस दिन हम प्रत्येक समुदाय में से एक गिरोह, ऐसे लोगों का जो हमारी आयतों को झुठलाते है, घेर लाएँगे। फिर उनकी दर्जाबन्दी की जाएगी
\end{hindi}}
\flushright{\begin{Arabic}
\quranayah[27][84]
\end{Arabic}}
\flushleft{\begin{hindi}
यहाँ तक कि जब वे आ जाएँगे तो वह कहेगा, "क्या तुमने मेरी आयतों को झुठलाया, हालाँकि अपने ज्ञान से तुम उनपर हावी न थे या फिर तुम क्या करते थे?"
\end{hindi}}
\flushright{\begin{Arabic}
\quranayah[27][85]
\end{Arabic}}
\flushleft{\begin{hindi}
और बात उनपर पूरी होकर रहेगी, इसलिए कि उन्होंने ज़ुल्म किया। अतः वे कुछ बोल न सकेंगे
\end{hindi}}
\flushright{\begin{Arabic}
\quranayah[27][86]
\end{Arabic}}
\flushleft{\begin{hindi}
क्या उन्होंने देखा नहीं कि हमने रात को (अँधेरी) बनाया, ताकि वे उसमें शान्ति और चैन प्राप्त करें। और दिन को प्रकाशमान बनाया (कि उसमें काम करें)? निश्चय ही इसमें उन लोगों के लिए निशानियाँ है, जो ईमान ले आएँ
\end{hindi}}
\flushright{\begin{Arabic}
\quranayah[27][87]
\end{Arabic}}
\flushleft{\begin{hindi}
और ख़याल करो जिस दिन सूर (नरसिंघा) में फूँक मारी जाएगी और जो आकाशों और धरती में है, घबरा उठेंगे, सिवाय उनके जिन्हें अल्लाह चाहे - और सब कान दबाए उसके समक्ष उपस्थित हो जाएँगे
\end{hindi}}
\flushright{\begin{Arabic}
\quranayah[27][88]
\end{Arabic}}
\flushleft{\begin{hindi}
और तुम पहाड़ों को देखकर समझते हो कि वे जमे हुए है, हालाँकि वे चल रहे होंगे, जिस प्रकार बादल चलते है। यह अल्लाह की कारीगरी है, जिसने हर चीज़ को सुदृढ़ किया। निस्संदेह वह उसकी ख़बर रखता है, जो कुछ तुम करते हो
\end{hindi}}
\flushright{\begin{Arabic}
\quranayah[27][89]
\end{Arabic}}
\flushleft{\begin{hindi}
जो कोई सुचरित लेकर आया उसको उससे भी अच्छा प्राप्त होगा; और ऐसे लोग घबराहट से उस दिन निश्चिन्त होंगे
\end{hindi}}
\flushright{\begin{Arabic}
\quranayah[27][90]
\end{Arabic}}
\flushleft{\begin{hindi}
और जो कुचरित लेकर आया तो ऐसे लोगों के मुँह आग में औधे होंगे। (और उनसे कहा जाएगा) "क्या तुम उसके सिवा किसी और चीज़ का बदला पा रहे हो, जो तुम करते रहे हो?"
\end{hindi}}
\flushright{\begin{Arabic}
\quranayah[27][91]
\end{Arabic}}
\flushleft{\begin{hindi}
मुझे तो बस यही आदेश मिला है कि इस नगर (मक्का) के रब की बन्दगी करूँ, जिसने इस आदरणीय ठहराया और उसी की हर चीज़ है। और मुझे आदेश मिला है कि मैं आज्ञाकारी बनकर रहूँ
\end{hindi}}
\flushright{\begin{Arabic}
\quranayah[27][92]
\end{Arabic}}
\flushleft{\begin{hindi}
और यह कि क़ुरआन पढ़कर सुनाऊँ। अब जिस किसी ने संमार्ग ग्रहण किया वह अपने ही लिए संमार्ग ग्रहण करेगा। और जो पथभ्रष्टि रहा तो कह दो, "मैं तो बस एक सचेत करनेवाला ही हूँ।"
\end{hindi}}
\flushright{\begin{Arabic}
\quranayah[27][93]
\end{Arabic}}
\flushleft{\begin{hindi}
और कहो, "सारी प्रशंसा अल्लाह के लिए है। जल्द ही वह तुम्हें अपनी निशानियाँ दिखा देगा और तुम उन्हें पहचान लोगे। और तेरा रब उससे बेख़बर नहीं है, जो कुछ तुम सब कर रहे हो।"
\end{hindi}}
\chapter{Al-Qasas (The Narrative)}
\begin{Arabic}
\Huge{\centerline{\basmalah}}\end{Arabic}
\flushright{\begin{Arabic}
\quranayah[28][1]
\end{Arabic}}
\flushleft{\begin{hindi}
ता॰ सीन॰ मीम॰
\end{hindi}}
\flushright{\begin{Arabic}
\quranayah[28][2]
\end{Arabic}}
\flushleft{\begin{hindi}
(जो आयतें अवतरित हो रही है) वे स्पष्ट। किताब की आयतें हैं
\end{hindi}}
\flushright{\begin{Arabic}
\quranayah[28][3]
\end{Arabic}}
\flushleft{\begin{hindi}
हम उन्हें मूसा और फ़िरऔन का कुछ वृत्तान्त ठीक-ठीक सुनाते है, उन लोगों के लिए जो ईमान लाना चाहें
\end{hindi}}
\flushright{\begin{Arabic}
\quranayah[28][4]
\end{Arabic}}
\flushleft{\begin{hindi}
निस्संदेह फ़िरऔन ने धरती में सरकशी की और उसके निवासियों को विभिन्न गिरोहों में विभक्त कर दिया। उनमें से एक गिरोह को कमज़ोर कर रखा था। वह उनके बेटों की हत्या करता और उनकी स्त्रियों को जीवित रहने देता। निश्चय ही वह बिगाड़ पैदा करनेवालों में से था
\end{hindi}}
\flushright{\begin{Arabic}
\quranayah[28][5]
\end{Arabic}}
\flushleft{\begin{hindi}
और हम यह चाहते थे कि उन लोगों पर उपकार करें, जो धरती में कमज़ोर पड़े थे और उन्हें नायक बनाएँ और उन्हीं को वारिस बनाएँ
\end{hindi}}
\flushright{\begin{Arabic}
\quranayah[28][6]
\end{Arabic}}
\flushleft{\begin{hindi}
और धरती में उन्हें सत्ताधिकार प्रदान करें और उनकी ओर से फ़िरऔन और हामान और उनकी सेनाओं को वह कुछ दिखाएँ, जिसकी उन्हें आशंका थी
\end{hindi}}
\flushright{\begin{Arabic}
\quranayah[28][7]
\end{Arabic}}
\flushleft{\begin{hindi}
हमने मूसा की माँ को संकेत किया कि "उसे दूध पिला फिर जब तुझे उसके विषय में भय हो, तो उसे दरिया में डाल दे और न तुझे कोई भय हो और न तू शोकाकुल हो। हम उसे तेरे पास लौटा लाएँगे और उसे रसूल बनाएँगे।"
\end{hindi}}
\flushright{\begin{Arabic}
\quranayah[28][8]
\end{Arabic}}
\flushleft{\begin{hindi}
अन्ततः फ़िरऔन के लोगों ने उसे उठा लिया, ताकि परिणामस्वरूप वह उनका शत्रु और उनके लिए दुख बने। निश्चय ही फ़िरऔन और हामान और उनकी सेनाओं से बड़ी चूक हुई
\end{hindi}}
\flushright{\begin{Arabic}
\quranayah[28][9]
\end{Arabic}}
\flushleft{\begin{hindi}
फ़िरऔन की स्त्री ने कहा, "यह मेरी और तुम्हारी आँखों की ठंडक है। इसकी हत्या न करो, कदाचित यह हमें लाभ पहुँचाए या हम इसे अपना बेटा ही बना लें।" और वे (परिणाम से) बेख़बर थे
\end{hindi}}
\flushright{\begin{Arabic}
\quranayah[28][10]
\end{Arabic}}
\flushleft{\begin{hindi}
और मूसा की माँ का हृदय विचलित हो गया। निकट था कि वह उसको प्रकट कर देती, यदि हम उसके दिल को इस ध्येय से न सँभालते कि वह मोमिनों में से हो
\end{hindi}}
\flushright{\begin{Arabic}
\quranayah[28][11]
\end{Arabic}}
\flushleft{\begin{hindi}
उसने उसकी बहन से कहा, "तू उसके पीछे-पीछे जा।" अतएव वह उसे दूर ही दूर से देखती रही और वे महसूस नहीं कर रहे थे
\end{hindi}}
\flushright{\begin{Arabic}
\quranayah[28][12]
\end{Arabic}}
\flushleft{\begin{hindi}
हमने पहले ही से दूध पिलानेवालियों को उसपर हराम कर दिया। अतः उसने (मूसा की बहन से) कहा कि "क्या मैं तुम्हें ऐसे घरवालों का पता बताऊँ जो तुम्हारे लिए इसके पालन-पोषण का ज़िम्मा लें और इसके शुभ-चिंतक हों?"
\end{hindi}}
\flushright{\begin{Arabic}
\quranayah[28][13]
\end{Arabic}}
\flushleft{\begin{hindi}
इस प्रकार हम उसे उसकी माँ के पास लौटा लाए, ताकि उसकी आँख ठंड़ी हो और वह शोकाकुल न हो और ताकि वह जान ले कि अल्लाह का वादा सच्चा है, किन्तु उनमें से अधिकतर लोग जानते नहीं
\end{hindi}}
\flushright{\begin{Arabic}
\quranayah[28][14]
\end{Arabic}}
\flushleft{\begin{hindi}
और जब वह अपनी जवानी को पहुँचा और भरपूर हो गया, तो हमने उसे निर्णय-शक्ति और ज्ञान प्रदान किया। और सुकर्मी लोगों को हम इसी प्रकार बदला देते है
\end{hindi}}
\flushright{\begin{Arabic}
\quranayah[28][15]
\end{Arabic}}
\flushleft{\begin{hindi}
उसने नगर में ऐसे समय प्रवेश किया जबकि वहाँ के लोग बेख़बर थे। उसने वहाँ दो आदमियों को लड़ते पाया। यह उसके अपने गिरोह का था और यह उसके शत्रुओं में से था। जो उसके गिरोह में से था उसने उसके मुक़ाबले में, जो उसके शत्रुओं में से था, सहायता के लिए उसे पुकारा। मूसा ने उसे घूँसा मारा और उसका काम तमाम कर दिया। कहा, "यह शैतान की कार्यवाई है। निश्चय ही वह खुला पथभ्रष्ट करनेवाला शत्रु है।"
\end{hindi}}
\flushright{\begin{Arabic}
\quranayah[28][16]
\end{Arabic}}
\flushleft{\begin{hindi}
उसने कहा, "ऐ मेरे रब, मैंने अपने आपपर ज़ुल्म किया। अतः तू मुझे क्षमा कर दे।" अतएव उसने उसे क्षमा कर दिया। निश्चय ही वही बड़ी क्षमाशील, अत्यन्त दयावान है
\end{hindi}}
\flushright{\begin{Arabic}
\quranayah[28][17]
\end{Arabic}}
\flushleft{\begin{hindi}
उसने कहा, "ऐ मेरे रब! जैसे तूने मुझपर अनुकम्पा दर्शाई है, अब मैं भी कभी अपराधियों का सहायक नहीं बनूँगा।"
\end{hindi}}
\flushright{\begin{Arabic}
\quranayah[28][18]
\end{Arabic}}
\flushleft{\begin{hindi}
फिर दूसरे दिन वह नगर में डरता, टोह लेता हुआ प्रविष्ट हुआ। इतने में अचानक क्या देखता है कि वही व्यक्ति जिसने कल उससे सहायता चाही थी, उसे पुकार रहा है। मूसा ने उससे कहा, "तू तो प्रत्यक्ष बहका हुआ व्यक्ति है।"
\end{hindi}}
\flushright{\begin{Arabic}
\quranayah[28][19]
\end{Arabic}}
\flushleft{\begin{hindi}
फिर जब उसने वादा किया कि उस व्यक्ति को पकड़े, जो उन लोगों का शत्रु था, तो वह बोल उठा, "ऐ मूसा, क्या तू चाहता है कि मुझे मार डाले, जिस प्रकार तूने कल एक व्यक्ति को मार डाला? धरती में बस तू निर्दय अत्याचारी बनकर रहना चाहता है और यह नहीं चाहता कि सुधार करनेवाला हो।"
\end{hindi}}
\flushright{\begin{Arabic}
\quranayah[28][20]
\end{Arabic}}
\flushleft{\begin{hindi}
इसके पश्चात एक आदमी नगर के परले सिरे से दौड़ता हुआ आया। उसने कहा, "ऐ मूसा, सरदार तेरे विषय में परामर्श कर रहे हैं कि तुझे मार डालें। अतः तू निकल जा! मैं तेरा हितैषी हूँ।"
\end{hindi}}
\flushright{\begin{Arabic}
\quranayah[28][21]
\end{Arabic}}
\flushleft{\begin{hindi}
फिर वह वहाँ से डरता और ख़तरा भाँपता हुआ निकल खड़ा हुआ। उसने कहा, "ऐ मेरे रब! मुझे ज़ालिम लोगों से छुटकारा दे।"
\end{hindi}}
\flushright{\begin{Arabic}
\quranayah[28][22]
\end{Arabic}}
\flushleft{\begin{hindi}
जब उसने मदयन का रुख़ किया तो कहा, "आशा है, मेरा रब मुझे ठीक रास्ते पर डाल देगा।"
\end{hindi}}
\flushright{\begin{Arabic}
\quranayah[28][23]
\end{Arabic}}
\flushleft{\begin{hindi}
और जब वह मदयन के पानी पर पहुँचा तो उसने उसपर पानी पिलाते लोगों को एक गिरोह पाया। और उनसे हटकर एक ओर दो स्त्रियों को पाया, जो अपन जानवरों को रोक रही थीं। उसने कहा, "तुम्हारा क्या मामला है?" उन्होंने कहा, "हम उस समय तक पानी नहीं पिला सकते, जब तक ये चरवाहे अपने जानवर निकाल न ले जाएँ, और हमारे बाप बहुत ही बूढ़े है।"
\end{hindi}}
\flushright{\begin{Arabic}
\quranayah[28][24]
\end{Arabic}}
\flushleft{\begin{hindi}
तब उसने उन दोनों के लिए पानी पिला दिया। फिर छाया की ओर पलट गया और कहा, "ऐ मेरे रब, जो भलाई भी तू मेरी उतार दे, मैं उसका ज़रूरतमंद हूँ।"
\end{hindi}}
\flushright{\begin{Arabic}
\quranayah[28][25]
\end{Arabic}}
\flushleft{\begin{hindi}
फिर उन दोनों में से एक लजाती हुई उसके पास आई। उसने कहा, "मेरे बाप आपको बुला रहे है, ताकि आपने हमारे लिए (जानवरों को) जो पानी पिलाया है, उसका बदला आपको दें।" फिर जब वह उसके पास पहुँचा और उसे अपने सारे वृत्तान्त सुनाए तो उसने कहा, "कुछ भय न करो। तुम ज़ालिम लोगों से छुटकारा पा गए हो।"
\end{hindi}}
\flushright{\begin{Arabic}
\quranayah[28][26]
\end{Arabic}}
\flushleft{\begin{hindi}
उन दोनों स्त्रियों में से एक ने कहा, "ऐ मेरे बाप! इसको मज़दूरी पर रख लीजिए। अच्छा व्यक्ति, जिसे आप मज़दूरी पर रखें, वही है जो बलवान, अमानतदार हो।"
\end{hindi}}
\flushright{\begin{Arabic}
\quranayah[28][27]
\end{Arabic}}
\flushleft{\begin{hindi}
उसने कहा, "मैं चाहता हूँ कि अपनी इन दोनों बेटियों में से एक का विवाह तुम्हारे साथ इस शर्त पर कर दूँ कि तुम आठ वर्ष तक मेरे यहाँ नौकरी करो। और यदि तुम दस वर्ष पूरे कर दो, तो यह तुम्हारी ओर से होगा। मैं तुम्हें कठिनाई में डालना नहीं चाहता। यदि अल्लाह ने चाहा तो तुम मुझे नेक पाओगे।"
\end{hindi}}
\flushright{\begin{Arabic}
\quranayah[28][28]
\end{Arabic}}
\flushleft{\begin{hindi}
कहा, "यह मेरे और आपके बीच निश्चय हो चुका। इन दोनों अवधियों में से जो भी मैं पूरी कर दूँ, तो तुझपर कोई ज़्यादती नहीं होगी। और जो कुछ हम कह रहे है, उसके विषय में अल्लाह पर भरोसा काफ़ी है।"
\end{hindi}}
\flushright{\begin{Arabic}
\quranayah[28][29]
\end{Arabic}}
\flushleft{\begin{hindi}
फिर जब मूसा ने अवधि पूरी कर दी और अपने घरवालों को लेकर चला तो तूर की ओर उसने एक आग-सी देखी। उसने अपने घरवालों से कहा, "ठहरो, मैंने एक आग का अवलोकन किया है। कदाचित मैं वहाँ से तुम्हारे पास कोई ख़बर ले आऊँ या उस आग से कोई अंगारा ही, ताकि तुम ताप सको।"
\end{hindi}}
\flushright{\begin{Arabic}
\quranayah[28][30]
\end{Arabic}}
\flushleft{\begin{hindi}
फिर जब वह वहाँ पहुँचा तो दाहिनी घाटी के किनारे से शुभ क्षेत्र में वृक्ष से आवाज़ आई, "ऐ मूसा! मैं ही अल्लाह हूँ, सारे संसार का रब!"
\end{hindi}}
\flushright{\begin{Arabic}
\quranayah[28][31]
\end{Arabic}}
\flushleft{\begin{hindi}
और यह कि "डाल दे अपनी लाठी।" फिर जब उसने देखा कि वह बल खा रही है जैसे कोई साँप हो तो वह पीठ फेरकर भागा और पीछे मुड़कर भी न देखा। "ऐ मूसा! आगे आ और भय न कर। निस्संदेह तेरे लिए कोई भय की बात नहीं
\end{hindi}}
\flushright{\begin{Arabic}
\quranayah[28][32]
\end{Arabic}}
\flushleft{\begin{hindi}
अपना हाथ अपने गिरेबान में डाल। बिना किसी ख़राबी के चमकता हुआ निकलेगा। और भय के समय अपनी भुजा को अपने से मिलाए रख। ये दो निशानियाँ है तेरे रब की ओर से फ़िरऔन और उसके दरबारियों के पास लेकर जाने के लिए। निश्चय ही वे बड़े अवज्ञाकारी लोग है।"
\end{hindi}}
\flushright{\begin{Arabic}
\quranayah[28][33]
\end{Arabic}}
\flushleft{\begin{hindi}
उसने कहा, "ऐ मेरे रब! मुझसे उनके एक आदमी की जान गई है। इसलिए मैं डरता हूँ कि वे मुझे मार डालेंगे
\end{hindi}}
\flushright{\begin{Arabic}
\quranayah[28][34]
\end{Arabic}}
\flushleft{\begin{hindi}
मेरे भाई हारून की ज़बान मुझसे बढ़कर धाराप्रवाह है। अतः उसे मेरे साथ सहायक के रूप में भेज कि वह मेरी पुष्टि करे। मुझे भय है कि वे मुझे झुठलाएँगे।"
\end{hindi}}
\flushright{\begin{Arabic}
\quranayah[28][35]
\end{Arabic}}
\flushleft{\begin{hindi}
कहा, "हम तेरे भाई के द्वारा तेरी भुजा मज़बूत करेंगे, और तुम दोनों को एक ओज प्रदान करेंगे कि वे फिर तुम तक न पहुँच सकेंगे। हमारी निशानियों के कारण तुम दोनों और जो तुम्हारे अनुयायी होंगे वे ही प्रभावी होंगे।"
\end{hindi}}
\flushright{\begin{Arabic}
\quranayah[28][36]
\end{Arabic}}
\flushleft{\begin{hindi}
फिर जब मूसा उनके पास हमारी खुली-खुली निशानियाँ लेकर आया तो उन्होंने कहा, "यह तो बस घड़ा हुआ जादू है। हमने तो यह बात अपने अगले बाप-दादा में कभी सुनी ही नहीं।"
\end{hindi}}
\flushright{\begin{Arabic}
\quranayah[28][37]
\end{Arabic}}
\flushleft{\begin{hindi}
मूसा ने कहा, "मेरा रब उस व्यक्ति को भली-भाँति जानता है जो उसके यहाँ से मार्गदर्शन लेकर आया है, और उसको भी जिसके लिए अंतिम घर है। निश्चय ही ज़ालिम सफल नहीं होते।"
\end{hindi}}
\flushright{\begin{Arabic}
\quranayah[28][38]
\end{Arabic}}
\flushleft{\begin{hindi}
फ़िरऔन ने कहा, "ऐ दरबारवालो, मैं तो अपने सिवा तुम्हारे किसी प्रभु को नहीं जानता। अच्छा तो ऐ हामान! तू मेरे लिए ईटें आग में पकवा। फिर मेरे लिए एक ऊँचा महल बना कि मैं मूसा के प्रभु को झाँक आऊँ। मैं तो उसे झूठा समझता हूँ।"
\end{hindi}}
\flushright{\begin{Arabic}
\quranayah[28][39]
\end{Arabic}}
\flushleft{\begin{hindi}
उसने और उसकी सेनाओं ने धरती में नाहक़ घमंड किया और समझा कि उन्हें हमारी ओर लौटना नहीं है
\end{hindi}}
\flushright{\begin{Arabic}
\quranayah[28][40]
\end{Arabic}}
\flushleft{\begin{hindi}
अन्ततः हमने उसे औऱ उसकी सेनाओं को पकड़ लिया और उन्हें गहरे पानी में फेंक दिया। अब देख लो कि ज़ालिमों का कैसा परिणाम हुआ
\end{hindi}}
\flushright{\begin{Arabic}
\quranayah[28][41]
\end{Arabic}}
\flushleft{\begin{hindi}
और हमने उन्हें आग की ओर बुलानेवाले पेशवा बना दिया और क़ियामत के दिन उन्हें कोई सहायता प्राप्त न होगी
\end{hindi}}
\flushright{\begin{Arabic}
\quranayah[28][42]
\end{Arabic}}
\flushleft{\begin{hindi}
और हमने इस दुनिया में उनके पीछे लानत लगा दी और क़ियामत दिन वही बदहाल होंगे
\end{hindi}}
\flushright{\begin{Arabic}
\quranayah[28][43]
\end{Arabic}}
\flushleft{\begin{hindi}
और अगली नस्लों को विनष्ट कर देने के पश्चात हमने मूसा को किताब प्रदान की, लोगों के लिए अन्तर्दृष्टियों की सामग्री, मार्गदर्शन और दयालुता बनाकर, ताकि वे ध्यान दें
\end{hindi}}
\flushright{\begin{Arabic}
\quranayah[28][44]
\end{Arabic}}
\flushleft{\begin{hindi}
तुम तो (नगर के) पश्चिमी किनारे पर नहीं थे, जब हमने मूसा को बात की निर्णित सूचना दी थी, और न तुम गवाहों में से थे
\end{hindi}}
\flushright{\begin{Arabic}
\quranayah[28][45]
\end{Arabic}}
\flushleft{\begin{hindi}
लेकिन हमने बहुत-सी नस्लें उठाईं और उनपर बहुत समय बीत गया। और न तुम मदयनवालों में रहते थे कि उन्हें हमारी आयतें सुना रहे होते, किन्तु रसूलों को भेजनेवाले हम ही रहे है
\end{hindi}}
\flushright{\begin{Arabic}
\quranayah[28][46]
\end{Arabic}}
\flushleft{\begin{hindi}
और तुम तूर के अंचल में भी उपस्थित न थे जब हमने पुकारा था, किन्तु यह तुम्हारे रब की दयालुता है - ताकि तुम ऐसे लोगों को सचेत कर दो जिनके पास तुमसे पहले कोई सचेत करनेवाला नहीं आया, ताकि वे ध्यान दें
\end{hindi}}
\flushright{\begin{Arabic}
\quranayah[28][47]
\end{Arabic}}
\flushleft{\begin{hindi}
(हम रसूल बनाकर न भेजते) यदि यह बात न होती कि जो कुछ उनके हाथ आगे भेज चुके है उसके कारण जब उनपर कोई मुसीबत आए तो वे कहने लगें, "ऐ हमारे रब, तूने क्यों न हमारी ओर कोई रसूल भेजा कि हम तेरी आयतों का (अनुसरण) करते और मोमिन होते?"
\end{hindi}}
\flushright{\begin{Arabic}
\quranayah[28][48]
\end{Arabic}}
\flushleft{\begin{hindi}
फिर जब उनके पास हमारे यहाँ से सत्य आ गया तो वे कहने लगे कि "जो चीज़ मूसा को मिली थी उसी तरह की चीज़ इसे क्यों न मिली?" क्या वे उसका इनकार नहीं कर चुके है, जो इससे पहले मूसा को प्रदान किया गया था? उन्होंने कहा, "दोनों जादू है जो एक-दूसरे की सहायता करते है।" और कहा, "हम तो हरेक का इनकार करते है।"
\end{hindi}}
\flushright{\begin{Arabic}
\quranayah[28][49]
\end{Arabic}}
\flushleft{\begin{hindi}
कहो, "अच्छा तो लाओ अल्लाह के यहाँ से कोई ऐसी किताब, जो इन दोनों से बढ़कर मार्गदर्शन करनेवाली हो कि मैं उसका अनुसरण करूँ, यदि तुम सच्चे हो?"
\end{hindi}}
\flushright{\begin{Arabic}
\quranayah[28][50]
\end{Arabic}}
\flushleft{\begin{hindi}
अब यदि वे तुम्हारी माँग पूरी न करें तो जान लो कि वे केवल अपनी इच्छाओं के पीछे चलते है। और उस व्यक्ति से बढ़कर भटका हुआ कौन होगा जो अल्लाह की ओर से किसी मार्गदर्शन के बिना अपनी इच्छा पर चले? निश्चय ही अल्लाह ज़ालिम लोगों को मार्ग नहीं दिखाता
\end{hindi}}
\flushright{\begin{Arabic}
\quranayah[28][51]
\end{Arabic}}
\flushleft{\begin{hindi}
और हम उनके लिए वाणी बराबर अवतरित करते रहे, शायद वे ध्यान दें
\end{hindi}}
\flushright{\begin{Arabic}
\quranayah[28][52]
\end{Arabic}}
\flushleft{\begin{hindi}
जिन लोगों को हमने इससे पूर्व किताब दी थी, वे इसपर ईमान लाते है
\end{hindi}}
\flushright{\begin{Arabic}
\quranayah[28][53]
\end{Arabic}}
\flushleft{\begin{hindi}
और जब यह उनको पढ़कर सुनाया जाता है तो वे कहते है, "हम इसपर ईमान लाए। निश्चय ही यह सत्य है हमारे रब की ओर से। हम तो इससे पहले ही से मुस्लिम (आज्ञाकारी) हैं।"
\end{hindi}}
\flushright{\begin{Arabic}
\quranayah[28][54]
\end{Arabic}}
\flushleft{\begin{hindi}
ये वे लोग है जिन्हें उनका प्रतिदान दुगना दिया जाएगा, क्योंकि वे जमे रहे और भलाई के द्वारा बुराई को दूर करते है और जो कुछ रोज़ी हमने उन्हें दी हैं, उसमें से ख़र्च करते है
\end{hindi}}
\flushright{\begin{Arabic}
\quranayah[28][55]
\end{Arabic}}
\flushleft{\begin{hindi}
और जब वे व्यर्थ की बात सुनते है तो यह कहते हुए उससे किनारा खींच लेते है कि "हमारे लिए हमारे कर्म है और तुम्हारे लिए तुम्हारे कर्म है। तुमको सलाम है! ज़ाहिलों को हम नहीं चाहते।"
\end{hindi}}
\flushright{\begin{Arabic}
\quranayah[28][56]
\end{Arabic}}
\flushleft{\begin{hindi}
तुम जिसे चाहो राह पर नहीं ला सकते, किन्तु अल्लाह जिसे चाहता है राह दिखाता है, और वही राह पानेवालों को भली-भाँति जानता है
\end{hindi}}
\flushright{\begin{Arabic}
\quranayah[28][57]
\end{Arabic}}
\flushleft{\begin{hindi}
वे कहते है, "यदि हम तुम्हारे साथ इस मार्गदर्शन का अनुसरण करें तो अपने भू-भाग से उचक लिए जाएँगे।" क्या ख़तरों से सुरक्षित हरम में हमने ठिकाना नहीं दिया, जिसकी ओर हमारी ओर से रोज़ी के रूप में हर चीज़ की पैदावार खिंची चली आती है? किन्तु उनमें से अधिकतर जानते नहीं
\end{hindi}}
\flushright{\begin{Arabic}
\quranayah[28][58]
\end{Arabic}}
\flushleft{\begin{hindi}
हमने कितनी ही बस्तियों को विनष्ट कर डाला, जिन्होंने अपनी गुज़र-बसर के संसाधन पर इतराते हुए अकृतज्ञता दिखाई। तो वे है उनके घर, जो उनके बाद आबाद थोड़े ही हुए। अन्ततः हम ही वारिस हुए
\end{hindi}}
\flushright{\begin{Arabic}
\quranayah[28][59]
\end{Arabic}}
\flushleft{\begin{hindi}
तेरा रब तो बस्तियों को विनष्ट करनेवाला नहीं जब तक कि उनकी केन्द्रीय बस्ती में कोई रसूल न भेज दे, जो हमारी आयतें सुनाए। और हम बस्तियों को विनष्ट करनेवाले नहीं सिवाय इस स्थिति के कि वहाँ के रहनेवाले ज़ालिम हों
\end{hindi}}
\flushright{\begin{Arabic}
\quranayah[28][60]
\end{Arabic}}
\flushleft{\begin{hindi}
जो चीज़ भी तुम्हें प्रदान की गई है वह तो सांसारिक जीवन की सामग्री और उसकी शोभा है। और जो कुछ अल्लाह के पास है वह उत्तम और शेष रहनेवाला है, तो क्या तुम बुद्धि से काम नहीं लेते?
\end{hindi}}
\flushright{\begin{Arabic}
\quranayah[28][61]
\end{Arabic}}
\flushleft{\begin{hindi}
भला वह व्यक्ति जिससे हमने अच्छा वादा किया है और वह उसे पानेवाला भी हो, वह उस व्यक्ति की तरह हो सकता है जिसे हमने सांसारिक जीवन की सामग्री दे दी हो, फिर वह क़ियामत के दिन पकड़कर पेश किया जानेवाला हो?
\end{hindi}}
\flushright{\begin{Arabic}
\quranayah[28][62]
\end{Arabic}}
\flushleft{\begin{hindi}
ख़याल करो जिस दिन वह उन्हें पुकारेगा और कहेगा, "कहाँ है मेरे वे साझीदार जिनका तुम्हें दावा था?"
\end{hindi}}
\flushright{\begin{Arabic}
\quranayah[28][63]
\end{Arabic}}
\flushleft{\begin{hindi}
जिनपर बात पूरी हो चुकी होगी, वे कहेंगे, "ऐ हमारे रब! ये वे लोग है जिन्हें हमने बहका दिया था। जैसे हम स्वयं बहके थे, इन्हें भी बहकाया। हमने तेरे आगे स्पष्ट कर दिया कि इनसे हमारा कोई सम्बन्ध नहीं। ये हमारी बन्दगी तो करते नहीं थे?"
\end{hindi}}
\flushright{\begin{Arabic}
\quranayah[28][64]
\end{Arabic}}
\flushleft{\begin{hindi}
कहा जाएगा, "पुकारो, अपने ठहराए हुए साझीदारों को!" तो वे उन्हें पुकारेंगे, किन्तु वे उनको कोई उत्तर न देंगे। और वे यातना देखकर रहेंगे। काश वे मार्ग पानेवाले होते!
\end{hindi}}
\flushright{\begin{Arabic}
\quranayah[28][65]
\end{Arabic}}
\flushleft{\begin{hindi}
और ख़याल करो, जिस दिन वह उन्हें पुकारेगा और कहेगा, "तुमने रसूलों को क्या उत्तर दिया था?"
\end{hindi}}
\flushright{\begin{Arabic}
\quranayah[28][66]
\end{Arabic}}
\flushleft{\begin{hindi}
उस दिन उन्हें बात न सूझेंगी, फिर वे आपस में भी पूछताछ न करेंगे
\end{hindi}}
\flushright{\begin{Arabic}
\quranayah[28][67]
\end{Arabic}}
\flushleft{\begin{hindi}
अलबत्ता जिस किसी ने तौबा कर ली और वह ईमान ले आया और अच्छा कर्म किया, तो आशा है वह सफल होनेवालों में से होगा
\end{hindi}}
\flushright{\begin{Arabic}
\quranayah[28][68]
\end{Arabic}}
\flushleft{\begin{hindi}
तेरा रब पैदा करता है जो कुछ चाहता है और ग्रहण करता है जो चाहता है। उन्हें कोई अधिकार प्राप्त नहीं। अल्लाह महान और उच्च है उस शिर्क से, जो वे करते है
\end{hindi}}
\flushright{\begin{Arabic}
\quranayah[28][69]
\end{Arabic}}
\flushleft{\begin{hindi}
और तेरा रब जानता है जो कुछ उनके सीने छिपाते है और जो कुछ वे लोग व्यक्त करते है
\end{hindi}}
\flushright{\begin{Arabic}
\quranayah[28][70]
\end{Arabic}}
\flushleft{\begin{hindi}
और वही अल्लाह है, उसके सिवा कोई इष्ट -पूज्य नहीं। सारी प्रशंसा उसी के लिए है पहले और पिछले जीवन में फ़ैसले का अधिकार उसी को है और उसी की ओर तुम लौटकर जाओगे
\end{hindi}}
\flushright{\begin{Arabic}
\quranayah[28][71]
\end{Arabic}}
\flushleft{\begin{hindi}
‍‍कहो, "क्या तुमने विचार किया कि यदि अल्लाह क़ियामत के दिन तक सदैव के लिए तुमपर रात कर दे तो अल्लाह के सिवा कौन इष्ट-प्रभु है जो तुम्हारे लिए प्रकाश लाए? तो क्या तुम देखते नहीं?"
\end{hindi}}
\flushright{\begin{Arabic}
\quranayah[28][72]
\end{Arabic}}
\flushleft{\begin{hindi}
कहो, "क्या तुमने विचार किया? यदि अल्लाह क़ियामत के दिन तक सदैव के लिए तुमपर दिन कर दे तो अल्लाह के सिवा दूसरा कौन इष्ट-पूज्य है जो तुम्हारे लिए रात लाए जिसमें तुम आराम पाते हो? तो क्या तुम देखते नहीं?
\end{hindi}}
\flushright{\begin{Arabic}
\quranayah[28][73]
\end{Arabic}}
\flushleft{\begin{hindi}
उसने अपनी दयालुता से तुम्हारे लिए रात और दिन बनाए, ताकि तुम उसमें (रात में) आराम पाओ और ताकि तुम (दिन में) उसका अनुग्रह (रोज़ी) तलाश करो और ताकि तुम कृतज्ञता दिखाओ।"
\end{hindi}}
\flushright{\begin{Arabic}
\quranayah[28][74]
\end{Arabic}}
\flushleft{\begin{hindi}
ख़याल करो, जिस दिन वह उन्हें पुकारेगा और कहेगा, "कहाँ है मेरे वे मेरे साझीदार, जिनका तुम्हे दावा था?"
\end{hindi}}
\flushright{\begin{Arabic}
\quranayah[28][75]
\end{Arabic}}
\flushleft{\begin{hindi}
और हम प्रत्येक समुदाय में से एक गवाह निकाल लाएँगे और कहेंगे, "लाओ अपना प्रमाण।" तब वे जान लेंगे कि सत्य अल्लाह की ओर से है और जो कुछ वे घड़ते थे, वह सब उनसे गुम होकर रह जाएगा
\end{hindi}}
\flushright{\begin{Arabic}
\quranayah[28][76]
\end{Arabic}}
\flushleft{\begin{hindi}
निश्चय ही क़ारून मूसा की क़ौम में से था, फिर उसने उनके विरुद्ध सिर उठाया और हमने उसे इतने ख़जाने दे रखें थे कि उनकी कुंजियाँ एक बलशाली दल को भारी पड़ती थी। जब उससे उसकी क़ौम के लोगों ने कहा, "इतरा मत, अल्लाह इतरानेवालों के पसन्द नही करता
\end{hindi}}
\flushright{\begin{Arabic}
\quranayah[28][77]
\end{Arabic}}
\flushleft{\begin{hindi}
जो कुछ अल्लाह ने तुझे दिया है, उसमें आख़िरत के घर का निर्माण कर और दुनिया में से अपना हिस्सा न भूल, और भलाई कर, जैसा कि अल्लाह ने तेरे साथ भलाई की है, और धरती का बिगाड़ मत चाह। निश्चय ही अल्लाह बिगाड़ पैदा करनेवालों को पसन्द नहीं करता।"
\end{hindi}}
\flushright{\begin{Arabic}
\quranayah[28][78]
\end{Arabic}}
\flushleft{\begin{hindi}
उसने कहा, "मुझे तो यह मेरे अपने व्यक्तिगत ज्ञान के कारण मिला है।" क्या उसने यह नहीं जाना कि अल्लाह उससे पहले कितनी ही नस्लों को विनष्ट कर चुका है जो शक्ति में उससे बढ़-चढ़कर और बाहुल्य में उससे अधिक थीं? अपराधियों से तो (उनकी तबाही के समय) उनके गुनाहों के विषय में पूछा भी नहीं जाता
\end{hindi}}
\flushright{\begin{Arabic}
\quranayah[28][79]
\end{Arabic}}
\flushleft{\begin{hindi}
फिर वह अपनी क़ौम के सामने अपने ठाठ-बाट में निकला। जो लोग सांसारिक जीवन के चाहनेवाले थे, उन्होंने कहा, "क्या ही अच्छा होता जैसा कुछ क़ारून को मिला है, हमें भी मिला होता! वह तो बड़ा ही भाग्यशाली है।"
\end{hindi}}
\flushright{\begin{Arabic}
\quranayah[28][80]
\end{Arabic}}
\flushleft{\begin{hindi}
किन्तु जिनको ज्ञान प्राप्त था, उन्होंने कहा, "अफ़सोस तुमपर! अल्लाह का प्रतिदान उत्तम है, उस व्यक्ति के लिए जो ईमान लाए और अच्छा कर्म करे, और यह बात उन्हीम के दिलों में पड़ती है जो धैर्यवान होते है।"
\end{hindi}}
\flushright{\begin{Arabic}
\quranayah[28][81]
\end{Arabic}}
\flushleft{\begin{hindi}
अन्ततः हमने उसको और उसके घर को धरती में धँसा दिया। और कोई ऐसा गिरोह न हुआ जो अल्लाह के मुक़ाबले में उसकी सहायता करता, और न वह स्वयं अपना बचाव कर सका
\end{hindi}}
\flushright{\begin{Arabic}
\quranayah[28][82]
\end{Arabic}}
\flushleft{\begin{hindi}
अब वही लोग, जो कल उसके पद की कामना कर रहे थे, कहने लगें, "अफ़सोस हम भूल गए थे कि अल्लाह अपने बन्दों में से जिसके लिए चाहता है रोज़ी कुशादा करता है और जिसे चाहता है नपी-तुली देता है। यदि अल्लाह ने हमपर उपकार न किया होता तो हमें भी धँसा देता। अफ़सोस हम भूल गए थे कि इनकार करनेवाले सफल नहीं हुआ करते।"
\end{hindi}}
\flushright{\begin{Arabic}
\quranayah[28][83]
\end{Arabic}}
\flushleft{\begin{hindi}
आख़िरत का घर हम उन लोगों के लिए ख़ास कर देंगे जो न धरती में अपनी बड़ाई चाहते है और न बिगाड़। परिणाम तो अन्ततः डर रखनेवालों के पक्ष में है
\end{hindi}}
\flushright{\begin{Arabic}
\quranayah[28][84]
\end{Arabic}}
\flushleft{\begin{hindi}
जो कोई अच्छा आचारण लेकर आया उसे उससे उत्तम प्राप्त होगा, और जो बुरा आचरण लेकर आया तो बुराइयाँ करनेवालों को तो वस वही मिलेगा जो वे करते थे
\end{hindi}}
\flushright{\begin{Arabic}
\quranayah[28][85]
\end{Arabic}}
\flushleft{\begin{hindi}
जिसने इस क़ुरआन की ज़िम्मेदारी तुमपर डाली है, वह तुम्हें उसके (अच्छे) अंजाम तक ज़रूर पहुँचाएगा। कहो, "मेरा रब उसे भली-भाँति जानता है जो मार्गदर्शन लेकर आया, और उसे भी जो खुली गुमराही में पड़ा है।"
\end{hindi}}
\flushright{\begin{Arabic}
\quranayah[28][86]
\end{Arabic}}
\flushleft{\begin{hindi}
तुम तो इसकी आशा नहीं रखते थे कि तुम्हारी ओर किताब उतारी जाएगी। इसकी संभावना तो केवल तुम्हारे रब की दयालुता के कारण हुई। अतः तुम इनकार करनेवालों के पृष्ठपोषक न बनो
\end{hindi}}
\flushright{\begin{Arabic}
\quranayah[28][87]
\end{Arabic}}
\flushleft{\begin{hindi}
और वे तुम्हें अल्लाह की आयतों से रोक न पाएँ, इसके पश्चात कि वे तुमपर अवतरित हो चुकी है। और अपने रब की ओर बुलाओ और बहुदेववादियों में कदापि सम्मिलित न होना
\end{hindi}}
\flushright{\begin{Arabic}
\quranayah[28][88]
\end{Arabic}}
\flushleft{\begin{hindi}
और अल्लाह के साथ किसी और इष्ट-पूज्य को न पुकारना। उसके सिवा कोई इष्ट-पूज्य नहीं। हर चीज़ नाशवान है सिवास उसके स्वरूप के। फ़ैसला और आदेश का अधिकार उसी को प्राप्त है और उसी की ओर तुम सबको लौटकर जाना है
\end{hindi}}
\chapter{Al-'Ankabut (The Spider)}
\begin{Arabic}
\Huge{\centerline{\basmalah}}\end{Arabic}
\flushright{\begin{Arabic}
\quranayah[29][1]
\end{Arabic}}
\flushleft{\begin{hindi}
अलिफ़॰ लाम॰ मीम॰
\end{hindi}}
\flushright{\begin{Arabic}
\quranayah[29][2]
\end{Arabic}}
\flushleft{\begin{hindi}
क्या लोगों ने यह समझ रखा है कि वे इतना कह देने मात्र से छोड़ दिए जाएँगे कि "हम ईमान लाए" और उनकी परीक्षा न की जाएगी?
\end{hindi}}
\flushright{\begin{Arabic}
\quranayah[29][3]
\end{Arabic}}
\flushleft{\begin{hindi}
हालाँकि हम उन लोगों की परीक्षा कर चुके है जो इनसे पहले गुज़र चुके है। अल्लाह तो उन लोगों को मालूम करके रहेगा, जो सच्चे है। और वह झूठों को भी मालूम करके रहेगा
\end{hindi}}
\flushright{\begin{Arabic}
\quranayah[29][4]
\end{Arabic}}
\flushleft{\begin{hindi}
या उन लोगों ने, जो बुरे कर्म करते है, यह समझ रखा है कि वे हमारे क़ाबू से बाहर निकल जाएँगे? बहुत बुरा है जो फ़ैसला वे कर रहे है
\end{hindi}}
\flushright{\begin{Arabic}
\quranayah[29][5]
\end{Arabic}}
\flushleft{\begin{hindi}
जो व्यक्ति अल्लाह से मिलने का आशा रखता है तो अल्लाह का नियत समय तो आने ही वाला है। और वह सब कुछ सुनता, जानता है
\end{hindi}}
\flushright{\begin{Arabic}
\quranayah[29][6]
\end{Arabic}}
\flushleft{\begin{hindi}
और जो व्यक्ति (अल्लाह के मार्ग में) संघर्ष करता है वह तो स्वयं अपने ही लिए संघर्ष करता है। निश्चय ही अल्लाह सारे संसार से निस्पृह है
\end{hindi}}
\flushright{\begin{Arabic}
\quranayah[29][7]
\end{Arabic}}
\flushleft{\begin{hindi}
और जो लोग ईमान लाए और उन्होंने अच्छा कर्म किए हम उनसे उनकी बुराइयों को दूर कर देंगे और उन्हें अवश्य ही उसका प्रतिदान प्रदान करेंगे, जो कुछ अच्छे कर्म वे करते रहे होंगे
\end{hindi}}
\flushright{\begin{Arabic}
\quranayah[29][8]
\end{Arabic}}
\flushleft{\begin{hindi}
और हमने मनुष्यों को अपने माँ-बाप के साथ अच्छा व्यवहार करने की ताकीद की है। किन्तु यदि वे तुमपर ज़ोर डालें कि तू किसी ऐसी चीज़ को मेरा साक्षी ठहराए, जिसका तुझे कोई ज्ञान नहीं, तो उनकी बात न मान। मेरी ही ओर तुम सबको पलटकर आना है, फिर मैं तुम्हें बता दूँगा जो कुछ कुम करते रहे होगे
\end{hindi}}
\flushright{\begin{Arabic}
\quranayah[29][9]
\end{Arabic}}
\flushleft{\begin{hindi}
और जो लोग ईमान लाए और उन्होंने अच्छे कर्म किए हम उन्हें अवश्य अच्छे लोगों में सम्मिलित करेंगे
\end{hindi}}
\flushright{\begin{Arabic}
\quranayah[29][10]
\end{Arabic}}
\flushleft{\begin{hindi}
लोगों में ऐसे भी है जो कहते है कि "हम अल्लाह पर ईमान लाए," किन्तु जब अल्लाह के मामले में वे सताए गए तो उन्होंने लोगों की ओर से आई हुई आज़माइश को अल्लाह की यातना समझ लिया। अब यदि तेरे रब की ओर से सहायता पहुँच गई तो कहेंगे, "हम तो तुम्हारे साथ थे।" क्या जो कुछ दुनियावालों के सीनों में है उसे अल्लाह भली-भाँति नहीं जानता?
\end{hindi}}
\flushright{\begin{Arabic}
\quranayah[29][11]
\end{Arabic}}
\flushleft{\begin{hindi}
और अल्लाह तो उन लोगों को मालूम करके रहेगा जो ईमान लाए, और वह कपटाचारियों को भी मालूम करके रहेगा
\end{hindi}}
\flushright{\begin{Arabic}
\quranayah[29][12]
\end{Arabic}}
\flushleft{\begin{hindi}
और इनकार करनेवाले ईमान लानेवालों से कहते है, "तुम हमारे मार्ग पर चलो, हम तुम्हारी ख़ताओं का बोझ उठा लेंगे।" हालाँकि वे उनकी ख़ताओं में से कुछ भी उठानेवाले नहीं है। वे निश्चय ही झूठे है
\end{hindi}}
\flushright{\begin{Arabic}
\quranayah[29][13]
\end{Arabic}}
\flushleft{\begin{hindi}
हाँ, अवश्य ही वे अपने बोझ भी उठाएँगे और अपने बोझों के साथ और बहुत-से बोझ भी। और क़ियामत के दिन अवश्य उनसे उसके विषय में पूछा जाएगा जो कुछ झूठ वे घड़ते रहे होंगे
\end{hindi}}
\flushright{\begin{Arabic}
\quranayah[29][14]
\end{Arabic}}
\flushleft{\begin{hindi}
हमने नूह को उसकी क़ौम की ओर भेजा। और वह पचास साल कम एक हजार वर्ष उनके बीच रहा। अन्ततः उनको तूफ़ान ने इस दशा में आ पकड़ा कि वे अत्याचारी था
\end{hindi}}
\flushright{\begin{Arabic}
\quranayah[29][15]
\end{Arabic}}
\flushleft{\begin{hindi}
फिर उसको और नौकावालों को हमने बचा लिया और उसे सारे संसार के लिए एक निशानी बना दिया
\end{hindi}}
\flushright{\begin{Arabic}
\quranayah[29][16]
\end{Arabic}}
\flushleft{\begin{hindi}
और इबराहीम को भी भेजा, जबकि उसने अपनी क़ौम के लोगों से कहा, "अल्लाह की बन्दगी करो और उसका डर रखो। यह तुम्हारे लिए अच्छा है, यदि तुम जानो
\end{hindi}}
\flushright{\begin{Arabic}
\quranayah[29][17]
\end{Arabic}}
\flushleft{\begin{hindi}
तुम तो अल्लाह से हटकर बस मूर्तियों को पूज रहे हो और झूठ घड़ रहे हो। तुम अल्लाह से हटकर जिनको पूजते हो वे तुम्हारे लिए रोज़ी का भी अधिकार नहीं रखते। अतः तुम अल्लाह ही के यहाँ रोज़ी तलाश करो और उसी की बन्दगी करो और उसके आभारी बनो। तुम्हें उसी की ओर लौटकर जाना है
\end{hindi}}
\flushright{\begin{Arabic}
\quranayah[29][18]
\end{Arabic}}
\flushleft{\begin{hindi}
और यदि तुम झुठलाते हो तो तुमसे पहले कितने ही समुदाय झुठला चुके है। रसूल पर तो बस केवल स्पष्ट रूप से (सत्य संदेश) पहुँचा देने की ज़िम्मेदारी है।"
\end{hindi}}
\flushright{\begin{Arabic}
\quranayah[29][19]
\end{Arabic}}
\flushleft{\begin{hindi}
क्या उन्होंने देखा नहीं कि अल्लाह किस प्रकार पैदाइश का आरम्भ करता है और फिर उसकी पुनरावृत्ति करता है? निस्संदेह यह अल्लाह के लिए अत्यन्त सरल है
\end{hindi}}
\flushright{\begin{Arabic}
\quranayah[29][20]
\end{Arabic}}
\flushleft{\begin{hindi}
कहो कि, "धरती में चलो-फिरो और देखो कि उसने किस प्रकार पैदाइश का आरम्भ किया। फिर अल्लाह पश्चात्वर्ती उठान उठाएगा। निश्चय ही अल्लाह को हर चीज़ की सामर्थ्य प्राप्त है
\end{hindi}}
\flushright{\begin{Arabic}
\quranayah[29][21]
\end{Arabic}}
\flushleft{\begin{hindi}
वह जिसे चाहे यातना दे और जिसपर चाहे दया करे। और उसी की ओर तुम्हें पलटकर जाना है।"
\end{hindi}}
\flushright{\begin{Arabic}
\quranayah[29][22]
\end{Arabic}}
\flushleft{\begin{hindi}
तुम न तो धरती में क़ाबू से बाहर निकल सकते हो और न आकाश में। और अल्लाह से हटकर न तो तुम्हारा कोई मित्र है और न सहायक
\end{hindi}}
\flushright{\begin{Arabic}
\quranayah[29][23]
\end{Arabic}}
\flushleft{\begin{hindi}
और जिन लोगों ने अल्लाह की आयतों और उससे मिलने का इनकार किया, वही लोग है जो मेरी दयालुता से निराश हुए और वही है जिनके लिए दुखद यातना है। -
\end{hindi}}
\flushright{\begin{Arabic}
\quranayah[29][24]
\end{Arabic}}
\flushleft{\begin{hindi}
फिर उनकी क़ौम के लोगों का उत्तर इसके सिवा और कुछ न था कि उन्होंने कहा, "मार डालो उसे या जला दो उसे!" अंततः अल्लाह ने उसको आग से बचा लिया। निश्चय ही इसमें उन लोगों के लिए निशानियाँ है, जो ईमान लाएँ
\end{hindi}}
\flushright{\begin{Arabic}
\quranayah[29][25]
\end{Arabic}}
\flushleft{\begin{hindi}
और उसने कहा, "अल्लाह से हटकर तुमने कुछ मूर्तियों को केवल सांसारिक जीवन में अपने पारस्परिक प्रेम के कारण पकड़ रखा है। फिर क़ियामत के दिन तुममें से एक-दूसरे का इनकार करेगा और तुममें से एक-दूसरे पर लानत करेगा। तुम्हारा ठौर-ठिकाना आग है और तुम्हारा कोई सहायक न होगा।"
\end{hindi}}
\flushright{\begin{Arabic}
\quranayah[29][26]
\end{Arabic}}
\flushleft{\begin{hindi}
फिर लूत ने उसकी बात मानी औऱ उसने कहा, "निस्संदेह मैं अपने रब की ओर हिजरत करता हूँ। निस्संदेह वह अत्यन्त प्रभुत्वशाली, तत्वदर्शी है।"
\end{hindi}}
\flushright{\begin{Arabic}
\quranayah[29][27]
\end{Arabic}}
\flushleft{\begin{hindi}
और हमने उसे इसहाक़ और याक़ूब प्रदान किए और उसकी संतति में नुबूवत (पैग़म्बरी) और किताब का सिलसिला जारी किया और हमने उसे संसार में भी उसका अच्छा प्रतिदान प्रदान किया। और निश्चय ही वह आख़िरत में अच्छे लोगों में से होगा
\end{hindi}}
\flushright{\begin{Arabic}
\quranayah[29][28]
\end{Arabic}}
\flushleft{\begin{hindi}
और हमने लूत को भेजा, जबकि उसने अपनी क़ौम के लोगों से कहा, "तुम जो वह अश्लील कर्म करते हो, जिसे तुमसे पहले सारे संसार में किसी ने नहीं किया
\end{hindi}}
\flushright{\begin{Arabic}
\quranayah[29][29]
\end{Arabic}}
\flushleft{\begin{hindi}
क्या तुम पुरुषों के पास जाते हो और बटमारी करते हो औऱ अपनी मजलिस में बुरा कर्म करते हो?" फिर उसकी क़ौम के लोगों का उत्तर बस यही था कि उन्होंने कहा, "ले आ हमपर अल्लाह की यातना, यदि तू सच्चा है।"
\end{hindi}}
\flushright{\begin{Arabic}
\quranayah[29][30]
\end{Arabic}}
\flushleft{\begin{hindi}
उसने कहास "ऐ मेरे रब! बिगाड़ पैदा करनेवाले लोगों के मुक़ावले में मेरी सहायता कर।"
\end{hindi}}
\flushright{\begin{Arabic}
\quranayah[29][31]
\end{Arabic}}
\flushleft{\begin{hindi}
हमारे भेजे हुए जब इबराहीम के पास शुभ सूचना लेकर आए तो उन्होंने कहा, "हम इस बस्ती के लोगों को विनष्ट करनेवाले है। निस्संदेह इस बस्ती के लोग ज़ालिम है।"
\end{hindi}}
\flushright{\begin{Arabic}
\quranayah[29][32]
\end{Arabic}}
\flushleft{\begin{hindi}
उसने कहाँ, "वहाँ तो लूत मौजूद है।" वे बोले, "जो कोई भी वहाँ है, हम भली-भाँति जानते है। हम उसको और उसके घरवालों को बचा लेंगे, सिवाय उसकी स्त्री के। वह पीछे रह जानेवालों में से है।"
\end{hindi}}
\flushright{\begin{Arabic}
\quranayah[29][33]
\end{Arabic}}
\flushleft{\begin{hindi}
जब यह हुआ कि हमारे भेजे हुए लूत के पास आए तो उनका आना उसे नागवार हुआ और उनके प्रति दिल को तंग पाया। किन्तु उन्होंने कहा, "डरो मत और न शोकाकुल हो। हम तुम्हें और तुम्हारे घरवालों को बचा लेंगे सिवाय तुम्हारी स्त्री के। वह पीछे रह जानेवालों में से है
\end{hindi}}
\flushright{\begin{Arabic}
\quranayah[29][34]
\end{Arabic}}
\flushleft{\begin{hindi}
निश्चय ही हम इस बस्ती के लोगों पर आकाश से एक यातना उतारनेवाले है, इस कारण कि वे बन्दगी की सीमा से निकलते रहे है।"
\end{hindi}}
\flushright{\begin{Arabic}
\quranayah[29][35]
\end{Arabic}}
\flushleft{\begin{hindi}
और हमने उस बस्ती से प्राप्त होनेवाली एक खुली निशानी उन लोगों के लिए छोड़ दी है, जो बुद्धि से काम लेना चाहे
\end{hindi}}
\flushright{\begin{Arabic}
\quranayah[29][36]
\end{Arabic}}
\flushleft{\begin{hindi}
और मदयन की ओर उनके भाई शुऐब को भेजा। उसने कहा, "ऐ मेरी क़ौम के लोगो, अल्लाह की बन्दगी करो। और अंतिम दिन की आशा रखो और धरती में बिगाड़ फैलाते मत फिरो।"
\end{hindi}}
\flushright{\begin{Arabic}
\quranayah[29][37]
\end{Arabic}}
\flushleft{\begin{hindi}
किन्तु उन्होंने उसे झुठला दिया। अन्ततः भूकम्प ने उन्हें आ लिया। और वे अपने घरों में औंधे पड़े रह गए
\end{hindi}}
\flushright{\begin{Arabic}
\quranayah[29][38]
\end{Arabic}}
\flushleft{\begin{hindi}
और आद और समूद को भी हमने विनष्ट किया। और उनके घरों और बस्तियों के अवशेषों से तुमपर स्पष्ट हो चुका है। शैतान ने उनके कर्मों को उनके लिए सुहाना बना दिया और उन्हें संमार्ग से रोक दिया। यद्यपि वे बड़े तीक्ष्ण स्पष्ट वाले थे
\end{hindi}}
\flushright{\begin{Arabic}
\quranayah[29][39]
\end{Arabic}}
\flushleft{\begin{hindi}
और क़ारून और फ़िरऔन और हामान को हमने विनष्ट किया। मूसा उनके पास खुली निशानियाँ लेकर आया। किन्तु उन्होंने धरती में घमंड किया, हालाँकि वे हमसे निकल जानेवाले न थे
\end{hindi}}
\flushright{\begin{Arabic}
\quranayah[29][40]
\end{Arabic}}
\flushleft{\begin{hindi}
अन्ततः हमने हरेक को उसके अपने गुनाह के कारण पकड़ लिया। फिर उनमें से कुछ पर तो हमने पथराव करनेवाली वायु भेजी और उनमें से कुछ को एक प्रचंड चीत्कार न आ लिया। और उनमें से कुछ को हमने धरती में धँसा दिया। और उनमें से कुछ को हमने डूबो दिया। अल्लाह तो ऐसा न था कि उनपर ज़ुल्म करता, किन्तु वे स्वयं अपने आपपर ज़ुल्म कर रहे थे
\end{hindi}}
\flushright{\begin{Arabic}
\quranayah[29][41]
\end{Arabic}}
\flushleft{\begin{hindi}
जिन लोगों ने अल्लाह से हटकर अपने दूसरे संरक्षक बना लिए है उनकी मिसाल मकड़ी जैसी है, जिसने अपना एक घर बनाया, और यह सच है कि सब घरों से कमज़ोर घर मकड़ी का घर ही होता है। क्या ही अच्छा होता कि वे जानते!
\end{hindi}}
\flushright{\begin{Arabic}
\quranayah[29][42]
\end{Arabic}}
\flushleft{\begin{hindi}
निस्संदेह अल्लाह उन चीज़ों को भली-भाँति जानता है, जिन्हें ये उससे हटकर पुकारते है। वह तो अत्यन्त प्रभुत्वशाली, तत्वदर्शी है
\end{hindi}}
\flushright{\begin{Arabic}
\quranayah[29][43]
\end{Arabic}}
\flushleft{\begin{hindi}
ये मिसालें हम लोगों के लिए पेश करते है, परन्तु इनको ज्ञानवान ही समझते है
\end{hindi}}
\flushright{\begin{Arabic}
\quranayah[29][44]
\end{Arabic}}
\flushleft{\begin{hindi}
अल्लाह ने आकाशों और धरती को सत्य के साथ पैदा किया। निश्चय ही इसमें ईमानवालों के लिए एक बड़ी निशानी है
\end{hindi}}
\flushright{\begin{Arabic}
\quranayah[29][45]
\end{Arabic}}
\flushleft{\begin{hindi}
उस किताब को पढ़ो जो तुम्हारी ओर प्रकाशना के द्वारा भेजी गई है, और नमाज़ का आयोजन करो। निस्संदेह नमाज़ अश्लीलता और बुराई से रोकती है। और अल्लाह का याद करना तो बहुत बड़ी चीज़ है। अल्लाह जानता है जो कुछ तुम रचते और बनाते हो
\end{hindi}}
\flushright{\begin{Arabic}
\quranayah[29][46]
\end{Arabic}}
\flushleft{\begin{hindi}
और किताबवालों से बस उत्तम रीति ही से वाद-विवाद करो - रहे वे लोग जो उनमें ज़ालिम हैं, उनकी बात दूसरी है - और कहो - "हम ईमान लाए उस चीज़ पर जो अवतरित हुई और तुम्हारी ओर भी अवतरित हुई। और हमारा पूज्य और तुम्हारा पूज्य अकेला ही है और हम उसी के आज्ञाकारी है।"
\end{hindi}}
\flushright{\begin{Arabic}
\quranayah[29][47]
\end{Arabic}}
\flushleft{\begin{hindi}
इसी प्रकार हमने तुम्हारी ओर किताब अवतरित की है, तो जिन्हें हमने किताब प्रदान की है वे उसपर ईमान लाएँगे। उनमें से कुछ उसपर ईमान ला भी रहे है। हमारी आयतों का इनकार तो केवल न माननेवाले ही करते है
\end{hindi}}
\flushright{\begin{Arabic}
\quranayah[29][48]
\end{Arabic}}
\flushleft{\begin{hindi}
इससे पहले तुम न कोई किताब पढ़ते थे और न उसे अपने हाथ से लिखते ही थे। ऐसा होता तो ये मिथ्यावादी सन्देह में पड़ सकते थे
\end{hindi}}
\flushright{\begin{Arabic}
\quranayah[29][49]
\end{Arabic}}
\flushleft{\begin{hindi}
नहीं, बल्कि वे तो उन लोगों के सीनों में विद्यमान खुली निशानियाँ है, जिन्हें ज्ञान प्राप्त हुआ है। हमारी आयतों का इनकार तो केवल ज़ालिम ही करते है
\end{hindi}}
\flushright{\begin{Arabic}
\quranayah[29][50]
\end{Arabic}}
\flushleft{\begin{hindi}
उनका कहना है कि "उसपर उसके रब की ओर से निशानियाँ क्यों नहीं अवतरित हुई?" कह दो, "निशानियाँ तो अल्लाह ही के पास है। मैं तो केवल स्पष्ट रूप से सचेत करनेवाला हूँ।"
\end{hindi}}
\flushright{\begin{Arabic}
\quranayah[29][51]
\end{Arabic}}
\flushleft{\begin{hindi}
क्या उनके लिए यह पर्याप्त नहीं कि हमने तुमपर किताब अवतरित की, जो उन्हें पढ़कर सुनाई जाती है? निस्संदेह उसमें उन लोगों के लिए दयालुता है और अनुस्मृति है जो ईमान लाएँ
\end{hindi}}
\flushright{\begin{Arabic}
\quranayah[29][52]
\end{Arabic}}
\flushleft{\begin{hindi}
कह दो, "मेरे और तुम्हारे बीच अल्लाह गवाह के रूप में काफ़ी है।" वह जानता है जो कुछ आकाशों और धरती में है। जो लोग असत्य पर ईमान लाए और अल्लाह का इनकार किया वही है जो घाटे में है
\end{hindi}}
\flushright{\begin{Arabic}
\quranayah[29][53]
\end{Arabic}}
\flushleft{\begin{hindi}
वे तुमसे यातना के लिए जल्दी मचा रहे है। यदि इसका एक नियत समय न होता तो उनपर अवश्य ही यातना आ जाती। वह तो अचानक उनपर आकर रहेगी कि उन्हें ख़बर भी न होगी
\end{hindi}}
\flushright{\begin{Arabic}
\quranayah[29][54]
\end{Arabic}}
\flushleft{\begin{hindi}
वे तुमसे यातना के लिए जल्दी मचा रहे है, हालाँकि जहन्नम इनकार करनेवालों को अपने घेरे में लिए हुए है
\end{hindi}}
\flushright{\begin{Arabic}
\quranayah[29][55]
\end{Arabic}}
\flushleft{\begin{hindi}
जिस दिन यातना उन्हें उनके ऊपर से ढाँक लेगी और उनके पाँव के नीचे से भी, और वह कहेगा, "चखो उसका मज़ा जो कुछ तुम करते रहे हो!"
\end{hindi}}
\flushright{\begin{Arabic}
\quranayah[29][56]
\end{Arabic}}
\flushleft{\begin{hindi}
ऐ मेरे बन्दों, जो ईमान लाए हो! निस्संदेह मेरी धरती विशाल है। अतः तुम मेरी ही बन्दगी करो
\end{hindi}}
\flushright{\begin{Arabic}
\quranayah[29][57]
\end{Arabic}}
\flushleft{\begin{hindi}
प्रत्येक जीव को मृत्यु का स्वाद चखना है। फिर तुम हमारी ओर वापस लौटोगे
\end{hindi}}
\flushright{\begin{Arabic}
\quranayah[29][58]
\end{Arabic}}
\flushleft{\begin{hindi}
जो लोग ईमान लाए और उन्होंने अच्छे कर्म किए उन्हें हम जन्नत की ऊपरी मंज़िल के कक्षों में जगह देंगे, जिनके नीचे नहरें बह रही होंगी। वे उसमें सदैव रहेंगे। क्या ही अच्छा प्रतिदान है कर्म करनेवालों का!
\end{hindi}}
\flushright{\begin{Arabic}
\quranayah[29][59]
\end{Arabic}}
\flushleft{\begin{hindi}
जिन्होंने धैर्य से काम लिया और जो अपने रब पर भरोसा रखते है
\end{hindi}}
\flushright{\begin{Arabic}
\quranayah[29][60]
\end{Arabic}}
\flushleft{\begin{hindi}
कितने ही चलनेवाले जीवधारी है, जो अपनी रोज़ी उठाए नहीं फिरते। अल्लाह ही उन्हें रोज़ी देता है और तुम्हें भी! वह सब कुछ सुनता, जानता है
\end{hindi}}
\flushright{\begin{Arabic}
\quranayah[29][61]
\end{Arabic}}
\flushleft{\begin{hindi}
और यदि तुम उनसे पूछो कि "किसने आकाशों और धरती को पैदा किया और सूर्य और चन्द्रमा को काम में लगाया?" तो वे बोल पड़ेगे, "अल्लाह ने!" फिर वे किधर उलटे फिरे जाते है?
\end{hindi}}
\flushright{\begin{Arabic}
\quranayah[29][62]
\end{Arabic}}
\flushleft{\begin{hindi}
अल्लाह अपने बन्दों में से जिसके लिए चाहता है आजीविका विस्तीर्ण कर देता है और जिसके लिए चाहता है नपी-तुली कर देता है। निस्संदेह अल्लाह हरेक चीज़ को भली-भाँति जानता है
\end{hindi}}
\flushright{\begin{Arabic}
\quranayah[29][63]
\end{Arabic}}
\flushleft{\begin{hindi}
और यदि तुम उनसे पूछो कि "किसने आकाश से पानी बरसाया; फिर उसके द्वारा धरती को उसके मुर्दा हो जाने के पश्चात जीवित किया?" तो वे बोल पड़ेंगे, "अल्लाह ने!" कहो, "सारी प्रशंसा अल्लाह ही के लिए है।" किन्तु उनमें से अधिकतर बुद्धि से काम नहीं लेते
\end{hindi}}
\flushright{\begin{Arabic}
\quranayah[29][64]
\end{Arabic}}
\flushleft{\begin{hindi}
और यह सांसारिक जीवन तो केवल दिल का बहलावा और खेल है। निस्संदेह पश्चात्वर्ती घर (का जीवन) ही वास्तविक जीवन है। क्या ही अच्छा होता कि वे जानते!
\end{hindi}}
\flushright{\begin{Arabic}
\quranayah[29][65]
\end{Arabic}}
\flushleft{\begin{hindi}
जब वे नौका में सवार होते है तो वे अल्लाह को उसके दीन (आज्ञापालन) के लिए निष्ठा वान होकर पुकारते है। किन्तु जब वह उन्हें बचाकर शु्ष्क भूमि तक ले आता है तो क्या देखते है कि वे लगे (अल्लाह का साथ) साझी ठहराने
\end{hindi}}
\flushright{\begin{Arabic}
\quranayah[29][66]
\end{Arabic}}
\flushleft{\begin{hindi}
ताकि जो कुछ हमने उन्हें दिया है उसके प्रति वे इस तरह कृतघ्नता दिखाएँ, और ताकि इस तरह से मज़े उड़ा ले। अच्छा तो वे शीघ्र ही जान लेंगे
\end{hindi}}
\flushright{\begin{Arabic}
\quranayah[29][67]
\end{Arabic}}
\flushleft{\begin{hindi}
क्या उन्होंने देखा नही कि हमने एक शान्तिमय हरम बनाया, हालाँकि उनके आसपास से लोग उचक लिए जाते है, तो क्या फिर भी वे असत्य पर ईमान रखते है और अल्लाह की अनुकम्पा के प्रति कृतघ्नता दिखलाते है?
\end{hindi}}
\flushright{\begin{Arabic}
\quranayah[29][68]
\end{Arabic}}
\flushleft{\begin{hindi}
उस व्यक्ति से बढ़कर ज़ालिम कौन होगा जो अल्लाह पर थोपकर झूठ घड़े या सत्य को झुठलाए, जबकि वह उसके पास आ चुका हो? क्या इनकार करनेवालों का ठौर-ठिकाना जहन्नम नें नहीं होगा?
\end{hindi}}
\flushright{\begin{Arabic}
\quranayah[29][69]
\end{Arabic}}
\flushleft{\begin{hindi}
रहे वे लोग जिन्होंने हमारे मार्ग में मिलकर प्रयास किया, हम उन्हें अवश्य अपने मार्ग दिखाएँगे। निस्संदेह अल्लाह सुकर्मियों के साथ है
\end{hindi}}
\chapter{Ar-Rum (The Romans)}
\begin{Arabic}
\Huge{\centerline{\basmalah}}\end{Arabic}
\flushright{\begin{Arabic}
\quranayah[30][1]
\end{Arabic}}
\flushleft{\begin{hindi}
अलिफ़॰ लाम॰ मीम॰
\end{hindi}}
\flushright{\begin{Arabic}
\quranayah[30][2]
\end{Arabic}}
\flushleft{\begin{hindi}
रूमी निकटवर्ती क्षेत्र में पराभूत हो गए हैं।
\end{hindi}}
\flushright{\begin{Arabic}
\quranayah[30][3]
\end{Arabic}}
\flushleft{\begin{hindi}
और वे अपने पराभव के पश्चात शीघ्र ही कुछ वर्षों में प्रभावी हो जाएँगे।
\end{hindi}}
\flushright{\begin{Arabic}
\quranayah[30][4]
\end{Arabic}}
\flushleft{\begin{hindi}
हुक्म तो अल्लाह ही का है पहले भी और उसके बाद भी। और उस दिन ईमानवाले अल्लाह की सहायता से प्रसन्न होंगे।
\end{hindi}}
\flushright{\begin{Arabic}
\quranayah[30][5]
\end{Arabic}}
\flushleft{\begin{hindi}
वह जिसकी चाहता है, सहायता करता है। वह अत्यन्त प्रभुत्वशाली, दयावान है
\end{hindi}}
\flushright{\begin{Arabic}
\quranayah[30][6]
\end{Arabic}}
\flushleft{\begin{hindi}
यह अल्लाह का वादा है! अल्लाह अपने वादे का उल्लंघन नहीं करता। किन्तु अधिकतर लोग जानते नहीं
\end{hindi}}
\flushright{\begin{Arabic}
\quranayah[30][7]
\end{Arabic}}
\flushleft{\begin{hindi}
वे सांसारिक जीवन के केवल वाह्य रूप को जानते है। किन्तु आख़िरत की ओर से वे बिलकुल असावधान है
\end{hindi}}
\flushright{\begin{Arabic}
\quranayah[30][8]
\end{Arabic}}
\flushleft{\begin{hindi}
क्या उन्होंने अपने आप में सोच-विचार नहीं किया? अल्लाह ने आकाशों और धरती को और जो कुछ उनके बीच है सत्य के साथ और एक नियत अवधि ही के लिए पैदा किया है। किन्तु बहुत-से लोग अपने प्रभु के मिलन का इनकार करते है
\end{hindi}}
\flushright{\begin{Arabic}
\quranayah[30][9]
\end{Arabic}}
\flushleft{\begin{hindi}
क्या वे धरती में चले-फिरे नहीं कि देखते कि उन लोगों का कैसा परिणाम हुआ जो उनसे पहले थे? वे शक्ति में उनसे अधिक बलवान थे और उन्होंने धरती को उपजाया और उससे कहीं अधिक उसे आबाद किया जितना उन्होंने आबाद किया था। और उनके पास उनके रसूल प्रत्यक्ष प्रमाण लेकर आए। फिर अल्लाह ऐसा न था कि उनपर ज़ुल्म करता। किन्तु वे स्वयं ही अपने आप पर ज़ुल्म करते थे
\end{hindi}}
\flushright{\begin{Arabic}
\quranayah[30][10]
\end{Arabic}}
\flushleft{\begin{hindi}
फिर जिन लोगों ने बुरा किया था उनका परिणाम बुरा हुआ, क्योंकि उन्होंने अल्लाह की आयतों को झुठलाया और उनका उपहास करते रहे
\end{hindi}}
\flushright{\begin{Arabic}
\quranayah[30][11]
\end{Arabic}}
\flushleft{\begin{hindi}
अल्लाह की सृष्टि का आरम्भ करता है। फिर वही उसकी पुनरावृति करता है। फिर उसी की ओर तुम पलटोगे
\end{hindi}}
\flushright{\begin{Arabic}
\quranayah[30][12]
\end{Arabic}}
\flushleft{\begin{hindi}
जिस दिन वह घड़ी आ खड़ी होगी, उस दिन अपराधी एकदम निराश होकर रह जाएँगे
\end{hindi}}
\flushright{\begin{Arabic}
\quranayah[30][13]
\end{Arabic}}
\flushleft{\begin{hindi}
उनके ठहराए हुए साझीदारों में से कोई उनका सिफ़ारिश करनेवाला न होगा और वे स्वयं भी अपने साझीदारों का इनकार करेंगे
\end{hindi}}
\flushright{\begin{Arabic}
\quranayah[30][14]
\end{Arabic}}
\flushleft{\begin{hindi}
और जिस दिन वह घड़ी आ खड़ी होगी, उस दिन वे सब अलग-अलग हो जाएँगे
\end{hindi}}
\flushright{\begin{Arabic}
\quranayah[30][15]
\end{Arabic}}
\flushleft{\begin{hindi}
अतः जो लोग ईमान लाए और उन्होंने अच्छे कर्म किए, वे एक बाग़ में प्रसन्नतापूर्वक रखे जाएँगे
\end{hindi}}
\flushright{\begin{Arabic}
\quranayah[30][16]
\end{Arabic}}
\flushleft{\begin{hindi}
किन्तु जिन लोगों ने इनकार किया और हमारी आयतों और आख़िरत की मुलाक़ात को झुठलाया, वे लाकर यातनाग्रस्त किए जाएँगे
\end{hindi}}
\flushright{\begin{Arabic}
\quranayah[30][17]
\end{Arabic}}
\flushleft{\begin{hindi}
अतः अब अल्लाह की तसबीह करो, जबकि तुम शाम करो और जब सुबह करो।
\end{hindi}}
\flushright{\begin{Arabic}
\quranayah[30][18]
\end{Arabic}}
\flushleft{\begin{hindi}
- और उसी के लिए प्रशंसा है आकाशों और धरती में - और पिछले पहर और जब तुमपर दोपहर हो
\end{hindi}}
\flushright{\begin{Arabic}
\quranayah[30][19]
\end{Arabic}}
\flushleft{\begin{hindi}
वह जीवित को मृत से निकालता है और मृत को जीवित से, और धरती को उसकी मृत्यु के पश्चात जीवन प्रदान करता है। इसी प्रकार तुम भी निकाले जाओगे
\end{hindi}}
\flushright{\begin{Arabic}
\quranayah[30][20]
\end{Arabic}}
\flushleft{\begin{hindi}
और यह उसकी निशानियों में से है कि उसने तुम्हें मिट्टी से पैदा किया। फिर क्या देखते है कि तुम मानव हो, फैलते जा रहे हो
\end{hindi}}
\flushright{\begin{Arabic}
\quranayah[30][21]
\end{Arabic}}
\flushleft{\begin{hindi}
और यह भी उसकी निशानियों में से है कि उसने तुम्हारी ही सहजाति से तुम्हारे लिए जोड़े पैदा किए, ताकि तुम उसके पास शान्ति प्राप्त करो। और उसने तुम्हारे बीच प्रेंम और दयालुता पैदा की। और निश्चय ही इसमें बहुत-सी निशानियाँ है उन लोगों के लिए जो सोच-विचार करते है
\end{hindi}}
\flushright{\begin{Arabic}
\quranayah[30][22]
\end{Arabic}}
\flushleft{\begin{hindi}
और उसकी निशानियों में से आकाशों और धरती का सृजन और तुम्हारी भाषाओं और तुम्हारे रंगों की विविधता भी है। निस्संदेह इसमें ज्ञानवानों के लिए बहुत-सी निशानियाँ है
\end{hindi}}
\flushright{\begin{Arabic}
\quranayah[30][23]
\end{Arabic}}
\flushleft{\begin{hindi}
और उसकी निशानियों में से तुम्हारा रात और दिन का सोना और तुम्हारा उसके अनुग्रह की तलाश करना भी है। निश्चय ही इसमें निशानियाँ है उन लोगों के लिए जो सुनते है
\end{hindi}}
\flushright{\begin{Arabic}
\quranayah[30][24]
\end{Arabic}}
\flushleft{\begin{hindi}
और उसकी निशानियों में से यह भी है कि वह तुम्हें बिजली की चमक भय और आशा उत्पन्न करने के लिए दिखाता है। और वह आकाश से पानी बरसाता है। फिर उसके द्वारा धरती को उसके निर्जीव हो जाने के पश्चात जीवन प्रदान करता है। निस्संदेह इसमें बहुत-सी निशानियाँ है उन लोगों के लिए जो बुद्धि से काम लेते है
\end{hindi}}
\flushright{\begin{Arabic}
\quranayah[30][25]
\end{Arabic}}
\flushleft{\begin{hindi}
और उसकी निशानियों में से यह भी है कि आकाश और धरती उसके आदेश से क़ायम है। फिर जब वह तुम्हे एक बार पुकारकर धरती में से बुलाएगा, तो क्या देखेंगे कि सहसा तुम निकल पड़े
\end{hindi}}
\flushright{\begin{Arabic}
\quranayah[30][26]
\end{Arabic}}
\flushleft{\begin{hindi}
आकाशों और धरती में जो कोई भी उसी का है। प्रत्येक उसी के निष्ठावान आज्ञाकारी है
\end{hindi}}
\flushright{\begin{Arabic}
\quranayah[30][27]
\end{Arabic}}
\flushleft{\begin{hindi}
वही है जो सृष्टि का आरम्भ करता है। फिर वही उसकी पुनरावृत्ति करेगा। और यह उसके लिए अधिक सरल है। आकाशों और धरती में उसी मिसाल (गुण) सर्वोच्च है। और वह अत्यन्त प्रभुत्वशाली, तत्वदर्शी हैं
\end{hindi}}
\flushright{\begin{Arabic}
\quranayah[30][28]
\end{Arabic}}
\flushleft{\begin{hindi}
उसने तुम्हारे लिए स्वयं तुम्हीं में से एक मिसाल पेश की है। क्या जो रोज़ी हमने तुम्हें दी है, उसमें तुम्हारे अधीनस्थों में से, कुछ तुम्हारे साझीदार है कि तुम सब उसमें बराबर के हो, तुम उनका ऐसा डर रखते हो जैसा अपने लोगों का डर रखते हो? - इसप्रकार हम उन लोगों के लिए आयतें खोल-खोलकर प्रस्तुत करते है जो बुद्धि से काम लेते है। -
\end{hindi}}
\flushright{\begin{Arabic}
\quranayah[30][29]
\end{Arabic}}
\flushleft{\begin{hindi}
नहीं, बल्कि ये ज़ालिम तो बिना ज्ञान के अपनी इच्छाओं के पीछे चल पड़े। तो अब कौन उसे मार्ग दिखाएगा जिसे अल्लाह ने भटका दिया हो? ऐसे लोगो का तो कोई सहायक नहीं
\end{hindi}}
\flushright{\begin{Arabic}
\quranayah[30][30]
\end{Arabic}}
\flushleft{\begin{hindi}
अतः एक ओर का होकर अपने रुख़ को 'दीन' (धर्म) की ओर जमा दो, अल्लाह की उस प्रकृति का अनुसरण करो जिसपर उसने लोगों को पैदा किया। अल्लाह की बनाई हुई संरचना बदली नहीं जा सकती। यही सीधा और ठीक धर्म है, किन्तु अधिकतर लोग जानते नहीं।
\end{hindi}}
\flushright{\begin{Arabic}
\quranayah[30][31]
\end{Arabic}}
\flushleft{\begin{hindi}
उसकी ओर रुजू करनेवाले (प्रवृत्त होनेवाले) रहो। और उसका डर रखो और नमाज़ का आयोजन करो और (अल्लाह का) साझी ठहरानेवालों में से न होना,
\end{hindi}}
\flushright{\begin{Arabic}
\quranayah[30][32]
\end{Arabic}}
\flushleft{\begin{hindi}
उन लोगों में से जिन्होंने अपनी दीन (धर्म) को टुकड़े-टुकड़े कर डाला और गिरोहों में बँट गए। हर गिरोह के पास जो कुछ है, उसी में मग्न है
\end{hindi}}
\flushright{\begin{Arabic}
\quranayah[30][33]
\end{Arabic}}
\flushleft{\begin{hindi}
और जब लोगों को कोई तकलीफ़ पहुँचती है तो वे अपने रब को, उसकी ओर रुजू (प्रवृत) होकर पुकारते है। फिर जब वह उन्हें अपनी दयालुता का रसास्वादन करा देता है, तो क्या देखते है कि उनमें से कुछ लोग अपने रब का साझी ठहराने लगे;
\end{hindi}}
\flushright{\begin{Arabic}
\quranayah[30][34]
\end{Arabic}}
\flushleft{\begin{hindi}
ताकि इस प्रकार वे उसके प्रति अकृतज्ञता दिखलाएँ जो कुछ हमने उन्हें दिया है। "अच्छा तो मज़े उड़ा लो, शीघ्र ही तुम जान लोगे।"
\end{hindi}}
\flushright{\begin{Arabic}
\quranayah[30][35]
\end{Arabic}}
\flushleft{\begin{hindi}
(क्या उनके देवताओं ने उनकी सहायता की थी) या हमने उनपर ऐसा कोई प्रमाण उतारा है कि वह उसके हक़ में बोलता हो, जो वे उसके साथ साझी ठहराते है
\end{hindi}}
\flushright{\begin{Arabic}
\quranayah[30][36]
\end{Arabic}}
\flushleft{\begin{hindi}
और जब हम लोगों को दयालुता का रसास्वादन कराते है तो वे उसपर इतराने लगते है; परन्तु जो कुछ उनके हाथों ने आगे भेजा है यदि उसके कारण उनपर कोई विपत्ति आ जाए, तो क्या देखते है कि वे निराश हो रहे है
\end{hindi}}
\flushright{\begin{Arabic}
\quranayah[30][37]
\end{Arabic}}
\flushleft{\begin{hindi}
क्या उन्होंने विचार नहीं किया कि अल्लाह जिसके लिए चाहता है रोज़ी कुशादा कर देता है और जिसके लिए चाहता है नपी-तुली कर देता है? निस्संदेह इसमें उन लोगों के लिए निशानियाँ है, जो ईमान लाएँ
\end{hindi}}
\flushright{\begin{Arabic}
\quranayah[30][38]
\end{Arabic}}
\flushleft{\begin{hindi}
अतः नातेदार को उसका हक़ दो और मुहताज और मुसाफ़िर को भी। यह अच्छा है उनके लिए जो अल्लाह की प्रसन्नता के इच्छुक हों और वही सफल है
\end{hindi}}
\flushright{\begin{Arabic}
\quranayah[30][39]
\end{Arabic}}
\flushleft{\begin{hindi}
तुम जो कुछ ब्याज पर देते हो, ताकि वह लोगों के मालों में सम्मिलित होकर बढ़ जाए, तो वह अल्लाह के यहाँ नहीं बढ़ता। किन्तु जो ज़कात तुमने अल्लाह की प्रसन्नता प्राप्त करने के लिए दी, तो ऐसे ही लोग (अल्लाह के यहाँ) अपना माल बढ़ाते है
\end{hindi}}
\flushright{\begin{Arabic}
\quranayah[30][40]
\end{Arabic}}
\flushleft{\begin{hindi}
अल्लाह ही है जिसने तुम्हें पैदा किया, फिर तुम्हें रोज़ी दी; फिर वह तुम्हें मृत्यु देता है; फिर तुम्हें जीवित करेगा। क्या तुम्हारे ठहराए हुए साझीदारों में भी कोई है, जो इन कामों में से कुछ कर सके? महान और उच्च है वह उसमें जो साझी वे ठहराते है
\end{hindi}}
\flushright{\begin{Arabic}
\quranayah[30][41]
\end{Arabic}}
\flushleft{\begin{hindi}
थल और जल में बिगाड़ फैल गया स्वयं लोगों ही के हाथों की कमाई के कारण, ताकि वह उन्हें उनकी कुछ करतूतों का मज़ा चखाए, कदाचित वे बाज़ आ जाएँ
\end{hindi}}
\flushright{\begin{Arabic}
\quranayah[30][42]
\end{Arabic}}
\flushleft{\begin{hindi}
कहो, "धरती में चल-फिरकर देखो कि उन लोगों का कैसा परिणाम हुआ जो पहले गुज़रे है। उनमें अधिकतर बहुदेववादी ही थे।"
\end{hindi}}
\flushright{\begin{Arabic}
\quranayah[30][43]
\end{Arabic}}
\flushleft{\begin{hindi}
अतः तुम अपना रुख़ सीधे व ठीक धर्म की ओर जमा दो, इससे पहले कि अल्लाह की ओर से वह दिन आ जाए जिसके लिए वापसी नहीं। उस दिन लोग अलग-अलग हो जाएँगे
\end{hindi}}
\flushright{\begin{Arabic}
\quranayah[30][44]
\end{Arabic}}
\flushleft{\begin{hindi}
जिस किसी ने इनकार किया तो उसका इनकार उसी के लिए घातक सिद्ध होगा, और जिन लोगों ने अच्छा कर्म किया वे अपने ही लिए आराम का साधन जुटा रहे है
\end{hindi}}
\flushright{\begin{Arabic}
\quranayah[30][45]
\end{Arabic}}
\flushleft{\begin{hindi}
ताकि वह अपने उदार अनुग्रह से उन लोगों को बदला दे जो ईमान लाए और उन्होंने अच्छे कर्म किए। निश्चय ही वह इनकार करनेवालों को पसन्द नहीं करता। -
\end{hindi}}
\flushright{\begin{Arabic}
\quranayah[30][46]
\end{Arabic}}
\flushleft{\begin{hindi}
और उसकी निशानियों में से यह भी है कि शुभ सूचना देनेवाली हवाएँ भेजता है (ताकि उनके द्वारा तुम्हें वर्षा की शुभ सूचना मिले) और ताकि वह तुम्हें अपनी दयालुता का रसास्वादन कराए और ताकि उसके आदेश से नौकाएँ चलें और ताकि तुम उसका अनुग्रह (रोज़ी) तलाश करो और कदाचित तुम कृतज्ञता दिखलाओ
\end{hindi}}
\flushright{\begin{Arabic}
\quranayah[30][47]
\end{Arabic}}
\flushleft{\begin{hindi}
हम तुमसे पहले कितने ही रसूलों को उनकी क़ौम की ओर भेज चुके है और वे उनके पास खुली निशानियाँ लेकर आए। फिर हम उन लोगों से बदला लेकर रहे जिन्होंने अपराध किया, और ईमानवालों की सहायता करना तो हमपर एक हक़ है
\end{hindi}}
\flushright{\begin{Arabic}
\quranayah[30][48]
\end{Arabic}}
\flushleft{\begin{hindi}
अल्लाह ही है जो हवाओं को भेजता है। फिर वे बादलों को उठाती हैं; फिर जिस तरह चाहता है उन्हें आकाश में फैला देता है और उन्हें परतों और टुकड़ियों का रूप दे देता है। फिर तुम देखते हो कि उनके बीच से वर्षा की बूँदें टपकी चली आती है। फिर जब वह अपने बन्दों में से जिनपर चाहता है, उसे बरसाता है। तो क्या देखते है कि वे हर्षित हो उठे
\end{hindi}}
\flushright{\begin{Arabic}
\quranayah[30][49]
\end{Arabic}}
\flushleft{\begin{hindi}
जबकि इससे पूर्व, इससे पहले कि वह उनपर उतरे, वे बिलकुल निराश थे
\end{hindi}}
\flushright{\begin{Arabic}
\quranayah[30][50]
\end{Arabic}}
\flushleft{\begin{hindi}
अतः देखों अल्लाह की दयालुता के चिन्ह! वह किस प्रकार धरती को उसके मृत हो जाने के पश्चात जीवन प्रदान करता है। निश्चय ही वह मुर्दों को जीवत करनेवाला है, और उसे हर चीज़ का सामर्थ्य प्राप्ती है
\end{hindi}}
\flushright{\begin{Arabic}
\quranayah[30][51]
\end{Arabic}}
\flushleft{\begin{hindi}
किन्तु यदि हम एक दूसरी हवा भेज दें, जिसके प्रभाव से वे उस (खेती) को पीली पड़ी हुई देखें तो इसके पश्चात वे कुफ़्र करने लग जाएँ
\end{hindi}}
\flushright{\begin{Arabic}
\quranayah[30][52]
\end{Arabic}}
\flushleft{\begin{hindi}
अतः तुम मुर्दों को नहीं सुना सकते और न बहरों को अपनी पुकार सुना सकते हो, जबकि वे पीठ फेरे चले जो रहे हों
\end{hindi}}
\flushright{\begin{Arabic}
\quranayah[30][53]
\end{Arabic}}
\flushleft{\begin{hindi}
और न तुम अंधों को उनकी गुमराही से फेरकर मार्ग पर ला सकते हो। तुम तो केवल उन्हीं को सुना सकते हो जो हमारी आयतों पर ईमान लाएँ। तो वही आज्ञाकारी हैं
\end{hindi}}
\flushright{\begin{Arabic}
\quranayah[30][54]
\end{Arabic}}
\flushleft{\begin{hindi}
अल्लाह ही है जिसनें तुम्हें निर्बल पैदा किया, फिर निर्बलता के पश्चात शक्ति प्रदान की; फिर शक्ति के पश्चात निर्बलता औऱ बुढापा दिया। वह जो कुछ चाहता है पैदा करता है। वह जाननेवाला, सामर्थ्यवान है
\end{hindi}}
\flushright{\begin{Arabic}
\quranayah[30][55]
\end{Arabic}}
\flushleft{\begin{hindi}
जिस दिन वह घड़ी आ खड़ी होगी अपराधी क़सम खाएँगे कि वे घड़ी भर से अधिक नहीं ठहरें। इसी प्रकार वे उलटे फिरे चले जाते थे
\end{hindi}}
\flushright{\begin{Arabic}
\quranayah[30][56]
\end{Arabic}}
\flushleft{\begin{hindi}
किन्तु जिन लोगों को ज्ञान और ईमान प्रदान हुआ, वे कहते, "अल्लाह के लेख में तो तुम जीवित होकर उठने के दिन ठहरे रहे हो। तो यही जीवित हो उठाने का दिन है। किन्तु तुम जानते न थे।"
\end{hindi}}
\flushright{\begin{Arabic}
\quranayah[30][57]
\end{Arabic}}
\flushleft{\begin{hindi}
अतः उस दिन ज़ुल्म करनेवालों को उनका कोई उज़्र (सफाई पेश करना) काम न आएगा और न उनसे यह चाहा जाएगा कि वे किसी यत्न से (अल्लाह के) प्रकोप को टाल सकें
\end{hindi}}
\flushright{\begin{Arabic}
\quranayah[30][58]
\end{Arabic}}
\flushleft{\begin{hindi}
हमने इस क़ुरआन में लोगों के लिए प्रत्येक मिसाल पेश कर दी है। यदि तुम कोई भी निशानी उनके पास ले आओ, जिन लोगों ने इनकार किया है, वे तो यही कहेंगे, "तुम तो बस झूठ घड़ते हो।"
\end{hindi}}
\flushright{\begin{Arabic}
\quranayah[30][59]
\end{Arabic}}
\flushleft{\begin{hindi}
इस प्रकार अल्लाह उन लोगों के दिलों पर ठप्पा लगा देता है जो अज्ञानी है
\end{hindi}}
\flushright{\begin{Arabic}
\quranayah[30][60]
\end{Arabic}}
\flushleft{\begin{hindi}
अतः धैर्य से काम लो निश्चय ही अल्लाह का वादा सच्चा है और जिन्हें विश्वास नहीं, वे तुम्हें कदापि हल्का न पाएँ
\end{hindi}}
\chapter{Luqman (Luqman)}
\begin{Arabic}
\Huge{\centerline{\basmalah}}\end{Arabic}
\flushright{\begin{Arabic}
\quranayah[31][1]
\end{Arabic}}
\flushleft{\begin{hindi}
अलिफ़॰ लाम॰ मीम॰
\end{hindi}}
\flushright{\begin{Arabic}
\quranayah[31][2]
\end{Arabic}}
\flushleft{\begin{hindi}
(जो आयतें उतर रही हैं) वे तत्वज्ञान से परिपूर्ण किताब की आयते हैं
\end{hindi}}
\flushright{\begin{Arabic}
\quranayah[31][3]
\end{Arabic}}
\flushleft{\begin{hindi}
मार्गदर्शन और दयालुता उत्तमकारों के लिए
\end{hindi}}
\flushright{\begin{Arabic}
\quranayah[31][4]
\end{Arabic}}
\flushleft{\begin{hindi}
जो नमाज़ का आयोजन करते है और ज़कात देते है और आख़िरत पर विश्वास रखते है
\end{hindi}}
\flushright{\begin{Arabic}
\quranayah[31][5]
\end{Arabic}}
\flushleft{\begin{hindi}
वही अपने रब की और से मार्ग पर हैं और वही सफल है
\end{hindi}}
\flushright{\begin{Arabic}
\quranayah[31][6]
\end{Arabic}}
\flushleft{\begin{hindi}
लोगों में से कोई ऐसा भी है जो दिल को लुभानेवाली बातों का ख़रीदार बनता है, ताकि बिना किसी ज्ञान के अल्लाह के मार्ग से (दूसरों को) भटकाए और उनका परिहास करे। वही है जिनके लिए अपमानजनक यातना है
\end{hindi}}
\flushright{\begin{Arabic}
\quranayah[31][7]
\end{Arabic}}
\flushleft{\begin{hindi}
जब उसे हमारी आयतें सुनाई जाती हैं तो वह स्वयं को बड़ा समझता हुआ पीठ फेरकर चल देता है, मानो उसने उन्हें सुना ही नहीं, मानो उसके काम बहरे है। अच्छा तो उसे एक दुखद यातना की शुभ सूचना दे दो
\end{hindi}}
\flushright{\begin{Arabic}
\quranayah[31][8]
\end{Arabic}}
\flushleft{\begin{hindi}
अलबत्ता जो लोग ईमान लाए और उन्होंने अच्छे कर्म किए उनके लिए नेमत भरी जन्नतें हैं,
\end{hindi}}
\flushright{\begin{Arabic}
\quranayah[31][9]
\end{Arabic}}
\flushleft{\begin{hindi}
जिनमें वे सदैव रहेंगे। यह अल्लाह का सच्चा वादा है और वह अत्यन्त प्रभुत्वशाली, तत्वदर्शी है
\end{hindi}}
\flushright{\begin{Arabic}
\quranayah[31][10]
\end{Arabic}}
\flushleft{\begin{hindi}
उसने आकाशों को पैदा किया, (जो थमें हुए हैं) बिना ऐसे स्तम्भों के जो तुम्हें दिखाई दें। और उसने धरती में पहाड़ डाल दिए कि ऐसा न हो कि तुम्हें लेकर डाँवाडोल हो जाए और उसने उसमें हर प्रकार के जानवर फैला दिए। और हमने ही आकाश से पानी उतारा, फिर उसमें हर प्रकार की उत्तम चीज़े उगाई
\end{hindi}}
\flushright{\begin{Arabic}
\quranayah[31][11]
\end{Arabic}}
\flushleft{\begin{hindi}
यह तो अल्लाह की संरचना है। अब तनिक मुझे दिखाओं कि उससे हटकर जो दूसरे हैं (तुम्हारे ठहराए हुए प्रुभ) उन्होंने क्या पैदा किया हैं! नहीं, बल्कि ज़ालिम तो एक खुली गुमराही में पड़े हुए है
\end{hindi}}
\flushright{\begin{Arabic}
\quranayah[31][12]
\end{Arabic}}
\flushleft{\begin{hindi}
निश्चय ही हमने लुकमान को तत्वदर्शिता प्रदान की थी कि अल्लाह के प्रति कृतज्ञता दिखलाओ और जो कोई कृतज्ञता दिखलाए, वह अपने ही भले के लिए कृतज्ञता दिखलाता है। और जिसने अकृतज्ञता दिखलाई तो अल्लाह वास्तव में निस्पृह, प्रशंसनीय है
\end{hindi}}
\flushright{\begin{Arabic}
\quranayah[31][13]
\end{Arabic}}
\flushleft{\begin{hindi}
याद करो जब लुकमान ने अपने बेटे से, उसे नसीहत करते हुए कहा, "ऐ मेरे बेटे! अल्लाह का साझी न ठहराना। निश्चय ही शिर्क (बहुदेववाद) बहुत बड़ा ज़ुल्म है।"
\end{hindi}}
\flushright{\begin{Arabic}
\quranayah[31][14]
\end{Arabic}}
\flushleft{\begin{hindi}
और हमने मनुष्य को उसके अपने माँ-बाप के मामले में ताकीद की है - उसकी माँ ने निढाल होकर उसे पेट में रखा और दो वर्ष उसके दूध छूटने में लगे - कि "मेरे प्रति कृतज्ञ हो और अपने माँ-बाप के प्रति भी। अंततः मेरी ही ओर आना है
\end{hindi}}
\flushright{\begin{Arabic}
\quranayah[31][15]
\end{Arabic}}
\flushleft{\begin{hindi}
किन्तु यदि वे तुझपर दबाव डाले कि तू किसी को मेरे साथ साझी ठहराए, जिसका तुझे ज्ञान नहीं, तो उसकी बात न मानना और दुनिया में उसके साथ भले तरीके से रहना। किन्तु अनुसरण उस व्यक्ति के मार्ग का करना जो मेरी ओर रुजू हो। फिर तुम सबको मेरी ही ओर पलटना है; फिर मैं तुम्हें बता दूँगा जो कुछ तुम करते रहे होगे।"-
\end{hindi}}
\flushright{\begin{Arabic}
\quranayah[31][16]
\end{Arabic}}
\flushleft{\begin{hindi}
"ऐ मेरे बेटे! इसमें सन्देह नहीं कि यदि वह राई के दाने के बराबर भी हो, फिर वह किसी चट्टान के बीच हो या आकाशों में हो या धरती में हो, अल्लाह उसे ला उपस्थित करेगा। निस्संदेह अल्लाह अत्यन्त सूक्ष्मदर्शी, ख़बर रखनेवाला है।
\end{hindi}}
\flushright{\begin{Arabic}
\quranayah[31][17]
\end{Arabic}}
\flushleft{\begin{hindi}
"ऐ मेरे बेटे! नमाज़ का आयोजन कर और भलाई का हुक्म दे और बुराई से रोक और जो मुसीबत भी तुझपर पड़े उसपर धैर्य से काम ले। निस्संदेह ये उन कामों में से है जो अनिवार्य और ढृढसंकल्प के काम है
\end{hindi}}
\flushright{\begin{Arabic}
\quranayah[31][18]
\end{Arabic}}
\flushleft{\begin{hindi}
"और लोगों से अपना रूख़ न फेर और न धरती में इतराकर चल। निश्चय ही अल्लाह किसी अहंकारी, डींग मारनेवाले को पसन्द नहीं करता
\end{hindi}}
\flushright{\begin{Arabic}
\quranayah[31][19]
\end{Arabic}}
\flushleft{\begin{hindi}
"और अपनी चाल में सहजता और संतुलन बनाए रख और अपनी आवाज़ धीमी रख। निस्संदेह आवाज़ों में सबसे बुरी आवाज़ गधों की आवाज़ होती है।"
\end{hindi}}
\flushright{\begin{Arabic}
\quranayah[31][20]
\end{Arabic}}
\flushleft{\begin{hindi}
क्या तुमने देखा नहीं कि अल्लाह ने, जो कुछ आकाशों में और जो कुछ धरती में है, सबको तुम्हारे काम में लगा रखा है और उसने तुमपर अपनी प्रकट और अप्रकट अनुकम्पाएँ पूर्ण कर दी है? इसपर भी कुछ लोग ऐसे है जो अल्लाह के विषय में बिना किसी ज्ञान, बिना किसी मार्गदर्शन और बिना किसी प्रकाशमान किताब के झगड़ते है
\end{hindi}}
\flushright{\begin{Arabic}
\quranayah[31][21]
\end{Arabic}}
\flushleft{\begin{hindi}
अब जब उनसे कहा जाता है कि "उस चीज़ का अनुसरण करो जो अल्लाह न उतारी है," तो कहते है, "नहीं, बल्कि हम तो उस चीज़ का अनुसरण करेंगे जिसपर हमने अपने बाप-दादा को पाया है।" क्या यद्यपि शैतान उनको भड़कती आग की यातना की ओर बुला रहा हो तो भी?
\end{hindi}}
\flushright{\begin{Arabic}
\quranayah[31][22]
\end{Arabic}}
\flushleft{\begin{hindi}
जो कोई आज्ञाकारिता के साथ अपना रुख़ अल्लाह की ओर करे, और वह उत्तमकर भी हो तो उसने मज़बूत सहारा थाम लिया। सारे मामलों की परिणति अल्लाह ही की ओर है
\end{hindi}}
\flushright{\begin{Arabic}
\quranayah[31][23]
\end{Arabic}}
\flushleft{\begin{hindi}
और जिस किसी ने इनकार किया तो उसका इनकार तुम्हें शोकाकुल न करे। हमारी ही ओर तो उन्हें पलटकर आना है। फिर जो कुछ वे करते रहे होंगे, उससे हम उन्हें अवगत करा देंगे। निस्संदेह अल्लाह सीनों की बात तक जानता है
\end{hindi}}
\flushright{\begin{Arabic}
\quranayah[31][24]
\end{Arabic}}
\flushleft{\begin{hindi}
हम उन्हें थोड़ा मज़ा उड़ाने देंगे। फिर उन्हें विवश करके एक कठोर यातना की ओर खींच ले जाएँगे
\end{hindi}}
\flushright{\begin{Arabic}
\quranayah[31][25]
\end{Arabic}}
\flushleft{\begin{hindi}
यदि तुम उनसे पूछो कि "आकाशों और धरती को किसने पैदा किया?" तो वे अवश्य कहेंगे कि "अल्लाह ने।" कहो, "प्रशंसा भी अल्लाह के लिए है।" वरन उनमें से अधिकांश जानते नहीं
\end{hindi}}
\flushright{\begin{Arabic}
\quranayah[31][26]
\end{Arabic}}
\flushleft{\begin{hindi}
आकाशों और धरती में जो कुछ है अल्लाह ही का है। निस्संदेह अल्लाह ही निस्पृह, स्वतः प्रशंसित है
\end{hindi}}
\flushright{\begin{Arabic}
\quranayah[31][27]
\end{Arabic}}
\flushleft{\begin{hindi}
धरती में जितने वृक्ष है, यदि वे क़लम हो जाएँ और समुद्र उसकी स्याही हो जाए, उसके बाद सात और समुद्र हों, तब भी अल्लाह के बोल समाप्त न हो सकेंगे। निस्संदेह अल्लाह अत्यन्त प्रभुत्वशाली, तत्वदर्शी है
\end{hindi}}
\flushright{\begin{Arabic}
\quranayah[31][28]
\end{Arabic}}
\flushleft{\begin{hindi}
तुम सबका पैदा करना और तुम सबका जीवित करके पुनः उठाना तो बस ऐसा है, जैसे एक जीव का। अल्लाह तो सब कुछ सुनता, देखता है
\end{hindi}}
\flushright{\begin{Arabic}
\quranayah[31][29]
\end{Arabic}}
\flushleft{\begin{hindi}
क्या तुमने देखा नहीं कि अल्लाह रात को दिन में प्रविष्ट करता है और दिन को रात में प्रविष्ट करता है। उसने सूर्य और चन्द्रमा को काम में लगा रखा है? प्रत्येक एक नियत समय तक चला जा रहा है और इसके साथ यह कि जो कुछ भी तुम करते हो, अल्लाह उसकी पूरी ख़बर रखता है
\end{hindi}}
\flushright{\begin{Arabic}
\quranayah[31][30]
\end{Arabic}}
\flushleft{\begin{hindi}
यह सब कुछ इस कारण से है कि अल्लाह ही सत्य है और यह कि उसे छोड़कर जिनको वे पुकारते है, वे असत्य है। और यह कि अल्लाह ही सर्वोच्च, महान है
\end{hindi}}
\flushright{\begin{Arabic}
\quranayah[31][31]
\end{Arabic}}
\flushleft{\begin{hindi}
क्या तुमने देखा नहीं कि नौका समुद्र में अल्लाह के अनुग्रह से चलती है, ताकि वह तुम्हें अपनी कुछ निशानियाँ दिखाए। निस्संदेह इसमें प्रत्येक धैर्यवान, कृतज्ञ के लिए निशानियाँ है
\end{hindi}}
\flushright{\begin{Arabic}
\quranayah[31][32]
\end{Arabic}}
\flushleft{\begin{hindi}
और जब कोई मौज छाया-छत्रों की तरह उन्हें ढाँक लेती है, तो वे अल्लाह को उसी के लिए अपने निष्ठाभाव के विशुद्ध करते हुए पुकारते है, फिर जब वह उन्हें बचाकर स्थल तक पहुँचा देता है, तो उनमें से कुछ लोग संतुलित मार्ग पर रहते है। (अधिकांश तो पुनः पथभ्रष्ट हो जाते है।) हमारी निशानियों का इनकार तो बस प्रत्येक वह व्यक्ति करता है जो विश्वासघाती, कृतध्न हो
\end{hindi}}
\flushright{\begin{Arabic}
\quranayah[31][33]
\end{Arabic}}
\flushleft{\begin{hindi}
ऐ लोगों! अपने रब का डर रखो और उस दिन से डरो जब न कोई बाप अपनी औलाद की ओर से बदला देगा और न कोई औलाद ही अपने बाप की ओर से बदला देनेवाली होगी। निश्चय ही अल्लाह का वादा सच्चा है। अतः सांसारिक जीवन कदापि तुम्हें धोखे में न डाले। और न अल्लाह के विषय में वह धोखेबाज़ तुम्हें धोखें में डाले
\end{hindi}}
\flushright{\begin{Arabic}
\quranayah[31][34]
\end{Arabic}}
\flushleft{\begin{hindi}
निस्संदेह उस घड़ी का ज्ञान अल्लाह ही के पास है। वही मेंह बरसाता है और जानता है जो कुछ गर्भाशयों में होता है। कोई क्यक्ति नहीं जानता कि कल वह क्या कमाएगा और कोई व्यक्ति नहीं जानता है कि किस भूभाग में उसक मृत्यु होगी। निस्संदेह अल्लाह जाननेवाला, ख़बर रखनेवाला है
\end{hindi}}
\chapter{As-Sajdah (The Adoration)}
\begin{Arabic}
\Huge{\centerline{\basmalah}}\end{Arabic}
\flushright{\begin{Arabic}
\quranayah[32][1]
\end{Arabic}}
\flushleft{\begin{hindi}
अलिफ़॰ लाम॰ मीम॰
\end{hindi}}
\flushright{\begin{Arabic}
\quranayah[32][2]
\end{Arabic}}
\flushleft{\begin{hindi}
इस किताब का अवतरण - इसमें सन्देह नहीं - सारे संसार के रब की ओर से है
\end{hindi}}
\flushright{\begin{Arabic}
\quranayah[32][3]
\end{Arabic}}
\flushleft{\begin{hindi}
(क्या वे इसपर विश्वास नहीं रखते) या वे कहते है कि "इस व्यक्ति ने इसे स्वयं ही घड़ लिया है?" नहीं, बल्कि वह सत्य है तेरे रब की ओर से, ताकि तू उन लोगों को सावधान कर दे जिनके पास तुझसे पहले कोई सावधान करनेवाला नहीं आया। कदाचित वे मार्ग पाएँ
\end{hindi}}
\flushright{\begin{Arabic}
\quranayah[32][4]
\end{Arabic}}
\flushleft{\begin{hindi}
अल्लाह ही है जिसने आकाशों और धरती को और जो कुछ दोनों के बीच है छह दिनों में पैदा किया। फिर सिंहासन पर विराजमान हुआ। उससे हटकर न तो तुम्हारा कोई संरक्षक मित्र है और न उसके मुक़ाबले में कोई सिफ़ारिस करनेवाला। फिर क्या तुम होश में न आओगे?
\end{hindi}}
\flushright{\begin{Arabic}
\quranayah[32][5]
\end{Arabic}}
\flushleft{\begin{hindi}
वह कार्य की व्यवस्था करता है आकाश से धरती तक - फिर सारे मामले उसी की तरफ़ लौटते है - एक दिन में, जिसकी माप तुम्हारी गणना के अनुसार एक हज़ार वर्ष है
\end{hindi}}
\flushright{\begin{Arabic}
\quranayah[32][6]
\end{Arabic}}
\flushleft{\begin{hindi}
वही है परोक्ष और प्रत्यक्ष का जाननेवाला अत्यन्त प्रभुत्वशाली, दयावान है
\end{hindi}}
\flushright{\begin{Arabic}
\quranayah[32][7]
\end{Arabic}}
\flushleft{\begin{hindi}
जिसने हरेक चीज़, जो बनाई ख़ूब ही बनाई और उसने मनुष्य की संरचना का आरम्भ गारे से किया
\end{hindi}}
\flushright{\begin{Arabic}
\quranayah[32][8]
\end{Arabic}}
\flushleft{\begin{hindi}
फिर उसकी सन्तति एक तुच्छ पानी के सत से चलाई
\end{hindi}}
\flushright{\begin{Arabic}
\quranayah[32][9]
\end{Arabic}}
\flushleft{\begin{hindi}
फिर उसे ठीक-ठीक किया और उसमें अपनी रूह (आत्मा) फूँकी। और तुम्हें कान और आँखें और दिल दिए। तुम आभारी थोड़े ही होते हो
\end{hindi}}
\flushright{\begin{Arabic}
\quranayah[32][10]
\end{Arabic}}
\flushleft{\begin{hindi}
और उन्होंने कहा, "जब हम धरती में रल-मिल जाएँगे तो फिर क्या हम वास्तब में नवीन काय में जीवित होंगे?" नहीं, बल्कि उन्हें अपने रब से मिलने का इनकार है
\end{hindi}}
\flushright{\begin{Arabic}
\quranayah[32][11]
\end{Arabic}}
\flushleft{\begin{hindi}
कहो, "मृत्यु का फ़रिश्ता जो तुमपर नियुक्त है, वह तुम्हें पूर्ण रूप से अपने क़ब्जे में ले लेता है। फिर तुम अपने रब की ओर वापस होंगे।"
\end{hindi}}
\flushright{\begin{Arabic}
\quranayah[32][12]
\end{Arabic}}
\flushleft{\begin{hindi}
और यदि कहीं तुम देखते जब वे अपराधी अपने रब के सामने अपने सिर झुकाए होंगे कि "हमारे रब! हमने देख लिया और सुन लिया। अब हमें वापस भेज दे, ताकि हम अच्छे कर्म करें। निस्संदेह अब हमें विश्वास हो गया।"
\end{hindi}}
\flushright{\begin{Arabic}
\quranayah[32][13]
\end{Arabic}}
\flushleft{\begin{hindi}
यदि हम चाहते तो प्रत्येक व्यक्ति को उसका अपना संमार्ग दिखा देते, तिन्तु मेरी ओर से बात सत्यापित हो चुकी है कि "मैं जहन्नम को जिन्नों और मनुष्यों, सबसे भरकर रहूँगा।"
\end{hindi}}
\flushright{\begin{Arabic}
\quranayah[32][14]
\end{Arabic}}
\flushleft{\begin{hindi}
अतः अब चखो मज़ा, इसका कि तुमने अपने इस दिन के मिलन को भुलाए रखा। तो हमने भी तुम्हें भुला दिया। शाश्वत यातना का रसास्वादन करो, उसके बदले में जो तुम करते रहे हो
\end{hindi}}
\flushright{\begin{Arabic}
\quranayah[32][15]
\end{Arabic}}
\flushleft{\begin{hindi}
हमारी आयतों पर जो बस वही लोग ईमान लाते है, जिन्हें उनके द्वारा जब याद दिलाया जाता है तो सजदे में गिर पड़ते है और अपने रब का गुणगान करते है और घमंड नहीं करते
\end{hindi}}
\flushright{\begin{Arabic}
\quranayah[32][16]
\end{Arabic}}
\flushleft{\begin{hindi}
उनके पहलू बिस्तरों से अलग रहते है कि वे अपने रब को भय और लालसा के साथ पुकारते है, और जो कुछ हमने उन्हें दिया है उसमें से ख़र्च करते है
\end{hindi}}
\flushright{\begin{Arabic}
\quranayah[32][17]
\end{Arabic}}
\flushleft{\begin{hindi}
फिर कोई प्राणी नहीं जानता आँखों की जो ठंडक उसके लिए छिपा रखी गई है उसके बदले में देने के ध्येय से जो वे करते रहे होंगे
\end{hindi}}
\flushright{\begin{Arabic}
\quranayah[32][18]
\end{Arabic}}
\flushleft{\begin{hindi}
भला जो व्यक्ति ईमानवाला हो वह उस व्यक्ति जैसा हो सकता है जो अवज्ञाकारी हो? वे बराबर नहीं हो सकते
\end{hindi}}
\flushright{\begin{Arabic}
\quranayah[32][19]
\end{Arabic}}
\flushleft{\begin{hindi}
रहे वे लोग जा ईमान लाए और उन्हें अच्छे कर्म किए, उनके लिए जो कर्म वे करते रहे उसके बदले में आतिथ्य स्वरूप रहने के बाग़ है
\end{hindi}}
\flushright{\begin{Arabic}
\quranayah[32][20]
\end{Arabic}}
\flushleft{\begin{hindi}
रहे वे लोग जिन्होंने सीमा का उल्लंघन किया, उनका ठिकाना आग है। जब कभी भी वे चाहेंगे कि उससे निकल जाएँ तो उसी में लौटा दिए जाएँगे और उनसे कहा जाएगा, "चखो उस आग की यातना का मज़ा, जिसे तुम झूठ समझते थे।"
\end{hindi}}
\flushright{\begin{Arabic}
\quranayah[32][21]
\end{Arabic}}
\flushleft{\begin{hindi}
हम बड़ी यातना से इतर उन्हें छोटी यातना का मज़ा चखाएँगे, कदाचित वे पलट आएँ
\end{hindi}}
\flushright{\begin{Arabic}
\quranayah[32][22]
\end{Arabic}}
\flushleft{\begin{hindi}
और उस व्यक्ति से बढकर अत्याचारी कौन होगा जिसे उसके रब की आयतों के द्वारा याद दिलाया जाए,फिर वह उनसे मुँह फेर ले? निश्चय ही हम अपराधियों से बदला लेकर रहेंगे
\end{hindi}}
\flushright{\begin{Arabic}
\quranayah[32][23]
\end{Arabic}}
\flushleft{\begin{hindi}
हमने मूसा को किताब प्रदान की थी - अतः उसके मिलने के प्रति तुम किसी सन्देह में न रहना और हमने इसराईल की सन्तान के लिए उस (किताब) को मार्गदर्शन बनाया था
\end{hindi}}
\flushright{\begin{Arabic}
\quranayah[32][24]
\end{Arabic}}
\flushleft{\begin{hindi}
और जब वे जमे रहे और उन्हें हमारी आयतों पर विश्वास था, तो हमने उनमें ऐसे नायक बनाए जो हमारे आदेश से मार्ग दिखाते थे
\end{hindi}}
\flushright{\begin{Arabic}
\quranayah[32][25]
\end{Arabic}}
\flushleft{\begin{hindi}
निश्चय ही तेरा रब ही क़ियामत के दिन उनके बीच उन बातों का फ़ैसला करेगा, जिनमें वे मतभेद करते रहे है
\end{hindi}}
\flushright{\begin{Arabic}
\quranayah[32][26]
\end{Arabic}}
\flushleft{\begin{hindi}
क्या उनके लिए यह चीज़ भी मार्गदर्शक सिद्ध नहीं हुई कि उनसे पहले कितनी ही नस्लों को हम विनष्ट कर चुके है, जिनके रहने-बसने की जगहों में वे चलते-फिरते है? निस्संदेह इसमें बहुत-सी निशानियाँ है। फिर क्या वे सुनने नहीं?
\end{hindi}}
\flushright{\begin{Arabic}
\quranayah[32][27]
\end{Arabic}}
\flushleft{\begin{hindi}
क्या उन्होंने देखा नहीं कि हम सूखी पड़ी भूमि की ओर पानी ले जाते है। फिर उससे खेती उगाते है, जिसमें से उनके चौपाए भी खाते है और वे स्वयं भी? तो क्या उन्हें सूझता नहीं?
\end{hindi}}
\flushright{\begin{Arabic}
\quranayah[32][28]
\end{Arabic}}
\flushleft{\begin{hindi}
वे कहते है कि "यह फ़ैसला कब होगा, यदि तुम सच्चे हो?"
\end{hindi}}
\flushright{\begin{Arabic}
\quranayah[32][29]
\end{Arabic}}
\flushleft{\begin{hindi}
कह दो कि "फ़ैसले के दिन इनकार करनेवालों का ईमान उनके लिए कुछ लाभदायक न होगा और न उन्हें ठील ही दी जाएगी।"
\end{hindi}}
\flushright{\begin{Arabic}
\quranayah[32][30]
\end{Arabic}}
\flushleft{\begin{hindi}
अच्छा, उन्हें उनके हाल पर छोड़ दो और प्रतीक्षा करो। वे भी परीक्षारत है
\end{hindi}}
\chapter{Al-Ahzab (The Allies)}
\begin{Arabic}
\Huge{\centerline{\basmalah}}\end{Arabic}
\flushright{\begin{Arabic}
\quranayah[33][1]
\end{Arabic}}
\flushleft{\begin{hindi}
ऐ नबी! अल्लाह का डर रखना और इनकार करनेवालों और कपटाचारियों का कहना न मानना। वास्तब में अल्लाह सर्वज्ञ, तत्वदर्शी है
\end{hindi}}
\flushright{\begin{Arabic}
\quranayah[33][2]
\end{Arabic}}
\flushleft{\begin{hindi}
और अनुकरण करना उस चीज़ का जो तुम्हारे रब की ओर से तुम्हें प्रकाशना की जा रही है। निश्चय ही अल्लाह उसकी ख़बर रखता है जो तुम करते हो
\end{hindi}}
\flushright{\begin{Arabic}
\quranayah[33][3]
\end{Arabic}}
\flushleft{\begin{hindi}
और अल्लाह पर भरोसा रखो। और अल्लाह भरोसे के लिए काफी है
\end{hindi}}
\flushright{\begin{Arabic}
\quranayah[33][4]
\end{Arabic}}
\flushleft{\begin{hindi}
अल्लाह ने किसी व्यक्ति के सीने में दो दिल नहीं रखे। और न उसने तुम्हारी उन पत्ऩियों को जिनसे तुम ज़िहार कर बैठते हो, वास्तव में तुम्हारी माँ बनाया, और न उसने तुम्हारे मुँह बोले बेटों को तुम्हारे वास्तविक बेटे बनाए। ये तो तुम्हारे मुँह की बातें है। किन्तु अल्लाह सच्ची बात कहता है और वही मार्ग दिखाता है
\end{hindi}}
\flushright{\begin{Arabic}
\quranayah[33][5]
\end{Arabic}}
\flushleft{\begin{hindi}
उन्हें उनके बापों का बेटा करकर पुकारो। अल्लाह के यहाँ यही अधिक न्यायसंगत बात है। और यदि तुम उनके बापों को न जानते हो, तो धर्म में वे तुम्हारे भाई तो है ही और तुम्हारे सहचर भी। इस सिलसिले में तुमसे जो ग़लती हुई हो उसमें तुमपर कोई गुनाह नहीं, किन्तु जिसका संकल्प तुम्हारे दिलों ने कर लिया, उसकी बात और है। वास्तव में अल्लाह अत्यन्त क्षमाशील, दयावान है
\end{hindi}}
\flushright{\begin{Arabic}
\quranayah[33][6]
\end{Arabic}}
\flushleft{\begin{hindi}
नबी का हक़ ईमानवालों पर स्वयं उनके अपने प्राणों से बढ़कर है। और उसकी पत्नियों उनकी माएँ है। और अल्लाह के विधान के अनुसार सामान्य मोमिनों और मुहाजिरों की अपेक्षा नातेदार आपस में एक-दूसरे से अधिक निकट है। यह और बात है कि तुम अपने साथियों के साथ कोई भलाई करो। यह बात किताब में लिखी हुई है
\end{hindi}}
\flushright{\begin{Arabic}
\quranayah[33][7]
\end{Arabic}}
\flushleft{\begin{hindi}
और याद करो जब हमने नबियों से वचन लिया, तुमसे भी और नूह और इबराहीम और मूसा और मरयम के बेटे ईसा से भी। इन सबसे हमने ढृढ़ वचन लिया,
\end{hindi}}
\flushright{\begin{Arabic}
\quranayah[33][8]
\end{Arabic}}
\flushleft{\begin{hindi}
ताकि वह सच्चे लोगों से उनकी सच्चाई के बारे में पूछे। और इनकार करनेवालों के लिए तो उसने दुखद यातना तैयार कर रखी है
\end{hindi}}
\flushright{\begin{Arabic}
\quranayah[33][9]
\end{Arabic}}
\flushleft{\begin{hindi}
ऐ ईमान लानेवालो! अल्लाह की उस अनुकम्पा को याद करो जो तुमपर हुई; जबकि सेनाएँ तुमपर चढ़ आई तो हमने उनपर एक हवा भेज दी और ऐसी सेनाएँ भी, जिनको तुमने देखा नहीं। और अल्लाह वह सब कुछ देखता है जो तुम करते हो
\end{hindi}}
\flushright{\begin{Arabic}
\quranayah[33][10]
\end{Arabic}}
\flushleft{\begin{hindi}
याद करो जब वे तुम्हारे ऊपर की ओर से और तुम्हारे नीचे की ओर से भी तुमपर चढ़ आए, और जब निगाहें टेढ़ी-तिरछी हो गई और उर (हृदय) कंठ को आ लगे। और तुम अल्लाह के बारे में तरह-तरह के गुमान करने लगे थे
\end{hindi}}
\flushright{\begin{Arabic}
\quranayah[33][11]
\end{Arabic}}
\flushleft{\begin{hindi}
उस समय ईमानवाले आज़माए गए और पूरी तरह हिला दिए गए
\end{hindi}}
\flushright{\begin{Arabic}
\quranayah[33][12]
\end{Arabic}}
\flushleft{\begin{hindi}
और जब कपटाचारी और वे लोग जिनके दिलों में रोग है कहने लगे, "अल्लाह और उसके रसूल ने हमसे जो वादा किया था वह तो धोखा मात्र था।"
\end{hindi}}
\flushright{\begin{Arabic}
\quranayah[33][13]
\end{Arabic}}
\flushleft{\begin{hindi}
और जबकि उनमें से एक गिरोह ने कहा, "ऐ यसरिबवालो, तुम्हारे लिए ठहरने का कोई मौक़ा नहीं। अतः लौट चलो।" और उनका एक गिरोह नबी से यह कहकर (वापस जाने की) अनुमति चाह रहा था कि "हमारे घर असुरक्षित है।" यद्यपि वे असुरक्षित न थे। वे तो बस भागना चाहते थे
\end{hindi}}
\flushright{\begin{Arabic}
\quranayah[33][14]
\end{Arabic}}
\flushleft{\begin{hindi}
और यदि उसके चतुर्दिक से उनपर हमला हो जाता, फिर उस समय उनसे उपद्रव के लिए कहा जाता, तो वे ऐसा कर डालते और इसमें विलम्ब थोड़े ही करते!
\end{hindi}}
\flushright{\begin{Arabic}
\quranayah[33][15]
\end{Arabic}}
\flushleft{\begin{hindi}
यद्यपि वे इससे पहले अल्लाह को वचन दे चुके थे कि वे पीठ न फेरेंगे, और अल्लाह से की गई प्रतिज्ञा के विषय में तो पूछा जाना ही है
\end{hindi}}
\flushright{\begin{Arabic}
\quranayah[33][16]
\end{Arabic}}
\flushleft{\begin{hindi}
कह दो, "यदि तुम मृत्यु और मारे जाने से भागो भी तो यह भागना तुम्हारे लिए कदापि लाभप्रद न होगा। और इस हालत में भी तुम सुख थोड़े ही प्राप्त कर सकोगे।"
\end{hindi}}
\flushright{\begin{Arabic}
\quranayah[33][17]
\end{Arabic}}
\flushleft{\begin{hindi}
कहो, "कहो है जो तुम्हें अल्लाह से बचा सकता है, यदि वह तुम्हारी कोई बुराई चाहे या वह तुम्हारे प्रति दयालुता का इरादा करे (तो कौन है जो उसकी दयालुता को रोक सके)?" वे अल्लाह के अल्लाह के अलावा न अपना कोई निकटवर्ती समर्थक पाएँगे और न (दूर का) सहायक
\end{hindi}}
\flushright{\begin{Arabic}
\quranayah[33][18]
\end{Arabic}}
\flushleft{\begin{hindi}
अल्लाह तुममें से उन लोगों को भली-भाँति जानता है जो (युद्ध से) रोकते है और अपने भाइयों से कहते है, "हमारे पास आ जाओ।" और वे लड़ाई में थोड़े ही आते है, (क्योंकि वे)
\end{hindi}}
\flushright{\begin{Arabic}
\quranayah[33][19]
\end{Arabic}}
\flushleft{\begin{hindi}
तुम्हारे साथ कृपणता से काम लेते है। अतः जब भय का समय आ जाता है, तो तुम उन्हें देखते हो कि वे तुम्हारी ओर इस प्रकार ताक रहे कि उनकी आँखें चक्कर खा रही है, जैसे किसी व्यक्ति पर मौत की बेहोशी छा रही हो। किन्तु जब भय जाता रहता है तो वे माल के लोभ में तेज़ ज़बाने तुमपर चलाते है। ऐसे लोग ईमान लाए ही नहीं। अतः अल्लाह ने उनके कर्म उनकी जान को लागू कर दिए। और यह अल्लाह के लिए बहुत सरल है
\end{hindi}}
\flushright{\begin{Arabic}
\quranayah[33][20]
\end{Arabic}}
\flushleft{\begin{hindi}
वे समझ रहे है कि (शत्रु के) सैन्य दल अभी गए नहीं हैं, और यदि वे गिरोह फिर आ जाएँ तो वे चाहेंगे कि किसी प्रकार बाहर (मरुस्थल में) बद्दु ओं के साथ हो रहें और वहीं से तुम्हारे बारे में समाचार पूछते रहे। और यदि वे तुम्हारे साथ होते भी तो लड़ाई में हिस्सा थोड़े ही लेते
\end{hindi}}
\flushright{\begin{Arabic}
\quranayah[33][21]
\end{Arabic}}
\flushleft{\begin{hindi}
निस्संदेह तुम्हारे लिए अल्लाह के रसूल में एक उत्तम आदर्श है अर्थात उस व्यक्ति के लिए जो अल्लाह और अन्तिम दिन की आशा रखता हो और अल्लाह को अधिक याद करे
\end{hindi}}
\flushright{\begin{Arabic}
\quranayah[33][22]
\end{Arabic}}
\flushleft{\begin{hindi}
और जब ईमानवालों ने सैन्य दलों को देखा तो वे पुकार उठे, "यह तो वही चीज़ है, जिसका अल्लाह और उसके रसूल ने हमसे वादा किया था। और अल्लाह और उसके रसूल ने सच कहा था।" इस चीज़ ने उनके ईमान और आज्ञाकारिता ही को बढ़ाया
\end{hindi}}
\flushright{\begin{Arabic}
\quranayah[33][23]
\end{Arabic}}
\flushleft{\begin{hindi}
ईमानवालों के रूप में ऐसे पुरुष मौजूद है कि जो प्रतिज्ञा उन्होंने अल्लाह से की थी उसे उन्होंने सच्चा कर दिखाया। फिर उनमें से कुछ तो अपना प्रण पूरा कर चुके और उनमें से कुछ प्रतीक्षा में है। और उन्होंने अपनी बात तनिक भी नहीं बदली
\end{hindi}}
\flushright{\begin{Arabic}
\quranayah[33][24]
\end{Arabic}}
\flushleft{\begin{hindi}
ताकि इसके परिणामस्वरूप अल्लाह सच्चों को उनकी सच्चाई का बदला दे और कपटाचारियों को चाहे तो यातना दे या उनकी तौबा क़बूल करे। निश्चय ही अल्लाह बड़ी क्षमाशील, दयावान है
\end{hindi}}
\flushright{\begin{Arabic}
\quranayah[33][25]
\end{Arabic}}
\flushleft{\begin{hindi}
अल्लाह ने इनकार करनेवालों को उनके अपने क्रोध के साथ फेर दिया। वे कोई भलाई प्राप्त न कर सके। अल्लाह ने मोमिनों को युद्ध करने से बचा लिया। अल्लाह तो है ही बड़ा शक्तिवान, प्रभुत्वशाली
\end{hindi}}
\flushright{\begin{Arabic}
\quranayah[33][26]
\end{Arabic}}
\flushleft{\begin{hindi}
और किताबवालों में सो जिन लोगों ने उसकी सहायता की थी, उन्हें उनकी गढ़ियों से उतार लाया। और उनके दिलों में धाक बिठा दी कि तुम एक गिरोह को जान से मारने लगे और एक गिरोह को बन्दी बनाने लगे
\end{hindi}}
\flushright{\begin{Arabic}
\quranayah[33][27]
\end{Arabic}}
\flushleft{\begin{hindi}
और उसने तुम्हें उनके भू-भाग और उनके घरों और उनके मालों का वारिस बना दिया और उस भू-भाग का भी जिसे तुमने पददलित नहीं किया। वास्तव में अल्लाह को हर चीज़ की सामर्थ्य प्राप्त है
\end{hindi}}
\flushright{\begin{Arabic}
\quranayah[33][28]
\end{Arabic}}
\flushleft{\begin{hindi}
ऐ नबी! अपनी पत्नि यों से कह दो कि "यदि तुम सांसारिक जीवन और उसकी शोभा चाहती हो तो आओ, मैं तुम्हें कुछ दे-दिलाकर भली रीति से विदा कर दूँ
\end{hindi}}
\flushright{\begin{Arabic}
\quranayah[33][29]
\end{Arabic}}
\flushleft{\begin{hindi}
"किन्तु यदि तुम अल्लाह और उसके रसूल और आख़िरत के घर को चाहती हो तो निश्चय ही अल्लाह ने तुममे से उत्तमकार स्त्रियों के लिए बड़ा प्रतिदान रख छोड़ा है।"
\end{hindi}}
\flushright{\begin{Arabic}
\quranayah[33][30]
\end{Arabic}}
\flushleft{\begin{hindi}
ऐ नबी की स्त्रियों! तुममें से जो कोई प्रत्यक्ष अनुचित कर्म करे तो उसके लिए दोहरी यातना होगी। और यह अल्लाह के लिए बहुत सरल है
\end{hindi}}
\flushright{\begin{Arabic}
\quranayah[33][31]
\end{Arabic}}
\flushleft{\begin{hindi}
किन्तु तुममें से जो अल्लाह और उसके रसूल के प्रति निष्ठापूर्वक आज्ञाकारिता की नीति अपनाए और अच्छा कर्म करे, उसे हम दोहरा प्रतिदान प्रदान करेंगे और उसके लिए हमने सम्मानपूर्ण आजीविका तैयार कर रखी है
\end{hindi}}
\flushright{\begin{Arabic}
\quranayah[33][32]
\end{Arabic}}
\flushleft{\begin{hindi}
ऐ नबी की स्त्रियों! तुम सामान्य स्त्रियों में से किसी की तरह नहीं हो, यदि तुम अल्लाह का डर रखो। अतः तुम्हारी बातों में लोच न हो कि वह व्यक्ति जिसके दिल में रोग है, वह लालच में पड़ जाए। तुम सामान्य रूप से बात करो
\end{hindi}}
\flushright{\begin{Arabic}
\quranayah[33][33]
\end{Arabic}}
\flushleft{\begin{hindi}
अपने घरों में टिककर रहो और विगत अज्ञानकाल की-सी सज-धज न दिखाती फिरना। नमाज़ का आयोजन करो और ज़कात दो। और अल्लाह और उसके रसूल की आज्ञा का पालन करो। अल्लाह तो बस यही चाहता है कि ऐ नबी के घरवालो, तुमसे गन्दगी को दूर रखे और तुम्हें तरह पाक-साफ़ रखे
\end{hindi}}
\flushright{\begin{Arabic}
\quranayah[33][34]
\end{Arabic}}
\flushleft{\begin{hindi}
तुम्हारे घरों में अल्लाह की जो आयतें और तत्वदर्शिता की बातें सुनाई जाती है उनकी चर्चा करती रहो। निश्चय ही अल्लाह अत्यन्त सूक्ष्मदर्शी, खबर रखनेवाला है
\end{hindi}}
\flushright{\begin{Arabic}
\quranayah[33][35]
\end{Arabic}}
\flushleft{\begin{hindi}
मुस्लिम पुरुष और मुस्लिम स्त्रियाँ, ईमानवाले पुरुष और ईमानवाली स्त्रियाँ, निष्ठा्पूर्वक आज्ञापालन करनेवाले पुरुष और निष्ठापूर्वक आज्ञापालन करनेवाली स्त्रियाँ, सत्यवादी पुरुष और सत्यवादी स्त्रियाँ, धैर्यवान पुरुष और धैर्य रखनेवाली स्त्रियाँ, विनम्रता दिखानेवाले पुरुष और विनम्रता दिखानेवाली स्त्रियाँ, सदक़ा (दान) देनेवाले पुरुष और सदक़ा देनेवाली स्त्रियाँ, रोज़ा रखनेवाले पुरुष और रोज़ा रखनेवाली स्त्रियाँ, अपने गुप्तांगों की रक्षा करनेवाले पुरुष और रक्षा करनेवाली स्त्रियाँ और अल्लाह को अधिक याद करनेवाले पुरुष और याद करनेवाली स्त्रियाँ - इनके लिए अल्लाह ने क्षमा और बड़ा प्रतिदान तैयार कर रखा है
\end{hindi}}
\flushright{\begin{Arabic}
\quranayah[33][36]
\end{Arabic}}
\flushleft{\begin{hindi}
न किसी ईमानवाले पुरुष और न किसी ईमानवाली स्त्री को यह अधिकार है कि जब अल्लाह और उसका रसूल किसी मामले का फ़ैसला कर दें, तो फिर उन्हें अपने मामले में कोई अधिकार शेष रहे। जो कोई अल्लाह और उसके रसूल की अवज्ञा करे तो वह खुली गुमराही में पड़ गया
\end{hindi}}
\flushright{\begin{Arabic}
\quranayah[33][37]
\end{Arabic}}
\flushleft{\begin{hindi}
याद करो (ऐ नबी), जबकि तुम उस व्यक्ति से कह रहे थे जिसपर अल्लाह ने अनुकम्पा की, और तुमने भी जिसपर अनुकम्पा की कि "अपनी पत्नी को अपने पास रोक रखो और अल्लाह का डर रखो, और तुम अपने जी में उस बात को छिपा रहे हो जिसको अल्लाह प्रकट करनेवाला है। तुम लोगों से डरते हो, जबकि अल्लाह इसका ज़्यादा हक़ रखता है कि तुम उससे डरो।" अतः जब ज़ैद उससे अपनी ज़रूरत पूरी कर चुका तो हमने उसका तुमसे विवाह कर दिया, ताकि ईमानवालों पर अपने मुँह बोले बेटों की पत्नियों के मामले में कोई तंगी न रहे जबकि वे उनसे अपनी ज़रूरत पूरी कर लें। अल्लाह का फ़ैसला तो पूरा होकर ही रहता है
\end{hindi}}
\flushright{\begin{Arabic}
\quranayah[33][38]
\end{Arabic}}
\flushleft{\begin{hindi}
नबी पर उस काम में कोई तंगी नहीं जो अल्लाह ने उसके लिए ठहराया हो। यही अल्लाह का दस्तूर उन लोगों के मामले में भी रहा है जो पहले गुज़र चुके है - और अल्लाह का काम तो जँचा-तुला होता है। -
\end{hindi}}
\flushright{\begin{Arabic}
\quranayah[33][39]
\end{Arabic}}
\flushleft{\begin{hindi}
जो अल्लाह के सन्देश पहुँचाते थे और उसी से डरते थे और अल्लाह के सिवा किसी से नहीं डरते थे। और हिसाब लेने के लिए अल्लाह काफ़ी है। -
\end{hindi}}
\flushright{\begin{Arabic}
\quranayah[33][40]
\end{Arabic}}
\flushleft{\begin{hindi}
मुहम्मद तुम्हारे पुरुषों में से किसी के बाप नहीं है, बल्कि वे अल्लाह के रसूल और नबियों के समापक है। अल्लाह को हर चीज़ का पूरा ज्ञान है
\end{hindi}}
\flushright{\begin{Arabic}
\quranayah[33][41]
\end{Arabic}}
\flushleft{\begin{hindi}
ऐ ईमान लानेवालो! अल्लाह को अधिक याद करो
\end{hindi}}
\flushright{\begin{Arabic}
\quranayah[33][42]
\end{Arabic}}
\flushleft{\begin{hindi}
और प्रातःकाल और सन्ध्या समय उसकी तसबीह करते रहो -
\end{hindi}}
\flushright{\begin{Arabic}
\quranayah[33][43]
\end{Arabic}}
\flushleft{\begin{hindi}
वही है जो तुमपर रहमत भेजता है और उसके फ़रिश्ते भी (दुआएँ करते है) - ताकि वह तुम्हें अँधरों से प्रकाश की ओर निकाल लाए। वास्तव में, वह ईमानवालों पर बहुत दयालु है
\end{hindi}}
\flushright{\begin{Arabic}
\quranayah[33][44]
\end{Arabic}}
\flushleft{\begin{hindi}
जिस दिन वे उससे मिलेंगे उनका अभिवादन होगा, सलाम और उनके लिए प्रतिष्ठामय प्रदान तैयार कर रखा है
\end{hindi}}
\flushright{\begin{Arabic}
\quranayah[33][45]
\end{Arabic}}
\flushleft{\begin{hindi}
ऐ नबी! हमने तुमको साक्षी और शुभ सूचना देनेवाला और सचेल करनेवाला बनाकर भेजा है;
\end{hindi}}
\flushright{\begin{Arabic}
\quranayah[33][46]
\end{Arabic}}
\flushleft{\begin{hindi}
और अल्लाह की अनुज्ञा से उसकी ओर बुलानेवाला और प्रकाशमान प्रदीप बनाकर
\end{hindi}}
\flushright{\begin{Arabic}
\quranayah[33][47]
\end{Arabic}}
\flushleft{\begin{hindi}
ईमानवालों को शुभ सूचना दे दो कि उनके लिए अल्लाह को ओर से बहुत बड़ा उदार अनुग्रह है
\end{hindi}}
\flushright{\begin{Arabic}
\quranayah[33][48]
\end{Arabic}}
\flushleft{\begin{hindi}
और इनकार करनेवालों और कपटाचारियों के कहने में न आना। उनकी पहुँचाई हुई तकलीफ़ का ख़याल न करो। और अल्लाह पर भरोसा रखो। अल्लाह इसके लिए काफ़ी है कि अपने मामले में उसपर भरोसा किया जाए
\end{hindi}}
\flushright{\begin{Arabic}
\quranayah[33][49]
\end{Arabic}}
\flushleft{\begin{hindi}
ऐ ईमान लानेवालो! जब तुम ईमान लानेवाली स्त्रियों से विवाह करो और फिर उन्हें हाथ लगाने से पहले तलाक़ दे दो तो तुम्हारे लिए उनपर कोई इद्दत नहीं, जिसकी तुम गिनती करो। अतः उन्हें कुछ सामान दे दो और भली रीति से विदा कर दो
\end{hindi}}
\flushright{\begin{Arabic}
\quranayah[33][50]
\end{Arabic}}
\flushleft{\begin{hindi}
ऐ नबी! हमने तुम्हारे लिए तुम्हारी वे पत्नियों वैध कर दी है जिनके मह्रक तुम दे चुके हो, और उन स्त्रियों को भी जो तुम्हारी मिल्कियत में आई, जिन्हें अल्लाह ने ग़नीमत के रूप में तुम्हें दी और तुम्हारी चचा की बेटियाँ और तुम्हारी फूफियों की बेटियाँ और तुम्हारे मामुओं की बेटियाँ और तुम्हारी ख़ालाओं की बेटियाँ जिन्होंने तुम्हारे साथ हिजरत की है और वह ईमानवाली स्त्री जो अपने आपको नबी के लिए दे दे, यदि नबी उससे विवाह करना चाहे। ईमानवालों से हटकर यह केवल तुम्हारे ही लिए है, हमें मालूम है जो कुछ हमने उनकी पत्ऩियों और उनकी लौड़ियों के बारे में उनपर अनिवार्य किया है - ताकि तुमपर कोई तंगी न रहे। अल्लाह बहुत क्षमाशील, दयावान है
\end{hindi}}
\flushright{\begin{Arabic}
\quranayah[33][51]
\end{Arabic}}
\flushleft{\begin{hindi}
तुम उनमें से जिसे चाहो अपने से अलग रखो और जिसे चाहो अपने पास रखो, और जिनको तुमने अलग रखा हो, उनमें से किसी के इच्छुक हो तो इसमें तुमपर कोई दोष नहीं, इससे इस बात की अधिक सम्भावना है कि उनकी आँखें ठंड़ी रहें और वे शोकाकुल न हों और जो कुछ तुम उन्हें दो उसपर वे राज़ी रहें। अल्लाह जानता है जो कुछ तुम्हारे दिलों में है। अल्लाह सर्वज्ञ, बहुत सहनशील है
\end{hindi}}
\flushright{\begin{Arabic}
\quranayah[33][52]
\end{Arabic}}
\flushleft{\begin{hindi}
इसके पश्चात तुम्हारे लिए दूसरी स्त्रियाँ वैध नहीं और न यह कि तुम उनकी जगह दूसरी पत्नियों ले आओ, यद्यपि उनका सौन्दर्य तुम्हें कितना ही भाए। उनकी बात औऱ है जो तुम्हारी लौंडियाँ हो। वास्तव में अल्लाह की स्पष्ट हर चीज़ पर है
\end{hindi}}
\flushright{\begin{Arabic}
\quranayah[33][53]
\end{Arabic}}
\flushleft{\begin{hindi}
ऐ ईमान लानेवालो! नबी के घरों में प्रवेश न करो, सिवाय इसके कि कभी तुम्हें खाने पर आने की अनुमति दी जाए। वह भी इस तरह कि उसकी (खाना पकने की) तैयारी की प्रतिक्षा में न रहो। अलबत्ता जब तुम्हें बुलाया जाए तो अन्दर जाओ, और जब तुम खा चुको तो उठकर चले जाओ, बातों में लगे न रहो। निश्चय ही यह हरकत नबी को तकलीफ़ देती है। किन्तु उन्हें तुमसे लज्जा आती है। किन्तु अल्लाह सच्ची बात कहने से लज्जा नहीं करता। और जब तुम उनसे कुछ माँगों तो उनसे परदे के पीछे से माँगो। यह अधिक शुद्धता की बात है तुम्हारे दिलों के लिए और उनके दिलों के लिए भी। तुम्हारे लिए वैध नहीं कि तुम अल्लाह के रसूल को तकलीफ़ पहुँचाओ और न यह कि उसके बाद कभी उसकी पत्नियों से विवाह करो। निश्चय ही अल्लाह की दृष्टि में यह बड़ी गम्भीर बात है
\end{hindi}}
\flushright{\begin{Arabic}
\quranayah[33][54]
\end{Arabic}}
\flushleft{\begin{hindi}
तुम चाहे किसी चीज़ को व्यक्त करो या उसे छिपाओ, अल्लाह को तो हर चीज़ का ज्ञान है
\end{hindi}}
\flushright{\begin{Arabic}
\quranayah[33][55]
\end{Arabic}}
\flushleft{\begin{hindi}
न उनके लिए अपने बापों के सामने होने में कोई दोष है और न अपने बेटों, न अपने भाइयों, न अपने भतीजों, न अपने भांजो, न अपने मेल की स्त्रियों और न जिनपर उन्हें स्वामित्व का अधिकार प्राप्त हो उनके सामने होने में। अल्लाह का डर रखो, निश्चय ही अल्लाह हर चीज़ का साक्षी है
\end{hindi}}
\flushright{\begin{Arabic}
\quranayah[33][56]
\end{Arabic}}
\flushleft{\begin{hindi}
निस्संदेह अल्लाह और उसके फ़रिश्ते नबी पर रहमत भेजते है। ऐ ईमान लानेवालो, तुम भी उसपर रहमत भेजो और ख़ूब सलाम भेजो
\end{hindi}}
\flushright{\begin{Arabic}
\quranayah[33][57]
\end{Arabic}}
\flushleft{\begin{hindi}
जो लोग अल्लाह और उसके रसूल को दुख पहुँचाते है, अल्लाह ने उनपर दुनिया और आख़िरत में लानत की है और उनके लिए अपमानजनक यातना तैयार कर रखी है
\end{hindi}}
\flushright{\begin{Arabic}
\quranayah[33][58]
\end{Arabic}}
\flushleft{\begin{hindi}
और जो लोग ईमानवाले पुरुषों और ईमानवाली स्त्रियों को, बिना इसके कि उन्होंने कुछ किया हो (आरोप लगाकर), दुख पहुँचाते है, उन्होंने तो बड़े मिथ्यारोपण और प्रत्यक्ष गुनाह का बोझ अपने ऊपर उठा लिया
\end{hindi}}
\flushright{\begin{Arabic}
\quranayah[33][59]
\end{Arabic}}
\flushleft{\begin{hindi}
ऐ नबी! अपनी पत्नि यों और अपनी बेटियों और ईमानवाली स्त्रियों से कह दो कि वे अपने ऊपर अपनी चादरों का कुछ हिस्सा लटका लिया करें। इससे इस बात की अधिक सम्भावना है कि वे पहचान ली जाएँ और सताई न जाएँ। अल्लाह बड़ा क्षमाशील, दयावान है
\end{hindi}}
\flushright{\begin{Arabic}
\quranayah[33][60]
\end{Arabic}}
\flushleft{\begin{hindi}
यदि कपटाचारी और वे लोग जिनके दिलों में रोग है और मदीना में खलबली पैदा करनेवाली अफ़वाहें फैलाने से बाज़ न आएँ तो हम तुम्हें उनके विरुद्ध उभार खड़ा करेंगे। फिर वे उसमें तुम्हारे साथ थोड़ा ही रहने पाएँगे,
\end{hindi}}
\flushright{\begin{Arabic}
\quranayah[33][61]
\end{Arabic}}
\flushleft{\begin{hindi}
फिटकारे हुए होंगे। जहाँ कही पाए गए पकड़े जाएँगे और बुरी तरह जान से मारे जाएँगे
\end{hindi}}
\flushright{\begin{Arabic}
\quranayah[33][62]
\end{Arabic}}
\flushleft{\begin{hindi}
यही अल्लाह की रीति रही है उन लोगों के विषय में भी जो पहले गुज़र चुके हैं। और तुम अल्लाह की रीति में कदापि परिवर्तन न पाओगे
\end{hindi}}
\flushright{\begin{Arabic}
\quranayah[33][63]
\end{Arabic}}
\flushleft{\begin{hindi}
लोग तुमसे क़ियामत की घड़ी के बारे में पूछते है। कह दो, "उसका ज्ञान तो बस अल्लाह ही के पास है। तुम्हें क्या मालूम? कदाचित वह घड़ी निकट ही हो।"
\end{hindi}}
\flushright{\begin{Arabic}
\quranayah[33][64]
\end{Arabic}}
\flushleft{\begin{hindi}
निश्चय ही अल्लाह ने इनकार करनेवालों पर लानत की है और उनके लिए भड़कती आग तैयार कर रखी है,
\end{hindi}}
\flushright{\begin{Arabic}
\quranayah[33][65]
\end{Arabic}}
\flushleft{\begin{hindi}
जिसमें वे सदैव रहेंगे। न वे कोई निकटवर्ती समर्थक पाएँगे और न (दूर का) सहायक
\end{hindi}}
\flushright{\begin{Arabic}
\quranayah[33][66]
\end{Arabic}}
\flushleft{\begin{hindi}
जिस दिन उनके चहेरे आग में उलटे-पलटे जाएँगे, वे कहेंगे, "क्या ही अच्छा होता कि हमने अल्लाह का आज्ञापालन किया होता और रसूल का आज्ञापालन किया होता!"
\end{hindi}}
\flushright{\begin{Arabic}
\quranayah[33][67]
\end{Arabic}}
\flushleft{\begin{hindi}
वे कहेंगे, "ऐ हमारे रब! वास्तव में हमने अपने सरदारों और अपने बड़ो का आज्ञा का पालन किया और उन्होंने हमें मार्ग से भटका दिया।
\end{hindi}}
\flushright{\begin{Arabic}
\quranayah[33][68]
\end{Arabic}}
\flushleft{\begin{hindi}
"ऐ हमारे रब! उन्हें दोहरी यातना दे और उनपर बड़ी लानत कर!"
\end{hindi}}
\flushright{\begin{Arabic}
\quranayah[33][69]
\end{Arabic}}
\flushleft{\begin{hindi}
ऐ ईमान लानेवालो! उन लोगों की तरह न हो जाना जिन्होंने मूसा को दुख पहुँचाया, तो अल्लाह ने उससे जो कुछ उन्होंने कहा था उसे बरी कर दिया। वह अल्लाह के यहाँ बड़ा गरिमावान था
\end{hindi}}
\flushright{\begin{Arabic}
\quranayah[33][70]
\end{Arabic}}
\flushleft{\begin{hindi}
ऐ ईमान लानेवालो! अल्लाह का डर रखो और बात कहो ठीक सधी हुई
\end{hindi}}
\flushright{\begin{Arabic}
\quranayah[33][71]
\end{Arabic}}
\flushleft{\begin{hindi}
वह तुम्हारे कर्मों को सँवार देगा और तुम्हारे गुनाहों को क्षमा कर देगा। और जो अल्लाह और उसके रसूल का आज्ञापालन करे, उसने बड़ी सफलता प्राप्त॥ कर ली है
\end{hindi}}
\flushright{\begin{Arabic}
\quranayah[33][72]
\end{Arabic}}
\flushleft{\begin{hindi}
हमने अमानत को आकाशों और धरती और पर्वतों के समक्ष प्रस्तुत किया, किन्तु उन्होंने उसके उठाने से इनकार कर दिया और उससे डर गए। लेकिन मनुष्य ने उसे उठा लिया। निश्चय ही वह बड़ी ज़ालिम, आवेश के वशीभूत हो जानेवाला है
\end{hindi}}
\flushright{\begin{Arabic}
\quranayah[33][73]
\end{Arabic}}
\flushleft{\begin{hindi}
ताकि अल्लाह कपटाचारी पुरुषों और कपटाचारी स्त्रियों और बहुदेववादी पुरुषों और बहुदेववादी स्त्रियों को यातना दे, और ईमानवाले पुरुषों और ईमानवाली स्त्रियों पर अल्लाह कृपा-स्पष्ट करे। वास्तव में अल्लाह बड़ा क्षमाशील, दयावान है
\end{hindi}}
\chapter{Al-Saba' (The Saba')}
\begin{Arabic}
\Huge{\centerline{\basmalah}}\end{Arabic}
\flushright{\begin{Arabic}
\quranayah[34][1]
\end{Arabic}}
\flushleft{\begin{hindi}
प्रशंसा अल्लाह ही के लिए है जिसका वह सब कुछ है जो आकाशों और धरती में है। और आख़िरत में भी उसी के लिए प्रशंसा है। और वही तत्वदर्शी, ख़बर रखनेवाला है
\end{hindi}}
\flushright{\begin{Arabic}
\quranayah[34][2]
\end{Arabic}}
\flushleft{\begin{hindi}
वह जानता है जो कुछ धरती में प्रविष्ट होता है और जो कुथ उससे निकलता है और जो कुछ आकाश से उतरता है और जो कुछ उसमें चढ़ता है। और वही अत्यन्त दयावान, क्षमाशील है
\end{hindi}}
\flushright{\begin{Arabic}
\quranayah[34][3]
\end{Arabic}}
\flushleft{\begin{hindi}
जिन लोगों ने इनकार किया उनका कहना है कि "हमपर क़ियामत की घड़ी नहीं आएगी।" कह दो, "क्यों नहीं, मेरे परोक्ष ज्ञाता रब की क़सम! वह तो तुमपर आकर रहेगी - उससे कणभर भी कोई चीज़ न तो आकाशों में ओझल है और न धरती में, और न इससे छोटी कोई चीज़ और न बड़ी। किन्तु वह एक स्पष्ट किताब में अंकित है। -
\end{hindi}}
\flushright{\begin{Arabic}
\quranayah[34][4]
\end{Arabic}}
\flushleft{\begin{hindi}
"ताकि वह उन लोगों को बदला दे, जो ईमान लाए और उन्होंने अच्छे कर्म किए। वहीं है जिनके लिए क्षमा और प्रतिष्ठामय आजीविका है
\end{hindi}}
\flushright{\begin{Arabic}
\quranayah[34][5]
\end{Arabic}}
\flushleft{\begin{hindi}
"रहे वे लोग जिन्होंने हमारी आयतों को मात करने का प्रयास किया, वह है जिनके लिए बहुत ही बुरे प्रकार की दुखद यातना है।"
\end{hindi}}
\flushright{\begin{Arabic}
\quranayah[34][6]
\end{Arabic}}
\flushleft{\begin{hindi}
जिन लोगों को ज्ञान प्राप्त हुआ है वे स्वयं देखते है कि जो कुछ तुम्हारे रब की ओर से तुम्हारी ओर अवतरित हुआ है वही सत्य है, और वह उसका मार्ग दिखाता है जो प्रभुत्वशाली, प्रशंसा का अधिकारी है
\end{hindi}}
\flushright{\begin{Arabic}
\quranayah[34][7]
\end{Arabic}}
\flushleft{\begin{hindi}
जिन लोगों ने इनकार किया वे कहते है कि "क्या हम तुम्हें एक ऐसा आदमी बताएँ जो तुम्हें ख़बर देता है कि जब तुम बिलकुल चूर्ण-विचूर्ण हो जाओगे तो तुम नवीन काय में जीवित होगे?"
\end{hindi}}
\flushright{\begin{Arabic}
\quranayah[34][8]
\end{Arabic}}
\flushleft{\begin{hindi}
क्या उसने अल्लाह पर झूठ घड़कर थोपा है, या उसे कुछ उन्माद है? नहीं, बल्कि जो लोग आख़िरत पर ईमान नहीं रखते वे यातना और परले दरजे की गुमराही में हैं
\end{hindi}}
\flushright{\begin{Arabic}
\quranayah[34][9]
\end{Arabic}}
\flushleft{\begin{hindi}
क्या उन्होंने आकाश और धरती को नहीं देखा, जो उनके आगे भी है और उनके पीछे भी? यदि हम चाहें तो उन्हें धरती में धँसा दें या उनपर आकाश से कुछ टुकड़े गिरा दें। निश्चय ही इसमें एक निशानी है हर उस बन्दे के लिए जो रुजू करनेवाला हो
\end{hindi}}
\flushright{\begin{Arabic}
\quranayah[34][10]
\end{Arabic}}
\flushleft{\begin{hindi}
हमने दाऊद को अपनी ओर से श्रेष्ठ ता प्रदान की, "ऐ पर्वतों! उसके साथ तसबीह को प्रतिध्वनित करो, और पक्षियों तुम भी!" और हमने उसके लिए लोहे को नर्म कर दिया
\end{hindi}}
\flushright{\begin{Arabic}
\quranayah[34][11]
\end{Arabic}}
\flushleft{\begin{hindi}
कि "पूरी कवचें बना और कड़ियों को ठीक अंदाज़ें से जोड।" - और तुम अच्छा कर्म करो। निस्संदेह जो कुछ तुम करते हो उसे मैं देखता हूँ
\end{hindi}}
\flushright{\begin{Arabic}
\quranayah[34][12]
\end{Arabic}}
\flushleft{\begin{hindi}
और सुलैमान के लिए वायु को वशीभुत कर दिया था। प्रातः समय उसका चलना एक महीने की राह तक और सायंकाल को उसका चलना एक महीने की राह तक - और हमने उसके लिए पिघले हुए ताँबे का स्रोत बहा दिया - और जिन्नों में से भी कुछ को (उसके वशीभूत कर दिया था,) जो अपने रब की अनुज्ञा से उसके आगे काम करते थे। (हमारा आदेशा था,) "उनमें से जो हमारे हुक्म से फिरेगा उसे हम भडकती आग का मज़ा चखाएँगे।"
\end{hindi}}
\flushright{\begin{Arabic}
\quranayah[34][13]
\end{Arabic}}
\flushleft{\begin{hindi}
वे उसके लिए बनाते, जो कुछ वह चाहता - बड़े-बड़े भवन, प्रतिमाएँ, बड़े हौज़ जैसे थाल और जमी रहनेवाली देगें - "ऐ दाऊद के लोगों! कर्म करो, कृतज्ञता दिखाने रूप में। मेरे बन्दों में कृतज्ञ थोड़े ही हैं।"
\end{hindi}}
\flushright{\begin{Arabic}
\quranayah[34][14]
\end{Arabic}}
\flushleft{\begin{hindi}
फिर जब हमने उसके लिए मौत का फ़ैसला लागू किया तो फिर उन जिन्नों को उसकी मौत का पता बस भूमि के उस कीड़े ने दिया जो उसकी लाठी को खा रहा था। फिर जब वह गिर पड़ा, तब जिन्नों पर प्रकट हुआ कि यदि वे परोक्ष के जाननेवाले होते तो इस अपमानजनक यातना में पड़े न रहते
\end{hindi}}
\flushright{\begin{Arabic}
\quranayah[34][15]
\end{Arabic}}
\flushleft{\begin{hindi}
सबा के लिए उनके निवास-स्थान ही में एक निशानी थी - दाएँ और बाएँ दो बाग, "खाओ अपने रब की रोज़ी, और उसके प्रति आभार प्रकट करो। भूमि भी अच्छी-सी और रब भी क्षमाशील।"
\end{hindi}}
\flushright{\begin{Arabic}
\quranayah[34][16]
\end{Arabic}}
\flushleft{\begin{hindi}
किन्तु वे ध्यान में न लाए तो हमने उनपर बँध-तोड़ बाढ़ भेज दी और उनके दोनों बाग़ों के बदले में उन्हें दो दूसरे बाग़ दिए, जिनमें कड़वे-कसैले फल और झाड़ थे, और कुछ थोड़ी-सी झड़-बेरियाँ
\end{hindi}}
\flushright{\begin{Arabic}
\quranayah[34][17]
\end{Arabic}}
\flushleft{\begin{hindi}
यह बदला हमने उन्हें इसलिए दिया कि उन्होंने कृतध्नता दिखाई। ऐसा बदला तो हम कृतध्न लोगों को ही देते है
\end{hindi}}
\flushright{\begin{Arabic}
\quranayah[34][18]
\end{Arabic}}
\flushleft{\begin{hindi}
और हमने उनके और उन बस्तियों के बीच जिनमें हमने बरकत रखी थी प्रत्यक्ष बस्तियाँ बसाई और उनमें सफ़र की मंज़िलें ख़ास अंदाज़े पर रखीं, "उनमें रात-दिन निश्चिन्त होकर चलो फिरो!"
\end{hindi}}
\flushright{\begin{Arabic}
\quranayah[34][19]
\end{Arabic}}
\flushleft{\begin{hindi}
किन्तु उन्होंने कहा, "ऐ हमारे रब! हमारी यात्राओं में दूरी कर दे।" उन्होंने स्वयं अपने ही ऊपर ज़ुल्म किया। अन्ततः हम उन्हें (अतीत की) कहानियाँ बनाकर रहे, औऱ उन्हें बिल्कुल छिन्न-भिन्न कर डाला। निश्चय ही इसमें निशानियाँ है प्रत्येक बड़े धैर्यवान, कृतज्ञ के लिए
\end{hindi}}
\flushright{\begin{Arabic}
\quranayah[34][20]
\end{Arabic}}
\flushleft{\begin{hindi}
इबलीस ने उनके विषय में अपना गुमान सत्य पाया और ईमानवालो के एक गिरोह के सिवा उन्होंने उसी का अनुसरण किया
\end{hindi}}
\flushright{\begin{Arabic}
\quranayah[34][21]
\end{Arabic}}
\flushleft{\begin{hindi}
यद्यपि उसको उनपर कोई ज़ोर और अधिकार प्राप्त न था, किन्तु यह इसलिए कि हम उन लोगों को जो आख़िरत पर ईमान रखते है उन लोगों से अलग जान ले जो उसकी ओर से किसी सन्देह में पड़े हुए है। तुम्हारा रब हर चीज़ का अभिरक्षक है
\end{hindi}}
\flushright{\begin{Arabic}
\quranayah[34][22]
\end{Arabic}}
\flushleft{\begin{hindi}
कह दो, "अल्लाह को छोड़कर जिनका तुम्हें (उपास्य होने का) दावा है, उन्हें पुकार कर देखो। वे न अल्लाह में कणभर चीज़ के मालिक है और न धरती में और न उन दोनों में उनका कोई साझी है और न उनमें से कोई उसका सहायक है।"
\end{hindi}}
\flushright{\begin{Arabic}
\quranayah[34][23]
\end{Arabic}}
\flushleft{\begin{hindi}
और उसके यहाँ कोई सिफ़ारिश काम नहीं आएगी, किन्तु उसी की जिसे उसने (सिफ़ारिश करने की) अनुमति दी हो। यहाँ तक कि जब उनके दिलों से घबराहट दूर हो जाएगी, तो वे कहेंगे, "तुम्हारे रब ने क्या कहा?" वे कहेंगे, "सर्वथा सत्य। और वह अत्यन्त उच्च, महान है।"
\end{hindi}}
\flushright{\begin{Arabic}
\quranayah[34][24]
\end{Arabic}}
\flushleft{\begin{hindi}
कहो, "कौन तुम्हें आकाशों और धरती में रोज़ी देता है?" कहो, "अल्लाह!" अब अवश्य ही हम है या तुम ही हो मार्ग पर, या खुली गुमराही में
\end{hindi}}
\flushright{\begin{Arabic}
\quranayah[34][25]
\end{Arabic}}
\flushleft{\begin{hindi}
कहो, "जो अपराध हमने किए, उसकी पूछ तुमसे न होगी और न उसकी पूछ हमसे होगी जो तुम कर रहे हो।"
\end{hindi}}
\flushright{\begin{Arabic}
\quranayah[34][26]
\end{Arabic}}
\flushleft{\begin{hindi}
कह दो, "हमारा रब हम सबको इकट्ठा करेगा। फिर हमारे बीच ठीक-ठीक फ़ैसला कर देगा। वही ख़ूब फ़ैसला करनेवाला, अत्यन्त ज्ञानवान है।"
\end{hindi}}
\flushright{\begin{Arabic}
\quranayah[34][27]
\end{Arabic}}
\flushleft{\begin{hindi}
कहो, "मुझे उनको दिखाओ तो, जिनको तुमने साझीदार बनाकर उसके साथ जोड रखा है। कुछ नहीं, बल्कि बही अल्लाह अत्यन्त प्रभुत्वशाली, तत्वदर्शी है।"
\end{hindi}}
\flushright{\begin{Arabic}
\quranayah[34][28]
\end{Arabic}}
\flushleft{\begin{hindi}
हमने तो तुम्हें सारे ही मनुष्यों को शुभ-सूचना देनेवाला और सावधान करनेवाला बनाकर भेजा, किन्तु अधिकतर लोग जानते नहीं
\end{hindi}}
\flushright{\begin{Arabic}
\quranayah[34][29]
\end{Arabic}}
\flushleft{\begin{hindi}
वे कहते है, "यह वादा कब पूरा होगा, यदि तुम सच्चे हो?"
\end{hindi}}
\flushright{\begin{Arabic}
\quranayah[34][30]
\end{Arabic}}
\flushleft{\begin{hindi}
कह दो, "तुम्हारे लिए एक विशेष दिन की अवधि नियत है, जिससे न एक घड़ी भर पीछे हटोगे और न आगे बढ़ोगे।"
\end{hindi}}
\flushright{\begin{Arabic}
\quranayah[34][31]
\end{Arabic}}
\flushleft{\begin{hindi}
जिन लोगों ने इनकार किया वे कहते है, "हम इस क़ुरआन को कदापि न मानेंगे और न उसको जो इसके आगे है।" और यदि तुम देख पाते जब ज़ालिम अपने रब के सामने खड़े कर दिए जाएँगे। वे आपस में एक-दूसरे पर इल्ज़ाम डाल रहे होंगे। जो लोग कमज़ोर समझे गए वे उन लोगों से जो बड़े बनते थे कहेंगे, "यदि तुम न होते तो हम अवश्य ही ईमानवाले होते।"
\end{hindi}}
\flushright{\begin{Arabic}
\quranayah[34][32]
\end{Arabic}}
\flushleft{\begin{hindi}
वे लोग जो बड़े बनते थे उन लोगों से जो कमज़ोर समझे गए थे, कहेंगे, "क्या हमने तुम्हे उस मार्गदर्शन से रोका था, वह तुम्हारे पास आया था? नहीं, बल्कि तुम स्वयं ही अपराधी हो।"
\end{hindi}}
\flushright{\begin{Arabic}
\quranayah[34][33]
\end{Arabic}}
\flushleft{\begin{hindi}
वे लोग कमज़ोर समझे गए थे बड़े बननेवालों से कहेंगे, "नहीं, बल्कि रात-दिन की मक्कारी थी जब तुम हमसे कहते थे कि हम अल्लाह के साथ कुफ़्र करें और दूसरों को उसका समकक्ष ठहराएँ।" जब वे यातना देखेंगे तो मन ही मन पछताएँगे और हम उन लोगों की गरदनों में जिन्होंने कुफ़्र की नीति अपनाई, तौक़ डाल देंगे। वे वही तो बदले में पाएँगे, जो वे करते रहे थे?
\end{hindi}}
\flushright{\begin{Arabic}
\quranayah[34][34]
\end{Arabic}}
\flushleft{\begin{hindi}
हमने जिस बस्ती में भी कोई सचेतकर्ता भेजा तो वहाँ के सम्पन्न लोगों ने यही कहा कि "जो कुछ देकर तुम्हें भेजा गया है, हम तो उसको नहीं मानते।"
\end{hindi}}
\flushright{\begin{Arabic}
\quranayah[34][35]
\end{Arabic}}
\flushleft{\begin{hindi}
उन्होंने यह भी कहा कि "हम तो धन और संतान में तुमसे बढ़कर है और हम यातनाग्रस्त होनेवाले नहीं।"
\end{hindi}}
\flushright{\begin{Arabic}
\quranayah[34][36]
\end{Arabic}}
\flushleft{\begin{hindi}
कहो, "निस्संदेह मेरा रब जिसके लिए चाहता है रोज़ी कुशादा कर देता है और जिसे चाहता है नपी-तुली देता है। किन्तु अधिकांश लोग जानते नहीं।"
\end{hindi}}
\flushright{\begin{Arabic}
\quranayah[34][37]
\end{Arabic}}
\flushleft{\begin{hindi}
वह चीज़ न तुम्हारे धन है और न तुम्हारी सन्तान, जो तुम्हें हमसे निकट कर दे। अलबता, जो कोई ईमान लाया और उसने अच्छा कर्म किया, तो ऐसे ही लोग है जिनके लिए उसका कई गुना बदला है, जो उन्होंने किया। और वे ऊपरी मंजिल के कक्षों में निश्चिन्तता-पूर्वक रहेंगे
\end{hindi}}
\flushright{\begin{Arabic}
\quranayah[34][38]
\end{Arabic}}
\flushleft{\begin{hindi}
रहे वे लोग जो हमारी आयतों को मात करने के लिए प्रयासरत है, वे लाकर यातनाग्रस्त किए जाएँगे
\end{hindi}}
\flushright{\begin{Arabic}
\quranayah[34][39]
\end{Arabic}}
\flushleft{\begin{hindi}
कह दो, "मेरा रब ही है जो अपने बन्दों में से जिसके लिए चाहता है रोज़ी कुशादा कर देता है और जिसके लिए चाहता है नपी-तुली कर देता है। और जो कुछ भी तुमने ख़र्च किया, उसकी जगह वह तुम्हें और देगा। वह सबसे अच्छा रोज़ी देनेवाला है।"
\end{hindi}}
\flushright{\begin{Arabic}
\quranayah[34][40]
\end{Arabic}}
\flushleft{\begin{hindi}
याद करो जिस दिन वह उन सबको इकट्ठा करेगा, फिर फ़रिश्तों से कहेगा, "क्या तुम्ही को ये पूजते रहे है?"
\end{hindi}}
\flushright{\begin{Arabic}
\quranayah[34][41]
\end{Arabic}}
\flushleft{\begin{hindi}
वे कहेंगे, "महान है तू, हमारा निकटता का मधुर सम्बन्ध तो तुझी से है, उनसे नहीं; बल्कि बात यह है कि वे जिन्नों को पूजते थे। उनमें से अधिकतर उन्हीं पर ईमान रखते थे।"
\end{hindi}}
\flushright{\begin{Arabic}
\quranayah[34][42]
\end{Arabic}}
\flushleft{\begin{hindi}
"अतः आज न तो तुम परस्पर एक-दूसरे के लाभ का अधिकार रखते हो और न हानि का।" और हम उन ज़ालिमों से कहेंगे, "अब उस आग की यातना का मज़ा चखो, जिसे तुम झुठलाते रहे हो।"
\end{hindi}}
\flushright{\begin{Arabic}
\quranayah[34][43]
\end{Arabic}}
\flushleft{\begin{hindi}
उन्हें जब हमारी स्पष्ट़ आयतें पढ़कर सुनाई जाती है तो वे कहते है, "यह तो बस ऐसा व्यक्ति है जो चाहता है कि तुम्हें उनसे रोक दें जिनको तुम्हारे बाप-दादा पूजते रहे है।" और कहते है, "यह तो एक घड़ा हुआ झूठ है।" जिन लोगों ने इनकार किया उन्होंने सत्य के विषय में, जबकि वह उनके पास आया, कह दिया, "यह तो बस एक प्रत्यक्ष जादू है।"
\end{hindi}}
\flushright{\begin{Arabic}
\quranayah[34][44]
\end{Arabic}}
\flushleft{\begin{hindi}
हमने उन्हें न तो किताबे दी थीं, जिनको वे पढ़ते हों और न तुमसे पहले उनकी ओर कोई सावधान करनेवाला ही भेजा था
\end{hindi}}
\flushright{\begin{Arabic}
\quranayah[34][45]
\end{Arabic}}
\flushleft{\begin{hindi}
और झूठलाया उन लोगों ने भी जो उनसे पहले थे। और जो कुछ हमने उन्हें दिया था ये तो उसके दसवें भाग को भी नहीं पहुँचे है। तो उन्होंने मेरे रसूलों को झुठलाया। तो फिर कैसी रही मेरी यातना!
\end{hindi}}
\flushright{\begin{Arabic}
\quranayah[34][46]
\end{Arabic}}
\flushleft{\begin{hindi}
कहो, "मैं तुम्हें बस एक बात की नसीहत करता हूँ कि अल्लाह के लिए दो-दो औऱ एक-एक करके उठ रखे हो; फिर विचार करो। तुम्हारे साथी को कोई उन्माद नहीं है। वह तो एक कठोर यातना से पहले तुम्हें सचेत करनेवाला ही है।"
\end{hindi}}
\flushright{\begin{Arabic}
\quranayah[34][47]
\end{Arabic}}
\flushleft{\begin{hindi}
कहो, "मैं तुमसे कोई बदला नहीं माँगता वह तुम्हें ही मुबारक हो। मेरा बदला तो बस अल्लाह के ज़िम्मे है और वह हर चीज का साक्षी है।"
\end{hindi}}
\flushright{\begin{Arabic}
\quranayah[34][48]
\end{Arabic}}
\flushleft{\begin{hindi}
कहो, "निश्चय ही मेरा रब सत्य को असत्य पर ग़ालिब करता है। वह परोक्ष की बातें भली-भाँथि जानता है।"
\end{hindi}}
\flushright{\begin{Arabic}
\quranayah[34][49]
\end{Arabic}}
\flushleft{\begin{hindi}
कह दो, "सत्य आ गया (असत्य मिट गया) और असत्य न तो आरम्भ करता है और न पुनरावृत्ति ही।"
\end{hindi}}
\flushright{\begin{Arabic}
\quranayah[34][50]
\end{Arabic}}
\flushleft{\begin{hindi}
कहो, "यदि मैं पथभ्रष्ट॥ हो जाऊँ तो पथभ्रष्ट होकर मैं अपना ही बुरा करूँगा, और यदि मैं सीधे मार्ग पर हूँ, तो इसका कारण वह प्रकाशना है जो मेरा रब मेरी ओर करता है। निस्संदेह वह सब कुछ सुनता है, निकट है।"
\end{hindi}}
\flushright{\begin{Arabic}
\quranayah[34][51]
\end{Arabic}}
\flushleft{\begin{hindi}
और यदि तुम देख लेते जब वे घबराए हुए होंगे; फिर बचकर भाग न सकेंगे और निकट स्थान ही से पकड़ लिए जाएँगे
\end{hindi}}
\flushright{\begin{Arabic}
\quranayah[34][52]
\end{Arabic}}
\flushleft{\begin{hindi}
और कहेंगे, "हम उसपर ईमान ले आए।" हालाँकि उनके लिए कहाँ सम्भव है कि इतने दूरस्थ स्थान से उसको पास सकें
\end{hindi}}
\flushright{\begin{Arabic}
\quranayah[34][53]
\end{Arabic}}
\flushleft{\begin{hindi}
इससे पहले तो उन्होंने उसका इनकार किया और दूरस्थ स्थान से बिन देखे तीर-तूक्के चलाते रहे
\end{hindi}}
\flushright{\begin{Arabic}
\quranayah[34][54]
\end{Arabic}}
\flushleft{\begin{hindi}
उनके और उनकी चाहतों के बीच रोक लगा दी जाएगी; जिस प्रकार इससे पहले उनके सहमार्गी लोगों के साथ मामला किया गया। निश्चय ही वे डाँवाडोल कर देनेवाले संदेह में पड़े रहे हैं
\end{hindi}}
\chapter{Al-Fatir (The Originator)}
\begin{Arabic}
\Huge{\centerline{\basmalah}}\end{Arabic}
\flushright{\begin{Arabic}
\quranayah[35][1]
\end{Arabic}}
\flushleft{\begin{hindi}
सब प्रशंसा अल्लाह के लिए है, जो आकाशों और धरती का पैदा करनेवाला है। दो-दो, तीन-तीन और चार-चार फ़रिश्तों को बाज़ुओंवालों सन्देशवाहक बनाकर नियुक्त करता है। वह संरचना में जैसी चाहता है, अभिवृद्धि करता है। निश्चय ही अल्लाह को हर चीज़ की सामर्थ्य प्राप्त है
\end{hindi}}
\flushright{\begin{Arabic}
\quranayah[35][2]
\end{Arabic}}
\flushleft{\begin{hindi}
अल्लाह जो दयालुता लोगों के लिए खोल दे उसे कोई रोकनेवाला नहीं और जिसे वह रोक ले तो उसके बाद उसे कोई जारी करनेवाला भी नहीं। वह अत्यन्त प्रभुत्वशाली, तत्वदर्शी है
\end{hindi}}
\flushright{\begin{Arabic}
\quranayah[35][3]
\end{Arabic}}
\flushleft{\begin{hindi}
ऐ लोगो! अल्लाह की तुमपर जो अनुकम्पा है, उसे याद करो। क्या अल्लाह के सिवा कोई और पैदा करनेवाला है, जो तुम्हें आकाश और धरती से रोज़ी देता हो? उसके सिवा कोई पूज्य-प्रभु नहीं। तो तुम कहाँ से उलटे भटके चले जा रहे हो?
\end{hindi}}
\flushright{\begin{Arabic}
\quranayah[35][4]
\end{Arabic}}
\flushleft{\begin{hindi}
और यदि वे तुम्हें झुठलाते तो तुमसे पहले भी कितने ही रसूल झुठलाए जा चुके है। सारे मामले अल्लाह ही की ओर पलटते हैं
\end{hindi}}
\flushright{\begin{Arabic}
\quranayah[35][5]
\end{Arabic}}
\flushleft{\begin{hindi}
ऐ लोगों! निश्चय ही अल्लाह का वादा सच्चा है। अतः सांसारिक जीवन तुम्हें धोखे में न डाले और न वह धोखेबाज़ अल्लाह के विषय में तुम्हें धोखा दे
\end{hindi}}
\flushright{\begin{Arabic}
\quranayah[35][6]
\end{Arabic}}
\flushleft{\begin{hindi}
निश्चय ही शैतान तुम्हारा शत्रु है। अतः तुम उसे शत्रु ही समझो। वह तो अपने गिरोह को केवल इसी लिए बुला रहा है कि वे दहकती आगवालों में सम्मिलित हो जाएँ
\end{hindi}}
\flushright{\begin{Arabic}
\quranayah[35][7]
\end{Arabic}}
\flushleft{\begin{hindi}
वे लोग है कि जिन्होंने इनकार किया उनके लिए कठोर यातना है। किन्तु जो ईमान लाए और उन्होंने अच्छे कर्म किए उनके लिए क्षमा और बड़ा प्रतिदान है
\end{hindi}}
\flushright{\begin{Arabic}
\quranayah[35][8]
\end{Arabic}}
\flushleft{\begin{hindi}
फिर क्या वह व्यक्ति जिसके लिए उसका बुरा कर्म सुहाना बना दिया गया हो और वह उसे अच्छा दिख रहा हो (तो क्या वह बुराई को छोड़ेगा)? निश्चय ही अल्लाह जिसे चाहता है मार्ग से वंचित रखता है और जिसे चाहता है सीधा मार्ग दिखाता है। अतः उनपर अफ़सोस करते-करते तुम्हारी जान न जाती रहे। अल्लाह भली-भाँति जानता है जो कुछ वे रच रहे है
\end{hindi}}
\flushright{\begin{Arabic}
\quranayah[35][9]
\end{Arabic}}
\flushleft{\begin{hindi}
अल्लाह ही तो है जिसने हवाएँ चलाई फिर वह बादलों को उभारती है, फिर हम उसे किसी शुष्क और निर्जीव भूभाग की ओर ले गए, और उसके द्वारा हमने धरती को उसके मुर्दा हो जाने के पश्चात जीवित कर दिया। इसी प्रकार (लोगों का नए सिरे से) जीवित होकर उठना भी है
\end{hindi}}
\flushright{\begin{Arabic}
\quranayah[35][10]
\end{Arabic}}
\flushleft{\begin{hindi}
जो कोई प्रभुत्व चाहता हो तो प्रभुत्व तो सारा का सारा अल्लाह के लिए है। उसी की ओर अच्छा-पवित्र बोल चढ़ता है और अच्छा कर्म उसे ऊँचा उठाता है। रहे वे लोग जो बुरी चालें चलते है, उनके लिए कठोर यातना है और उनकी चालबाज़ी मटियामेट होकर रहेगी
\end{hindi}}
\flushright{\begin{Arabic}
\quranayah[35][11]
\end{Arabic}}
\flushleft{\begin{hindi}
अल्लाह ने तुम्हें मिट्टी से पैदा किया, फिर वीर्य से, फिर तुम्हें जोड़े-जोड़े बनाया। उसके ज्ञान के बिना न कोई स्त्री गर्भवती होती है और न जन्म देती है। और जो कोई आयु को प्राप्ति करनेवाला आयु को प्राप्त करता है और जो कुछ उसकी आयु में कमी होती है। अनिवार्यतः यह सब एक किताब में लिखा होता है। निश्चय ही यह सब अल्लाह के लिए अत्यन्त सरल है
\end{hindi}}
\flushright{\begin{Arabic}
\quranayah[35][12]
\end{Arabic}}
\flushleft{\begin{hindi}
दोनों सागर समान नहीं, यह मीठा सुस्वाद है जिससे प्यास जाती रहे, पीने में रुचिकर। और यह खारा-कडुवा है। और तुम प्रत्येक में से तरोताज़ा माँस खाते हो और आभूषण निकालते हो, जिसे तुम पहनते हो। और तुम नौकाओं को देखते हो कि चीरती हुई उसमें चली जा रही हैं, ताकि तुम उसका उदार अनुग्रह तलाश करो और कदाचित तुम आभारी बनो
\end{hindi}}
\flushright{\begin{Arabic}
\quranayah[35][13]
\end{Arabic}}
\flushleft{\begin{hindi}
वह रात को दिन में प्रविष्ट करता है और दिन को रात में प्रविष्ट करता हैं। उसने सूर्य और चन्द्रमा को काम में लगा रखा है। प्रत्येक एक नियत समय पूरी करने के लिए चल रहा है। वही अल्लाह तुम्हारा रब है। उसी की बादशाही है। उससे हटकर जिनको तुम पुकारते हो वे एक तिनके के भी मालिक नहीं
\end{hindi}}
\flushright{\begin{Arabic}
\quranayah[35][14]
\end{Arabic}}
\flushleft{\begin{hindi}
यदि तुम उन्हें पुकारो तो वे तुम्हारी पुकार सुनेगे नहीं। और यदि वे सुनते तो भी तुम्हारी याचना स्वीकार न कर सकते और क़ियामत के दिन वे तुम्हारे साझी ठहराने का इनकार कर देंगे। पूरी ख़बर रखनेवाला (अल्लाह) की तरह तुम्हें कोई न बताएगा
\end{hindi}}
\flushright{\begin{Arabic}
\quranayah[35][15]
\end{Arabic}}
\flushleft{\begin{hindi}
ऐ लोगों! तुम्ही अल्लाह के मुहताज हो और अल्लाह तो निस्पृह, स्वप्रशंसित है
\end{hindi}}
\flushright{\begin{Arabic}
\quranayah[35][16]
\end{Arabic}}
\flushleft{\begin{hindi}
यदि वह चाहे तो तुम्हें हटा दे और एक नई संसृति ले आए
\end{hindi}}
\flushright{\begin{Arabic}
\quranayah[35][17]
\end{Arabic}}
\flushleft{\begin{hindi}
और यह अल्लाह के लिए कुछ भी कठिन नहीं
\end{hindi}}
\flushright{\begin{Arabic}
\quranayah[35][18]
\end{Arabic}}
\flushleft{\begin{hindi}
कोई बोझ उठानेवाला किसी दूसरे का बोझ न उठाएगा। और यदि कोई कोई से दबा हुआ व्यक्ति अपना बोझ उठाने के लिए पुकारे तो उसमें से कुछ भी न उठाया, यद्यपि वह निकट का सम्बन्धी ही क्यों न हो। तुम तो केवल सावधान कर रहे हो। जो परोक्ष में रहते हुए अपने रब से डरते हैं और नमाज़ के पाबन्द हो चुके है (उनकी आत्मा का विकास हो गया) । और जिसने स्वयं को विकसित किया वह अपने ही भले के लिए अपने आपको विकसित करेगा। और पलटकर जाना तो अल्लाह ही की ओर है
\end{hindi}}
\flushright{\begin{Arabic}
\quranayah[35][19]
\end{Arabic}}
\flushleft{\begin{hindi}
अंधा और आँखोंवाला बराबर नहीं,
\end{hindi}}
\flushright{\begin{Arabic}
\quranayah[35][20]
\end{Arabic}}
\flushleft{\begin{hindi}
और न अँधेरे और प्रकाश,
\end{hindi}}
\flushright{\begin{Arabic}
\quranayah[35][21]
\end{Arabic}}
\flushleft{\begin{hindi}
और न छाया और धूप
\end{hindi}}
\flushright{\begin{Arabic}
\quranayah[35][22]
\end{Arabic}}
\flushleft{\begin{hindi}
और न जीवित और मृत बराबर है। निश्चय ही अल्लाह जिसे चाहता है सुनाता है। तुम उन लोगों को नहीं सुना सकते, जो क़ब्रों में हो।
\end{hindi}}
\flushright{\begin{Arabic}
\quranayah[35][23]
\end{Arabic}}
\flushleft{\begin{hindi}
तुम तो बस एक सचेतकर्ता हो
\end{hindi}}
\flushright{\begin{Arabic}
\quranayah[35][24]
\end{Arabic}}
\flushleft{\begin{hindi}
हमने तुम्हें सत्य के साथ भेजा है, शुभ-सूचना देनेवाला और सचेतकर्ता बनाकर। और जो भी समुदाय गुज़रा है, उसमें अनिवार्यतः एक सचेतकर्ता हुआ है
\end{hindi}}
\flushright{\begin{Arabic}
\quranayah[35][25]
\end{Arabic}}
\flushleft{\begin{hindi}
यदि वे तुम्हें झुठलाते है तो जो उनसे पहले थे वे भी झुठला चुके है। उनके रसूल उनके पास स्पष्ट और ज़बूरें और प्रकाशमान किताब लेकर आए थे
\end{hindi}}
\flushright{\begin{Arabic}
\quranayah[35][26]
\end{Arabic}}
\flushleft{\begin{hindi}
फिर मैं उन लोगों को, जिन्होंने इनकार किया, पकड़ लिया (तो फिर कैसा रहा मेरा इनकार!)
\end{hindi}}
\flushright{\begin{Arabic}
\quranayah[35][27]
\end{Arabic}}
\flushleft{\begin{hindi}
क्या तुमने नहीं देखा कि अल्लाह ने आकाश से पानी बरसाया, फिर उसके द्वारा हमने फल निकाले, जिनके रंग विभिन्न प्रकार के होते है? और पहाड़ो में भी श्वेत और लाल विभिन्न रंगों की धारियाँ पाई जाती है, और भुजंग काली भी
\end{hindi}}
\flushright{\begin{Arabic}
\quranayah[35][28]
\end{Arabic}}
\flushleft{\begin{hindi}
और मनुष्यों और जानवरों और चौपायों के रंग भी इसी प्रकार भिन्न हैं। अल्लाह से डरते तो उसके वही बन्दे हैं, जो बाख़बर है। निश्चय ही अल्लाह अत्यन्त प्रभुत्वशाली, क्षमाशील है
\end{hindi}}
\flushright{\begin{Arabic}
\quranayah[35][29]
\end{Arabic}}
\flushleft{\begin{hindi}
निश्चय ही जो लोग अल्लाह की किताब पढ़ते हैं, इस हाल मे कि नमाज़ के पाबन्द हैं, और जो कुछ हमने उन्हें दिया है, उसमें से छिपे और खुले ख़र्च किया है, वे एक ऐसे व्यापार की आशा रखते है जो कभी तबाह न होगा
\end{hindi}}
\flushright{\begin{Arabic}
\quranayah[35][30]
\end{Arabic}}
\flushleft{\begin{hindi}
परिणामस्वरूप वह उन्हें उनके प्रतिदान पूरे-पूरे दे और अपने उदार अनुग्रह से उन्हें और अधिक भी प्रदान करे। निस्संदेह वह बहुत क्षमाशील, अत्यन्त गुणग्राहक है
\end{hindi}}
\flushright{\begin{Arabic}
\quranayah[35][31]
\end{Arabic}}
\flushleft{\begin{hindi}
जो किताब हमने तुम्हारी ओर प्रकाशना द्वारा भेजी है, वही सत्य है। अपने से पहले (की किताबों) की पुष्टि में है। निश्चय ही अल्लाह अपने बन्दों की ख़बर पूरी रखनेवाला, देखनेवाला है
\end{hindi}}
\flushright{\begin{Arabic}
\quranayah[35][32]
\end{Arabic}}
\flushleft{\begin{hindi}
फिर हमने इस किताब का उत्तराधिकारी उन लोगों को बनाया, जिन्हें हमने अपने बन्दो में से चुन लिया है। अब कोई तो उनमें से अपने आप पर ज़ुल्म करता है और कोई उनमें से मध्य श्रेणी का है और कोई उनमें से अल्लाह के कृपायोग से भलाइयों में अग्रसर है। यही है बड़ी श्रेष्ठता। -
\end{hindi}}
\flushright{\begin{Arabic}
\quranayah[35][33]
\end{Arabic}}
\flushleft{\begin{hindi}
सदैव रहने के बाग है, जिनमें वे प्रवेश करेंगे। वहाँ उन्हें सोने के कंगनों और मोती से आभूषित किया जाएगा। और वहाँ उनका वस्त्र रेशम होगा
\end{hindi}}
\flushright{\begin{Arabic}
\quranayah[35][34]
\end{Arabic}}
\flushleft{\begin{hindi}
और वे कहेंगे, "सब प्रशंसा अल्लाह के लिए है, जिसने हमसे ग़म दूर कर दिया। निश्चय ही हमारा रब अत्यन्त क्षमाशील, बड़ा गुणग्राहक है
\end{hindi}}
\flushright{\begin{Arabic}
\quranayah[35][35]
\end{Arabic}}
\flushleft{\begin{hindi}
जिसने हमें अपने उदार अनुग्रह से रहने के ऐसे घर में उतारा जहाँ न हमें कोई मशक़्क़त उठानी पड़ती है और न हमें कोई थकान ही आती है।"
\end{hindi}}
\flushright{\begin{Arabic}
\quranayah[35][36]
\end{Arabic}}
\flushleft{\begin{hindi}
रहे वे लोग जिन्होंने इनकार किया, उनके लिए जहन्नम की आग है, न उनका काम तमाम किया जाएगा कि मर जाएँ और न उनसे उसकी यातना ही कुछ हल्की की जाएगी। हम ऐसा ही बदला प्रत्येक अकृतज्ञ को देते है
\end{hindi}}
\flushright{\begin{Arabic}
\quranayah[35][37]
\end{Arabic}}
\flushleft{\begin{hindi}
वे वहाँ चिल्लाएँगे कि "ऐ हमारे रब! हमें निकाल ले। हम अच्छा कर्म करेंगे, उससे भिन्न जो हम करते रहे।" "क्या हमने तुम्हें इतनी आयु नहीं दी कि जिसमें कोई होश में आना चाहता तो होश में आ जाता? और तुम्हारे पास सचेतकर्ता भी आया था, तो अब मज़ा चखते रहो! ज़ालिमोंं को कोई सहायक नहीं!"
\end{hindi}}
\flushright{\begin{Arabic}
\quranayah[35][38]
\end{Arabic}}
\flushleft{\begin{hindi}
निस्संदेह अल्लाह आकाशों और धरती की छिपी बात को जानता है। वह तो सीनो तक की बात जानता है
\end{hindi}}
\flushright{\begin{Arabic}
\quranayah[35][39]
\end{Arabic}}
\flushleft{\begin{hindi}
वही तो है जिसने तुम्हे धरती में ख़लीफ़ा बनाया। अब तो कोई इनकार करेगा, उसके इनकार का वबाल उसी पर है। इनकार करनेवालों का इनकार उनके रब के यहाँ केवल प्रकोप ही को बढ़ाता है, और इनकार करनेवालों का इनकार केवल घाटे में ही अभिवृद्धि करता है
\end{hindi}}
\flushright{\begin{Arabic}
\quranayah[35][40]
\end{Arabic}}
\flushleft{\begin{hindi}
कहो, "क्या तुमने अपने ठहराए हुए साझीदारो का अवलोकन भी किया, जिन्हें तुम अल्लाह को छोड़कर पुकारते हो? मुझे दिखाओ उन्होंने धरती का कौन-सा भाग पैदा किया है या आकाशों में उनकी कोई भागीदारी है?" या हमने उन्हें कोई किताब ही है कि उसका कोई स्पष्ट प्रमाण उनके पक्ष में हो? नहीं, बल्कि वे ज़ालिम आपस में एक-दूसरे से केवल धोखे का वादा कर रहे है
\end{hindi}}
\flushright{\begin{Arabic}
\quranayah[35][41]
\end{Arabic}}
\flushleft{\begin{hindi}
अल्लाह ही आकाशों और धरती को थामे हुए है कि वे टल न जाएँ और यदि वे टल जाएँ तो उसके पश्चात कोई भी नहीं जो उन्हें थाम सके। निस्संदेह, वह बहुत सहनशील, क्षमा करनेवाला है
\end{hindi}}
\flushright{\begin{Arabic}
\quranayah[35][42]
\end{Arabic}}
\flushleft{\begin{hindi}
उन्होंने अल्लाह की कड़ी-कड़ी क़समें खाई थी कि यदि उनके पास कोई सचेतकर्ता आए तो वे समुदायों में से प्रत्येक से बढ़कर सीधे मार्ग पर होंगे। किन्तु जब उनके पास एक सचेतकर्ता आ गया तो इस चीज़ ने धरती में उनके घमंड और बुरी चालों के कारण उनकी नफ़रत ही में अभिवृद्धि की,
\end{hindi}}
\flushright{\begin{Arabic}
\quranayah[35][43]
\end{Arabic}}
\flushleft{\begin{hindi}
हालाँकि बुरी चाल अपने ही लोगों को घेर लेती है। तो अब क्या जो रीति अगलों के सिलसिले में रही है वे बस उसी रीति की प्रतिक्षा कर रहे है? तो तुम अल्लाह की रीति में कदापि कोई परिवर्तन न पाओगे और न तुम अल्लाह की रीति को कभी टलते ही पाओगे
\end{hindi}}
\flushright{\begin{Arabic}
\quranayah[35][44]
\end{Arabic}}
\flushleft{\begin{hindi}
क्या वे धरती में चले-फिरे नहीं कि देखते कि उन लोगों का कैसा परिणाम हुआ है जो उनसे पहले गुज़रे हैं? हालाँकि वे शक्ति में उनसे कही बढ़-चढ़कर थे। अल्लाह ऐसा नहीं कि आकाशों में कोई चीज़ उसे मात कर सके और न धरती ही में। निस्संदेह वह सर्वज्ञ, सामर्थ्यमान है
\end{hindi}}
\flushright{\begin{Arabic}
\quranayah[35][45]
\end{Arabic}}
\flushleft{\begin{hindi}
यदि अल्लाह लोगों को उनकी कमाई के कारण पकड़ने पर आ जाए तो इस धरती की पीठ पर किसी जीवधारी को भी न छोड़े। किन्तु वह उन्हें एक नियत समय तक ढील देता है, फिर जब उनका नियत समय आ जाता है तो निश्चय ही अल्लाह तो अपने बन्दों को देख ही रहा है
\end{hindi}}
\chapter{Ya Sin (Ya Sin)}
\begin{Arabic}
\Huge{\centerline{\basmalah}}\end{Arabic}
\flushright{\begin{Arabic}
\quranayah[36][1]
\end{Arabic}}
\flushleft{\begin{hindi}
या॰ सीन॰
\end{hindi}}
\flushright{\begin{Arabic}
\quranayah[36][2]
\end{Arabic}}
\flushleft{\begin{hindi}
गवाह है हिकमतवाला क़ुरआन
\end{hindi}}
\flushright{\begin{Arabic}
\quranayah[36][3]
\end{Arabic}}
\flushleft{\begin{hindi}
- कि तुम निश्चय ही रसूलों में से हो
\end{hindi}}
\flushright{\begin{Arabic}
\quranayah[36][4]
\end{Arabic}}
\flushleft{\begin{hindi}
एक सीधे मार्ग पर
\end{hindi}}
\flushright{\begin{Arabic}
\quranayah[36][5]
\end{Arabic}}
\flushleft{\begin{hindi}
- क्या ही ख़ूब है, प्रभुत्वशाली, अत्यन्त दयावाल का इसको अवतरित करना!
\end{hindi}}
\flushright{\begin{Arabic}
\quranayah[36][6]
\end{Arabic}}
\flushleft{\begin{hindi}
ताकि तुम ऐसे लोगों को सावधान करो, जिनके बाप-दादा को सावधान नहीं किया गया; इस कारण वे गफ़लत में पड़े हुए है
\end{hindi}}
\flushright{\begin{Arabic}
\quranayah[36][7]
\end{Arabic}}
\flushleft{\begin{hindi}
उनमें से अधिकतर लोगों पर बात सत्यापित हो चुकी है। अतः वे ईमान नहीं लाएँगे।
\end{hindi}}
\flushright{\begin{Arabic}
\quranayah[36][8]
\end{Arabic}}
\flushleft{\begin{hindi}
हमने उनकी गर्दनों में तौक़ डाल दिए है जो उनकी ठोड़ियों से लगे है। अतः उनके सिर ऊपर को उचके हुए है
\end{hindi}}
\flushright{\begin{Arabic}
\quranayah[36][9]
\end{Arabic}}
\flushleft{\begin{hindi}
और हमने उनके आगे एक दीवार खड़ी कर दी है और एक दीवार उनके पीछे भी। इस तरह हमने उन्हें ढाँक दिया है। अतः उन्हें कुछ सुझाई नहीं देता
\end{hindi}}
\flushright{\begin{Arabic}
\quranayah[36][10]
\end{Arabic}}
\flushleft{\begin{hindi}
उनके लिए बराबर है तुमने सचेत किया या उन्हें सचेत नहीं किया, वे ईमान नहीं लाएँगे
\end{hindi}}
\flushright{\begin{Arabic}
\quranayah[36][11]
\end{Arabic}}
\flushleft{\begin{hindi}
तुम तो बस सावधान कर रहे हो। जो कोई अनुस्मृति का अनुसरण करे और परोक्ष में रहते हुए रहमान से डरे, अतः क्षमा और प्रतिष्ठामय बदले की शुभ सूचना दे दो
\end{hindi}}
\flushright{\begin{Arabic}
\quranayah[36][12]
\end{Arabic}}
\flushleft{\begin{hindi}
निस्संदेह हम मुर्दों को जीवित करेंगे और हम लिखेंगे जो कुछ उन्होंने आगे के लिए भेजा और उनके चिन्हों को (जो पीछे रहा) । हर चीज़ हमने एक स्पष्ट किताब में गिन रखी है
\end{hindi}}
\flushright{\begin{Arabic}
\quranayah[36][13]
\end{Arabic}}
\flushleft{\begin{hindi}
उनके लिए बस्तीवालों की एक मिसाल पेश करो, जबकि वहाँ भेजे हुए दूत आए
\end{hindi}}
\flushright{\begin{Arabic}
\quranayah[36][14]
\end{Arabic}}
\flushleft{\begin{hindi}
जबकि हमने उनकी ओर दो दूत भेजे, तो उन्होंने झुठला दिया। तब हमने तीसरे के द्वारा शक्ति पहुँचाई, तो उन्होंने कहा, "हम तुम्हारी ओर भेजे गए हैं।"
\end{hindi}}
\flushright{\begin{Arabic}
\quranayah[36][15]
\end{Arabic}}
\flushleft{\begin{hindi}
वे बोले, "तुम तो बस हमारे ही जैसे मनुष्य हो। रहमान ने तो कोई भी चीज़ अवतरित नहीं की है। तुम केवल झूठ बोलते हो।"
\end{hindi}}
\flushright{\begin{Arabic}
\quranayah[36][16]
\end{Arabic}}
\flushleft{\begin{hindi}
उन्होंने कहा, "हमारा रब जानता है कि हम निश्चय ही तुम्हारी ओर भेजे गए है
\end{hindi}}
\flushright{\begin{Arabic}
\quranayah[36][17]
\end{Arabic}}
\flushleft{\begin{hindi}
औऱ हमारी ज़िम्मेदारी तो केवल स्पष्ट रूप से संदेश पहुँचा देने की हैं।"
\end{hindi}}
\flushright{\begin{Arabic}
\quranayah[36][18]
\end{Arabic}}
\flushleft{\begin{hindi}
वे बोले, "हम तो तुम्हें अपशकुन समझते है, यदि तुम बाज न आए तो हम तुम्हें पथराव करके मार डालेंगे और तुम्हें अवश्य हमारी ओर से दुखद यातना पहुँचेगी।"
\end{hindi}}
\flushright{\begin{Arabic}
\quranayah[36][19]
\end{Arabic}}
\flushleft{\begin{hindi}
उन्होंने कहा, "तुम्हारा अवशकुन तो तुम्हारे अपने ही साथ है। क्या यदि तुम्हें याददिहानी कराई जाए (तो यह कोई क्रुद्ध होने की बात है)? नहीं, बल्कि तुम मर्यादाहीन लोग हो।"
\end{hindi}}
\flushright{\begin{Arabic}
\quranayah[36][20]
\end{Arabic}}
\flushleft{\begin{hindi}
इतने में नगर के दूरवर्ती सिरे से एक व्यक्ति दौड़ता हुआ आया। उसने कहा, "ऐ मेरी क़ौम के लोगो! उनका अनुवर्तन करो, जो भेजे गए है।
\end{hindi}}
\flushright{\begin{Arabic}
\quranayah[36][21]
\end{Arabic}}
\flushleft{\begin{hindi}
उसका अनुवर्तन करो जो तुमसे कोई बदला नहीं माँगते और वे सीधे मार्ग पर है
\end{hindi}}
\flushright{\begin{Arabic}
\quranayah[36][22]
\end{Arabic}}
\flushleft{\begin{hindi}
"और मुझे क्या हुआ है कि मैं उसकी बन्दगी न करूँ, जिसने मुझे पैदा किया और उसी की ओर तुम्हें लौटकर जाना है?
\end{hindi}}
\flushright{\begin{Arabic}
\quranayah[36][23]
\end{Arabic}}
\flushleft{\begin{hindi}
"क्या मैं उससे इतर दूसरे उपास्य बना लूँ? यदि रहमान मुझे कोई तकलीफ़ पहुँचाना चाहे तो उनकी सिफ़ारिश मेरे कुछ काम नहीं आ सकती और न वे मुझे छुडा ही सकते है
\end{hindi}}
\flushright{\begin{Arabic}
\quranayah[36][24]
\end{Arabic}}
\flushleft{\begin{hindi}
"तब तो मैं अवश्य स्पष्ट गुमराही में पड़ जाऊँगा
\end{hindi}}
\flushright{\begin{Arabic}
\quranayah[36][25]
\end{Arabic}}
\flushleft{\begin{hindi}
"मैं तो तुम्हारे रब पर ईमान ले आया, अतः मेरी सुनो!"
\end{hindi}}
\flushright{\begin{Arabic}
\quranayah[36][26]
\end{Arabic}}
\flushleft{\begin{hindi}
कहा गया, "प्रवेश करो जन्नत में!" उसने कहा, "ऐ काश! मेरी क़ौम के लोग जानते
\end{hindi}}
\flushright{\begin{Arabic}
\quranayah[36][27]
\end{Arabic}}
\flushleft{\begin{hindi}
कि मेरे रब ने मुझे क्षमा कर दिया और मुझे प्रतिष्ठित लोगों में सम्मिलित कर दिया।"
\end{hindi}}
\flushright{\begin{Arabic}
\quranayah[36][28]
\end{Arabic}}
\flushleft{\begin{hindi}
उसके पश्चात उसकी क़ौम पर हमने आकाश से कोई सेना नहीं उतारी और हम इस तरह उतारा नहीं करते
\end{hindi}}
\flushright{\begin{Arabic}
\quranayah[36][29]
\end{Arabic}}
\flushleft{\begin{hindi}
वह तो केवल एक प्रचंड चीत्कार थी। तो सहसा क्या देखते है कि वे बुझकर रह गए
\end{hindi}}
\flushright{\begin{Arabic}
\quranayah[36][30]
\end{Arabic}}
\flushleft{\begin{hindi}
ऐ अफ़सोस बन्दो पर! जो रसूल भी उनके पास आया, वे उसका परिहास ही करते रहे
\end{hindi}}
\flushright{\begin{Arabic}
\quranayah[36][31]
\end{Arabic}}
\flushleft{\begin{hindi}
क्या उन्होंने नहीं देखा कि उनसे पहले कितनी ही नस्लों को हमने विनष्ट किया कि वे उनकी ओर पलटकर नहीं आएँगे?
\end{hindi}}
\flushright{\begin{Arabic}
\quranayah[36][32]
\end{Arabic}}
\flushleft{\begin{hindi}
और जितने भी है, सबके सब हमारे ही सामने उपस्थित किए जाएँगे
\end{hindi}}
\flushright{\begin{Arabic}
\quranayah[36][33]
\end{Arabic}}
\flushleft{\begin{hindi}
और एक निशानी उनके लिए मृत भूमि है। हमने उसे जीवित किया और उससे अनाज निकाला, तो वे खाते है
\end{hindi}}
\flushright{\begin{Arabic}
\quranayah[36][34]
\end{Arabic}}
\flushleft{\begin{hindi}
और हमने उसमें खजूरों और अंगूरों के बाग लगाए और उसमें स्रोत प्रवाहित किए;
\end{hindi}}
\flushright{\begin{Arabic}
\quranayah[36][35]
\end{Arabic}}
\flushleft{\begin{hindi}
ताकि वे उसके फल खाएँ - हालाँकि यह सब कुछ उनके हाथों का बनाया हुआ नहीं है। - तो क्या वे आभार नहीं प्रकट करते?
\end{hindi}}
\flushright{\begin{Arabic}
\quranayah[36][36]
\end{Arabic}}
\flushleft{\begin{hindi}
महिमावान है वह जिसने सबके जोड़े पैदा किए धरती जो चीजें उगाती है उनमें से भी और स्वयं उनकी अपनी जाति में से भी और उन चीज़ो में से भी जिनको वे नहीं जानते
\end{hindi}}
\flushright{\begin{Arabic}
\quranayah[36][37]
\end{Arabic}}
\flushleft{\begin{hindi}
और एक निशानी उनके लिए रात है। हम उसपर से दिन को खींच लेते है। फिर क्या देखते है कि वे अँधेरे में रह गए
\end{hindi}}
\flushright{\begin{Arabic}
\quranayah[36][38]
\end{Arabic}}
\flushleft{\begin{hindi}
और सूर्य अपने नियत ठिकाने के लिए चला जा रहा है। यह बाँधा हुआ हिसाब है प्रभुत्वशाली, ज्ञानवान का
\end{hindi}}
\flushright{\begin{Arabic}
\quranayah[36][39]
\end{Arabic}}
\flushleft{\begin{hindi}
और रहा चन्द्रमा, तो उसकी नियति हमने मंज़िलों के क्रम में रखी, यहाँ तक कि वह फिर खजूर की पूरानी टेढ़ी टहनी के सदृश हो जाता है
\end{hindi}}
\flushright{\begin{Arabic}
\quranayah[36][40]
\end{Arabic}}
\flushleft{\begin{hindi}
न सूर्य ही से हो सकता है कि चाँद को जा पकड़े और न रात दिन से आगे बढ़ सकती है। सब एक-एक कक्षा में तैर रहे हैं
\end{hindi}}
\flushright{\begin{Arabic}
\quranayah[36][41]
\end{Arabic}}
\flushleft{\begin{hindi}
और एक निशानी उनके लिए यह है कि हमने उनके अनुवर्तियों को भरी हुई नौका में सवार किया
\end{hindi}}
\flushright{\begin{Arabic}
\quranayah[36][42]
\end{Arabic}}
\flushleft{\begin{hindi}
और उनके लिए उसी के सदृश और भी ऐसी चीज़े पैदा की, जिनपर वे सवार होते है
\end{hindi}}
\flushright{\begin{Arabic}
\quranayah[36][43]
\end{Arabic}}
\flushleft{\begin{hindi}
और यदि हम चाहें तो उन्हें डूबो दें। फिर न तो उनकी कोई चीख-पुकार हो और न उन्हें बचाया जा सके
\end{hindi}}
\flushright{\begin{Arabic}
\quranayah[36][44]
\end{Arabic}}
\flushleft{\begin{hindi}
यह तो बस हमारी दयालुता और एक नियत समय तक की सुख-सामग्री है
\end{hindi}}
\flushright{\begin{Arabic}
\quranayah[36][45]
\end{Arabic}}
\flushleft{\begin{hindi}
और जब उनसे कहा जाता है कि उस चीज़ का डर रखो, जो तुम्हारे आगे है और जो तुम्हारे पीछे है, ताकि तुमपर दया कि जाए! (तो चुप्पी साझ लेते है)
\end{hindi}}
\flushright{\begin{Arabic}
\quranayah[36][46]
\end{Arabic}}
\flushleft{\begin{hindi}
उनके पास उनके रब की आयतों में से जो आयत भी आती है, वे उससे कतराते ही है
\end{hindi}}
\flushright{\begin{Arabic}
\quranayah[36][47]
\end{Arabic}}
\flushleft{\begin{hindi}
और जब उनसे कहा जाता है कि "अल्लाह ने जो कुछ रोज़ी तुम्हें दी है उनमें से ख़र्च करो।" तो जिन लोगों ने इनकार किया है, वे उन लोगों से, जो ईमान लाए है, कहते है, "क्या हम उसको खाना खिलाएँ जिसे .दि अल्लाह चाहता तो स्वयं खिला देता? तुम तो बस खुली गुमराही में पड़े हो।"
\end{hindi}}
\flushright{\begin{Arabic}
\quranayah[36][48]
\end{Arabic}}
\flushleft{\begin{hindi}
और वे कहते है कि "यह वादा कब पूरा होगा, यदि तुम सच्चे हो?"
\end{hindi}}
\flushright{\begin{Arabic}
\quranayah[36][49]
\end{Arabic}}
\flushleft{\begin{hindi}
वे तो बस एक प्रचंड चीत्कार की प्रतीक्षा में है, जो उन्हें आ पकड़ेगी, जबकि वे झगड़ते होंगे
\end{hindi}}
\flushright{\begin{Arabic}
\quranayah[36][50]
\end{Arabic}}
\flushleft{\begin{hindi}
फिर न तो वे कोई वसीयत कर पाएँगे और न अपने घरवालों की ओर लौट ही सकेंगे
\end{hindi}}
\flushright{\begin{Arabic}
\quranayah[36][51]
\end{Arabic}}
\flushleft{\begin{hindi}
और नरसिंघा में फूँक मारी जाएगी। फिर क्या देखेंगे कि वे क़ब्रों से निकलकर अपने रब की ओर चल पड़े हैं
\end{hindi}}
\flushright{\begin{Arabic}
\quranayah[36][52]
\end{Arabic}}
\flushleft{\begin{hindi}
कहेंगे, "ऐ अफ़सोस हम पर! किसने हमें सोते से जगा दिया? यह वही चीज़ है जिसका रहमान ने वादा किया था और रसूलों ने सच कहा था।"
\end{hindi}}
\flushright{\begin{Arabic}
\quranayah[36][53]
\end{Arabic}}
\flushleft{\begin{hindi}
बस एक ज़ोर की चिंघाड़ होगी। फिर क्या देखेंगे कि वे सबके-सब हमारे सामने उपस्थित कर दिए गए
\end{hindi}}
\flushright{\begin{Arabic}
\quranayah[36][54]
\end{Arabic}}
\flushleft{\begin{hindi}
अब आज किसी जीव पर कुछ भी ज़ुल्म न होगा और तुम्हें बदले में वही मिलेगा जो कुछ तुम करते रहे हो
\end{hindi}}
\flushright{\begin{Arabic}
\quranayah[36][55]
\end{Arabic}}
\flushleft{\begin{hindi}
निश्चय ही जन्नतवाले आज किसी न किसी काम नें व्यस्त आनन्द ले रहे है
\end{hindi}}
\flushright{\begin{Arabic}
\quranayah[36][56]
\end{Arabic}}
\flushleft{\begin{hindi}
वे और उनकी पत्नियों छायों में मसहरियों पर तकिया लगाए हुए है,
\end{hindi}}
\flushright{\begin{Arabic}
\quranayah[36][57]
\end{Arabic}}
\flushleft{\begin{hindi}
उनके लिए वहाँ मेवे है। औऱ उनके लिए वह सब कुछ मौजूद है, जिसकी वे माँग करें
\end{hindi}}
\flushright{\begin{Arabic}
\quranayah[36][58]
\end{Arabic}}
\flushleft{\begin{hindi}
(उनपर) सलाम है, दयामय रब का उच्चारित किया हुआ
\end{hindi}}
\flushright{\begin{Arabic}
\quranayah[36][59]
\end{Arabic}}
\flushleft{\begin{hindi}
"और ऐ अपराधियों! आज तुम छँटकर अलग हो जाओ
\end{hindi}}
\flushright{\begin{Arabic}
\quranayah[36][60]
\end{Arabic}}
\flushleft{\begin{hindi}
क्या मैंने तुम्हें ताकीद नहीं की थी, ऐ आदम के बेटो! कि शैतान की बन्दगी न करे। वास्तव में वह तुम्हारा खुला शत्रु है
\end{hindi}}
\flushright{\begin{Arabic}
\quranayah[36][61]
\end{Arabic}}
\flushleft{\begin{hindi}
और यह कि मेरी बन्दगी करो? यही सीधा मार्ग है
\end{hindi}}
\flushright{\begin{Arabic}
\quranayah[36][62]
\end{Arabic}}
\flushleft{\begin{hindi}
उसने तो तुममें से बहुत-से गिरोहों को पथभ्रष्ट कर दिया। तो क्या तुम बुद्धि नहीं रखते थे?
\end{hindi}}
\flushright{\begin{Arabic}
\quranayah[36][63]
\end{Arabic}}
\flushleft{\begin{hindi}
यह वही जहन्नम है जिसकी तुम्हें धमकी दी जाती रही है
\end{hindi}}
\flushright{\begin{Arabic}
\quranayah[36][64]
\end{Arabic}}
\flushleft{\begin{hindi}
जो इनकार तुम करते रहे हो, उसके बदले में आज इसमें प्रविष्ट हो जाओ।"
\end{hindi}}
\flushright{\begin{Arabic}
\quranayah[36][65]
\end{Arabic}}
\flushleft{\begin{hindi}
आज हम उनके मुँह पर मुहर लगा देंगे और उनके हाथ हमसे बोलेंगे और जो कुछ वे कमाते रहे है, उनके पाँव उसकी गवाही देंगे
\end{hindi}}
\flushright{\begin{Arabic}
\quranayah[36][66]
\end{Arabic}}
\flushleft{\begin{hindi}
यदि हम चाहें तो उनकी आँखें मेट दें क्योंकि वे (अपने रूढ़) मार्ग की और लपके हुए है। फिर उन्हें सुझाई कहाँ से देगा?
\end{hindi}}
\flushright{\begin{Arabic}
\quranayah[36][67]
\end{Arabic}}
\flushleft{\begin{hindi}
यदि हम चाहें तो उनकी जगह पर ही उनके रूप बिगाड़कर रख दें क्योंकि वे सत्य की ओर न चल सके और वे (गुमराही से) बाज़ नहीं आते।
\end{hindi}}
\flushright{\begin{Arabic}
\quranayah[36][68]
\end{Arabic}}
\flushleft{\begin{hindi}
जिसको हम दीर्धायु देते है, उसको उसकी संरचना में उल्टा फेर देते है। तो क्या वे बुद्धि से काम नहीं लेते?
\end{hindi}}
\flushright{\begin{Arabic}
\quranayah[36][69]
\end{Arabic}}
\flushleft{\begin{hindi}
हमने उस (नबी) को कविता नहीं सिखाई और न वह उसके लिए शोभनीय है। वह तो केवल अनुस्मृति और स्पष्ट क़ुरआन है;
\end{hindi}}
\flushright{\begin{Arabic}
\quranayah[36][70]
\end{Arabic}}
\flushleft{\begin{hindi}
ताकि वह उसे सचेत कर दे जो जीवन्त हो और इनकार करनेवालों पर (यातना की) बात स्थापित हो जाए
\end{hindi}}
\flushright{\begin{Arabic}
\quranayah[36][71]
\end{Arabic}}
\flushleft{\begin{hindi}
क्या उन्होंने देखा नहीं कि हमने उनके लिए अपने हाथों की बनाई हुई चीज़ों में से चौपाए पैदा किए और अब वे उनके मालिक है?
\end{hindi}}
\flushright{\begin{Arabic}
\quranayah[36][72]
\end{Arabic}}
\flushleft{\begin{hindi}
और उन्हें उनके बस में कर दिया कि उनमें से कुछ तो उनकी सवारियाँ हैं और उनमें से कुछ को खाते है।
\end{hindi}}
\flushright{\begin{Arabic}
\quranayah[36][73]
\end{Arabic}}
\flushleft{\begin{hindi}
और उनके लिए उनमें कितने ही लाभ है और पेय भी है। तो क्या वे कृतज्ञता नहीं दिखलाते?
\end{hindi}}
\flushright{\begin{Arabic}
\quranayah[36][74]
\end{Arabic}}
\flushleft{\begin{hindi}
उन्होंने अल्लाह से इतर कितने ही उपास्य बना लिए है कि शायद उन्हें मदद पहुँचे।
\end{hindi}}
\flushright{\begin{Arabic}
\quranayah[36][75]
\end{Arabic}}
\flushleft{\begin{hindi}
वे उनकी सहायता करने की सामर्थ्य नहीं रखते, हालाँकि वे (बहुदेववादियों की अपनी स्पष्ट में) उनके लिए उपस्थित सेनाएँ हैं
\end{hindi}}
\flushright{\begin{Arabic}
\quranayah[36][76]
\end{Arabic}}
\flushleft{\begin{hindi}
अतः उनकी बात तुम्हें शोकाकुल न करे। हम जानते है जो कुछ वे छिपाते और जो कुछ व्यक्त करते है
\end{hindi}}
\flushright{\begin{Arabic}
\quranayah[36][77]
\end{Arabic}}
\flushleft{\begin{hindi}
क्या (इनकार करनेवाले) मनुष्य को नहीं देखा कि हमने उसे वीर्य से पैदा किया? फिर क्या देखते है कि वह प्रत्क्षय विरोधी झगड़ालू बन गया
\end{hindi}}
\flushright{\begin{Arabic}
\quranayah[36][78]
\end{Arabic}}
\flushleft{\begin{hindi}
और उसने हमपर फबती कसी और अपनी पैदाइश को भूल गया। कहता है, "कौन हड्डियों में जान डालेगा, जबकि वे जीर्ण-शीर्ण हो चुकी होंगी?"
\end{hindi}}
\flushright{\begin{Arabic}
\quranayah[36][79]
\end{Arabic}}
\flushleft{\begin{hindi}
कह दो, "उनमें वही जाल डालेगा जिसने उनको पहली बार पैदा किया। वह तो प्रत्येक संसृति को भली-भाँति जानता है
\end{hindi}}
\flushright{\begin{Arabic}
\quranayah[36][80]
\end{Arabic}}
\flushleft{\begin{hindi}
वही है जिसने तुम्हारे लिए हरे-भरे वृक्ष से आग पैदा कर दी। तो लगे हो तुम उससे जलाने।"
\end{hindi}}
\flushright{\begin{Arabic}
\quranayah[36][81]
\end{Arabic}}
\flushleft{\begin{hindi}
क्या जिसने आकाशों और धरती को पैदा किया उसे इसकी सामर्थ्य नहीं कि उन जैसों को पैदा कर दे? क्यों नहीं, जबकि वह महान स्रष्टा , अत्यन्त ज्ञानवान है
\end{hindi}}
\flushright{\begin{Arabic}
\quranayah[36][82]
\end{Arabic}}
\flushleft{\begin{hindi}
उसका मामला तो बस यह है कि जब वह किसी चीज़ (के पैदा करने) का इरादा करता है तो उससे कहता है, "हो जा!" और वह हो जाती है
\end{hindi}}
\flushright{\begin{Arabic}
\quranayah[36][83]
\end{Arabic}}
\flushleft{\begin{hindi}
अतः महिमा है उसकी, जिसके हाथ में हर चीज़ का पूरा अधिकार है। और उसी की ओर तुम लौटकर जाओगे
\end{hindi}}
\chapter{As-Saffat (Those Ranging in Ranks)}
\begin{Arabic}
\Huge{\centerline{\basmalah}}\end{Arabic}
\flushright{\begin{Arabic}
\quranayah[37][1]
\end{Arabic}}
\flushleft{\begin{hindi}
गवाह है परा जमाकर पंक्तिबद्ध होनेवाले;
\end{hindi}}
\flushright{\begin{Arabic}
\quranayah[37][2]
\end{Arabic}}
\flushleft{\begin{hindi}
फिर डाँटनेवाले;
\end{hindi}}
\flushright{\begin{Arabic}
\quranayah[37][3]
\end{Arabic}}
\flushleft{\begin{hindi}
फिर यह ज़िक्र करनेवाले
\end{hindi}}
\flushright{\begin{Arabic}
\quranayah[37][4]
\end{Arabic}}
\flushleft{\begin{hindi}
कि तुम्हारा पूज्य-प्रभु अकेला है।
\end{hindi}}
\flushright{\begin{Arabic}
\quranayah[37][5]
\end{Arabic}}
\flushleft{\begin{hindi}
वह आकाशों और धरती और जो कुछ उनके बीच है सबका रब है और पूर्व दिशाओं का भी रब है
\end{hindi}}
\flushright{\begin{Arabic}
\quranayah[37][6]
\end{Arabic}}
\flushleft{\begin{hindi}
हमने दुनिया के आकाश को सजावट अर्थात तारों से सुसज्जित किया, (रात में मुसाफ़िरों को मार्ग दिखाने के लिए)
\end{hindi}}
\flushright{\begin{Arabic}
\quranayah[37][7]
\end{Arabic}}
\flushleft{\begin{hindi}
और प्रत्येक सरकश शैतान से सुरक्षित रखने के लिए
\end{hindi}}
\flushright{\begin{Arabic}
\quranayah[37][8]
\end{Arabic}}
\flushleft{\begin{hindi}
वे (शैतान) "मलए आला" की ओर कान नहीं लगा पाते और हर ओर से फेंक मारे जाते है भगाने-धुतकारने के लिए।
\end{hindi}}
\flushright{\begin{Arabic}
\quranayah[37][9]
\end{Arabic}}
\flushleft{\begin{hindi}
और उनके लिए अनवरत यातना है
\end{hindi}}
\flushright{\begin{Arabic}
\quranayah[37][10]
\end{Arabic}}
\flushleft{\begin{hindi}
किन्तु यह और बात है कि कोई कुछ उचक ले, इस दशा में एक तेज़ दहकती उल्का उसका पीछा करती है
\end{hindi}}
\flushright{\begin{Arabic}
\quranayah[37][11]
\end{Arabic}}
\flushleft{\begin{hindi}
अब उनके पूछो कि उनके पैदा करने का काम अधिक कठिन है या उन चीज़ों का, जो हमने पैदा कर रखी है। निस्संदेह हमने उनको लेसकर मिट्टी से पैदा किया।
\end{hindi}}
\flushright{\begin{Arabic}
\quranayah[37][12]
\end{Arabic}}
\flushleft{\begin{hindi}
बल्कि तुम तो आश्चर्य में हो और वे है कि परिहास कर रहे है
\end{hindi}}
\flushright{\begin{Arabic}
\quranayah[37][13]
\end{Arabic}}
\flushleft{\begin{hindi}
और जब उन्हें याद दिलाया जाता है, तो वे याद नहीं करते,
\end{hindi}}
\flushright{\begin{Arabic}
\quranayah[37][14]
\end{Arabic}}
\flushleft{\begin{hindi}
और जब कोई निशानी देखते है तो हँसी उड़ाते है
\end{hindi}}
\flushright{\begin{Arabic}
\quranayah[37][15]
\end{Arabic}}
\flushleft{\begin{hindi}
और कहते है, "यह तो बस एक प्रत्यक्ष जादू है
\end{hindi}}
\flushright{\begin{Arabic}
\quranayah[37][16]
\end{Arabic}}
\flushleft{\begin{hindi}
क्या जब हम मर चुके होंगे और मिट्टी और हड्डियाँ होकर रह जाएँगे, तो क्या फिर हम उठाए जाएँगे?
\end{hindi}}
\flushright{\begin{Arabic}
\quranayah[37][17]
\end{Arabic}}
\flushleft{\begin{hindi}
क्या और हमारे पहले के बाप-दादा भी?"
\end{hindi}}
\flushright{\begin{Arabic}
\quranayah[37][18]
\end{Arabic}}
\flushleft{\begin{hindi}
कह दो, "हाँ! और तुम अपमानित भी होंगे।"
\end{hindi}}
\flushright{\begin{Arabic}
\quranayah[37][19]
\end{Arabic}}
\flushleft{\begin{hindi}
वह तो बस एक झिड़की होगी। फिर क्या देखेंगे कि वे ताकने लगे है
\end{hindi}}
\flushright{\begin{Arabic}
\quranayah[37][20]
\end{Arabic}}
\flushleft{\begin{hindi}
और वे कहेंगे, "ऐ अफ़सोस हमपर! यह तो बदले का दिन है।"
\end{hindi}}
\flushright{\begin{Arabic}
\quranayah[37][21]
\end{Arabic}}
\flushleft{\begin{hindi}
यह वही फ़ैसले का दिन है जिसे तुम झुठलाते रहे हो
\end{hindi}}
\flushright{\begin{Arabic}
\quranayah[37][22]
\end{Arabic}}
\flushleft{\begin{hindi}
(कहा जाएगा) "एकत्र करो उन लोगों को जिन्होंने ज़ुल्म किया और उनके जोड़ीदारों को भी और उनको भी जिनकी अल्लाह से हटकर वे बन्दगी करते रहे है।
\end{hindi}}
\flushright{\begin{Arabic}
\quranayah[37][23]
\end{Arabic}}
\flushleft{\begin{hindi}
फिर उन सबको भड़कती हुई आग की राह दिखाओ!"
\end{hindi}}
\flushright{\begin{Arabic}
\quranayah[37][24]
\end{Arabic}}
\flushleft{\begin{hindi}
और तनिक उन्हें ठहराओ, उनसे पूछना है,
\end{hindi}}
\flushright{\begin{Arabic}
\quranayah[37][25]
\end{Arabic}}
\flushleft{\begin{hindi}
"तुम्हें क्या हो गया, जो तुम एक-दूसरे की सहायता नहीं कर रहे हो?"
\end{hindi}}
\flushright{\begin{Arabic}
\quranayah[37][26]
\end{Arabic}}
\flushleft{\begin{hindi}
बल्कि वे तो आज बड़े आज्ञाकारी हो गए है
\end{hindi}}
\flushright{\begin{Arabic}
\quranayah[37][27]
\end{Arabic}}
\flushleft{\begin{hindi}
वे एक-दूसरे की ओर रुख़ करके पूछते हुए कहेंगे,
\end{hindi}}
\flushright{\begin{Arabic}
\quranayah[37][28]
\end{Arabic}}
\flushleft{\begin{hindi}
"तुम तो हमारे पास आते थे दाहिने से (और बाएँ से)"
\end{hindi}}
\flushright{\begin{Arabic}
\quranayah[37][29]
\end{Arabic}}
\flushleft{\begin{hindi}
वे कहेंगे, "नहीं, बल्कि तुम स्वयं ही ईमानवाले न थे
\end{hindi}}
\flushright{\begin{Arabic}
\quranayah[37][30]
\end{Arabic}}
\flushleft{\begin{hindi}
और हमारा तो तुमपर कोई ज़ोर न था, बल्कि तुम स्वयं ही सरकश लोग थे
\end{hindi}}
\flushright{\begin{Arabic}
\quranayah[37][31]
\end{Arabic}}
\flushleft{\begin{hindi}
अन्ततः हमपर हमारे रब की बात सत्यापित होकर रही। निस्संदेह हमें (अपनी करतूत का) मजा़ चखना ही होगा
\end{hindi}}
\flushright{\begin{Arabic}
\quranayah[37][32]
\end{Arabic}}
\flushleft{\begin{hindi}
सो हमने तुम्हे बहकाया। निश्चय ही हम स्वयं बहके हुए थे।"
\end{hindi}}
\flushright{\begin{Arabic}
\quranayah[37][33]
\end{Arabic}}
\flushleft{\begin{hindi}
अतः वे सब उस दिन यातना में एक-दूसरे के सह-भागी होंगे
\end{hindi}}
\flushright{\begin{Arabic}
\quranayah[37][34]
\end{Arabic}}
\flushleft{\begin{hindi}
हम अपराधियों के साथ ऐसा ही किया करते है
\end{hindi}}
\flushright{\begin{Arabic}
\quranayah[37][35]
\end{Arabic}}
\flushleft{\begin{hindi}
उनका हाल यह था कि जब उनसे कहा जाता कि "अल्लाह के सिवा कोई पूज्य-प्रभु नहीं हैं।" तो वे घमंड में आ जाते थे
\end{hindi}}
\flushright{\begin{Arabic}
\quranayah[37][36]
\end{Arabic}}
\flushleft{\begin{hindi}
और कहते थे, "क्या हम एक उन्मादी कवि के लिए अपने उपास्यों को छोड़ दें?"
\end{hindi}}
\flushright{\begin{Arabic}
\quranayah[37][37]
\end{Arabic}}
\flushleft{\begin{hindi}
"नहीं, बल्कि वह सत्य लेकर आया है और वह (पिछले) रसूलों की पुष्टि॥ में है।
\end{hindi}}
\flushright{\begin{Arabic}
\quranayah[37][38]
\end{Arabic}}
\flushleft{\begin{hindi}
निश्चय ही तुम दुखद यातना का मज़ा चखोगे। -
\end{hindi}}
\flushright{\begin{Arabic}
\quranayah[37][39]
\end{Arabic}}
\flushleft{\begin{hindi}
"तुम बदला वही तो पाओगे जो तुम करते हो।"
\end{hindi}}
\flushright{\begin{Arabic}
\quranayah[37][40]
\end{Arabic}}
\flushleft{\begin{hindi}
अलबत्ता अल्लाह के उन बन्दों की बात और है, जिनको उसने चुन लिया है
\end{hindi}}
\flushright{\begin{Arabic}
\quranayah[37][41]
\end{Arabic}}
\flushleft{\begin{hindi}
वही लोग है जिनके लिए जानी-बूझी रोज़ी है,
\end{hindi}}
\flushright{\begin{Arabic}
\quranayah[37][42]
\end{Arabic}}
\flushleft{\begin{hindi}
स्वादिष्ट फल।
\end{hindi}}
\flushright{\begin{Arabic}
\quranayah[37][43]
\end{Arabic}}
\flushleft{\begin{hindi}
और वे नेमत भरी जन्नतों
\end{hindi}}
\flushright{\begin{Arabic}
\quranayah[37][44]
\end{Arabic}}
\flushleft{\begin{hindi}
में सम्मानपूर्वक होंगे, तख़्तों पर आमने-सामने विराजमान होंगे;
\end{hindi}}
\flushright{\begin{Arabic}
\quranayah[37][45]
\end{Arabic}}
\flushleft{\begin{hindi}
उनके बीच विशुद्ध पेय का पात्र फिराया जाएगा,
\end{hindi}}
\flushright{\begin{Arabic}
\quranayah[37][46]
\end{Arabic}}
\flushleft{\begin{hindi}
बिलकुल साफ़, उज्जवल, पीनेवालों के लिए सर्वथा सुस्वादु
\end{hindi}}
\flushright{\begin{Arabic}
\quranayah[37][47]
\end{Arabic}}
\flushleft{\begin{hindi}
न उसमें कोई ख़ुमार होगा और न वे उससे निढाल और मदहोश होंगे।
\end{hindi}}
\flushright{\begin{Arabic}
\quranayah[37][48]
\end{Arabic}}
\flushleft{\begin{hindi}
और उनके पास निगाहें बचाए रखनेवाली, सुन्दर आँखोंवाली स्त्रियाँ होंगी,
\end{hindi}}
\flushright{\begin{Arabic}
\quranayah[37][49]
\end{Arabic}}
\flushleft{\begin{hindi}
मानो वे सुरक्षित अंडे है
\end{hindi}}
\flushright{\begin{Arabic}
\quranayah[37][50]
\end{Arabic}}
\flushleft{\begin{hindi}
फिर वे एक-दूसरे की ओर रुख़ करके आपस में पूछेंगे
\end{hindi}}
\flushright{\begin{Arabic}
\quranayah[37][51]
\end{Arabic}}
\flushleft{\begin{hindi}
उनमें से एक कहनेवाला कहेगा, "मेरा एक साथी था;
\end{hindi}}
\flushright{\begin{Arabic}
\quranayah[37][52]
\end{Arabic}}
\flushleft{\begin{hindi}
जो कहा करता था क्या तुम भी पुष्टि करनेवालों में से हो?
\end{hindi}}
\flushright{\begin{Arabic}
\quranayah[37][53]
\end{Arabic}}
\flushleft{\begin{hindi}
क्या जब हम मर चुके होंगे और मिट्टी और हड्डियाँ होकर रह जाएँगे, तो क्या हम वास्तव में बदला पाएँगे?"
\end{hindi}}
\flushright{\begin{Arabic}
\quranayah[37][54]
\end{Arabic}}
\flushleft{\begin{hindi}
वह कहेगा, "क्या तुम झाँककर देखोगे?"
\end{hindi}}
\flushright{\begin{Arabic}
\quranayah[37][55]
\end{Arabic}}
\flushleft{\begin{hindi}
फिर वह झाँकेगा तो उसे भड़कती हुई आग के बीच में देखेगा
\end{hindi}}
\flushright{\begin{Arabic}
\quranayah[37][56]
\end{Arabic}}
\flushleft{\begin{hindi}
कहेगा, "अल्लाह की क़सम! तुम तो मुझे तबाह ही करने को थे
\end{hindi}}
\flushright{\begin{Arabic}
\quranayah[37][57]
\end{Arabic}}
\flushleft{\begin{hindi}
यदि मेरे रब की अनुकम्पा न होती तो अवश्य ही मैं भी पकड़कर हाज़िर किए गए लोगों में से होता
\end{hindi}}
\flushright{\begin{Arabic}
\quranayah[37][58]
\end{Arabic}}
\flushleft{\begin{hindi}
है ना अब ऐसा कि हम मरने के नहीं।
\end{hindi}}
\flushright{\begin{Arabic}
\quranayah[37][59]
\end{Arabic}}
\flushleft{\begin{hindi}
हमें जो मृत्यु आनी थी वह बस पहले आ चुकी। और हमें कोई यातना ही दी जाएगी!"
\end{hindi}}
\flushright{\begin{Arabic}
\quranayah[37][60]
\end{Arabic}}
\flushleft{\begin{hindi}
निश्चय ही यही बड़ी सफलता है
\end{hindi}}
\flushright{\begin{Arabic}
\quranayah[37][61]
\end{Arabic}}
\flushleft{\begin{hindi}
ऐसी की चीज़ के लिए कर्म करनेवालों को कर्म करना चाहिए
\end{hindi}}
\flushright{\begin{Arabic}
\quranayah[37][62]
\end{Arabic}}
\flushleft{\begin{hindi}
क्या वह आतिथ्य अच्छा है या 'ज़क़्क़ूम' का वृक्ष?
\end{hindi}}
\flushright{\begin{Arabic}
\quranayah[37][63]
\end{Arabic}}
\flushleft{\begin{hindi}
निश्चय ही हमने उस (वृक्ष) को ज़ालिमों के लिए परीक्षा बना दिया है
\end{hindi}}
\flushright{\begin{Arabic}
\quranayah[37][64]
\end{Arabic}}
\flushleft{\begin{hindi}
वह एक वृक्ष है जो भड़कती हुई आग की तह से निकलता है
\end{hindi}}
\flushright{\begin{Arabic}
\quranayah[37][65]
\end{Arabic}}
\flushleft{\begin{hindi}
उसके गाभे मानो शैतानों के सिर (साँपों के फन) है
\end{hindi}}
\flushright{\begin{Arabic}
\quranayah[37][66]
\end{Arabic}}
\flushleft{\begin{hindi}
तो वे उसे खाएँगे और उसी से पेट भरेंगे
\end{hindi}}
\flushright{\begin{Arabic}
\quranayah[37][67]
\end{Arabic}}
\flushleft{\begin{hindi}
फिर उनके लिए उसपर खौलते हुए पानी का मिश्रण होगा
\end{hindi}}
\flushright{\begin{Arabic}
\quranayah[37][68]
\end{Arabic}}
\flushleft{\begin{hindi}
फिर उनकी वापसी भड़कती हुई आग की ओर होगी
\end{hindi}}
\flushright{\begin{Arabic}
\quranayah[37][69]
\end{Arabic}}
\flushleft{\begin{hindi}
निश्चय ही उन्होंने अपने बाप-दादा को पथभ्रष्ट॥ पाया।
\end{hindi}}
\flushright{\begin{Arabic}
\quranayah[37][70]
\end{Arabic}}
\flushleft{\begin{hindi}
फिर वे उन्हीं के पद-चिन्हों पर दौड़ते रहे
\end{hindi}}
\flushright{\begin{Arabic}
\quranayah[37][71]
\end{Arabic}}
\flushleft{\begin{hindi}
और उनसे पहले भी पूर्ववर्ती लोगों में अधिकांश पथभ्रष्ट हो चुके है,
\end{hindi}}
\flushright{\begin{Arabic}
\quranayah[37][72]
\end{Arabic}}
\flushleft{\begin{hindi}
हमने उनमें सचेत करनेवाले भेजे थे।
\end{hindi}}
\flushright{\begin{Arabic}
\quranayah[37][73]
\end{Arabic}}
\flushleft{\begin{hindi}
तो अब देख लो उन लोगों का कैसा परिणाम हुआ, जिन्हे सचेत किया गया था
\end{hindi}}
\flushright{\begin{Arabic}
\quranayah[37][74]
\end{Arabic}}
\flushleft{\begin{hindi}
अलबत्ता अल्लाह के बन्दों की बात और है, जिनको उसने चुन लिया है
\end{hindi}}
\flushright{\begin{Arabic}
\quranayah[37][75]
\end{Arabic}}
\flushleft{\begin{hindi}
नूह ने हमको पुकारा था, तो हम कैसे अच्छे है निवेदन स्वीकार करनेवाले!
\end{hindi}}
\flushright{\begin{Arabic}
\quranayah[37][76]
\end{Arabic}}
\flushleft{\begin{hindi}
हमने उसे और उसके लोगों को बड़ी घुटन और बेचैनी से छुटकारा दिया
\end{hindi}}
\flushright{\begin{Arabic}
\quranayah[37][77]
\end{Arabic}}
\flushleft{\begin{hindi}
और हमने उसकी सतति (औलाद व अनुयायी) ही को बाक़ी रखा
\end{hindi}}
\flushright{\begin{Arabic}
\quranayah[37][78]
\end{Arabic}}
\flushleft{\begin{hindi}
और हमने पीछे आनेवाली नस्लों में उसका अच्छा ज़िक्र छोड़ा
\end{hindi}}
\flushright{\begin{Arabic}
\quranayah[37][79]
\end{Arabic}}
\flushleft{\begin{hindi}
कि "सलाम है नूह पर सम्पूर्ण संसारवालों में!"
\end{hindi}}
\flushright{\begin{Arabic}
\quranayah[37][80]
\end{Arabic}}
\flushleft{\begin{hindi}
निस्संदेह हम उत्तमकारों को ऐसा बदला देते है
\end{hindi}}
\flushright{\begin{Arabic}
\quranayah[37][81]
\end{Arabic}}
\flushleft{\begin{hindi}
निश्चय ही वह हमारे ईमानवाले बन्दों में से था
\end{hindi}}
\flushright{\begin{Arabic}
\quranayah[37][82]
\end{Arabic}}
\flushleft{\begin{hindi}
फिर हमने दूसरो को डूबो दिया।
\end{hindi}}
\flushright{\begin{Arabic}
\quranayah[37][83]
\end{Arabic}}
\flushleft{\begin{hindi}
और इबराहीम भी उसी के सहधर्मियों में से था।
\end{hindi}}
\flushright{\begin{Arabic}
\quranayah[37][84]
\end{Arabic}}
\flushleft{\begin{hindi}
याद करो, जब वह अपने रब के समक्ष भला-चंगा हृदय लेकर आया;
\end{hindi}}
\flushright{\begin{Arabic}
\quranayah[37][85]
\end{Arabic}}
\flushleft{\begin{hindi}
जबकि उसने अपने बाप और अपनी क़ौम के लोगों से कहा, "तुम किस चीज़ की पूजा करते हो?
\end{hindi}}
\flushright{\begin{Arabic}
\quranayah[37][86]
\end{Arabic}}
\flushleft{\begin{hindi}
क्या अल्लाह से हटकर मनघड़ंत उपास्यों को चाह रहे हो?
\end{hindi}}
\flushright{\begin{Arabic}
\quranayah[37][87]
\end{Arabic}}
\flushleft{\begin{hindi}
आख़िर सारे संसार के रब के विषय में तुम्हारा क्या गुमान है?"
\end{hindi}}
\flushright{\begin{Arabic}
\quranayah[37][88]
\end{Arabic}}
\flushleft{\begin{hindi}
फिर उसने एक दृष्टि तारों पर डाली
\end{hindi}}
\flushright{\begin{Arabic}
\quranayah[37][89]
\end{Arabic}}
\flushleft{\begin{hindi}
और कहा, "मैं तो निढाल हूँ।"
\end{hindi}}
\flushright{\begin{Arabic}
\quranayah[37][90]
\end{Arabic}}
\flushleft{\begin{hindi}
अतएव वे उसे छोड़कर चले गए पीठ फेरकर
\end{hindi}}
\flushright{\begin{Arabic}
\quranayah[37][91]
\end{Arabic}}
\flushleft{\begin{hindi}
फिर वह आँख बचाकर उनके देवताओं की ओर गया और कहा, "क्या तुम खाते नहीं?
\end{hindi}}
\flushright{\begin{Arabic}
\quranayah[37][92]
\end{Arabic}}
\flushleft{\begin{hindi}
तुम्हें क्या हुआ है कि तुम बोलते नहीं?"
\end{hindi}}
\flushright{\begin{Arabic}
\quranayah[37][93]
\end{Arabic}}
\flushleft{\begin{hindi}
फिर वह भरपूर हाथ मारते हुए उनपर पिल पड़ा
\end{hindi}}
\flushright{\begin{Arabic}
\quranayah[37][94]
\end{Arabic}}
\flushleft{\begin{hindi}
फिर वे लोग झपटते हुए उसकी ओर आए
\end{hindi}}
\flushright{\begin{Arabic}
\quranayah[37][95]
\end{Arabic}}
\flushleft{\begin{hindi}
उसने कहा, "क्या तुम उनको पूजते हो, जिन्हें स्वयं तराशते हो,
\end{hindi}}
\flushright{\begin{Arabic}
\quranayah[37][96]
\end{Arabic}}
\flushleft{\begin{hindi}
जबकि अल्लाह ने तुम्हे भी पैदा किया है और उनको भी, जिन्हें तुम बनाते हो?"
\end{hindi}}
\flushright{\begin{Arabic}
\quranayah[37][97]
\end{Arabic}}
\flushleft{\begin{hindi}
वे बोले, "उनके लिए एक मकान (अर्थात अग्नि-कुंड) तैयार करके उसे भड़कती आग में डाल दो!"
\end{hindi}}
\flushright{\begin{Arabic}
\quranayah[37][98]
\end{Arabic}}
\flushleft{\begin{hindi}
अतः उन्होंने उसके साथ एक चाल चलनी चाही, किन्तु हमने उन्हीं को नीचा दिखा दिया
\end{hindi}}
\flushright{\begin{Arabic}
\quranayah[37][99]
\end{Arabic}}
\flushleft{\begin{hindi}
उसने कहा, "मैं अपने रब की ओर जा रहा हूँ, वह मेरा मार्गदर्शन करेगा
\end{hindi}}
\flushright{\begin{Arabic}
\quranayah[37][100]
\end{Arabic}}
\flushleft{\begin{hindi}
ऐ मेरे रब! मुझे कोई नेक संतान प्रदान कर।"
\end{hindi}}
\flushright{\begin{Arabic}
\quranayah[37][101]
\end{Arabic}}
\flushleft{\begin{hindi}
तो हमने उसे एक सहनशील पुत्र की शुभ सूचना दी
\end{hindi}}
\flushright{\begin{Arabic}
\quranayah[37][102]
\end{Arabic}}
\flushleft{\begin{hindi}
फिर जब वह उसके साथ दौड़-धूप करने की अवस्था को पहुँचा तो उसने कहा, "ऐ मेरे प्रिय बेटे! मैं स्वप्न में देखता हूँ कि तुझे क़ुरबान कर रहा हूँ। तो अब देख, तेरा क्या विचार है?" उसने कहा, "ऐ मेरे बाप! जो कुछ आपको आदेश दिया जा रहा है उसे कर डालिए। अल्लाह ने चाहा तो आप मुझे धैर्यवान पाएँगे।"
\end{hindi}}
\flushright{\begin{Arabic}
\quranayah[37][103]
\end{Arabic}}
\flushleft{\begin{hindi}
अन्ततः जब दोनों ने अपने आपको (अल्लाह के आगे) झुका दिया और उसने (इबाराहीम ने) उसे कनपटी के बल लिटा दिया (तो उस समय क्या दृश्य रहा होगा, सोचो!)
\end{hindi}}
\flushright{\begin{Arabic}
\quranayah[37][104]
\end{Arabic}}
\flushleft{\begin{hindi}
और हमने उसे पुकारा, "ऐ इबराहीम!
\end{hindi}}
\flushright{\begin{Arabic}
\quranayah[37][105]
\end{Arabic}}
\flushleft{\begin{hindi}
तूने स्वप्न को सच कर दिखाया। निस्संदेह हम उत्तमकारों को इसी प्रकार बदला देते है।"
\end{hindi}}
\flushright{\begin{Arabic}
\quranayah[37][106]
\end{Arabic}}
\flushleft{\begin{hindi}
निस्संदेह यह तो एक खुली हूई परीक्षा थी
\end{hindi}}
\flushright{\begin{Arabic}
\quranayah[37][107]
\end{Arabic}}
\flushleft{\begin{hindi}
और हमने उसे (बेटे को) एक बड़ी क़ुरबानी के बदले में छुड़ा लिया
\end{hindi}}
\flushright{\begin{Arabic}
\quranayah[37][108]
\end{Arabic}}
\flushleft{\begin{hindi}
और हमने पीछे आनेवाली नस्लों में उसका ज़िक्र छोड़ा,
\end{hindi}}
\flushright{\begin{Arabic}
\quranayah[37][109]
\end{Arabic}}
\flushleft{\begin{hindi}
कि "सलाम है इबराहीम पर।"
\end{hindi}}
\flushright{\begin{Arabic}
\quranayah[37][110]
\end{Arabic}}
\flushleft{\begin{hindi}
उत्तमकारों को हम ऐसा ही बदला देते है
\end{hindi}}
\flushright{\begin{Arabic}
\quranayah[37][111]
\end{Arabic}}
\flushleft{\begin{hindi}
निश्चय ही वह हमारे ईमानवाले बन्दों में से था
\end{hindi}}
\flushright{\begin{Arabic}
\quranayah[37][112]
\end{Arabic}}
\flushleft{\begin{hindi}
और हमने उसे इसहाक़ की शुभ सूचना दी, अच्छों में से एक नबी
\end{hindi}}
\flushright{\begin{Arabic}
\quranayah[37][113]
\end{Arabic}}
\flushleft{\begin{hindi}
और हमने उसे और इसहाक़ को बरकत दी। और उन दोनों की संतति में कोई तो उत्तमकार है और कोई अपने आप पर खुला ज़ुल्म करनेवाला
\end{hindi}}
\flushright{\begin{Arabic}
\quranayah[37][114]
\end{Arabic}}
\flushleft{\begin{hindi}
और हम मूसा और हारून पर भी उपकार कर चुके है
\end{hindi}}
\flushright{\begin{Arabic}
\quranayah[37][115]
\end{Arabic}}
\flushleft{\begin{hindi}
और हमने उन्हें और उनकी क़ौम को बड़ी घुटन और बेचैनी से छुटकारा दिया
\end{hindi}}
\flushright{\begin{Arabic}
\quranayah[37][116]
\end{Arabic}}
\flushleft{\begin{hindi}
हमने उनकी सहायता की, तो वही प्रभावी रहे
\end{hindi}}
\flushright{\begin{Arabic}
\quranayah[37][117]
\end{Arabic}}
\flushleft{\begin{hindi}
हमने उनको अत्यन्त स्पष्टा किताब प्रदान की।
\end{hindi}}
\flushright{\begin{Arabic}
\quranayah[37][118]
\end{Arabic}}
\flushleft{\begin{hindi}
और उन्हें सीधा मार्ग दिखाया
\end{hindi}}
\flushright{\begin{Arabic}
\quranayah[37][119]
\end{Arabic}}
\flushleft{\begin{hindi}
और हमने पीछे आनेवाली नस्लों में उसका अच्छा ज़िक्र छोड़ा
\end{hindi}}
\flushright{\begin{Arabic}
\quranayah[37][120]
\end{Arabic}}
\flushleft{\begin{hindi}
कि "सलाम है मूसा और हारून पर!"
\end{hindi}}
\flushright{\begin{Arabic}
\quranayah[37][121]
\end{Arabic}}
\flushleft{\begin{hindi}
निस्संदेह हम उत्तमकारों को ऐसा बदला देते है
\end{hindi}}
\flushright{\begin{Arabic}
\quranayah[37][122]
\end{Arabic}}
\flushleft{\begin{hindi}
निश्चय ही वे दोनों हमारे ईमानवाले बन्दों में से थे
\end{hindi}}
\flushright{\begin{Arabic}
\quranayah[37][123]
\end{Arabic}}
\flushleft{\begin{hindi}
और निस्संदेह इलयास भी रसूलों में से था।
\end{hindi}}
\flushright{\begin{Arabic}
\quranayah[37][124]
\end{Arabic}}
\flushleft{\begin{hindi}
याद करो, जब उसने अपनी क़ौम के लोगों से कहा, "क्या तुम डर नहीं रखते?
\end{hindi}}
\flushright{\begin{Arabic}
\quranayah[37][125]
\end{Arabic}}
\flushleft{\begin{hindi}
क्या तुम 'बअत' (देवता) को पुकारते हो और सर्वोत्तम सृष्टा। को छोड़ देते हो;
\end{hindi}}
\flushright{\begin{Arabic}
\quranayah[37][126]
\end{Arabic}}
\flushleft{\begin{hindi}
अपने रब और अपने अगले बाप-दादा के रब, अल्लाह को!"
\end{hindi}}
\flushright{\begin{Arabic}
\quranayah[37][127]
\end{Arabic}}
\flushleft{\begin{hindi}
किन्तु उन्होंने उसे झुठला दिया। सौ वे निश्चय ही पकड़कर हाज़िर किए जाएँगे
\end{hindi}}
\flushright{\begin{Arabic}
\quranayah[37][128]
\end{Arabic}}
\flushleft{\begin{hindi}
अल्लाह के बन्दों की बात और है, जिनको उसने चुन लिया है
\end{hindi}}
\flushright{\begin{Arabic}
\quranayah[37][129]
\end{Arabic}}
\flushleft{\begin{hindi}
और हमने पीछे आनेवाली नस्लों में उसका अच्छा ज़िक्र छोड़ा
\end{hindi}}
\flushright{\begin{Arabic}
\quranayah[37][130]
\end{Arabic}}
\flushleft{\begin{hindi}
कि "सलाम है इलयास पर!"
\end{hindi}}
\flushright{\begin{Arabic}
\quranayah[37][131]
\end{Arabic}}
\flushleft{\begin{hindi}
निस्संदेह हम उत्तमकारों को ऐसा ही बदला देते है
\end{hindi}}
\flushright{\begin{Arabic}
\quranayah[37][132]
\end{Arabic}}
\flushleft{\begin{hindi}
निश्चय ही वह हमारे ईमानवाले बन्दों में से था
\end{hindi}}
\flushright{\begin{Arabic}
\quranayah[37][133]
\end{Arabic}}
\flushleft{\begin{hindi}
और निश्चय ही लूत भी रसूलों में से था
\end{hindi}}
\flushright{\begin{Arabic}
\quranayah[37][134]
\end{Arabic}}
\flushleft{\begin{hindi}
याद करो, जब हमने उसे और उसके सभी लोगों को बचा लिया,
\end{hindi}}
\flushright{\begin{Arabic}
\quranayah[37][135]
\end{Arabic}}
\flushleft{\begin{hindi}
सिवाय एक बुढ़िया के, जो पीछे रह जानेवालों में से थी
\end{hindi}}
\flushright{\begin{Arabic}
\quranayah[37][136]
\end{Arabic}}
\flushleft{\begin{hindi}
फिर दूसरों को हमने तहस-नहस करके रख दिया
\end{hindi}}
\flushright{\begin{Arabic}
\quranayah[37][137]
\end{Arabic}}
\flushleft{\begin{hindi}
और निस्संदेह तुम उनपर (उनके क्षेत्र) से गुज़रते हो कभी प्रातः करते हुए
\end{hindi}}
\flushright{\begin{Arabic}
\quranayah[37][138]
\end{Arabic}}
\flushleft{\begin{hindi}
और रात में भी। तो क्या तुम बुद्धि से काम नहीं लेते?
\end{hindi}}
\flushright{\begin{Arabic}
\quranayah[37][139]
\end{Arabic}}
\flushleft{\begin{hindi}
और निस्संदेह यूनुस भी रसूलो में से था
\end{hindi}}
\flushright{\begin{Arabic}
\quranayah[37][140]
\end{Arabic}}
\flushleft{\begin{hindi}
याद करो, जब वह भरी नौका की ओर भाग निकला,
\end{hindi}}
\flushright{\begin{Arabic}
\quranayah[37][141]
\end{Arabic}}
\flushleft{\begin{hindi}
फिर पर्ची डालने में शामिल हुआ और उसमें मात खाई
\end{hindi}}
\flushright{\begin{Arabic}
\quranayah[37][142]
\end{Arabic}}
\flushleft{\begin{hindi}
फिर उसे मछली ने निगल लिया और वह निन्दनीय दशा में ग्रस्त हो गया था।
\end{hindi}}
\flushright{\begin{Arabic}
\quranayah[37][143]
\end{Arabic}}
\flushleft{\begin{hindi}
अब यदि वह तसबीह करनेवाला न होता
\end{hindi}}
\flushright{\begin{Arabic}
\quranayah[37][144]
\end{Arabic}}
\flushleft{\begin{hindi}
तो उसी के भीतर उस दिन तक पड़ा रह जाता, जबकि लोग उठाए जाएँगे।
\end{hindi}}
\flushright{\begin{Arabic}
\quranayah[37][145]
\end{Arabic}}
\flushleft{\begin{hindi}
अन्ततः हमने उसे इस दशा में कि वह निढ़ाल था, साफ़ मैदान में डाल दिया।
\end{hindi}}
\flushright{\begin{Arabic}
\quranayah[37][146]
\end{Arabic}}
\flushleft{\begin{hindi}
हमने उसपर बेलदार वृक्ष उगाया था
\end{hindi}}
\flushright{\begin{Arabic}
\quranayah[37][147]
\end{Arabic}}
\flushleft{\begin{hindi}
और हमने उसे एक लाख या उससे अधिक (लोगों) की ओर भेजा
\end{hindi}}
\flushright{\begin{Arabic}
\quranayah[37][148]
\end{Arabic}}
\flushleft{\begin{hindi}
फिर वे ईमान लाए तो हमने उन्हें एक अवधि कर सुख भोगने का अवसर दिया।
\end{hindi}}
\flushright{\begin{Arabic}
\quranayah[37][149]
\end{Arabic}}
\flushleft{\begin{hindi}
अब उनसे पूछो, "क्या तुम्हारे रब के लिए तो बेटियाँ हों और उनके अपने लिए बेटे?
\end{hindi}}
\flushright{\begin{Arabic}
\quranayah[37][150]
\end{Arabic}}
\flushleft{\begin{hindi}
क्या हमने फ़रिश्तों को औरतें बनाया और यह उनकी आँखों देखी बात हैं?"
\end{hindi}}
\flushright{\begin{Arabic}
\quranayah[37][151]
\end{Arabic}}
\flushleft{\begin{hindi}
सुन लो, निश्चय ही वे अपनी मनघड़ंत कहते है
\end{hindi}}
\flushright{\begin{Arabic}
\quranayah[37][152]
\end{Arabic}}
\flushleft{\begin{hindi}
कि "अल्लाह के औलाद हुई है!" निश्चय ही वे झूठे है।
\end{hindi}}
\flushright{\begin{Arabic}
\quranayah[37][153]
\end{Arabic}}
\flushleft{\begin{hindi}
क्या उसने बेटों की अपेक्षा बेटियाँ चुन ली है?
\end{hindi}}
\flushright{\begin{Arabic}
\quranayah[37][154]
\end{Arabic}}
\flushleft{\begin{hindi}
तुम्हें क्या हो गया है? तुम कैसा फ़ैसला करते हो?
\end{hindi}}
\flushright{\begin{Arabic}
\quranayah[37][155]
\end{Arabic}}
\flushleft{\begin{hindi}
तो क्या तुम होश से काम नहीं लेते?
\end{hindi}}
\flushright{\begin{Arabic}
\quranayah[37][156]
\end{Arabic}}
\flushleft{\begin{hindi}
क्या तुम्हारे पास कोई स्पष्ट प्रमाण है?
\end{hindi}}
\flushright{\begin{Arabic}
\quranayah[37][157]
\end{Arabic}}
\flushleft{\begin{hindi}
तो लाओ अपनी किताब, यदि तुम सच्चे हो
\end{hindi}}
\flushright{\begin{Arabic}
\quranayah[37][158]
\end{Arabic}}
\flushleft{\begin{hindi}
उन्होंने अल्लाह और जिन्नों के बीच नाता जोड़ रखा है, हालाँकि जिन्नों को भली-भाँति मालूम है कि वे अवश्य पकड़कर हाज़िर किए जाएँगे-
\end{hindi}}
\flushright{\begin{Arabic}
\quranayah[37][159]
\end{Arabic}}
\flushleft{\begin{hindi}
महान और उच्च है अल्लाह उससे, जो वे बयान करते है। -
\end{hindi}}
\flushright{\begin{Arabic}
\quranayah[37][160]
\end{Arabic}}
\flushleft{\begin{hindi}
अल्लाह के उन बन्दों की बात और है, जिन्हें उसने चुन लिया
\end{hindi}}
\flushright{\begin{Arabic}
\quranayah[37][161]
\end{Arabic}}
\flushleft{\begin{hindi}
अतः तुम और जिनको तुम पूजते हो वे,
\end{hindi}}
\flushright{\begin{Arabic}
\quranayah[37][162]
\end{Arabic}}
\flushleft{\begin{hindi}
तुम सब अल्लाह के विरुद्ध किसी को बहका नहीं सकते,
\end{hindi}}
\flushright{\begin{Arabic}
\quranayah[37][163]
\end{Arabic}}
\flushleft{\begin{hindi}
सिवाय उसके जो जहन्नम की भड़कती आग में पड़ने ही वाला हो
\end{hindi}}
\flushright{\begin{Arabic}
\quranayah[37][164]
\end{Arabic}}
\flushleft{\begin{hindi}
और हमारी ओर से उसके लिए अनिवार्यतः एक ज्ञात और नियत स्थान है
\end{hindi}}
\flushright{\begin{Arabic}
\quranayah[37][165]
\end{Arabic}}
\flushleft{\begin{hindi}
और हम ही पंक्तिबद्ध करते है।
\end{hindi}}
\flushright{\begin{Arabic}
\quranayah[37][166]
\end{Arabic}}
\flushleft{\begin{hindi}
और हम ही महानता बयान करते है
\end{hindi}}
\flushright{\begin{Arabic}
\quranayah[37][167]
\end{Arabic}}
\flushleft{\begin{hindi}
वे तो कहा करते थे,
\end{hindi}}
\flushright{\begin{Arabic}
\quranayah[37][168]
\end{Arabic}}
\flushleft{\begin{hindi}
"यदि हमारे पास पिछलों की कोई शिक्षा होती
\end{hindi}}
\flushright{\begin{Arabic}
\quranayah[37][169]
\end{Arabic}}
\flushleft{\begin{hindi}
तो हम अल्लाह के चुने हुए बन्दे होते।"
\end{hindi}}
\flushright{\begin{Arabic}
\quranayah[37][170]
\end{Arabic}}
\flushleft{\begin{hindi}
किन्तु उन्होंने इनकार कर दिया, तो अब जल्द ही वे जान लेंगे
\end{hindi}}
\flushright{\begin{Arabic}
\quranayah[37][171]
\end{Arabic}}
\flushleft{\begin{hindi}
और हमारे अपने उन बन्दों के हक़ में, जो रसूल बनाकर भेजे गए, हमारी बात पहले ही निश्चित हो चुकी है
\end{hindi}}
\flushright{\begin{Arabic}
\quranayah[37][172]
\end{Arabic}}
\flushleft{\begin{hindi}
कि निश्चय ही उन्हीं की सहायता की जाएगी।
\end{hindi}}
\flushright{\begin{Arabic}
\quranayah[37][173]
\end{Arabic}}
\flushleft{\begin{hindi}
और निश्चय ही हमारी सेना ही प्रभावी रहेगी
\end{hindi}}
\flushright{\begin{Arabic}
\quranayah[37][174]
\end{Arabic}}
\flushleft{\begin{hindi}
अतः एक अवधि तक के लिए उनसे रुख़ फेर लो
\end{hindi}}
\flushright{\begin{Arabic}
\quranayah[37][175]
\end{Arabic}}
\flushleft{\begin{hindi}
और उन्हें देखते रहो। वे भी जल्द ही (अपना परिणाम) देख लेंगे
\end{hindi}}
\flushright{\begin{Arabic}
\quranayah[37][176]
\end{Arabic}}
\flushleft{\begin{hindi}
क्या वे हमारी यातना के लिए जल्दी मचा रहे हैं?
\end{hindi}}
\flushright{\begin{Arabic}
\quranayah[37][177]
\end{Arabic}}
\flushleft{\begin{hindi}
तो जब वह उनके आँगन में उतरेगी तो बड़ी ही बुरी सुबह होगी उन लोगों की, जिन्हें सचेत किया जा चुका है!
\end{hindi}}
\flushright{\begin{Arabic}
\quranayah[37][178]
\end{Arabic}}
\flushleft{\begin{hindi}
एक अवधि तक के लिए उनसे रुख़ फेर लो
\end{hindi}}
\flushright{\begin{Arabic}
\quranayah[37][179]
\end{Arabic}}
\flushleft{\begin{hindi}
और देखते रहो, वे जल्द ही देख लेंगे
\end{hindi}}
\flushright{\begin{Arabic}
\quranayah[37][180]
\end{Arabic}}
\flushleft{\begin{hindi}
महान और उच्च है तुम्हारा रब, प्रताप का स्वामी, उन बातों से जो वे बताते है!
\end{hindi}}
\flushright{\begin{Arabic}
\quranayah[37][181]
\end{Arabic}}
\flushleft{\begin{hindi}
और सलाम है रसूलों पर;
\end{hindi}}
\flushright{\begin{Arabic}
\quranayah[37][182]
\end{Arabic}}
\flushleft{\begin{hindi}
औऱ सब प्रशंसा अल्लाह, सारे संसार के रब के लिए है
\end{hindi}}
\chapter{Sad (Sad)}
\begin{Arabic}
\Huge{\centerline{\basmalah}}\end{Arabic}
\flushright{\begin{Arabic}
\quranayah[38][1]
\end{Arabic}}
\flushleft{\begin{hindi}
साद। क़सम है, याददिहानी-वाले क़ुरआन की (जिसमें कोई कमी नहीं कि धर्मविरोधी सत्य को न समझ सकें)
\end{hindi}}
\flushright{\begin{Arabic}
\quranayah[38][2]
\end{Arabic}}
\flushleft{\begin{hindi}
बल्कि जिन्होंने इनकार किया वे गर्व और विरोध में पड़े हुए है
\end{hindi}}
\flushright{\begin{Arabic}
\quranayah[38][3]
\end{Arabic}}
\flushleft{\begin{hindi}
उनसे पहले हमने कितनी ही पीढ़ियों को विनष्ट किया, तो वे लगे पुकारने। किन्तु वह समय हटने-बचने का न था
\end{hindi}}
\flushright{\begin{Arabic}
\quranayah[38][4]
\end{Arabic}}
\flushleft{\begin{hindi}
उन्होंने आश्चर्य किया इसपर कि उनके पास उन्हीं में से एक सचेतकर्ता आया और इनकार करनेवाले कहने लगे, "यह जादूगर है बड़ा झूठा
\end{hindi}}
\flushright{\begin{Arabic}
\quranayah[38][5]
\end{Arabic}}
\flushleft{\begin{hindi}
क्या उसने सारे उपास्यों को अकेला एक उपास्य ठहरा दिया? निस्संदेह यह तो बहुत अचम्भेवाली चीज़ है!"
\end{hindi}}
\flushright{\begin{Arabic}
\quranayah[38][6]
\end{Arabic}}
\flushleft{\begin{hindi}
और उनके सरदार (यह कहते हुए) चल खड़े हुए कि "चलते रहो और अपने उपास्यों पर जमें रहो। निस्संदेह यह वांछिच चीज़ है
\end{hindi}}
\flushright{\begin{Arabic}
\quranayah[38][7]
\end{Arabic}}
\flushleft{\begin{hindi}
यह बात तो हमने पिछले धर्म में सुनी ही नहीं। यह तो बस मनघड़त है
\end{hindi}}
\flushright{\begin{Arabic}
\quranayah[38][8]
\end{Arabic}}
\flushleft{\begin{hindi}
क्या हम सबमें से (चुनकर) इसी पर अनुस्मृति अवतरित हुई है?" नहीं, बल्कि वे मेरी अनुस्मृति के विषय में संदेह में है, बल्कि उन्होंने अभी तक मेरी यातना का मज़ा चखा ही नहीं है
\end{hindi}}
\flushright{\begin{Arabic}
\quranayah[38][9]
\end{Arabic}}
\flushleft{\begin{hindi}
या, तेरे प्रभुत्वशाली, बड़े दाता रब की दयालुता के ख़ज़ाने उनके पास है?
\end{hindi}}
\flushright{\begin{Arabic}
\quranayah[38][10]
\end{Arabic}}
\flushleft{\begin{hindi}
या, आकाशों और धरती और जो कुछ उनके बीच है, उन सबकी बादशाही उन्हीं की है? फिर तो चाहिए कि वे रस्सियों द्वारा ऊपर चढ़ जाए
\end{hindi}}
\flushright{\begin{Arabic}
\quranayah[38][11]
\end{Arabic}}
\flushleft{\begin{hindi}
वह एक साधारण सेना है (विनष्ट होनेवाले) दलों में से, वहाँ मात खाना जिसकी नियति है
\end{hindi}}
\flushright{\begin{Arabic}
\quranayah[38][12]
\end{Arabic}}
\flushleft{\begin{hindi}
उनसे पहले नूह की क़ौम और आद और मेखोंवाले फ़िरऔन ने झुठलाया
\end{hindi}}
\flushright{\begin{Arabic}
\quranayah[38][13]
\end{Arabic}}
\flushleft{\begin{hindi}
और समूद और लूत की क़ौम और 'ऐकावाले' भी, ये है वे दल
\end{hindi}}
\flushright{\begin{Arabic}
\quranayah[38][14]
\end{Arabic}}
\flushleft{\begin{hindi}
उनमें से प्रत्येक ने रसूलों को झुठलाया, तो मेरी ओर से दंड अवश्यम्भावी होकर रहा
\end{hindi}}
\flushright{\begin{Arabic}
\quranayah[38][15]
\end{Arabic}}
\flushleft{\begin{hindi}
इन्हें बस एक चीख की प्रतीक्षा है जिसमें तनिक भी अवकाश न होगा
\end{hindi}}
\flushright{\begin{Arabic}
\quranayah[38][16]
\end{Arabic}}
\flushleft{\begin{hindi}
वे कहते है, "ऐ हमारे रब! हिसाब के दिन से पहले ही शीघ्र हमारा हिस्सा दे दे।"
\end{hindi}}
\flushright{\begin{Arabic}
\quranayah[38][17]
\end{Arabic}}
\flushleft{\begin{hindi}
वे जो कुछ कहते है उसपर धैर्य से काम लो और ज़ोर व शक्तिवाले हमारे बन्दे दाऊद को याद करो। निश्चय ही वह (अल्लाह की ओर) बहुत रुजू करनेवाला था
\end{hindi}}
\flushright{\begin{Arabic}
\quranayah[38][18]
\end{Arabic}}
\flushleft{\begin{hindi}
हमने पर्वतों को उसके साथ वशीभूत कर दिया था कि प्रातःकाल और सन्ध्य समय तसबीह करते रहे।
\end{hindi}}
\flushright{\begin{Arabic}
\quranayah[38][19]
\end{Arabic}}
\flushleft{\begin{hindi}
और पक्षियों को भी, जो एकत्र हो जाते थे। प्रत्येक उसके आगे रुजू रहता
\end{hindi}}
\flushright{\begin{Arabic}
\quranayah[38][20]
\end{Arabic}}
\flushleft{\begin{hindi}
हमने उसका राज्य सुदृढ़ कर दिया था और उसे तत्वदर्शिता प्रदान की थी और निर्णायक बात कहने की क्षमता प्रदान की थी
\end{hindi}}
\flushright{\begin{Arabic}
\quranayah[38][21]
\end{Arabic}}
\flushleft{\begin{hindi}
और क्या तुम्हें उन विवादियों की ख़बर पहुँची है? जब वे दीवार पर चढ़कर मेहराब (एकान्त कक्ष) मे आ पहुँचे
\end{hindi}}
\flushright{\begin{Arabic}
\quranayah[38][22]
\end{Arabic}}
\flushleft{\begin{hindi}
जब वे दाऊद के पास पहुँचे तो वह उनसे सहम गया। वे बोले, "डरिए नहीं, हम दो विवादी हैं। हममें से एक ने दूसरे पर ज़्यादती की है; तो आप हमारे बीच ठीक-ठीक फ़ैसला कर दीजिए। और बात को दूर न डालिए और हमें ठीक मार्ग बता दीजिए
\end{hindi}}
\flushright{\begin{Arabic}
\quranayah[38][23]
\end{Arabic}}
\flushleft{\begin{hindi}
यह मेरा भाई है। इसके पास निन्यानबे दुंबियाँ है और मेरे पास एक दुंबी है। अब इसका कहना है कि इसे भी मुझे सौप दे और बातचीत में इसने मुझे दबा लिया।"
\end{hindi}}
\flushright{\begin{Arabic}
\quranayah[38][24]
\end{Arabic}}
\flushleft{\begin{hindi}
उसने कहा, "इसने अपनी दुंबियों के साथ तेरी दुंबी को मिला लेने की माँग करके निश्चय ही तुझपर ज़ुल्म किया है। और निस्संदेह बहुत-से साथ मिलकर रहनेवाले एक-दूसरे पर ज़्यादती करते है, सिवाय उन लोगों के जो ईमान लाए और उन्होंने अच्छे कर्म किए। किन्तु ऐसे लोग थोड़े ही है।" अब दाऊद समझ गया कि यह तो हमने उसे परीक्षा में डाला है। अतः उसने अपने रब से क्षमा-याचना की और झुककर (सीधे सजदे में) गिर पड़ा और रुजू हुआ
\end{hindi}}
\flushright{\begin{Arabic}
\quranayah[38][25]
\end{Arabic}}
\flushleft{\begin{hindi}
तो हमने उसका वह क़सूर माफ़ कर दिया। और निश्चय ही हमारे यहाँ उसके लिए अनिवार्यतः सामीप्य और उत्तम ठिकाना है
\end{hindi}}
\flushright{\begin{Arabic}
\quranayah[38][26]
\end{Arabic}}
\flushleft{\begin{hindi}
"ऐ दाऊद! हमने धरती में तुझे ख़लीफ़ा (उत्तराधिकारी) बनाया है। अतः तू लोगों के बीच हक़ के साथ फ़ैसला करना और अपनी इच्छा का अनुपालन न करना कि वह तुझे अल्लाह के मार्ग से भटका दे। जो लोग अल्लाह के मार्ग से भटकते है, निश्चय ही उनके लिए कठोर यातना है, क्योंकि वे हिसाब के दिन को भूले रहे।-
\end{hindi}}
\flushright{\begin{Arabic}
\quranayah[38][27]
\end{Arabic}}
\flushleft{\begin{hindi}
हमने आकाश और धरती को और जो कुछ उनके बीच है, व्यर्थ नहीं पैदा किया। यह तो उन लोगों का गुमान है जिन्होंने इनकार किया। अतः आग में झोंके जाने के कारण इनकार करनेवालों की बड़ी दुर्गति है
\end{hindi}}
\flushright{\begin{Arabic}
\quranayah[38][28]
\end{Arabic}}
\flushleft{\begin{hindi}
(क्या हम उनको जो समझते है कि जगत की संरचना व्यर्थ नहीं है, उनके समान कर देंगे जो जगत को निरर्थक मानते है।) या हम उन लोगों को जो ईमान लाए और उन्होंने अच्छे कर्म किए, उनके समान कर देंगे जो धरती में बिगाड़ पैदा करते है; या डर रखनेवालों को हम दुराचारियों जैसा कर देंगे?
\end{hindi}}
\flushright{\begin{Arabic}
\quranayah[38][29]
\end{Arabic}}
\flushleft{\begin{hindi}
यह एक इसकी आयतों पर सोच-विचार करें और ताकि बुद्धि और समझवाले इससे शिक्षा ग्रहण करें।-
\end{hindi}}
\flushright{\begin{Arabic}
\quranayah[38][30]
\end{Arabic}}
\flushleft{\begin{hindi}
और हमने दाऊद को सुलैमान प्रदान किया। वह कितना अच्छा बन्दा था! निश्चय ही वह बहुत ही रुजू रहनेवाला था।
\end{hindi}}
\flushright{\begin{Arabic}
\quranayah[38][31]
\end{Arabic}}
\flushleft{\begin{hindi}
याद करो, जबकि सन्ध्या समय उसके सामने सधे हुए द्रुतगामी घोड़े हाज़िर किए गए
\end{hindi}}
\flushright{\begin{Arabic}
\quranayah[38][32]
\end{Arabic}}
\flushleft{\begin{hindi}
तो उसने कहा, "मैंने इनके प्रति प्रेम अपने रब की याद के कारण अपनाया है।" यहाँ तक कि वे (घोड़े) ओट में छिप गए
\end{hindi}}
\flushright{\begin{Arabic}
\quranayah[38][33]
\end{Arabic}}
\flushleft{\begin{hindi}
"उन्हें मेरे पास वापस लाओ!" फिर वह उनकी पिंडलियों और गरदनों पर हाथ फेरने लगा
\end{hindi}}
\flushright{\begin{Arabic}
\quranayah[38][34]
\end{Arabic}}
\flushleft{\begin{hindi}
निश्चय ही हमने सुलैमान को भी परीक्षा में डाला। और हमने उसके तख़्त पर एक धड़ डाल दिया। फिर वह रुजू हुआ
\end{hindi}}
\flushright{\begin{Arabic}
\quranayah[38][35]
\end{Arabic}}
\flushleft{\begin{hindi}
उसने कहा, "ऐ मेरे रब! मुझे क्षमा कर दे और मुझे वह राज्य प्रदान कर, जो मेरे पश्चात किसी के लिए शोभनीय न हो। निश्चय ही तू बड़ा दाता है।"
\end{hindi}}
\flushright{\begin{Arabic}
\quranayah[38][36]
\end{Arabic}}
\flushleft{\begin{hindi}
तब हमने वायु को उसके लिए वशीभूत कर दिया, जो उसके आदेश से, जहाँ वह पहुँचना चाहता, सरलतापूर्वक चलती थी
\end{hindi}}
\flushright{\begin{Arabic}
\quranayah[38][37]
\end{Arabic}}
\flushleft{\begin{hindi}
और शैतानों को भी (वशीभुत कर दिया), प्रत्येक निर्माता और ग़ोताख़ोर को
\end{hindi}}
\flushright{\begin{Arabic}
\quranayah[38][38]
\end{Arabic}}
\flushleft{\begin{hindi}
और दूसरों को भी जो ज़जीरों में जकड़े हुए रहत
\end{hindi}}
\flushright{\begin{Arabic}
\quranayah[38][39]
\end{Arabic}}
\flushleft{\begin{hindi}
"यह हमारी बेहिसाब देन है। अब एहसान करो या रोको।"
\end{hindi}}
\flushright{\begin{Arabic}
\quranayah[38][40]
\end{Arabic}}
\flushleft{\begin{hindi}
और निश्चय ही हमारे यहाँ उसके लिए अनिवार्यतः समीप्य और उत्तम ठिकाना है
\end{hindi}}
\flushright{\begin{Arabic}
\quranayah[38][41]
\end{Arabic}}
\flushleft{\begin{hindi}
हमारे बन्दे अय्यूब को भी याद करो, जब उसने अपने रब को पुकारा कि "शैतान ने मुझे दुख और पीड़ा पहुँचा रखी है।"
\end{hindi}}
\flushright{\begin{Arabic}
\quranayah[38][42]
\end{Arabic}}
\flushleft{\begin{hindi}
"अपना पाँव (धरती पर) मार, यह है ठंडा (पानी) नहाने को और पीने को।"
\end{hindi}}
\flushright{\begin{Arabic}
\quranayah[38][43]
\end{Arabic}}
\flushleft{\begin{hindi}
और हमने उसे उसके परिजन दिए और उनके साथ वैसे ही और भी; अपनी ओर से दयालुता के रूप में और बुद्धि और समझ रखनेवालों के लिए शिक्षा के रूप में।
\end{hindi}}
\flushright{\begin{Arabic}
\quranayah[38][44]
\end{Arabic}}
\flushleft{\begin{hindi}
"और अपने हाथ में तिनकों का एक मुट्ठा ले और उससे मार और अपनी क़सम न तोड़।" निश्चय ही हमने उसे धैर्यवान पाया, क्या ही अच्छा बन्दा! निस्संदेह वह बड़ा ही रुजू रहनेवाला था
\end{hindi}}
\flushright{\begin{Arabic}
\quranayah[38][45]
\end{Arabic}}
\flushleft{\begin{hindi}
हमारे बन्दों, इबराहीम और इसहाक़ और याक़ूब को भी याद करो, जो हाथों (शक्ति) और निगाहोंवाले (ज्ञान-चक्षुवाले) थे
\end{hindi}}
\flushright{\begin{Arabic}
\quranayah[38][46]
\end{Arabic}}
\flushleft{\begin{hindi}
निस्संदेह हमने उन्हें एक विशिष्ट बात के लिए चुन लिया था और वह वास्तविक घर (आख़िरत) की याद थी
\end{hindi}}
\flushright{\begin{Arabic}
\quranayah[38][47]
\end{Arabic}}
\flushleft{\begin{hindi}
और निश्चय ही वे हमारे यहाँ चुने हुए नेक लोगों में से है
\end{hindi}}
\flushright{\begin{Arabic}
\quranayah[38][48]
\end{Arabic}}
\flushleft{\begin{hindi}
इसमाईल और अल-यसअ और ज़ुलकिफ़्ल को भी याद करो। इनमें से प्रत्येक ही अच्छा रहा है
\end{hindi}}
\flushright{\begin{Arabic}
\quranayah[38][49]
\end{Arabic}}
\flushleft{\begin{hindi}
यह एक अनुस्मृति है। और निश्चय ही डर रखनेवालों के लिए अच्छा ठिकाना है
\end{hindi}}
\flushright{\begin{Arabic}
\quranayah[38][50]
\end{Arabic}}
\flushleft{\begin{hindi}
सदैव रहने के बाग़ है, जिनके द्वार उनके लिए खुले होंगे
\end{hindi}}
\flushright{\begin{Arabic}
\quranayah[38][51]
\end{Arabic}}
\flushleft{\begin{hindi}
उनमें वे तकिया लगाए हुए होंगे। वहाँ वे बहुत-से मेवे और पेय मँगवाते होंगे
\end{hindi}}
\flushright{\begin{Arabic}
\quranayah[38][52]
\end{Arabic}}
\flushleft{\begin{hindi}
और उनके पास निगाहें बचाए रखनेवाली स्त्रियाँ होंगी, जो समान अवस्था की होंगी
\end{hindi}}
\flushright{\begin{Arabic}
\quranayah[38][53]
\end{Arabic}}
\flushleft{\begin{hindi}
यह है वह चीज़, जिसका हिसाब के दिन के लिए तुमसे वादा किया जाता है
\end{hindi}}
\flushright{\begin{Arabic}
\quranayah[38][54]
\end{Arabic}}
\flushleft{\begin{hindi}
यह हमारा दिया है, जो कभी समाप्त न होगा
\end{hindi}}
\flushright{\begin{Arabic}
\quranayah[38][55]
\end{Arabic}}
\flushleft{\begin{hindi}
एक और यह है, किन्तु सरकशों के लिए बहुत बुरा ठिकाना है;
\end{hindi}}
\flushright{\begin{Arabic}
\quranayah[38][56]
\end{Arabic}}
\flushleft{\begin{hindi}
जहन्नम, जिसमें वे प्रवेश करेंगे। तो वह बहुत ही बुरा विश्राम-स्थल है!
\end{hindi}}
\flushright{\begin{Arabic}
\quranayah[38][57]
\end{Arabic}}
\flushleft{\begin{hindi}
यह है, अब उन्हें इसे चखना है - खौलता हुआ पानी और रक्तयुक्त पीप
\end{hindi}}
\flushright{\begin{Arabic}
\quranayah[38][58]
\end{Arabic}}
\flushleft{\begin{hindi}
और इसी प्रकार की दूसरी और भी चीज़ें
\end{hindi}}
\flushright{\begin{Arabic}
\quranayah[38][59]
\end{Arabic}}
\flushleft{\begin{hindi}
"यह एक भीड़ है जो तुम्हारे साथ घुसी चली आ रही है। कोई आवभगत उनके लिए नहीं। वे तो आग में पड़नेवाले है।"
\end{hindi}}
\flushright{\begin{Arabic}
\quranayah[38][60]
\end{Arabic}}
\flushleft{\begin{hindi}
वे कहेंगे, "नहीं, तुम नहीं। तुम्हारे लिए कोई आवभगत नहीं। तुम्ही यह हमारे आगे लाए हो। तो बहुत ही बुरी है यह ठहरने की जगह!"
\end{hindi}}
\flushright{\begin{Arabic}
\quranayah[38][61]
\end{Arabic}}
\flushleft{\begin{hindi}
वे कहेंगे, "ऐ हमारे रब! जो हमारे आगे यह (मुसीबत) लाया उसे आग में दोहरी यातना दे!"
\end{hindi}}
\flushright{\begin{Arabic}
\quranayah[38][62]
\end{Arabic}}
\flushleft{\begin{hindi}
और वे कहेंगे, "क्या बात है कि हम उन लोगों को नहीं देखते जिनकी गणना हम बुरों में करते थे?
\end{hindi}}
\flushright{\begin{Arabic}
\quranayah[38][63]
\end{Arabic}}
\flushleft{\begin{hindi}
क्या हमने यूँ ही उनका मज़ाक बनाया था, यह उनसे निगाहें चूक गई हैं?"
\end{hindi}}
\flushright{\begin{Arabic}
\quranayah[38][64]
\end{Arabic}}
\flushleft{\begin{hindi}
निस्संदेह आग में पड़नेवालों का यह आपस का झगड़ा तो अवश्य होना है
\end{hindi}}
\flushright{\begin{Arabic}
\quranayah[38][65]
\end{Arabic}}
\flushleft{\begin{hindi}
कह दो, "मैं तो बस एक सचेत करनेवाला हूँ। कोई पूज्य-प्रभु नहीं सिवाय अल्लाह के, जो अकेला है, सबपर क़ाबू रखनेवाला;
\end{hindi}}
\flushright{\begin{Arabic}
\quranayah[38][66]
\end{Arabic}}
\flushleft{\begin{hindi}
आकाशों और धरती का रब है, और जो कुछ इन दोनों के बीच है उसका भी, अत्यन्त प्रभुत्वशाली, बड़ा क्षमाशील।"
\end{hindi}}
\flushright{\begin{Arabic}
\quranayah[38][67]
\end{Arabic}}
\flushleft{\begin{hindi}
कह दो, "वह एक बड़ी ख़बर है, ‘
\end{hindi}}
\flushright{\begin{Arabic}
\quranayah[38][68]
\end{Arabic}}
\flushleft{\begin{hindi}
जिसे तुम ध्यान में नहीं ला रहे हो
\end{hindi}}
\flushright{\begin{Arabic}
\quranayah[38][69]
\end{Arabic}}
\flushleft{\begin{hindi}
मुझे 'मलए आला' (ऊपरी लोक के फ़रिश्तों) का कोई ज्ञान नहीं था, जब वे वाद-विवाद कर रहे थे
\end{hindi}}
\flushright{\begin{Arabic}
\quranayah[38][70]
\end{Arabic}}
\flushleft{\begin{hindi}
मेरी ओर तो बस इसलिए प्रकाशना की जाती है कि मैं खुल्लम-खुल्ला सचेत करनेवाला हूँ।"
\end{hindi}}
\flushright{\begin{Arabic}
\quranayah[38][71]
\end{Arabic}}
\flushleft{\begin{hindi}
याद करो जब तुम्हारे रब ने फ़रिश्तों से कहा कि "मैं मिट्टी से एक मनुष्य पैदा करनेवाला हूँ
\end{hindi}}
\flushright{\begin{Arabic}
\quranayah[38][72]
\end{Arabic}}
\flushleft{\begin{hindi}
तो जब मैं उसको ठीक-ठाक कर दूँ औऱ उसमें अपनी रूह फूँक दूँ, तो तुम उसके आगे सजदे में गिर जाना।"
\end{hindi}}
\flushright{\begin{Arabic}
\quranayah[38][73]
\end{Arabic}}
\flushleft{\begin{hindi}
तो सभी फ़रिश्तों ने सजदा किया, सिवाय इबलीस के।
\end{hindi}}
\flushright{\begin{Arabic}
\quranayah[38][74]
\end{Arabic}}
\flushleft{\begin{hindi}
उसने घमंड किया और इनकार करनेवालों में से हो गया
\end{hindi}}
\flushright{\begin{Arabic}
\quranayah[38][75]
\end{Arabic}}
\flushleft{\begin{hindi}
कहा, "ऐ इबलीस! तूझे किस चीज़ ने उसको सजदा करने से रोका जिसे मैंने अपने दोनों हाथों से बनाया? क्या तूने घमंड किया, या तू कोई ऊँची हस्ती है?"
\end{hindi}}
\flushright{\begin{Arabic}
\quranayah[38][76]
\end{Arabic}}
\flushleft{\begin{hindi}
उसने कहा, "मैं उससे उत्तम हूँ। तूने मुझे आग से पैदा किया और उसे मिट्टी से पैदा किया।"
\end{hindi}}
\flushright{\begin{Arabic}
\quranayah[38][77]
\end{Arabic}}
\flushleft{\begin{hindi}
कहा, "अच्छा, निकल जा यहाँ से, क्योंकि तू धुत्कारा हुआ है
\end{hindi}}
\flushright{\begin{Arabic}
\quranayah[38][78]
\end{Arabic}}
\flushleft{\begin{hindi}
और निश्चय ही बदला दिए जाने के दिन तक तुझपर मेरी लानत है।"
\end{hindi}}
\flushright{\begin{Arabic}
\quranayah[38][79]
\end{Arabic}}
\flushleft{\begin{hindi}
उसने कहा, "ऐ मेरे रब! फिर तू मुझे उस दिन तक के लिए मुहल्लत दे, जबकि लोग (जीवित करके) उठाए जाएँगे।"
\end{hindi}}
\flushright{\begin{Arabic}
\quranayah[38][80]
\end{Arabic}}
\flushleft{\begin{hindi}
कहा, "अच्छा, तुझे निश्चित एवं
\end{hindi}}
\flushright{\begin{Arabic}
\quranayah[38][81]
\end{Arabic}}
\flushleft{\begin{hindi}
ज्ञात समय तक मुहलत है।"
\end{hindi}}
\flushright{\begin{Arabic}
\quranayah[38][82]
\end{Arabic}}
\flushleft{\begin{hindi}
उसने कहा, "तेरे प्रताप की सौगन्ध! मैं अवश्य उन सबको बहकाकर रहूँगा,
\end{hindi}}
\flushright{\begin{Arabic}
\quranayah[38][83]
\end{Arabic}}
\flushleft{\begin{hindi}
सिवाय उनमें से तेरे उन बन्दों के, जो चुने हुए है।"
\end{hindi}}
\flushright{\begin{Arabic}
\quranayah[38][84]
\end{Arabic}}
\flushleft{\begin{hindi}
कहा, "तो यह सत्य है और मैं सत्य ही कहता हूँ
\end{hindi}}
\flushright{\begin{Arabic}
\quranayah[38][85]
\end{Arabic}}
\flushleft{\begin{hindi}
कि मैं जहन्नम को तुझसे और उन सबसे भर दूँगा, जिन्होंने उनमें से तेरा अनुसरण किया होगा।"
\end{hindi}}
\flushright{\begin{Arabic}
\quranayah[38][86]
\end{Arabic}}
\flushleft{\begin{hindi}
कह दो, "मैं इसपर तुमसे कोई पारिश्रमिक नहीं माँगता और न मैं बनानट करनेवालों में से हूँ।"
\end{hindi}}
\flushright{\begin{Arabic}
\quranayah[38][87]
\end{Arabic}}
\flushleft{\begin{hindi}
वह तो एक अनुस्मृति है सारे संसारवालों के लिए
\end{hindi}}
\flushright{\begin{Arabic}
\quranayah[38][88]
\end{Arabic}}
\flushleft{\begin{hindi}
और थोड़ी ही अवधि के पश्चात उसकी दी हुई ख़बर तुम्हे मालूम हो जाएगी
\end{hindi}}
\chapter{Az-Zumar (The Companies)}
\begin{Arabic}
\Huge{\centerline{\basmalah}}\end{Arabic}
\flushright{\begin{Arabic}
\quranayah[39][1]
\end{Arabic}}
\flushleft{\begin{hindi}
इस किताब का अवतरण अल्लाह अत्यन्त प्रभुत्वशाली, तत्वदर्शी की ओर से है
\end{hindi}}
\flushright{\begin{Arabic}
\quranayah[39][2]
\end{Arabic}}
\flushleft{\begin{hindi}
निस्संदेह हमने यह किताब तुम्हारी ओर सत्य के साथ अवतरित की है
\end{hindi}}
\flushright{\begin{Arabic}
\quranayah[39][3]
\end{Arabic}}
\flushleft{\begin{hindi}
जान रखो कि विशुद्ध धर्म अल्लाह ही के लिए है। रहे वे लोग जिन्होंने उससे हटकर दूसरे समर्थक औऱ संरक्षक बना रखे है (कहते है,) "हम तो उनकी बन्दगी इसी लिए करते है कि वे हमें अल्लाह का सामीप्य प्राप्त करा दें।" निश्चय ही अल्लाह उनके बीच उस बात का फ़ैसला कर देगा जिसमें वे विभेद कर रहे है। अल्लाह उसे मार्ग नहीं दिखाता जो झूठा और बड़ा अकृतज्ञ हो
\end{hindi}}
\flushright{\begin{Arabic}
\quranayah[39][4]
\end{Arabic}}
\flushleft{\begin{hindi}
यदि अल्लाह अपनी कोई सन्तान बनाना चाहता तो वह उसमें से, जिन्हें पैदा कर रहा है, चुन लेता। महान और उच्च है वह! वह अल्लाह है अकेला, सब पर क़ाबू रखनेवाला
\end{hindi}}
\flushright{\begin{Arabic}
\quranayah[39][5]
\end{Arabic}}
\flushleft{\begin{hindi}
उसने आकाशों और धरती को सत्य के साथ पैदा किया। रात को दिन पर लपेटता है और दिन को रात पर लपेटता है। और उसने सूर्य और चन्द्रमा को वशीभुत कर रखा है। प्रत्येक एक नियत समय को पूरा करने के लिए चल रहा है। जान रखो, वही प्रभुत्वशाली, बड़ा क्षमाशील है
\end{hindi}}
\flushright{\begin{Arabic}
\quranayah[39][6]
\end{Arabic}}
\flushleft{\begin{hindi}
उसने तुम्हें अकेली जान पैदा किया; फिर उसी से उसका जोड़ा बनाया औऱ तुम्हारे लिए चौपायों में से आठ नर-मादा उतारे। वह तुम्हारी माँओं के पेटों में तीन अँधेरों के भीतर तुम्हें एक सृजनरूप के पश्चात अन्य एक सृजनरूप देता चला जाता है। वही अल्लाह तुम्हारा रब है। बादशाही उसी की है, उसके अतिरिक्त कोई पूज्य-प्रभु नहीं। फिर तुम कहाँ फिरे जाते हो?
\end{hindi}}
\flushright{\begin{Arabic}
\quranayah[39][7]
\end{Arabic}}
\flushleft{\begin{hindi}
यदि तुम इनकार करोगे तो अल्लाह तुमसे निस्पृह है। यद्यपि वह अपने बन्दों के लिए इनकार को पसन्द नहीं करता, किन्तु यदि तुम कृतज्ञता दिखाओगे, तो उसे वह तुम्हारे लिए पसन्द करता है। कोई बोझ न उठाएगा। फिर तुम्हारी वापसी अपने रब ही की ओर है। और वह तुम्हे बता देगा, जो कुछ तुम करते रहे होगे। निश्चय ही वह सीनों तक की बातें जानता है
\end{hindi}}
\flushright{\begin{Arabic}
\quranayah[39][8]
\end{Arabic}}
\flushleft{\begin{hindi}
जब मनुष्य को कोई तकलीफ़ पहुँचती है तो वह अपने रब को उसी की ओर रुजू होकर पुकारने लगता है, फिर जब वह उसपर अपनी अनुकम्पा करता है, तो वह उस चीज़ को भूल जाता है जिसके लिए पहले पुकार रहा था और (दूसरो को) अल्लाह के समकक्ष ठहराने लगता है, ताकि इसके परिणामस्वरूप वह उसकी राह से भटका दे। कह दो, "अपने इनकार का थोड़ा मज़ा ले लो। निस्संदेह तुम आगवालों में से हो।"
\end{hindi}}
\flushright{\begin{Arabic}
\quranayah[39][9]
\end{Arabic}}
\flushleft{\begin{hindi}
(क्या उक्त व्यक्ति अच्छा है) या वह व्यक्ति जो रात की घड़ियों में सजदा करता और खड़ा रहता है, आख़िरत से डरता है और अपने रब की दयालुता की आशा रखता हुआ विनयशीलता के साथ बन्दगी में लगा रहता है? कहो, "क्या वे लोग जो जानते है और वे लोग जो नहीं जानते दोनों समान होंगे? शिक्षा तो बुद्धि और समझवाले ही ग्रहण करते है।"
\end{hindi}}
\flushright{\begin{Arabic}
\quranayah[39][10]
\end{Arabic}}
\flushleft{\begin{hindi}
कह दो कि "ऐ मेरे बन्दो, जो ईमान लाए हो! अपने रब का डर रखो। जिन लोगों ने अच्छा कर दिखाया उनके लिए इस संसार में अच्छाई है, और अल्लाह की धरती विस्तृत है। जमे रहनेवालों को तो उनका बदला बेहिसाब मिलकर रहेगा।"
\end{hindi}}
\flushright{\begin{Arabic}
\quranayah[39][11]
\end{Arabic}}
\flushleft{\begin{hindi}
कह दो, "मुझे तो आदेश दिया गया है कि मैं अल्लाह की बन्दगी करूँ, धर्म (भक्तिभाव एवं निष्ठान) को उसी के लिए विशुद्ध करते हुए
\end{hindi}}
\flushright{\begin{Arabic}
\quranayah[39][12]
\end{Arabic}}
\flushleft{\begin{hindi}
और मुझे आदेश दिया गया है कि सबसे बढ़कर मैं स्वयं आज्ञाकारी बनूँ।"
\end{hindi}}
\flushright{\begin{Arabic}
\quranayah[39][13]
\end{Arabic}}
\flushleft{\begin{hindi}
कहो, "यदि मैं अपने रब की अवज्ञा करूँ तो मुझे एक बड़े दिन की यातना का भय है।"
\end{hindi}}
\flushright{\begin{Arabic}
\quranayah[39][14]
\end{Arabic}}
\flushleft{\begin{hindi}
कहो, "मैं तो अल्लाह ही की बन्दगी करता हूँ, अपने धर्म को उसी के लिए विशुद्ध करते हुए
\end{hindi}}
\flushright{\begin{Arabic}
\quranayah[39][15]
\end{Arabic}}
\flushleft{\begin{hindi}
अब तुम उससे हटकर जिसकी चाहो बन्दगी करो।" कह दो, "वास्तव में घाटे में पड़नेवाले तो वही है, जिन्होंने अपने आपको और अपने लोगों को क़ियामत के दिन घाटे में डाल दिया। जान रखो, यही खुला घाटा है
\end{hindi}}
\flushright{\begin{Arabic}
\quranayah[39][16]
\end{Arabic}}
\flushleft{\begin{hindi}
उनके लिए उनके ऊपर से भी आग की छतरियाँ होंगी और उनके नीचे से भी छतरियाँ होंगी। यही वह चीज़ है, जिससे अल्लाह अपने बन्दों को डराता है, "ऐ मेरे बन्दो! अतः तुम मेरा डर रखो।"
\end{hindi}}
\flushright{\begin{Arabic}
\quranayah[39][17]
\end{Arabic}}
\flushleft{\begin{hindi}
रहे वे लोग जो इससे बचे कि वे ताग़ूत (बढ़े हुए फ़सादी) की बन्दगी करते है और अल्लाह की ओर रुजू हुए, उनके लिए शुभ सूचना है।
\end{hindi}}
\flushright{\begin{Arabic}
\quranayah[39][18]
\end{Arabic}}
\flushleft{\begin{hindi}
अतः मेरे उन बन्दों को शुभ सूचना दे दो जो बात को ध्यान से सुनते है; फिर उस अच्छी से अच्छी बात का अनुपालन करते है। वही हैं, जिन्हें अल्लाह ने मार्ग दिखाया है और वही बुद्धि और समझवाले है
\end{hindi}}
\flushright{\begin{Arabic}
\quranayah[39][19]
\end{Arabic}}
\flushleft{\begin{hindi}
तो क्या वह व्यक्ति जिसपर यातना की बात सत्यापित हो चुकी है (यातना से बच सकता है)? तो क्या तुम छुड़ा लोगे उसको जो आग में है
\end{hindi}}
\flushright{\begin{Arabic}
\quranayah[39][20]
\end{Arabic}}
\flushleft{\begin{hindi}
अलबत्ता जो लोग अपने रब से डरकर रहे उनके लिए ऊपरी मंज़िल पर कक्ष होंगे, जिनके ऊपर भी निर्मित कक्ष होंगे। उनके नीचे नहरें बह रही होगी। यह अल्लाह का वादा है। अल्लाह अपने वादे का उल्लंघन नहीं करता
\end{hindi}}
\flushright{\begin{Arabic}
\quranayah[39][21]
\end{Arabic}}
\flushleft{\begin{hindi}
क्या तुमने नहीं देखा कि अल्लाह ने आकाश से पानी उतारा, फिर धरती में उसके स्रोत प्रवाहित कर दिए; फिर उसने द्वारा खेती निकालता है, जिसके विभिन्न रंग होते है; फिर वह सूखने लगती है; फिर तुम देखते हो कि वह पीली पड़ गई; फिर वह उसे चूर्ण-विचूर्ण कर देता है? निस्संदेह इसमें बुद्धि और समझवालों के लिए बड़ी याददिहानी है
\end{hindi}}
\flushright{\begin{Arabic}
\quranayah[39][22]
\end{Arabic}}
\flushleft{\begin{hindi}
अब क्या वह व्यक्ति जिसका सीना (हृदय) अल्लाह ने इस्लाम के लिए खोल दिया, अतः वह अपने रब की ओर से प्रकाश पर है, (उस व्यक्ति के समान होगा जो कठोर हृदय और अल्लाह की याद से ग़ाफ़िल है)? अतः तबाही है उन लोगों के लिए जिनके दि कठोर हो चुके है, अल्लाह की याद से ख़ाली होकर! वही खुली गुमराही में पड़े हुए है
\end{hindi}}
\flushright{\begin{Arabic}
\quranayah[39][23]
\end{Arabic}}
\flushleft{\begin{hindi}
अल्लाह ने सर्वोत्तम वाणी अवतरित की, एक ऐसी किताब जिसके सभी भाग परस्पर मिलते-जुलते है, जो रुख़ फेर देनेवाली (क्रांतिकारी) है। उससे उन लोगों के रोंगटे खड़े हो जाते है जो अपने रब से डरते है। फिर उनकी खालें (शरीर) और उनके दिल नर्म होकर अल्लाह की याद की ओर झुक जाते है। वह अल्लाह का मार्गदर्शन है, उसके द्वारा वह सीधे मार्ग पर ले आता है, जिसे चाहता है। और जिसको अल्लाह पथभ्रष्ट रहने दे, फिर उसके लिए कोई मार्गदर्शक नहीं
\end{hindi}}
\flushright{\begin{Arabic}
\quranayah[39][24]
\end{Arabic}}
\flushleft{\begin{hindi}
अब क्या जो क़ियामत के दिन अपने चहरें को बुरी यातना (से बचने) की ढाल बनाएगा वह (यातना से सुरक्षित लोगों जैसा होगा)? और ज़ालिमों से कहा जाएगा, "चखों मज़ा उस कमाई का, जो तुम करते रहे थे!"
\end{hindi}}
\flushright{\begin{Arabic}
\quranayah[39][25]
\end{Arabic}}
\flushleft{\begin{hindi}
जो लोग उनसे पहले थे उन्होंने भी झूठलाया। अन्ततः उनपर वहाँ से यातना आ पहुँची, जिसका उन्हें कोई पता न था
\end{hindi}}
\flushright{\begin{Arabic}
\quranayah[39][26]
\end{Arabic}}
\flushleft{\begin{hindi}
फिर अल्लाह ने उन्हें सांसारिक जीवन में भी रुसवाई का मज़ा चखाया और आख़िरत की यातना तो इससे भी बड़ी है। काश! वे जानते
\end{hindi}}
\flushright{\begin{Arabic}
\quranayah[39][27]
\end{Arabic}}
\flushleft{\begin{hindi}
हमने इस क़ुरआन में लोगों के लिए हर प्रकार की मिसालें पेश कर दी हैं, ताकि वे शिक्षा ग्रहण करें
\end{hindi}}
\flushright{\begin{Arabic}
\quranayah[39][28]
\end{Arabic}}
\flushleft{\begin{hindi}
एक अरबी क़ुरआन के रूप में, जिसमें कोई टेढ़ नहीं, ताकि वे धर्मपरायणता अपनाएँ
\end{hindi}}
\flushright{\begin{Arabic}
\quranayah[39][29]
\end{Arabic}}
\flushleft{\begin{hindi}
अल्लाह एक मिसाल पेश करता है कि एक व्यक्ति है, जिसके मालिक होने में कई क्यक्ति साक्षी है, आपस में खींचातानी करनेवाले, और एक क्यक्ति वह है जो पूरा का पूरा एक ही व्यक्ति का है। क्या दोनों का हाल एक जैसा होगा? सारी प्रशंसा अल्लाह ही के लिए है, किन्तु उनमें से अधिकांश लोग नहीं जानते
\end{hindi}}
\flushright{\begin{Arabic}
\quranayah[39][30]
\end{Arabic}}
\flushleft{\begin{hindi}
तुम्हें भी मरना है और उन्हें भी मरना है
\end{hindi}}
\flushright{\begin{Arabic}
\quranayah[39][31]
\end{Arabic}}
\flushleft{\begin{hindi}
फिर निश्चय ही तुम सब क़ियामत के दिन अपने रब के समक्ष झगड़ोगे
\end{hindi}}
\flushright{\begin{Arabic}
\quranayah[39][32]
\end{Arabic}}
\flushleft{\begin{hindi}
फिर उस व्यक्ति से बढ़कर अत्याचारी कौन होगा, जिसने झूठ घड़कर अल्लाह पर थोपा और सत्य को झूठला दिया जब वह उसके पास आया। क्या जहन्नम में इनकार करनेवालों का ठिकाना नहीं हैं?
\end{hindi}}
\flushright{\begin{Arabic}
\quranayah[39][33]
\end{Arabic}}
\flushleft{\begin{hindi}
और जो व्यक्ति सच्चाई लेकर आया और उसने उसकी पुष्टि की, ऐसे ही लोग डर रखते है
\end{hindi}}
\flushright{\begin{Arabic}
\quranayah[39][34]
\end{Arabic}}
\flushleft{\begin{hindi}
उनके लिए उनके रब के पास वह सब कुछ है, जो वे चाहेंगे। यह है उत्तमकारों का बदला
\end{hindi}}
\flushright{\begin{Arabic}
\quranayah[39][35]
\end{Arabic}}
\flushleft{\begin{hindi}
ताकि जो निकृष्टतम कर्म उन्होंने किए अल्लाह उन (के बुरे प्रभाव) को उनसे दूर कर दे। औऱ जो उत्तम कर्म वे करते रहे उसका उन्हें बदला प्रदान करे
\end{hindi}}
\flushright{\begin{Arabic}
\quranayah[39][36]
\end{Arabic}}
\flushleft{\begin{hindi}
क्या अल्लाह अपने बन्दे के लिए काफ़ी नहीं है, यद्यपि वे तुम्हें उनसे डराते है, जो उसके सिवा (उन्होंने अपने सहायक बना रखे) है? अल्लाह जिसे गुमराही में डाल दे उसे मार्ग दिखानेवाला कोई नही
\end{hindi}}
\flushright{\begin{Arabic}
\quranayah[39][37]
\end{Arabic}}
\flushleft{\begin{hindi}
और जिसे अल्लाह मार्ग दिखाए उसे गुमराह करनेवाला भी कोई नहीं। क्या अल्लाह प्रभुत्वशाली, बदला लेनेवाला नहीं है?
\end{hindi}}
\flushright{\begin{Arabic}
\quranayah[39][38]
\end{Arabic}}
\flushleft{\begin{hindi}
यदि तुम उनसे पूछो कि "आकाशों और धरती को किसने पैदा किया?" को वे अवश्य कहेंगे, "अल्लाह ने।" कहो, "तुम्हारा क्या विचार है? यदि अल्लाह मुझे कोई तकलीफ़ पहुँचानी चाहे तो क्या अल्लाह से हटकर जिनको तुम पुकारते हो वे उसकी पहुँचाई हुई तकलीफ़ को दूर कर सकते है? या वह मुझपर कोई दयालुता दर्शानी चाहे तो क्या वे उसकी दयालुता को रोक सकते है?" कह दो, "मेरे लिए अल्लाह काफ़ी है। भरोसा करनेवाले उसी पर भरोसा करते है।"
\end{hindi}}
\flushright{\begin{Arabic}
\quranayah[39][39]
\end{Arabic}}
\flushleft{\begin{hindi}
कह दो, "ऐ मेरी क़ौम के लोगो! तुम अपनी जगह काम करो। मैं (अपनी जगह) काम करता हूँ। तो शीघ्र ही तुम जान लोगे
\end{hindi}}
\flushright{\begin{Arabic}
\quranayah[39][40]
\end{Arabic}}
\flushleft{\begin{hindi}
कि किस पर वह यातना आती है जो उसे रुसवा कर देगी और किसपर अटल यातना उतरती है।"
\end{hindi}}
\flushright{\begin{Arabic}
\quranayah[39][41]
\end{Arabic}}
\flushleft{\begin{hindi}
निश्चय ही हमने लोगों के लिए हक़ के साथ तुमपर किताब अवतरित की है। अतः जिसने सीधा मार्ग ग्रहण किया तो अपने ही लिए, और जो भटका, तो वह भटककर अपने ही को हानि पहुँचाता है। तुम उनके ज़िम्मेदार नहीं हो
\end{hindi}}
\flushright{\begin{Arabic}
\quranayah[39][42]
\end{Arabic}}
\flushleft{\begin{hindi}
अल्लाह ही प्राणों को उनकी मृत्यु के समय ग्रस्त कर लेता है और जिसकी मृत्यु नहीं आई उसे उसकी निद्रा की अवस्था में (ग्रस्त कर लेता है) । फिर जिसकी मृत्यु का फ़ैसला कर दिया है उसे रोक रखता है। और दूसरों को एक नियत समय तक के लिए छोड़ देता है। निश्चय ही इसमें कितनी ही निशानियाँ है सोच-विचार करनेवालों के लिए
\end{hindi}}
\flushright{\begin{Arabic}
\quranayah[39][43]
\end{Arabic}}
\flushleft{\begin{hindi}
(क्या उनके उपास्य प्रभुता में साझीदार है) या उन्होंने अल्लाह से हटकर दूसरों को सिफ़ारिशी बना रखा है? कहो, "क्या यद्यपि वे किसी चीज़ का अधिकार न रखते हों और न कुछ समझते ही हो तब भी?"
\end{hindi}}
\flushright{\begin{Arabic}
\quranayah[39][44]
\end{Arabic}}
\flushleft{\begin{hindi}
कहो, "सिफ़ारिश तो सारी की सारी अल्लाह के अधिकार में है। आकाशों और धरती की बादशाही उसी की है। फिर उसी की ओर तुम लौटाए जाओगे।"
\end{hindi}}
\flushright{\begin{Arabic}
\quranayah[39][45]
\end{Arabic}}
\flushleft{\begin{hindi}
अकेले अल्लाह का ज़िक्र किया जाता है तो जो लोग आख़िरत पर ईमान नहीं रखते उनके दिल भिंचने लगते है, किन्तु जब उसके सिवा दूसरों का ज़िक्र होता है तो क्या देखते है कि वे खुशी से खिले जा रहे है
\end{hindi}}
\flushright{\begin{Arabic}
\quranayah[39][46]
\end{Arabic}}
\flushleft{\begin{hindi}
कहो, "ऐ अल्लाह, आकाशो और धरती को पैदा करनेवाले; परोक्ष और प्रत्यक्ष के जाननेवाले! तू ही अपने बन्दों के बीच उस चीज़ का फ़ैसला करेगा, जिसमें वे विभेद कर रहे है।"
\end{hindi}}
\flushright{\begin{Arabic}
\quranayah[39][47]
\end{Arabic}}
\flushleft{\begin{hindi}
जिन लोगों ने ज़ुल्म किया यदि उनके पास वह सब कुछ हो जो धरती में है और उसके साथ उतना ही और भी, तो वे क़ियामत के दिन बुरी यातना से बचने के लिए वह सब फ़िदया (प्राण-मुक्ति के बदले) में दे डाले। बात यह है कि अल्लाह की ओर से उनके सामने वह कुछ आ जाएगा जिसका वे गुमान तक न करते थे
\end{hindi}}
\flushright{\begin{Arabic}
\quranayah[39][48]
\end{Arabic}}
\flushleft{\begin{hindi}
और जो कुछ उन्होंने कमाया उसकी बुराइयाँ उनपर प्रकट हो जाएँगी। और वही चीज़ उन्हें घेर लेगी जिसकी वे हँसी उड़ाया करते थे
\end{hindi}}
\flushright{\begin{Arabic}
\quranayah[39][49]
\end{Arabic}}
\flushleft{\begin{hindi}
अतः जब मनुष्य को कोई तकलीफ़ पहुँचती है तो वह हमें पुकारने लगता है, फिर जब हमारी ओर से उसपर कोई अनुकम्पा होती है तो कहता है, "यह तो मुझे ज्ञान के कारण प्राप्त हुआ।" नहीं, बल्कि यह तो एक परीक्षा है, किन्तु उनमें से अधिकतर जानते नहीं
\end{hindi}}
\flushright{\begin{Arabic}
\quranayah[39][50]
\end{Arabic}}
\flushleft{\begin{hindi}
यही बात वे लोग भी कह चुके है, जो उनसे पहले गुज़रे है। किन्तु जो कुछ कमाई वे करते है, वह उनके कुछ काम न आई
\end{hindi}}
\flushright{\begin{Arabic}
\quranayah[39][51]
\end{Arabic}}
\flushleft{\begin{hindi}
फिर जो कुछ उन्होंने कमाया, उसकी बुराइयाँ उनपर आ पड़ी और इनमें से भी जिन लोगों ने ज़ुल्म किया, उनपर भी जो कुछ उन्होंने कमाया उसकी बुराइयाँ जल्द ही आ पड़ेगी। और वे काबू से बाहर निकलनेवाले नहीं
\end{hindi}}
\flushright{\begin{Arabic}
\quranayah[39][52]
\end{Arabic}}
\flushleft{\begin{hindi}
क्या उन्हें मालूम नहीं कि अल्लाह जिसके लिए चाहता है रोज़ी कुशादा कर देता है और जिसके लिए चाहता है नपी-तुली कर देता है? निस्संदेह इसमें उन लोगों के लिए बड़ी निशानियाँ है जो ईमान लाएँ
\end{hindi}}
\flushright{\begin{Arabic}
\quranayah[39][53]
\end{Arabic}}
\flushleft{\begin{hindi}
कह दो, "ऐ मेरे बन्दो, जिन्होंने अपने आप पर ज्यादती की है, अल्लाह की दयालुता से निराश न हो। निस्संदेह अल्लाह सारे ही गुनाहों का क्षमा कर देता है। निश्चय ही वह बड़ा क्षमाशील, अत्यन्त दयावान है
\end{hindi}}
\flushright{\begin{Arabic}
\quranayah[39][54]
\end{Arabic}}
\flushleft{\begin{hindi}
रुजू हो अपने रब की ओर और उसके आज्ञाकारी बन जाओ, इससे पहले कि तुमपर यातना आ जाए। फिर तुम्हारी सहायता न की जाएगी
\end{hindi}}
\flushright{\begin{Arabic}
\quranayah[39][55]
\end{Arabic}}
\flushleft{\begin{hindi}
और अनुसर्ण करो उस सर्वोत्तम चीज़ का जो तुम्हारे रब की ओर से अवतरित हुई है, इससे पहले कि तुम पर अचानक यातना आ जाए और तुम्हें पता भी न हो।"
\end{hindi}}
\flushright{\begin{Arabic}
\quranayah[39][56]
\end{Arabic}}
\flushleft{\begin{hindi}
कहीं ऐसा न हो कि कोई व्यक्ति कहने लगे, "हाय, अफ़सोस उसपर! जो कोताही अल्लाह के हक़ में मैंने की। और मैं तो परिहास करनेवालों मं ही सम्मिलित रहा।"
\end{hindi}}
\flushright{\begin{Arabic}
\quranayah[39][57]
\end{Arabic}}
\flushleft{\begin{hindi}
या, कहने लगे कि "यदि अल्लाह मुझे मार्ग दिखाता तो अवश्य ही मैं डर रखनेवालों में से होता।"
\end{hindi}}
\flushright{\begin{Arabic}
\quranayah[39][58]
\end{Arabic}}
\flushleft{\begin{hindi}
या, जब वह यातना देखे तो कहने लगे, "काश! मुझे एक बार फिर लौटकर जाना हो, तो मैं उत्तमकारों में सम्मिलित हो जाऊँ।"
\end{hindi}}
\flushright{\begin{Arabic}
\quranayah[39][59]
\end{Arabic}}
\flushleft{\begin{hindi}
"क्यों नहीं, मेरी आयतें तेरे पास आ चुकी थीं, किन्तु तूने उनको झूठलाया और घमंड किया और इनकार करनेवालों में सम्मिलित रहा
\end{hindi}}
\flushright{\begin{Arabic}
\quranayah[39][60]
\end{Arabic}}
\flushleft{\begin{hindi}
और क़ियामत के दिन तुम उन लोगों को देखोगे जिन्होंने अल्लाह पर झूठ घड़कर थोपा है कि उनके चेहरे स्याह है। क्या अहंकारियों का ठिकाना जहन्नम में नहीं हैं?"
\end{hindi}}
\flushright{\begin{Arabic}
\quranayah[39][61]
\end{Arabic}}
\flushleft{\begin{hindi}
इसके विपरीत अल्लाह उन लोगों को जिन्होंने डर रखा उन्हें उनकी सफलता के साथ मुक्ति प्रदान करेगा। न तो उन्हें कोई अनिष्ट् छू सकेगा और न वे शोकाकुल होंगे
\end{hindi}}
\flushright{\begin{Arabic}
\quranayah[39][62]
\end{Arabic}}
\flushleft{\begin{hindi}
अल्लाह हर चीज़ का स्रष्टा है और वही हर चीज़ का ज़िम्मा लेता है
\end{hindi}}
\flushright{\begin{Arabic}
\quranayah[39][63]
\end{Arabic}}
\flushleft{\begin{hindi}
उसी के पास आकाशों और धरती की कुँजियाँ है। और जिन लोगों ने हमारी आयतों का इनकार किया, वही है जो घाटे में है
\end{hindi}}
\flushright{\begin{Arabic}
\quranayah[39][64]
\end{Arabic}}
\flushleft{\begin{hindi}
कहो, "क्या फिर भी तुम मुझसे कहते हो कि मैं अल्लाह के सिवा किसी और की बन्दगी करूँ, ऐ अज्ञानियों?"
\end{hindi}}
\flushright{\begin{Arabic}
\quranayah[39][65]
\end{Arabic}}
\flushleft{\begin{hindi}
तुम्हारी ओर और जो तुमसे पहले गुज़र चुके हैं उनकी ओर भी वह्यस की जा चुकी है कि "यदि तुमने शिर्क किया तो तुम्हारा किया-धरा अनिवार्यतः अकारथ जाएगा और तुम अवश्य ही घाटे में पड़नेवालों में से हो जाओगे।"
\end{hindi}}
\flushright{\begin{Arabic}
\quranayah[39][66]
\end{Arabic}}
\flushleft{\begin{hindi}
नहीं, बल्कि अल्लाह ही की बन्दगी करो और कृतज्ञता दिखानेवालों में से हो जाओ
\end{hindi}}
\flushright{\begin{Arabic}
\quranayah[39][67]
\end{Arabic}}
\flushleft{\begin{hindi}
उन्होंने अल्लाह की क़द्र न जानी, जैसी क़द्र उसकी जाननी चाहिए थी। हालाँकि क़ियामत के दिन सारी की सारी धरती उसकी मुट्ठी में होगी और आकाश उसके दाएँ हाथ में लिपटे हुए होंगे। महान और उच्च है वह उससे, जो वे साझी ठहराते है
\end{hindi}}
\flushright{\begin{Arabic}
\quranayah[39][68]
\end{Arabic}}
\flushleft{\begin{hindi}
और सूर (नरसिंघा) फूँका जाएगा, तो जो कोई आकाशों और जो कोई धरती में होगा वह अचेत हो जाएगा सिवाय उसके जिसको अल्लाह चाहे। फिर उसे दूबारा फूँका जाएगा, तो क्या देखेगे कि सहसा वे खड़े देख रहे है
\end{hindi}}
\flushright{\begin{Arabic}
\quranayah[39][69]
\end{Arabic}}
\flushleft{\begin{hindi}
और धरती रब के प्रकाश से जगमगा उठेगी, और किताब रखी जाएगी और नबियों और गवाहों को लाया जाएगा और लोगों के बीच हक़ के साथ फ़ैसला कर दिया जाएगा, और उनपर कोई ज़ुल्म न होगा
\end{hindi}}
\flushright{\begin{Arabic}
\quranayah[39][70]
\end{Arabic}}
\flushleft{\begin{hindi}
और प्रत्येक व्यक्ति को उसका किया भरपूर दिया जाएगा। और वह भली-भाँति जानता है, जो कुछ वे करते है
\end{hindi}}
\flushright{\begin{Arabic}
\quranayah[39][71]
\end{Arabic}}
\flushleft{\begin{hindi}
जिन लोगों ने इनकार किया, वे गिरोह के गिरोह जहन्नम की ओर ले जाए जाएँगे, यहाँ तक कि जब वे वहाँ पहुँचेगे तो उसके द्वार खोल दिए जाएँगे और उसके प्रहरी उनसे कहेंगे, "क्या तुम्हारे पास तुम्हीं में से रसूल नहीं आए थे जो तुम्हें तुम्हारे रब की आयतें सुनाते रहे हों और तुम्हें इस दिन की मुलाक़ात से सचेत करते रहे हों?" वे कहेंगे, "क्यों नहीं (वे तो आए थे) ।" किन्तु इनकार करनेवालों पर यातना की बात सत्यापित होकर रही
\end{hindi}}
\flushright{\begin{Arabic}
\quranayah[39][72]
\end{Arabic}}
\flushleft{\begin{hindi}
कहा जाएगा, "जहन्नम के द्वारों में प्रवेश करो। उसमें सदैव रहने के लिए।" तो बहुत ही बुरा ठिकाना है अहंकारियों का!
\end{hindi}}
\flushright{\begin{Arabic}
\quranayah[39][73]
\end{Arabic}}
\flushleft{\begin{hindi}
और जो लोग अपने रब का डर रखते थे, वे गिरोह के गिरोह जन्नत की ओर ले जाएँगे, यहाँ तक कि जब वे वहाँ पहुँचेंगे इस हाल में कि उसके द्वार खुले होंगे। और उसके प्रहरी उनसे कहेंगे, "सलाम हो तुमपर! बहुत अच्छे रहे! अतः इसमें प्रवेश करो सदैव रहने के लिए तो (उनकी ख़ुशियों का क्या हाल होगा!)
\end{hindi}}
\flushright{\begin{Arabic}
\quranayah[39][74]
\end{Arabic}}
\flushleft{\begin{hindi}
और वे कहेंगे, "प्रशंसा अल्लाह के लिए, जिसने हमारे साथ अपना वादा सच कर दिखाया, और हमें इस भूमि का वारिस बनाया कि हम जन्नत में जहाँ चाहें वहाँ रहें-बसे।" अतः क्या ही अच्छा प्रतिदान है कर्म करनेवालों का!-
\end{hindi}}
\flushright{\begin{Arabic}
\quranayah[39][75]
\end{Arabic}}
\flushleft{\begin{hindi}
और तुम फ़रिश्तों को देखोगे कि वे सिंहासन के गिर्द घेरा बाँधे हुए, अपने रब का गुणगान कर रहे है। और लोगों के बीच ठीक-ठीक फ़ैसला कर दिया जाएगा और कहा जाएगा, "सारी प्रशंसा अल्लाह, सारे संसार के रब, के लिए है।"
\end{hindi}}
\chapter{Al-Mu'min (The Believer)}
\begin{Arabic}
\Huge{\centerline{\basmalah}}\end{Arabic}
\flushright{\begin{Arabic}
\quranayah[40][1]
\end{Arabic}}
\flushleft{\begin{hindi}
हा॰ मीम॰
\end{hindi}}
\flushright{\begin{Arabic}
\quranayah[40][2]
\end{Arabic}}
\flushleft{\begin{hindi}
इस किताब का अवतरण प्रभुत्वशाली, सर्वज्ञ अल्लाह की ओर से है,
\end{hindi}}
\flushright{\begin{Arabic}
\quranayah[40][3]
\end{Arabic}}
\flushleft{\begin{hindi}
जो गुनाह क्षमा करनेवाला, तौबा क़बूल करनेवाला, कठोर दंड देनेवाला, शक्तिमान है। उसके अतिरिक्त कोई पूज्य-प्रभु नहीं। अन्ततः उसी की ओर जाना है
\end{hindi}}
\flushright{\begin{Arabic}
\quranayah[40][4]
\end{Arabic}}
\flushleft{\begin{hindi}
अल्लाह की आयतों के बारे में बस वही लोग झगड़ते हैं जिन्होंने इनकार किया, तो नगरों में उसकी चलत-फिरत तुम्हें धोखे में न डाले
\end{hindi}}
\flushright{\begin{Arabic}
\quranayah[40][5]
\end{Arabic}}
\flushleft{\begin{hindi}
उनसे पहले नूह की क़ौम ने और उनके पश्चात दूसरों गिरोहों ने भी झुठलाया और हर समुदाय के लोगों ने अपने रसूलों के बारे में इरादा किया कि उन्हें पकड़ लें और वे सत्य का सहारा लेकर झगडे, ताकि उसके द्वारा सत्य को उखाड़ दें। अन्ततः मैंने उन्हें पकड़ लिया। तौ कैसी रही मेरी सज़ा!
\end{hindi}}
\flushright{\begin{Arabic}
\quranayah[40][6]
\end{Arabic}}
\flushleft{\begin{hindi}
और (जैसे दुनिया में सज़ा मिली) उसी प्रकार तेरे रब की यह बात भी उन लोगों पर सत्यापित हो गई है, जिन्होंने इनकार किया कि वे आग में पड़नेवाले है;
\end{hindi}}
\flushright{\begin{Arabic}
\quranayah[40][7]
\end{Arabic}}
\flushleft{\begin{hindi}
जो सिंहासन को उठाए हुए है और जो उसके चतुर्दिक हैं, अपने रब का गुणगान करते है और उस पर ईमान रखते है और उन लोगों के लिए क्षमा की प्रार्थना करते है जो ईमान लाए कि "ऐ हमारे रब! तू हर चीज़ को व्याप्त है। अतः जिन लोगों ने तौबा की और तेरे मार्ग का अनुसरण किया, उन्हें क्षमा कर दे और भड़कती हुई आग की यातना से बचा लें
\end{hindi}}
\flushright{\begin{Arabic}
\quranayah[40][8]
\end{Arabic}}
\flushleft{\begin{hindi}
ऐ हमारे रब! और उन्हें सदैव रहने के बागों में दाख़िल कर जिनका तूने उनसे वादा किया है और उनके बाप-दादा और उनकी पत्नि यों और उनकी सन्ततियों में से जो योग्य हुए उन्हें भी। निस्संदेह तू प्रभुत्वशाली, अत्यन्त तत्वदर्शी है
\end{hindi}}
\flushright{\begin{Arabic}
\quranayah[40][9]
\end{Arabic}}
\flushleft{\begin{hindi}
और उन्हें अनिष्टों से बचा। जिसे उस दिन तूने अनिष्टों से बचा लिया, तो निश्चय ही उसपर तूने दया की। और वही बड़ी सफलता है।"
\end{hindi}}
\flushright{\begin{Arabic}
\quranayah[40][10]
\end{Arabic}}
\flushleft{\begin{hindi}
निश्चय ही जिन लोगों ने इनकार किया उन्हें पुकारकर कहा जाएगा कि "अपने आपसे जो तुम्हें विद्वेष एवं क्रोध है, तुम्हारे प्रति अल्लाह का क्रोध एवं द्वेष उससे कहीं बढकर है कि जब तुम्हें ईमान की ओर बुलाया जाता था तो तुम इनकार करते थे।"
\end{hindi}}
\flushright{\begin{Arabic}
\quranayah[40][11]
\end{Arabic}}
\flushleft{\begin{hindi}
वे कहेंगे, "ऐ हमारे रब! तूने हमें दो बार मृत रखा और दो बार जीवन प्रदान किया। अब हमने अपने गुनाहों को स्वीकार किया, तो क्या अब (यहाँ से) निकलने का भी कोई मार्ग है?"
\end{hindi}}
\flushright{\begin{Arabic}
\quranayah[40][12]
\end{Arabic}}
\flushleft{\begin{hindi}
वह (बुरा परिणाम) तो इसलिए सामने आएगा कि जब अकेला अल्लाह को पुकारा जाता है तो तुम इनकार करते हो। किन्तु यदि उसके साथ साझी ठहराया जाए तो तुम मान लेते हो। तो अब फ़ैसला तो अल्लाह ही के हाथ में है, जो सर्वोच्च बड़ा महान है। -
\end{hindi}}
\flushright{\begin{Arabic}
\quranayah[40][13]
\end{Arabic}}
\flushleft{\begin{hindi}
वही है जो तुम्हें अपनी निशानियाँ दिखाता है और तुम्हारे लिए आकाश से रोज़ी उतारता है, किन्तु याददिहानी तो बस वही हासिल करता है जो (उसकी ओर) रुजू करे
\end{hindi}}
\flushright{\begin{Arabic}
\quranayah[40][14]
\end{Arabic}}
\flushleft{\begin{hindi}
अतः तुम अल्लाह ही को, धर्म को उसी के लिए विशुद्ध करते हुए, पुकारो, यद्यपि इनकार करनेवालों को अप्रिय ही लगे। -
\end{hindi}}
\flushright{\begin{Arabic}
\quranayah[40][15]
\end{Arabic}}
\flushleft{\begin{hindi}
वह ऊँचे दर्जोवाला, सिंहासनवाला है, अपने बन्दों में से जिसपर चाहता है, अपने हुक्म में से जिसपर चाहता है, अपने हुक्म से रूह उतारता है, ताकि वह मुलाक़ात के दिन से सावधान कर दे
\end{hindi}}
\flushright{\begin{Arabic}
\quranayah[40][16]
\end{Arabic}}
\flushleft{\begin{hindi}
जिस दिन वे खुले रूप में सामने उपस्थित होंगे, उनकी कोई चीज़ अल्लाह से छिपी न रहेगी, "आज किसकी बादशाही है?" "अल्लाह की, जो अकेला सबपर क़ाबू रखनेवाला है।"
\end{hindi}}
\flushright{\begin{Arabic}
\quranayah[40][17]
\end{Arabic}}
\flushleft{\begin{hindi}
आज प्रत्येक व्यक्ति को उसकी कमाई का बदला दिया जाएगा। आज कोई ज़ुल्म न होगा। निश्चय ही अल्लाह हिसाब लेने में बहुत तेज है
\end{hindi}}
\flushright{\begin{Arabic}
\quranayah[40][18]
\end{Arabic}}
\flushleft{\begin{hindi}
(उन्हें अल्लाह की ओर बुलाओ) और उन्हें निकट आ जानेवाले (क़ियामत के) दिन से सावधान कर दो, जबकि उर (हृदय) कंठ को आ लगे होंगे और वे दबा रहे होंगे। ज़ालिमों का न कोई घनिष्ट मित्र होगा और न ऐसा सिफ़ारिशी जिसकी बात मानी जाए
\end{hindi}}
\flushright{\begin{Arabic}
\quranayah[40][19]
\end{Arabic}}
\flushleft{\begin{hindi}
वह निगाहों की चोरी तक को जानता है और उसे भी जो सीने छिपा रहे होते है
\end{hindi}}
\flushright{\begin{Arabic}
\quranayah[40][20]
\end{Arabic}}
\flushleft{\begin{hindi}
अल्लाह ठीक-ठीक फ़ैसला कर देगा। रहे वे लोग जिन्हें वे अल्लाह को छोड़कर पुकारते हैं, वे किसी चीज़ का भी फ़ैसला करनेवाले नहीं। निस्संदेह अल्लाह ही है जो सुनता, देखता है
\end{hindi}}
\flushright{\begin{Arabic}
\quranayah[40][21]
\end{Arabic}}
\flushleft{\begin{hindi}
क्या वे धरती में चले-फिरे नहीं कि देखते कि उन लोगों का कैसा परिणाम हुआ, जो उनसे पहले गुज़र चुके है? वे शक्ति और धरती में अपने चिन्हों की दृष्टि\ से उनसे कहीं बढ़-चढ़कर थे, फिर उनके गुनाहों के कारण अल्लाह ने उन्हें पकड़ लिया। और अल्लाह से उन्हें बचानेवाला कोई न हुआ
\end{hindi}}
\flushright{\begin{Arabic}
\quranayah[40][22]
\end{Arabic}}
\flushleft{\begin{hindi}
वह (बुरा परिणाम) तो इसलिए सामने आया कि उनके पास उनके रसूल स्पष्ट प्रमाण लेकर आते रहे, किन्तु उन्होंने इनकार किया। अन्ततः अल्लाह ने उन्हें पकड़ लिया। निश्चय ही वह बड़ी शक्तिवाला, सज़ा देने में अत्याधिक कठोर है
\end{hindi}}
\flushright{\begin{Arabic}
\quranayah[40][23]
\end{Arabic}}
\flushleft{\begin{hindi}
और हमने मूसा को भी अपनी निशानियों और स्पष्ट प्रमाण के साथ
\end{hindi}}
\flushright{\begin{Arabic}
\quranayah[40][24]
\end{Arabic}}
\flushleft{\begin{hindi}
फ़िरऔन औऱ हामान और क़ारून की ओर भेजा था, किन्तु उन्होंने कहा, "यह तो जादूगर है, बड़ा झूठा!"
\end{hindi}}
\flushright{\begin{Arabic}
\quranayah[40][25]
\end{Arabic}}
\flushleft{\begin{hindi}
फिर जब वह उनके सामने हमारे पास से सत्य लेकर आया तो उन्होंने कहा, "जो लोग ईमान लेकर उसके साथ है, उनके बेटों को मार डालो औऱ उनकी स्त्रियों को जीवित छोड़ दो।" किन्तु इनकार करनेवालों की चाल तो भटकने ही के लिए होती है
\end{hindi}}
\flushright{\begin{Arabic}
\quranayah[40][26]
\end{Arabic}}
\flushleft{\begin{hindi}
फ़िरऔन ने कहा, "मुझे छोड़ो, मैं मूसा को मार डालूँ और उसे चाहिए कि वह अपने रब को (अपनी सहायता के लिए) पुकारे। मुझे डर है कि ऐसा न हो कि वह तुम्हारे धर्म को बदल डाले या यह कि वह देश में बिगाड़ पैदा करे।"
\end{hindi}}
\flushright{\begin{Arabic}
\quranayah[40][27]
\end{Arabic}}
\flushleft{\begin{hindi}
मूसा ने कहा, "मैंने हर अहंकारी के मुक़ाबले में, जो हिसाब के दिन पर ईमान नहीं रखता, अपने रब और तुम्हारे रब की शरण ले ली है।"
\end{hindi}}
\flushright{\begin{Arabic}
\quranayah[40][28]
\end{Arabic}}
\flushleft{\begin{hindi}
फ़िरऔन के लोगों में से एक ईमानवाले व्यक्ति ने, जो अपने ईमान को छिपा रहा था, कहा, "क्या तुम एक ऐसे व्यक्ति को इसलिए मार डालोगे कि वह कहता है कि मेरा रब अल्लाह है और वह तुम्हारे पास तुम्हारे रब की ओर से खुले प्रमाण भी लेकर आया है? यदि वह झूठा है तो उसके झूठ का वबाल उसी पर पड़ेगा। किन्तु यदि वह सच्चा है तो जिस चीज़ की वह तुम्हें धमकी दे रहा है, उसमें से कुछ न कुछ तो तुमपर पड़कर रहेगा। निश्चय ही अल्लाह उसको मार्ग नहीं दिखाता जो मर्यादाहीन, बड़ा झूठा हो
\end{hindi}}
\flushright{\begin{Arabic}
\quranayah[40][29]
\end{Arabic}}
\flushleft{\begin{hindi}
ऐ मेरी क़ौम के लोगो! आज तुम्हारी बादशाही है। धरती में प्रभावी हो। किन्तु अल्लाह की यातना के मुक़ाबले में कौन हमारी सहायता करेगा, यदि वह हम पर आ जाए?" फ़िरऔन ने कहा, "मैं तो तुम्हें बस वही दिखा रहा हूँ जो मैं स्वयं देख रहा हूँ और मैं तुम्हें बस ठीक रास्ता दिखा रहा हूँ, जो बुद्धिसंगत भी है।"
\end{hindi}}
\flushright{\begin{Arabic}
\quranayah[40][30]
\end{Arabic}}
\flushleft{\begin{hindi}
उस व्यक्ति ने, जो ईमान ला चुका था, कहा, "ऐ मेरी क़ौम के लोगो! मुझे भय है कि तुमपर (विनाश का) ऐसा दिन न आ पड़े, जैसा दूसरे विगत समुदायों पर आ पड़ा था।
\end{hindi}}
\flushright{\begin{Arabic}
\quranayah[40][31]
\end{Arabic}}
\flushleft{\begin{hindi}
जैसे नूह की क़ौम और आद और समूद और उनके पश्चात्वर्ती लोगों का हाल हुआ। अल्लाह तो ऐसा नहीं कि बन्दों पर कोई ज़ुल्म करना चाहे
\end{hindi}}
\flushright{\begin{Arabic}
\quranayah[40][32]
\end{Arabic}}
\flushleft{\begin{hindi}
और ऐ मेरी क़ौम के लोगो! मुझे तुम्हारे बारे में चीख़-पुकार के दिन का भय है,
\end{hindi}}
\flushright{\begin{Arabic}
\quranayah[40][33]
\end{Arabic}}
\flushleft{\begin{hindi}
जिस दिन तुम पीठ फेरकर भागोगे, तुम्हें अल्लाह से बचानेवाला कोई न होगा - और जिसे अल्लाह ही भटका दे उसे मार्ग दिखानेवाला कोई नहीं। -
\end{hindi}}
\flushright{\begin{Arabic}
\quranayah[40][34]
\end{Arabic}}
\flushleft{\begin{hindi}
हमने पहले भी तुम्हारे पास यूसुफ़ खुले प्रमाण लेकर आ चुके है, किन्तु जो कुछ वे लेकर तुम्हारे पास आए थे, उसके बारे में तुम बराबर सन्देह में पड़े रहे, यहाँ तक कि जब उनकी मृत्यु हो गई तो तुम कहने लगे, "अल्लाह उनके पश्चात कदापि कोई रसूल न भेजेगा।" इसी प्रकार अल्लाह उसे गुमराही में डाल देता है जो मर्यादाहीन, सन्देहों में पड़नेवाला हो। -
\end{hindi}}
\flushright{\begin{Arabic}
\quranayah[40][35]
\end{Arabic}}
\flushleft{\begin{hindi}
ऐसे लोगो को (गुमराही में डालता है) जो अल्लाह की आयतों में झगड़ते है, बिना इसके कि उनके पास कोई प्रमाण आया हो, अल्लाह की दृष्टि) में और उन लोगों की दृष्टि में जो ईमान लाए यह (बात) अत्यन्त अप्रिय है। इसी प्रकार अल्लाह हर अहंकारी, निर्दय- अत्याचारी के दिल पर मुहर लगा देता है। -
\end{hindi}}
\flushright{\begin{Arabic}
\quranayah[40][36]
\end{Arabic}}
\flushleft{\begin{hindi}
फ़िरऔन ने कहा, "ऐ हामान! मेरे एक उच्च भवन बना, ताकि मैं साधनों तक पहुँच सकूँ,
\end{hindi}}
\flushright{\begin{Arabic}
\quranayah[40][37]
\end{Arabic}}
\flushleft{\begin{hindi}
आकाशों को साधनों (और क्षत्रों) तक। फिर मूसा के पूज्य को झाँककर देखूँ। मैं तो उसे झूठा ही समझता हूँ।" इस प्रकार फ़िरऔन को लिए उसका दुष्कर्म सुहाना बना दिया गया और उसे मार्ग से रोक दिया गया। फ़िरऔन की चाल तो बस तबाही के सिलसिले में रही
\end{hindi}}
\flushright{\begin{Arabic}
\quranayah[40][38]
\end{Arabic}}
\flushleft{\begin{hindi}
उस व्यक्ति ने, जो ईमान लाया था, कहा, "ऐ मेरी क़ौम के लोगो! मेरा अनुसरण करो, मैं तुम्हे भलाई का ठीक रास्ता दिखाऊँगा
\end{hindi}}
\flushright{\begin{Arabic}
\quranayah[40][39]
\end{Arabic}}
\flushleft{\begin{hindi}
ऐ मेरी क़ौम के लोगो! यह सांसारिक जीवन तो बस अस्थायी उपभोग है। निश्चय ही स्थायी रूप से ठहरनेका घर तो आख़िरत ही है
\end{hindi}}
\flushright{\begin{Arabic}
\quranayah[40][40]
\end{Arabic}}
\flushleft{\begin{hindi}
जिस किसी ने बुराई की तो उसे वैसा ही बदला मिलेगा, किन्तु जिस किसी ने अच्छा कर्म किया, चाहे वह पुरुष हो या स्त्री, किन्तु हो वह मोमिन, तो ऐसे लोग जन्नत में प्रवेश करेंगे। वहाँ उन्हें बेहिसाब दिया जाएगा
\end{hindi}}
\flushright{\begin{Arabic}
\quranayah[40][41]
\end{Arabic}}
\flushleft{\begin{hindi}
ऐ मेरी क़ौम के लोगो! यह मेरे साथ क्या मामला है कि मैं तो तुम्हें मुक्ति की ओर बुलाता हूँ और तुम मुझे आग की ओर बुला रहे हो?
\end{hindi}}
\flushright{\begin{Arabic}
\quranayah[40][42]
\end{Arabic}}
\flushleft{\begin{hindi}
तुम मुझे बुला रहे हो कि मैं अल्लाह के साथ कुफ़्र करूँ और उसके साथ उसे साझी ठहराऊँ जिसका मुझे कोई ज्ञान नहीं, जबकि मैं तुम्हें बुला रहा हूँ उसकी ओर जो प्रभुत्वशाली, अत्यन्त क्षमाशील है
\end{hindi}}
\flushright{\begin{Arabic}
\quranayah[40][43]
\end{Arabic}}
\flushleft{\begin{hindi}
निस्संदेह तुम मुझे जिसकी ओर बुलाते हो उसके लिए न संसार में आमंत्रण है और न आख़िरत (परलोक) में और यह की हमें लौटना भी अल्लाह ही की ओर है और यह कि जो मर्यादाही है, वही आग (में पड़नेवाले) वाले है
\end{hindi}}
\flushright{\begin{Arabic}
\quranayah[40][44]
\end{Arabic}}
\flushleft{\begin{hindi}
अतः शीघ्र ही तुम याद करोगे, जो कुछ मैं तुमसे कह रहा हूँ। मैं तो अपना मामला अल्लाह को सौंपता हूँ। निस्संदेह अल्लाह की दृष्टि सब बन्दों पर है
\end{hindi}}
\flushright{\begin{Arabic}
\quranayah[40][45]
\end{Arabic}}
\flushleft{\begin{hindi}
अन्ततः जो चाल वे चल रहे थे, उसकी बुराइयों से अल्लाह ने उसे बचा लिया और फ़िरऔनियों को बुरी यातना ने आ घेरा;
\end{hindi}}
\flushright{\begin{Arabic}
\quranayah[40][46]
\end{Arabic}}
\flushleft{\begin{hindi}
अर्थात आग ने; जिसके सामने वे प्रातःकाल और सायंकाल पेश किए जाते है। और जिन दिन क़ियामत की घड़ी घटित होगी (कहा जाएगा), "फ़िरऔन के लोगों को निकृष्ट तम यातना में प्रविष्टी कराओ!"
\end{hindi}}
\flushright{\begin{Arabic}
\quranayah[40][47]
\end{Arabic}}
\flushleft{\begin{hindi}
और सोचो जबकि वे आग के भीतर एक-दूसरे से झगड़ रहे होंगे, तो कमज़ोर लोग उन लोगों से, जो बड़े बनते थे, कहेंगे, "हम तो तुम्हारे पीछे चलनेवाले थे। अब क्या तुम हमपर से आग का कुछ भाग हटा सकते हो?"
\end{hindi}}
\flushright{\begin{Arabic}
\quranayah[40][48]
\end{Arabic}}
\flushleft{\begin{hindi}
वे लोग, जो बड़े बनते थे, कहेंगे, "हममें से प्रत्येक इसी में पड़ा है। निश्चय ही अल्लाह बन्दों के बीच फ़ैसला कर चुका।"
\end{hindi}}
\flushright{\begin{Arabic}
\quranayah[40][49]
\end{Arabic}}
\flushleft{\begin{hindi}
जो लोग आग में होंगे वे जहन्नम के प्रहरियों से कहेंगे कि "अपने रब को पुकारो कि वह हमपर से एक दिन यातना कुछ हल्की कर दे!"
\end{hindi}}
\flushright{\begin{Arabic}
\quranayah[40][50]
\end{Arabic}}
\flushleft{\begin{hindi}
वे कहेंगे, "क्या तुम्हारे पास तुम्हारे रसूल खुले प्रमाण लेकर नहीं आते रहे?" कहेंगे, "क्यों नहीं!" वे कहेंगे, "फिर तो तुम्ही पुकारो।" किन्तु इनकार करनेवालों की पुकार तो बस भटककर ही रह जाती है
\end{hindi}}
\flushright{\begin{Arabic}
\quranayah[40][51]
\end{Arabic}}
\flushleft{\begin{hindi}
निश्चय ही हम अपने रसूलों की और उन लोगों की जो ईमान लाए अवश्य सहायता करते है, सांसारिक जीवन में भी और उस दिन भी, जबकि गवाह खड़े होंगे
\end{hindi}}
\flushright{\begin{Arabic}
\quranayah[40][52]
\end{Arabic}}
\flushleft{\begin{hindi}
जिस दिन ज़ालिमों को उनका उज्र (सफ़ाई पेश करना) कुछ भी लाभ न पहुँचाएगा, बल्कि उनके लिए तो लानत है और उनके लिए बुरा घर है
\end{hindi}}
\flushright{\begin{Arabic}
\quranayah[40][53]
\end{Arabic}}
\flushleft{\begin{hindi}
मूसा को भी हम मार्ग दिखा चुके है, और इसराईल की सन्तान को हमने किताब का उत्ताराधिकारी बनाया,
\end{hindi}}
\flushright{\begin{Arabic}
\quranayah[40][54]
\end{Arabic}}
\flushleft{\begin{hindi}
जो बुद्धि और समझवालों के लिए मार्गदर्शन और अनुस्मृति थी
\end{hindi}}
\flushright{\begin{Arabic}
\quranayah[40][55]
\end{Arabic}}
\flushleft{\begin{hindi}
अतः धैर्य से काम लो। निश्चय ही अल्लाह का वादा सच्चा है और अपने क़सूर की क्षमा चाहो और संध्या समय और प्रातः की घड़ियों में अपने रब की प्रशंसा की तसबीह करो
\end{hindi}}
\flushright{\begin{Arabic}
\quranayah[40][56]
\end{Arabic}}
\flushleft{\begin{hindi}
जो लोग बिना किसी ऐसे प्रमाण के जो उनके पास आया हो अल्लाह की आयतों में झगड़ते है उनके सीनों में केवल अहंकार है जिसतक वे पहुँचनेवाले नहीं। अतः अल्लाह की शरण लो। निश्चय ही वह सुनता, देखता है
\end{hindi}}
\flushright{\begin{Arabic}
\quranayah[40][57]
\end{Arabic}}
\flushleft{\begin{hindi}
निस्संदेह, आकाशों और धरती को पैदा करना लोगों को पैदा करने की अपेक्षा अधिक बड़ा (कठिन) काम है। किन्तु अधिकतर लोग नहीं जानते
\end{hindi}}
\flushright{\begin{Arabic}
\quranayah[40][58]
\end{Arabic}}
\flushleft{\begin{hindi}
अंधा और आँखोंवाला बराबर नहीं होते, और वे लोग भी परस्पर बराबर नहीं होते जिन्होंने ईमान लाकर अच्छे कर्म किए, और न बुरे कर्म करनेवाले ही परस्पर बराबर हो सकते है। तुम होश से काम थोड़े ही लेते हो!
\end{hindi}}
\flushright{\begin{Arabic}
\quranayah[40][59]
\end{Arabic}}
\flushleft{\begin{hindi}
निश्चय ही क़ियामत की घड़ी आनेवाली है, इसमें कोई सन्देह नहीं। किन्तु अधिकतर लोग मानते नही
\end{hindi}}
\flushright{\begin{Arabic}
\quranayah[40][60]
\end{Arabic}}
\flushleft{\begin{hindi}
तुम्हारे रब ने कहा कि "तुम मुझे पुकारो, मैं तुम्हारी प्रार्थनाएँ स्वीकार करूँगा।" जो लोग मेरी बन्दगी के मामले में घमंड से काम लेते है निश्चय ही वे शीघ्र ही अपमानित होकर जहन्नम में प्रवेश करेंगे
\end{hindi}}
\flushright{\begin{Arabic}
\quranayah[40][61]
\end{Arabic}}
\flushleft{\begin{hindi}
अल्लाह ही है जिसने तुम्हारे लिए रात (अंधकारमय) बनाई, तुम उसमें शान्ति प्राप्त करो औऱ दिन को प्रकाशमान बनाया (ताकि उसमें दौड़-धूप करो) । निस्संदेह अल्लाह लोगों के लिए बड़ा उदार अनुग्रहवाला हैं, किन्तु अधिकतर लोग कृतज्ञता नहीं दिखाते
\end{hindi}}
\flushright{\begin{Arabic}
\quranayah[40][62]
\end{Arabic}}
\flushleft{\begin{hindi}
वह है अल्लाह, तुम्हारा रब, हर चीज़ का पैदा करनेवाला! उसके सिवा कोई पूज्य-प्रभु नहीं। फिर तुम कहाँ उलटे फिरे जा रहे हो?
\end{hindi}}
\flushright{\begin{Arabic}
\quranayah[40][63]
\end{Arabic}}
\flushleft{\begin{hindi}
इसी प्रकार वे भी उलटे फिरे जाते थे जो अल्लाह की निशानियों का इनकार करते थे
\end{hindi}}
\flushright{\begin{Arabic}
\quranayah[40][64]
\end{Arabic}}
\flushleft{\begin{hindi}
अल्लाह ही है जिसने तुम्हारे लिए धरती को ठहरने का स्थान बनाया और आकाश को एक भवन के रूप में बनाया, और तुम्हें रूप दिए तो क्या ही अच्छे रूप दिए, और तुम्हें अच्छी पाक चीज़ों की रोज़ी दी। वह है अल्लाह, तुम्हारा रब। तो बड़ी बरकतवाला है अल्लाह, सारे संसार का रब
\end{hindi}}
\flushright{\begin{Arabic}
\quranayah[40][65]
\end{Arabic}}
\flushleft{\begin{hindi}
वह जीवन्त है। उसके सिवा कोई पूज्य-प्रभु नहीं। अतः उसी को पुकारो, धर्म को उसी के लिए विशुद्ध करके। सारी प्रशंसा अल्लाह ही के लिए है, जो सारे संसार का रब है
\end{hindi}}
\flushright{\begin{Arabic}
\quranayah[40][66]
\end{Arabic}}
\flushleft{\begin{hindi}
कह दो, "मुझे इससे रोक दिया गया है कि मैं उनकी बन्दगी करूँ जिन्हें अल्लाह से हटकर पुकारते हो, जबकि मेरे पास मेरे रब की ओर से खुले प्रमाण आ चुके है। मुझे तो हुक्म हुआ है कि मैं सारे संसार के रब के आगे नतमस्तक हो जाऊँ।" -
\end{hindi}}
\flushright{\begin{Arabic}
\quranayah[40][67]
\end{Arabic}}
\flushleft{\begin{hindi}
वही है जिसने तुम्हें मिट्टी से पैदा, फिर वीर्य से, फिर रक्त के लोथड़े से; फिर वह तुम्हें एक बच्चे के रूप में निकालता है, फिर (तुम्हें बढ़ाता है) ताकि अपनी प्रौढ़ता को प्राप्ति हो, फिर मुहलत देता है कि तुम बुढापे को पहुँचो - यद्यपि तुममें से कोई इससे पहले भी उठा लिया जाता है - और यह इसलिए करता है कि तुम एक नियत अवधि तक पहुँच जाओ और ऐसा इसलिए है कि तुम समझो
\end{hindi}}
\flushright{\begin{Arabic}
\quranayah[40][68]
\end{Arabic}}
\flushleft{\begin{hindi}
वही है जो जीवन और मृत्यु देता है, और जब वह किसी काम का फ़ैसला करता है, तो उसके लिए बस कह देता है कि 'हो जा' तो वह हो जाता है
\end{hindi}}
\flushright{\begin{Arabic}
\quranayah[40][69]
\end{Arabic}}
\flushleft{\begin{hindi}
क्या तुमने उन लोगों को नहीं देखा जो अल्लाह की आयतों के बारे में झगड़ते है, वे कहाँ फिरे जाते हैं?
\end{hindi}}
\flushright{\begin{Arabic}
\quranayah[40][70]
\end{Arabic}}
\flushleft{\begin{hindi}
जिन लोगों ने किताब को झुठलाया और उसे भी जिसके साथ हमने अपने रसूलों को भेजा था। तो शीघ्र ही उन्हें मालूम हो जाएगा
\end{hindi}}
\flushright{\begin{Arabic}
\quranayah[40][71]
\end{Arabic}}
\flushleft{\begin{hindi}
जबकि तौक़ उनकी गरदनों में होंगे और ज़ंजीरें (उनके पैरों में)
\end{hindi}}
\flushright{\begin{Arabic}
\quranayah[40][72]
\end{Arabic}}
\flushleft{\begin{hindi}
वे खौलते हुए पानी में घसीटे जाएँगे, फिर आग में झोंक दिए जाएँगे
\end{hindi}}
\flushright{\begin{Arabic}
\quranayah[40][73]
\end{Arabic}}
\flushleft{\begin{hindi}
फिर उनसे कहा जाएगा, "कहाँ है वे जिन्हें प्रभुत्व में साझी ठहराकर तुम अल्लाह के सिवा पूजते थे?"
\end{hindi}}
\flushright{\begin{Arabic}
\quranayah[40][74]
\end{Arabic}}
\flushleft{\begin{hindi}
वे कहेंगे, "वे हमसे गुम होकर रह गए, बल्कि हम इससे पहले किसी चीज़ को नहीं पुकारते थे।" इसी प्रकार अल्लाह इनकार करनेवालों को भटकता छोड़ देता है
\end{hindi}}
\flushright{\begin{Arabic}
\quranayah[40][75]
\end{Arabic}}
\flushleft{\begin{hindi}
"यह इसलिए कि तुम धरती में नाहक़ मग्न थे और इसलिए कि तुम इतराते रहे हो
\end{hindi}}
\flushright{\begin{Arabic}
\quranayah[40][76]
\end{Arabic}}
\flushleft{\begin{hindi}
प्रवेश करो जहन्नम के द्वारों में, उसमे सदैव रहने के लिए।" अतः बहुत ही बुरा ठिकाना है अहंकारियों का!
\end{hindi}}
\flushright{\begin{Arabic}
\quranayah[40][77]
\end{Arabic}}
\flushleft{\begin{hindi}
अतः धैर्य से काम लो। निश्चय ही अल्लाह का वादा सच्चा है। तो जिस चीज़ की हम उन्हें धमकी दे रहे है उसमें से कुछ यदि हम तुम्हें दिखा दें या हम तुम्हे उठा लें, हर हाल में उन्हें लौटना तो हमारी ही ओर है
\end{hindi}}
\flushright{\begin{Arabic}
\quranayah[40][78]
\end{Arabic}}
\flushleft{\begin{hindi}
हम तुमसे पहले कितने ही रसूल भेज चुके है। उनमें से कुछ तो वे है जिनके वृत्तान्त का उल्लेख हमने तुमसे किया है और उनमें ऐसे भी है जिनके वृत्तान्त का उल्लेख हमने तुमसे नहीं किया। किसी रसूल को भी यह सामर्थ्य प्राप्त न थी कि वह अल्लाह की अनुज्ञा के बिना कोई निशानी ले आए। फिर जब अल्लाह का आदेश आ जाता है तो ठीक-ठीक फ़ैसला कर दिया जाता है। और उस समय झूठवाले घाटे में पड़ जाते है
\end{hindi}}
\flushright{\begin{Arabic}
\quranayah[40][79]
\end{Arabic}}
\flushleft{\begin{hindi}
अल्लाह ही है जिसने तुम्हारे लिए चौपाए बनाए ताकि उनमें से कुछ पर तुम सवारी करो और उनमें से कुछ को तुम खाते भी हो
\end{hindi}}
\flushright{\begin{Arabic}
\quranayah[40][80]
\end{Arabic}}
\flushleft{\begin{hindi}
उनमें तुम्हारे लिए और भी फ़ायदे है - और ताकि उनके द्वारा तुम उस आवश्यकता की पूर्ति कर सको जो तुम्हारे सीनों में हो, और उनपर भी और नौकाओं पर भी सवार होते हो
\end{hindi}}
\flushright{\begin{Arabic}
\quranayah[40][81]
\end{Arabic}}
\flushleft{\begin{hindi}
और वह तुम्हें अपनी निशानियाँ दिखाता है। आख़िर तुम अल्लाह की कौन-सी निशानी को नहीं पहचानते?
\end{hindi}}
\flushright{\begin{Arabic}
\quranayah[40][82]
\end{Arabic}}
\flushleft{\begin{hindi}
फिर क्या वे धरती में चले-फिरे नहीं कि देखते कि उन लोगों का कैसा परिणाम हुआ, जो उनसे पहले गुज़र चुके है। वे उनसे अधिक थे और शक्ति और अपनी छोड़ी हुई निशानियों की दृष्टि से भी बढ़-चढ़कर थे। किन्तु जो कुछ वे कमाते थे, वह उनके कुछ भी काम न आया
\end{hindi}}
\flushright{\begin{Arabic}
\quranayah[40][83]
\end{Arabic}}
\flushleft{\begin{hindi}
फिर जब उनके रसूल उनके पास स्पष्ट प्रमाणों के साथ आए तो जो ज्ञान उनके अपने पास था वे उसी पर मग्न होते रहे और उनको उसी चीज़ ने आ घेरा जिसका वे परिहास करते थे
\end{hindi}}
\flushright{\begin{Arabic}
\quranayah[40][84]
\end{Arabic}}
\flushleft{\begin{hindi}
फिर जब उन्होंने हमारी यातना देखी तो कहने लगे, "हम ईमान लाए अल्लाह पर जो अकेला है और उसका इनकार किया जिसे हम उसका साझी ठहराते थे।"
\end{hindi}}
\flushright{\begin{Arabic}
\quranayah[40][85]
\end{Arabic}}
\flushleft{\begin{hindi}
किन्तु उनका ईमान उनको कुछ भी लाभ नहीं पहुँचा सकता था जबकि उन्होंने हमारी यातना को देख लिया - यही अल्लाह की रीति है, जो उसके बन्दों में पहले से चली आई है - और उस समय इनकार करनेवाले घाटे में पड़कर रहे
\end{hindi}}
\chapter{Ha Mim (Ha Mim)}
\begin{Arabic}
\Huge{\centerline{\basmalah}}\end{Arabic}
\flushright{\begin{Arabic}
\quranayah[41][1]
\end{Arabic}}
\flushleft{\begin{hindi}
हा॰ मीम॰
\end{hindi}}
\flushright{\begin{Arabic}
\quranayah[41][2]
\end{Arabic}}
\flushleft{\begin{hindi}
यह अवतरण है बड़े कृपाशील, अत्यन्त दयावान की ओर से,
\end{hindi}}
\flushright{\begin{Arabic}
\quranayah[41][3]
\end{Arabic}}
\flushleft{\begin{hindi}
एक किताब, जिसकी आयतें खोल-खोलकर बयान हुई है; अरबी क़ुरआन के रूप में, उन लोगों के लिए जो जानना चाहें;
\end{hindi}}
\flushright{\begin{Arabic}
\quranayah[41][4]
\end{Arabic}}
\flushleft{\begin{hindi}
शुभ सूचक एवं सचेतकर्त्ता किन्तु उनमें से अधिकतर कतरा गए तो वे सुनते ही नहीं
\end{hindi}}
\flushright{\begin{Arabic}
\quranayah[41][5]
\end{Arabic}}
\flushleft{\begin{hindi}
और उनका कहना है कि "जिसकी ओर तुम हमें बुलाते हो उसके लिए तो हमारे दिल आवरणों में है। और हमारे कानों में बोझ है। और हमारे और तुम्हारे बीच एक ओट है; अतः तुम अपना काम करो, हम तो अपना काम करते है।"
\end{hindi}}
\flushright{\begin{Arabic}
\quranayah[41][6]
\end{Arabic}}
\flushleft{\begin{hindi}
कह दो, "मैं तो तुम्हीं जैसा मनुष्य हूँ। मेरी ओर प्रकाशना की जाती है कि तुम्हारा पूज्य-प्रभु बस अकेला पूज्य-प्रभु है। अतः तुम सीधे उसी का रुख करो और उसी से क्षमा-याचना करो - साझी ठहरानेवालों के लिए तो बड़ी तबाही है,
\end{hindi}}
\flushright{\begin{Arabic}
\quranayah[41][7]
\end{Arabic}}
\flushleft{\begin{hindi}
जो ज़कात नहीं देते और वही है जो आख़िरत का इनकार करते है। -
\end{hindi}}
\flushright{\begin{Arabic}
\quranayah[41][8]
\end{Arabic}}
\flushleft{\begin{hindi}
रहे वे लोग जो ईमान लाए और उन्होंने अच्छे कर्म किए, उनके लिए ऐसा बदला है जिसका क्रम टूटनेवाला नहीं।"
\end{hindi}}
\flushright{\begin{Arabic}
\quranayah[41][9]
\end{Arabic}}
\flushleft{\begin{hindi}
कहो, "क्या तुम उसका इनकार करते हो, जिसने धरती को दो दिनों (काल) में पैदा किया और तुम उसके समकक्ष ठहराते हो? वह तो सारे संसार का रब है
\end{hindi}}
\flushright{\begin{Arabic}
\quranayah[41][10]
\end{Arabic}}
\flushleft{\begin{hindi}
और उसने उस (धरती) में उसके ऊपर से पहाड़ जमाए और उसमें बरकत रखी और उसमें उसकी ख़ुराकों को ठीक अंदाज़े से रखा। माँग करनेवालों के लिए समान रूप से यह सब चार दिन में हुआ
\end{hindi}}
\flushright{\begin{Arabic}
\quranayah[41][11]
\end{Arabic}}
\flushleft{\begin{hindi}
फिर उसने आकाश की ओर रुख़ किया, जबकि वह मात्र धुआँ था- और उसने उससे और धरती से कहा, 'आओ, स्वेच्छा के साथ या अनिच्छा के साथ।' उन्होंने कहा, 'हम स्वेच्छा के साथ आए।' -
\end{hindi}}
\flushright{\begin{Arabic}
\quranayah[41][12]
\end{Arabic}}
\flushleft{\begin{hindi}
फिर दो दिनों में उनको अर्थात सात आकाशों को बनाकर पूरा किया और प्रत्येक आकाश में उससे सम्बन्धित आदेश की प्रकाशना कर दी औऱ दुनिया के (निकटवर्ती) आकाश को हमने दीपों से सजाया (रात के यात्रियों के दिशा-निर्देश आदि के लिए) और सुरक्षित करने के उद्देश्य से। यह अत्.न्त प्रभुत्वशाली, सर्वज्ञ का ठहराया हुआ है।"
\end{hindi}}
\flushright{\begin{Arabic}
\quranayah[41][13]
\end{Arabic}}
\flushleft{\begin{hindi}
अब यदि वे लोग ध्यान में न लाएँ तो कह दो, "मैं तुम्हें उसी तरह के कड़का (वज्रवात) से डराता हूँ, जैसा कड़का आद और समूद पर हुआ था।"
\end{hindi}}
\flushright{\begin{Arabic}
\quranayah[41][14]
\end{Arabic}}
\flushleft{\begin{hindi}
जब उनके पास रसूल उनके आगे और उनके पीछे से आए कि "अल्लाह के सिवा किसी की बन्दगी न करो।" तो उन्होंने कहा, "यदि हमारा रब चाहता तो फ़रिश्तों को उतार देता। अतः जिस चीज़ के साथ तुम्हें भेजा गया है, हम उसे नहीं मानते।"
\end{hindi}}
\flushright{\begin{Arabic}
\quranayah[41][15]
\end{Arabic}}
\flushleft{\begin{hindi}
रहे आद, तो उन्होंने नाहक़ धरती में घमंड किया और कहा, "कौन हमसे शक्ति में बढ़कर है?" क्या उन्होंने नहीं देखा कि अल्लाह, जिसने उन्हें पैदा किया, वह उनसे शक्ति में बढ़कर है? वे तो हमारी आयतों का इनकार ही करते रहे
\end{hindi}}
\flushright{\begin{Arabic}
\quranayah[41][16]
\end{Arabic}}
\flushleft{\begin{hindi}
अन्ततः हमने कुछ अशुभ दिनों में उनपर एक शीत-झंझावात चलाई, ताकि हम उन्हें सांसारिक जीवन में अपमान और रुसवाई की यातना का मज़ा चखा दें। और आख़िरत की यातना तो इससे कहीं बढ़कर रुसवा करनेवाली है। और उनको कोई सहायता भी न मिल सकेगी
\end{hindi}}
\flushright{\begin{Arabic}
\quranayah[41][17]
\end{Arabic}}
\flushleft{\begin{hindi}
और रहे समूद, तो हमने उनके सामने सीधा मार्ग दिखाया, किन्तु मार्गदर्शन के मुक़ाबले में उन्होंने अन्धा रहना ही पसन्द किया। परिणामतः जो कुछ वे कमाई करते रहे थे उसके बदले में अपमानजनक यातना के कड़के ने उन्हें आ पकड़ा
\end{hindi}}
\flushright{\begin{Arabic}
\quranayah[41][18]
\end{Arabic}}
\flushleft{\begin{hindi}
और हमने उन लोगों को बचा लिया जो ईमान लाए थे और डर रखते थे
\end{hindi}}
\flushright{\begin{Arabic}
\quranayah[41][19]
\end{Arabic}}
\flushleft{\begin{hindi}
और विचार करो जिस दिन अल्लाह के शत्रु आग की ओर एकत्र करके लाए जाएँगे, फिर उन्हें श्रेणियों में क्रमबद्ध किया जाएगा, यहाँ तक की जब वे उसके पास पहुँच जाएँगे
\end{hindi}}
\flushright{\begin{Arabic}
\quranayah[41][20]
\end{Arabic}}
\flushleft{\begin{hindi}
तो उनके कान और उनकी आँखें और उनकी खालें उनके विरुद्ध उन बातों की गवाही देंगी, जो कुछ वे करते रहे होंगे
\end{hindi}}
\flushright{\begin{Arabic}
\quranayah[41][21]
\end{Arabic}}
\flushleft{\begin{hindi}
वे अपनी खालों से कहेंगे, "तुमने हमारे विरुद्ध क्यों गवाही दी?" वे कहेंगी, "हमें उसी अल्लाह ने वाक्-शक्ति प्रदान की है, जिसने प्रत्येक चीज़ को वाक्-शक्ति प्रदान की।" - उसी ने तुम्हें पहली बार पैदा किया और उसी की ओर तुम्हें लौटना है
\end{hindi}}
\flushright{\begin{Arabic}
\quranayah[41][22]
\end{Arabic}}
\flushleft{\begin{hindi}
तुम इस भय से छिपते न थे कि तुम्हारे कान तुम्हारे विरुद्ध गवाही देंगे, और न इसलिए कि तुम्हारी आँखें गवाही देंगी और न इस कारण से कि तुम्हारी खाले गवाही देंगी, बल्कि तुमने तो यह समझ रखा था कि अल्लाह तुम्हारे बहुत-से कामों को जानता ही नहीं
\end{hindi}}
\flushright{\begin{Arabic}
\quranayah[41][23]
\end{Arabic}}
\flushleft{\begin{hindi}
और तुम्हारे उस गुमान ने तुम्हे बरबाद किया जो तुमने अपने रब के साथ किया; अतः तुम घाटे में पड़कर रहे
\end{hindi}}
\flushright{\begin{Arabic}
\quranayah[41][24]
\end{Arabic}}
\flushleft{\begin{hindi}
अब यदि वे धैर्य दिखाएँ तब भी आग ही उनका ठिकाना है। और यदि वे किसी प्रकार (उसके) क्रोध को दूर करना चाहें, तब भी वे ऐसे नहीं कि वे राज़ी कर सकें
\end{hindi}}
\flushright{\begin{Arabic}
\quranayah[41][25]
\end{Arabic}}
\flushleft{\begin{hindi}
हमने उनके लिए कुछ साथी नियुक्त कर दिए थे। फिर उन्होंने उनके आगे और उनके पीछे जो कुछ था उसे सुहाना बनाकर उन्हें दिखाया। अन्ततः उनपर भी जिन्नों और मनुष्यों के उन गिरोहों के साथ फ़ैसला सत्यापित होकर रहा, जो उनसे पहले गुज़र चुके थे। निश्चय ही वे घाटा उठानेवाले थे
\end{hindi}}
\flushright{\begin{Arabic}
\quranayah[41][26]
\end{Arabic}}
\flushleft{\begin{hindi}
जिन लोगों ने इनकार किया उन्होंने कहा, "इस क़ुरआन को सुनो ही मत और इसके बीच में शोर-गुल मचाओ, ताकि तुम प्रभावी रहो।"
\end{hindi}}
\flushright{\begin{Arabic}
\quranayah[41][27]
\end{Arabic}}
\flushleft{\begin{hindi}
अतः हम अवश्य ही उन लोगों को, जिन्होंने इनकार किया, कठोर यातना का मजा चखाएँगे, और हम अवश्य उन्हें उसका बदला देंगे, जो निकृष्टतम कर्म वे करते रहे है
\end{hindi}}
\flushright{\begin{Arabic}
\quranayah[41][28]
\end{Arabic}}
\flushleft{\begin{hindi}
वह है अल्लाह के शत्रुओं का बदला - आग। उसी में उसका सदा का घर है, उसके बदले में जो वे हमारी आयतों का इनकार करते रहे
\end{hindi}}
\flushright{\begin{Arabic}
\quranayah[41][29]
\end{Arabic}}
\flushleft{\begin{hindi}
और जिन लोगों ने इनकार किया वे कहेंगे, "ऐ हमारे रब! हमें दिखा दे उन जिन्नों और मनुष्यों को, जिन्होंने हमको पथभ्रष़्ट किया कि हम उन्हें अपने पैरों तले डाल दे ताकि वे सबसे नीचे जा पड़े
\end{hindi}}
\flushright{\begin{Arabic}
\quranayah[41][30]
\end{Arabic}}
\flushleft{\begin{hindi}
जिन लोगों ने कहा कि "हमारा रब अल्लाह है।" फिर इस पर दृढ़तापूर्वक जमें रहे, उनपर फ़रिश्ते उतरते है कि "न डरो और न शोकाकुल हो, और उस जन्नत की शुभ सूचना लो जिसका तुमसे वादा किया गया है
\end{hindi}}
\flushright{\begin{Arabic}
\quranayah[41][31]
\end{Arabic}}
\flushleft{\begin{hindi}
हम सांसारिक जीवन में भी तुम्हारे सहचर मित्र है और आख़िरत में भी। और वहाँ तुम्हारे लिए वह सब कुछ है, जिसकी इच्छा तुम्हारे जी को होगी। और वहाँ तुम्हारे लिए वह सब कुछ होगा, जिसका तुम माँग करोगे
\end{hindi}}
\flushright{\begin{Arabic}
\quranayah[41][32]
\end{Arabic}}
\flushleft{\begin{hindi}
आतिथ्य के रूप में क्षमाशील, दयावान सत्ता की ओर से"
\end{hindi}}
\flushright{\begin{Arabic}
\quranayah[41][33]
\end{Arabic}}
\flushleft{\begin{hindi}
और उस व्यक्ति से बात में अच्छा कौन हो सकता है जो अल्लाह की ओर बुलाए और अच्छे कर्म करे और कहे, "निस्संदेह मैं मुस्लिम (आज्ञाकारी) हूँ?"
\end{hindi}}
\flushright{\begin{Arabic}
\quranayah[41][34]
\end{Arabic}}
\flushleft{\begin{hindi}
भलाई और बुराई समान नहीं है। तुम (बुरे आचरण की बुराई को) अच्छे से अच्छे आचरण के द्वारा दूर करो। फिर क्या देखोगे कि वही व्यक्ति तुम्हारे और जिसके बीच वैर पड़ा हुआ था, जैसे वह कोई घनिष्ठ मित्र है
\end{hindi}}
\flushright{\begin{Arabic}
\quranayah[41][35]
\end{Arabic}}
\flushleft{\begin{hindi}
किन्तु यह चीज़ केवल उन लोगों को प्राप्त होती है जो धैर्य से काम लेते है, और यह चीज़ केवल उसको प्राप्त होती है जो बड़ा भाग्यशाली होता है
\end{hindi}}
\flushright{\begin{Arabic}
\quranayah[41][36]
\end{Arabic}}
\flushleft{\begin{hindi}
और यदि शैतान की ओर से कोई उकसाहट तुम्हें चुभे तो अल्लाह की शरण माँग लो। निश्चय ही वह सबकुछ सुनता, जानता है
\end{hindi}}
\flushright{\begin{Arabic}
\quranayah[41][37]
\end{Arabic}}
\flushleft{\begin{hindi}
रात और दिन और सूर्य और चन्द्रमा उसकी निशानियों में से है। तुम न तो सूर्य को सजदा करो और न चन्द्रमा को, बल्कि अल्लाह को सजदा करो जिसने उन्हें पैदा किया, यदि तुम उसी की बन्दगी करनेवाले हो
\end{hindi}}
\flushright{\begin{Arabic}
\quranayah[41][38]
\end{Arabic}}
\flushleft{\begin{hindi}
लेकिन यदि वे घमंड करें (और अल्लाह को याद न करें), तो जो फ़रिश्ते तुम्हारे रब के पास है वे तो रात और दिन उसकी तसबीह करते ही रहते है और वे उकताते नहीं
\end{hindi}}
\flushright{\begin{Arabic}
\quranayah[41][39]
\end{Arabic}}
\flushleft{\begin{hindi}
और यह चीज़ भी उसकी निशानियों में से है कि तुम देखते हो कि धरती दबी पड़ी है; फिर ज्यों ही हमने उसपर पानी बरसाया कि वह फबक उठी और फूल गई। निश्चय ही जिसने उसे जीवित किया, वही मुर्दों को जीवित करनेवाला है। निस्संदेह उसे हर चीज़ की सामर्थ्य प्राप्त है
\end{hindi}}
\flushright{\begin{Arabic}
\quranayah[41][40]
\end{Arabic}}
\flushleft{\begin{hindi}
जो लोग हमारी आयतों में कुटिलता की नीति अपनाते है वे हमसे छिपे हुए नहीं हैं, तो क्या जो व्यक्ति आग में डाला जाए वह अच्छा है या वह जो क़ियामत के दिन निश्चिन्त होकर आएगा? जो चाहो कर लो, तुम जो कुछ करते हो वह तो उसे देख ही रहा है
\end{hindi}}
\flushright{\begin{Arabic}
\quranayah[41][41]
\end{Arabic}}
\flushleft{\begin{hindi}
जिन लोगों ने अनुस्मृति का इनकार किया, जबकि वह उनके पास आई, हालाँकि वह एक प्रभुत्वशाली किताब है, (तो न पूछो कि उनका कितना बुरा परिणाम होगा)
\end{hindi}}
\flushright{\begin{Arabic}
\quranayah[41][42]
\end{Arabic}}
\flushleft{\begin{hindi}
असत्य उस तक न उसके आगे से आ सकता है और न उसके पीछे से; अवतरण है उसकी ओर से जो अत्यन्त तत्वदर्शी, प्रशंसा के योग्य है
\end{hindi}}
\flushright{\begin{Arabic}
\quranayah[41][43]
\end{Arabic}}
\flushleft{\begin{hindi}
तुम्हें बस वही कहा जा रहा है, जो उन रसूलों को कहा जा चुका है, जो तुमसे पहले गुज़र चुके है। निस्संदेह तुम्हारा रब बड़ा क्षमाशील है और दुखद दंड देनेवाला भी
\end{hindi}}
\flushright{\begin{Arabic}
\quranayah[41][44]
\end{Arabic}}
\flushleft{\begin{hindi}
यदि हम उसे ग़ैर अरबी क़ुरआन बनाते तो वे कहते कि "उसकी आयतें क्यों नहीं (हमारी भाषा में) खोलकर बयान की गई? यह क्या कि वाणी तो ग़ैर अरबी है और व्यक्ति अरबी?" कहो, "वह उन लोगों के लिए जो ईमान लाए मार्गदर्शन और आरोग्य है, किन्तु जो लोग ईमान नहीं ला रहे है उनके कानों में बोझ है और वह (क़ुरआन) उनके लिए अन्धापन (सिद्ध हो रहा) है, वे ऐसे है जिनको किसी दूर के स्थान से पुकारा जा रहा हो।"
\end{hindi}}
\flushright{\begin{Arabic}
\quranayah[41][45]
\end{Arabic}}
\flushleft{\begin{hindi}
हमने मूसा को भी किताब प्रदान की थी, फिर उसमें भी विभेद किया गया। यदि तुम्हारे रब की ओर से पहले ही से एक बात निश्चित न हो चुकी होती तो उनके बीत फ़ैसला चुका दिया जाता। हालाँकि वे उसकी ओर से उलझन में डाल देनेवाले सन्देह में पड़े हुए है
\end{hindi}}
\flushright{\begin{Arabic}
\quranayah[41][46]
\end{Arabic}}
\flushleft{\begin{hindi}
जिस किसी ने अच्छा कर्म किया तो अपने ही लिए और जिस किसी ने बुराई की, तो उसका वबाल भी उसी पर पड़ेगा। वास्तव में तुम्हारा रब अपने बन्दों पर तनिक भी ज़ुल्म नहीं करता
\end{hindi}}
\flushright{\begin{Arabic}
\quranayah[41][47]
\end{Arabic}}
\flushleft{\begin{hindi}
उस घड़ी का ज्ञान अल्लाह की ओर फिरता है। जो फल भी अपने कोषों से निकलते है और जो मादा भी गर्भवती होती है और बच्चा जनती है, अनिवार्यतः उसे इन सबका ज्ञान होता है। जिस दिन वह उन्हें पुकारेगा, "कहाँ है मेरे साझीदार?" वे कहेंगे, "हम तेरे समक्ष खुल्लम-ख़ुल्ला कह चुके है कि हममें से कोई भी इसका गवाह नहीं।"
\end{hindi}}
\flushright{\begin{Arabic}
\quranayah[41][48]
\end{Arabic}}
\flushleft{\begin{hindi}
और जिन्हें वे पहले पुकारा करते थे वे उनसे गुम हो जाएँगे। और वे समझ लेंगे कि उनके लिए कोई भी भागने की जगह नहीं है
\end{hindi}}
\flushright{\begin{Arabic}
\quranayah[41][49]
\end{Arabic}}
\flushleft{\begin{hindi}
मनुष्य भलाई माँगने से नहीं उकताता, किन्तु यदि उसे कोई तकलीफ़ छू जाती है तो वह निराश होकर आस छोड़ बैठता है
\end{hindi}}
\flushright{\begin{Arabic}
\quranayah[41][50]
\end{Arabic}}
\flushleft{\begin{hindi}
और यदि उस तकलीफ़ के बाद, जो उसे पहुँची, हम उसे अपनी दयालुता का आस्वादन करा दें तो वह निश्चय ही कहेंगा, "यह तो मेरा हक़ ही है। मैं तो यह नहीं समझता कि वह, क़ियामत की घड़ी, घटित होगी और यदि मैं अपने रब की ओर लौटाया भी गया तो अवश्य ही उसके पास मेरे लिए अच्छा पारितोषिक होगा।" फिर हम उन लोगों को जिन्होंने इनकार किया, अवश्य बताकर रहेंगे, जो कुछ उन्होंने किया होगा। और हम उन्हें अवश्य ही कठोर यातना का मज़ा चखाएँगे
\end{hindi}}
\flushright{\begin{Arabic}
\quranayah[41][51]
\end{Arabic}}
\flushleft{\begin{hindi}
जब हम मनुष्य पर अनुकम्पा करते है तो वह ध्यान में नहीं लाता और अपना पहलू फेर लेता है। किन्तु जब उसे तकलीफ़ छू जाती है, तो वह लम्बी-चौड़ी प्रार्थनाएँ करने लगता है
\end{hindi}}
\flushright{\begin{Arabic}
\quranayah[41][52]
\end{Arabic}}
\flushleft{\begin{hindi}
कह दो, "क्या तुमने विचार किया, यदि वह (क़ुरआन) अल्लाह की ओर सो ही हुआ और तुमने उसका इनकार किया तो उससे बढ़कर भटका हुआ और कौन होगा जो विरोध में बहुत दूर जा पड़ा हो?"
\end{hindi}}
\flushright{\begin{Arabic}
\quranayah[41][53]
\end{Arabic}}
\flushleft{\begin{hindi}
शीघ्र ही हम उन्हें अपनी निशानियाँ वाह्य क्षेत्रों में दिखाएँगे और स्वयं उनके अपने भीतर भी, यहाँ तक कि उनपर स्पष्टा हो जाएगा कि वह (क़ुरआन) सत्य है। क्या तुम्हारा रब इस दृष्टि, से काफ़ी नहीं कि वह हर चीज़ का साक्षी है
\end{hindi}}
\flushright{\begin{Arabic}
\quranayah[41][54]
\end{Arabic}}
\flushleft{\begin{hindi}
जान लो कि वे लोग अपने रब से मिलन के बारे में संदेह में पड़े हुए है। जान लो कि निश्चय ही वह हर चीज़ को अपने घेरे में लिए हुए है
\end{hindi}}
\chapter{Ash-Shura (Counsel)}
\begin{Arabic}
\Huge{\centerline{\basmalah}}\end{Arabic}
\flushright{\begin{Arabic}
\quranayah[42][1]
\end{Arabic}}
\flushleft{\begin{hindi}
हा॰ मीम॰
\end{hindi}}
\flushright{\begin{Arabic}
\quranayah[42][2]
\end{Arabic}}
\flushleft{\begin{hindi}
ऐन॰ सीन॰ क़ाफ़॰
\end{hindi}}
\flushright{\begin{Arabic}
\quranayah[42][3]
\end{Arabic}}
\flushleft{\begin{hindi}
इसी प्रकार अल्लाह प्रुभत्वशाली, तत्वदर्शी तुम्हारी ओर और उन लोगों की ओर प्रकाशना (वह्यप) करता रहा है, जो तुमसे पहले गुज़र चुके है
\end{hindi}}
\flushright{\begin{Arabic}
\quranayah[42][4]
\end{Arabic}}
\flushleft{\begin{hindi}
आकाशों और धरती में जो कुछ है उसी का है और वह सर्वोच्च महिमावान है
\end{hindi}}
\flushright{\begin{Arabic}
\quranayah[42][5]
\end{Arabic}}
\flushleft{\begin{hindi}
लगता है कि आकाश स्वयं अपने ऊपर से फट पड़े। हाल यह है कि फ़रिश्ते अपने रब का गुणगान कर रहे, और उन लोगों के लिए जो धरती में है, क्षमा की प्रार्थना करते रहते है। सुन लो! निश्चय ही अल्लाह ही क्षमाशील, अत्यन्त दयावान है
\end{hindi}}
\flushright{\begin{Arabic}
\quranayah[42][6]
\end{Arabic}}
\flushleft{\begin{hindi}
और जिन लोगों ने उससे हटकर अपने कुछ दूसरे संरक्षक बना रखे हैं, अल्लाह उनपर निगरानी रखे हुए है। तुम उनके कोई ज़िम्मेदार नहीं हो
\end{hindi}}
\flushright{\begin{Arabic}
\quranayah[42][7]
\end{Arabic}}
\flushleft{\begin{hindi}
और (जैसे हम स्पष्ट आयतें उतारते है) उसी प्रकार हमने तुम्हारी ओर एक अरबी क़ुरआन की प्रकाशना की है, ताकि तुम बस्तियों के केन्द्र (मक्का) को और जो लोग उसके चतुर्दिक है उनको सचेत कर दो और सचेत करो इकट्ठा होने के दिन से, जिसमें कोई सन्देह नहीं। एक गिरोह जन्नत में होगा और एक गिरोह भड़कती आग में
\end{hindi}}
\flushright{\begin{Arabic}
\quranayah[42][8]
\end{Arabic}}
\flushleft{\begin{hindi}
यदि अल्लाह चाहता तो उन्हें एक ही समुदाय बना देता, किन्तु वह जिसे चाहता है अपनी दयालुता में दाख़िल करता है। रहे ज़ालिम, तो उनका न तो कोई निकटवर्ती मित्र है और न कोई (दूर का) सहायक
\end{hindi}}
\flushright{\begin{Arabic}
\quranayah[42][9]
\end{Arabic}}
\flushleft{\begin{hindi}
(क्या उन्होंने अल्लाह से हटकर दूसरे सहायक बना लिए है,) या उन्होंने उससे हटकर दूसरे संरक्षक बना रखे है? संरक्षक तो अल्लाह ही है। वही मुर्दों को जीवित करता है और उसे हर चीज़ की सामर्थ्य प्राप्त है
\end{hindi}}
\flushright{\begin{Arabic}
\quranayah[42][10]
\end{Arabic}}
\flushleft{\begin{hindi}
(रसूल ने कहा,) "जिस चीज़ में तुमने विभेद किया है उसका फ़ैसला तो अल्लाह के हवाले है। वही अल्लाह मेरा रब है। उसी पर मैंने भरोसा किया है, और उसी की ओर में रुजू करता हूँ
\end{hindi}}
\flushright{\begin{Arabic}
\quranayah[42][11]
\end{Arabic}}
\flushleft{\begin{hindi}
वह आकाशों और धरती का पैदा करनेवाला है। उसने तुम्हारे लिए तुम्हारी अपनी सहजाति से जोड़े बनाए और चौपायों के जोड़े भी। फैला रहा है वह तुमको अपने में। उसके सदृश कोई चीज़ नहीं। वही सबकुछ सुनता, देखता है
\end{hindi}}
\flushright{\begin{Arabic}
\quranayah[42][12]
\end{Arabic}}
\flushleft{\begin{hindi}
आकाशों और धरती की कुंजियाँ उसी के पास हैं। वह जिसके लिए चाहता है रोज़ी कुशादा कर देता है और जिसके लिए चाहता है नपी-तुली कर देता है। निस्संदेह उसे हर चीज़ का ज्ञान है
\end{hindi}}
\flushright{\begin{Arabic}
\quranayah[42][13]
\end{Arabic}}
\flushleft{\begin{hindi}
उसने तुम्हारे लिए वही धर्म निर्धारित किया जिसकी ताकीद उसने नूह को की थी।" और वह (जीवन्त आदेश) जिसकी प्रकाशना हमने तुम्हारी ओर की है और वह जिसकी ताकीद हमने इबराहीम और मूसा और ईसा को की थी यह है कि "धर्म को क़ायम करो और उसके विषय में अलग-अलग न हो जाओ।" बहुदेववादियों को वह चीज़ बहुत अप्रिय है, जिसकी ओर तुम उन्हें बुलाते हो। अल्लाह जिसे चाहता है अपनी ओर छाँट लेता है और अपनी ओर का मार्ग उसी को दिखाता है जो उसकी ओर रुजू करता है
\end{hindi}}
\flushright{\begin{Arabic}
\quranayah[42][14]
\end{Arabic}}
\flushleft{\begin{hindi}
उन्होंने तो परस्पर एक-दूसरे पर ज़्यादती करने के उद्देश्य से इसके पश्चात विभेद किया कि उनके पास ज्ञान आ चुका था। और यदि तुम्हारे रब की ओर से एक नियत अवधि तक के लिए बात पहले निश्चित न हो चुकी होती तो उनके बीच फ़ैसला चुका दिया गया होता। किन्तु जो लोग उनके पश्चात किताब के वारिस हुए वे उसकी ओर से एक उलझन में डाल देनेवाले संदेह में पड़े हुए है
\end{hindi}}
\flushright{\begin{Arabic}
\quranayah[42][15]
\end{Arabic}}
\flushleft{\begin{hindi}
अतः इसी लिए (उन्हें सत्य की ओर) बुलाओ, और जैसा कि तुम्हें हुक्म दिया गया है स्वयं क़ायम रहो, और उनकी इच्छाओं का पालन न करना और कह दो, "अल्लाह ने जो किताब अवतरित की है, मैं उसपर ईमान लाया। मुझे तो आदेश हुआ है कि मैं तुम्हारे बीच न्याय करूँ। अल्लाह ही हमारा भी रब है और तुम्हारा भी। हमारे लिए हमारे कर्म है और तुम्हारे लिए तुम्हारे कर्म। हममें और तुममें कोई झगड़ा नहीं। अल्लाह हम सबको इकट्ठा करेगा और अन्ततः उसी की ओर जाना है।"
\end{hindi}}
\flushright{\begin{Arabic}
\quranayah[42][16]
\end{Arabic}}
\flushleft{\begin{hindi}
जो लोग अल्लाह के विषय में झगड़ते है, इसके पश्चात कि उसकी पुकार स्वीकार कर ली गई, उनका झगड़ना उनके रब की स्पष्ट में बिलकुल न ठहरनेवाला (असत्य) है। प्रकोप है उनपर और उनके लिए कड़ी यातना है
\end{hindi}}
\flushright{\begin{Arabic}
\quranayah[42][17]
\end{Arabic}}
\flushleft{\begin{hindi}
वह अल्लाह ही है जिसने हक़ के साथ किताब और तुला अवतरित की। और तुम्हें क्या मालूम कदाचित क़ियामत की घड़ी निकट ही आ लगी हो
\end{hindi}}
\flushright{\begin{Arabic}
\quranayah[42][18]
\end{Arabic}}
\flushleft{\begin{hindi}
उसकी जल्दी वे लोग मचाते है जो उसपर ईमान नहीं रखते, किन्तु जो ईमान रखते है वे तो उससे डरते है और जानते है कि वह सत्य है। जान लो, जो लोग उस घड़ी के बारे में सन्देह डालनेवाली बहसें करते है, वे परले दरजे की गुमराही में पड़े हुए है
\end{hindi}}
\flushright{\begin{Arabic}
\quranayah[42][19]
\end{Arabic}}
\flushleft{\begin{hindi}
अल्लाह अपने बन्दों पर अत्यन्त दयालु है। वह जिसे चाहता है रोज़ी देता है। वह शक्तिमान, अत्यन्त प्रभुत्वशाली है
\end{hindi}}
\flushright{\begin{Arabic}
\quranayah[42][20]
\end{Arabic}}
\flushleft{\begin{hindi}
जो कोई आख़िरत की खेती चाहता है, हम उसके लिए उसकी खेती में बढ़ोत्तरी प्रदान करेंगे और जो कोई दुनिया की खेती चाहता है, हम उसमें से उसे कुछ दे देते है, किन्तु आख़िरत में उसका कोई हिस्सा नहीं
\end{hindi}}
\flushright{\begin{Arabic}
\quranayah[42][21]
\end{Arabic}}
\flushleft{\begin{hindi}
(क्या उन्हें समझ नहीं) या उनके कुछ ऐसे (ठहराए हुए) साझीदार है, जिन्होंन उनके लिए कोई ऐसा धर्म निर्धारित कर दिया है जिसकी अनुज्ञा अल्लाह ने नहीं दी? यदि फ़ैसले की बात निश्चित न हो गई होती तो उनके बीच फ़ैसला हो चुका होता। निश्चय ही ज़ालिमों के लिए दुखद यातना है
\end{hindi}}
\flushright{\begin{Arabic}
\quranayah[42][22]
\end{Arabic}}
\flushleft{\begin{hindi}
तुम ज़ालिमों को देखोगे कि उन्होंने जो कुछ कमाया उससे डर रहे होंगे, किन्तु वह तो उनपर पड़कर रहेगा। किन्तु जो लोग ईमान लाए और उन्होंने अच्छे कर्म किए, वे जन्न्तों की वाटिकाओं में होंगे। उनके लिए उनके रब के पास वह सब कुछ है जिसकी वे इच्छा करेंगे। वही तो बड़ा उदार अनुग्रह है
\end{hindi}}
\flushright{\begin{Arabic}
\quranayah[42][23]
\end{Arabic}}
\flushleft{\begin{hindi}
उसी की शुभ सूचना अल्लाह अपने बन्दों को देता है जो ईमान लाए और उन्होंने अच्छे कर्म किए। कहो, "मैं इसका तुमसे कोई पारिश्रमिक नहीं माँगता, बस निकटता का प्रेम-भाव चाहता हूँ, जो कोई नेकी कमाएगा हम उसके लिए उसमें अच्छाई की अभिवृद्धि करेंगे। निश्चय ही अल्लाह अत्यन्त क्षमाशील, गुणग्राहक है।"
\end{hindi}}
\flushright{\begin{Arabic}
\quranayah[42][24]
\end{Arabic}}
\flushleft{\begin{hindi}
(क्या वे ईमान नहीं लाएँगे) या उनका कहना है कि "इस व्यक्ति ने अल्लाह पर मिथ्यारोपण किया है?" यदि अल्लाह चाहे तो तुम्हारे दिल पर मुहर लगा दे (जिस प्रकार उसने इनकार करनेवालों के दिल पर मुहर लगा दी है) । अल्लाह तो असत्य को मिटा रहा है और सत्य को अपने बोलों से सिद्ध कर रहा है। निश्चय ही वह सीनों तक की बात को भी भली-भाँति जानता है
\end{hindi}}
\flushright{\begin{Arabic}
\quranayah[42][25]
\end{Arabic}}
\flushleft{\begin{hindi}
वही है जो अपने बन्दों की तौबा क़बूल करता है और बुराइयों को माफ़ करता है, हालाँकि वह जानता है, जो कुछ तुम करते हो
\end{hindi}}
\flushright{\begin{Arabic}
\quranayah[42][26]
\end{Arabic}}
\flushleft{\begin{hindi}
और वह उन लोगों की प्रार्थनाएँ स्वीकार करता है जो ईमान लाए और उन्होंने अच्छे कर्म किए। और अपने उदार अनुग्रह से उन्हें और अधिक प्रदान करता है। रहे इनकार करनेवाले, तो उनके लिए कड़ा यातना है
\end{hindi}}
\flushright{\begin{Arabic}
\quranayah[42][27]
\end{Arabic}}
\flushleft{\begin{hindi}
यदि अल्लाह अपने बन्दों के लिए रोज़ी कुशादा कर देता तो वे धरती में सरकशी करने लगते। किन्तु वह एक अंदाज़े के साथ जो चाहता है, उतारता है। निस्संदेह वह अपने बन्दों की ख़बर रखनेवाला है। वह उनपर निगाह रखता है
\end{hindi}}
\flushright{\begin{Arabic}
\quranayah[42][28]
\end{Arabic}}
\flushleft{\begin{hindi}
वही है जो इसके पश्चात कि लोग निराश हो चुके होते है, मेंह बरसाता है और अपनी दयालुता को फैला देता है। और वही है संरक्षक मित्र, प्रशंसनीय!
\end{hindi}}
\flushright{\begin{Arabic}
\quranayah[42][29]
\end{Arabic}}
\flushleft{\begin{hindi}
और उसकी निशानियों में से है आकाशों और धरती को पैदा करना, और वे जीवधारी भी जो उसने इन दोनों में फैला रखे है। वह जब चाहे उन्हें इकट्ठा करने की सामर्थ्य भी रखता है
\end{hindi}}
\flushright{\begin{Arabic}
\quranayah[42][30]
\end{Arabic}}
\flushleft{\begin{hindi}
जो मुसीबत तुम्हें पहुँची वह तो तुम्हारे अपने हाथों की कमाई से पहुँची और बहुत कुछ तो वह माफ़ कर देता है
\end{hindi}}
\flushright{\begin{Arabic}
\quranayah[42][31]
\end{Arabic}}
\flushleft{\begin{hindi}
तुम धरती में काबू से निकल जानेवाले नहीं हो, और न अल्लाह से हटकर तुम्हारा कोई संरक्षक मित्र है और न सहायक ही
\end{hindi}}
\flushright{\begin{Arabic}
\quranayah[42][32]
\end{Arabic}}
\flushleft{\begin{hindi}
उसकी निशानियों में से समुद्र में पहाड़ो के सदृश चलते जहाज़ भी है
\end{hindi}}
\flushright{\begin{Arabic}
\quranayah[42][33]
\end{Arabic}}
\flushleft{\begin{hindi}
यदि वह चाहे तो वायु को ठहरा दे, तो वे समुद्र की पीठ पर ठहरे रह जाएँ - निश्चय ही इसमें कितनी ही निशानियाँ है हर उस व्यक्ति के लिए जो अत्यन्त धैर्यवान, कृतज्ञ हो
\end{hindi}}
\flushright{\begin{Arabic}
\quranayah[42][34]
\end{Arabic}}
\flushleft{\begin{hindi}
या उनको उनकी कमाई के कारण विनष्ट कर दे और बहुतो को माफ़ भी कर दे
\end{hindi}}
\flushright{\begin{Arabic}
\quranayah[42][35]
\end{Arabic}}
\flushleft{\begin{hindi}
और परिणामतः वे लोग जान लें जो हमारी आयतों में झगड़ते है कि उनके लिए भागने की कोई जगह नहीं
\end{hindi}}
\flushright{\begin{Arabic}
\quranayah[42][36]
\end{Arabic}}
\flushleft{\begin{hindi}
तुम्हें जो चीज़ भी मिली है वह तो सांसारिक जीवन की अस्थायी सुख-सामग्री है। किन्तु जो कुछ अल्लाह के पास है वह उत्तम है और शेष रहनेवाला भी, वह उन्ही के लिए है जो ईमान लाए और अपने रब पर भरोसा रखते है;
\end{hindi}}
\flushright{\begin{Arabic}
\quranayah[42][37]
\end{Arabic}}
\flushleft{\begin{hindi}
जो बड़े-बड़े गुनाहों और अश्लील कर्मों से बचते है और जब उन्हे (किसी पर) क्रोध आता है तो वे क्षमा कर देते हैं;
\end{hindi}}
\flushright{\begin{Arabic}
\quranayah[42][38]
\end{Arabic}}
\flushleft{\begin{hindi}
और जिन्होंने अपने रब का हुक्म माना और नमाज़ क़ायम की, और उनका मामला उनके पारस्परिक परामर्श से चलता है, और जो कुछ हमने उन्हें दिया है उसमें से ख़र्च करते है;
\end{hindi}}
\flushright{\begin{Arabic}
\quranayah[42][39]
\end{Arabic}}
\flushleft{\begin{hindi}
और जो ऐसे है कि जब उनपर ज़्यादती होती है तो वे प्रतिशोध करते है
\end{hindi}}
\flushright{\begin{Arabic}
\quranayah[42][40]
\end{Arabic}}
\flushleft{\begin{hindi}
बुराई का बदला वैसी ही बुराई है किन्तु जो क्षमा कर दे और सुधार करे तो उसका बदला अल्लाह के ज़िम्मे है। निश्चय ही वह ज़ालिमों को पसन्द नहीं करता
\end{hindi}}
\flushright{\begin{Arabic}
\quranayah[42][41]
\end{Arabic}}
\flushleft{\begin{hindi}
और जो कोई अपने ऊपर ज़ु्ल्म होने के पश्चात बदला ले ले, तो ऐसे लोगों पर कोई इलज़ाम नहीं
\end{hindi}}
\flushright{\begin{Arabic}
\quranayah[42][42]
\end{Arabic}}
\flushleft{\begin{hindi}
इलज़ाम तो केवल उनपर आता है जो लोगों पर ज़ुल्म करते है और धरती में नाहक़ ज़्यादती करते है। ऐसे लोगों के लिए दुखद यातना है
\end{hindi}}
\flushright{\begin{Arabic}
\quranayah[42][43]
\end{Arabic}}
\flushleft{\begin{hindi}
किन्तु जिसने धैर्य से काम लिया और क्षमा कर दिया तो निश्चय ही वह उन कामों में से है जो (सफलता के लिए) आवश्यक ठहरा दिए गए है
\end{hindi}}
\flushright{\begin{Arabic}
\quranayah[42][44]
\end{Arabic}}
\flushleft{\begin{hindi}
जिस व्यक्ति को अल्लाह गुमराही में डाल दे, तो उसके पश्चात उसे सम्भालनेवाला कोई भी नहीं। तुम ज़ालिमों को देखोगे कि जब वे यातना को देख लेंगे तो कह रहे होंगे, "क्या लौटने का भी कोई मार्ग है?"
\end{hindi}}
\flushright{\begin{Arabic}
\quranayah[42][45]
\end{Arabic}}
\flushleft{\begin{hindi}
और तुम उन्हें देखोगे कि वे उस (जहन्नम) पर इस दशा में लाए जा रहे है कि बेबसी और अपमान के कारण दबे हुए है। कनखियों से देख रहे है। जो लोग ईमान लाए, वे उस समय कहेंगे कि "निश्चय ही घाटे में पड़नेवाले वही है जिन्होंने क़ियामत के दिन अपने आपको और अपने लोगों को घाटे में डाल दिया। सावधान! निश्चय ही ज़ालिम स्थिर रहनेवाली यातना में होंगे
\end{hindi}}
\flushright{\begin{Arabic}
\quranayah[42][46]
\end{Arabic}}
\flushleft{\begin{hindi}
और उनके कुछ संरक्षक भी न होंगे, जो सहायता करके उन्हें अल्लाह से बचा सकें। जिसे अल्लाह गुमराही में डाल दे तो उसके लिए फिर कोई मार्ग नहीं।"
\end{hindi}}
\flushright{\begin{Arabic}
\quranayah[42][47]
\end{Arabic}}
\flushleft{\begin{hindi}
अपने रब की बात मान लो इससे पहले कि अल्लाह की ओर से वह दिन आ जाए जो पलटने का नहीं। उस दिन तुम्हारे लिए न कोई शरण-स्थल होगा और न तुम किसी चीज़ को रद्द कर सकोगे
\end{hindi}}
\flushright{\begin{Arabic}
\quranayah[42][48]
\end{Arabic}}
\flushleft{\begin{hindi}
अब यदि वे ध्यान में न लाएँ तो हमने तो तुम्हें उनपर कोई रक्षक बनाकर तो भेजा नहीं है। तुमपर तो केवल (संदेश) पहुँचा देने की ज़िम्मेदारी है। हाल यह है कि जब हम मनुष्य को अपनी ओर से किसी दयालुता का आस्वादन कराते है तो वह उसपर इतराने लगता है, किन्तु ऐसे लोगों के हाथों ने जो कुछ आगे भेजा है उसके कारण यदि उन्हें कोई तकलीफ़ पहुँचती है तो निश्चय ही वही मनुष्य बड़ा कृतघ्न बन जाता है
\end{hindi}}
\flushright{\begin{Arabic}
\quranayah[42][49]
\end{Arabic}}
\flushleft{\begin{hindi}
अल्लाह ही की है आकाशों और धरती की बादशाही। वह जो चाहता है पैदा करता है, जिसे चाहता है लड़कियाँ देता है और जिसे चाहता है लड़के देता है।
\end{hindi}}
\flushright{\begin{Arabic}
\quranayah[42][50]
\end{Arabic}}
\flushleft{\begin{hindi}
या उन्हें लड़के और लड़कियाँ मिला-जुलाकर देता है और जिसे चाहता है निस्संतान रखता है। निश्चय ही वह सर्वज्ञ, सामर्थ्यवान है
\end{hindi}}
\flushright{\begin{Arabic}
\quranayah[42][51]
\end{Arabic}}
\flushleft{\begin{hindi}
किसी मनुष्य की यह शान नहीं कि अल्लाह उससे बात करे, सिवाय इसके कि प्रकाशना के द्वारा या परदे के पीछे से (बात करे) । या यह कि वह एक रसूल (फ़रिश्ता) भेज दे, फिर वह उसकी अनुज्ञा से जो कुछ वह चाहता है प्रकाशना कर दे। निश्चय ही वह सर्वोच्च अत्यन्त तत्वदर्शी है
\end{hindi}}
\flushright{\begin{Arabic}
\quranayah[42][52]
\end{Arabic}}
\flushleft{\begin{hindi}
और इसी प्रकार हमने अपने आदेश से एक रूह (क़ुरआन) की प्रकाशना तुम्हारी ओर की है। तुम नहीं जानते थे कि किताब क्या होती है और न ईमान को (जानते थे), किन्तु हमने इस (प्रकाशना) को एक प्रकाश बनाया, जिसके द्वारा हम अपने बन्दों में से जिसे चाहते है मार्ग दिखाते है। निश्चय ही तुम एक सीधे मार्ग की ओर पथप्रदर्शन कर रहे हो-
\end{hindi}}
\flushright{\begin{Arabic}
\quranayah[42][53]
\end{Arabic}}
\flushleft{\begin{hindi}
उस अल्लाह के मार्ग की ओर जिसका वह सब कुछ है, जो आकाशों में है और जो धरती में है। सुन लो, सारे मामले अन्ततः अल्लाह ही की ओर पलटते हैं
\end{hindi}}
\chapter{Az-Zukhruf (Gold)}
\begin{Arabic}
\Huge{\centerline{\basmalah}}\end{Arabic}
\flushright{\begin{Arabic}
\quranayah[43][1]
\end{Arabic}}
\flushleft{\begin{hindi}
हा॰ मीम॰
\end{hindi}}
\flushright{\begin{Arabic}
\quranayah[43][2]
\end{Arabic}}
\flushleft{\begin{hindi}
गवाह है स्पष्ट किताब
\end{hindi}}
\flushright{\begin{Arabic}
\quranayah[43][3]
\end{Arabic}}
\flushleft{\begin{hindi}
हमने उसे अरबी क़ुरआन बनाया, ताकि तुम समझो
\end{hindi}}
\flushright{\begin{Arabic}
\quranayah[43][4]
\end{Arabic}}
\flushleft{\begin{hindi}
और निश्चय ही वह मूल किताब में अंकित है, हमारे यहाँ बहुच उच्च कोटि की, तत्वदर्शिता से परिपूर्ण है
\end{hindi}}
\flushright{\begin{Arabic}
\quranayah[43][5]
\end{Arabic}}
\flushleft{\begin{hindi}
क्या इसलिए कि तुम मर्यादाहीन लोग हो, हम तुमपर से बिलकुल ही नज़र फेर लेंगे?
\end{hindi}}
\flushright{\begin{Arabic}
\quranayah[43][6]
\end{Arabic}}
\flushleft{\begin{hindi}
हमने पहले के लोगों में कितने ही रसूल भेजे
\end{hindi}}
\flushright{\begin{Arabic}
\quranayah[43][7]
\end{Arabic}}
\flushleft{\begin{hindi}
किन्तु जो भी नबी उनके पास आया, वे उसका परिहास ही करते रहे
\end{hindi}}
\flushright{\begin{Arabic}
\quranayah[43][8]
\end{Arabic}}
\flushleft{\begin{hindi}
अन्ततः हमने उनको पकड़ में लेकर विनष्ट कर दिया जो उनसे कहीं अधिक बलशाली थे। और पहले के लोगों की मिसाल गुज़र-चुकी है
\end{hindi}}
\flushright{\begin{Arabic}
\quranayah[43][9]
\end{Arabic}}
\flushleft{\begin{hindi}
यदि तुम उनसे पूछो कि "आकाशों और धरती को किसने पैदा किया?" तो वे अवश्य कहेंगे, "उन्हें अत्यन्त प्रभुत्वशाली, सर्वज्ञ सत्ता ने पैदा किया।"
\end{hindi}}
\flushright{\begin{Arabic}
\quranayah[43][10]
\end{Arabic}}
\flushleft{\begin{hindi}
जिसने तुम्हारे लिए धरती को गहवारा बनाया औऱ उसमें तुम्हारे लिए मार्ग बना दिए. ताकि तुम्हें मार्गदर्शन प्राप्त हो
\end{hindi}}
\flushright{\begin{Arabic}
\quranayah[43][11]
\end{Arabic}}
\flushleft{\begin{hindi}
और जिसने आकाश से एक अन्दाज़े से पानी उतारा। और हमने उसके द्वारा मृत भूमि को जीवित कर दिया। इसी तरह तुम भी (जीवित करके) निकाले जाओगे
\end{hindi}}
\flushright{\begin{Arabic}
\quranayah[43][12]
\end{Arabic}}
\flushleft{\begin{hindi}
और जिसने विभिन्न प्रकार की चीज़े पैदा कीं, और तुम्हारे लिए वे नौकाएँ और जानवर बनाए जिनपर तुम सवार होते हो
\end{hindi}}
\flushright{\begin{Arabic}
\quranayah[43][13]
\end{Arabic}}
\flushleft{\begin{hindi}
ताकि तुम उनकी पीठों पर जमकर बैठो, फिर याद करो अपने रब की अनुकम्पा को जब तुम उनपर बैठ जाओ और कहो, "कितना महिमावान है वह जिसने इसको हमारे वश में किया, अन्यथा हम तो इसे क़ाबू में कर सकनेवाले न थे
\end{hindi}}
\flushright{\begin{Arabic}
\quranayah[43][14]
\end{Arabic}}
\flushleft{\begin{hindi}
और निश्चय ही हम अपने रब की ओर लौटनेवाले है।"
\end{hindi}}
\flushright{\begin{Arabic}
\quranayah[43][15]
\end{Arabic}}
\flushleft{\begin{hindi}
उन्होंने उसके बन्दों में से कुछ को उसका अंश ठहरा दिया! निश्चय ही मनुष्य खुला कृतघ्न है
\end{hindi}}
\flushright{\begin{Arabic}
\quranayah[43][16]
\end{Arabic}}
\flushleft{\begin{hindi}
(क्या किसी ने अल्लाह को इससे रोक दिया है कि वह अपने लिए बेटे चुनता) या जो कुछ वह पैदा करता है उसमें से उसने स्वयं ही अपने लिए तो बेटियाँ लीं और तुम्हें चुन लिया बेटों के लिए?
\end{hindi}}
\flushright{\begin{Arabic}
\quranayah[43][17]
\end{Arabic}}
\flushleft{\begin{hindi}
और हाल यह है कि जब उनमें से किसी को उसकी मंगल सूचना दी जाती है, जो वह रहमान के लिए बयान करता है, तो उसके मुँह पर कलौंस छा जाती है और वह ग़म के मारे घुटा-घुटा रहने लगता है
\end{hindi}}
\flushright{\begin{Arabic}
\quranayah[43][18]
\end{Arabic}}
\flushleft{\begin{hindi}
और क्या वह जो आभूषणों में पले और वह जो वाद-विवाद और झगड़े में खुल न पाए (ऐसी अबला को अल्लाह की सन्तान घोषित करते हो)?
\end{hindi}}
\flushright{\begin{Arabic}
\quranayah[43][19]
\end{Arabic}}
\flushleft{\begin{hindi}
उन्होंने फ़रिश्तों को, जो रहमान के बन्दे है, स्त्रियाँ ठहरा ली है। क्या वे उनकी संरचना के समय मौजूद थे? उनकी गवाही लिख ली जाएगी और उनसे पूछ होगी
\end{hindi}}
\flushright{\begin{Arabic}
\quranayah[43][20]
\end{Arabic}}
\flushleft{\begin{hindi}
वे कहते है कि "यदि रहमान चाहता तो हम उन्हें न पूजते।" उन्हें इसका कुछ ज्ञान नहीं। वे तो बस अटकल दौड़ाते है
\end{hindi}}
\flushright{\begin{Arabic}
\quranayah[43][21]
\end{Arabic}}
\flushleft{\begin{hindi}
(क्या हमने इससे पहले उनके पास कोई रसूल भेजा है) या हमने इससे पहले उनको कोई किताब दी है तो वे उसे दृढ़तापूर्वक थामें हुए है?
\end{hindi}}
\flushright{\begin{Arabic}
\quranayah[43][22]
\end{Arabic}}
\flushleft{\begin{hindi}
नहीं, बल्कि वे कहते है, "हमने तो अपने बाप-दादा को एक तरीक़े पर पाया और हम उन्हीं के पद-चिन्हों पर हैं, सीधे मार्ग पर चल रहे है।"
\end{hindi}}
\flushright{\begin{Arabic}
\quranayah[43][23]
\end{Arabic}}
\flushleft{\begin{hindi}
इसी प्रकार हमने जिस किसी बस्ती में तुमसे पहले कोई सावधान करनेवाला भेजा तो वहाँ के सम्पन्न लोगों ने बस यही कहा कि "हमने तो अपने बाप-दादा को एक तरीक़े पर पाया और हम उन्हीं के पद-चिन्हों पर है, उनका अनुसरण कर रहे है।"
\end{hindi}}
\flushright{\begin{Arabic}
\quranayah[43][24]
\end{Arabic}}
\flushleft{\begin{hindi}
उसने कहा, "क्या यदि मैं उससे उत्तम मार्गदर्शन लेकर आया हूँ, जिसपर तूने अपने बाप-दादा को पाया है, तब भी (तुम अपने बाप-दादा के पद-चिह्मों का ही अनुसरण करोगं)?" उन्होंने कहा, "तुम्हें जो कुछ देकर भेजा गया है, हम तो उसका इनकार करते है।"
\end{hindi}}
\flushright{\begin{Arabic}
\quranayah[43][25]
\end{Arabic}}
\flushleft{\begin{hindi}
अन्ततः हमने उनसे बदला लिया। तो देख लो कि झुठलानेवालों का कैसा परिणाम हुआ?
\end{hindi}}
\flushright{\begin{Arabic}
\quranayah[43][26]
\end{Arabic}}
\flushleft{\begin{hindi}
याद करो, जबकि इबराहीम ने अपने बाप और अपनी क़ौम से कहा, "तुम जिनको पूजते हो उनसे मेरा कोई सम्बन्ध नहीं,
\end{hindi}}
\flushright{\begin{Arabic}
\quranayah[43][27]
\end{Arabic}}
\flushleft{\begin{hindi}
सिवाय उसके जिसने मुझे पैदा किया। अतः निश्चय ही वह मुझे मार्ग दिखाएगा।"
\end{hindi}}
\flushright{\begin{Arabic}
\quranayah[43][28]
\end{Arabic}}
\flushleft{\begin{hindi}
और यही बात वह अपने पीछे (अपनी सन्तान में) बाक़ी छोड़ गया, ताकि वे रुजू करें
\end{hindi}}
\flushright{\begin{Arabic}
\quranayah[43][29]
\end{Arabic}}
\flushleft{\begin{hindi}
नहीं,बल्कि मैं उन्हें और उनके बाप-दादा को जीवन-सुख प्रदान करता रहा, यहाँ तक कि उनके पास सत्य और खोल-खोलकर बतानेवाला रसूल आ गया
\end{hindi}}
\flushright{\begin{Arabic}
\quranayah[43][30]
\end{Arabic}}
\flushleft{\begin{hindi}
किन्तु जब वह हक़ लेकर उनके पास आया तो वे कहने लगे, "यह तो जादू है। और हम तो इसका इनकार करते है।"
\end{hindi}}
\flushright{\begin{Arabic}
\quranayah[43][31]
\end{Arabic}}
\flushleft{\begin{hindi}
वे कहते है, "यह क़ुरआन इन दो बस्तियों के किसी बड़े आदमी पर क्यों नहीं अवतरित हुआ?"
\end{hindi}}
\flushright{\begin{Arabic}
\quranayah[43][32]
\end{Arabic}}
\flushleft{\begin{hindi}
क्या वे तुम्हारे रब की दयालुता को बाँटते है? सांसारिक जीवन में उनके जीवन-यापन के साधन हमने उनके बीच बाँटे है और हमने उनमें से कुछ लोगों को दूसरे कुछ लोगों से श्रेणियों की दृष्टि से उच्च रखा है, ताकि उनमें से वे एक-दूसरे से काम लें। और तुम्हारे रब की दयालुता उससे कहीं उत्तम है जिसे वे समेट रहे है
\end{hindi}}
\flushright{\begin{Arabic}
\quranayah[43][33]
\end{Arabic}}
\flushleft{\begin{hindi}
यदि इस बात की सम्भावना न होती कि सब लोग एक ही समुदाय (अधर्मी) हो जाएँगे, तो जो लोग रहमान के साथ कुफ़्र करते है उनके लिए हम उनके घरों की छतें चाँदी की कर देते है और सीढ़ियाँ भी जिनपर वे चढ़ते।
\end{hindi}}
\flushright{\begin{Arabic}
\quranayah[43][34]
\end{Arabic}}
\flushleft{\begin{hindi}
और उनके घरों के दरवाज़े भी और वे तख़्त भी जिनपर वे टेक लगाते
\end{hindi}}
\flushright{\begin{Arabic}
\quranayah[43][35]
\end{Arabic}}
\flushleft{\begin{hindi}
और सोने द्वारा सजावट का आयोजन भी कर देते। यह सब तो कुछ भी नहीं, बस सांसारिक जीवन की अस्थायी सुख-सामग्री है। और आख़िरत तुम्हारे रब के यहाँ डर रखनेवालों के लिए है
\end{hindi}}
\flushright{\begin{Arabic}
\quranayah[43][36]
\end{Arabic}}
\flushleft{\begin{hindi}
जो रहमान के स्मरण की ओर से अंधा बना रहा है, हम उसपर एक शैतान नियुक्त कर देते है तो वही उसका साथी होता है
\end{hindi}}
\flushright{\begin{Arabic}
\quranayah[43][37]
\end{Arabic}}
\flushleft{\begin{hindi}
औऱ वे (शैतान) उन्हें मार्ग से रोकते है और वे (इनकार करनेवाले) यह समझते है कि वे मार्ग पर है
\end{hindi}}
\flushright{\begin{Arabic}
\quranayah[43][38]
\end{Arabic}}
\flushleft{\begin{hindi}
यहाँ तक कि जब वह हमारे पास आएगा तो (शैतान से) कहेगा, "ऐ काश, मेरे और तेरे बीच पूरब के दोनों किनारों की दूरी होती! तू तो बहुत ही बुरा साथी निकला!"
\end{hindi}}
\flushright{\begin{Arabic}
\quranayah[43][39]
\end{Arabic}}
\flushleft{\begin{hindi}
और जबकि तुम ज़ालिम ठहरे तो आज यह बात तुम्हें कुछ लाभ न पहुँचा सकेगी कि यातना में तुम एक-दूसरे के साझी हो
\end{hindi}}
\flushright{\begin{Arabic}
\quranayah[43][40]
\end{Arabic}}
\flushleft{\begin{hindi}
क्या तुम बहरों को सुनाओगे या अंधो को और जो खुली गुमराही में पड़ा हुआ हो उसको राह दिखाओगे?
\end{hindi}}
\flushright{\begin{Arabic}
\quranayah[43][41]
\end{Arabic}}
\flushleft{\begin{hindi}
फिर यदि तुम्हें उठा भी लें तब भी हम उनसे बदला लेकर रहेंगे
\end{hindi}}
\flushright{\begin{Arabic}
\quranayah[43][42]
\end{Arabic}}
\flushleft{\begin{hindi}
या हम तुम्हें वह चीज़ दिखा देंगे जिसका हमने वादा किया है। निस्संदेह हमें उनपर पूरी सामर्थ्य प्राप्त है
\end{hindi}}
\flushright{\begin{Arabic}
\quranayah[43][43]
\end{Arabic}}
\flushleft{\begin{hindi}
अतः तुम उस चीज़ को मज़बूती से थामे रहो जिसकी तुम्हारी ओर प्रकाशना की गई। निश्चय ही तु सीधे मार्ग पर हो
\end{hindi}}
\flushright{\begin{Arabic}
\quranayah[43][44]
\end{Arabic}}
\flushleft{\begin{hindi}
निश्चय ही वह अनुस्मृति है तुम्हारे लिए और तुम्हारी क़ौम के लिए। शीघ्र ही तुम सबसे पूछा जाएगा
\end{hindi}}
\flushright{\begin{Arabic}
\quranayah[43][45]
\end{Arabic}}
\flushleft{\begin{hindi}
तुम हमारे रसूलों से, जिन्हें हमने तुमसे पहले भेजा, पूछ लो कि क्या हमने रहमान के सिवा भी कुछ उपास्य ठहराए थे, जिनकी बन्दगी की जाए?
\end{hindi}}
\flushright{\begin{Arabic}
\quranayah[43][46]
\end{Arabic}}
\flushleft{\begin{hindi}
और हमने मूसा को अपनी निशानियों के साथ फ़िरऔन और उसके सरदारों के पास भेजा तो उसने कहा, "मैं सारे संसार के रब का रसूल हूँ।"
\end{hindi}}
\flushright{\begin{Arabic}
\quranayah[43][47]
\end{Arabic}}
\flushleft{\begin{hindi}
लेकिन जब वह उनके पास हमारी निशानियाँ लेकर आया तो क्या देखते है कि वे लगे उनकी हँसी उड़ाने
\end{hindi}}
\flushright{\begin{Arabic}
\quranayah[43][48]
\end{Arabic}}
\flushleft{\begin{hindi}
और हम उन्हें जो निशानी भी दिखाते वह अपने प्रकार की पहली निशानी से बढ़-चढ़कर होती और हमने उन्हें यातना से ग्रस्त कर लिया, ताकि वे रुजू करें
\end{hindi}}
\flushright{\begin{Arabic}
\quranayah[43][49]
\end{Arabic}}
\flushleft{\begin{hindi}
उनका कहना था, "ऐ जादूगर! अपने रब से हमारे लिए प्रार्थना कर, उस प्रतिज्ञा के आधार पर जो उसने तुझसे कर रखी है। निश्चय ही हम सीधे मार्ग पर चलेंगे।"
\end{hindi}}
\flushright{\begin{Arabic}
\quranayah[43][50]
\end{Arabic}}
\flushleft{\begin{hindi}
फिर जब भी हम उनपर ले यातना हटा देते है, तो क्या देखते है कि वे प्रतिज्ञा-भंग कर रहे है
\end{hindi}}
\flushright{\begin{Arabic}
\quranayah[43][51]
\end{Arabic}}
\flushleft{\begin{hindi}
फ़िरऔन ने अपनी क़ौम के बीच पुकारकर कहा, "ऐ मेरी क़ौम के लोगो! क्या मिस्र का राज्य मेरा नहीं और ये मेरे नीचे बहती नहरें? तो क्या तुम देखते नहीं?
\end{hindi}}
\flushright{\begin{Arabic}
\quranayah[43][52]
\end{Arabic}}
\flushleft{\begin{hindi}
(यह अच्छा है) या मैं इससे अच्छा हूँ जो तुच्छ है, और साफ़ बोल भी नहीं पाता?
\end{hindi}}
\flushright{\begin{Arabic}
\quranayah[43][53]
\end{Arabic}}
\flushleft{\begin{hindi}
(यदि वह रसूल है तो) फिर ऐसा क्यों न हुआ कि उसके लिए ऊपर से सोने के कंगन डाले गए होते या उसके साथ पार्श्ववर्ती होकर फ़रिश्ते आए होते?"
\end{hindi}}
\flushright{\begin{Arabic}
\quranayah[43][54]
\end{Arabic}}
\flushleft{\begin{hindi}
तो उसने अपनी क़ौम के लोगों को मूर्ख बनाया औऱ उन्होंने उसकी बात मान ली। निश्चय ही वे अवज्ञाकारी लोग थे
\end{hindi}}
\flushright{\begin{Arabic}
\quranayah[43][55]
\end{Arabic}}
\flushleft{\begin{hindi}
अन्ततः जब उन्होंने हमें अप्रसन्न कर दिया तो हमने उनसे बदला लिया और हमने उन सबको डूबो दिया
\end{hindi}}
\flushright{\begin{Arabic}
\quranayah[43][56]
\end{Arabic}}
\flushleft{\begin{hindi}
अतः हमने उन्हें अग्रगामी और बादवालों के लिए शिक्षाप्रद उदाहरण बना दिया
\end{hindi}}
\flushright{\begin{Arabic}
\quranayah[43][57]
\end{Arabic}}
\flushleft{\begin{hindi}
और जब मरयम के बेटे की मिसाल दी गई तो क्या देखते है कि उसपर तुम्हारी क़ौम के लोग लगे चिल्लाने
\end{hindi}}
\flushright{\begin{Arabic}
\quranayah[43][58]
\end{Arabic}}
\flushleft{\begin{hindi}
और कहने लगे, "क्या हमारे उपास्य अच्छे नहीं या वह (मसीह)?" उन्होंने यह बात तुमसे केवल झगड़ने के लिए कही, बल्कि वे तो है ही झगड़ालू लोग
\end{hindi}}
\flushright{\begin{Arabic}
\quranayah[43][59]
\end{Arabic}}
\flushleft{\begin{hindi}
वह (ईसा मसीह) तो बस एक बन्दा था, जिसपर हमने अनुकम्पा की और उसे हमने इसराईल की सन्तान के लिए एक आदर्श बनाया
\end{hindi}}
\flushright{\begin{Arabic}
\quranayah[43][60]
\end{Arabic}}
\flushleft{\begin{hindi}
और यदि हम चाहते हो तुममें से फ़रिश्ते पैदा कर देते, जो धरती में उत्ताराधिकारी होते
\end{hindi}}
\flushright{\begin{Arabic}
\quranayah[43][61]
\end{Arabic}}
\flushleft{\begin{hindi}
निश्चय ही वह उस घड़ी (जिसका वादा किया गया है) के ज्ञान का साधन है। अतः तुम उसके बारे में संदेह न करो और मेरा अनुसरण करो। यही सीधा मार्ग है
\end{hindi}}
\flushright{\begin{Arabic}
\quranayah[43][62]
\end{Arabic}}
\flushleft{\begin{hindi}
और शैतान तुम्हें रोक न दे, निश्चय ही वह तुम्हारा खुला शत्रु है
\end{hindi}}
\flushright{\begin{Arabic}
\quranayah[43][63]
\end{Arabic}}
\flushleft{\begin{hindi}
जब ईसा स्पष्ट प्रमाणों के साथ आया तो उसने कहा, "मैं तुम्हारे पास तत्वदर्शिता लेकर आया हूँ (ताकि उसकी शिक्षा तुम्हें दूँ) और ताकि कुछ ऐसी बातें तुमपर खोल दूँ, जिनमं तुम मतभेद करते हो। अतः अल्लाह का डर रखो और मेरी बात मानो
\end{hindi}}
\flushright{\begin{Arabic}
\quranayah[43][64]
\end{Arabic}}
\flushleft{\begin{hindi}
वास्तव में अल्लाह ही मेरा भी रब है और तुम्हारा भी रब है, तो उसी की बन्दगी करो। यही सीधा मार्ग है।"
\end{hindi}}
\flushright{\begin{Arabic}
\quranayah[43][65]
\end{Arabic}}
\flushleft{\begin{hindi}
किन्तु उनमें के कितने ही गिरोहों ने आपस में विभेद किया। अतः तबाही है एक दुखद दिन की यातना से, उन लोगों के लिए जिन्होंने ज़ुल्म किया
\end{hindi}}
\flushright{\begin{Arabic}
\quranayah[43][66]
\end{Arabic}}
\flushleft{\begin{hindi}
क्या वे बस उस (क़ियामत की) घड़ी की प्रतीक्षा कर रहे है कि वह सहसा उनपर आ पड़े और उन्हें ख़बर भी न हो
\end{hindi}}
\flushright{\begin{Arabic}
\quranayah[43][67]
\end{Arabic}}
\flushleft{\begin{hindi}
उस दिन सभी मित्र परस्पर एक-दूसरे के शत्रु होंगे सिवाय डर रखनेवालों के। -
\end{hindi}}
\flushright{\begin{Arabic}
\quranayah[43][68]
\end{Arabic}}
\flushleft{\begin{hindi}
"ऐ मेरे बन्दों! आज न तुम्हें कोई भय है और न तुम शोकाकुल होगे।" -
\end{hindi}}
\flushright{\begin{Arabic}
\quranayah[43][69]
\end{Arabic}}
\flushleft{\begin{hindi}
वह जो हमारी आयतों पर ईमान लाए और आज्ञाकारी रहे;
\end{hindi}}
\flushright{\begin{Arabic}
\quranayah[43][70]
\end{Arabic}}
\flushleft{\begin{hindi}
"प्रवेश करो जन्नत में, तुम भी और तुम्हारे जोड़े भी, हर्षित होकर!"
\end{hindi}}
\flushright{\begin{Arabic}
\quranayah[43][71]
\end{Arabic}}
\flushleft{\begin{hindi}
उनके आगे सोने की तशतरियाँ और प्याले गर्दिश करेंगे और वहाँ वह सब कुछ होगा, जो दिलों को भाए और आँखे जिससे लज़्ज़त पाएँ। "और तुम उसमें सदैव रहोगे
\end{hindi}}
\flushright{\begin{Arabic}
\quranayah[43][72]
\end{Arabic}}
\flushleft{\begin{hindi}
यह वह जन्नत है जिसके तुम वारिस उसके बदले में हुए जो कर्म तुम करते रहे।
\end{hindi}}
\flushright{\begin{Arabic}
\quranayah[43][73]
\end{Arabic}}
\flushleft{\begin{hindi}
तुम्हारे लिए वहाँ बहुत-से स्वादिष्ट फल है जिन्हें तुम खाओगे।"
\end{hindi}}
\flushright{\begin{Arabic}
\quranayah[43][74]
\end{Arabic}}
\flushleft{\begin{hindi}
निस्संदेह अपराधी लोग सदैव जहन्नम की यातना में रहेंगे
\end{hindi}}
\flushright{\begin{Arabic}
\quranayah[43][75]
\end{Arabic}}
\flushleft{\begin{hindi}
वह (यातना) कभी उनपर से हल्की न होगी और वे उसी में निराश पड़े रहेंगे
\end{hindi}}
\flushright{\begin{Arabic}
\quranayah[43][76]
\end{Arabic}}
\flushleft{\begin{hindi}
हमने उनपर कोई ज़ुल्म नहीं किया, परन्तु वे खुद ही ज़ालिम थे
\end{hindi}}
\flushright{\begin{Arabic}
\quranayah[43][77]
\end{Arabic}}
\flushleft{\begin{hindi}
वे पुकारेंगे, "ऐ मालिक! तुम्हारा रब हमारा काम ही तमाम कर दे!" वह कहेगा, "तुम्हें तो इसी दशा में रहना है।"
\end{hindi}}
\flushright{\begin{Arabic}
\quranayah[43][78]
\end{Arabic}}
\flushleft{\begin{hindi}
"निश्चय ही हम तुम्हारे पास सत्य लेकर आए है, किन्तु तुममें से अधिकतर लोगों को सत्य प्रिय नहीं
\end{hindi}}
\flushright{\begin{Arabic}
\quranayah[43][79]
\end{Arabic}}
\flushleft{\begin{hindi}
(क्या उन्होंने कुछ निश्चय नहीं किया है) या उन्होंने किसी बात का निश्चय कर लिया है? अच्छा तो हमने भी निश्चय कर लिया है
\end{hindi}}
\flushright{\begin{Arabic}
\quranayah[43][80]
\end{Arabic}}
\flushleft{\begin{hindi}
या वे समझते है कि हम उनकी छिपी बात और उनकी कानाफूसी को सुनते नही? क्यों नहीं, और हमारे भेजे हुए (फ़रिश्ते) उनके समीप हैं, वे लिखते रहते है।"
\end{hindi}}
\flushright{\begin{Arabic}
\quranayah[43][81]
\end{Arabic}}
\flushleft{\begin{hindi}
कहो, "यदि रहमान की कोई सन्तान होती तो सबसे पहले मैं (उसे) पूजता
\end{hindi}}
\flushright{\begin{Arabic}
\quranayah[43][82]
\end{Arabic}}
\flushleft{\begin{hindi}
आकाशों और धरती का रब, सिंहासन का स्वामी, उससे महान और उच्च है जो वे बयान करते है।"
\end{hindi}}
\flushright{\begin{Arabic}
\quranayah[43][83]
\end{Arabic}}
\flushleft{\begin{hindi}
अच्छा, छोड़ो उन्हें कि वे व्यर्थ की बहस में पड़े रहे और खेलों में लगे रहें। यहाँ तक कि उनकी भेंट अपने उस दिन से हो जिसका वादा उनसे किया जाता है
\end{hindi}}
\flushright{\begin{Arabic}
\quranayah[43][84]
\end{Arabic}}
\flushleft{\begin{hindi}
वही है जो आकाशों में भी पूज्य है और धरती में भी पूज्य है और वह तत्वदर्शी, सर्वज्ञ है
\end{hindi}}
\flushright{\begin{Arabic}
\quranayah[43][85]
\end{Arabic}}
\flushleft{\begin{hindi}
बड़ी ही बरकतवाली है वह सत्ता, जिसके अधिकार में है आकाशों और धरती की बादशाही और जो कुछ उन दिनों के बीच है उसकी भी। और उसी के पास उस घड़ी का ज्ञान है, और उसी की ओर तुम लौटाए जाओगे।
\end{hindi}}
\flushright{\begin{Arabic}
\quranayah[43][86]
\end{Arabic}}
\flushleft{\begin{hindi}
और जिन्हें वे उसके और अपने बीच माध्यम ठहराकर पुकारते है, उन्हें सिफ़ारिश का कुछ भी अधिकार नहीं, बस उसे ही यह अधिकार प्राप्त, है जो हक की गवाही दे, और ऐसे लोग जानते है।-
\end{hindi}}
\flushright{\begin{Arabic}
\quranayah[43][87]
\end{Arabic}}
\flushleft{\begin{hindi}
यदि तुम उनसे पूछो कि "उन्हें किसने पैदा किया?" तो वे अवश्य कहेंगे, "अल्लाह ने।" तो फिर वे कहाँ उलटे फिर जाते है?-
\end{hindi}}
\flushright{\begin{Arabic}
\quranayah[43][88]
\end{Arabic}}
\flushleft{\begin{hindi}
और उसका कहना हो कि "ऐ मेरे रब! निश्चय ही ये वे लोग है, जो ईमान नहीं रखते थे।"
\end{hindi}}
\flushright{\begin{Arabic}
\quranayah[43][89]
\end{Arabic}}
\flushleft{\begin{hindi}
अच्छा तो उनसे नज़र फेर लो और कह दो, "सलाम है तुम्हें!" अन्ततः शीघ्र ही वे स्वयं जान लेंगे
\end{hindi}}
\chapter{Ad-Dukhan (The Drought)}
\begin{Arabic}
\Huge{\centerline{\basmalah}}\end{Arabic}
\flushright{\begin{Arabic}
\quranayah[44][1]
\end{Arabic}}
\flushleft{\begin{hindi}
हा॰ मीम॰
\end{hindi}}
\flushright{\begin{Arabic}
\quranayah[44][2]
\end{Arabic}}
\flushleft{\begin{hindi}
गवाह है स्पष्ट किताब
\end{hindi}}
\flushright{\begin{Arabic}
\quranayah[44][3]
\end{Arabic}}
\flushleft{\begin{hindi}
निस्संदेह हमने उसे एक बरकत भरी रात में अवतरित किया है। - निश्चय ही हम सावधान करनेवाले है।-
\end{hindi}}
\flushright{\begin{Arabic}
\quranayah[44][4]
\end{Arabic}}
\flushleft{\begin{hindi}
उस (रात) में तमाम तत्वदर्शिता युक्त मामलों का फ़ैसला किया जाता है,
\end{hindi}}
\flushright{\begin{Arabic}
\quranayah[44][5]
\end{Arabic}}
\flushleft{\begin{hindi}
हमारे यहाँ से आदेश के रूप में। निस्संदेह रसूलों को भेजनेवाले हम ही है। -
\end{hindi}}
\flushright{\begin{Arabic}
\quranayah[44][6]
\end{Arabic}}
\flushleft{\begin{hindi}
तुम्हारे रब की दयालुता के कारण। निस्संदेह वही सब कुछ सुननेवाला, जाननेवाला है
\end{hindi}}
\flushright{\begin{Arabic}
\quranayah[44][7]
\end{Arabic}}
\flushleft{\begin{hindi}
आकाशों और धरती का रब और जो कुछ उन दोनों के बीच है उसका भी, यदि तुम विश्वास रखनेवाले हो (तो विश्वास करो कि किताब का अवतरण अल्लाह की दयालुता है)
\end{hindi}}
\flushright{\begin{Arabic}
\quranayah[44][8]
\end{Arabic}}
\flushleft{\begin{hindi}
उसके अतिरिक्त कोई पूज्य-प्रभु नहीं; वही जीवित करता और मारता है; तुम्हारा रब और तुम्हारे अगले बाप-दादों का रब है
\end{hindi}}
\flushright{\begin{Arabic}
\quranayah[44][9]
\end{Arabic}}
\flushleft{\begin{hindi}
बल्कि वे संदेह में पड़े रहे हैं
\end{hindi}}
\flushright{\begin{Arabic}
\quranayah[44][10]
\end{Arabic}}
\flushleft{\begin{hindi}
अच्छा तो तुम उस दिन की प्रतीक्षा करो, जब आकाश प्रत्यक्ष धुँआ लाएगा।
\end{hindi}}
\flushright{\begin{Arabic}
\quranayah[44][11]
\end{Arabic}}
\flushleft{\begin{hindi}
वह लोगों का ढाँक लेगा। यह है दुखद यातना!
\end{hindi}}
\flushright{\begin{Arabic}
\quranayah[44][12]
\end{Arabic}}
\flushleft{\begin{hindi}
वे कहेंगे, "ऐ हमारे रब! हमपर से यातना हटा दे। हम ईमान लाते है।"
\end{hindi}}
\flushright{\begin{Arabic}
\quranayah[44][13]
\end{Arabic}}
\flushleft{\begin{hindi}
अब उनके होश में आने का मौक़ा कहाँ बाक़ी रहा। उनका हाल तो यह है कि उनके पास साफ़-साफ़ बतानेवाला एक रसूल आ चुका है।
\end{hindi}}
\flushright{\begin{Arabic}
\quranayah[44][14]
\end{Arabic}}
\flushleft{\begin{hindi}
फिर उन्होंने उसकी ओर से मुँह मोड़ लिया और कहने लगे, "यह तो एक सिखाया-पढ़ाया दीवाना है।"
\end{hindi}}
\flushright{\begin{Arabic}
\quranayah[44][15]
\end{Arabic}}
\flushleft{\begin{hindi}
"हम यातना थोड़ा हटा देते है तो तुम पुनः फिर जाते हो।
\end{hindi}}
\flushright{\begin{Arabic}
\quranayah[44][16]
\end{Arabic}}
\flushleft{\begin{hindi}
याद रखो, जिस दिन हम बड़ी पकड़ पकड़ेंगे, तो निश्चय ही हम बदला लेकर रहेंगे
\end{hindi}}
\flushright{\begin{Arabic}
\quranayah[44][17]
\end{Arabic}}
\flushleft{\begin{hindi}
उनसे पहले हम फ़िरऔन की क़ौम के लोगों को परीक्षा में डाल चुके हैं, जबकि उनके पास एक अत्यन्त सज्जन रसूल आया
\end{hindi}}
\flushright{\begin{Arabic}
\quranayah[44][18]
\end{Arabic}}
\flushleft{\begin{hindi}
कि "तुम अल्लाह के बन्दों को मेरे हवाले कर दो। निश्चय ही मै तुम्हारे लिए एक विश्वसनीय रसूल हूँ
\end{hindi}}
\flushright{\begin{Arabic}
\quranayah[44][19]
\end{Arabic}}
\flushleft{\begin{hindi}
और अल्लाह के मुक़ाबले में सरकशी न करो, मैं तुम्हारे लिए एक स्पष्ट प्रमाण लेकर आया हूँ
\end{hindi}}
\flushright{\begin{Arabic}
\quranayah[44][20]
\end{Arabic}}
\flushleft{\begin{hindi}
और मैं इससे अपने रब और तुम्हारे रब की शरण ले चुका हूँ कि तुम मुझ पर पथराव करके मार डालो
\end{hindi}}
\flushright{\begin{Arabic}
\quranayah[44][21]
\end{Arabic}}
\flushleft{\begin{hindi}
किन्तु यदि तुम मेरी बात नहीं मानते तो मुझसे अलग हो जाओ!"
\end{hindi}}
\flushright{\begin{Arabic}
\quranayah[44][22]
\end{Arabic}}
\flushleft{\begin{hindi}
अन्ततः उसने अपने रब को पुकारा कि "ये अपराधी लोग है।"
\end{hindi}}
\flushright{\begin{Arabic}
\quranayah[44][23]
\end{Arabic}}
\flushleft{\begin{hindi}
"अच्छा तुम रातों रात मेरे बन्दों को लेकर चले जाओ। निश्चय ही तुम्हारा पीछा किया जाएगा
\end{hindi}}
\flushright{\begin{Arabic}
\quranayah[44][24]
\end{Arabic}}
\flushleft{\begin{hindi}
और सागर को स्थिर छोड़ दो। वे तो एक सेना दल हैं, डूब जानेवाले।"
\end{hindi}}
\flushright{\begin{Arabic}
\quranayah[44][25]
\end{Arabic}}
\flushleft{\begin{hindi}
वे छोड़ गये कितनॆ ही बाग़ और स्रोत
\end{hindi}}
\flushright{\begin{Arabic}
\quranayah[44][26]
\end{Arabic}}
\flushleft{\begin{hindi}
और ख़ेतियां और उत्तम आवास-
\end{hindi}}
\flushright{\begin{Arabic}
\quranayah[44][27]
\end{Arabic}}
\flushleft{\begin{hindi}
और सुख सामग्री जिनमें वे मज़े कर रहे थे।
\end{hindi}}
\flushright{\begin{Arabic}
\quranayah[44][28]
\end{Arabic}}
\flushleft{\begin{hindi}
हम ऐसा ही मामला करते है, और उन चीज़ों का वारिस हमने दूसरे लोगों को बनाया
\end{hindi}}
\flushright{\begin{Arabic}
\quranayah[44][29]
\end{Arabic}}
\flushleft{\begin{hindi}
फिर न तो आकाश और धरती ने उनपर विलाप किया और न उन्हें मुहलत ही मिली
\end{hindi}}
\flushright{\begin{Arabic}
\quranayah[44][30]
\end{Arabic}}
\flushleft{\begin{hindi}
इस प्रकार हमने इसराईल की सन्तान को अपमानजनक यातना से
\end{hindi}}
\flushright{\begin{Arabic}
\quranayah[44][31]
\end{Arabic}}
\flushleft{\begin{hindi}
अर्थात फ़िरऔन से छुटकारा दिया। निश्चय ही वह मर्यादाहीन लोगों में से बड़ा ही सरकश था
\end{hindi}}
\flushright{\begin{Arabic}
\quranayah[44][32]
\end{Arabic}}
\flushleft{\begin{hindi}
और हमने (उनकी स्थिति को) जानते हुए उन्हें सारे संसारवालों के मुक़ाबले मं चुन लिया
\end{hindi}}
\flushright{\begin{Arabic}
\quranayah[44][33]
\end{Arabic}}
\flushleft{\begin{hindi}
और हमने उन्हें निशानियों के द्वारा वह चीज़ दी जिसमें स्पष्ट परीक्षा थी
\end{hindi}}
\flushright{\begin{Arabic}
\quranayah[44][34]
\end{Arabic}}
\flushleft{\begin{hindi}
ये लोग बड़ी दृढ़तापूर्वक कहते है,
\end{hindi}}
\flushright{\begin{Arabic}
\quranayah[44][35]
\end{Arabic}}
\flushleft{\begin{hindi}
"बस यह हमारी पहली मृत्यु ही है, हम दोबारा उठाए जानेवाले नहीं हैं
\end{hindi}}
\flushright{\begin{Arabic}
\quranayah[44][36]
\end{Arabic}}
\flushleft{\begin{hindi}
तो ले आओ हमारे बाप-दादा को, यदि तुम सच्चे हो!"
\end{hindi}}
\flushright{\begin{Arabic}
\quranayah[44][37]
\end{Arabic}}
\flushleft{\begin{hindi}
क्या वे अच्छे है या तुब्बा की क़ौम या वे लोग जो उनसे पहले गुज़र चुके है? हमने उन्हें विनष्ट कर दिया, निश्चय ही वे अपराधी थे
\end{hindi}}
\flushright{\begin{Arabic}
\quranayah[44][38]
\end{Arabic}}
\flushleft{\begin{hindi}
हमने आकाशों और धरती को और जो कुछ उनके बीच है उन्हें खेल नहीं बनाया
\end{hindi}}
\flushright{\begin{Arabic}
\quranayah[44][39]
\end{Arabic}}
\flushleft{\begin{hindi}
हमने उन्हें हक़ के साथ पैदा किया, किन्तु उनमें से अधिककर लोग जानते नहीं
\end{hindi}}
\flushright{\begin{Arabic}
\quranayah[44][40]
\end{Arabic}}
\flushleft{\begin{hindi}
निश्चय ही फ़ैसले का दिन उन सबका नियत समय है,
\end{hindi}}
\flushright{\begin{Arabic}
\quranayah[44][41]
\end{Arabic}}
\flushleft{\begin{hindi}
जिस दिन कोई अपना किसी अपने के कुछ काम न आएगा और न कोई सहायता पहुँचेगी,
\end{hindi}}
\flushright{\begin{Arabic}
\quranayah[44][42]
\end{Arabic}}
\flushleft{\begin{hindi}
सिवाय उस व्यक्ति के जिसपर अल्लाह दया करे। निश्चय ही वह प्रभुत्वशाली, अत्यन्त दयावान है
\end{hindi}}
\flushright{\begin{Arabic}
\quranayah[44][43]
\end{Arabic}}
\flushleft{\begin{hindi}
निस्संदेह ज़क़्क़ूम का वृक्ष
\end{hindi}}
\flushright{\begin{Arabic}
\quranayah[44][44]
\end{Arabic}}
\flushleft{\begin{hindi}
गुनहगार का भोजन होगा,
\end{hindi}}
\flushright{\begin{Arabic}
\quranayah[44][45]
\end{Arabic}}
\flushleft{\begin{hindi}
तेल की तलछट जैसा, वह पेटों में खौलता होगा,
\end{hindi}}
\flushright{\begin{Arabic}
\quranayah[44][46]
\end{Arabic}}
\flushleft{\begin{hindi}
जैसे गर्म पानी खौलता है
\end{hindi}}
\flushright{\begin{Arabic}
\quranayah[44][47]
\end{Arabic}}
\flushleft{\begin{hindi}
"पकड़ो उसे, और भड़कती हुई आग के बीच तक घसीट ले जाओ,
\end{hindi}}
\flushright{\begin{Arabic}
\quranayah[44][48]
\end{Arabic}}
\flushleft{\begin{hindi}
फिर उसके सिर पर खौलते हुए पानी का यातना उंडेल दो!"
\end{hindi}}
\flushright{\begin{Arabic}
\quranayah[44][49]
\end{Arabic}}
\flushleft{\begin{hindi}
"मज़ा चख, तू तो बड़ा बलशाली, सज्जन और आदरणीय है!
\end{hindi}}
\flushright{\begin{Arabic}
\quranayah[44][50]
\end{Arabic}}
\flushleft{\begin{hindi}
यही तो है जिसके विषय में तुम संदेह करते थे।"
\end{hindi}}
\flushright{\begin{Arabic}
\quranayah[44][51]
\end{Arabic}}
\flushleft{\begin{hindi}
निस्संदेह डर रखनेवाले निश्चिन्तता की जगह होंगे,
\end{hindi}}
\flushright{\begin{Arabic}
\quranayah[44][52]
\end{Arabic}}
\flushleft{\begin{hindi}
बाग़ों और स्रोतों में
\end{hindi}}
\flushright{\begin{Arabic}
\quranayah[44][53]
\end{Arabic}}
\flushleft{\begin{hindi}
बारीक और गाढ़े रेशम के वस्त्र पहने हुए, एक-दूसरे के आमने-सामने उपस्थित होंगे
\end{hindi}}
\flushright{\begin{Arabic}
\quranayah[44][54]
\end{Arabic}}
\flushleft{\begin{hindi}
ऐसा ही उनके साथ मामला होगा। और हम साफ़ गोरी, बड़ी नेत्रोवाली स्त्रियों से उनका विवाह कर देंगे
\end{hindi}}
\flushright{\begin{Arabic}
\quranayah[44][55]
\end{Arabic}}
\flushleft{\begin{hindi}
वे वहाँ निश्चिन्तता के साथ हर प्रकार के स्वादिष्ट फल मँगवाते होंगे
\end{hindi}}
\flushright{\begin{Arabic}
\quranayah[44][56]
\end{Arabic}}
\flushleft{\begin{hindi}
वहाँ वे मृत्यु का मज़ा कभी न चखेगे। बस पहली मृत्यु जो हुई, सो हुई। और उसने उन्हें भड़कती हुई आग की यातना से बचा लिया
\end{hindi}}
\flushright{\begin{Arabic}
\quranayah[44][57]
\end{Arabic}}
\flushleft{\begin{hindi}
यह सब तुम्हारे रब के विशेष उदार अनुग्रह के कारण होगा, वही बड़ी सफलता है
\end{hindi}}
\flushright{\begin{Arabic}
\quranayah[44][58]
\end{Arabic}}
\flushleft{\begin{hindi}
हमने तो इस (क़ुरआन) को बस तुम्हारी भाषा में सहज एवं सुगम बना दिया है ताकि वे याददिहानी प्राप्त (करें
\end{hindi}}
\flushright{\begin{Arabic}
\quranayah[44][59]
\end{Arabic}}
\flushleft{\begin{hindi}
अच्छा तुम भी प्रतीक्षा करो, वे भी प्रतीक्षा में हैं
\end{hindi}}
\chapter{Al-Jathiyah (The Kneeling)}
\begin{Arabic}
\Huge{\centerline{\basmalah}}\end{Arabic}
\flushright{\begin{Arabic}
\quranayah[45][1]
\end{Arabic}}
\flushleft{\begin{hindi}
हा॰ मीम॰
\end{hindi}}
\flushright{\begin{Arabic}
\quranayah[45][2]
\end{Arabic}}
\flushleft{\begin{hindi}
इस किताब का अवतरण अल्लाह की ओर से है, जो अत्यन्त प्रभुत्वशाली, तत्वदर्शी है। -
\end{hindi}}
\flushright{\begin{Arabic}
\quranayah[45][3]
\end{Arabic}}
\flushleft{\begin{hindi}
निस्संदेह आकाशों और धरती में ईमानवालों के लिए बहुत-सी निशानियाँ है
\end{hindi}}
\flushright{\begin{Arabic}
\quranayah[45][4]
\end{Arabic}}
\flushleft{\begin{hindi}
और तुम्हारी संरचना में, और उनकी भी जो जानवर वह फैलाता रहता है, निशानियाँ है उन लोगों के लिए जो विश्वास करें
\end{hindi}}
\flushright{\begin{Arabic}
\quranayah[45][5]
\end{Arabic}}
\flushleft{\begin{hindi}
और रात और दिन के उलट-फेर में भी, और उस रोज़ी (पानी) में भी जिसे अल्लाह ने आकाश से उतारा, फिर उसके द्वारा धरती को उसके उन मुर्दा हो जाने के पश्चात जीवित किया और हवाओं की गर्दिश में भीलोगों के लिए बहुत-सी निशानियाँ है जो बुद्धि से काम लें
\end{hindi}}
\flushright{\begin{Arabic}
\quranayah[45][6]
\end{Arabic}}
\flushleft{\begin{hindi}
ये अल्लाह की आयतें हैं, हम उन्हें हक़ के साथ तुमको सुना रहे हैं। अब आख़िर अल्लाह और उसकी आयतों के पश्चात और कौन-सी बात है जिसपर वे ईमान लाएँगे?
\end{hindi}}
\flushright{\begin{Arabic}
\quranayah[45][7]
\end{Arabic}}
\flushleft{\begin{hindi}
तबाही है हर झूठ घड़नेवाले गुनहगार के लिए,
\end{hindi}}
\flushright{\begin{Arabic}
\quranayah[45][8]
\end{Arabic}}
\flushleft{\begin{hindi}
जो अल्लाह की उन आयतों को सुनता है जो उसे पढ़कर सुनाई जाती है। फिर घमंड के साथ अपनी (इनकार की) नीति पर अड़ा रहता है मानो उसने उनको सुना ही नहीं। अतः उसको दुखद यातना की शुभ सूचना दे दो
\end{hindi}}
\flushright{\begin{Arabic}
\quranayah[45][9]
\end{Arabic}}
\flushleft{\begin{hindi}
जब हमारी आयतों में से कोई बात वह जान लेता है तो वह उनका परिहास करता है, ऐसे लोगों के लिए रुसवा कर देनेवाली यातना है
\end{hindi}}
\flushright{\begin{Arabic}
\quranayah[45][10]
\end{Arabic}}
\flushleft{\begin{hindi}
उनके आगे जहन्नम है, जो उन्होंने कमाया वह उनके कुछ काम न आएगा और न यही कि उन्होंने अल्लाह को छोड़कर अपने संरक्षक ठहरा रखे है। उनके लिए तो बड़ी यातना है
\end{hindi}}
\flushright{\begin{Arabic}
\quranayah[45][11]
\end{Arabic}}
\flushleft{\begin{hindi}
यह सर्वथा मार्गदर्शन है। और जिन लोगों ने अपने रब की आयतों को इनकार किया, उनके लिए हिला देनेवाली दुखद यातना है
\end{hindi}}
\flushright{\begin{Arabic}
\quranayah[45][12]
\end{Arabic}}
\flushleft{\begin{hindi}
वह अल्लाह ही है जिसने समुद्र को तुम्हारे लिए वशीभूत कर दिया है, ताकि उसके आदेश से नौकाएँ उसमें चलें; और ताकि तुम उसका उदार अनुग्रह तलाश करो; और इसलिए कि तुम कृतज्ञता दिखाओ
\end{hindi}}
\flushright{\begin{Arabic}
\quranayah[45][13]
\end{Arabic}}
\flushleft{\begin{hindi}
जो चीज़ें आकाशों में है और जो धरती में हैं, उसने उन सबको अपनी ओर से तुम्हारे काम में लगा रखा है। निश्चय ही इसमें उन लोगों के लिए निशानियाँ है जो सोच-विचार से काम लेते है
\end{hindi}}
\flushright{\begin{Arabic}
\quranayah[45][14]
\end{Arabic}}
\flushleft{\begin{hindi}
जो लोग ईमान लाए उनसे कह दो कि, "वे उन लोगों को क्षमा करें (उनकी करतूतों पर ध्यान न दे) अल्लाह के दिनों की आशा नहीं रखते, ताकि वह इसके परिणामस्वरूप उन लोगों को उनकी अपनी कमाई का बदला दे
\end{hindi}}
\flushright{\begin{Arabic}
\quranayah[45][15]
\end{Arabic}}
\flushleft{\begin{hindi}
जो कुछ अच्छा कर्म करता है तो अपने ही लिए करेगा और जो कोई बुरा कर्म करता है तो उसका वबाल उसी पर होगा। फिर तुम अपने रब की ओर लौटाये जाओगे
\end{hindi}}
\flushright{\begin{Arabic}
\quranayah[45][16]
\end{Arabic}}
\flushleft{\begin{hindi}
निश्चय ही हमने इसराईल की सन्तान को किताब और हुक्म और पैग़म्बरी प्रदान की थी। और हमने उन्हें पवित्र चीज़ो की रोज़ी दी और उन्हें सारे संसारवालों पर श्रेष्ठता प्रदान की
\end{hindi}}
\flushright{\begin{Arabic}
\quranayah[45][17]
\end{Arabic}}
\flushleft{\begin{hindi}
और हमने उन्हें इस मामले के विषय में स्पष्ट निशानियाँ प्रदान कीं। फिर जो भी विभेद उन्होंने किया, वह इसके पश्चात ही किया कि उनके पास ज्ञान आ चुका था और इस कारण कि वे परस्पर एक-दूसरे पर ज़्यादती करना चाहते थे। निश्चय ही तुम्हारा रब क़ियामत के दिन उनके बीच उन चीज़ों के बारे में फ़ैसला कर देगा, जिनमें वे परस्पर विभेद करते रहे है
\end{hindi}}
\flushright{\begin{Arabic}
\quranayah[45][18]
\end{Arabic}}
\flushleft{\begin{hindi}
फिर हमने तुम्हें इस मामलें में एक खुले मार्ग (शरीअत) पर कर दिया। अतः तुम उसी पर चलो और उन लोगों की इच्छाओं का अनुपालन न करना जो जानते नहीं
\end{hindi}}
\flushright{\begin{Arabic}
\quranayah[45][19]
\end{Arabic}}
\flushleft{\begin{hindi}
वे अल्लाह के मुक़ाबले में तुम्हारे कदापि कुछ काम नहीं आ सकते। निश्चय ही ज़ालिम लोग एक-दूसरे के साथी है और डर रखनेवालों का साथी अल्लाह है
\end{hindi}}
\flushright{\begin{Arabic}
\quranayah[45][20]
\end{Arabic}}
\flushleft{\begin{hindi}
वह लोगों के लिए सूझ के प्रकाशों का पुंज है, और मार्गदर्शन और दयालुता है उन लोगों के लिए जो विश्वास करें
\end{hindi}}
\flushright{\begin{Arabic}
\quranayah[45][21]
\end{Arabic}}
\flushleft{\begin{hindi}
(क्या मार्गदर्शन और पथभ्रष्ट ता समान है) या वे लोग, जिन्होंने बुराइयाँ कमाई है, यह समझ बैठे हैं कि हम उन्हें उन लोगों जैसा कर देंगे जो ईमान लाए और उन्होंने अच्छे कर्म किए कि उनका जीना और मरना समान हो जाए? बहुत ही बुरा है जो निर्णय वे करते है!
\end{hindi}}
\flushright{\begin{Arabic}
\quranayah[45][22]
\end{Arabic}}
\flushleft{\begin{hindi}
अल्लाह ने आकाशों और धरती को हक़ के साथ पैदा किया और इसलिए कि प्रत्येक व्यक्ति को उसकी कमाई का बदला दिया जाए और उनपर ज़ुल्म न किया जाए
\end{hindi}}
\flushright{\begin{Arabic}
\quranayah[45][23]
\end{Arabic}}
\flushleft{\begin{hindi}
क्या तुमने उस व्यक्ति को नहीं देखा जिसने अपनी इच्छा ही को अपना उपास्य बना लिया? अल्लाह ने (उसकी स्थिति) जानते हुए उसे गुमराही में डाल दिया, और उसके कान और उसके दिल पर ठप्पा लगा दिया और उसकी आँखों पर परदा डाल दिया। फिर अब अल्लाह के पश्चात कौन उसे मार्ग पर ला सकता है? तो क्या तुम शिक्षा नहीं ग्रहण करते?
\end{hindi}}
\flushright{\begin{Arabic}
\quranayah[45][24]
\end{Arabic}}
\flushleft{\begin{hindi}
वे कहते है, "वह तो बस हमारा सांसारिक जीवन ही है। हम मरते और जीते है। हमें तो बस काल (समय) ही विनष्ट करता है।" हालाँकि उनके पास इसका कोई ज्ञान नहीं। वे तो बस अटकलें ही दौड़ाते है
\end{hindi}}
\flushright{\begin{Arabic}
\quranayah[45][25]
\end{Arabic}}
\flushleft{\begin{hindi}
और जब उनके सामने हमारी स्पष्ट आयतें पढ़ी जाती है, तो उनकी हुज्जत इसके सिवा कुछ और नहीं होती कि वे कहते है, "यदि तुम सच्चे हो तो हमारे बाप-दादा को ले आओ।"
\end{hindi}}
\flushright{\begin{Arabic}
\quranayah[45][26]
\end{Arabic}}
\flushleft{\begin{hindi}
कह दो, "अल्लाह ही तुम्हें जीवन प्रदान करता है। फिर वहीं तुम्हें मृत्यु देता है। फिर वही तुम्हें क़ियामत के दिन तक इकट्ठा कर रहा है, जिसमें कोई संदेह नहीं। किन्तु अधिकतर लोग जानते नहीं
\end{hindi}}
\flushright{\begin{Arabic}
\quranayah[45][27]
\end{Arabic}}
\flushleft{\begin{hindi}
आकाशों और धरती की बादशाही अल्लाह ही की है। और जिस दिन वह घड़ी घटित होगी उस दिन झूठवाले घाटे में होंगे
\end{hindi}}
\flushright{\begin{Arabic}
\quranayah[45][28]
\end{Arabic}}
\flushleft{\begin{hindi}
और तुम प्रत्येक गिरोह को घुटनों के बल झुका हुआ देखोगे। प्रत्येक गिरोह अपनी किताब की ओर बुलाया जाएगा, "आज तुम्हें उसी का बदला दिया जाएगा, जो तुम करते थे
\end{hindi}}
\flushright{\begin{Arabic}
\quranayah[45][29]
\end{Arabic}}
\flushleft{\begin{hindi}
"यह हमारी किताब है, जो तुम्हारे मुक़ाबले में ठीक-ठीक बोल रही है। निश्चय ही हम लिखवाते रहे हैं जो कुछ तुम करते थे।"
\end{hindi}}
\flushright{\begin{Arabic}
\quranayah[45][30]
\end{Arabic}}
\flushleft{\begin{hindi}
अतः जो लोग ईमान लाए और उन्होंने अच्छे कर्म किए उन्हें उनका रब अपनी दयालुता में दाख़िल करेगा, यही स्पष्ट सफलता है
\end{hindi}}
\flushright{\begin{Arabic}
\quranayah[45][31]
\end{Arabic}}
\flushleft{\begin{hindi}
रहे वे लोग जिन्होंने इनकार किया (उनसे कहा जाएगा,) "क्या तुम्हें हमारी आयतें पढ़कर नहीं सुनाई जाती थी? किन्तु तुमने घमंड किया और तुम थे ही अपराधी लोग
\end{hindi}}
\flushright{\begin{Arabic}
\quranayah[45][32]
\end{Arabic}}
\flushleft{\begin{hindi}
और जब कहा जाता था कि अल्लाह का वादा सच्चा है और (क़ियामत की) घड़ी में कोई संदेह नहीं हैं। तो तुम कहते थे, "हम नहीं जानते कि वह घड़ी क्या हैं? तो तुम कहते थे, 'हम नहीं जानते कि वह घड़ी क्या है? हमें तो बस एक अनुमान-सा प्रतीत होता है और हमें विश्वास नहीं होता।'"
\end{hindi}}
\flushright{\begin{Arabic}
\quranayah[45][33]
\end{Arabic}}
\flushleft{\begin{hindi}
और जो कुछ वे करते रहे उसकी बुराइयाँ उनपर प्रकट हो गई और जिस चीज़ का वे परिहास करते थे उसी ने उन्हें आ घेरा
\end{hindi}}
\flushright{\begin{Arabic}
\quranayah[45][34]
\end{Arabic}}
\flushleft{\begin{hindi}
और कह दिया जाएगा कि "आज हम तुम्हें भुला देते हैं जैसे तुमने इस दिन की भेंट को भुला रखा था। तुम्हारा ठिकाना अब आग है और तुम्हारा कोई सहायक नहीं
\end{hindi}}
\flushright{\begin{Arabic}
\quranayah[45][35]
\end{Arabic}}
\flushleft{\begin{hindi}
यह इस कारण कि तुमने अल्लाह की आयतों की हँसी उड़ाई थी और सांसारिक जीवन ने तुम्हें धोखे में डाले रखा।" अतः आज वे न तो उससे निकाले जाएँगे और न उनसे यह चाहा जाएगा कि वे किसी उपाय से (अल्लाह के) प्रकोप को दूर कर सकें
\end{hindi}}
\flushright{\begin{Arabic}
\quranayah[45][36]
\end{Arabic}}
\flushleft{\begin{hindi}
अतः सारी प्रशंसा अल्लाह ही के लिए है जो आकाशों का रब और घरती का रब, सारे संसार का रब है
\end{hindi}}
\flushright{\begin{Arabic}
\quranayah[45][37]
\end{Arabic}}
\flushleft{\begin{hindi}
आकाशों और धरती में बड़ाई उसी के लिए है, और वही प्रभुत्वशाली, अत्यन्त तत्वदर्शी है
\end{hindi}}
\chapter{Al-Ahqaf (The Sandhills)}
\begin{Arabic}
\Huge{\centerline{\basmalah}}\end{Arabic}
\flushright{\begin{Arabic}
\quranayah[46][1]
\end{Arabic}}
\flushleft{\begin{hindi}
हा॰ मीम॰
\end{hindi}}
\flushright{\begin{Arabic}
\quranayah[46][2]
\end{Arabic}}
\flushleft{\begin{hindi}
इस किताब का अवतरण अल्लाह की ओर से है, जो प्रभुत्वशाली, अत्यन्त तत्वदर्शी है
\end{hindi}}
\flushright{\begin{Arabic}
\quranayah[46][3]
\end{Arabic}}
\flushleft{\begin{hindi}
हमने आकाशों और धरती को और जो कुछ उन दोनों के मध्य है उसे केवल हक़ के साथ और एक नियत अवधि तक के लिए पैदा किया है। किन्तु जिन लोगों ने इनकार किया है, वे उस चीज़ को ध्यान में नहीं लाते जिससे उन्हें सावधान किया गया है
\end{hindi}}
\flushright{\begin{Arabic}
\quranayah[46][4]
\end{Arabic}}
\flushleft{\begin{hindi}
कहो, "क्या तुमने उनको देखा भी, जिन्हें तुम अल्लाह को छोड़कर पुकारते हो? मुझे दिखाओ उन्होंने धरती की चीज़ों में से क्या पैदा किया है या आकाशों में उनकी कोई साझेदारी है? मेरे पास इससे पहले की कोई किताब ले आओ या ज्ञान की कोई अवशेष बात ही, यदि तुम सच्चे हो।"
\end{hindi}}
\flushright{\begin{Arabic}
\quranayah[46][5]
\end{Arabic}}
\flushleft{\begin{hindi}
आख़़िर उस व्यक्ति से बढ़कर पथभ्रष्ट और कौन होगा, जो अल्लाह से हटकर उन्हें पुकारता हो जो क़ियामत के दिन तक उसकी पुकार को स्वीकार नहीं कर सकते, बल्कि वे तो उनकी पुकार से भी बेख़बर है;
\end{hindi}}
\flushright{\begin{Arabic}
\quranayah[46][6]
\end{Arabic}}
\flushleft{\begin{hindi}
और जब लोग इकट्ठे किए जाएँगे तो वे उनके शत्रु होंगे औऱ उनकी बन्दगी का इनकार करेंगे
\end{hindi}}
\flushright{\begin{Arabic}
\quranayah[46][7]
\end{Arabic}}
\flushleft{\begin{hindi}
जब हमारी स्पष्ट आयतें उन्हें पढ़कर सुनाई जाती है तो वे लोग जिन्होंने इनकार किया, सत्य के विषय में, जबकि वह उनके पास आ गया, कहते है कि "यह तो खुला जादू है।"
\end{hindi}}
\flushright{\begin{Arabic}
\quranayah[46][8]
\end{Arabic}}
\flushleft{\begin{hindi}
(क्या ईमान लाने से उन्हें कोई चीज़ रोक रही है) या वे कहते है, "उसने इसे स्वयं ही घड़ लिया है?" कहो, "यदि मैंने इसे स्वयं घड़ा है तो अल्लाह के विरुद्ध मेरे लिए तुम कुछ भी अधिकार नहीं रखते। जिसके विषय में तुम बातें बनाने में लगे हो, वह उसे भली-भाँति जानता है। और वह मेरे और तुम्हारे बीच गवाह की हैसियत से काफ़ी है। और वही बड़ा क्षमाशील, अत्यन्त दयावान है।"
\end{hindi}}
\flushright{\begin{Arabic}
\quranayah[46][9]
\end{Arabic}}
\flushleft{\begin{hindi}
कह दो, "मैं कोई पहला रसूल तो नहीं हूँ। और मैं नहीं जानता कि मेरे साथ क्या किया जाएगा और न यह कि तुम्हारे साथ क्या किया जाएगा। मैं तो बस उसी का अनुगामी हूँ, जिसकी प्रकाशना मेरी ओर की जाती है और मैं तो केवल एक स्पष्ट सावधान करनेवाला हूँ।"
\end{hindi}}
\flushright{\begin{Arabic}
\quranayah[46][10]
\end{Arabic}}
\flushleft{\begin{hindi}
कहो, "क्या तुमने सोचा भी (कि तुम्हारा क्या परिणाम होगा)? यदि वह (क़ुरआन) अल्लाह के यहाँ से हुआ और तुमने उसका इनकार कर दिया, हालाँकि इसराईल की सन्तान में से एक गवाह ने उसके एक भाग की गवाही भी दी। सो वह ईमान ले आया और तुम घमंड में पड़े रहे। अल्लाह तो ज़ालिम लोगों को मार्ग नहीं दिखाता।"
\end{hindi}}
\flushright{\begin{Arabic}
\quranayah[46][11]
\end{Arabic}}
\flushleft{\begin{hindi}
जिन लोगों ने इनकार किया, वे ईमान लानेवालों के बारे में कहते है, "यदि वह अच्छा होता तो वे उसकी ओर (बढ़ने में) हमसे अग्रसर न रहते।" औऱ जब उन्होंने उससे मार्ग ग्रहण नहीं किया तो अब अवश्य कहेंगे, "यह तो पुराना झूठ है!"
\end{hindi}}
\flushright{\begin{Arabic}
\quranayah[46][12]
\end{Arabic}}
\flushleft{\begin{hindi}
हालाँकि इससे पहले मूसा की किताब पथप्रदर्शक और दयालुता रही है। और यह किताब, जो अरबी भाषा में है, उसकी पुष्टि में है, ताकि उन लोगों को सचेत कर दे जिन्होंने ज़ु्ल्म किया और शुभ सूचना हो उत्तमकारों के लिए
\end{hindi}}
\flushright{\begin{Arabic}
\quranayah[46][13]
\end{Arabic}}
\flushleft{\begin{hindi}
निश्चय ही जिन लोगों ने कहा, "हमारा रब अल्लाह है।" फिर वे उसपर जमे रहे, तो उन्हें न तो कोई भय होगा और न वे शोकाकुल होगे
\end{hindi}}
\flushright{\begin{Arabic}
\quranayah[46][14]
\end{Arabic}}
\flushleft{\begin{hindi}
वही जन्नतवाले है, वहाँ वे सदैव रहेंगे उसके बदले में जो वे करते रहे है
\end{hindi}}
\flushright{\begin{Arabic}
\quranayah[46][15]
\end{Arabic}}
\flushleft{\begin{hindi}
हमने मनुष्य को अपने माँ-बाप के साथ अच्छा व्यवहार करने की ताकीद की। उसकी माँ ने उसे (पेट में) तकलीफ़ के साथ उठाए रखा और उसे जना भी तकलीफ़ के साथ। और उसके गर्भ की अवस्था में रहने और दूध छुड़ाने की अवधि तीस माह है, यहाँ तक कि जब वह अपनी पूरी शक्ति को पहुँचा और चालीस वर्ष का हुआ तो उसने कहा, "ऐ मेरे रब! मुझे सम्भाल कि मैं तेरी उस अनुकम्पा के प्रति कृतज्ञता दिखाऊँ, जो तुने मुझपर और मेरे माँ-बाप पर की है। और यह कि मैं ऐसा अच्छा कर्म करूँ जो तुझे प्रिय हो और मेरे लिए मेरी संतति में भलाई रख दे। मैं तेरे आगे तौबा करता हूँ औऱ मैं मुस्लिम (आज्ञाकारी) हूँ।"
\end{hindi}}
\flushright{\begin{Arabic}
\quranayah[46][16]
\end{Arabic}}
\flushleft{\begin{hindi}
ऐसे ही लोग जिनसे हम अच्छे कर्म, जो उन्होंने किए होंगे, स्वीकार कर लेगें और उनकी बुराइयों को टाल जाएँगे। इस हाल में कि वे जन्नतवालों में होंगे, उस सच्चे वादे के अनुरूप जो उनसे किया जाता रहा है
\end{hindi}}
\flushright{\begin{Arabic}
\quranayah[46][17]
\end{Arabic}}
\flushleft{\begin{hindi}
किन्तु वह व्यक्ति जिसने अपने माँ-बाप से कहा, "धिक है तुम पर! क्या तुम मुझे डराते हो कि मैं (क़ब्र से) निकाला जाऊँगा, हालाँकि मुझसे पहले कितनी ही नस्लें गुज़र चुकी है?" और वे दोनों अल्लाह से फ़रियाद करते है - "अफ़सोस है तुमपर! मान जा। निस्संदेह अल्लाह का वादा सच्चा है।" किन्तु वह कहता है, "ये तो बस पहले के लोगों की कहानियाँ है।"
\end{hindi}}
\flushright{\begin{Arabic}
\quranayah[46][18]
\end{Arabic}}
\flushleft{\begin{hindi}
ऐसे ही लोग है जिनपर उन गिरोहों के साथ यातना की बात सत्यापित होकर रही जो जिन्नों और मनुष्यों में से उनसे पहले गुज़र चुके है। निश्चय ही वे घाटे में रहे
\end{hindi}}
\flushright{\begin{Arabic}
\quranayah[46][19]
\end{Arabic}}
\flushleft{\begin{hindi}
उनमें से प्रत्येक के दरजे उनके अपने किए हुए कर्मों के अनुसार होंगे (ताकि अल्लाह का वादा पूरा हो) और वह उन्हें उनके कर्मों का पूरा-पूरा बदला चुका दे और उनपर कदापि ज़ुल्म न होगा
\end{hindi}}
\flushright{\begin{Arabic}
\quranayah[46][20]
\end{Arabic}}
\flushleft{\begin{hindi}
और याद करो जिस दिन वे लोग जिन्होंने इनकार किया, आग के सामने पेश किए जाएँगे। (कहा जाएगा), "तुम अपने सांसारिक जीवन में अच्छी रुचिकर चीज़े नष्ट कर बैठे और उनका मज़ा ले चुके। अतः आज तुम्हे अपमानजनक यातना दी जाएगी, क्योंकि तुम धरती में बिना किसी हक़ के घमंड करते रहे और इसलिए कि तुम आज्ञा का उल्लंघन करते रहे।"
\end{hindi}}
\flushright{\begin{Arabic}
\quranayah[46][21]
\end{Arabic}}
\flushleft{\begin{hindi}
आद के भाई को याद करो, जबकि उसने अपनी क़ौम के लोगों को अहक़ाफ़ में सावधान किया, और उसके आगे और पीछे भी सावधान करनेवाले गुज़र चुके थे - कि, "अल्लाह के सिवा किसी की बन्दगी न करो। मुझे तुम्हारे बारे में एक बड़े दिन की यातना का भय है।"
\end{hindi}}
\flushright{\begin{Arabic}
\quranayah[46][22]
\end{Arabic}}
\flushleft{\begin{hindi}
उन्होंने कहा, "क्या तू हमारे पास इसलिए आया है कि झूठ बोलकर हमको अपने उपास्यों से विमुख कर दे? अच्छा, तो हमपर ले आ, जिसकी तू हमें धमकी देता है, यदि तू सच्चा है।"
\end{hindi}}
\flushright{\begin{Arabic}
\quranayah[46][23]
\end{Arabic}}
\flushleft{\begin{hindi}
उसने कहा, "ज्ञान तो अल्लाह ही के पास है (कि वह कब यातना लाएगा) । और मैं तो तुम्हें वह संदेश पहुँचा रहा हूँ जो मुझे देकर भेजा गया है। किन्तु मैं तुम्हें देख रहा हूँ कि तुम अज्ञानता से काम ले रहे हो।"
\end{hindi}}
\flushright{\begin{Arabic}
\quranayah[46][24]
\end{Arabic}}
\flushleft{\begin{hindi}
फिर जब उन्होंने उसे बादल के रूप में देखा, जिसका रुख़ उनकी घाटियों की ओर था, तो वे कहने लगे, "यह बादल है जो हमपर बरसनेवाला है!' "नहीं, बल्कि यह तो वही चीज़ है जिसके लिए तुमने जल्दी मचा रखी थी। - यह वायु है जिसमें दुखद यातना है
\end{hindi}}
\flushright{\begin{Arabic}
\quranayah[46][25]
\end{Arabic}}
\flushleft{\begin{hindi}
हर चीज़ को अपने रब के आदेश से विनष्ट- कर देगी।" अन्ततः वे ऐसे हो गए कि उनके रहने की जगहों के सिवा कुछ नज़र न आता था। अपराधी लोगों को हम इसी तरह बदला देते है
\end{hindi}}
\flushright{\begin{Arabic}
\quranayah[46][26]
\end{Arabic}}
\flushleft{\begin{hindi}
हमने उन्हें उन चीज़ों में जमाव और सामर्थ्य प्रदान की थी, जिनमें तुम्हें जमाव और सामर्थ्य नहीं प्रदान की। औऱ हमने उन्हें कान, आँखें और दिल दिए थे। किन्तु न तो उनके कान उनके कुछ काम आए औऱ न उनकी आँखे और न उनके दिल ही। क्योंकि वे अल्लाह की आयतों का इनकार करते थे और जिस चीज़ की वे हँसी उड़ाते थे, उसी ने उन्हें आ घेरा
\end{hindi}}
\flushright{\begin{Arabic}
\quranayah[46][27]
\end{Arabic}}
\flushleft{\begin{hindi}
हम तुम्हारे आस-पास की बस्तियों को विनष्ट कर चुके हैं, हालाँकि हमने तरह-तरह से आयते पेश की थीं, ताकि वे रुजू करें
\end{hindi}}
\flushright{\begin{Arabic}
\quranayah[46][28]
\end{Arabic}}
\flushleft{\begin{hindi}
फिर क्यों न उन बस्तियों ने उसकी सहायता की जिनको उन्होंने अपने और अल्लाह का बीच माध्यम ठहराकर सामीप्य प्राप्त करने के लिए उपास्य बना लिया था? बल्कि वे उनसे गुम हो गए, और यह था उनका मिथ्यारोपण और वह कुछ जो वे घड़ते थे
\end{hindi}}
\flushright{\begin{Arabic}
\quranayah[46][29]
\end{Arabic}}
\flushleft{\begin{hindi}
और याद करो (ऐ नबी) जब हमने कुछ जिन्नों को तुम्हारी ओर फेर दिया जो क़ुरआन सुनने लगे थे, तो जब वे वहाँ पहुँचे तो कहने लगे, "चुप हो जाओ!" फिर जब वह (क़ुरआन का पाठ) पूरा हुआ तो वे अपनी क़ौम की ओर सावधान करनेवाले होकर लौटे
\end{hindi}}
\flushright{\begin{Arabic}
\quranayah[46][30]
\end{Arabic}}
\flushleft{\begin{hindi}
उन्होंने कहा, "ऐ मेरी क़ौम के लोगो! हमने एक किताब सुनी है, जो मूसा के पश्चात अवतरित की गई है। उसकी पुष्टि में हैं जो उससे पहले से मौजूद है, सत्य की ओर और सीधे मार्ग की ओर मार्गदर्शन करती है
\end{hindi}}
\flushright{\begin{Arabic}
\quranayah[46][31]
\end{Arabic}}
\flushleft{\begin{hindi}
ऐ हमारी क़ौम के लोगो! अल्लाह के आमंत्रणकर्त्ता का आमंत्रण स्वीकार करो और उसपर ईमान लाओ। अल्लाह तुम्हें क्षमा करके गुनाहों से तुम्हें पाक कर देगा और दुखद यातना से तुम्हें बचाएगा
\end{hindi}}
\flushright{\begin{Arabic}
\quranayah[46][32]
\end{Arabic}}
\flushleft{\begin{hindi}
और जो कोई अल्लाह के आमंत्रणकर्त्ता का आमंत्रण स्वीकार नहीं करेगा तो वह धरती में क़ाबू से बच निकलनेवाला नहीं है और न अल्लाह से हटकर उसके संरक्षक होंगे। ऐसे ही लोग खुली गुमराही में हैं।"
\end{hindi}}
\flushright{\begin{Arabic}
\quranayah[46][33]
\end{Arabic}}
\flushleft{\begin{hindi}
क्या उन्होंने देखा नहीं कि जिस अल्लाह ने आकाशों और धरती को पैदा किया और उनके पैदा करने से थका नहीं; क्या ऐसा नहीं कि वह मुर्दों को जीवित कर दे? क्यों नहीं, निश्चय ही उसे हर चीज़ की सामर्थ्य प्राप्त है
\end{hindi}}
\flushright{\begin{Arabic}
\quranayah[46][34]
\end{Arabic}}
\flushleft{\begin{hindi}
और याद करो जिस दिन वे लोग, जिन्होंने इनकार किया, आग के सामने पेश किए जाएँगे, (कहा जाएगा) "क्या यह सत्य नहीं है?" वे कहेंगे, "नहीं, हमारे रब की क़सम!" वह कहेगा, "तो अब यातना का मज़ा चखो, उउस इनकार के बदले में जो तुम करते रहे थे।"
\end{hindi}}
\flushright{\begin{Arabic}
\quranayah[46][35]
\end{Arabic}}
\flushleft{\begin{hindi}
अतः धैर्य से काम लो, जिस प्रकार संकल्पवान रसूलों ने धैर्य से काम लिया। और उनके लिए जल्दी न करो। जिस दिन वे लोग उस चीज़ को देख लेंगे जिसका उनसे वादा किया जाता है, तो वे महसूस करेंगे कि जैसे वे बस दिन की एक घड़ी भर ही ठहरे थे। यह (संदेश) साफ़-साफ़ पहुँचा देना है। अब क्या अवज्ञाकारी लोगों के अतिरिक्त कोई और विनष्ट होगा?
\end{hindi}}
\chapter{Muhammad (Muhammad)}
\begin{Arabic}
\Huge{\centerline{\basmalah}}\end{Arabic}
\flushright{\begin{Arabic}
\quranayah[47][1]
\end{Arabic}}
\flushleft{\begin{hindi}
जिन लोगों ने इनकार किया और अल्लाह के मार्ग से रोका उनके कर्म उसने अकारथ कर दिए
\end{hindi}}
\flushright{\begin{Arabic}
\quranayah[47][2]
\end{Arabic}}
\flushleft{\begin{hindi}
रहे वे लोग जो ईमान लाए और उन्होंने अच्छे कर्म किए और उस चीज़ पर ईमान लाए जो मुहम्मद पर अवतरित किया गया - और वही सत्य है उनके रब की ओर से - उसने उसकी बुराइयाँ उनसे दूर कर दीं और उनका हाल ठीक कर दिया
\end{hindi}}
\flushright{\begin{Arabic}
\quranayah[47][3]
\end{Arabic}}
\flushleft{\begin{hindi}
यह इसलिए कि जिन लोगों ने इनकार किया उन्होंने असत्य का अनुसरण किया और यह कि जो लोग ईमान लाए उन्होंने सत्य का अनुसरण किया, जो उनके रब की ओर से है। इस प्रकार अल्लाह लोगों के लिए उनकी मिसालें बयान करता है
\end{hindi}}
\flushright{\begin{Arabic}
\quranayah[47][4]
\end{Arabic}}
\flushleft{\begin{hindi}
अतः जब इनकार करनेवालो से तुम्हारी मुठभेड़ हो तो (उनकी) गरदनें मारना है, यहाँ तक कि जब उन्हें अच्छी तरह कुचल दो तो बन्धनों में जकड़ो, फिर बाद में या तो एहसान करो या फ़िदया (अर्थ-दंड) का मामला करो, यहाँ तक कि युद्ध अपने बोझ उतारकर रख दे। यह भली-भाँति समझ लो, यदि अल्लाह चाहे तो स्वयं उनसे निपट ले। किन्तु (उसने या आदेश इसलिए दिया) ताकि तुम्हारी एक-दूसरे की परीक्षा ले। और जो लोग अल्लाह के मार्ग में मारे जाते है उनके कर्म वह कदापि अकारथ न करेगा
\end{hindi}}
\flushright{\begin{Arabic}
\quranayah[47][5]
\end{Arabic}}
\flushleft{\begin{hindi}
वह उनका मार्गदर्शन करेगा और उनका हाल ठीक कर देगा
\end{hindi}}
\flushright{\begin{Arabic}
\quranayah[47][6]
\end{Arabic}}
\flushleft{\begin{hindi}
और उन्हें जन्नत में दाख़िल करेगा, जिससे वह उन्हें परिचित करा चुका है
\end{hindi}}
\flushright{\begin{Arabic}
\quranayah[47][7]
\end{Arabic}}
\flushleft{\begin{hindi}
ऐ लोगों, जो ईमान लाए हो, यदि तुम अल्लाह की सहायता करोगे तो वह तुम्हारी सहायता करेगा और तुम्हारे क़दम जमा देगा
\end{hindi}}
\flushright{\begin{Arabic}
\quranayah[47][8]
\end{Arabic}}
\flushleft{\begin{hindi}
रहे वे लोग जिन्होंने इनकार किया, तो उनके लिए तबाही है। और उनके कर्मों को अल्लाह ने अकारथ कर दिया
\end{hindi}}
\flushright{\begin{Arabic}
\quranayah[47][9]
\end{Arabic}}
\flushleft{\begin{hindi}
यह इसलिए कि उन्होंने उस चीज़ को नापसन्द किया जिसे अल्लाह ने अवतरित किया, तो उसने उनके कर्म अकारथ कर दिए
\end{hindi}}
\flushright{\begin{Arabic}
\quranayah[47][10]
\end{Arabic}}
\flushleft{\begin{hindi}
क्या वे धरती में चले-फिरे नहीं कि देखते कि उन लोगों का कैसा परिणाम हुआ जो उनसे पहले गुज़र चुके है? अल्लाह ने उन्हें तहस-नहस कर दिया और इनकार करनेवालों के लिए ऐसे ही मामले होने है
\end{hindi}}
\flushright{\begin{Arabic}
\quranayah[47][11]
\end{Arabic}}
\flushleft{\begin{hindi}
यह इसलिए कि जो लोग ईमान लाए उनका संरक्षक अल्लाह है और यह कि इनकार करनेवालों को कोई संरक्षक नहीं
\end{hindi}}
\flushright{\begin{Arabic}
\quranayah[47][12]
\end{Arabic}}
\flushleft{\begin{hindi}
निश्चय ही अल्लाह उन लोगों को जो ईमान लाए और उन्होंने अच्छे कर्म किए ऐसे बागों में दाख़िल करेगा जिनके नीचे नहरें बह रही होंगी। रहे वे लोग जिन्होंने इनकार किया, वे कुछ दिनों का सुख भोग रहे है और खा रहे है, जिसे चौपाए खाते है। और आग उनका ठिकाना है
\end{hindi}}
\flushright{\begin{Arabic}
\quranayah[47][13]
\end{Arabic}}
\flushleft{\begin{hindi}
कितनी ही बस्तियाँ थी जो शक्ति में तुम्हारी उस बस्ती से, जिसने तुम्हें निकाल दिया, बढ़-चढ़कर थीं। हमने उन्हे विनष्टम कर दिया! फिर कोई उनका सहायक न हुआ
\end{hindi}}
\flushright{\begin{Arabic}
\quranayah[47][14]
\end{Arabic}}
\flushleft{\begin{hindi}
तो क्या जो व्यक्ति अपने रब की ओर से एक स्पष्ट प्रमाण पर हो वह उन लोगों जैसा हो सकता है, जिन्हें उनका बुरा कर्म ही सुहाना लगता हो और वे अपनी इच्छाओं के पीछे ही चलने लग गए हो?
\end{hindi}}
\flushright{\begin{Arabic}
\quranayah[47][15]
\end{Arabic}}
\flushleft{\begin{hindi}
उस जन्नत की शान, जिसका वादा डर रखनेवालों से किया गया है, यह है कि ऐसे पानी की नहरें होगी जो प्रदूषित नहीं होता। और ऐसे दूध की नहरें होंगी जिसके स्वाद में तनिक भी अन्तर न आया होगा, और ऐसे पेय की नहरें होंगी जो पीनेवालों के लिए मज़ा ही मज़ा होंगी, और साफ़-सुधरे शहद की नहरें भी होंगी। और उनके लिए वहाँ हर प्रकार के फल होंगे और क्षमा उनके अपने रब की ओर से - क्या वे उन जैसे हो सकते है, जो सदैव आग में रहनेवाले है और जिन्हें खौलता हुआ पानी पिलाया जाएगा, जो उनकी आँतों को टुकड़े-टुकड़े करके रख देगा?
\end{hindi}}
\flushright{\begin{Arabic}
\quranayah[47][16]
\end{Arabic}}
\flushleft{\begin{hindi}
और उनमें कुछ लोग ऐसे है जो तुम्हारी ओर कान लगाते है, यहाँ तक कि जब वे तुम्हारे पास से निकलते है तो उन लोगों से, जिन्हें ज्ञान प्रदान हुआ है कहते है, "उन्होंने अभी-अभी क्या कहा?" वही वे लोग है जिनके दिलों पर अल्लाह ने ठप्पा लगा दिया है और वे अपनी इच्छाओं के पीछे चले है
\end{hindi}}
\flushright{\begin{Arabic}
\quranayah[47][17]
\end{Arabic}}
\flushleft{\begin{hindi}
रहे वे लोग जिन्होंने सीधा रास्ता अपनाया, (अल्लाह ने) उनके मार्गदर्शन में अभिवृद्धि कर दी और उन्हें उनकी परहेज़गारी प्रदान की
\end{hindi}}
\flushright{\begin{Arabic}
\quranayah[47][18]
\end{Arabic}}
\flushleft{\begin{hindi}
अब क्या वे लोग बस उस घड़ी की प्रतीक्षा कर रहे है कि वह उनपर अचानक आ जाए? उसके लक्षण तो सामने आ चुके है, जब वह स्वयं भी उनपर आ जाएगी तो फिर उनके लिए होश में आने का अवसर कहाँ शेष रहेगा?
\end{hindi}}
\flushright{\begin{Arabic}
\quranayah[47][19]
\end{Arabic}}
\flushleft{\begin{hindi}
अतः जान रखों कि अल्लाह के अतिरिक्त कोई पूज्य-प्रभु नहीं। और अपने गुनाहों के लिए क्षमा-याचना करो और मोमिन पुरुषों और मोमिन स्त्रियों के लिए भी। अल्लाह तुम्हारी चलत-फिरत को भी जानता है और तुम्हारे ठिकाने को भी
\end{hindi}}
\flushright{\begin{Arabic}
\quranayah[47][20]
\end{Arabic}}
\flushleft{\begin{hindi}
जो लोग ईमान लाए वे कहते है, "कोई सूरा क्यों नहीं उतरी?" किन्तु जब एक पक्की सूरा अवतरित की जाती है, जिसमें युद्ध का उल्लेख होता है, तो तुम उन लोगों को देखते हो जिनके दिलों में रोग है कि वे तुम्हारी ओर इस प्रकार देखते है जैसे किसी पर मृत्यु की बेहोशी छा गई हो। तो अफ़सोस है उनके हाल पर!
\end{hindi}}
\flushright{\begin{Arabic}
\quranayah[47][21]
\end{Arabic}}
\flushleft{\begin{hindi}
उनके लिए उचित है आज्ञापालन और अच्छी-भली बात। फिर जब (युद्ध की) बात पक्की हो जाए (तो युद्ध करना चाहिए) तो यदि वे अल्लाह के लिए सच्चे साबित होते तो उनके लिए ही अच्छा होता
\end{hindi}}
\flushright{\begin{Arabic}
\quranayah[47][22]
\end{Arabic}}
\flushleft{\begin{hindi}
यदि तुम उल्टे फिर गए तो क्या तुम इससे निकट हो कि धरती में बिगाड़ पैदा करो और अपने नातों-रिश्तों को काट डालो?
\end{hindi}}
\flushright{\begin{Arabic}
\quranayah[47][23]
\end{Arabic}}
\flushleft{\begin{hindi}
ये वे लोग है जिनपर अल्लाह ने लानत की और उन्हें बहरा और उनकी आँखों को अन्धा कर दिया
\end{hindi}}
\flushright{\begin{Arabic}
\quranayah[47][24]
\end{Arabic}}
\flushleft{\begin{hindi}
तो क्या वे क़ुरआन में सोच-विचार नहीं करते या उनके दिलों पर ताले लगे हैं?
\end{hindi}}
\flushright{\begin{Arabic}
\quranayah[47][25]
\end{Arabic}}
\flushleft{\begin{hindi}
वे लोग जो पीठ-फेरकर पलट गए, इसके पश्चात कि उनपर मार्ग स्पष्ट॥ हो चुका था, उन्हें शैतान ने बहका दिया और उसने उन्हें ढील दे दी
\end{hindi}}
\flushright{\begin{Arabic}
\quranayah[47][26]
\end{Arabic}}
\flushleft{\begin{hindi}
यह इसलिए कि उन्होंने उन लोगों से, जिन्होंने उस चीज़ को नापसन्द किया जो कुछ अल्लाह ने उतारा है, कहा कि "हम कुछ मामलों में तुम्हारी बात मान लेंगे।" अल्लाह उनकी गुप्त बातों को भली-भाँति जानता है
\end{hindi}}
\flushright{\begin{Arabic}
\quranayah[47][27]
\end{Arabic}}
\flushleft{\begin{hindi}
फिर उस समय क्या हाल होगा जब फ़रिश्तें उनके चहरों और उनकी पीठों पर मारते हुए उनकी रूह क़ब्ज़ करेंगे?
\end{hindi}}
\flushright{\begin{Arabic}
\quranayah[47][28]
\end{Arabic}}
\flushleft{\begin{hindi}
यह इसलिए कि उन्होंने उस चीज़ का अनुसरण किया जो अल्लाह को अप्रसन्न करनेवाली थी और उन्होंने उसकी ख़ुशी को नापसन्द किया तो उसने उनके कर्मों को अकारथ कर दिया
\end{hindi}}
\flushright{\begin{Arabic}
\quranayah[47][29]
\end{Arabic}}
\flushleft{\begin{hindi}
(क्या अल्लाह से कोई चीज़ छिपी है) या जिन लोगों के दिलों में रोग है वे समझ बैठे है कि अल्लाह उनके द्वेषों को कदापि प्रकट न करेगा?
\end{hindi}}
\flushright{\begin{Arabic}
\quranayah[47][30]
\end{Arabic}}
\flushleft{\begin{hindi}
यदि हम चाहें तो उन्हें तुम्हें दिखा दें, फिर तुम उन्हें उनके लक्षणों से पहचान लो; किन्तु तुम उन्हें उनकी बातचीत के ढब से अवश्य पहचान लोगे। अल्लाह तो तुम्हारे कर्मों को जानता ही है
\end{hindi}}
\flushright{\begin{Arabic}
\quranayah[47][31]
\end{Arabic}}
\flushleft{\begin{hindi}
हम अवश्य तुम्हारी परीक्षा करेंगे, यहाँ तक कि हम तुममें से जो जिहाद करनेवाले है और जो दृढ़तापूर्वक जमे रहनेवाले है उनको जान ले और तुम्हारी हालतों को जाँच लें
\end{hindi}}
\flushright{\begin{Arabic}
\quranayah[47][32]
\end{Arabic}}
\flushleft{\begin{hindi}
जिन लोगों ने इसके पश्चात कि मार्ग उनपर स्पष्ट हो चुका था, इनकार किया और अल्लाह के मार्ग से रोका और रसूल का विरोध किया, वे अल्लाह को कदापि कोई हानि नहीं पहुँचा सकेंगे, बल्कि वही उनका सब किया-कराया उनकी जान को लागू कर देगा
\end{hindi}}
\flushright{\begin{Arabic}
\quranayah[47][33]
\end{Arabic}}
\flushleft{\begin{hindi}
ऐ ईमान लानेवालों! अल्लाह का आज्ञापालन करो और रसूल का आज्ञापालन करो और अपने कर्मों को विनष्ट न करो
\end{hindi}}
\flushright{\begin{Arabic}
\quranayah[47][34]
\end{Arabic}}
\flushleft{\begin{hindi}
निश्चय ही जिन लोगों ने इनकार किया और अल्लाह के मार्ग से रोका और इनकार करनेवाले ही रहकर मर गए, अल्लाह उन्हें कदापि क्षमा न करेगा
\end{hindi}}
\flushright{\begin{Arabic}
\quranayah[47][35]
\end{Arabic}}
\flushleft{\begin{hindi}
अतः ऐसा न हो कि तुम हिम्मत हार जाओ और सुलह का निमंत्रण देने लगो, जबकि तुम ही प्रभावी हो। अल्लाह तुम्हारे साथ है और वह तुम्हारे कर्मों (के फल) में तुम्हें कदापि हानि न पहुँचाएगा
\end{hindi}}
\flushright{\begin{Arabic}
\quranayah[47][36]
\end{Arabic}}
\flushleft{\begin{hindi}
सांसारिक जीवन तो बस एक खेल और तमाशा है। और यदि तुम ईमान लाओ और डर रखो तो वह तुम्हारे कर्मफल तुम्हें प्रदान करेगा और तुमसे धन नही माँगेगा। -
\end{hindi}}
\flushright{\begin{Arabic}
\quranayah[47][37]
\end{Arabic}}
\flushleft{\begin{hindi}
और यदि वह उनको तुमसे माँगे और समेटकर तुमसे माँगे तो तुम कंजूसी करोगे। और वह तुम्हारे द्वेष को निकाल बाहर कर देगा
\end{hindi}}
\flushright{\begin{Arabic}
\quranayah[47][38]
\end{Arabic}}
\flushleft{\begin{hindi}
सुनो! यह तुम्ही लोग हो कि तुम्हें आमंत्रण दिया जा रहा है कि "अल्लाह के मार्ग में ख़र्च करो।" फिर तुमसे कुछ लोग है जो कंजूसी करते है। हालाँकि जो कंजूसी करता है वह वास्तव में अपने आप ही से कंजूसी करता है। अल्लाह तो निस्पृह है, तुम्हीं मुहताज हो। और यदि तुम फिर जाओ तो वह तुम्हारी जगह अन्य लोगों को ले आएगा; फिर वे तुम जैसे न होंगे
\end{hindi}}
\chapter{Al-Fath (The Victory)}
\begin{Arabic}
\Huge{\centerline{\basmalah}}\end{Arabic}
\flushright{\begin{Arabic}
\quranayah[48][1]
\end{Arabic}}
\flushleft{\begin{hindi}
निश्चय ही हमने तुम्हारे लिए एक खुली विजय प्रकट की,
\end{hindi}}
\flushright{\begin{Arabic}
\quranayah[48][2]
\end{Arabic}}
\flushleft{\begin{hindi}
ताकि अल्लाह तुम्हारे अगले और पिछले गुनाहों को क्षमा कर दे और तुमपर अपनी अनुकम्पा पूर्ण कर दे और तुम्हें सीधे मार्ग पर चलाए,
\end{hindi}}
\flushright{\begin{Arabic}
\quranayah[48][3]
\end{Arabic}}
\flushleft{\begin{hindi}
और अल्लाह तुम्हें प्रभावकारी सहायता प्रदान करे
\end{hindi}}
\flushright{\begin{Arabic}
\quranayah[48][4]
\end{Arabic}}
\flushleft{\begin{hindi}
वहीं है जिसने ईमानवालों के दिलों में सकीना (प्रशान्ति) उतारी, ताकि अपने ईमान के साथ वे और ईमान की अभिवृद्धि करें - आकाशों और धरती की सभी सेनाएँ अल्लाह ही की है, और अल्लाह सर्वज्ञ, तत्वदर्शी है। -
\end{hindi}}
\flushright{\begin{Arabic}
\quranayah[48][5]
\end{Arabic}}
\flushleft{\begin{hindi}
ताकि वह मोमिन पुरुषों औप मोमिन स्त्रियों को ऐसे बाग़ों में दाख़िल करे जिनके नीचे नहरें बहती होंगी कि वे उनमें सदैव रहें और उनसे उनकी बुराईयाँ दूर कर दे - यह अल्लाह के यहाँ बड़ी सफलता है। -
\end{hindi}}
\flushright{\begin{Arabic}
\quranayah[48][6]
\end{Arabic}}
\flushleft{\begin{hindi}
और कपटाचारी पुरुषों और कपटाचारी स्त्रियों और बहुदेववादी पुरुषों और बहुदेववादी स्त्रियों को, जो अल्लाह के बारे में बुरा गुमान रखते है, यातना दे। उन्हीं पर बुराई की गर्दिश है। उनपर अल्लाह का क्रोध हुआ और उसने उनपर लानत की, और उसने उनके लिए जहन्नम तैयार कर रखा है, और वह अत्यन्त बुरा ठिकाना है!
\end{hindi}}
\flushright{\begin{Arabic}
\quranayah[48][7]
\end{Arabic}}
\flushleft{\begin{hindi}
आकाशों और धरती की सब सेनाएँ अल्लाह ही की है। अल्लाह प्रभुत्वशाली, अत्यन्त तत्वदर्शी है
\end{hindi}}
\flushright{\begin{Arabic}
\quranayah[48][8]
\end{Arabic}}
\flushleft{\begin{hindi}
निश्चय ही हमने तुम्हें गवाही देनेवाला और शुभ सूचना देनेवाला और सचेतकर्त्ता बनाकर भेजा,
\end{hindi}}
\flushright{\begin{Arabic}
\quranayah[48][9]
\end{Arabic}}
\flushleft{\begin{hindi}
ताकि तुम अल्लाह और उसके रसूल पर ईमान लाओ, उसे सहायता पहुँचाओ और उसका आदर करो, और प्रातःकाल और संध्या समय उसकी तसबीह करते रहो
\end{hindi}}
\flushright{\begin{Arabic}
\quranayah[48][10]
\end{Arabic}}
\flushleft{\begin{hindi}
(ऐ नबी) वे लोग जो तुमसे बैअत करते है वे तो वास्तव में अल्लाह ही से बैअत करते है। उनके हाथों के ऊपर अल्लाह का हाथ होता है। फिर जिस किसी ने वचन भंग किया तो वह वचन भंग करके उसका बवाल अपने ही सिर लेता है, किन्तु जिसने उस प्रतिज्ञा को पूरा किया जो उसने अल्लाह से की है तो उसे वह बड़ा बदला प्रदान करेगा
\end{hindi}}
\flushright{\begin{Arabic}
\quranayah[48][11]
\end{Arabic}}
\flushleft{\begin{hindi}
जो बद्‌दू पीछे रह गए थे, वे अब तुमसे कहेगे, "हमारे माल और हमारे घरवालों ने हमें व्यस्त कर रखा था; तो आप हमारे लिए क्षमा की प्रार्थना कीजिए।" वे अपनी ज़बानों से वे बातें कहते है जो उनके दिलों में नहीं। कहना कि, "कौन है जो अल्लाह के मुक़ाबले में तुम्हारे किए किसी चीज़ का अधिकार रखता है, यदि वह तुम्हें कोई हानि पहुँचानी चाहे या वह तुम्हें कोई लाभ पहुँचाने का इरादा करे? बल्कि जो कुछ तुम करते हो अल्लाह उसकी ख़बर रखता है। -
\end{hindi}}
\flushright{\begin{Arabic}
\quranayah[48][12]
\end{Arabic}}
\flushleft{\begin{hindi}
"नहीं, बल्कि तुमने यह समझा कि रसूल और ईमानवाले अपने घरवालों की ओर लौटकर कभी न आएँगे और यह तुम्हारे दिलों को अच्छा लगा। तुमने तो बहुत बुरे गुमान किए और तुम्हीं लोग हुए तबाही में पड़नेवाले।"
\end{hindi}}
\flushright{\begin{Arabic}
\quranayah[48][13]
\end{Arabic}}
\flushleft{\begin{hindi}
और अल्लाह और उसके रसूल पर ईमान न लाया, तो हमने भी इनकार करनेवालों के लिए भड़कती आग तैयार कर रखी है
\end{hindi}}
\flushright{\begin{Arabic}
\quranayah[48][14]
\end{Arabic}}
\flushleft{\begin{hindi}
अल्लाह ही की है आकाशों और धरती की बादशाही। वह जिसे चाहे क्षमा करे और जिसे चाहे यातना दे। और अल्लाह बड़ा क्षमाशील, अत्यन्त दयावान है
\end{hindi}}
\flushright{\begin{Arabic}
\quranayah[48][15]
\end{Arabic}}
\flushleft{\begin{hindi}
जब तुम ग़नीमतों को प्राप्त करने के लिए उनकी ओर चलोगे तो पीछे रहनेवाले कहेंगे, "हमें भी अनुमति दी जाए कि हम तुम्हारे साथ चले।" वे चाहते है कि अल्लाह का कथन को बदल दे। कह देना, "तुम हमारे साथ कदापि नहीं चल सकते। अल्लाह ने पहले ही ऐसा कह दिया है।" इसपर वे कहेंगे, "नहीं, बल्कि तुम हमसे ईर्ष्या कर रहे हो।" नहीं, बल्कि वे लोग समझते थोड़े ही है
\end{hindi}}
\flushright{\begin{Arabic}
\quranayah[48][16]
\end{Arabic}}
\flushleft{\begin{hindi}
पीछे रह जानेवाले बद्‌दूओं से कहना, "शीघ्र ही तुम्हें ऐसे लोगों की ओर बुलाया जाएगा जो बड़े युद्धवीर है कि तुम उनसे लड़ो या वे आज्ञाकारी हो जाएँ। तो यदि तुम आज्ञाकारी हो जाएँ। तो यदि तुम आज्ञापालन करोगे तो अल्लाह तुम्हें अच्छा बदला प्रदान करेगा। किन्तु यदि तुम फिर गए, जैसे पहले फिर गए थे, तो वह तुम्हें दुखद यातना देगा।"
\end{hindi}}
\flushright{\begin{Arabic}
\quranayah[48][17]
\end{Arabic}}
\flushleft{\begin{hindi}
न अन्धे के लिए कोई हरज है, न लँगडे के लिए कोई हरज है और न बीमार के लिए कोई हरज है। जो भी अल्लाह और उसके रसूल की आज्ञा का पालन करेगा, उसे वह ऐसे बाग़ों में दाख़िल करके, जिनके नीचे नहरे बह रही होगी, किन्तु जो मुँह फेरेगा उसे वह दुखद यातना देगा
\end{hindi}}
\flushright{\begin{Arabic}
\quranayah[48][18]
\end{Arabic}}
\flushleft{\begin{hindi}
निश्चय ही अल्लाह मोमिनों से प्रसन्न हुआ, जब वे वृक्ष के नीचे तुमसे बैअत कर रहे थे। उसने जान लिया जो कुछ उनके दिलों में था। अतः उनपर उसने सकीना (प्रशान्ति) उतारी और बदले में उन्हें मिलनेवाली विजय निश्चित कर दी;
\end{hindi}}
\flushright{\begin{Arabic}
\quranayah[48][19]
\end{Arabic}}
\flushleft{\begin{hindi}
और बहुत-सी ग़नीमतें भी, जिन्हें वे प्राप्त करेंगे। अल्लाह प्रभुत्वशाली, तत्वदर्शी है
\end{hindi}}
\flushright{\begin{Arabic}
\quranayah[48][20]
\end{Arabic}}
\flushleft{\begin{hindi}
अल्लाह ने तुमसे बहुत-सी गंनीमतों का वादा किया हैं, जिन्हें तुम प्राप्त करोगे। यह विजय तो उसने तुम्हें तात्कालिक रूप से निश्चित कर दी। और लोगों के हाथ तुमसे रोक दिए (कि वे तुमपर आक्रमण करने का साहस न कर सकें) और ताकि ईमानवालों के लिए एक निशानी हो। और वह सीधे मार्ग की ओर तुम्हारा मार्गदर्शन करे
\end{hindi}}
\flushright{\begin{Arabic}
\quranayah[48][21]
\end{Arabic}}
\flushleft{\begin{hindi}
इसके अतिरिक्त दूसरी और विजय का भी वादा है, जिसकी सामर्थ्य अभी तुम्हे प्राप्त नहीं, उन्हें अल्लाह ने घेर रखा है। अल्लाह को हर चीज़ की सामर्थ्य प्राप्त है
\end{hindi}}
\flushright{\begin{Arabic}
\quranayah[48][22]
\end{Arabic}}
\flushleft{\begin{hindi}
यदि (मक्का के) इनकार करनेवाले तुमसे लड़ते तो अवश्य ही पीठ फेर जाते। फिर यह भी कि वे न तो कोई संरक्षक पाएँगे और न कोई सहायक
\end{hindi}}
\flushright{\begin{Arabic}
\quranayah[48][23]
\end{Arabic}}
\flushleft{\begin{hindi}
यह अल्लाह की उस रीति के अनुकूल है जो पहले से चली आई है, और तुम अल्लाह की रीति में कदापि कोई परिवर्तन न पाओगे
\end{hindi}}
\flushright{\begin{Arabic}
\quranayah[48][24]
\end{Arabic}}
\flushleft{\begin{hindi}
वही है जिसने उसके हाथ तुमसे और तुम्हारे हाथ उनसे मक्के की घाटी में रोक दिए इसके पश्चात कि वह तुम्हें उनपर प्रभावी कर चुका था। अल्लाह उसे देख रहा था जो कुछ तुम कर रहे थे
\end{hindi}}
\flushright{\begin{Arabic}
\quranayah[48][25]
\end{Arabic}}
\flushleft{\begin{hindi}
ये वही लोग तो है जिन्होंने इनकार किया और तुम्हें मस्जिदे हराम (काबा) से रोक दिया और क़ुरबानी के बँधे हुए जानवरों को भी इससे रोके रखा कि वे अपने ठिकाने पर पहुँचे। यदि यह ख़याल न होता कि बहुत-से मोमिन पुरुष और मोमिन स्त्रियाँ (मक्का में) मौजूद है, जिन्हें तुम नहीं जानते, उन्हें कुचल दोगे, फिर उनके सिलसिले में अनजाने तुमपर इल्ज़ाम आएगा (तो युद्ध की अनुमति दे दी जाती, अनुमति इसलिए नहीं दी गई) ताकि अल्लाह जिसे चाहे अपनी दयालुता में दाख़िल कर ले। यदि वे ईमानवाले अलग हो गए होते तो उनमें से जिन लोगों ने इनकार किया उनको हम अवश्य दुखद यातना देते
\end{hindi}}
\flushright{\begin{Arabic}
\quranayah[48][26]
\end{Arabic}}
\flushleft{\begin{hindi}
याद करो जब इनकार करनेवाले लोगों ने अपने दिलों में हठ को जगह दी, अज्ञानपूर्ण हठ को; तो अल्लाह ने अपने रसूल पर और ईमानवालो पर सकीना (प्रशान्ति) उतारी और उन्हें परहेज़गारी (धर्मपरायणता) की बात का पाबन्द रखा। वे इसके ज़्यादा हक़दार और इसके योग्य भी थे। अल्लाह तो हर चीज़ जानता है
\end{hindi}}
\flushright{\begin{Arabic}
\quranayah[48][27]
\end{Arabic}}
\flushleft{\begin{hindi}
निश्चय ही अल्लाह ने अपने रसूल को हक़ के साथ सच्चा स्वप्न दिखाया, "यदि अल्लाह ने चाहा तो तुम अवश्य मस्जिदे हराम (काबा) में प्रवेश करोगे बेखटके, अपने सिर के बाल मुड़ाते और कतरवाते हुए, तुम्हें कोई भय न होगा।" हुआ यह कि उसने वह बात जान ली जो तुमने नहीं जानी। अतः इससे पहले उसने शीघ्र प्राप्त होनेवाली विजय तुम्हारे लिए निश्चिंत कर दी
\end{hindi}}
\flushright{\begin{Arabic}
\quranayah[48][28]
\end{Arabic}}
\flushleft{\begin{hindi}
वही है जिसने अपने रसूल को मार्गदर्शन और सत्यधर्म के साथ भेजा, ताकि उसे पूरे के पूरे धर्म पर प्रभुत्व प्रदान करे और गवाह की हैसियत से अल्लाह काफ़ी है
\end{hindi}}
\flushright{\begin{Arabic}
\quranayah[48][29]
\end{Arabic}}
\flushleft{\begin{hindi}
अल्लाह के रसूल मुहम्मद और जो लोग उनके साथ हैं, वे इनकार करनेवालों पर भारी हैं, आपस में दयालु है। तुम उन्हें रुकू में, सजदे में, अल्लाह का उदार अनुग्रह और उसकी प्रसन्नता चाहते हुए देखोगे। वे अपने चहरों से पहचाने जाते हैं जिनपर सजदों का प्रभाव है। यही उनकी विशेषता तौरात में और उनकी विशेषता इंजील में उस खेती की तरह उल्लिखित है जिसने अपना अंकुर निकाला; फिर उसे शक्ति पहुँचाई तो वह मोटा हुआ और वह अपने तने पर सीधा खड़ा हो गया। खेती करनेवालों को भा रहा है, ताकि उनसे इनकार करनेवालों का भी जी जलाए। उनमें से जो लोग ईमान लाए और उन्होंने अच्छे कर्म किए उनसे अल्लाह ने क्षमा और बदले का वादा किया है
\end{hindi}}
\chapter{Al-Hujurat (The Apartments)}
\begin{Arabic}
\Huge{\centerline{\basmalah}}\end{Arabic}
\flushright{\begin{Arabic}
\quranayah[49][1]
\end{Arabic}}
\flushleft{\begin{hindi}
ऐ ईमानवालो! अल्लाह और उसके रसूल से आगे न बढो और अल्लाह का डर रखो। निश्चय ही अल्लाह सुनता, जानता है
\end{hindi}}
\flushright{\begin{Arabic}
\quranayah[49][2]
\end{Arabic}}
\flushleft{\begin{hindi}
ऐ लोगो, जो ईमान लाए हो! तुम अपनी आवाज़ों को नबी की आवाज़ से ऊँची न करो। और जिस तरह तुम आपस में एक-दूसरे से ज़ोर से बोलते हो, उससे ऊँची आवाज़ में बात न करो। कहीं ऐसा न हो कि तुम्हारे कर्म अकारथ हो जाएँ और तुम्हें ख़बर भी न हो
\end{hindi}}
\flushright{\begin{Arabic}
\quranayah[49][3]
\end{Arabic}}
\flushleft{\begin{hindi}
वे लोग जो अल्लाह के रसूल के समक्ष अपनी आवाज़ों को दबी रखते है, वही लोग है जिनके दिलों को अल्लाह ने परहेज़गारी के लिए जाँचकर चुन लिया है। उनके लिए क्षमा और बड़ा बदला है
\end{hindi}}
\flushright{\begin{Arabic}
\quranayah[49][4]
\end{Arabic}}
\flushleft{\begin{hindi}
जो लोग (ऐ नबी) तुम्हें कमरों के बाहर से पुकारते है उनमें से अधिकतर बुद्धि से काम नहीं लेते
\end{hindi}}
\flushright{\begin{Arabic}
\quranayah[49][5]
\end{Arabic}}
\flushleft{\begin{hindi}
यदि वे धैर्य से काम लेते यहाँ तक कि तुम स्वयं निकलकर उनके पास आ जाते तो यह उनके लिए अच्छा होता। किन्तु अल्लाह बड़ा क्षमाशील, अत्यन्त दयावान है
\end{hindi}}
\flushright{\begin{Arabic}
\quranayah[49][6]
\end{Arabic}}
\flushleft{\begin{hindi}
ऐ लोगों, जो ईमान लाए हो! यदि कोई अवज्ञाकारी तुम्हारे पास कोई ख़बर लेकर आए तो उसकी छानबीन कर लिया करो। कहीं ऐसा न हो कि तुम किसी गिरोह को अनजाने में तकलीफ़ और नुक़सान पहुँचा बैठो, फिर अपने किए पर पछताओ
\end{hindi}}
\flushright{\begin{Arabic}
\quranayah[49][7]
\end{Arabic}}
\flushleft{\begin{hindi}
जान लो कि तुम्हारे बीच अल्लाह का रसूल मौजूद है। बहुत-से मामलों में यदि वह तुम्हारी बात मान ले तो तुम कठिनाई में पड़ जाओ। किन्तु अल्लाह ने तुम्हारे लिए ईमान को प्रिय बना दिया और उसे तुम्हारे दिलों में सुन्दरता दे दी और इनकार, उल्लंघन और अवज्ञा को तुम्हारे लिए बहुत अप्रिय बना दिया।
\end{hindi}}
\flushright{\begin{Arabic}
\quranayah[49][8]
\end{Arabic}}
\flushleft{\begin{hindi}
ऐसे ही लोग अल्लाह के उदार अनुग्रह और अनुकम्पा से सूझबूझवाले है। और अल्लाह सब कुछ जाननेवाला, तत्वदर्शी है
\end{hindi}}
\flushright{\begin{Arabic}
\quranayah[49][9]
\end{Arabic}}
\flushleft{\begin{hindi}
यदि मोमिनों में से दो गिरोह आपस में लड़ पड़े तो उनके बीच सुलह करा दो। फिर यदि उनमें से एक गिरोह दूसरे पर ज़्यादती करे, तो जो गिरोह ज़्यादती कर रहा हो उससे लड़ो, यहाँ तक कि वह अल्लाह के आदेश की ओर पलट आए। फिर यदि वह पलट आए तो उनके बीच न्याय के साथ सुलह करा दो, और इनसाफ़ करो। निश्चय ही अल्लाह इनसाफ़ करनेवालों को पसन्द करता है
\end{hindi}}
\flushright{\begin{Arabic}
\quranayah[49][10]
\end{Arabic}}
\flushleft{\begin{hindi}
मोमिन तो भाई-भाई ही है। अतः अपने दो भाईयो के बीच सुलह करा दो और अल्लाह का डर रखो, ताकि तुमपर दया की जाए
\end{hindi}}
\flushright{\begin{Arabic}
\quranayah[49][11]
\end{Arabic}}
\flushleft{\begin{hindi}
ऐ लोगो, जो ईमान लाए हो! न पुरुषों का कोई गिरोह दूसरे पुरुषों की हँसी उड़ाए, सम्भव है वे उनसे अच्छे हों और न स्त्रियाँ स्त्रियों की हँसी उड़ाए, सम्भव है वे उनसे अच्छी हों, और न अपनों पर ताने कसो और न आपस में एक-दूसरे को बुरी उपाधियों से पुकारो। ईमान के पश्चात अवज्ञाकारी का नाम जुडना बहुत ही बुरा है। और जो व्यक्ति बाज़ न आए, तो ऐसे ही व्यक्ति ज़ालिम है
\end{hindi}}
\flushright{\begin{Arabic}
\quranayah[49][12]
\end{Arabic}}
\flushleft{\begin{hindi}
ऐ ईमान लानेवालो! बहुत से गुमानों से बचो, क्योंकि कतिपय गुमान गुनाह होते है। और न टोह में पड़ो और न तुममें से कोई किसी की पीठ पीछे निन्दा करे - क्या तुममें से कोई इसको पसन्द करेगा कि वह मरे हुए भाई का मांस खाए? वह तो तुम्हें अप्रिय होगी ही। - और अल्लाह का डर रखो। निश्चय ही अल्लाह तौबा क़बूल करनेवाला, अत्यन्त दयावान है
\end{hindi}}
\flushright{\begin{Arabic}
\quranayah[49][13]
\end{Arabic}}
\flushleft{\begin{hindi}
ऐ लोगो! हमनें तुम्हें एक पुरुष और एक स्त्री से पैदा किया और तुम्हें बिरादरियों और क़बिलों का रूप दिया, ताकि तुम एक-दूसरे को पहचानो। वास्तव में अल्लाह के यहाँ तुममें सबसे अधिक प्रतिष्ठित वह है, जो तुममे सबसे अधिक डर रखता है। निश्चय ही अल्लाह सबकुछ जाननेवाला, ख़बर रखनेवाला है
\end{hindi}}
\flushright{\begin{Arabic}
\quranayah[49][14]
\end{Arabic}}
\flushleft{\begin{hindi}
बद्‌दुओं ने कहा कि, "हम ईमान लाए।" कह दो, "तुम ईमान नहीं लाए। किन्तु यूँ कहो, 'हम तो आज्ञाकारी हुए' ईमान तो अभी तुम्हारे दिलों में दाख़िल ही नहीं हुआ। यदि तुम अल्लाह और उसके रसूल की आज्ञा का पालन करो तो वह तुम्हारे कर्मों में से तुम्हारे लिए कुछ कम न करेगा। निश्चय ही अल्लाह बड़ा क्षमाशील, अत्यन्त दयावान है।"
\end{hindi}}
\flushright{\begin{Arabic}
\quranayah[49][15]
\end{Arabic}}
\flushleft{\begin{hindi}
मोमिन तो बस वही लोग है जो अल्लाह और उसके रसूल पर ईमान लाए, फिर उन्होंने कोई सन्देह नहीं किया और अपने मालों और अपनी जानों से अल्लाह के मार्ग में जिहाद किया। वही लोग सच्चे है
\end{hindi}}
\flushright{\begin{Arabic}
\quranayah[49][16]
\end{Arabic}}
\flushleft{\begin{hindi}
कहो, "क्या तुम अल्लाह को अपने धर्म की सूचना दे रहे हो। हालाँकि जो कुछ आकाशों में और जो कुछ धरती में है, अल्लाह सब जानता है? अल्लाह को हर चीज़ का ज्ञान है।"
\end{hindi}}
\flushright{\begin{Arabic}
\quranayah[49][17]
\end{Arabic}}
\flushleft{\begin{hindi}
वे तुमपर एहसान जताते है कि उन्होंने इस्लाम क़बूल कर लिया। कह दो, "मुझ पर अपने इस्लाम का एहसान न रखो, बल्कि यदि तुम सच्चे हो तो अल्लाह ही तुमपर एहसान रखता है कि उसने तुम्हें ईमान की राह दिखाई।-
\end{hindi}}
\flushright{\begin{Arabic}
\quranayah[49][18]
\end{Arabic}}
\flushleft{\begin{hindi}
"निश्चय ही अल्लाह आकाशों और धरती के अदृष्ट को जानता है। और अल्लाह देख रहा है जो कुछ तुम करते हो।"
\end{hindi}}
\chapter{Qaf (Qaf)}
\begin{Arabic}
\Huge{\centerline{\basmalah}}\end{Arabic}
\flushright{\begin{Arabic}
\quranayah[50][1]
\end{Arabic}}
\flushleft{\begin{hindi}
क़ाफ़॰; गवाह है क़ुरआन मजीद! -
\end{hindi}}
\flushright{\begin{Arabic}
\quranayah[50][2]
\end{Arabic}}
\flushleft{\begin{hindi}
बल्कि उन्हें तो इस बात पर आश्चर्य हुआ कि उनके पास उन्हीं में से एक सावधान करनेवाला आ गया। फिर इनकार करनेवाले कहने लगे, "यह तो आश्चर्य की बात है
\end{hindi}}
\flushright{\begin{Arabic}
\quranayah[50][3]
\end{Arabic}}
\flushleft{\begin{hindi}
"क्या जब हम मर जाएँगे और मिट्टी हो जाएँगे (तो फिर हम जीवि होकर पलटेंगे)? यह पलटना तो बहुत दूर की बात है!"
\end{hindi}}
\flushright{\begin{Arabic}
\quranayah[50][4]
\end{Arabic}}
\flushleft{\begin{hindi}
हम जानते है कि धरती उनमें जो कुछ कमी करती है और हमारे पास सुरक्षित रखनेवाली एक किताब भी है
\end{hindi}}
\flushright{\begin{Arabic}
\quranayah[50][5]
\end{Arabic}}
\flushleft{\begin{hindi}
बल्कि उन्होंने सत्य को झुठला दिया जब वह उनके पास आया। अतः वे एक उलझन भी बात में पड़े हुए है
\end{hindi}}
\flushright{\begin{Arabic}
\quranayah[50][6]
\end{Arabic}}
\flushleft{\begin{hindi}
अच्छा तो क्या उन्होंने अपने ऊपर आकाश को नहीं देखा, हमने उसे कैसा बनाया और उसे सजाया। और उसमें कोई दरार नहीं
\end{hindi}}
\flushright{\begin{Arabic}
\quranayah[50][7]
\end{Arabic}}
\flushleft{\begin{hindi}
और धरती को हमने फैलाया और उसमे अटल पहाड़ डाल दिए। और हमने उसमें हर प्रकार की सुन्दर चीज़े उगाई,
\end{hindi}}
\flushright{\begin{Arabic}
\quranayah[50][8]
\end{Arabic}}
\flushleft{\begin{hindi}
आँखें खोलने और याद दिलाने के उद्देश्य से, हर बन्दे के लिए जो रुजू करनेवाला हो
\end{hindi}}
\flushright{\begin{Arabic}
\quranayah[50][9]
\end{Arabic}}
\flushleft{\begin{hindi}
और हमने आकाश से बरकतवाला पानी उतारा, फिर उससे बाग़ और फ़सल के अनाज।
\end{hindi}}
\flushright{\begin{Arabic}
\quranayah[50][10]
\end{Arabic}}
\flushleft{\begin{hindi}
और ऊँचे-ऊँचे खजूर के वृक्ष उगाए जिनके गुच्छे तह पर तह होते है,
\end{hindi}}
\flushright{\begin{Arabic}
\quranayah[50][11]
\end{Arabic}}
\flushleft{\begin{hindi}
बन्दों की रोजी के लिए। और हमने उस (पानी) के द्वारा निर्जीव धरती को जीवन प्रदान किया। इसी प्रकार निकलना भी हैं
\end{hindi}}
\flushright{\begin{Arabic}
\quranayah[50][12]
\end{Arabic}}
\flushleft{\begin{hindi}
उनसे पहले नूह की क़ौम, 'अर्-रस' वाले, समूद,
\end{hindi}}
\flushright{\begin{Arabic}
\quranayah[50][13]
\end{Arabic}}
\flushleft{\begin{hindi}
आद, फ़िरऔन , लूत के भाई,
\end{hindi}}
\flushright{\begin{Arabic}
\quranayah[50][14]
\end{Arabic}}
\flushleft{\begin{hindi}
'अल-ऐका' वाले और तुब्बा के लोग भी झुठला चुके है। प्रत्येक ने रसूलों को झुठलाया। अन्ततः मेरी धमकी सत्यापित होकर रही
\end{hindi}}
\flushright{\begin{Arabic}
\quranayah[50][15]
\end{Arabic}}
\flushleft{\begin{hindi}
क्या हम पहली बार पैदा करने से असमर्थ रहे? नहीं, बल्कि वे एक नई सृष्टि के विषय में सन्देह में पड़े है
\end{hindi}}
\flushright{\begin{Arabic}
\quranayah[50][16]
\end{Arabic}}
\flushleft{\begin{hindi}
हमने मनुष्य को पैदा किया है और हम जानते है जो बातें उसके जी में आती है। और हम उससे उसकी गरदन की रग से भी अधिक निकट है
\end{hindi}}
\flushright{\begin{Arabic}
\quranayah[50][17]
\end{Arabic}}
\flushleft{\begin{hindi}
जब दो प्राप्त करनेवाले (फ़रिशते) प्राप्त कर रहे होते है, दाएँ से और बाएँ से वे लगे बैठे होते है
\end{hindi}}
\flushright{\begin{Arabic}
\quranayah[50][18]
\end{Arabic}}
\flushleft{\begin{hindi}
कोई बात उसने कही नहीं कि उसके पास एक निरीक्षक तैयार रहता है
\end{hindi}}
\flushright{\begin{Arabic}
\quranayah[50][19]
\end{Arabic}}
\flushleft{\begin{hindi}
और मौत की बेहोशी ले आई अविश्व नीय चीज़! यही वह चीज़ है जिससे तू कतराता था
\end{hindi}}
\flushright{\begin{Arabic}
\quranayah[50][20]
\end{Arabic}}
\flushleft{\begin{hindi}
और नरसिंघा फूँक दिया गया। यही है वह दिन जिसकी धमकी दी गई थी
\end{hindi}}
\flushright{\begin{Arabic}
\quranayah[50][21]
\end{Arabic}}
\flushleft{\begin{hindi}
हर व्यक्ति इस दशा में आ गया कि उसके साथ एक लानेवाला है और एक गवाही देनेवाला
\end{hindi}}
\flushright{\begin{Arabic}
\quranayah[50][22]
\end{Arabic}}
\flushleft{\begin{hindi}
तू इस चीज़ की ओर से ग़फ़लत में था। अब हमने तुझसे तेरा परदा हटा दिया, तो आज तेरी निगाह बड़ी तेज़ है
\end{hindi}}
\flushright{\begin{Arabic}
\quranayah[50][23]
\end{Arabic}}
\flushleft{\begin{hindi}
उसके साथी ने कहा, "यह है (तेरी सजा)! मेरे पास कुछ (सहायता के लिए) मौजूद नहीं।"
\end{hindi}}
\flushright{\begin{Arabic}
\quranayah[50][24]
\end{Arabic}}
\flushleft{\begin{hindi}
"डाल दो, डाल दो, जहन्नम में! हर अकृतज्ञ द्वेष रखने वाले,
\end{hindi}}
\flushright{\begin{Arabic}
\quranayah[50][25]
\end{Arabic}}
\flushleft{\begin{hindi}
भलाई से रोकनेवाले, सीमा का अतिक्रमण करनेवाले, सन्देहग्रस्त को
\end{hindi}}
\flushright{\begin{Arabic}
\quranayah[50][26]
\end{Arabic}}
\flushleft{\begin{hindi}
जिसने अल्लाह के साथ किसी दूसरे को पूज्य-प्रभु ठहराया। तो डाल दो उसे कठोर यातना में।"
\end{hindi}}
\flushright{\begin{Arabic}
\quranayah[50][27]
\end{Arabic}}
\flushleft{\begin{hindi}
उसका साथी बोला, "ऐ हमारे रब! मैंने उसे सरकश नहीं बनाया, बल्कि वह स्वयं ही परले दरजे की गुमराही में था।"
\end{hindi}}
\flushright{\begin{Arabic}
\quranayah[50][28]
\end{Arabic}}
\flushleft{\begin{hindi}
कहा, "मेरे सामने मत झगड़ो। मैं तो तुम्हें पहले ही अपनी धमकी से सावधान कर चुका था। -
\end{hindi}}
\flushright{\begin{Arabic}
\quranayah[50][29]
\end{Arabic}}
\flushleft{\begin{hindi}
"मेरे यहाँ बात बदला नहीं करती और न मैं अपने बन्दों पर तनिक भी अत्याचार करता हूँ।"
\end{hindi}}
\flushright{\begin{Arabic}
\quranayah[50][30]
\end{Arabic}}
\flushleft{\begin{hindi}
जिस दिन हम जहन्नम से कहेंगे, "क्या तू भर गई?" और वह कहेगी, "क्या अभी और भी कुछ है?"
\end{hindi}}
\flushright{\begin{Arabic}
\quranayah[50][31]
\end{Arabic}}
\flushleft{\begin{hindi}
और जन्नत डर रखनेवालों के लिए निकट कर दी गई, कुछ भी दूर न रही
\end{hindi}}
\flushright{\begin{Arabic}
\quranayah[50][32]
\end{Arabic}}
\flushleft{\begin{hindi}
"यह है वह चीज़ जिसका तुमसे वादा किया जाता था हर रुजू करनेवाले, बड़ी निगरानी रखनेवाले के लिए; -
\end{hindi}}
\flushright{\begin{Arabic}
\quranayah[50][33]
\end{Arabic}}
\flushleft{\begin{hindi}
"जो रहमान से डरा परोक्ष में और आया रुजू रहनेवाला हृदय लेकर -
\end{hindi}}
\flushright{\begin{Arabic}
\quranayah[50][34]
\end{Arabic}}
\flushleft{\begin{hindi}
"प्रवेश करो उस (जन्नत) में सलामती के साथ" वह शाश्वत दिवस है
\end{hindi}}
\flushright{\begin{Arabic}
\quranayah[50][35]
\end{Arabic}}
\flushleft{\begin{hindi}
उनके लिए उसमें वह सब कुछ है जो वे चाहे और हमारे पास उससे अधिक भी है
\end{hindi}}
\flushright{\begin{Arabic}
\quranayah[50][36]
\end{Arabic}}
\flushleft{\begin{hindi}
उनसे पहले हम कितनी ही नस्लों को विनष्ट कर चुके है। वे लोग शक्ति में उनसे कहीं बढ़-चढ़कर थे। (पनाह की तलाश में) उन्होंने नगरों को छान मारा, कोई है भागने को ठिकाना?
\end{hindi}}
\flushright{\begin{Arabic}
\quranayah[50][37]
\end{Arabic}}
\flushleft{\begin{hindi}
निश्चय ही इसमें उस व्यक्ति के लिए शिक्षा-सामग्री है जिसके पास दिल हो या वह (दिल से) हाजिर रहकर कान लगाए
\end{hindi}}
\flushright{\begin{Arabic}
\quranayah[50][38]
\end{Arabic}}
\flushleft{\begin{hindi}
हमने आकाशों और धरती को और जो कुछ उनके बीच है छः दिनों में पैदा कर दिया और हमें कोई थकान न छू सकी
\end{hindi}}
\flushright{\begin{Arabic}
\quranayah[50][39]
\end{Arabic}}
\flushleft{\begin{hindi}
अतः जो कुछ वे कहते है उसपर धैर्य से काम लो और अपने रब की प्रशंसा की तसबीह करो; सूर्योदय से पूर्व और सूर्यास्त के पूर्व,
\end{hindi}}
\flushright{\begin{Arabic}
\quranayah[50][40]
\end{Arabic}}
\flushleft{\begin{hindi}
और रात की घड़ियों में फिर उसकी तसबीह करो और सजदों के पश्चात भी
\end{hindi}}
\flushright{\begin{Arabic}
\quranayah[50][41]
\end{Arabic}}
\flushleft{\begin{hindi}
और कान लगाकर सुन लेगा जिस दिन पुकारनेवाला अत्यन्त निकट के स्थान से पुकारेगा,
\end{hindi}}
\flushright{\begin{Arabic}
\quranayah[50][42]
\end{Arabic}}
\flushleft{\begin{hindi}
जिस दिन लोग भयंकर चीख़ को सत्यतः सुन रहे होंगे। वही दिन होगा निकलने का।-
\end{hindi}}
\flushright{\begin{Arabic}
\quranayah[50][43]
\end{Arabic}}
\flushleft{\begin{hindi}
हम ही जीलन प्रदान करते और मृत्यु देते है और हमारी ही ओर अन्ततः आना है। -
\end{hindi}}
\flushright{\begin{Arabic}
\quranayah[50][44]
\end{Arabic}}
\flushleft{\begin{hindi}
जिस दिन धरती उनपर से फट जाएगी और वे तेजी से निकल पड़ेंगे। यह इकट्ठा करना हमारे लिए अत्यन्त सरल है
\end{hindi}}
\flushright{\begin{Arabic}
\quranayah[50][45]
\end{Arabic}}
\flushleft{\begin{hindi}
हम जानते है जो कुछ वे कहते है, तुम उनपर कोई ज़बरदस्ती करनेवाले तो हो नहीं। अतः तुम क़ुरआन के द्वारा उसे नसीहत करो जो हमारी चेतावनी से डरे
\end{hindi}}
\chapter{Ad-Dhariyat (The Scatterers)}
\begin{Arabic}
\Huge{\centerline{\basmalah}}\end{Arabic}
\flushright{\begin{Arabic}
\quranayah[51][1]
\end{Arabic}}
\flushleft{\begin{hindi}
गवाह है (हवाएँ) जो गर्द-ग़ुबार उड़ाती फिरती है;
\end{hindi}}
\flushright{\begin{Arabic}
\quranayah[51][2]
\end{Arabic}}
\flushleft{\begin{hindi}
फिर बोझ उठाती है;
\end{hindi}}
\flushright{\begin{Arabic}
\quranayah[51][3]
\end{Arabic}}
\flushleft{\begin{hindi}
फिर नरमी से चलती है;
\end{hindi}}
\flushright{\begin{Arabic}
\quranayah[51][4]
\end{Arabic}}
\flushleft{\begin{hindi}
फिर मामले को अलग-अलग करती है;
\end{hindi}}
\flushright{\begin{Arabic}
\quranayah[51][5]
\end{Arabic}}
\flushleft{\begin{hindi}
निश्चय ही तुमसे जिस चीज़ का वादा किया जाता है, वह सत्य है;
\end{hindi}}
\flushright{\begin{Arabic}
\quranayah[51][6]
\end{Arabic}}
\flushleft{\begin{hindi}
और (कर्मों का) बदला अवश्य सामने आकर रहेगा
\end{hindi}}
\flushright{\begin{Arabic}
\quranayah[51][7]
\end{Arabic}}
\flushleft{\begin{hindi}
गवाह है धारियोंवाला आकाश।
\end{hindi}}
\flushright{\begin{Arabic}
\quranayah[51][8]
\end{Arabic}}
\flushleft{\begin{hindi}
निश्चय ही तुम उस बात में पड़े हुए हो जिनमें कथन भिन्न-भिन्न है
\end{hindi}}
\flushright{\begin{Arabic}
\quranayah[51][9]
\end{Arabic}}
\flushleft{\begin{hindi}
इसमें कोई सरफिरा ही विमुख होता है
\end{hindi}}
\flushright{\begin{Arabic}
\quranayah[51][10]
\end{Arabic}}
\flushleft{\begin{hindi}
मारे जाएँ अटकल दौड़ानेवाले;
\end{hindi}}
\flushright{\begin{Arabic}
\quranayah[51][11]
\end{Arabic}}
\flushleft{\begin{hindi}
जो ग़फ़लत में पड़े हुए हैं भूले हुए
\end{hindi}}
\flushright{\begin{Arabic}
\quranayah[51][12]
\end{Arabic}}
\flushleft{\begin{hindi}
पूछते है, "बदले का दिन कब आएगा?"
\end{hindi}}
\flushright{\begin{Arabic}
\quranayah[51][13]
\end{Arabic}}
\flushleft{\begin{hindi}
जिस दिन वे आग पर तपाए जाएँगे,
\end{hindi}}
\flushright{\begin{Arabic}
\quranayah[51][14]
\end{Arabic}}
\flushleft{\begin{hindi}
"चखों मज़ा. अपने फ़ितने (उपद्रव) का! यहीं है जिसके लिए तुम जल्दी मचा रहे थे।"
\end{hindi}}
\flushright{\begin{Arabic}
\quranayah[51][15]
\end{Arabic}}
\flushleft{\begin{hindi}
निश्चय ही डर रखनेवाले बाग़ों और स्रोतों में होंगे
\end{hindi}}
\flushright{\begin{Arabic}
\quranayah[51][16]
\end{Arabic}}
\flushleft{\begin{hindi}
जो कुछ उनके रब ने उन्हें दिया, वे उसे ले रहे होंगे। निस्संदेह वे इससे पहले उत्तमकारों में से थे
\end{hindi}}
\flushright{\begin{Arabic}
\quranayah[51][17]
\end{Arabic}}
\flushleft{\begin{hindi}
रातों को थोड़ा ही सोते थे,
\end{hindi}}
\flushright{\begin{Arabic}
\quranayah[51][18]
\end{Arabic}}
\flushleft{\begin{hindi}
और वही प्रातः की घड़ियों में क्षमा की प्रार्थना करते थे
\end{hindi}}
\flushright{\begin{Arabic}
\quranayah[51][19]
\end{Arabic}}
\flushleft{\begin{hindi}
और उनके मालों में माँगनेवाले और धनहीन का हक़ था
\end{hindi}}
\flushright{\begin{Arabic}
\quranayah[51][20]
\end{Arabic}}
\flushleft{\begin{hindi}
और धरती में विश्वास करनेवालों के लिए बहुत-सी निशानियाँ है,
\end{hindi}}
\flushright{\begin{Arabic}
\quranayah[51][21]
\end{Arabic}}
\flushleft{\begin{hindi}
और ,स्वयं तुम्हारे अपने आप में भी। तो क्या तुम देखते नहीं?
\end{hindi}}
\flushright{\begin{Arabic}
\quranayah[51][22]
\end{Arabic}}
\flushleft{\begin{hindi}
और आकाश मे ही तुम्हारी रोज़ी है और वह चीज़ भी जिसका तुमसे वादा किया जा रहा है
\end{hindi}}
\flushright{\begin{Arabic}
\quranayah[51][23]
\end{Arabic}}
\flushleft{\begin{hindi}
अतः सौगन्ध है आकाश और धरती के रब की। निश्चय ही वह सत्य बात है ऐसे ही जैसे तुम बोलते हो
\end{hindi}}
\flushright{\begin{Arabic}
\quranayah[51][24]
\end{Arabic}}
\flushleft{\begin{hindi}
क्या इबराईम के प्रतिष्ठित अतिथियों का वृतान्त तुम तक पहँचा?
\end{hindi}}
\flushright{\begin{Arabic}
\quranayah[51][25]
\end{Arabic}}
\flushleft{\begin{hindi}
जब वे उसके पास आए तो कहा, "सलाम है तुमपर!" उसने भी कहा, "सलाम है आप लोगों पर भी!" (और जी में कहा) "ये तो अपरिचित लोग हैं।"
\end{hindi}}
\flushright{\begin{Arabic}
\quranayah[51][26]
\end{Arabic}}
\flushleft{\begin{hindi}
फिर वह चुपके से अपने घरवालों के पास गया और एक मोटा-ताज़ा बछड़ा (का भूना हुआ मांस) ले आया
\end{hindi}}
\flushright{\begin{Arabic}
\quranayah[51][27]
\end{Arabic}}
\flushleft{\begin{hindi}
और उसे उनके सामने पेश किया। कहा, "क्या आप खाते नहीं?"
\end{hindi}}
\flushright{\begin{Arabic}
\quranayah[51][28]
\end{Arabic}}
\flushleft{\begin{hindi}
फिर उसने दिल में उनसे डर महसूस किया। उन्होंने कहा, "डरिए नहीं।" और उन्होंने उसे एक ज्ञानवान लड़के की मंगल-सूचना दी
\end{hindi}}
\flushright{\begin{Arabic}
\quranayah[51][29]
\end{Arabic}}
\flushleft{\begin{hindi}
इसपर उसकी स्त्री (चकित होकर) आगे बढ़ी और उसने अपना मुँह पीट लिया और कहने लगी, "एक बूढ़ी बाँझ (के यहाँ बच्चा पैदा होगा)!"
\end{hindi}}
\flushright{\begin{Arabic}
\quranayah[51][30]
\end{Arabic}}
\flushleft{\begin{hindi}
उन्होंने कहा, "ऐसी ही तेरे रब ने कहा है। निश्चय ही वह बड़ा तत्वदर्शी, ज्ञानवान है।"
\end{hindi}}
\flushright{\begin{Arabic}
\quranayah[51][31]
\end{Arabic}}
\flushleft{\begin{hindi}
उसने कहा, "ऐ (अल्लाह के भेजे हुए) दूतों, तुम्हारे सामने क्या मुहिम है?"
\end{hindi}}
\flushright{\begin{Arabic}
\quranayah[51][32]
\end{Arabic}}
\flushleft{\begin{hindi}
उन्होंने कहा, "हम एक अपराधी क़ौम की ओर भेजे गए है;
\end{hindi}}
\flushright{\begin{Arabic}
\quranayah[51][33]
\end{Arabic}}
\flushleft{\begin{hindi}
"ताकि उनके ऊपर मिट्टी के पत्थर (कंकड़) बरसाएँ,
\end{hindi}}
\flushright{\begin{Arabic}
\quranayah[51][34]
\end{Arabic}}
\flushleft{\begin{hindi}
जो आपके रब के यहाँ सीमा का अतिक्रमण करनेवालों के लिए चिन्हित है।"
\end{hindi}}
\flushright{\begin{Arabic}
\quranayah[51][35]
\end{Arabic}}
\flushleft{\begin{hindi}
फिर वहाँ जो ईमानवाले थे, उन्हें हमने निकाल लिया;
\end{hindi}}
\flushright{\begin{Arabic}
\quranayah[51][36]
\end{Arabic}}
\flushleft{\begin{hindi}
किन्तु हमने वहाँ एक घर के अतिरिक्त मुसलमानों (आज्ञाकारियों) का और कोई घर न पाया
\end{hindi}}
\flushright{\begin{Arabic}
\quranayah[51][37]
\end{Arabic}}
\flushleft{\begin{hindi}
इसके पश्चात हमने वहाँ उन लोगों के लिए एक निशानी छोड़ दी, जो दुखद यातना से डरते है
\end{hindi}}
\flushright{\begin{Arabic}
\quranayah[51][38]
\end{Arabic}}
\flushleft{\begin{hindi}
और मूसा के वृतान्त में भी (निशानी है) जब हमने फ़िरऔन के पास के स्पष्ट प्रमाण के साथ भेजा,
\end{hindi}}
\flushright{\begin{Arabic}
\quranayah[51][39]
\end{Arabic}}
\flushleft{\begin{hindi}
किन्तु उसने अपनी शक्ति के कारण मुँह फेर लिया और कहा, "जादूगर है या दीवाना।"
\end{hindi}}
\flushright{\begin{Arabic}
\quranayah[51][40]
\end{Arabic}}
\flushleft{\begin{hindi}
अन्ततः हमने उसे और उसकी सेनाओं को पकड़ लिया और उन्हें गहरे पानी में फेंक दिया, इस दशा में कि वह निन्दनीय था
\end{hindi}}
\flushright{\begin{Arabic}
\quranayah[51][41]
\end{Arabic}}
\flushleft{\begin{hindi}
और आद में भी (तुम्हारे लिए निशानी है) जबकि हमने उनपर अशुभ वायु चला दी
\end{hindi}}
\flushright{\begin{Arabic}
\quranayah[51][42]
\end{Arabic}}
\flushleft{\begin{hindi}
वह जिस चीज़ पर से गुज़री उसे उसने जीर्ण-शीर्ण करके रख दिया
\end{hindi}}
\flushright{\begin{Arabic}
\quranayah[51][43]
\end{Arabic}}
\flushleft{\begin{hindi}
और समुद्र में भी (तुम्हारे लिए निशानी है) जबकि उनसे कहा गया, "एक समय तक मज़े कर लो!"
\end{hindi}}
\flushright{\begin{Arabic}
\quranayah[51][44]
\end{Arabic}}
\flushleft{\begin{hindi}
किन्तु उन्होंने अपने रब के आदेश की अवहेलना की; फिर कड़क ने उन्हें आ लिया और वे देखते रहे
\end{hindi}}
\flushright{\begin{Arabic}
\quranayah[51][45]
\end{Arabic}}
\flushleft{\begin{hindi}
फिर वे न खड़े ही हो सके और न अपना बचाव ही कर सके
\end{hindi}}
\flushright{\begin{Arabic}
\quranayah[51][46]
\end{Arabic}}
\flushleft{\begin{hindi}
और इससे पहले नूह की क़ौम को भी पकड़ा। निश्चय ही वे अवज्ञाकारी लोग थे
\end{hindi}}
\flushright{\begin{Arabic}
\quranayah[51][47]
\end{Arabic}}
\flushleft{\begin{hindi}
आकाश को हमने अपने हाथ के बल से बनाया और हम बड़ी समाई रखनेवाले है
\end{hindi}}
\flushright{\begin{Arabic}
\quranayah[51][48]
\end{Arabic}}
\flushleft{\begin{hindi}
और धरती को हमने बिछाया, तो हम क्या ही ख़ूब बिछानेवाले है
\end{hindi}}
\flushright{\begin{Arabic}
\quranayah[51][49]
\end{Arabic}}
\flushleft{\begin{hindi}
और हमने हर चीज़ के जोड़े बनाए, ताकि तुम ध्यान दो
\end{hindi}}
\flushright{\begin{Arabic}
\quranayah[51][50]
\end{Arabic}}
\flushleft{\begin{hindi}
अतः अल्लाह की ओर दौड़ो। मैं उसकी ओर से तुम्हारे लिए एक प्रत्यक्ष सावधान करनेवाला हूँ
\end{hindi}}
\flushright{\begin{Arabic}
\quranayah[51][51]
\end{Arabic}}
\flushleft{\begin{hindi}
और अल्लाह के साथ कोई दूसरा पूज्य-प्रभु न ठहराओ। मैं उसकी ओर से तुम्हारे लिए एक प्रत्यक्ष सावधान करनेवाला हूँ
\end{hindi}}
\flushright{\begin{Arabic}
\quranayah[51][52]
\end{Arabic}}
\flushleft{\begin{hindi}
इसी तरह उन लोगों के पास भी, जो उनसे पहले गुज़र चुके है, जो भी रसूल आया तो उन्होंने बस यही कहा, "जादूगर है या दीवाना!"
\end{hindi}}
\flushright{\begin{Arabic}
\quranayah[51][53]
\end{Arabic}}
\flushleft{\begin{hindi}
क्या उन्होंने एक-दूसरे को इसकी वसीयत कर रखी है? नहीं, बल्कि वे है ही सरकश लोग
\end{hindi}}
\flushright{\begin{Arabic}
\quranayah[51][54]
\end{Arabic}}
\flushleft{\begin{hindi}
अतः उनसे मुँह फेर लो अब तुमपर कोई मलामत नहीं
\end{hindi}}
\flushright{\begin{Arabic}
\quranayah[51][55]
\end{Arabic}}
\flushleft{\begin{hindi}
और याद दिलाते रहो, क्योंकि याद दिलाना ईमानवालों को लाभ पहुँचाता है
\end{hindi}}
\flushright{\begin{Arabic}
\quranayah[51][56]
\end{Arabic}}
\flushleft{\begin{hindi}
मैंने तो जिन्नों और मनुष्यों को केवल इसलिए पैदा किया है कि वे मेरी बन्दगी करे
\end{hindi}}
\flushright{\begin{Arabic}
\quranayah[51][57]
\end{Arabic}}
\flushleft{\begin{hindi}
मैं उनसे कोई रोज़ी नहीं चाहता और न यह चाहता हूँ कि वे मुझे खिलाएँ
\end{hindi}}
\flushright{\begin{Arabic}
\quranayah[51][58]
\end{Arabic}}
\flushleft{\begin{hindi}
निश्चय ही अल्लाह ही है रोज़ी देनेवाला, शक्तिशाली, दृढ़
\end{hindi}}
\flushright{\begin{Arabic}
\quranayah[51][59]
\end{Arabic}}
\flushleft{\begin{hindi}
अतः जिन लोगों ने ज़ुल्म किया है उनके लिए एक नियत पैमाना है; जैसा उनके साथियों का नियत पैमाना था। अतः वे मुझसे जल्दी न मचाएँ!
\end{hindi}}
\flushright{\begin{Arabic}
\quranayah[51][60]
\end{Arabic}}
\flushleft{\begin{hindi}
अतः इनकार करनेवालों के लिए बड़ी खराबी है, उनके उस दिन के कारण जिसकी उन्हें धमकी दी जा रही है
\end{hindi}}
\chapter{At-Tur (The Mountain)}
\begin{Arabic}
\Huge{\centerline{\basmalah}}\end{Arabic}
\flushright{\begin{Arabic}
\quranayah[52][1]
\end{Arabic}}
\flushleft{\begin{hindi}
गवाह है तूर पर्वत,
\end{hindi}}
\flushright{\begin{Arabic}
\quranayah[52][2]
\end{Arabic}}
\flushleft{\begin{hindi}
और लिखी हुई किताब;
\end{hindi}}
\flushright{\begin{Arabic}
\quranayah[52][3]
\end{Arabic}}
\flushleft{\begin{hindi}
फैले हुए झिल्ली के पन्ने में
\end{hindi}}
\flushright{\begin{Arabic}
\quranayah[52][4]
\end{Arabic}}
\flushleft{\begin{hindi}
और बसा हुआ घर;
\end{hindi}}
\flushright{\begin{Arabic}
\quranayah[52][5]
\end{Arabic}}
\flushleft{\begin{hindi}
और ऊँची छत;
\end{hindi}}
\flushright{\begin{Arabic}
\quranayah[52][6]
\end{Arabic}}
\flushleft{\begin{hindi}
और उफनता समुद्र
\end{hindi}}
\flushright{\begin{Arabic}
\quranayah[52][7]
\end{Arabic}}
\flushleft{\begin{hindi}
कि तेरे रब की यातना अवश्य घटित होकर रहेगी;
\end{hindi}}
\flushright{\begin{Arabic}
\quranayah[52][8]
\end{Arabic}}
\flushleft{\begin{hindi}
जिसे टालनेवाला कोई नहीं;
\end{hindi}}
\flushright{\begin{Arabic}
\quranayah[52][9]
\end{Arabic}}
\flushleft{\begin{hindi}
जिस दिल आकाश बुरी तरह डगमगाएगा;
\end{hindi}}
\flushright{\begin{Arabic}
\quranayah[52][10]
\end{Arabic}}
\flushleft{\begin{hindi}
और पहाड़ चलते-फिरते होंगे;
\end{hindi}}
\flushright{\begin{Arabic}
\quranayah[52][11]
\end{Arabic}}
\flushleft{\begin{hindi}
तो तबाही है उस दिन, झुठलानेवालों के लिए;
\end{hindi}}
\flushright{\begin{Arabic}
\quranayah[52][12]
\end{Arabic}}
\flushleft{\begin{hindi}
जो बात बनाने में लगे हुए खेल रहे है
\end{hindi}}
\flushright{\begin{Arabic}
\quranayah[52][13]
\end{Arabic}}
\flushleft{\begin{hindi}
जिस दिन वे धक्के दे-देकर जहन्नम की ओर ढकेले जाएँगे
\end{hindi}}
\flushright{\begin{Arabic}
\quranayah[52][14]
\end{Arabic}}
\flushleft{\begin{hindi}
(कहा जाएगा), "यही है वह आग जिसे तुम झुठलाते थे
\end{hindi}}
\flushright{\begin{Arabic}
\quranayah[52][15]
\end{Arabic}}
\flushleft{\begin{hindi}
"अब भला (बताओ) यह कोई जादू है या तुम्हे सुझाई नहीं देता?
\end{hindi}}
\flushright{\begin{Arabic}
\quranayah[52][16]
\end{Arabic}}
\flushleft{\begin{hindi}
"जाओ, झुलसो उसमें! अब धैर्य से काम लो या धैर्य से काम न लो; तुम्हारे लिए बराबर है। तुम वही बदला पा रहे हो, जो तुम करते रहे थे।"
\end{hindi}}
\flushright{\begin{Arabic}
\quranayah[52][17]
\end{Arabic}}
\flushleft{\begin{hindi}
निश्चय ही डर रखनेवाले बाग़ों और नेमतों में होंगे
\end{hindi}}
\flushright{\begin{Arabic}
\quranayah[52][18]
\end{Arabic}}
\flushleft{\begin{hindi}
जो कुछ उनके रब ने उन्हें दिया होगा, उसका आनन्द ले रहे होंगे और इस बात से कि उनके रब ने उन्हें भड़कती हुई आग से बचा लिया -
\end{hindi}}
\flushright{\begin{Arabic}
\quranayah[52][19]
\end{Arabic}}
\flushleft{\begin{hindi}
"मज़े से खाओ और पियो उन कर्मों के बदले में जो तुम करते रहे हो।"
\end{hindi}}
\flushright{\begin{Arabic}
\quranayah[52][20]
\end{Arabic}}
\flushleft{\begin{hindi}
- पंक्तिबद्ध तख़्तो पर तकिया लगाए हुए होंगे और हम बड़ी आँखोंवाली हूरों (परम रूपवती स्त्रियों) से उनका विवाह कर देंगे
\end{hindi}}
\flushright{\begin{Arabic}
\quranayah[52][21]
\end{Arabic}}
\flushleft{\begin{hindi}
जो लोग ईमान लाए और उनकी सन्तान ने भी ईमान के साथ उसका अनुसरण किया, उनकी सन्तान को भी हम उनसे मिला देंगे, और उनके कर्म में से कुछ भी कम करके उन्हें नहीं देंगे। हर व्यक्ति अपनी कमाई के बदले में बन्धक है
\end{hindi}}
\flushright{\begin{Arabic}
\quranayah[52][22]
\end{Arabic}}
\flushleft{\begin{hindi}
और हम उन्हें मेवे और मांस, जिसकी वे इच्छा करेंगे दिए चले जाएँगे
\end{hindi}}
\flushright{\begin{Arabic}
\quranayah[52][23]
\end{Arabic}}
\flushleft{\begin{hindi}
वे वहाँ आपस में प्याले हाथोंहाथ ले रहे होंगे, जिसमें न कोई बेहूदगी होगी और न गुनाह पर उभारनेवाली कोई बात,
\end{hindi}}
\flushright{\begin{Arabic}
\quranayah[52][24]
\end{Arabic}}
\flushleft{\begin{hindi}
और उनकी सेवा में सुरक्षित मोतियों के सदृश किशोर दौड़ते फिरते होंगे, जो ख़ास उन्हीं (की सेवा) के लिए होंगे
\end{hindi}}
\flushright{\begin{Arabic}
\quranayah[52][25]
\end{Arabic}}
\flushleft{\begin{hindi}
उनमें से कुछ व्यक्ति कुछ व्यक्तियों की ओर हाल पूछते हुए रुख़ करेंगे,
\end{hindi}}
\flushright{\begin{Arabic}
\quranayah[52][26]
\end{Arabic}}
\flushleft{\begin{hindi}
कहेंगे, "निश्चय ही हम पहले अपने घरवालों में डरते रहे है,
\end{hindi}}
\flushright{\begin{Arabic}
\quranayah[52][27]
\end{Arabic}}
\flushleft{\begin{hindi}
"अन्ततः अल्लाह ने हमपर एहसास किया और हमें गर्म विषैली वायु की यातना से बचा लिया
\end{hindi}}
\flushright{\begin{Arabic}
\quranayah[52][28]
\end{Arabic}}
\flushleft{\begin{hindi}
"इससे पहले हम उसे पुकारते रहे है। निश्चय ही वह सदव्यवहार करनेवाला, अत्यन्त दयावान है।"
\end{hindi}}
\flushright{\begin{Arabic}
\quranayah[52][29]
\end{Arabic}}
\flushleft{\begin{hindi}
अतः तुम याद दिलाते रहो। अपने रब की अनुकम्पा से न तुम काहिन (ढोंगी भविष्यवक्ता) हो और न दीवाना
\end{hindi}}
\flushright{\begin{Arabic}
\quranayah[52][30]
\end{Arabic}}
\flushleft{\begin{hindi}
या वे कहते है, "वह कवि है जिसके लिए हम काल-चक्र की प्रतीक्षा कर रहे है?"
\end{hindi}}
\flushright{\begin{Arabic}
\quranayah[52][31]
\end{Arabic}}
\flushleft{\begin{hindi}
कह दो, "प्रतीक्षा करो! मैं भी तुम्हारे साथ प्रतीक्षा करता हूँ।"
\end{hindi}}
\flushright{\begin{Arabic}
\quranayah[52][32]
\end{Arabic}}
\flushleft{\begin{hindi}
या उनकी बुद्धियाँ यही आदेश दे रही है, या वे ही है सरकश लोग?
\end{hindi}}
\flushright{\begin{Arabic}
\quranayah[52][33]
\end{Arabic}}
\flushleft{\begin{hindi}
या वे कहते है, "उसने उस (क़ुरआन) को स्वयं ही कह लिया है?" नहीं, बल्कि वे ईमान नहीं लाते
\end{hindi}}
\flushright{\begin{Arabic}
\quranayah[52][34]
\end{Arabic}}
\flushleft{\begin{hindi}
अच्छा यदि वे सच्चे है तो उन्हें उस जैसी वाणी ले आनी चाहिए
\end{hindi}}
\flushright{\begin{Arabic}
\quranayah[52][35]
\end{Arabic}}
\flushleft{\begin{hindi}
या वे बिना किसी चीज़ के पैदा हो गए? या वे स्वयं ही अपने स्रष्टाँ है?
\end{hindi}}
\flushright{\begin{Arabic}
\quranayah[52][36]
\end{Arabic}}
\flushleft{\begin{hindi}
या उन्होंने आकाशों और धरती को पैदा किया?
\end{hindi}}
\flushright{\begin{Arabic}
\quranayah[52][37]
\end{Arabic}}
\flushleft{\begin{hindi}
या उनके पास तुम्हारे रब के खज़ाने है? या वही उनके परिरक्षक है?
\end{hindi}}
\flushright{\begin{Arabic}
\quranayah[52][38]
\end{Arabic}}
\flushleft{\begin{hindi}
या उनके पास कोई सीढ़ी है जिसपर चढ़कर वे (कान लगाकर) सुन लेते है? फिर उनमें से जिसने सुन लिया हो तो वह ले आए स्पष्ट प्रमाण
\end{hindi}}
\flushright{\begin{Arabic}
\quranayah[52][39]
\end{Arabic}}
\flushleft{\begin{hindi}
या उस (अल्लाह) के लिए बेटियाँ है और तुम्हारे अपने लिए बेटे?
\end{hindi}}
\flushright{\begin{Arabic}
\quranayah[52][40]
\end{Arabic}}
\flushleft{\begin{hindi}
या तुम उनसे कोई पारिश्रामिक माँगते हो कि वे तावान के बोझ से दबे जा रहे है?
\end{hindi}}
\flushright{\begin{Arabic}
\quranayah[52][41]
\end{Arabic}}
\flushleft{\begin{hindi}
या उनके पास परोक्ष (स्पष्ट) है जिसके आधार पर वे लिए रहे हो?
\end{hindi}}
\flushright{\begin{Arabic}
\quranayah[52][42]
\end{Arabic}}
\flushleft{\begin{hindi}
या वे कोई चाल चलना चाहते है? तो जिन लोगों ने इनकार किया वही चाल की लपेट में आनेवाले है
\end{hindi}}
\flushright{\begin{Arabic}
\quranayah[52][43]
\end{Arabic}}
\flushleft{\begin{hindi}
या अल्लाह के अतिरिक्त उनका कोई और पूज्य-प्रभु है? अल्लाह महान और उच्च है उससे जो वे साझी ठहराते है
\end{hindi}}
\flushright{\begin{Arabic}
\quranayah[52][44]
\end{Arabic}}
\flushleft{\begin{hindi}
यदि वे आकाश का कोई टुकटा गिरता हुआ देखें तो कहेंगे, "यह तो परत पर परत बादल है!"
\end{hindi}}
\flushright{\begin{Arabic}
\quranayah[52][45]
\end{Arabic}}
\flushleft{\begin{hindi}
अतः छोडो उन्हें, यहाँ तक कि वे अपने उस दिन का सामना करें जिसमें उनपर वज्रपात होगा;
\end{hindi}}
\flushright{\begin{Arabic}
\quranayah[52][46]
\end{Arabic}}
\flushleft{\begin{hindi}
जिस दिन उनकी चाल उनके कुछ भी काम न आएगी और न उन्हें कोई सहायता ही मिलेगी;
\end{hindi}}
\flushright{\begin{Arabic}
\quranayah[52][47]
\end{Arabic}}
\flushleft{\begin{hindi}
और निश्चय ही जिन लोगों ने ज़ुल्म किया उनके लिए एक यातना है उससे हटकर भी, परन्तु उनमें से अधिकतर जानते नहीं
\end{hindi}}
\flushright{\begin{Arabic}
\quranayah[52][48]
\end{Arabic}}
\flushleft{\begin{hindi}
अपने रब का फ़ैसला आने तक धैर्य से काम लो, तुम तो हमारी आँखों में हो, और जब उठो तो अपने रब का गुणगान करो;
\end{hindi}}
\flushright{\begin{Arabic}
\quranayah[52][49]
\end{Arabic}}
\flushleft{\begin{hindi}
रात की कुछ घड़ियों में भी उसकी तसबीह करो, और सितारों के पीठ फेरने के समय (प्रातःकाल) भी
\end{hindi}}
\chapter{An-Najm (The Star)}
\begin{Arabic}
\Huge{\centerline{\basmalah}}\end{Arabic}
\flushright{\begin{Arabic}
\quranayah[53][1]
\end{Arabic}}
\flushleft{\begin{hindi}
गवाह है तारा, जब वह नीचे को आए
\end{hindi}}
\flushright{\begin{Arabic}
\quranayah[53][2]
\end{Arabic}}
\flushleft{\begin{hindi}
तुम्हारी साथी (मुहम्मह सल्ल॰) न गुमराह हुआ और न बहका;
\end{hindi}}
\flushright{\begin{Arabic}
\quranayah[53][3]
\end{Arabic}}
\flushleft{\begin{hindi}
और न वह अपनी इच्छा से बोलता है;
\end{hindi}}
\flushright{\begin{Arabic}
\quranayah[53][4]
\end{Arabic}}
\flushleft{\begin{hindi}
वह तो बस एक प्रकाशना है, जो की जा रही है
\end{hindi}}
\flushright{\begin{Arabic}
\quranayah[53][5]
\end{Arabic}}
\flushleft{\begin{hindi}
उसे बड़ी शक्तियोंवाले ने सिखाया,
\end{hindi}}
\flushright{\begin{Arabic}
\quranayah[53][6]
\end{Arabic}}
\flushleft{\begin{hindi}
स्थिर रीतिवाले ने।
\end{hindi}}
\flushright{\begin{Arabic}
\quranayah[53][7]
\end{Arabic}}
\flushleft{\begin{hindi}
अतः वह भरपूर हुआ, इस हाल में कि वह क्षितिज के उच्चतम छोर पर है
\end{hindi}}
\flushright{\begin{Arabic}
\quranayah[53][8]
\end{Arabic}}
\flushleft{\begin{hindi}
फिर वह निकट हुआ और उतर गया
\end{hindi}}
\flushright{\begin{Arabic}
\quranayah[53][9]
\end{Arabic}}
\flushleft{\begin{hindi}
अब दो कमानों के बराबर या उससे भी अधिक निकट हो गया
\end{hindi}}
\flushright{\begin{Arabic}
\quranayah[53][10]
\end{Arabic}}
\flushleft{\begin{hindi}
तब उसने अपने बन्दे की ओर प्रकाशना की, जो कुछ प्रकाशना की।
\end{hindi}}
\flushright{\begin{Arabic}
\quranayah[53][11]
\end{Arabic}}
\flushleft{\begin{hindi}
दिल ने कोई धोखा नहीं दिया, जो कुछ उसने देखा;
\end{hindi}}
\flushright{\begin{Arabic}
\quranayah[53][12]
\end{Arabic}}
\flushleft{\begin{hindi}
अब क्या तुम उस चीज़ पर झगड़ते हो, जिसे वह देख रहा है? -
\end{hindi}}
\flushright{\begin{Arabic}
\quranayah[53][13]
\end{Arabic}}
\flushleft{\begin{hindi}
और निश्चय ही वह उसे एक बार और
\end{hindi}}
\flushright{\begin{Arabic}
\quranayah[53][14]
\end{Arabic}}
\flushleft{\begin{hindi}
'सिदरतुल मुन्तहा' (परली सीमा के बेर) के पास उतरते देख चुका है
\end{hindi}}
\flushright{\begin{Arabic}
\quranayah[53][15]
\end{Arabic}}
\flushleft{\begin{hindi}
उसी के निकट 'जन्नतुल मावा' (ठिकानेवाली जन्नत) है। -
\end{hindi}}
\flushright{\begin{Arabic}
\quranayah[53][16]
\end{Arabic}}
\flushleft{\begin{hindi}
जबकि छा रहा था उस बेर पर, जो कुछ छा रहा था
\end{hindi}}
\flushright{\begin{Arabic}
\quranayah[53][17]
\end{Arabic}}
\flushleft{\begin{hindi}
निगाह न तो टेढ़ी हुइ और न हद से आगे बढ़ी
\end{hindi}}
\flushright{\begin{Arabic}
\quranayah[53][18]
\end{Arabic}}
\flushleft{\begin{hindi}
निश्चय ही उसने अपने रब की बड़ी-बड़ी निशानियाँ देखीं
\end{hindi}}
\flushright{\begin{Arabic}
\quranayah[53][19]
\end{Arabic}}
\flushleft{\begin{hindi}
तो क्या तुमने लात और उज़्ज़ा
\end{hindi}}
\flushright{\begin{Arabic}
\quranayah[53][20]
\end{Arabic}}
\flushleft{\begin{hindi}
और तीसरी एक और (देवी) मनात पर विचार किया?
\end{hindi}}
\flushright{\begin{Arabic}
\quranayah[53][21]
\end{Arabic}}
\flushleft{\begin{hindi}
क्या तुम्हारे लिए तो बेटे है उनके लिए बेटियाँ?
\end{hindi}}
\flushright{\begin{Arabic}
\quranayah[53][22]
\end{Arabic}}
\flushleft{\begin{hindi}
तब तो यह बहुत बेढ़ंगा और अन्यायपूर्ण बँटवारा हुआ!
\end{hindi}}
\flushright{\begin{Arabic}
\quranayah[53][23]
\end{Arabic}}
\flushleft{\begin{hindi}
वे तो बस कुछ नाम है जो तुमने और तुम्हारे बाप-दादा ने रख लिए है। अल्लाह ने उनके लिए कोई सनद नहीं उतारी। वे तो केवल अटकल के पीछे चले रहे है और उनके पीछे जो उनके मन की इच्छा होती है। हालाँकि उनके पास उनके रब की ओर से मार्गदर्शन आ चुका है
\end{hindi}}
\flushright{\begin{Arabic}
\quranayah[53][24]
\end{Arabic}}
\flushleft{\begin{hindi}
(क्या उनकी देवियाँ उन्हें लाभ पहुँचा सकती है) या मनुष्य वह कुछ पा लेगा, जिसकी वह कामना करता है?
\end{hindi}}
\flushright{\begin{Arabic}
\quranayah[53][25]
\end{Arabic}}
\flushleft{\begin{hindi}
आख़िरत और दुनिया का मालिक तो अल्लाह ही है
\end{hindi}}
\flushright{\begin{Arabic}
\quranayah[53][26]
\end{Arabic}}
\flushleft{\begin{hindi}
आकाशों में कितने ही फ़रिश्ते है, उनकी सिफ़ारिश कुछ काम नहीं आएगी; यदि काम आ सकती है तो इसके पश्चात ही कि अल्लाह अनुमति दे, जिसे चाहे और पसन्द करे।
\end{hindi}}
\flushright{\begin{Arabic}
\quranayah[53][27]
\end{Arabic}}
\flushleft{\begin{hindi}
जो लोग आख़िरत को नहीं मानते, वे फ़रिश्तों के देवियों के नाम से अभिहित करते है,
\end{hindi}}
\flushright{\begin{Arabic}
\quranayah[53][28]
\end{Arabic}}
\flushleft{\begin{hindi}
हालाँकि इस विषय में उन्हें कोई ज्ञान नहीं। वे केवल अटकल के पीछे चलते है, हालाँकि सत्य से जो लाभ पहुँचता है वह अटकल से कदापि नहीं पहुँच सकता।
\end{hindi}}
\flushright{\begin{Arabic}
\quranayah[53][29]
\end{Arabic}}
\flushleft{\begin{hindi}
अतः तुम उसको ध्यान में न लाओ जो हमारे ज़िक्र से मुँह मोड़ता है और सांसारिक जीवन के सिवा उसने कुछ नहीं चाहा
\end{hindi}}
\flushright{\begin{Arabic}
\quranayah[53][30]
\end{Arabic}}
\flushleft{\begin{hindi}
ऐसे लोगों के ज्ञान की पहुँच बस यहीं तक है। निश्चय ही तुम्हारा रब ही उसे भली-भाँति जानता है जो उसके मार्ग से भटक गया और वही उसे भी भली-भाँति जानता है जिसने सीधा मार्ग अपनाया
\end{hindi}}
\flushright{\begin{Arabic}
\quranayah[53][31]
\end{Arabic}}
\flushleft{\begin{hindi}
अल्लाह ही का है जो कुछ आकाशों में है और जो कुछ धरती में है, ताकि जिन लोगों ने बुराई की वह उन्हें उनके किए का बदला दे। और जिन लोगों ने भलाई की उन्हें अच्छा बदला दे;
\end{hindi}}
\flushright{\begin{Arabic}
\quranayah[53][32]
\end{Arabic}}
\flushleft{\begin{hindi}
वे लोग जो बड़े गुनाहों और अश्लील कर्मों से बचते है, यह और बात है कि संयोगबश कोई छोटी बुराई उनसे हो जाए। निश्चय ही तुम्हारा रब क्षमाशीलता मे बड़ा व्यापक है। वह तुम्हें उस समय से भली-भाँति जानता है, जबकि उसने तुम्हें धरती से पैदा किया और जबकि तुम अपनी माँओ के पेटों में भ्रुण अवस्था में थे। अतः अपने मन की पवित्रता और निखार का दावा न करो। वह उस व्यक्ति को भली-भाँति जानता है, जिसने डर रखा
\end{hindi}}
\flushright{\begin{Arabic}
\quranayah[53][33]
\end{Arabic}}
\flushleft{\begin{hindi}
क्या तुमने उस व्यक्ति को देखा जिसने मुँह फेरा,
\end{hindi}}
\flushright{\begin{Arabic}
\quranayah[53][34]
\end{Arabic}}
\flushleft{\begin{hindi}
और थोड़ा-सा देकर रुक गया;
\end{hindi}}
\flushright{\begin{Arabic}
\quranayah[53][35]
\end{Arabic}}
\flushleft{\begin{hindi}
क्या उसके पास परोक्ष का ज्ञान है कि वह देख रहा है;
\end{hindi}}
\flushright{\begin{Arabic}
\quranayah[53][36]
\end{Arabic}}
\flushleft{\begin{hindi}
या उसको उन बातों की ख़बर नहीं पहुँची, जो मूसा की किताबों में है
\end{hindi}}
\flushright{\begin{Arabic}
\quranayah[53][37]
\end{Arabic}}
\flushleft{\begin{hindi}
और इबराहीम की (किताबों में है), जिसने अल्लाह की बन्दगी का) पूरा-पूरा हक़ अदा कर दिया?
\end{hindi}}
\flushright{\begin{Arabic}
\quranayah[53][38]
\end{Arabic}}
\flushleft{\begin{hindi}
यह कि कोई बोझ उठानेवाला किसी दूसरे का बोझ न उठाएगा;
\end{hindi}}
\flushright{\begin{Arabic}
\quranayah[53][39]
\end{Arabic}}
\flushleft{\begin{hindi}
और यह कि मनुष्य के लिए बस वही है जिसके लिए उसने प्रयास किया;
\end{hindi}}
\flushright{\begin{Arabic}
\quranayah[53][40]
\end{Arabic}}
\flushleft{\begin{hindi}
और यह कि उसका प्रयास शीघ्र ही देखा जाएगा।
\end{hindi}}
\flushright{\begin{Arabic}
\quranayah[53][41]
\end{Arabic}}
\flushleft{\begin{hindi}
फिर उसे पूरा बदला दिया जाएगा;
\end{hindi}}
\flushright{\begin{Arabic}
\quranayah[53][42]
\end{Arabic}}
\flushleft{\begin{hindi}
और यह कि अन्त में पहुँचना तुम्हारे रब ही की ओर है;
\end{hindi}}
\flushright{\begin{Arabic}
\quranayah[53][43]
\end{Arabic}}
\flushleft{\begin{hindi}
और यह कि वही है जो हँसाता और रुलाता है;
\end{hindi}}
\flushright{\begin{Arabic}
\quranayah[53][44]
\end{Arabic}}
\flushleft{\begin{hindi}
और यह कि वही जो मारता और जिलाता है;
\end{hindi}}
\flushright{\begin{Arabic}
\quranayah[53][45]
\end{Arabic}}
\flushleft{\begin{hindi}
और यह कि वही है जिसने नर और मादा के जोड़े पैदा किए,
\end{hindi}}
\flushright{\begin{Arabic}
\quranayah[53][46]
\end{Arabic}}
\flushleft{\begin{hindi}
एक बूँद से, जब वह टपकाई जाती है;
\end{hindi}}
\flushright{\begin{Arabic}
\quranayah[53][47]
\end{Arabic}}
\flushleft{\begin{hindi}
और यह कि उसी के ज़िम्मे दोबारा उठाना भी है;
\end{hindi}}
\flushright{\begin{Arabic}
\quranayah[53][48]
\end{Arabic}}
\flushleft{\begin{hindi}
और यह कि वही है जिसने धनी और पूँजीपति बनाया;
\end{hindi}}
\flushright{\begin{Arabic}
\quranayah[53][49]
\end{Arabic}}
\flushleft{\begin{hindi}
और यह कि वही है जो शेअरा (नामक तारे) का रब है
\end{hindi}}
\flushright{\begin{Arabic}
\quranayah[53][50]
\end{Arabic}}
\flushleft{\begin{hindi}
और यह कि वही है उसी ने प्राचीन आद को विनष्ट किया;
\end{hindi}}
\flushright{\begin{Arabic}
\quranayah[53][51]
\end{Arabic}}
\flushleft{\begin{hindi}
और समूद को भी। फिर किसी को बाक़ी न छोड़ा।
\end{hindi}}
\flushright{\begin{Arabic}
\quranayah[53][52]
\end{Arabic}}
\flushleft{\begin{hindi}
और उससे पहले नूह की क़ौम को भी। बेशक वे ज़ालिम और सरकश थे
\end{hindi}}
\flushright{\begin{Arabic}
\quranayah[53][53]
\end{Arabic}}
\flushleft{\begin{hindi}
उलट जानेवाली बस्ती को भी फेंक दिया।
\end{hindi}}
\flushright{\begin{Arabic}
\quranayah[53][54]
\end{Arabic}}
\flushleft{\begin{hindi}
तो ढँक लिया उसे जिस चीज़ ने ढँक लिया;
\end{hindi}}
\flushright{\begin{Arabic}
\quranayah[53][55]
\end{Arabic}}
\flushleft{\begin{hindi}
फिर तू अपने रब के चमत्कारों में से किस-किस के विषय में संदेह करेगा?
\end{hindi}}
\flushright{\begin{Arabic}
\quranayah[53][56]
\end{Arabic}}
\flushleft{\begin{hindi}
यह पहले के सावधान-कर्ताओं के सदृश एक सावधान करनेवाला है
\end{hindi}}
\flushright{\begin{Arabic}
\quranayah[53][57]
\end{Arabic}}
\flushleft{\begin{hindi}
निकट आनेवाली (क़ियामत की घड़ी) निकट आ गई
\end{hindi}}
\flushright{\begin{Arabic}
\quranayah[53][58]
\end{Arabic}}
\flushleft{\begin{hindi}
अल्लाह के सिवा कोई नहीं जो उसे प्रकट कर दे
\end{hindi}}
\flushright{\begin{Arabic}
\quranayah[53][59]
\end{Arabic}}
\flushleft{\begin{hindi}
अब क्या तुम इस वाणी पर आश्चर्य करते हो;
\end{hindi}}
\flushright{\begin{Arabic}
\quranayah[53][60]
\end{Arabic}}
\flushleft{\begin{hindi}
और हँसते हो और रोते नहीं?
\end{hindi}}
\flushright{\begin{Arabic}
\quranayah[53][61]
\end{Arabic}}
\flushleft{\begin{hindi}
जबकि तुम घमंडी और ग़ाफिल हो
\end{hindi}}
\flushright{\begin{Arabic}
\quranayah[53][62]
\end{Arabic}}
\flushleft{\begin{hindi}
अतः अल्लाह को सजदा करो और बन्दगी करो
\end{hindi}}
\chapter{Al-Qamar (The Moon)}
\begin{Arabic}
\Huge{\centerline{\basmalah}}\end{Arabic}
\flushright{\begin{Arabic}
\quranayah[54][1]
\end{Arabic}}
\flushleft{\begin{hindi}
वह घड़ी निकट और लगी और चाँद फट गया;
\end{hindi}}
\flushright{\begin{Arabic}
\quranayah[54][2]
\end{Arabic}}
\flushleft{\begin{hindi}
किन्तु हाल यह है कि यदि वे कोई निशानी देख भी लें तो टाल जाएँगे और कहेंगे, "यह तो जादू है, पहले से चला आ रहा है!"
\end{hindi}}
\flushright{\begin{Arabic}
\quranayah[54][3]
\end{Arabic}}
\flushleft{\begin{hindi}
उन्होंने झुठलाया और अपनी इच्छाओं का अनुसरण किया; किन्तु हर मामले के लिए एक नियत अवधि है।
\end{hindi}}
\flushright{\begin{Arabic}
\quranayah[54][4]
\end{Arabic}}
\flushleft{\begin{hindi}
उनके पास अतीत को ऐसी खबरें आ चुकी है, जिनमें ताड़ना अर्थात पूर्णतः तत्वदर्शीता है।
\end{hindi}}
\flushright{\begin{Arabic}
\quranayah[54][5]
\end{Arabic}}
\flushleft{\begin{hindi}
किन्तु चेतावनियाँ उनके कुछ काम नहीं आ रही है! -
\end{hindi}}
\flushright{\begin{Arabic}
\quranayah[54][6]
\end{Arabic}}
\flushleft{\begin{hindi}
अतः उनसे रुख़ फेर लो - जिस दिन पुकारनेवाला एक अत्यन्त अप्रिय चीज़ की ओर पुकारेगा;
\end{hindi}}
\flushright{\begin{Arabic}
\quranayah[54][7]
\end{Arabic}}
\flushleft{\begin{hindi}
वे अपनी झुकी हुई निगाहों के साथ अपनी क्रबों से निकल रहे होंगे, मानो वे बिखरी हुई टिड्डियाँ है;
\end{hindi}}
\flushright{\begin{Arabic}
\quranayah[54][8]
\end{Arabic}}
\flushleft{\begin{hindi}
दौड़ पड़ने को पुकारनेवाले की ओर। इनकार करनेवाले कहेंगे, "यह तो एक कठिन दिन है!"
\end{hindi}}
\flushright{\begin{Arabic}
\quranayah[54][9]
\end{Arabic}}
\flushleft{\begin{hindi}
उनसे पहले नूह की क़ौम ने भी झुठलाया। उन्होंने हमारे बन्दे को झूठा ठहराया और कहा, "यह तो दीवाना है!" और वह बुरी तरह झिड़का गया
\end{hindi}}
\flushright{\begin{Arabic}
\quranayah[54][10]
\end{Arabic}}
\flushleft{\begin{hindi}
अन्त में उसने अपने रब को पुकारा कि "मैं दबा हुआ हूँ। अब तू बदला ले।"
\end{hindi}}
\flushright{\begin{Arabic}
\quranayah[54][11]
\end{Arabic}}
\flushleft{\begin{hindi}
तब हमने मूसलाधार बरसते हुए पानी से आकाश के द्वार खोल दिए;
\end{hindi}}
\flushright{\begin{Arabic}
\quranayah[54][12]
\end{Arabic}}
\flushleft{\begin{hindi}
और धरती को प्रवाहित स्रोतों में परिवर्तित कर दिया, और सारा पानी उस काम के लिए मिल गया जो नियत हो चुका था
\end{hindi}}
\flushright{\begin{Arabic}
\quranayah[54][13]
\end{Arabic}}
\flushleft{\begin{hindi}
और हमने उसे एक तख़्तों और कीलोंवाली (नौका) पर सवार किया,
\end{hindi}}
\flushright{\begin{Arabic}
\quranayah[54][14]
\end{Arabic}}
\flushleft{\begin{hindi}
जो हमारी निगाहों के सामने चल रही थी - यह बदला था उस व्यक्ति के लिए जिसकी क़द्र नहीं की गई।
\end{hindi}}
\flushright{\begin{Arabic}
\quranayah[54][15]
\end{Arabic}}
\flushleft{\begin{hindi}
हमने उसे एक निशानी बनाकर छोड़ दिया; फिर क्या कोई नसीहत हासिल करनेवाला?
\end{hindi}}
\flushright{\begin{Arabic}
\quranayah[54][16]
\end{Arabic}}
\flushleft{\begin{hindi}
फिर कैसी रही मेरी यातना और मेरे डरावे?
\end{hindi}}
\flushright{\begin{Arabic}
\quranayah[54][17]
\end{Arabic}}
\flushleft{\begin{hindi}
और हमने क़ुरआन को नसीहत के लिए अनुकूल और सहज बना दिया है। फिर क्या है कोई नसीहत करनेवाला?
\end{hindi}}
\flushright{\begin{Arabic}
\quranayah[54][18]
\end{Arabic}}
\flushleft{\begin{hindi}
आद ने भी झुठलाया, फिर कैसी रही मेरी यातना और मेरा डराना?
\end{hindi}}
\flushright{\begin{Arabic}
\quranayah[54][19]
\end{Arabic}}
\flushleft{\begin{hindi}
निश्चय ही हमने एक निरन्तर अशुभ दिन में तेज़ प्रचंड ठंडी हवा भेजी, उसे उनपर मुसल्लत कर दिया, तो वह लोगों को उखाड़ फेंक रही थी
\end{hindi}}
\flushright{\begin{Arabic}
\quranayah[54][20]
\end{Arabic}}
\flushleft{\begin{hindi}
मानो वे उखड़े खजूर के तने हो
\end{hindi}}
\flushright{\begin{Arabic}
\quranayah[54][21]
\end{Arabic}}
\flushleft{\begin{hindi}
फिर कैसी रही मेरी यातना और मेरे डरावे?
\end{hindi}}
\flushright{\begin{Arabic}
\quranayah[54][22]
\end{Arabic}}
\flushleft{\begin{hindi}
और हमने क़ुरआन को नसीहत के लिए अनुकूल और सहज बना दिया है। फिर क्या है कोई नसीहत हासिल करनेवाला?
\end{hindi}}
\flushright{\begin{Arabic}
\quranayah[54][23]
\end{Arabic}}
\flushleft{\begin{hindi}
समूद ने चेतावनियों को झुठलाया;
\end{hindi}}
\flushright{\begin{Arabic}
\quranayah[54][24]
\end{Arabic}}
\flushleft{\begin{hindi}
और कहने लगे, "एक अकेला आदमी, जो हम ही में से है, क्या हम उसके पीछे चलेंगे? तब तो वास्तव में हम गुमराही और दीवानापन में पड़ गए!
\end{hindi}}
\flushright{\begin{Arabic}
\quranayah[54][25]
\end{Arabic}}
\flushleft{\begin{hindi}
"क्या हमारे बीच उसी पर अनुस्मृति उतारी है? नहीं, बल्कि वह तो परले दरजे का झूठा, बड़ा आत्मश्लाघी है।"
\end{hindi}}
\flushright{\begin{Arabic}
\quranayah[54][26]
\end{Arabic}}
\flushleft{\begin{hindi}
"कल को ही वे जान लेंगे कि कौन परले दरजे का झूठा, बड़ा आत्मश्लाघी है।
\end{hindi}}
\flushright{\begin{Arabic}
\quranayah[54][27]
\end{Arabic}}
\flushleft{\begin{hindi}
हम ऊँटनी को उनके लिए परीक्षा के रूप में भेज रहे है। अतः तुम उन्हें देखते जाओ और धैर्य से काम लो
\end{hindi}}
\flushright{\begin{Arabic}
\quranayah[54][28]
\end{Arabic}}
\flushleft{\begin{hindi}
"और उन्हें सूचित कर दो कि पानी उनके बीच बाँट दिया गया है। हर एक पीने की बारी पर बारीवाला उपस्थित होगा।"
\end{hindi}}
\flushright{\begin{Arabic}
\quranayah[54][29]
\end{Arabic}}
\flushleft{\begin{hindi}
अन्ततः उन्होंने अपने साथी को पुकारा, तो उसने ज़िम्मा लिया फिर उसने उसकी कूचें काट दी
\end{hindi}}
\flushright{\begin{Arabic}
\quranayah[54][30]
\end{Arabic}}
\flushleft{\begin{hindi}
फिर कैसी रही मेरी यातना और मेरे डरावे?
\end{hindi}}
\flushright{\begin{Arabic}
\quranayah[54][31]
\end{Arabic}}
\flushleft{\begin{hindi}
हमने उनपर एक धमाका छोड़ा, फिर वे बाड़ लगानेवाले की रौंदी हुई बाड़ की तरह चूरा होकर रह गए
\end{hindi}}
\flushright{\begin{Arabic}
\quranayah[54][32]
\end{Arabic}}
\flushleft{\begin{hindi}
हमने क़ुरआन को नसीहत के लिए अनुकूल और सहज बना दिया है। फिर क्या कोई नसीहत हासिल करनेवाला?
\end{hindi}}
\flushright{\begin{Arabic}
\quranayah[54][33]
\end{Arabic}}
\flushleft{\begin{hindi}
लूत की क़ौम ने भी चेतावनियों को झुठलाया
\end{hindi}}
\flushright{\begin{Arabic}
\quranayah[54][34]
\end{Arabic}}
\flushleft{\begin{hindi}
हमने लूत के घरवालों के सिवा उनपर पथराव करनेवाली तेज़ वायु भेजी।
\end{hindi}}
\flushright{\begin{Arabic}
\quranayah[54][35]
\end{Arabic}}
\flushleft{\begin{hindi}
हमने अपनी विशेष अनुकम्पा से प्रातःकाल उन्हें बचा लिया। हम इसी तरह उस व्यक्ति को बदला देते है जो कृतज्ञता दिखाए
\end{hindi}}
\flushright{\begin{Arabic}
\quranayah[54][36]
\end{Arabic}}
\flushleft{\begin{hindi}
उसने जो उन्हें हमारी पकड़ से सावधान कर दिया था। किन्तु वे चेतावनियों के विषय में संदेह करते रहे
\end{hindi}}
\flushright{\begin{Arabic}
\quranayah[54][37]
\end{Arabic}}
\flushleft{\begin{hindi}
उन्होंने उसे फुसलाकर उसके पास से उसके अतिथियों को बलाना चाहा। अन्ततः हमने उसकी आँखें मेट दीं, "लो, अब चखो मज़ा मेरी यातना और चेतावनियों का!"
\end{hindi}}
\flushright{\begin{Arabic}
\quranayah[54][38]
\end{Arabic}}
\flushleft{\begin{hindi}
सुबह सवेरे ही एक अटल यातना उनपर आ पहुँची,
\end{hindi}}
\flushright{\begin{Arabic}
\quranayah[54][39]
\end{Arabic}}
\flushleft{\begin{hindi}
"लो, अब चखो मज़ा मेरी यातना और चेतावनियों का!"
\end{hindi}}
\flushright{\begin{Arabic}
\quranayah[54][40]
\end{Arabic}}
\flushleft{\begin{hindi}
और हमने क़ुरआन को नसीहत के लिए अनुकूल और सहज बना दिया है। फिर क्या है कोई नसीहत हासिल करनेवाला?
\end{hindi}}
\flushright{\begin{Arabic}
\quranayah[54][41]
\end{Arabic}}
\flushleft{\begin{hindi}
और फ़िरऔनियों के पास चेतावनियाँ आई;
\end{hindi}}
\flushright{\begin{Arabic}
\quranayah[54][42]
\end{Arabic}}
\flushleft{\begin{hindi}
उन्होंने हमारी सारी निशानियों को झुठला दिया। अन्ततः हमने उन्हें पकड़ लिया, जिस प्रकार एक ज़बरदस्त प्रभुत्वशाली पकड़ता है
\end{hindi}}
\flushright{\begin{Arabic}
\quranayah[54][43]
\end{Arabic}}
\flushleft{\begin{hindi}
क्या तुम्हारे काफ़िर कुछ उन लोगो से अच्छे है या किताबों में तुम्हारे लिए कोई छुटकारा लिखा हुआ है?
\end{hindi}}
\flushright{\begin{Arabic}
\quranayah[54][44]
\end{Arabic}}
\flushleft{\begin{hindi}
या वे कहते है, "और हम मुक़ाबले की शक्ति रखनेवाले एक जत्था है?"
\end{hindi}}
\flushright{\begin{Arabic}
\quranayah[54][45]
\end{Arabic}}
\flushleft{\begin{hindi}
शीघ्र ही वह जत्था पराजित होकर रहेगा और वे पीठ दिखा जाएँगे
\end{hindi}}
\flushright{\begin{Arabic}
\quranayah[54][46]
\end{Arabic}}
\flushleft{\begin{hindi}
नहीं, बल्कि वह घड़ी है, जिसका समय उनके लिए नियत है और वह बड़ी आपदावाली और कटु घड़ी है!
\end{hindi}}
\flushright{\begin{Arabic}
\quranayah[54][47]
\end{Arabic}}
\flushleft{\begin{hindi}
निस्संदेह, अपराधी लोग गुमराही और दीवानेपन में पड़े हुए है
\end{hindi}}
\flushright{\begin{Arabic}
\quranayah[54][48]
\end{Arabic}}
\flushleft{\begin{hindi}
जिस दिन वे अपने मुँह के बल आग में घसीटे जाएँगे, "चखो मज़ा आग की लपट का!"
\end{hindi}}
\flushright{\begin{Arabic}
\quranayah[54][49]
\end{Arabic}}
\flushleft{\begin{hindi}
निश्चय ही हमने हर चीज़ एक अंदाज़े के साथ पैदा की है
\end{hindi}}
\flushright{\begin{Arabic}
\quranayah[54][50]
\end{Arabic}}
\flushleft{\begin{hindi}
और हमारा आदेश (और काम) तो बस एक दम की बात होती है जैसे आँख का झपकना
\end{hindi}}
\flushright{\begin{Arabic}
\quranayah[54][51]
\end{Arabic}}
\flushleft{\begin{hindi}
और हम तुम्हारे जैसे लोगों को विनष्ट कर चुके है। फिर क्या है कोई नसीहत हासिल करनेवाला?
\end{hindi}}
\flushright{\begin{Arabic}
\quranayah[54][52]
\end{Arabic}}
\flushleft{\begin{hindi}
जो कुछ उन्होंने किया है, वह पन्नों में अंकित है
\end{hindi}}
\flushright{\begin{Arabic}
\quranayah[54][53]
\end{Arabic}}
\flushleft{\begin{hindi}
और हर छोटी और बड़ी चीज़ लिखित है
\end{hindi}}
\flushright{\begin{Arabic}
\quranayah[54][54]
\end{Arabic}}
\flushleft{\begin{hindi}
निश्चय ही डर रखनेवाले बाग़ो और नहरों के बीच होंगे,
\end{hindi}}
\flushright{\begin{Arabic}
\quranayah[54][55]
\end{Arabic}}
\flushleft{\begin{hindi}
प्रतिष्ठित स्थान पर, प्रभुत्वशाली सम्राट के निकट
\end{hindi}}
\chapter{Ar-Rahman (The Beneficent)}
\begin{Arabic}
\Huge{\centerline{\basmalah}}\end{Arabic}
\flushright{\begin{Arabic}
\quranayah[55][1]
\end{Arabic}}
\flushleft{\begin{hindi}
रहमान ने
\end{hindi}}
\flushright{\begin{Arabic}
\quranayah[55][2]
\end{Arabic}}
\flushleft{\begin{hindi}
क़ुरआन सिखाया;
\end{hindi}}
\flushright{\begin{Arabic}
\quranayah[55][3]
\end{Arabic}}
\flushleft{\begin{hindi}
उसी ने मनुष्य को पैदा किया;
\end{hindi}}
\flushright{\begin{Arabic}
\quranayah[55][4]
\end{Arabic}}
\flushleft{\begin{hindi}
उसे बोलना सिखाया;
\end{hindi}}
\flushright{\begin{Arabic}
\quranayah[55][5]
\end{Arabic}}
\flushleft{\begin{hindi}
सूर्य और चन्द्रमा एक हिसाब के पाबन्द है;
\end{hindi}}
\flushright{\begin{Arabic}
\quranayah[55][6]
\end{Arabic}}
\flushleft{\begin{hindi}
और तारे और वृक्ष सजदा करते है;
\end{hindi}}
\flushright{\begin{Arabic}
\quranayah[55][7]
\end{Arabic}}
\flushleft{\begin{hindi}
उसने आकाश को ऊँचा किया और संतुलन स्थापित किया -
\end{hindi}}
\flushright{\begin{Arabic}
\quranayah[55][8]
\end{Arabic}}
\flushleft{\begin{hindi}
कि तुम भी तुला में सीमा का उल्लंघन न करो
\end{hindi}}
\flushright{\begin{Arabic}
\quranayah[55][9]
\end{Arabic}}
\flushleft{\begin{hindi}
न्याय के साथ ठीक-ठीक तौलो और तौल में कमी न करो। -
\end{hindi}}
\flushright{\begin{Arabic}
\quranayah[55][10]
\end{Arabic}}
\flushleft{\begin{hindi}
और धरती को उसने सृष्टल प्राणियों के लिए बनाया;
\end{hindi}}
\flushright{\begin{Arabic}
\quranayah[55][11]
\end{Arabic}}
\flushleft{\begin{hindi}
उसमें स्वादिष्ट फल है और खजूर के वृक्ष है, जिनके फल आवरणों में लिपटे हुए है,
\end{hindi}}
\flushright{\begin{Arabic}
\quranayah[55][12]
\end{Arabic}}
\flushleft{\begin{hindi}
और भुसवाले अनाज भी और सुगंधित बेल-बूटा भी
\end{hindi}}
\flushright{\begin{Arabic}
\quranayah[55][13]
\end{Arabic}}
\flushleft{\begin{hindi}
तो तुम दोनों अपने रब की अनुकम्पाओं में से किस-किस को झुठलाओगे?
\end{hindi}}
\flushright{\begin{Arabic}
\quranayah[55][14]
\end{Arabic}}
\flushleft{\begin{hindi}
उसने मनुष्य को ठीकरी जैसी खनखनाती हुए मिट्टी से पैदा किया;
\end{hindi}}
\flushright{\begin{Arabic}
\quranayah[55][15]
\end{Arabic}}
\flushleft{\begin{hindi}
और जिन्न को उसने आग की लपट से पैदा किया
\end{hindi}}
\flushright{\begin{Arabic}
\quranayah[55][16]
\end{Arabic}}
\flushleft{\begin{hindi}
फिर तुम दोनों अपने रब की सामर्थ्यों में से किस-किस को झुठलाओगे?
\end{hindi}}
\flushright{\begin{Arabic}
\quranayah[55][17]
\end{Arabic}}
\flushleft{\begin{hindi}
वह दो पूर्व का रब है और दो पश्चिम का रब भी।
\end{hindi}}
\flushright{\begin{Arabic}
\quranayah[55][18]
\end{Arabic}}
\flushleft{\begin{hindi}
फिर तुम दोनों अपने रब की महानताओं में से किस-किस को झुठलाओगे?
\end{hindi}}
\flushright{\begin{Arabic}
\quranayah[55][19]
\end{Arabic}}
\flushleft{\begin{hindi}
उसने दो समुद्रो को प्रवाहित कर दिया, जो आपस में मिल रहे होते है।
\end{hindi}}
\flushright{\begin{Arabic}
\quranayah[55][20]
\end{Arabic}}
\flushleft{\begin{hindi}
उन दोनों के बीच एक परदा बाधक होता है, जिसका वे अतिक्रमण नहीं करते
\end{hindi}}
\flushright{\begin{Arabic}
\quranayah[55][21]
\end{Arabic}}
\flushleft{\begin{hindi}
तो तुम दोनों अपने रब के चमत्कारों में से किस-किस को झुठलाओगे?
\end{hindi}}
\flushright{\begin{Arabic}
\quranayah[55][22]
\end{Arabic}}
\flushleft{\begin{hindi}
उन (समुद्रों) से मोती और मूँगा निकलता है।
\end{hindi}}
\flushright{\begin{Arabic}
\quranayah[55][23]
\end{Arabic}}
\flushleft{\begin{hindi}
अतः तुम दोनों अपने रब के चमत्कारों में से किस-किस को झुठलाओगे?
\end{hindi}}
\flushright{\begin{Arabic}
\quranayah[55][24]
\end{Arabic}}
\flushleft{\begin{hindi}
उसी के बस में है समुद्र में पहाड़ो की तरह उठे हुए जहाज़
\end{hindi}}
\flushright{\begin{Arabic}
\quranayah[55][25]
\end{Arabic}}
\flushleft{\begin{hindi}
तो तुम दोनों अपने रब की अनुकम्पाओं में से किस-किस को झुठलाओग?
\end{hindi}}
\flushright{\begin{Arabic}
\quranayah[55][26]
\end{Arabic}}
\flushleft{\begin{hindi}
प्रत्येक जो भी इस (धरती) पर है, नाशवान है
\end{hindi}}
\flushright{\begin{Arabic}
\quranayah[55][27]
\end{Arabic}}
\flushleft{\begin{hindi}
किन्तु तुम्हारे रब का प्रतापवान और उदार स्वरूप शेष रहनेवाला है
\end{hindi}}
\flushright{\begin{Arabic}
\quranayah[55][28]
\end{Arabic}}
\flushleft{\begin{hindi}
अतः तुम दोनों अपने रब के चमत्कारों में से किस-किस को झुठलाओगं?
\end{hindi}}
\flushright{\begin{Arabic}
\quranayah[55][29]
\end{Arabic}}
\flushleft{\begin{hindi}
आकाशों और धरती में जो भी है उसी से माँगता है। उसकी नित्य नई शान है
\end{hindi}}
\flushright{\begin{Arabic}
\quranayah[55][30]
\end{Arabic}}
\flushleft{\begin{hindi}
अतः तुम दोनों अपने रब की अनुकम्पाओं में से किस-किस को झुठलाओगे?
\end{hindi}}
\flushright{\begin{Arabic}
\quranayah[55][31]
\end{Arabic}}
\flushleft{\begin{hindi}
ऐ दोनों बोझों! शीघ्र ही हम तुम्हारे लिए निवृत हुए जाते है
\end{hindi}}
\flushright{\begin{Arabic}
\quranayah[55][32]
\end{Arabic}}
\flushleft{\begin{hindi}
तो तुम दोनों अपने रब की अनुकम्पाओं में से किस-किस को झुठलाओगे?
\end{hindi}}
\flushright{\begin{Arabic}
\quranayah[55][33]
\end{Arabic}}
\flushleft{\begin{hindi}
ऐ जिन्नों और मनुष्यों के गिरोह! यदि तुममें हो सके कि आकाशों और धरती की सीमाओं को पार कर सको, तो पार कर जाओ; तुम कदापि पार नहीं कर सकते बिना अधिकार-शक्ति के
\end{hindi}}
\flushright{\begin{Arabic}
\quranayah[55][34]
\end{Arabic}}
\flushleft{\begin{hindi}
अतः तुम दोनों अपने रब की सामर्थ्यों में से किस-किस को झुठलाओगे?
\end{hindi}}
\flushright{\begin{Arabic}
\quranayah[55][35]
\end{Arabic}}
\flushleft{\begin{hindi}
अतः तुम दोनों पर अग्नि-ज्वाला और धुएँवाला अंगारा (पिघला ताँबा) छोड़ दिया जाएगा, फिर तुम मुक़ाबला न कर सकोगे।
\end{hindi}}
\flushright{\begin{Arabic}
\quranayah[55][36]
\end{Arabic}}
\flushleft{\begin{hindi}
अतः तुम दोनों अपने रब की सामर्थ्यों में से किस-किस को झुठलाओगे?
\end{hindi}}
\flushright{\begin{Arabic}
\quranayah[55][37]
\end{Arabic}}
\flushleft{\begin{hindi}
फिर जब आकाश फट जाएगा और लाल चमड़े की तरह लाल हो जाएगा।
\end{hindi}}
\flushright{\begin{Arabic}
\quranayah[55][38]
\end{Arabic}}
\flushleft{\begin{hindi}
- अतः तुम दोनों अपने रब के चमत्कारों में से किस-किस को झुठलाओगे?
\end{hindi}}
\flushright{\begin{Arabic}
\quranayah[55][39]
\end{Arabic}}
\flushleft{\begin{hindi}
फिर उस दिन न किसी मनुष्य से उसके गुनाह के विषय में पूछा जाएगा न किसी जिन्न से
\end{hindi}}
\flushright{\begin{Arabic}
\quranayah[55][40]
\end{Arabic}}
\flushleft{\begin{hindi}
अतः तुम दोनों अपने रब के चमत्कारों में से किस-किस को झुठलाओगे?
\end{hindi}}
\flushright{\begin{Arabic}
\quranayah[55][41]
\end{Arabic}}
\flushleft{\begin{hindi}
अपराधी अपने चहरों से पहचान लिए जाएँगे और उनके माथे के बालों और टाँगों द्वारा पकड़ लिया जाएगा
\end{hindi}}
\flushright{\begin{Arabic}
\quranayah[55][42]
\end{Arabic}}
\flushleft{\begin{hindi}
अतः तुम दोनों अपने रब की सामर्थ्यों में से किस-किस को झुठलाओगे?
\end{hindi}}
\flushright{\begin{Arabic}
\quranayah[55][43]
\end{Arabic}}
\flushleft{\begin{hindi}
यही वह जहन्नम है जिसे अपराधी लोग झूठ ठहराते रहे है
\end{hindi}}
\flushright{\begin{Arabic}
\quranayah[55][44]
\end{Arabic}}
\flushleft{\begin{hindi}
वे उनके और खौलते हुए पानी के बीच चक्कर लगा रहें होंगे
\end{hindi}}
\flushright{\begin{Arabic}
\quranayah[55][45]
\end{Arabic}}
\flushleft{\begin{hindi}
फिर तुम दोनों अपने रब के सामर्थ्यों में से किस-किस को झुठलाओगे?
\end{hindi}}
\flushright{\begin{Arabic}
\quranayah[55][46]
\end{Arabic}}
\flushleft{\begin{hindi}
किन्तु जो अपने रब के सामने खड़े होने का डर रखता होगा, उसके लिए दो बाग़ है। -
\end{hindi}}
\flushright{\begin{Arabic}
\quranayah[55][47]
\end{Arabic}}
\flushleft{\begin{hindi}
तो तुम दोनों अपने रब की अनुकम्पाओं में से किस-किस को झुठलाओगे?
\end{hindi}}
\flushright{\begin{Arabic}
\quranayah[55][48]
\end{Arabic}}
\flushleft{\begin{hindi}
घनी डालियोंवाले;
\end{hindi}}
\flushright{\begin{Arabic}
\quranayah[55][49]
\end{Arabic}}
\flushleft{\begin{hindi}
अतः तुम दोनों अपने रब के उपकारों में से किस-किस को झुठलाओगे?
\end{hindi}}
\flushright{\begin{Arabic}
\quranayah[55][50]
\end{Arabic}}
\flushleft{\begin{hindi}
उन दोनो (बाग़ो) में दो प्रवाहित स्रोत है।
\end{hindi}}
\flushright{\begin{Arabic}
\quranayah[55][51]
\end{Arabic}}
\flushleft{\begin{hindi}
अतः तुम दोनों अपने रब की अनुकम्पाओं में से किस-किस को झुठलाओगे?
\end{hindi}}
\flushright{\begin{Arabic}
\quranayah[55][52]
\end{Arabic}}
\flushleft{\begin{hindi}
उन दोनों (बाग़ो) मे हर स्वादिष्ट फल की दो-दो किस्में हैं;
\end{hindi}}
\flushright{\begin{Arabic}
\quranayah[55][53]
\end{Arabic}}
\flushleft{\begin{hindi}
अतः तुम दोनो रब के चमत्कारों में से किस-किस को झुठलाओगे?
\end{hindi}}
\flushright{\begin{Arabic}
\quranayah[55][54]
\end{Arabic}}
\flushleft{\begin{hindi}
वे ऐसे बिछौनो पर तकिया लगाए हुए होंगे जिनके अस्तर गाढे रेशम के होंगे, और दोनों बाग़ो के फल झुके हुए निकट ही होंगे।
\end{hindi}}
\flushright{\begin{Arabic}
\quranayah[55][55]
\end{Arabic}}
\flushleft{\begin{hindi}
अतः तुम अपने रब के चमत्कारों में से किस-किस को झुठलाओगे?
\end{hindi}}
\flushright{\begin{Arabic}
\quranayah[55][56]
\end{Arabic}}
\flushleft{\begin{hindi}
उन (अनुकम्पाओं) में निगाह बचाए रखनेवाली (सुन्दर) स्त्रियाँ होंगी, जिन्हें उनसे पहले न किसी मनुष्य ने हाथ लगाया और न किसी जिन्न ने
\end{hindi}}
\flushright{\begin{Arabic}
\quranayah[55][57]
\end{Arabic}}
\flushleft{\begin{hindi}
फिर तुम दोनों अपने रब की अनुकम्पाओं में से किस-किस को झुठलाओगे?
\end{hindi}}
\flushright{\begin{Arabic}
\quranayah[55][58]
\end{Arabic}}
\flushleft{\begin{hindi}
मानो वे लाल (याकूत) और प्रवाल (मूँगा) है।
\end{hindi}}
\flushright{\begin{Arabic}
\quranayah[55][59]
\end{Arabic}}
\flushleft{\begin{hindi}
अतः तुम दोनों अपने रब की अनुकम्पाओं में से किस-किस को झुठलाओगे?
\end{hindi}}
\flushright{\begin{Arabic}
\quranayah[55][60]
\end{Arabic}}
\flushleft{\begin{hindi}
अच्छाई का बदला अच्छाई के सिवा और क्या हो सकता है?
\end{hindi}}
\flushright{\begin{Arabic}
\quranayah[55][61]
\end{Arabic}}
\flushleft{\begin{hindi}
अतः तुम दोनों अपने रब की अनुकम्पाओं में से किस-किस को झुठलाओगे?
\end{hindi}}
\flushright{\begin{Arabic}
\quranayah[55][62]
\end{Arabic}}
\flushleft{\begin{hindi}
उन दोनों से हटकर दो और बाग़ है।
\end{hindi}}
\flushright{\begin{Arabic}
\quranayah[55][63]
\end{Arabic}}
\flushleft{\begin{hindi}
फिर तुम दोनों अपने रब की अनुकम्पाओं में से किस-किस को झुठलाओगे?
\end{hindi}}
\flushright{\begin{Arabic}
\quranayah[55][64]
\end{Arabic}}
\flushleft{\begin{hindi}
गहरे हरित;
\end{hindi}}
\flushright{\begin{Arabic}
\quranayah[55][65]
\end{Arabic}}
\flushleft{\begin{hindi}
अतः तुम दोनों अपने रब की अनुकम्पाओं में से किस-किस को झुठलाओगे?
\end{hindi}}
\flushright{\begin{Arabic}
\quranayah[55][66]
\end{Arabic}}
\flushleft{\begin{hindi}
उन दोनों (बाग़ो) में दो स्रोत है जोश मारते हुए
\end{hindi}}
\flushright{\begin{Arabic}
\quranayah[55][67]
\end{Arabic}}
\flushleft{\begin{hindi}
अतः तुम दोनों अपने रब के चमत्कारों में से किस-किस को झुठलाओगे?
\end{hindi}}
\flushright{\begin{Arabic}
\quranayah[55][68]
\end{Arabic}}
\flushleft{\begin{hindi}
उनमें है स्वादिष्ट फल और खजूर और अनार;
\end{hindi}}
\flushright{\begin{Arabic}
\quranayah[55][69]
\end{Arabic}}
\flushleft{\begin{hindi}
अतः तुम दोनों अपने रब की अनुकम्पाओं में से किस-किस को झुठलाओगे?
\end{hindi}}
\flushright{\begin{Arabic}
\quranayah[55][70]
\end{Arabic}}
\flushleft{\begin{hindi}
उनमें भली और सुन्दर स्त्रियाँ होंगी।
\end{hindi}}
\flushright{\begin{Arabic}
\quranayah[55][71]
\end{Arabic}}
\flushleft{\begin{hindi}
तो तुम दोनों अपने रब की अनुकम्पाओं में से किस-किस को झुठलाओगे?
\end{hindi}}
\flushright{\begin{Arabic}
\quranayah[55][72]
\end{Arabic}}
\flushleft{\begin{hindi}
हूरें (परम रूपवती स्त्रियाँ) ख़ेमों में रहनेवाली;
\end{hindi}}
\flushright{\begin{Arabic}
\quranayah[55][73]
\end{Arabic}}
\flushleft{\begin{hindi}
अतः तुम दोनों अपने रब के चमत्कारों में से किस-किस को झुठलाओगे?
\end{hindi}}
\flushright{\begin{Arabic}
\quranayah[55][74]
\end{Arabic}}
\flushleft{\begin{hindi}
जिन्हें उससे पहले न किसी मनुष्य ने हाथ लगाया होगा और न किसी जिन्न ने।
\end{hindi}}
\flushright{\begin{Arabic}
\quranayah[55][75]
\end{Arabic}}
\flushleft{\begin{hindi}
अतः तुम दोनों अपने रब की अनुकम्पाओं में से किस-किस को झुठलाओगे?
\end{hindi}}
\flushright{\begin{Arabic}
\quranayah[55][76]
\end{Arabic}}
\flushleft{\begin{hindi}
वे हरे रेशमी गद्दो और उत्कृष्ट् और असाधारण क़ालीनों पर तकिया लगाए होंगे;
\end{hindi}}
\flushright{\begin{Arabic}
\quranayah[55][77]
\end{Arabic}}
\flushleft{\begin{hindi}
अतः तुम दोनों अपने रब की अनुकम्पाओं में से किस-किस को झुठलाओगे?
\end{hindi}}
\flushright{\begin{Arabic}
\quranayah[55][78]
\end{Arabic}}
\flushleft{\begin{hindi}
बड़ा ही बरकतवाला नाम है तुम्हारे प्रतापवान और उदार रब का
\end{hindi}}
\chapter{Al-Waqi'ah (The Event)}
\begin{Arabic}
\Huge{\centerline{\basmalah}}\end{Arabic}
\flushright{\begin{Arabic}
\quranayah[56][1]
\end{Arabic}}
\flushleft{\begin{hindi}
जब घटित होनेवाली (घड़ी) घटित हो जाएगी;
\end{hindi}}
\flushright{\begin{Arabic}
\quranayah[56][2]
\end{Arabic}}
\flushleft{\begin{hindi}
उसके घटित होने में कुछ भी झुठ नहीं;
\end{hindi}}
\flushright{\begin{Arabic}
\quranayah[56][3]
\end{Arabic}}
\flushleft{\begin{hindi}
पस्त करनेवाली होगी, ऊँचा करनेवाली थी;
\end{hindi}}
\flushright{\begin{Arabic}
\quranayah[56][4]
\end{Arabic}}
\flushleft{\begin{hindi}
जब धरती थरथराकर काँप उठेगी;
\end{hindi}}
\flushright{\begin{Arabic}
\quranayah[56][5]
\end{Arabic}}
\flushleft{\begin{hindi}
और पहाड़ टूटकर चूर्ण-विचुर्ण हो जाएँगे
\end{hindi}}
\flushright{\begin{Arabic}
\quranayah[56][6]
\end{Arabic}}
\flushleft{\begin{hindi}
कि वे बिखरे हुए धूल होकर रह जाएँगे
\end{hindi}}
\flushright{\begin{Arabic}
\quranayah[56][7]
\end{Arabic}}
\flushleft{\begin{hindi}
और तुम लोग तीन प्रकार के हो जाओगे -
\end{hindi}}
\flushright{\begin{Arabic}
\quranayah[56][8]
\end{Arabic}}
\flushleft{\begin{hindi}
तो दाहिने हाथ वाले (सौभाग्यशाली), कैसे होंगे दाहिने हाथ वाले!
\end{hindi}}
\flushright{\begin{Arabic}
\quranayah[56][9]
\end{Arabic}}
\flushleft{\begin{hindi}
और बाएँ हाथ वाले (दुर्भाग्यशाली), कैसे होंगे बाएँ हाथ वाले!
\end{hindi}}
\flushright{\begin{Arabic}
\quranayah[56][10]
\end{Arabic}}
\flushleft{\begin{hindi}
और आगे बढ़ जानेवाले तो आगे बढ़ जानेवाले ही है
\end{hindi}}
\flushright{\begin{Arabic}
\quranayah[56][11]
\end{Arabic}}
\flushleft{\begin{hindi}
वही (अल्लाह के) निकटवर्ती है;
\end{hindi}}
\flushright{\begin{Arabic}
\quranayah[56][12]
\end{Arabic}}
\flushleft{\begin{hindi}
नेमत भरी जन्नतों में होंगे;
\end{hindi}}
\flushright{\begin{Arabic}
\quranayah[56][13]
\end{Arabic}}
\flushleft{\begin{hindi}
अगलों में से तो बहुत-से होंगे,
\end{hindi}}
\flushright{\begin{Arabic}
\quranayah[56][14]
\end{Arabic}}
\flushleft{\begin{hindi}
किन्तु पिछलों में से कम ही
\end{hindi}}
\flushright{\begin{Arabic}
\quranayah[56][15]
\end{Arabic}}
\flushleft{\begin{hindi}
जड़ित तख़्तो पर;
\end{hindi}}
\flushright{\begin{Arabic}
\quranayah[56][16]
\end{Arabic}}
\flushleft{\begin{hindi}
तकिया लगाए आमने-सामने होंगे;
\end{hindi}}
\flushright{\begin{Arabic}
\quranayah[56][17]
\end{Arabic}}
\flushleft{\begin{hindi}
उनके पास किशोर होंगे जो सदैव किशोरावस्था ही में रहेंगे,
\end{hindi}}
\flushright{\begin{Arabic}
\quranayah[56][18]
\end{Arabic}}
\flushleft{\begin{hindi}
प्याले और आफ़ताबे (जग) और विशुद्ध पेय से भरा हुआ पात्र लिए फिर रहे होंगे
\end{hindi}}
\flushright{\begin{Arabic}
\quranayah[56][19]
\end{Arabic}}
\flushleft{\begin{hindi}
- जिस (के पीने) से न तो उन्हें सिर दर्द होगा और न उनकी बुद्धि में विकार आएगा
\end{hindi}}
\flushright{\begin{Arabic}
\quranayah[56][20]
\end{Arabic}}
\flushleft{\begin{hindi}
और स्वादिष्ट॥ फल जो वे पसन्द करें;
\end{hindi}}
\flushright{\begin{Arabic}
\quranayah[56][21]
\end{Arabic}}
\flushleft{\begin{hindi}
और पक्षी का मांस जो वे चाह;
\end{hindi}}
\flushright{\begin{Arabic}
\quranayah[56][22]
\end{Arabic}}
\flushleft{\begin{hindi}
और बड़ी आँखोंवाली हूरें,
\end{hindi}}
\flushright{\begin{Arabic}
\quranayah[56][23]
\end{Arabic}}
\flushleft{\begin{hindi}
मानो छिपाए हुए मोती हो
\end{hindi}}
\flushright{\begin{Arabic}
\quranayah[56][24]
\end{Arabic}}
\flushleft{\begin{hindi}
यह सब उसके बदले में उन्हें प्राप्त होगा जो कुछ वे करते रहे
\end{hindi}}
\flushright{\begin{Arabic}
\quranayah[56][25]
\end{Arabic}}
\flushleft{\begin{hindi}
उसमें वे न कोई व्यर्थ बात सुनेंगे और न गुनाह की बात;
\end{hindi}}
\flushright{\begin{Arabic}
\quranayah[56][26]
\end{Arabic}}
\flushleft{\begin{hindi}
सिवाय इस बात के कि "सलाम हो, सलाम हो!"
\end{hindi}}
\flushright{\begin{Arabic}
\quranayah[56][27]
\end{Arabic}}
\flushleft{\begin{hindi}
रहे सौभाग्यशाली लोग, तो सौभाग्यशालियों का क्या कहना!
\end{hindi}}
\flushright{\begin{Arabic}
\quranayah[56][28]
\end{Arabic}}
\flushleft{\begin{hindi}
वे वहाँ होंगे जहाँ बिन काँटों के बेर होंगे;
\end{hindi}}
\flushright{\begin{Arabic}
\quranayah[56][29]
\end{Arabic}}
\flushleft{\begin{hindi}
और गुच्छेदार केले;
\end{hindi}}
\flushright{\begin{Arabic}
\quranayah[56][30]
\end{Arabic}}
\flushleft{\begin{hindi}
दूर तक फैली हुई छाँव;
\end{hindi}}
\flushright{\begin{Arabic}
\quranayah[56][31]
\end{Arabic}}
\flushleft{\begin{hindi}
बहता हुआ पानी;
\end{hindi}}
\flushright{\begin{Arabic}
\quranayah[56][32]
\end{Arabic}}
\flushleft{\begin{hindi}
बहुत-सा स्वादिष्ट; फल,
\end{hindi}}
\flushright{\begin{Arabic}
\quranayah[56][33]
\end{Arabic}}
\flushleft{\begin{hindi}
जिसका सिलसिला टूटनेवाला न होगा और न उसपर कोई रोक-टोक होगी
\end{hindi}}
\flushright{\begin{Arabic}
\quranayah[56][34]
\end{Arabic}}
\flushleft{\begin{hindi}
उच्चकोटि के बिछौने होंगे;
\end{hindi}}
\flushright{\begin{Arabic}
\quranayah[56][35]
\end{Arabic}}
\flushleft{\begin{hindi}
(और वहाँ उनकी पत्नियों को) निश्चय ही हमने एक विशेष उठान पर उठान पर उठाया
\end{hindi}}
\flushright{\begin{Arabic}
\quranayah[56][36]
\end{Arabic}}
\flushleft{\begin{hindi}
और हमने उन्हे कुँवारियाँ बनाया;
\end{hindi}}
\flushright{\begin{Arabic}
\quranayah[56][37]
\end{Arabic}}
\flushleft{\begin{hindi}
प्रेम दर्शानेवाली और समायु;
\end{hindi}}
\flushright{\begin{Arabic}
\quranayah[56][38]
\end{Arabic}}
\flushleft{\begin{hindi}
सौभाग्यशाली लोगों के लिए;
\end{hindi}}
\flushright{\begin{Arabic}
\quranayah[56][39]
\end{Arabic}}
\flushleft{\begin{hindi}
वे अगलों में से भी अधिक होगे
\end{hindi}}
\flushright{\begin{Arabic}
\quranayah[56][40]
\end{Arabic}}
\flushleft{\begin{hindi}
और पिछलों में से भी अधिक होंगे
\end{hindi}}
\flushright{\begin{Arabic}
\quranayah[56][41]
\end{Arabic}}
\flushleft{\begin{hindi}
रहे दुर्भाग्यशाली लोग, तो कैसे होंगे दुर्भाग्यशाली लोग!
\end{hindi}}
\flushright{\begin{Arabic}
\quranayah[56][42]
\end{Arabic}}
\flushleft{\begin{hindi}
गर्म हवा और खौलते हुए पानी में होंगे;
\end{hindi}}
\flushright{\begin{Arabic}
\quranayah[56][43]
\end{Arabic}}
\flushleft{\begin{hindi}
और काले धुएँ की छाँव में,
\end{hindi}}
\flushright{\begin{Arabic}
\quranayah[56][44]
\end{Arabic}}
\flushleft{\begin{hindi}
जो न ठंडी होगी और न उत्तम और लाभप्रद
\end{hindi}}
\flushright{\begin{Arabic}
\quranayah[56][45]
\end{Arabic}}
\flushleft{\begin{hindi}
वे इससे पहले सुख-सम्पन्न थे;
\end{hindi}}
\flushright{\begin{Arabic}
\quranayah[56][46]
\end{Arabic}}
\flushleft{\begin{hindi}
और बड़े गुनाह पर अड़े रहते थे
\end{hindi}}
\flushright{\begin{Arabic}
\quranayah[56][47]
\end{Arabic}}
\flushleft{\begin{hindi}
कहते थे, "क्या जब हम मर जाएँगे और मिट्टी और हड्डियाँ होकर रहे जाएँगे, तो क्या हम वास्तव में उठाए जाएँगे?
\end{hindi}}
\flushright{\begin{Arabic}
\quranayah[56][48]
\end{Arabic}}
\flushleft{\begin{hindi}
"और क्या हमारे पहले के बाप-दादा भी?"
\end{hindi}}
\flushright{\begin{Arabic}
\quranayah[56][49]
\end{Arabic}}
\flushleft{\begin{hindi}
कह दो, "निश्चय ही अगले और पिछले भी
\end{hindi}}
\flushright{\begin{Arabic}
\quranayah[56][50]
\end{Arabic}}
\flushleft{\begin{hindi}
एक नियत समय पर इकट्ठे कर दिए जाएँगे, जिसका दिन ज्ञात और नियत है
\end{hindi}}
\flushright{\begin{Arabic}
\quranayah[56][51]
\end{Arabic}}
\flushleft{\begin{hindi}
"फिर तुम ऐ गुमराहो, झुठलानेवालो!
\end{hindi}}
\flushright{\begin{Arabic}
\quranayah[56][52]
\end{Arabic}}
\flushleft{\begin{hindi}
ज़क्कूम के वृक्ष में से खाओंगे;
\end{hindi}}
\flushright{\begin{Arabic}
\quranayah[56][53]
\end{Arabic}}
\flushleft{\begin{hindi}
"और उसी से पेट भरोगे;
\end{hindi}}
\flushright{\begin{Arabic}
\quranayah[56][54]
\end{Arabic}}
\flushleft{\begin{hindi}
"और उसके ऊपर से खौलता हुआ पानी पीओगे;
\end{hindi}}
\flushright{\begin{Arabic}
\quranayah[56][55]
\end{Arabic}}
\flushleft{\begin{hindi}
"और तौस लगे ऊँट की तरह पीओगे।"
\end{hindi}}
\flushright{\begin{Arabic}
\quranayah[56][56]
\end{Arabic}}
\flushleft{\begin{hindi}
यह बदला दिए जाने के दिन उनका पहला सत्कार होगा
\end{hindi}}
\flushright{\begin{Arabic}
\quranayah[56][57]
\end{Arabic}}
\flushleft{\begin{hindi}
हमने तुम्हें पैदा किया; फिर तुम सच क्यों नहीं मानते?
\end{hindi}}
\flushright{\begin{Arabic}
\quranayah[56][58]
\end{Arabic}}
\flushleft{\begin{hindi}
तो क्या तुमने विचार किया जो चीज़ तुम टपकाते हो?
\end{hindi}}
\flushright{\begin{Arabic}
\quranayah[56][59]
\end{Arabic}}
\flushleft{\begin{hindi}
क्या तुम उसे आकार देते हो, या हम है आकार देनेवाले?
\end{hindi}}
\flushright{\begin{Arabic}
\quranayah[56][60]
\end{Arabic}}
\flushleft{\begin{hindi}
हमने तुम्हारे बीच मृत्यु को नियत किया है और हमारे बस से यह बाहर नहीं है
\end{hindi}}
\flushright{\begin{Arabic}
\quranayah[56][61]
\end{Arabic}}
\flushleft{\begin{hindi}
कि हम तुम्हारे जैसों को बदल दें और तुम्हें ऐसी हालत में उठा खड़ा करें जिसे तुम जानते नहीं
\end{hindi}}
\flushright{\begin{Arabic}
\quranayah[56][62]
\end{Arabic}}
\flushleft{\begin{hindi}
तुम तो पहली पैदाइश को जान चुके हो, फिर तुम ध्यान क्यों नहीं देते?
\end{hindi}}
\flushright{\begin{Arabic}
\quranayah[56][63]
\end{Arabic}}
\flushleft{\begin{hindi}
फिर क्या तुमने देखा तो कुछ तुम खेती करते हो?
\end{hindi}}
\flushright{\begin{Arabic}
\quranayah[56][64]
\end{Arabic}}
\flushleft{\begin{hindi}
क्या उसे तुम उगाते हो या हम उसे उगाते है?
\end{hindi}}
\flushright{\begin{Arabic}
\quranayah[56][65]
\end{Arabic}}
\flushleft{\begin{hindi}
यदि हम चाहें तो उसे चूर-चूर कर दें। फिर तुम बातें बनाते रह जाओ
\end{hindi}}
\flushright{\begin{Arabic}
\quranayah[56][66]
\end{Arabic}}
\flushleft{\begin{hindi}
कि "हमपर उलटा डाँड पड़ गया,
\end{hindi}}
\flushright{\begin{Arabic}
\quranayah[56][67]
\end{Arabic}}
\flushleft{\begin{hindi}
बल्कि हम वंचित होकर रह गए!"
\end{hindi}}
\flushright{\begin{Arabic}
\quranayah[56][68]
\end{Arabic}}
\flushleft{\begin{hindi}
फिर क्या तुमने उस पानी को देखा जिसे तुम पीते हो?
\end{hindi}}
\flushright{\begin{Arabic}
\quranayah[56][69]
\end{Arabic}}
\flushleft{\begin{hindi}
क्या उसे बादलों से तुमने पानी बरसाया या बरसानेवाले हम है?
\end{hindi}}
\flushright{\begin{Arabic}
\quranayah[56][70]
\end{Arabic}}
\flushleft{\begin{hindi}
यदि हम चाहें तो उसे अत्यन्त खारा बनाकर रख दें। फिर तुम कृतज्ञता क्यों नहीं दिखाते?
\end{hindi}}
\flushright{\begin{Arabic}
\quranayah[56][71]
\end{Arabic}}
\flushleft{\begin{hindi}
फिर क्या तुमने उस आग को देखा जिसे तुम सुलगाते हो?
\end{hindi}}
\flushright{\begin{Arabic}
\quranayah[56][72]
\end{Arabic}}
\flushleft{\begin{hindi}
क्या तुमने उसके वृक्ष को पैदा किया है या पैदा करनेवाले हम है?
\end{hindi}}
\flushright{\begin{Arabic}
\quranayah[56][73]
\end{Arabic}}
\flushleft{\begin{hindi}
हमने उसे एक अनुस्मृति और मरुभुमि के मुसाफ़िरों और ज़रूरतमन्दों के लिए लाभप्रद बनाया
\end{hindi}}
\flushright{\begin{Arabic}
\quranayah[56][74]
\end{Arabic}}
\flushleft{\begin{hindi}
अतः तुम अपने महान रब के नाम की तसबीह करो
\end{hindi}}
\flushright{\begin{Arabic}
\quranayah[56][75]
\end{Arabic}}
\flushleft{\begin{hindi}
अतः नहीं! मैं क़समों खाता हूँ सितारों की स्थितियों की -
\end{hindi}}
\flushright{\begin{Arabic}
\quranayah[56][76]
\end{Arabic}}
\flushleft{\begin{hindi}
और यह बहुत बड़ी गवाही है, यदि तुम जानो -
\end{hindi}}
\flushright{\begin{Arabic}
\quranayah[56][77]
\end{Arabic}}
\flushleft{\begin{hindi}
निश्चय ही यह प्रतिष्ठित क़ुरआन है
\end{hindi}}
\flushright{\begin{Arabic}
\quranayah[56][78]
\end{Arabic}}
\flushleft{\begin{hindi}
एक सुरक्षित किताब में अंकित है।
\end{hindi}}
\flushright{\begin{Arabic}
\quranayah[56][79]
\end{Arabic}}
\flushleft{\begin{hindi}
उसे केवल पाक-साफ़ व्यक्ति ही हाथ लगाते है
\end{hindi}}
\flushright{\begin{Arabic}
\quranayah[56][80]
\end{Arabic}}
\flushleft{\begin{hindi}
उसका अवतरण सारे संसार के रब की ओर से है।
\end{hindi}}
\flushright{\begin{Arabic}
\quranayah[56][81]
\end{Arabic}}
\flushleft{\begin{hindi}
फिर क्या तुम उस वाणी के प्रति उपेक्षा दर्शाते हो?
\end{hindi}}
\flushright{\begin{Arabic}
\quranayah[56][82]
\end{Arabic}}
\flushleft{\begin{hindi}
और तुम इसको अपनी वृत्ति बना रहे हो कि झुठलाते हो?
\end{hindi}}
\flushright{\begin{Arabic}
\quranayah[56][83]
\end{Arabic}}
\flushleft{\begin{hindi}
फिर ऐसा क्यों नहीं होता, जबकि प्राण कंठ को आ लगते है
\end{hindi}}
\flushright{\begin{Arabic}
\quranayah[56][84]
\end{Arabic}}
\flushleft{\begin{hindi}
और उस समय तुम देख रहे होते हो -
\end{hindi}}
\flushright{\begin{Arabic}
\quranayah[56][85]
\end{Arabic}}
\flushleft{\begin{hindi}
और हम तुम्हारी अपेक्षा उससे अधिक निकट होते है। किन्तु तुम देखते नहीं –
\end{hindi}}
\flushright{\begin{Arabic}
\quranayah[56][86]
\end{Arabic}}
\flushleft{\begin{hindi}
फिर ऐसा क्यों नहीं होता कि यदि तुम अधीन नहीं हो
\end{hindi}}
\flushright{\begin{Arabic}
\quranayah[56][87]
\end{Arabic}}
\flushleft{\begin{hindi}
तो उसे (प्राण को) लौटा दो, यदि तुम सच्चे हो
\end{hindi}}
\flushright{\begin{Arabic}
\quranayah[56][88]
\end{Arabic}}
\flushleft{\begin{hindi}
फिर यदि वह (अल्लाह के) निकटवर्तियों में से है;
\end{hindi}}
\flushright{\begin{Arabic}
\quranayah[56][89]
\end{Arabic}}
\flushleft{\begin{hindi}
तो (उसके लिए) आराम, सुख-सामग्री और सुगंध है, और नेमतवाला बाग़ है
\end{hindi}}
\flushright{\begin{Arabic}
\quranayah[56][90]
\end{Arabic}}
\flushleft{\begin{hindi}
और यदि वह भाग्यशालियों में से है,
\end{hindi}}
\flushright{\begin{Arabic}
\quranayah[56][91]
\end{Arabic}}
\flushleft{\begin{hindi}
तो "सलाम है तुम्हें कि तुम सौभाग्यशाली में से हो।"
\end{hindi}}
\flushright{\begin{Arabic}
\quranayah[56][92]
\end{Arabic}}
\flushleft{\begin{hindi}
किन्तु यदि वह झुठलानेवालों, गुमराहों में से है;
\end{hindi}}
\flushright{\begin{Arabic}
\quranayah[56][93]
\end{Arabic}}
\flushleft{\begin{hindi}
तो उसका पहला सत्कार खौलते हुए पानी से होगा
\end{hindi}}
\flushright{\begin{Arabic}
\quranayah[56][94]
\end{Arabic}}
\flushleft{\begin{hindi}
फिर भड़कती हुई आग में उन्हें झोंका जाना है
\end{hindi}}
\flushright{\begin{Arabic}
\quranayah[56][95]
\end{Arabic}}
\flushleft{\begin{hindi}
निस्संदेह यही विश्वसनीय सत्य है
\end{hindi}}
\flushright{\begin{Arabic}
\quranayah[56][96]
\end{Arabic}}
\flushleft{\begin{hindi}
अतः तुम अपने महान रब की तसबीह करो
\end{hindi}}
\chapter{Al-Hadid (Iron)}
\begin{Arabic}
\Huge{\centerline{\basmalah}}\end{Arabic}
\flushright{\begin{Arabic}
\quranayah[57][1]
\end{Arabic}}
\flushleft{\begin{hindi}
अल्लाह की तसबीह की हर उस चीज़ ने जो आकाशों और धरती में है। वही प्रभुत्वशाली, तत्वशाली है
\end{hindi}}
\flushright{\begin{Arabic}
\quranayah[57][2]
\end{Arabic}}
\flushleft{\begin{hindi}
आकाशों और धरती की बादशाही उसी की है। वही जीवन प्रदान करता है और मृत्यु देता है, और उसे हर चीज़ की सामर्थ्य प्राप्त है
\end{hindi}}
\flushright{\begin{Arabic}
\quranayah[57][3]
\end{Arabic}}
\flushleft{\begin{hindi}
वही आदि है और अन्त भी और वही व्यक्त है और अव्यक्त भी। और वह हर चीज़ को जानता है
\end{hindi}}
\flushright{\begin{Arabic}
\quranayah[57][4]
\end{Arabic}}
\flushleft{\begin{hindi}
वही है जिसने आकाशों और धरती को छह दिनों में पैदा किया; फिर सिंहासन पर विराजमान हुआ। वह जानता है जो कुछ धरती में प्रवेश करता है और जो कुछ उससे निकलता है और जो कुछ आकाश से उतरता है और जो कुछ उसमें चढ़ता है। और तुम जहाँ कहीं भी हो, वह तुम्हारे साथ है। और अल्लाह देखता है जो कुछ तुम करते हो
\end{hindi}}
\flushright{\begin{Arabic}
\quranayah[57][5]
\end{Arabic}}
\flushleft{\begin{hindi}
आकाशों और धरती की बादशाही उसी की है और अल्लाह ही की है ओर सारे मामले पलटते है
\end{hindi}}
\flushright{\begin{Arabic}
\quranayah[57][6]
\end{Arabic}}
\flushleft{\begin{hindi}
वह रात को दिन में प्रविष्ट कराता है और दिन को रात में प्रविष्ट कराता है। वह सीनों में छिपी बात तक को जानता है
\end{hindi}}
\flushright{\begin{Arabic}
\quranayah[57][7]
\end{Arabic}}
\flushleft{\begin{hindi}
ईमान लाओ अल्लाह और उसके रसूल पर और उसमें से ख़र्च करो जिसका उसने तु्म्हें अधिकारी बनाया है। तो तुममें से जो लोग ईमान लाए और उन्होंने ख़र्च किया, उसने लिए बड़ा प्रतिदान है
\end{hindi}}
\flushright{\begin{Arabic}
\quranayah[57][8]
\end{Arabic}}
\flushleft{\begin{hindi}
तुम्हें क्या हो गया है कि तुम अल्लाह पर ईमान नहीं लाते; जबकि रसूल तुम्हें निमंत्रण दे रहा है कि तुम अपने रब पर ईमान लाओ और वह तुमसे दृढ़ वचन भी ले चुका है, यदि तुम मोमिन हो
\end{hindi}}
\flushright{\begin{Arabic}
\quranayah[57][9]
\end{Arabic}}
\flushleft{\begin{hindi}
वही है जो अपने बन्दों पर स्पष्ट आयतें उतार रहा है, ताकि वह तुम्हें अंधकारों से प्रकाश की ओर ले आए। और वास्तविकता यह है कि अल्लाह तुमपर अत्यन्त करुणामय, दयावान है
\end{hindi}}
\flushright{\begin{Arabic}
\quranayah[57][10]
\end{Arabic}}
\flushleft{\begin{hindi}
और तुम्हें क्यो हुआ है कि तुम अल्लाह के मार्ग में ख़र्च न करो, हालाँकि आकाशों और धरती की विरासत अल्लाह ही के लिए है? तुममें से जिन लोगों ने विजय से पूर्व ख़र्च किया और लड़े वे परस्पर एक-दूसरे के समान नहीं है। वे तो दरजे में उनसे बढ़कर है जिन्होंने बाद में ख़र्च किया और लड़े। यद्यपि अल्लाह ने प्रत्येक से अच्छा वादा किया है। अल्लाह उसकी ख़बर रखता है, जो कुछ तुम करते हो
\end{hindi}}
\flushright{\begin{Arabic}
\quranayah[57][11]
\end{Arabic}}
\flushleft{\begin{hindi}
कौन है जो अल्लाह को ऋण दे, अच्छा ऋण कि वह उसे उसके लिए कई गुना कर दे। और उसके लिए सम्मानित प्रतिदान है
\end{hindi}}
\flushright{\begin{Arabic}
\quranayah[57][12]
\end{Arabic}}
\flushleft{\begin{hindi}
जिस दिन तुम मोमिन पुरुषों और मोमिन स्त्रियों को देखोगे कि उनका प्रकाश उनके आगे-आगे दौड़ रहा है और उनके दाएँ हाथ में है। (कहा जाएगा,) "आज शुभ सूचना है तुम्हारे लिए ऐसी जन्नतों की जिनके नीचे नहरें बह रही है, जिनमें सदैव रहना है। वही बड़ी सफलता है।"
\end{hindi}}
\flushright{\begin{Arabic}
\quranayah[57][13]
\end{Arabic}}
\flushleft{\begin{hindi}
जिस दिन कपटाचारी पुरुष और कपटाचारी स्त्रियाँ मोमिनों से कहेंगी, "तनिक हमारी प्रतिक्षा करो। हम भी तुम्हारे प्रकाश मे से कुछ प्रकाश ले लें!" कहा जाएगा, "अपने पीछे लौट जाओ। फिर प्रकाश तलाश करो!" इतने में उनके बीच एक दीवार खड़ी कर दी जाएगी, जिसमें एक द्वार होगा। उसके भीतर का हाल यह होगा कि उसमें दयालुता होगी और उसके बाहर का यह कि उस ओर से यातना होगी
\end{hindi}}
\flushright{\begin{Arabic}
\quranayah[57][14]
\end{Arabic}}
\flushleft{\begin{hindi}
वे उन्हें पुकारकर कहेंगे, "क्या हम तुम्हारे साथी नहीं थे?" वे कहेंगे, "क्यों नहीं? किन्तु तुमने तो अपने आपको फ़ितने (गुमराही) में डाला और प्रतीक्षा करते रहे और सन्देह में पड़े रहे और कामनाओं ने तुम्हें धोखे में डाले रखा है
\end{hindi}}
\flushright{\begin{Arabic}
\quranayah[57][15]
\end{Arabic}}
\flushleft{\begin{hindi}
"अब आज न तुमसे कोई फ़िदया (मुक्ति-प्रतिदान) लिया जाएगा और न उन लोगों से जिन्होंने इनकार किया। तुम्हारा ठिकाना आग है, और वही तुम्हारी संरक्षिका है। और बहुत ही बुरी जगह है अन्त में पहुँचने की!"
\end{hindi}}
\flushright{\begin{Arabic}
\quranayah[57][16]
\end{Arabic}}
\flushleft{\begin{hindi}
क्या उन लोगों के लिए, जो ईमान लाए, अभी वह समय नहीं आया कि उनके दिल अल्लाह की याद के लिए और जो सत्य अवतरित हुआ है उसके आगे झुक जाएँ? और वे उन लोगों की तरह न हो जाएँ, जिन्हें किताब दी गई थी, फिर उनपर दीर्ध समय बीत गया। अन्ततः उनके दिल कठोर हो गए और उनमें से अधिकांश अवज्ञाकारी रहे
\end{hindi}}
\flushright{\begin{Arabic}
\quranayah[57][17]
\end{Arabic}}
\flushleft{\begin{hindi}
जान लो, अल्लाह धरती को उसकी मृत्यु के पश्चात जीवन प्रदान करता है। हमने तुम्हारे लिए आयतें खोल-खोलकर बयान कर दी है, ताकि तुम बुद्धि से काम लो
\end{hindi}}
\flushright{\begin{Arabic}
\quranayah[57][18]
\end{Arabic}}
\flushleft{\begin{hindi}
निश्चय ही जो सदका देनेवाले पुरुष और सदका देनेवाली स्त्रियाँ है और उन्होंने अल्लाह को अच्छा ऋण दिया, उसे उसके लिए कई गुना कर दिया जाएगा। और उनके लिए सम्मानित प्रतिदान है
\end{hindi}}
\flushright{\begin{Arabic}
\quranayah[57][19]
\end{Arabic}}
\flushleft{\begin{hindi}
जो लोग अल्लाह और उसके रसूल पर ईमान लाए, वही अपने रब के यहाण सिद्दीक और शहीद है। उनके लिए उनका प्रतिदान और उनका प्रकाश है। किन्तु जिन लोगों ने इनकार किया और हमारी आयतों को झुठलाया, वही भड़कती आगवाले हैं
\end{hindi}}
\flushright{\begin{Arabic}
\quranayah[57][20]
\end{Arabic}}
\flushleft{\begin{hindi}
जान लो, सांसारिक जीवन तो बस एक खेल और तमाशा है और एक साज-सज्जा, और तुम्हारा आपस में एक-दूसरे पर बड़ाई जताना, और धन और सन्तान में परस्पर एक-दूसरे से बढ़ा हुआ प्रदर्शित करना। वर्षा का मिसाल की तरह जिसकी वनस्पति ने किसान का दिल मोह लिया। फिर वह पक जाती है; फिर तुम उसे देखते हो कि वह पीली हो गई। फिर वह चूर्ण-विचूर्ण होकर रह जाती है, जबकि आख़िरत में कठोर यातना भी है और अल्लाह की क्षमा और प्रसन्नता भी। सांसारिक जीवन तो केवल धोखे की सुख-सामग्री है
\end{hindi}}
\flushright{\begin{Arabic}
\quranayah[57][21]
\end{Arabic}}
\flushleft{\begin{hindi}
अपने रब की क्षमा और उस जन्नत की ओर अग्रसर होने में एक-दूसरे से बाज़ी ले जाओ, जिसका विस्तार आकाश और धरती के विस्तार जैसा है, जो उन लोगों के लिए तैयार की गई है जो अल्लाह और उसके रसूलों पर ईमान लाए हों। यह अल्लाह का उदार अनुग्रह है, जिसे चाहता है प्रदान करता है। अल्लाह बड़े उदार अनुग्रह का मालिक है
\end{hindi}}
\flushright{\begin{Arabic}
\quranayah[57][22]
\end{Arabic}}
\flushleft{\begin{hindi}
जो मुसीबतें भी धरती में आती है और तुम्हारे अपने ऊपर, वह अनिवार्यतः एक किताब में अंकित है, इससे पहले कि हम उसे अस्तित्व में लाएँ - निश्चय ही यह अल्लाह के लिए आसान है -
\end{hindi}}
\flushright{\begin{Arabic}
\quranayah[57][23]
\end{Arabic}}
\flushleft{\begin{hindi}
(यह बात तुम्हें इसलिए बता दी गई) ताकि तुम उस चीज़ का अफ़सोस न करो जो तुम पर जाती रहे और न उसपर फूल जाओ जो उसने तुम्हें प्रदान की हो। अल्लाह किसी इतरानेवाले, बड़ाई जतानेवाले को पसन्द नहीं करता
\end{hindi}}
\flushright{\begin{Arabic}
\quranayah[57][24]
\end{Arabic}}
\flushleft{\begin{hindi}
जो स्वयं कंजूसी करते है और लोगों को भी कंजूसी करने पर उकसाते है, और जो कोई मुँह मोड़े तो अल्लाह तो निस्पृह प्रशंसनीय है
\end{hindi}}
\flushright{\begin{Arabic}
\quranayah[57][25]
\end{Arabic}}
\flushleft{\begin{hindi}
निश्चय ही हमने अपने रसूलों को स्पष्ट प्रमाणों के साथ भेजा और उनके लिए किताब और तुला उतारी, ताकि लोग इनसाफ़ पर क़ायम हों। और लोहा भी उतारा, जिसमें बड़ी दहशत है और लोगों के लिए कितने ही लाभ है., और (किताब एवं तुला इसलिए भी उतारी) ताकि अल्लाह जान ले कि कौन परोक्ष में रहते हुए उसकी और उसके रसूलों की सहायता करता है। निश्चय ही अल्लाह शक्तिशाली, प्रभुत्वशाली है
\end{hindi}}
\flushright{\begin{Arabic}
\quranayah[57][26]
\end{Arabic}}
\flushleft{\begin{hindi}
हमने नूह और इबराहीम को भेजा और उन दोनों की सन्तान में पैग़म्बरी और क़िताब रख दी। फिर उनमें से किसी ने तो संमार्ग अपनाया; किन्तु उनमें से अधिकतर अवज्ञाकारी थे
\end{hindi}}
\flushright{\begin{Arabic}
\quranayah[57][27]
\end{Arabic}}
\flushleft{\begin{hindi}
फिर उनके पीछ उन्हीं के पद-चिन्हों पर हमने अपने दूसरे रसूलों को भेजा और हमने उनके पीछे मरयम के बेटे ईसा को भेजा और उसे इंजील प्रदान की। और जिन लोगों ने उसका अनुसरण किया, उनके दिलों में हमने करुणा और दया रख दी। रहा संन्यास, तो उसे उन्होंने स्वयं घड़ा था। हमने उसे उनके लिए अनिवार्य नहीं किया था, यदि अनिवार्य किया था तो केवल अल्लाह की प्रसन्नता की चाहत। फिर वे उसका निर्वाह न कर सकें, जैसा कि उनका निर्वाह करना चाहिए था। अतः उन लोगों को, जो उनमें से वास्तव में ईमान लाए थे, उनका बदला हमने (उन्हें) प्रदान किया। किन्तु उनमें से अधिकतर अवज्ञाकारी ही है
\end{hindi}}
\flushright{\begin{Arabic}
\quranayah[57][28]
\end{Arabic}}
\flushleft{\begin{hindi}
ऐ लोगों, जो ईमान लाए हो! अल्लाह का डर रखो और उसके रसूल पर ईमान लाओ। वह तुम्हें अपनी दयालुता का दोहरा हिस्सा प्रदान करेगा और तुम्हारे लिए एक प्रकाश कर देगा, जिसमें तुम चलोगे और तुम्हें क्षमा कर देगा। अल्लाह बड़ा क्षमाशील, अत्यन्त दयावान है
\end{hindi}}
\flushright{\begin{Arabic}
\quranayah[57][29]
\end{Arabic}}
\flushleft{\begin{hindi}
ताकि किताबवाले यह न समझें कि अल्लाह के अनुग्रह में से वे किसी चीज़ पर अधिकार न प्राप्त कर सकेंगे और यह कि अनुग्रह अल्लाह के हाथ में है, जिसे चाहता है प्रदान करता है। अल्लाह बड़े अनुग्रह का मालिक है
\end{hindi}}
\chapter{Al-Mujadilah (The Pleading Woman)}
\begin{Arabic}
\Huge{\centerline{\basmalah}}\end{Arabic}
\flushright{\begin{Arabic}
\quranayah[58][1]
\end{Arabic}}
\flushleft{\begin{hindi}
अल्लाह ने उस स्त्री की बात सुन ली जो अपने पति के विषय में तुमसे झगड़ रही है और अल्लाह से शिकायत किए जाती है। अल्लाह तुम दोनों की बातचीत सुन रहा है। निश्चय ही अल्लाह सब कुछ सुननेवाला, देखनेवाला है
\end{hindi}}
\flushright{\begin{Arabic}
\quranayah[58][2]
\end{Arabic}}
\flushleft{\begin{hindi}
तुममें से जो लोग अपनी स्त्रियों से ज़िहार करते हैं, उनकी माएँ वे नहीं है, उनकी माएँ तो वही है जिन्होंने उनको जन्म दिया है। यह अवश्य है कि वे लोग एक अनुचित बात और झूठ कहते है। और निश्चय ही अल्लाह टाल जानेवाला अत्यन्त क्षमाशील है
\end{hindi}}
\flushright{\begin{Arabic}
\quranayah[58][3]
\end{Arabic}}
\flushleft{\begin{hindi}
जो लोग अपनी स्त्रियों से ज़िहार करते हैं; फिर जो बात उन्होंने कही थी उससे रुजू करते है, तो इससे पहले कि दोनों एक-दूसरे को हाथ लगाएँ एक गर्दन आज़ाद करनी होगी। यह वह बात है जिसकी तुम्हें नसीहत की जाती है, और तुम जो कुछ करते हो अल्लाह उसकी ख़बर रखता है
\end{hindi}}
\flushright{\begin{Arabic}
\quranayah[58][4]
\end{Arabic}}
\flushleft{\begin{hindi}
किन्तु जिस किसी को ग़ुलाम प्राप्त न हो तो वह निरन्तर दो माह रोज़े रखे, इससे पहले कि वे दोनों एक-दूसरे को हाथ लगाएँ और जिस किसी को इसकी भी सामर्थ्य न हो तो साठ मुहताजों को भोजन कराना होगा। यह इसलिए कि तुम अल्लाह और उसके रसूल पर ईमानवाले सिद्ध हो सको। ये अल्लाह की निर्धारित की हुई सीमाएँ है। और इनकार करनेवाले के लिए दुखद यातना है
\end{hindi}}
\flushright{\begin{Arabic}
\quranayah[58][5]
\end{Arabic}}
\flushleft{\begin{hindi}
जो लोग अल्लाह और उसके रसूल का विरोध करते हैं, वे अपमानित और तिरस्कृत होकर रहेंगे, जैसे उनसे पहले के लोग अपमानित और तिरस्कृत हो चुके है। हमने स्पष्ट आयतें अवतरित कर दी है और इनकार करनेवालों के लिए अपमानजनक यातना है
\end{hindi}}
\flushright{\begin{Arabic}
\quranayah[58][6]
\end{Arabic}}
\flushleft{\begin{hindi}
जिस दिन अल्लाह उन सबको उठा खड़ा करेगा और जो कुछ उन्होंने किया होगा, उससे उन्हें अवगत करा देगा। अल्लाह ने उसकी गणना कर रखी है, और वे उसे भूले हुए है, और अल्लाह हर चीज़ का साक्षी है
\end{hindi}}
\flushright{\begin{Arabic}
\quranayah[58][7]
\end{Arabic}}
\flushleft{\begin{hindi}
क्या तुमने इसको नहीं देखा कि अल्लाह जानता है जो कुछ आकाशों में है और जो कुछ धरती में है। कभी ऐसा नहीं होता कि तीन आदमियों की गुप्त वार्ता हो और उनके बीच चौथा वह (अल्लाह) न हो। और न पाँच आदमियों की होती है जिसमें छठा वह न होता हो। और न इससे कम की कोई होती है और न इससे अधिक की भी, किन्तु वह उनके साथ होता है, जहाँ कहीं भी वे हो; फिर जो कुछ भी उन्होंने किया होगा क़ियामत के दिन उससे वह उन्हें अवगत करा देगा। निश्चय ही अल्लाह को हर चीज़ का ज्ञान है
\end{hindi}}
\flushright{\begin{Arabic}
\quranayah[58][8]
\end{Arabic}}
\flushleft{\begin{hindi}
क्या तुमने नहीं देखा जिन्हें कानाफूसी से रोका गया था, फिर वे वही करते रहे जिससे उन्हें रोका गया था। वे आपस में गुनाह और ज़्यादती और रसूल की अवज्ञा की कानाफूसी करते है। और जब तुम्हारे पास आते है तो तुम्हारे प्रति अभिवादन के ऐसे शब्द प्रयोग में लाते है जो शब्द अल्लाह ने तुम्हारे लिए अभिवादन के लिए नहीं कहे। और अपने जी में कहते है, "जो कुछ हम कहते है उसपर अल्लाह हमें यातना क्यों नहीं देता?" उनके लिए जहन्नम ही काफ़ी है जिसमें वे प्रविष्ट होंगे। वह तो बहुत बुरी जगह है, अन्त नें पहुँचने की!
\end{hindi}}
\flushright{\begin{Arabic}
\quranayah[58][9]
\end{Arabic}}
\flushleft{\begin{hindi}
ऐ ईमान लानेवालो! जब तुम आपस में गुप्त॥ वार्ता करो तो गुनाह और ज़्यादती और रसूल की अवज्ञा की गुप्त वार्ता न करो, बल्कि नेकी और परहेज़गारी के विषय में आपस में एकान्त वार्ता करो। और अल्लाह का डर रखो, जिसके पास तुम इकट्ठे होगे
\end{hindi}}
\flushright{\begin{Arabic}
\quranayah[58][10]
\end{Arabic}}
\flushleft{\begin{hindi}
वह कानाफूसी तो केवल शैतान की ओर से है, ताकि वह उन्हें ग़म में डाले जो ईमान लाए है। हालाँकि अल्लाह की अवज्ञा के बिना उसे कुछ भी हानि पहुँचाने की सामर्थ्य प्राप्त नहीं। और ईमानवालों को तो अल्लाह ही पर भरोसा रखना चाहिए
\end{hindi}}
\flushright{\begin{Arabic}
\quranayah[58][11]
\end{Arabic}}
\flushleft{\begin{hindi}
ऐ ईमान लानेवालो! जब तुमसे कहा जाए कि मजलिसों में जगह कुशादा कर दे, तो कुशादगी पैदा कर दो। अल्लाह तुम्हारे लिए कुशादगी पैदा करेगा। और जब कहा जाए कि उठ जाओ, तो उठ जाया करो। तुममें से जो लोग ईमान लाए है और उन्हें ज्ञान प्रदान किया गया है, अल्लाह उनके दरजों को उच्चता प्रदान करेगा। जो कुछ तुम करते हो अल्लाह उसकी पूरी ख़बर रखता है
\end{hindi}}
\flushright{\begin{Arabic}
\quranayah[58][12]
\end{Arabic}}
\flushleft{\begin{hindi}
ऐ ईमान लानेवालो! जब तुम रसूल से अकेले में बात करो तो अपनी गुप्त वार्ता से पहले सदक़ा दो। यह तुम्हारे लिए अच्छा और अधिक पवित्र है। फिर यदि तुम अपने को इसमें असमर्थ पाओ, तो निश्चय ही अल्लाह बड़ा क्षमाशील, अत्यन्त दयावान है
\end{hindi}}
\flushright{\begin{Arabic}
\quranayah[58][13]
\end{Arabic}}
\flushleft{\begin{hindi}
क्या तुम इससे डर गए कि अपनी गुप्त वार्ता से पहले सदक़े दो? जो जब तुमने यह न किया और अल्लाह ने तुम्हें क्षमा कर दिया. तो नमाज़ क़ायम करो, ज़कात देते रहो और अल्लाह और उसके रसूल की आज्ञा का पालन करो। और तुम जो कुछ भी करते हो अल्लाह उसकी पूरी ख़बर रखता है
\end{hindi}}
\flushright{\begin{Arabic}
\quranayah[58][14]
\end{Arabic}}
\flushleft{\begin{hindi}
क्या तुमने उन लोगों को नहीं देखा जिन्होंने ऐसे लोगों को मित्र बनाया जिनपर अल्लाह का प्रकोप हुआ है? वे न तुममें से है और न उनमें से। और वे जानते-बूझते झूठी बात पर क़सम खाते है
\end{hindi}}
\flushright{\begin{Arabic}
\quranayah[58][15]
\end{Arabic}}
\flushleft{\begin{hindi}
अल्लाह ने उनके लिए कठोर यातना तैयार कर रखी है। निश्चय ही बुरा है जो वे कर रहे है
\end{hindi}}
\flushright{\begin{Arabic}
\quranayah[58][16]
\end{Arabic}}
\flushleft{\begin{hindi}
उन्होंने अपनी क़समों को ढाल बना रखा है। अतः वे अल्लाह के मार्ग से (लोगों को) रोकते है। तो उनके लिए रुसवा करनेवाली यातना है
\end{hindi}}
\flushright{\begin{Arabic}
\quranayah[58][17]
\end{Arabic}}
\flushleft{\begin{hindi}
अल्लाह से बचाने के लिए न उनके माल उनके कुछ काम आएँगे और न उनकी सन्तान। वे आगवाले हैं। उसी में वे सदैव रहेंगे
\end{hindi}}
\flushright{\begin{Arabic}
\quranayah[58][18]
\end{Arabic}}
\flushleft{\begin{hindi}
जिस दिन अल्लाह उन सबको उठाएगा तो वे उसके सामने भी इसी तरह क़समें खाएँगे, जिस तरह तुम्हारे सामने क़समें खाते है और समझते हैं कि वे किसी बुनियाद पर है। सावधान रहो, निश्चय ही वही झूठे है!
\end{hindi}}
\flushright{\begin{Arabic}
\quranayah[58][19]
\end{Arabic}}
\flushleft{\begin{hindi}
उनपर शैतान ने पूरी तरह अपना प्रभाव जमा लिया है। अतः उसने अल्लाह की याद को उनसे भुला दिया। वे शैतान की पार्टीवाले हैं। सावधान रहो शैतान की पार्टीवाले ही घाटे में रहनेवाले हैं!
\end{hindi}}
\flushright{\begin{Arabic}
\quranayah[58][20]
\end{Arabic}}
\flushleft{\begin{hindi}
निश्चय ही जो लोग अल्लाह और उसके रसूल का विरोध करते है वे अत्यन्त अपमानित लोगों में से है
\end{hindi}}
\flushright{\begin{Arabic}
\quranayah[58][21]
\end{Arabic}}
\flushleft{\begin{hindi}
अल्लाह ने लिए दिया है, "मैं और मेरे रसूल ही विजयी होकर रहेंगे।" निस्संदेह अल्लाह शक्तिमान, प्रभुत्वशाली है
\end{hindi}}
\flushright{\begin{Arabic}
\quranayah[58][22]
\end{Arabic}}
\flushleft{\begin{hindi}
तुम उन लोगों को ऐसा कभी नहीं पाओगे जो अल्लाह और अन्तिम दि पर ईमान रखते है कि वे उन लोगों से प्रेम करते हो जिन्होंने अल्लाह और उसके रसूल का विरोध किया, यद्यपि वे उनके अपने बाप हों या उनके अपने बेटे हो या उनके अपने भाई या उनके अपने परिवारवाले ही हो। वही लोग हैं जिनके दिलों में अल्लाह ने ईमान को अंकित कर दिया है और अपनी ओर से एक आत्मा के द्वारा उन्हें शक्ति दी है। और उन्हें वह ऐसे बाग़ों में दाख़िल करेगा जिनके नीचे नहरें बह रही होंगी; जहाँ वे सदैव रहेंगे। अल्लाह उनसे राज़ी हुआ और वे भी उससे राज़ी हुए। वे अल्लाह की पार्टी के लोग है। सावधान रहो, निश्चय ही अल्लाह की पार्टीवाले ही सफल है
\end{hindi}}
\chapter{Al-Hashr (The Banishment)}
\begin{Arabic}
\Huge{\centerline{\basmalah}}\end{Arabic}
\flushright{\begin{Arabic}
\quranayah[59][1]
\end{Arabic}}
\flushleft{\begin{hindi}
अल्लाह की तसबीह की है हर उस चीज़ ने जो आकाशों और धरती में है, और वही प्रभुत्वशाली, तत्वदर्शी है
\end{hindi}}
\flushright{\begin{Arabic}
\quranayah[59][2]
\end{Arabic}}
\flushleft{\begin{hindi}
वही है जिसने किताबवालों में से उन लोगों को जिन्होंने इनकार किया, उनके घरों से पहले ही जमावड़े में निकल बाहर किया। तुम्हें गुमान न था कि उनकी गढ़ियाँ अल्लाह से उन्हें बचा लेंगी। किन्तु अल्लाह उनपर वहाँ से आया जिसका उन्हें गुमान भी न था। और उसने उनके दिलों में रोब डाल दिया कि वे अपने घरों को स्वयं अपने हाथों और ईमानवालों के हाथों भी उजाड़ने लगे। अतः शिक्षा ग्रहण करो, ऐ दृष्टि रखनेवालो!
\end{hindi}}
\flushright{\begin{Arabic}
\quranayah[59][3]
\end{Arabic}}
\flushleft{\begin{hindi}
यदि अल्लाह ने उनके लिए देश निकाला न लिख दिया होता तो दुनिया में ही वह उन्हें अवश्य यातना दे देता, और आख़िरत में तो उनके लिए आग की यातना है ही
\end{hindi}}
\flushright{\begin{Arabic}
\quranayah[59][4]
\end{Arabic}}
\flushleft{\begin{hindi}
यह इसलिए कि उन्होंने अल्लाह और उसके रसूल का मुक़ाला करने की कोशिश की। और जो कोई अल्लाह का मुक़ाबला करता है तो निश्चय ही अल्लाह की यातना बहुत कठोर है
\end{hindi}}
\flushright{\begin{Arabic}
\quranayah[59][5]
\end{Arabic}}
\flushleft{\begin{hindi}
तुमने खजूर के जो वृक्ष काटे या उन्हें उनकी जड़ों पर खड़ा छोड़ दिया तो यह अल्लाह ही की अनुज्ञा से हुआ (ताकि ईमानवालों के लिए आसानी पैदा करे) और इसलिए कि वह अवज्ञाकारियों को रुसवा करे
\end{hindi}}
\flushright{\begin{Arabic}
\quranayah[59][6]
\end{Arabic}}
\flushleft{\begin{hindi}
और अल्लाह ने उनसे लेकर अपने रसूल की ओर जो कुछ पलटाया, उसके लिए न तो तुमने घोड़े दौड़ाए और न ऊँट। किन्तु अल्लाह अपने रसूलों को जिसपर चाहता है प्रभुत्व प्रदान कर देता है। अल्लाह को तो हर चीज़ की सामर्थ्य प्राप्ति है
\end{hindi}}
\flushright{\begin{Arabic}
\quranayah[59][7]
\end{Arabic}}
\flushleft{\begin{hindi}
जो कुछ अल्लाह ने अपने रसूल की ओर बस्तियोंवालों से लेकर पलटाया वह अल्लाह और रसूल और (मुहताज) नातेदार और अनाथों और मुहताजों और मुसाफ़िर के लिए है, ताकि वह (माल) तुम्हारे मालदारों ही के बीच चक्कर न लगाता रहे - रसूल जो कुछ तुम्हें दे उसे ले लो और जिस चीज़ से तुम्हें रोक दे उससे रुक जाओ, और अल्लाह का डर रखो। निश्चय ही अल्लाह की यातना बहुत कठोर है। -
\end{hindi}}
\flushright{\begin{Arabic}
\quranayah[59][8]
\end{Arabic}}
\flushleft{\begin{hindi}
वह ग़रीब मुहाजिरों के लिए है, जो अपने घरों और अपने मालों से इस हालत में निकाल बाहर किए गए है कि वे अल्लाह का उदार अनुग्रह और उसकी प्रसन्नता की तलाश में है और अल्लाह और उसके रसूल की सहायता कर रहे है, और वही वास्तव में सच्चे है
\end{hindi}}
\flushright{\begin{Arabic}
\quranayah[59][9]
\end{Arabic}}
\flushleft{\begin{hindi}
और उनके लिए जो उनसे पहले ही से हिजरत के घर (मदीना) में ठिकाना बनाए हुए है और ईमान पर जमे हुए है, वे उनसे प्रेम करते है जो हिजरत करके उनके यहाँ आए है और जो कुछ भी उन्हें दिया गया उससे वे अपने सीनों में कोई खटक नहीं पाते और वे उन्हें अपने मुक़ाबले में प्राथमिकता देते है, यद्यपि अपनी जगह वे स्वयं मुहताज ही हों। और जो अपने मन के लोभ और कृपणता से बचा लिया जाए ऐसे लोग ही सफल है
\end{hindi}}
\flushright{\begin{Arabic}
\quranayah[59][10]
\end{Arabic}}
\flushleft{\begin{hindi}
और (इस माल में उनका भी हिस्सा है) जो उनके बाद आए, वे कहते है, "ऐ हमारे रब! हमें क्षमा कर दे और हमारे उन भाइयों को भी जो ईमानलाने में हमसे अग्रसर रहे और हमारे दिलों में ईमानवालों के लिए कोई विद्वेष न रख। ऐ हमारे रब! तू निश्चय ही बड़ा करुणामय, अत्यन्त दयावान है।"
\end{hindi}}
\flushright{\begin{Arabic}
\quranayah[59][11]
\end{Arabic}}
\flushleft{\begin{hindi}
क्या तुमने उन लोगों को नहीं देखा जिन्होंने कपटाचार की नीति अपनाई हैं, वे अपने किताबवाले उन भाइयों से, जो इनकार की नीति अपनाए हुए है, कहते है, "यदि तुम्हें निकाला गया तो हम भी अवश्य ही तुम्हारे साथ निकल जाएँगे और तुम्हारे मामले में किसी की बात कभी भी नहीं मानेंगे। और यदि तुमसे युद्ध किया गया तो हम अवश्य तुम्हारी सहायता करेंगे।" किन्तु अल्लाह गवाही देता है कि वे बिलकुल झूठे है
\end{hindi}}
\flushright{\begin{Arabic}
\quranayah[59][12]
\end{Arabic}}
\flushleft{\begin{hindi}
यदि वे निकाले गए तो वे उनके साथ नहीं निकलेंगे और यदि उनसे युद्ध हुआ तो वे उनकी सहायता कदापि न करेंगे और यदि उनकी सहायता करें भी तो पीठ फेंर जाएँगे। फिर उन्हें कोई सहायता प्राप्त न होगी
\end{hindi}}
\flushright{\begin{Arabic}
\quranayah[59][13]
\end{Arabic}}
\flushleft{\begin{hindi}
उनके दिलों में अल्लाह से बढ़कर तुम्हारा भय समाया हुआ है। यह इसलिए कि वे ऐसे लोग है जो समझते नहीं
\end{hindi}}
\flushright{\begin{Arabic}
\quranayah[59][14]
\end{Arabic}}
\flushleft{\begin{hindi}
वे इकट्ठे होकर भी तुमसे (खुले मैदान में) नहीं लड़ेगे, क़िलाबन्द बस्तियों या दीवारों के पीछ हों तो यह और बात है। उनकी आपस में सख़्त लड़ाई है। तुम उन्हें इकट्ठा समझते हो! हालाँकि उनके दिल फटे हुए है। यह इसलिए कि वे ऐसे लोग है जो बुद्धि से काम नहीं लेते
\end{hindi}}
\flushright{\begin{Arabic}
\quranayah[59][15]
\end{Arabic}}
\flushleft{\begin{hindi}
उनकी हालत उन्हीं लोगों जैसी है जो उनसे पहले निकट काल में अपने किए के वबाल का मज़ा चख चुके है, और उनके लिए दुखद यातना भी है
\end{hindi}}
\flushright{\begin{Arabic}
\quranayah[59][16]
\end{Arabic}}
\flushleft{\begin{hindi}
इनकी मिसाल शैतान जैसी है कि जब उसने मनुष्य से कहा, "क़ुफ़्र कर!" फिर जब वह कुफ़्र कर बैठा तो कहने लगा, "मैं तुम्हारी ज़िम्मेदारी से बरी हूँ। मैं तो सारे संसार के रब अल्लाह से डरता हूँ।"
\end{hindi}}
\flushright{\begin{Arabic}
\quranayah[59][17]
\end{Arabic}}
\flushleft{\begin{hindi}
फिर उन दोनों का परिणाम यह हुआ कि दोनों आग में गए, जहाँ सदैव रहेंगे। और ज़ालिमों का यही बदला है
\end{hindi}}
\flushright{\begin{Arabic}
\quranayah[59][18]
\end{Arabic}}
\flushleft{\begin{hindi}
ऐ ईमान लानेवालो! अल्लाह का डर रखो। और प्रत्येक व्यक्ति को यह देखना चाहिए कि उसने कल के लिए क्या भेजा है। और अल्लाह का डर रखो। जो कुछ भी तुम करते हो निश्चय ही अल्लाह उसकी पूरी ख़बर रखता है
\end{hindi}}
\flushright{\begin{Arabic}
\quranayah[59][19]
\end{Arabic}}
\flushleft{\begin{hindi}
और उन लोगों की तरह न हो जाना जिन्होंने अल्लाह को भुला दिया। तो उसने भी ऐसा किया कि वे स्वयं अपने आपको भूल बैठे। वही अवज्ञाकारी है
\end{hindi}}
\flushright{\begin{Arabic}
\quranayah[59][20]
\end{Arabic}}
\flushleft{\begin{hindi}
आगवाले और बाग़वाले (जहन्नमवाले और जन्नतवाले) कभी समान नहीं हो सकते। बाग़वाले ही सफ़ल है
\end{hindi}}
\flushright{\begin{Arabic}
\quranayah[59][21]
\end{Arabic}}
\flushleft{\begin{hindi}
यदि हमने इस क़ुरआन को किसी पर्वत पर भी उतार दिया होता तो तुम अवश्य देखते कि अल्लाह के भय से वह दबा हुआ और फटा जाता है। ये मिशालें लोगों के लिए हम इसलिए पेश करते है कि वे सोच-विचार करें
\end{hindi}}
\flushright{\begin{Arabic}
\quranayah[59][22]
\end{Arabic}}
\flushleft{\begin{hindi}
वही अल्लाह है जिसके सिवा कोई पूज्य-प्रभु नहीं, परोक्ष और प्रत्यक्ष को जानता है। वह बड़ा कृपाशील, अत्यन्त दयावान है
\end{hindi}}
\flushright{\begin{Arabic}
\quranayah[59][23]
\end{Arabic}}
\flushleft{\begin{hindi}
वही अल्लाह है जिसके सिवा कोई पूज्य नहीं। बादशाह है अत्यन्त पवित्र, सर्वथा सलामती, निश्चिन्तता प्रदान करनेवाला, संरक्षक, प्रभुत्वशाली, प्रभावशाली (टुटे हुए को जोड़नेवाला), अपनी बड़ाई प्रकट करनेवाला। महान और उच्च है अल्लाह उस शिर्क से जो वे करते है
\end{hindi}}
\flushright{\begin{Arabic}
\quranayah[59][24]
\end{Arabic}}
\flushleft{\begin{hindi}
वही अल्लाह है जो संरचना का प्रारूपक है, अस्तित्व प्रदान करनेवाला, रूप देनेवाला है। उसी के लिए अच्छे नाम है। जो चीज़ भी आकाशों और धरती में है, उसी की तसबीह कर रही है। और वह प्रभुत्वशाली, तत्वदर्शी है
\end{hindi}}
\chapter{Al-Mumtahanah (The Woman who is Examined)}
\begin{Arabic}
\Huge{\centerline{\basmalah}}\end{Arabic}
\flushright{\begin{Arabic}
\quranayah[60][1]
\end{Arabic}}
\flushleft{\begin{hindi}
ऐ ईमान लानेवालो! यदि तुम मेरे मार्ग में जिहाद के लिए और मेरी प्रसन्नता की तलाश में निकले हो तो मेरे शत्रुओं और अपने शत्रुओं को मित्र न बनाओ कि उनके प्रति प्रेम दिखाओं, जबकि तुम्हारे पास जो सत्य आया है उसका वे इनकार कर चुके है। वे रसूल को और तुम्हें इसलिए निर्वासित करते है कि तुम अपने रब - अल्लाह पर ईमान लाए हो। तुम गुप्त रूप से उनसे मित्रता की बातें करते हो। हालाँकि मैं भली-भाँति जानता हूँ जो कुछ तुम छिपाते हो और व्यक्त करते हो। और जो कोई भी तुममें से भटक गया
\end{hindi}}
\flushright{\begin{Arabic}
\quranayah[60][2]
\end{Arabic}}
\flushleft{\begin{hindi}
यदि वे तुम्हें पा जाएँ तो तुम्हारे शत्रु हो जाएँ और कष्ट पहुँचाने के लिए तुमपर हाथ और ज़बान चलाएँ। वे तो चाहते है कि काश! तुम भी इनकार करनेवाले हो जाओ
\end{hindi}}
\flushright{\begin{Arabic}
\quranayah[60][3]
\end{Arabic}}
\flushleft{\begin{hindi}
क़ियामत के दिन तुम्हारी नातेदारियाँ कदापि तुम्हें लाभ न पहुँचाएँगी और न तुम्हारी सन्तान ही। उस दिन वह (अल्लाह) तुम्हारे बीच जुदाई डाल देगा। जो कुछ भी तुम करते हो अल्लाह उसे देख रहा होता है
\end{hindi}}
\flushright{\begin{Arabic}
\quranayah[60][4]
\end{Arabic}}
\flushleft{\begin{hindi}
तुम लोगों के लिए इबराहीम में और उन लोगों में जो उसके साथ थे अच्छा आदर्श है, जबकि उन्होंने अपनी क़ौम के लोगों से कह दिया कि "हम तुमसे और अल्लाह से हटकर जिन्हें तुम पूजते हो उनसे विरक्त है। हमने तुम्हारा इनकार किया और हमारे और तुम्हारे बीच सदैव के लिए वैर और विद्वेष प्रकट हो चुका जब तक अकेले अल्लाह पर तुम ईमान न लाओ।" इूबराहीम का अपने बाप से यह कहना अपवाद है कि "मैं आपके लिए क्षमा की प्रार्थना अवश्य करूँगा, यद्यपि अल्लाह के मुक़ाबले में आपके लिए मैं किसी चीज़ पर अधिकार नहीं रखता।" "ऐ हमारे रब! हमने तुझी पर भरोसा किया और तेरी ही ओर रुजू हुए और तेरी ही ओर अन्त में लौटना हैं। -
\end{hindi}}
\flushright{\begin{Arabic}
\quranayah[60][5]
\end{Arabic}}
\flushleft{\begin{hindi}
"ऐ हमारे रब! हमें इनकार करनेवालों के लिए फ़ितना न बना और ऐ हमारे रब! हमें क्षमा कर दे। निश्चय ही तू प्रभुत्वशाली, तत्वदर्शी है।"
\end{hindi}}
\flushright{\begin{Arabic}
\quranayah[60][6]
\end{Arabic}}
\flushleft{\begin{hindi}
निश्चय ही तुम्हारे लिए उनमें अच्छा आदर्श है और हर उस व्यक्ति के लिए जो अल्लाह और अंतिम दिन की आशा रखता हो। और जो कोई मुँह फेरे तो अल्लाह तो निस्पृह, अपने आप में स्वयं प्रशंसित है
\end{hindi}}
\flushright{\begin{Arabic}
\quranayah[60][7]
\end{Arabic}}
\flushleft{\begin{hindi}
आशा है कि अल्लाह तुम्हारे और उनके बीच, जिनके बीच, जिनसे तुमने शत्रुता मोल ली है, प्रेम-भाव उत्पन्न कर दे। अल्लाह बड़ी सामर्थ्य रखता है और अल्लाह बहुत क्षमाशील, अत्यन्त दयावान है
\end{hindi}}
\flushright{\begin{Arabic}
\quranayah[60][8]
\end{Arabic}}
\flushleft{\begin{hindi}
अल्लाह तुम्हें इससे नहीं रोकता कि तुम उन लोगों के साथ अच्छा व्यवहार करो और उनके साथ न्याय करो, जिन्होंने तुमसे धर्म के मामले में युद्ध नहीं किया और न तुम्हें तुम्हारे अपने घरों से निकाला। निस्संदेह अल्लाह न्याय करनेवालों को पसन्द करता है
\end{hindi}}
\flushright{\begin{Arabic}
\quranayah[60][9]
\end{Arabic}}
\flushleft{\begin{hindi}
अल्लाह तो तुम्हें केवल उन लोगों से मित्रता करने से रोकता है जिन्होंने धर्म के मामले में तुमसे युद्ध किया और तुम्हें तुम्हारे अपने घरों से निकाला और तुम्हारे निकाले जाने के सम्बन्ध में सहायता की। जो लोग उनसे मित्रता करें वही ज़ालिम है
\end{hindi}}
\flushright{\begin{Arabic}
\quranayah[60][10]
\end{Arabic}}
\flushleft{\begin{hindi}
ऐ ईमान लानेवालो! जब तुम्हारे पास ईमान की दावेदार स्त्रियाँ हिजरत करके आएँ तो तुम उन्हें जाँच लिया करो। यूँ तो अल्लाह उनके ईमान से भली-भाँति परिचित है। फिर यदि वे तुम्हें ईमानवाली मालूम हो, तो उन्हें इनकार करनेवालों (अधर्मियों) की ओर न लौटाओ। न तो वे स्त्रियाँ उनके लिए वैद्य है और न वे उन स्त्रियों के लिए वैद्य है। और जो कुछ उन्होंने ख़र्च किया हो तुम उन्हें दे दो और इसमें तुम्हारे लिए कोई गुनाह नहीं कि तुम उनसे विवाह कर लो, जबकि तुम उन्हें महर अदा कर दो। और तुम स्वयं भी इनकार करनेवाली स्त्रियों के सतीत्व को अपने अधिकार में न रखो। और जो कुछ तुमने ख़र्च किया हो माँग लो। और उन्हें भी चाहिए कि जो कुछ उन्होंने ख़र्च किया हो माँग ले। यह अल्लाह का आदेश है। वह तुम्हारे बीच फ़ैसला करता है। अल्लाह सर्वज्ञ, तत्वदर्शी है
\end{hindi}}
\flushright{\begin{Arabic}
\quranayah[60][11]
\end{Arabic}}
\flushleft{\begin{hindi}
और यदि तुम्हारी पत्नि यो (के मह्रों) में से कुछ तुम्हारे हाथ से निकल जाए और इनकार करनेवालों (अधर्मियों) की ओर रह जाए, फिर तुम्हारी नौबत आए, जो जिन लोगों की पत्नियों चली गई है, उन्हें जितना उन्होंने ख़र्च किया हो दे दो। और अल्लाह का डर रखो, जिसपर तुम ईमान रखते हो
\end{hindi}}
\flushright{\begin{Arabic}
\quranayah[60][12]
\end{Arabic}}
\flushleft{\begin{hindi}
ऐ नबी! जब तुम्हारे पास ईमानवाली स्त्रियाँ आकर तुमसे इसपर 'बैअत' करे कि वे अल्लाह के साथ किसी चीज़ को साझी नहीं ठहराएँगी और न चोरी करेंगी और न व्यभिचार करेंगी, और न अपनी औलाद की हत्या करेंगी और न अपने हाथों और पैरों को बीच कोई आरोप घड़कर लाएँगी. और न किसी भले काम में तुम्हारी अवज्ञा करेंगी, तो उनसे 'बैअत' ले लो और उनके लिए अल्लाह से क्षमा की प्रार्थना करो। निश्चय ही अत्यन्त बहुत क्षमाशील, अत्यन्त दयावान है
\end{hindi}}
\flushright{\begin{Arabic}
\quranayah[60][13]
\end{Arabic}}
\flushleft{\begin{hindi}
ऐ ईमान लानेवालो! ऐसे लोगों से मित्रता न करो जिनपर अल्लाह का प्रकोप हुआ, वे आख़िरत से निराश हो चुके है, जिस प्रकार इनकार करनेवाले क़ब्रवालों से निराश हो चुके है
\end{hindi}}
\chapter{As-Saff (The Ranks)}
\begin{Arabic}
\Huge{\centerline{\basmalah}}\end{Arabic}
\flushright{\begin{Arabic}
\quranayah[61][1]
\end{Arabic}}
\flushleft{\begin{hindi}
अल्लाह की तसबीह की हर उस चीज़ ने जो आकाशों और धरती में है। वही प्रभुत्वशाली, तत्वदर्शी है
\end{hindi}}
\flushright{\begin{Arabic}
\quranayah[61][2]
\end{Arabic}}
\flushleft{\begin{hindi}
ऐ ईमान लानेवालो! तुम वह बात क्यों कहते हो जो करते नहीं?
\end{hindi}}
\flushright{\begin{Arabic}
\quranayah[61][3]
\end{Arabic}}
\flushleft{\begin{hindi}
अल्लाह के यहाँ यह अत्यन्त अप्रिय बात है कि तुम वह बात कहो, जो करो नहीं
\end{hindi}}
\flushright{\begin{Arabic}
\quranayah[61][4]
\end{Arabic}}
\flushleft{\begin{hindi}
अल्लाह तो उन लोगों से प्रेम रखता है जो उसके मार्ग में पंक्तिबद्ध होकर लड़ते है मानो वे सीसा पिलाई हुए दीवार है
\end{hindi}}
\flushright{\begin{Arabic}
\quranayah[61][5]
\end{Arabic}}
\flushleft{\begin{hindi}
और याद करो जब मूसा ने अपनी क़ौम के लोगों से कहा, "ऐ मेरी क़ौम के लोगों! तुम मुझे क्यो दुख देते हो, हालाँकि तुम जानते हो कि मैं तुम्हारी ओर भेजा हुआ अल्लाह का रसूल हूँ?" फिर जब उन्होंने टेढ़ अपनाई तो अल्लाह ने भी उनके दिल टेढ़ कर दिए। अल्लाह अवज्ञाकारियों को सीधा मार्ग नहीं दिखाता
\end{hindi}}
\flushright{\begin{Arabic}
\quranayah[61][6]
\end{Arabic}}
\flushleft{\begin{hindi}
और याद करो जबकि मरयम के बेटे ईसा ने कहा, "ऐ इसराईल की संतान! मैं तुम्हारी ओर भेजा हुआ अल्लाह का रसूल हूँ। मैं तौरात की (उस भविष्यवाणी की) पुष्टि करता हूँ जो मुझसे पहले से विद्यमान है और एक रसूल की शुभ सूचना देता हूँ जो मेरे बाद आएगा, उसका नाम अहमद होगा।" किन्तु वह जब उनके पास स्पट्जो प्रमाणों के साथ आया तो उन्होंने कहा, "यह तो जादू है।"
\end{hindi}}
\flushright{\begin{Arabic}
\quranayah[61][7]
\end{Arabic}}
\flushleft{\begin{hindi}
अब उस व्यक्ति से बढ़कर ज़ालिम कौन होगा, जो अल्लाह पर थोपकर झूठ घड़े जबकि इस्लाम (अल्लाह के आगे समर्पण करने) का ओर बुलाया जा रहा हो? अल्लाह ज़ालिम लोगों को सीधा मार्ग नहीं दिखाया करता
\end{hindi}}
\flushright{\begin{Arabic}
\quranayah[61][8]
\end{Arabic}}
\flushleft{\begin{hindi}
वे चाहते है कि अल्लाह के प्रकाश को अपने मुँह की फूँक से बुझा दे, किन्तु अल्लाह अपने प्रकाश को पूर्ण करके ही रहेगा, यद्यपि इनकार करनेवालों को अप्रिय ही लगे
\end{hindi}}
\flushright{\begin{Arabic}
\quranayah[61][9]
\end{Arabic}}
\flushleft{\begin{hindi}
वही है जिसने अपने रसूल को मार्गदर्शन और सत्यधर्म के साथ भेजा, ताकि उसे पूरे के पूरे धर्म पर प्रभुत्व प्रदान कर दे, यद्यपि बहुदेवादियों को अप्रिय ही लगे
\end{hindi}}
\flushright{\begin{Arabic}
\quranayah[61][10]
\end{Arabic}}
\flushleft{\begin{hindi}
ऐ ईमान लानेवालो! क्या मैं तुम्हें एक ऐसा व्यापार बताऊँ जो तुम्हें दुखद यातना से बचा ले?
\end{hindi}}
\flushright{\begin{Arabic}
\quranayah[61][11]
\end{Arabic}}
\flushleft{\begin{hindi}
तुम्हें ईमान लाना है अल्लाह और उसके रसूल पर, और जिहाद करना है अल्लाह के मार्ग में अपने मालों और अपनी जानों से। यही तुम्हारे लिए उत्तम है, यदि तुम जानो
\end{hindi}}
\flushright{\begin{Arabic}
\quranayah[61][12]
\end{Arabic}}
\flushleft{\begin{hindi}
वह तुम्हारे गुनाहों को क्षमा कर देगा और तुम्हें ऐसे बागों में दाखिल करेगा जिनके नीचे नहरें बह रही होगी और उन अच्छे घरों में भी जो सदाबहार बाग़ों में होंगे। यही बड़ी सफलता है
\end{hindi}}
\flushright{\begin{Arabic}
\quranayah[61][13]
\end{Arabic}}
\flushleft{\begin{hindi}
और दूसरी चीज़ भी जो तुम्हें प्रिय है (प्रदान करेगा), "अल्लाह की ओर से सहायता और निकट प्राप्त होनेवाली विजय," ईमानवालों को शुभसूचना दे दो!
\end{hindi}}
\flushright{\begin{Arabic}
\quranayah[61][14]
\end{Arabic}}
\flushleft{\begin{hindi}
ऐ ईमान लानेवालों! अल्लाह के सहायक बनो, जैसा कि मरयम के बेटे ईसा ने हवारियों (साथियों) से कहा था, "कौन है अल्लाह की ओर (बुलाने में) मेरे सहायक?" हवारियों ने कहा, "हम है अल्लाह के सहायक।" फिर इसराईल की संतान में से एक गिरोह ईमान ले आया और एक गिरोह न इनकार किया। अतः हमने उन लोगों को, जो ईमान लाए थे, उनके अपने शत्रुओं के मुकाबले में शक्ति प्रदान की, तो वे छाकर रहे
\end{hindi}}
\chapter{Al-Jumu'ah (The Congregation)}
\begin{Arabic}
\Huge{\centerline{\basmalah}}\end{Arabic}
\flushright{\begin{Arabic}
\quranayah[62][1]
\end{Arabic}}
\flushleft{\begin{hindi}
अल्लाह की तसबीह कर रही है हर वह चीज़ जो आकाशों में है और जो धरती में है, जो सम्राट है, अत्यन्त पवित्र, प्रभुत्वशाली तत्वदर्शी
\end{hindi}}
\flushright{\begin{Arabic}
\quranayah[62][2]
\end{Arabic}}
\flushleft{\begin{hindi}
वही है जिसने उम्मियों में उन्हीं में से एक रसूल उठाया जो उन्हें उसकी आयतें पढ़कर सुनाता है, उन्हें निखारता है और उन्हें किताब और हिकमत (तत्वदर्शिता) की शिक्षा देता है, यद्यपि इससे पहले तो वे खुली हुई गुमराही में पड़े हुए थे, -
\end{hindi}}
\flushright{\begin{Arabic}
\quranayah[62][3]
\end{Arabic}}
\flushleft{\begin{hindi}
और उन दूसरे लोगों को भी (किताब और हिकमत की शिक्षा दे) जो अभी उनसे मिले नहीं है, वे उन्हीं में से होंगे। और वही प्रभुत्वशाली, तत्वशाली है
\end{hindi}}
\flushright{\begin{Arabic}
\quranayah[62][4]
\end{Arabic}}
\flushleft{\begin{hindi}
यह अल्लाह का उदार अनुग्रह है, जिसको चाहता है उसे प्रदान करता है। अल्लाह बड़े अनुग्रह का मालिक है
\end{hindi}}
\flushright{\begin{Arabic}
\quranayah[62][5]
\end{Arabic}}
\flushleft{\begin{hindi}
जिन लोगों पर तारात का बोझ डाला गया, किन्तु उन्होंने उसे न उठाया, उनकी मिसाल उस गधे की-सी है जो किताबे लादे हुए हो। बहुत ही बुरी मिसाल है उन लोगों की जिन्होंने अल्लाह की आयतों को झुठला दिया। अल्लाह ज़ालिमों को सीधा मार्ग नहीं दिखाया करता
\end{hindi}}
\flushright{\begin{Arabic}
\quranayah[62][6]
\end{Arabic}}
\flushleft{\begin{hindi}
कह दो, "ऐ लोगों, जो यहूदी हुए हो! यदि तुम्हें यह गुमान है कि सारे मनुष्यों को छोड़कर तुम ही अल्लाह के प्रेमपात्र हो तो मृत्यु की कामना करो, यदि तुम सच्चे हो।"
\end{hindi}}
\flushright{\begin{Arabic}
\quranayah[62][7]
\end{Arabic}}
\flushleft{\begin{hindi}
किन्तु वे कभी भी उसकी कामना करेंगे, उस (कर्म) के कारण जो उनके हाथों ने आगे भेजा है। अल्लाह ज़ालिमों को भली-भाँति जानता है
\end{hindi}}
\flushright{\begin{Arabic}
\quranayah[62][8]
\end{Arabic}}
\flushleft{\begin{hindi}
कह दो, "मृत्यु जिससे तुम भागते हो, वह तो तुम्हें मिलकर रहेगी, फिर तुम उसकी ओर लौटाए जाओगे जो छिपे और खुले का जाननेवाला है। और वह तुम्हें उससे अवगत करा देगा जो कुछ तुम करते रहे होगे।" -
\end{hindi}}
\flushright{\begin{Arabic}
\quranayah[62][9]
\end{Arabic}}
\flushleft{\begin{hindi}
ऐ ईमान लानेवालो, जब जुमा के दिन नमाज़ के लिए पुकारा जाए तो अल्लाह की याद की ओर दौड़ पड़ो और क्रय-विक्रय छोड़ दो। यह तुम्हारे लिए अच्छा है, यदि तुम जानो
\end{hindi}}
\flushright{\begin{Arabic}
\quranayah[62][10]
\end{Arabic}}
\flushleft{\begin{hindi}
फिर जब नमाज़ पूरी हो जाए तो धरती में फैल जाओ और अल्लाह का उदार अनुग्रह (रोजी) तलाश करो, और अल्लाह को बहुत ज़्यादा याद करते रहो, ताकि तुम सफल हो। -
\end{hindi}}
\flushright{\begin{Arabic}
\quranayah[62][11]
\end{Arabic}}
\flushleft{\begin{hindi}
किन्तु जब वे व्यवहार और खेल-तमाशा देखते है तो उसकी ओर टूट पड़ते है और तुम्हें खड़ा छोड़ देते है। कह दो, "जो कुछ अल्लाह के पास है वह तमाशे और व्यापार से कहीं अच्छा है। और अल्लाह सबसे अच्छा आजीविका प्रदान करनेवाला है।"
\end{hindi}}
\chapter{Al-Munafiqun (The Hypocrites)}
\begin{Arabic}
\Huge{\centerline{\basmalah}}\end{Arabic}
\flushright{\begin{Arabic}
\quranayah[63][1]
\end{Arabic}}
\flushleft{\begin{hindi}
जब मुनाफ़िक (कपटाचारी) तुम्हारे पास आते है तो कहते है, "हम गवाही देते है कि निश्चय ही आप अल्लाह के रसूल है।" अल्लाह जानता है कि निस्संदेह तुम उसके रसूल हो, किेन्तु अल्लाह गवाही देता है कि ये मुनाफ़िक बिलकुल झूठे है
\end{hindi}}
\flushright{\begin{Arabic}
\quranayah[63][2]
\end{Arabic}}
\flushleft{\begin{hindi}
उन्होंने अपनी क़समों को ढाल बना रखा है, इस प्रकार वे अल्लाह के मार्ग से रोकते है। निश्चय ही बुरा है जो वे कर रहे है
\end{hindi}}
\flushright{\begin{Arabic}
\quranayah[63][3]
\end{Arabic}}
\flushleft{\begin{hindi}
यह इस कारण कि वे ईमान लाए, फिर इनकार किया, अतः उनके दिलों पर मुहर लगा दी गई, अब वे कुछ नहीं समझते
\end{hindi}}
\flushright{\begin{Arabic}
\quranayah[63][4]
\end{Arabic}}
\flushleft{\begin{hindi}
तुम उन्हें देखते हो तो उनके शरीर (बाह्य रूप) तुम्हें अच्छे लगते है, औरयदि वे बात करें तो उनकी बात तुम सुनते रह जाओ। किन्तु यह ऐसा ही है मानो वे लकड़ी के कुंदे है, जिन्हें (दीवार के सहारे) खड़ा कर दिया गया हो। हर ज़ोर की आवाज़ को वे अपने ही विरुद्ध समझते है। वही वास्तविक शत्रु हैं, अतः उनसे बचकर रहो। अल्लाह की मार उनपर। वे कहाँ उल्टे फिरे जा रहे है!
\end{hindi}}
\flushright{\begin{Arabic}
\quranayah[63][5]
\end{Arabic}}
\flushleft{\begin{hindi}
और जब उनसे कहा जाता है, "आओ, अल्लाह का रसूल तुम्हारे लिए क्षमा की प्रार्थना करे।" तो वे अपने सिर मटकाते है और तुम देखते हो कि घमंड के साथ खिंचे रहते है
\end{hindi}}
\flushright{\begin{Arabic}
\quranayah[63][6]
\end{Arabic}}
\flushleft{\begin{hindi}
उनके लिए बराबर है चाहे तुम उनके किए क्षमा की प्रार्थना करो या उनके लिए क्षमा की प्रार्थना न करो। अल्लाह उन्हें कदापि क्षमा न करेगा। निश्चय ही अल्लाह अवज्ञाकारियों को सीधा मार्ग नहीं दिखाया करता
\end{hindi}}
\flushright{\begin{Arabic}
\quranayah[63][7]
\end{Arabic}}
\flushleft{\begin{hindi}
वे वहीं लोग है जो कहते है, "उन लोगों पर ख़र्च न करो जो अल्लाह के रसूल के पास रहनेवाले है, ताकि वे तितर-बितर हो जाएँ।" हालाँकि आकाशों और धरती के ख़जाने अल्लाह ही के है, किन्तु वे मुनाफ़िक़ समझते नहीं
\end{hindi}}
\flushright{\begin{Arabic}
\quranayah[63][8]
\end{Arabic}}
\flushleft{\begin{hindi}
वे कहते है, "यदि हम मदीना लौटकर गए तो जो अधिक शक्तिवाला है, वह हीनतर (व्यक्तियों) को वहाँ से निकाल बाहर करेगा।" हालाँकि शक्ति अल्लाह और उसके रसूल और मोमिनों के लिए है, किन्तु वे मुनाफ़िक़ जानते नहीं
\end{hindi}}
\flushright{\begin{Arabic}
\quranayah[63][9]
\end{Arabic}}
\flushleft{\begin{hindi}
ऐ ईमान लानेवालो! तुम्हारे माल तुम्हें अल्लाह की याद से ग़ाफ़िल न कर दें और न तुम्हारी सन्तान ही। जो कोई ऐसा करे तो ऐसे ही लोग घाटे में रहनेवाले है
\end{hindi}}
\flushright{\begin{Arabic}
\quranayah[63][10]
\end{Arabic}}
\flushleft{\begin{hindi}
हमने तुम्हें जो कुछ दिया है उसमें से ख़र्च करो इससे पहले कि तुममें से किसी की मृत्यु आ जाए और उस समय वह करने लगे, "ऐ मेरे रब! तूने मुझे कुछ थोड़े समय तक और मुहलत क्यों न दी कि मैं सदक़ा (दान) करता (मुझे मुहलत दे कि मैं सदक़ा करूँ) और अच्छे लोगों में सम्मिलित हो जाऊँ।"
\end{hindi}}
\flushright{\begin{Arabic}
\quranayah[63][11]
\end{Arabic}}
\flushleft{\begin{hindi}
किन्तु अल्लाह, किसी व्यक्ति को जब तक उसका नियत समय आ जाता है, कदापि मुहलत नहीं देता। और जो कुछ तुम करते हो अल्लाह उसकी पूरी ख़बर रखता है
\end{hindi}}
\chapter{At-Taghabun (The Manifestation of Losses)}
\begin{Arabic}
\Huge{\centerline{\basmalah}}\end{Arabic}
\flushright{\begin{Arabic}
\quranayah[64][1]
\end{Arabic}}
\flushleft{\begin{hindi}
अल्लाह की तसबीह कर रही है हर वह चीज़ जो आकाशों में है और जो धरती में है। उसी की बादशाही है और उसी के लिए प्रशंसा है और उसे हर चीज़ की सामर्थ्य प्राप्त है
\end{hindi}}
\flushright{\begin{Arabic}
\quranayah[64][2]
\end{Arabic}}
\flushleft{\begin{hindi}
वही है जिसने तुम्हें पैदा किया, फिर तुममें से कोई तो इनकार करनेवाला है और तुममें से कोई ईमानवाला है, और तुम जो कुछ भी करते हो अल्लाह उसे देख रहा होता है
\end{hindi}}
\flushright{\begin{Arabic}
\quranayah[64][3]
\end{Arabic}}
\flushleft{\begin{hindi}
उसने आकाशों और धरती को हक़ के साथ पैदा किया और तुम्हारा रूप बनाया, तो बहुत ही अच्छे बनाए तुम्हारे रूप और उसी की ओर अन्ततः जाना है
\end{hindi}}
\flushright{\begin{Arabic}
\quranayah[64][4]
\end{Arabic}}
\flushleft{\begin{hindi}
वह जानता है जो कुछ आकाशों और धरती में है और उसे भी जानता है जो कुछ तुम छिपाते हो और जो कुछ तुम प्रकट करते हो। अल्लाह तो सीनों में छिपी बात तक को जानता है
\end{hindi}}
\flushright{\begin{Arabic}
\quranayah[64][5]
\end{Arabic}}
\flushleft{\begin{hindi}
क्या तुम्हें उन लोगों की ख़बर नहीं पहुँची जिन्होंने इससे पहले इनकार किया था, फिर उन्होंने अपने कर्म के वबाल का मज़ा चखा और उनके लिए एक दुखद यातना भी है
\end{hindi}}
\flushright{\begin{Arabic}
\quranayah[64][6]
\end{Arabic}}
\flushleft{\begin{hindi}
यह इस कारण कि उनके पास उनके रसूल स्पष्ट प्रमाण लेकर आते रहे, किन्तु उन्होंने कहा, "क्या मनुष्य हमें मार्ग दिखाएँगे?" इस प्रकार उन्होंने इनकार किया और मुँह फेर लिया, तब अल्लाह भी उनसे बेपरवाह हो गया। अल्लाह तो है ही निस्पृह, अपने आप में स्वयं प्रशंसित
\end{hindi}}
\flushright{\begin{Arabic}
\quranayah[64][7]
\end{Arabic}}
\flushleft{\begin{hindi}
जिन लोगों ने इनकार किया उन्होंने दावा किया वे मरने के पश्चात कदापि न उठाए जाएँगे। कह दो, "क्यों नहीं, मेरे रब की क़सम! तुम अवश्य उठाए जाओगे, फिर जो कुछ तुमने किया है उससे तुम्हें अवगत करा दिया जाएगा। और अल्लाह के लिए यह अत्यन्त सरल है।"
\end{hindi}}
\flushright{\begin{Arabic}
\quranayah[64][8]
\end{Arabic}}
\flushleft{\begin{hindi}
अतः ईमान लाओ, अल्लाह पर और उसके रसूल पर और उस प्रकाश पर जिसे हमने अवतरित किया है। तुम जो कुछ भी करते हो अल्लाह उसकी पूरी ख़बर रखता है
\end{hindi}}
\flushright{\begin{Arabic}
\quranayah[64][9]
\end{Arabic}}
\flushleft{\begin{hindi}
इकट्ठा होने के दिन वह तुम्हें इकट्ठा करेगा, वह परस्पर लाभ-हानि का दिन होगा। जो भी अल्लाह पर ईमान लाए और अच्छा कर्म करे उसकी बुराईयाँ अल्लाह उससे दूर कर देगा और उसे ऐसे बाग़ों में दाख़िल करेगा जिनके नीचे नहरें बह रही होंगी, उनमें वे सदैव रहेंगे। यही बड़ी सफलता है
\end{hindi}}
\flushright{\begin{Arabic}
\quranayah[64][10]
\end{Arabic}}
\flushleft{\begin{hindi}
रहे वे लोग जिन्होंने इनकार किया और हमारी आयतों को झुठलाया, वही आगवाले है जिसमें वे सदैव रहेंगे। अन्ततः लौटकर पहुँचने की वह बहुत ही बुरी जगह है
\end{hindi}}
\flushright{\begin{Arabic}
\quranayah[64][11]
\end{Arabic}}
\flushleft{\begin{hindi}
अल्लाह की अनुज्ञा के बिना कोई भी मुसीबत नहीं आती। जो अल्लाह पर ईमान ले आए अल्लाह उसके दिल को मार्ग दिखाता है, और अल्लाह हर चीज को भली-भाँति जानता है
\end{hindi}}
\flushright{\begin{Arabic}
\quranayah[64][12]
\end{Arabic}}
\flushleft{\begin{hindi}
अल्लाह की आज्ञा का पालन करो और रसूल की आज्ञा का पालन करो, किन्तु यदि तुम मुँह मोड़ते हो तो हमारे रसूल पर बस स्पष्ट रूप से (संदेश) पहुँचा देने की ज़िम्मेदारी है
\end{hindi}}
\flushright{\begin{Arabic}
\quranayah[64][13]
\end{Arabic}}
\flushleft{\begin{hindi}
अल्लाह वह है जिसके सिवा कोई पूज्य-प्रभु नहीं। अतः अल्लाह ही पर ईमानवालों को भरोसा करना चाहिए
\end{hindi}}
\flushright{\begin{Arabic}
\quranayah[64][14]
\end{Arabic}}
\flushleft{\begin{hindi}
ऐ ईमान लानेवालो, तुम्हारी पत्नियों और तुम्हारी सन्तान में से कुछ ऐसे भी है जो तुम्हारे शत्रु है। अतः उनसे होशियार रहो। और यदि तुम माफ़ कर दो और टाल जाओ और क्षमा कर दो निश्चय ही अल्लाह बड़ा क्षमाशील, अत्यन्त दयावान है
\end{hindi}}
\flushright{\begin{Arabic}
\quranayah[64][15]
\end{Arabic}}
\flushleft{\begin{hindi}
तुम्हारे माल और तुम्हारी सन्तान तो केवल एक आज़माइश है, और अल्लाह ही है जिसके पास बड़ा प्रतिदान है
\end{hindi}}
\flushright{\begin{Arabic}
\quranayah[64][16]
\end{Arabic}}
\flushleft{\begin{hindi}
अतः जहाँ तक तुम्हारे बस में हो अल्लाह का डर रखो और सुनो और आज्ञापालन करो और ख़र्च करो अपनी भलाई के लिए। और जो अपने मन के लोभ एवं कृपणता से सुरक्षित रहा तो ऐसे ही लोग सफल है
\end{hindi}}
\flushright{\begin{Arabic}
\quranayah[64][17]
\end{Arabic}}
\flushleft{\begin{hindi}
यदि तुम अल्लाह को अच्छा ऋण दो तो वह उसे तुम्हारे लिए कई गुना बढ़ा देगा और तुम्हें क्षमा कर देगा। अल्लाह बड़ा गुणग्राहक और सहनशील है,
\end{hindi}}
\flushright{\begin{Arabic}
\quranayah[64][18]
\end{Arabic}}
\flushleft{\begin{hindi}
परोक्ष और प्रत्यक्ष को जानता है, प्रभुत्वशाली, तत्वदर्शी है
\end{hindi}}
\chapter{At-Talaq (Divorce)}
\begin{Arabic}
\Huge{\centerline{\basmalah}}\end{Arabic}
\flushright{\begin{Arabic}
\quranayah[65][1]
\end{Arabic}}
\flushleft{\begin{hindi}
ऐ नबी! जब तुम लोग स्त्रियों को तलाक़ दो तो उन्हें तलाक़ उनकी इद्दत के हिसाब से दो। और इद्दत की गणना करो और अल्लाह का डर रखो, जो तुम्हारा रब है। उन्हें उनके घरों से न निकालो और न वे स्वयं निकलें, सिवाय इसके कि वे कोई स्पष्ट। अशोभनीय कर्म कर बैठें। ये अल्लाह की नियत की हुई सीमाएँ है - और जो अल्लाह की सीमाओं का उल्लंघन करे तो उसने स्वयं अपने आप पर ज़ुल्म किया - तुम नहीं जानते, कदाचित इस (तलाक़) के पश्चात अल्लाह कोई सूरत पैदा कर दे
\end{hindi}}
\flushright{\begin{Arabic}
\quranayah[65][2]
\end{Arabic}}
\flushleft{\begin{hindi}
फिर जब वे अपनी नियत इद्दत को पहुँचे तो या तो उन्हें भली रीति से रोक लो या भली रीति से अलग कर दो। और अपने में से दो न्यायप्रिय आदमियों को गवाह बना दो और अल्लाह के लिए गवाही को दुरुस्त रखो। इसकी नसीहत उस व्यक्ति को की जाती है जो अल्लाह और अन्तिम दिन पर ईमान रखेगा उसके लिए वह (परेशानी से) निकलने का राह पैदा कर देगा
\end{hindi}}
\flushright{\begin{Arabic}
\quranayah[65][3]
\end{Arabic}}
\flushleft{\begin{hindi}
और उसे वहाँ से रोज़ी देगा जिसका उसे गुमान भी न होगा। जो अल्लाह पर भरोसा करे तो वह उसके लिए काफ़ी है। निश्चय ही अल्लाह अपना काम पूरा करके रहता है। अल्लाह ने हर चीज़ का एक अन्दाजा नियत कर रखा है
\end{hindi}}
\flushright{\begin{Arabic}
\quranayah[65][4]
\end{Arabic}}
\flushleft{\begin{hindi}
और तुम्हारी स्त्रियों में से जो मासिक धर्म से निराश हो चुकी हों, यदि तुम्हें संदेह हो तो उनकी इद्दत तीन मास है और इसी प्रकार उनकी भी जो अभी रजस्वला नहीं हुई। और जो गर्भवती स्त्रियाँ हो उनकी इद्दत उनके शिशु-प्रसव तक है। जो कोई अल्लाह का डर रखेगा उसके मामले में वह आसानी पैदा कर देगा
\end{hindi}}
\flushright{\begin{Arabic}
\quranayah[65][5]
\end{Arabic}}
\flushleft{\begin{hindi}
यह अल्लाह का आदेश है जो उसने तुम्हारी ओर उतारा है। और जो कोई अल्लाह का डर रखेगा उससे वह उसकी बुराईयाँ दूर कर देगा और उसके प्रतिदान को बड़ा कर देगा
\end{hindi}}
\flushright{\begin{Arabic}
\quranayah[65][6]
\end{Arabic}}
\flushleft{\begin{hindi}
अपनी हैसियत के अनुसार यहाँ तुम स्वयं रहते हो उन्हें भी उसी जगह रखो। और उन्हें तंग करने के लिए उन्हें हानि न पहुँचाओ। और यदि वे गर्भवती हो तो उनपर ख़र्च करते रहो जब तक कि उनका शिशु-प्रसव न हो जाए। फिर यदि वे तुम्हारे लिए (शिशु को) दूध पिलाएँ तो तुम उन्हें उनका पारिश्रामिक दो और आपस में भली रीति से परस्पर बातचीत के द्वार कोई बात तय कर लो। और यदि तुम दोनों में कोई कठिनाई हो तो फिर कोई दूसरी स्त्री उसके लिए दूध पिला देगी
\end{hindi}}
\flushright{\begin{Arabic}
\quranayah[65][7]
\end{Arabic}}
\flushleft{\begin{hindi}
चाहिए कि समाई (सामर्थ्य) वाला अपनी समाई के अनुसार ख़र्च करे और जिसे उसकी रोज़ी नपी-तुली मिली हो तो उसे चाहिए कि अल्लाह ने उसे जो कुछ भी दिया है उसी में से वह ख़र्च करे। जितना कुछ दिया है उससे बढ़कर अल्लाह किसी व्यक्ति पर ज़िम्मेदारी का बोझ नहीं डालता। जल्द ही अल्लाह कठिनाई के बाद आसानी पैदा कर देगा
\end{hindi}}
\flushright{\begin{Arabic}
\quranayah[65][8]
\end{Arabic}}
\flushleft{\begin{hindi}
कितनी ही बस्तियाँ हैं जिन्होंने रब और उसके रसूलों के आदेश के मुक़ाबले में सरकशी की, तो हमने उनकी सख़्त पकड़ की और उन्हें बुरी यातना दी
\end{hindi}}
\flushright{\begin{Arabic}
\quranayah[65][9]
\end{Arabic}}
\flushleft{\begin{hindi}
अतः उन्होंने अपने किए के वबाल का मज़ा चख लिया और उनकी कार्य-नीति का परिणाम घाटा ही रहा
\end{hindi}}
\flushright{\begin{Arabic}
\quranayah[65][10]
\end{Arabic}}
\flushleft{\begin{hindi}
अल्लाह ने उनके लिए कठोर यातना तैयार कर रखी है। अतः ऐ बुद्धि और समझवालो जो ईमान लाए हो! अल्लाह का डर रखो। अल्लाह ने तुम्हारी ओर एक याददिहानी उतार दी है
\end{hindi}}
\flushright{\begin{Arabic}
\quranayah[65][11]
\end{Arabic}}
\flushleft{\begin{hindi}
(अर्थात) एक रसूल जो तुम्हें अल्लाह की स्पष्ट आयतें पढ़कर सुनाता है, ताकि वह उन लोगों को, जो ईमान लाए और उन्होंने अच्छे कर्म किए, अँधेरों से निकालकर प्रकाश की ओर ले आए। जो कोई अल्लाह पर ईमान लाए और अच्छा कर्म करे, उसे वह ऐसे बाग़़ों में दाख़िल करेगा जिनके नीचे नहरें बह रही होगी - ऐसे लोग उनमें सदैव रहेंगे - अल्लाह ने उनके लिए उत्तम रोज़ी रखी है
\end{hindi}}
\flushright{\begin{Arabic}
\quranayah[65][12]
\end{Arabic}}
\flushleft{\begin{hindi}
अल्लाह ही है जिसने सात आकाश बनाए और उन्ही के सदृश धरती से भी। उनके बीच (उसका) आदेश उतरता रहता है ताकि तुम जान लो कि अल्लाह को हर चीज़ का सामर्थ्य प्राप्त है और यह कि अल्लाह हर चीज़ को अपनी ज्ञान-परिधि में लिए हुए है
\end{hindi}}
\chapter{At-Tahrim (The Prohibition)}
\begin{Arabic}
\Huge{\centerline{\basmalah}}\end{Arabic}
\flushright{\begin{Arabic}
\quranayah[66][1]
\end{Arabic}}
\flushleft{\begin{hindi}
ऐ नबी! जिस चीज़ को अल्लाह ने तुम्हारे लिए वैध ठहराया है उसे तुम अपनी पत्नियों की प्रसन्नता प्राप्त करने के लिए क्यो अवैध करते हो? अल्लाह बड़ा क्षमाशील, अत्यन्त दयावान है
\end{hindi}}
\flushright{\begin{Arabic}
\quranayah[66][2]
\end{Arabic}}
\flushleft{\begin{hindi}
अल्लाह ने तुम लोगों के लिए तुम्हारी अपनी क़समों की पाबंदी से निकलने का उपाय निश्चित कर दिया है। अल्लाह तुम्हारा संरक्षक है और वही सर्वज्ञ, अत्यन्त तत्वदर्शी है
\end{hindi}}
\flushright{\begin{Arabic}
\quranayah[66][3]
\end{Arabic}}
\flushleft{\begin{hindi}
जब नबी ने अपनी पत्ऩियों में से किसी से एक गोपनीय बात कही, फिर जब उसने उसकी ख़बर कर दी और अल्लाह ने उसे उसपर ज़ाहिर कर दिया, तो उसने उसे किसी हद तक बता दिया और किसी हद तक टाल गया। फिर जब उसने उसकी उसे ख़बर की तो वह बोली, "आपको इसकी ख़बर किसने दी?" उसने कहा, "मुझे उसने ख़बर दी जो सब कुछ जाननेवाला, ख़बर रखनेवाला है।"
\end{hindi}}
\flushright{\begin{Arabic}
\quranayah[66][4]
\end{Arabic}}
\flushleft{\begin{hindi}
यदि तुम दोनों अल्लाह की ओर रुजू हो तो तुम्हारे दिल तो झुक ही चुके हैं, किन्तु यदि तुम उसके विरुद्ध एक-दूसरे की सहायता करोगी तो अल्लाह उसकी संरक्षक है, और जिबरील और नेक ईमानवाले भी, और इसके बाद फ़रिश्ते भी उसके सहायक है
\end{hindi}}
\flushright{\begin{Arabic}
\quranayah[66][5]
\end{Arabic}}
\flushleft{\begin{hindi}
इसकी बहुत सम्भावना है कि यदि वह तुम्हें तलाक़ दे दे तो उसका रब तुम्हारे बदले में तुमसे अच्छी पत्ऩियाँ उसे प्रदान करे - मुस्लिम, ईमानवाली, आज्ञाकारिणी, तौबा करनेवाली, इबादत करनेवाली, (अल्लाह के मार्ग में) सफ़र करनेवाली, विवाहिता और कुँवारियाँ भी
\end{hindi}}
\flushright{\begin{Arabic}
\quranayah[66][6]
\end{Arabic}}
\flushleft{\begin{hindi}
ऐ ईमान लानेवालो! अपने आपको और अपने घरवालों को उस आग से बचाओ जिसका ईधन मनुष्य और पत्थर होंगे, जिसपर कठोर स्वभाव के ऐसे बलशाली फ़रिश्ते नियुक्त होंगे जो अल्लाह की अवज्ञा उसमें नहीं करेंगे जो आदेश भी वह उन्हें देगा, और वे वही करेंगे जिसका उन्हें आदेश दिया जाएगा
\end{hindi}}
\flushright{\begin{Arabic}
\quranayah[66][7]
\end{Arabic}}
\flushleft{\begin{hindi}
ऐ इनकार करनेवालो! आज उज़्र पेश न करो। तुम्हें बदले में वही तो दिया जा रहा है जो कुछ तुम करते रहे हो
\end{hindi}}
\flushright{\begin{Arabic}
\quranayah[66][8]
\end{Arabic}}
\flushleft{\begin{hindi}
ऐ ईमान लानेवाले! अल्लाह के आगे तौबा करो, विशुद्ध तौबा। बहुत सम्भव है कि तुम्हारा रब तुम्हारी बुराइयाँ तुमसे दूर कर दे और तुम्हें ऐसे बाग़ों में दाख़िल करे जिनके नीचे नहरे बह रही होंगी, जिस दिन अल्लाह नबी को और उनको जो ईमान लाकर उसके साथ हुए, रुसवा न करेगा। उनका प्रकाश उनके आगे-आगे दौड़ रहा होगा और उनके दाहिने हाथ मे होगा। वे कह रहे होंगे, "ऐ हमारे रब! हमारे लिए हमारे प्रकाश को पूर्ण कर दे और हमें क्षमा कर। निश्चय ही तू हर चीज़ की सामर्थ्य रखता है।"
\end{hindi}}
\flushright{\begin{Arabic}
\quranayah[66][9]
\end{Arabic}}
\flushleft{\begin{hindi}
ऐ नबी! इनकार करनेवालों और कपटाचारियों से जिहाद करो और उनके साथ सख़्ती से पेश आओ। उनका ठिकाना जहन्नम है और वह अन्ततः पहुँचने की बहुत बुरी जगह है
\end{hindi}}
\flushright{\begin{Arabic}
\quranayah[66][10]
\end{Arabic}}
\flushleft{\begin{hindi}
अल्लाह ने इनकार करनेवालों के लिए नूह की स्त्री और लूत की स्त्री की मिसाल पेश की है। वे हमारे बन्दों में से दो नेक बन्दों के अधीन थीं। किन्तु उन दोनों स्त्रियों ने उनसे विश्वासघात किया तो अल्लाह के मुक़ाबले में उनके कुछ काम न आ सके और कह दिया गया, "प्रवेश करनेवालों के साथ दोनों आग में प्रविष्ट हो जाओ।"
\end{hindi}}
\flushright{\begin{Arabic}
\quranayah[66][11]
\end{Arabic}}
\flushleft{\begin{hindi}
और ईमान लानेवालों के लिए अल्लाह ने फ़िरऔन की स्त्री की मिसाल पेश की है, जबकि उसने कहा, "ऐ मेरे रब! तू मेरे लिए अपने पास जन्नत में एक घर बना और मुझे फ़िरऔन और उसके कर्म से छुटकारा दे, और छुटकारा दे मुझे ज़ालिम लोगों से।"
\end{hindi}}
\flushright{\begin{Arabic}
\quranayah[66][12]
\end{Arabic}}
\flushleft{\begin{hindi}
और इमरान की बेटी मरयम की मिसाल पेश ही है जिसने अपने सतीत्व की रक्षा की थी, फिर हमने उस स्त्री के भीतर अपनी रूह फूँक दी और उसने अपने रब के बोलों और उसकी किताबों की पुष्टि की और वह भक्ति-प्रवृत्त आज्ञाकारियों में से थी
\end{hindi}}
\chapter{Al-Mulk (The Kingdom)}
\begin{Arabic}
\Huge{\centerline{\basmalah}}\end{Arabic}
\flushright{\begin{Arabic}
\quranayah[67][1]
\end{Arabic}}
\flushleft{\begin{hindi}
बड़ा बरकतवाला है वह जिसके हाथ में सारी बादशाही है और वह हर चीज़ की सामर्थ्य रखता है। -
\end{hindi}}
\flushright{\begin{Arabic}
\quranayah[67][2]
\end{Arabic}}
\flushleft{\begin{hindi}
जिसने पैदा किया मृत्यु और जीवन को, ताकि तुम्हारी परीक्षा करे कि तुममें कर्म की दृष्टि से कौन सबसे अच्छा है। वह प्रभुत्वशाली, बड़ा क्षमाशील है। -
\end{hindi}}
\flushright{\begin{Arabic}
\quranayah[67][3]
\end{Arabic}}
\flushleft{\begin{hindi}
जिसने ऊपर-तले सात आकाश बनाए। तुम रहमान की रचना में कोई असंगति और विषमता न देखोगे। फिर नज़र डालो, "क्या तुम्हें कोई बिगाड़ दिखाई देता है?"
\end{hindi}}
\flushright{\begin{Arabic}
\quranayah[67][4]
\end{Arabic}}
\flushleft{\begin{hindi}
फिर दोबारा नज़र डालो। निगाह रद्द होकर और थक-हारकर तुम्हारी ओर पलट आएगी
\end{hindi}}
\flushright{\begin{Arabic}
\quranayah[67][5]
\end{Arabic}}
\flushleft{\begin{hindi}
हमने निकटवर्ती आकाश को दीपों से सजाया और उन्हें शैतानों के मार भगाने का साधन बनाया और उनके लिए हमने भड़कती आग की यातना तैयार कर रखी है
\end{hindi}}
\flushright{\begin{Arabic}
\quranayah[67][6]
\end{Arabic}}
\flushleft{\begin{hindi}
जिन लोगों ने अपने रब के साथ कुफ़्र किया उनके लिए जहन्नम की यातना है और वह बहुत ही बुरा ठिकाना है
\end{hindi}}
\flushright{\begin{Arabic}
\quranayah[67][7]
\end{Arabic}}
\flushleft{\begin{hindi}
जब वे उसमें डाले जाएँगे तो उसकी दहाड़ने की भयानक आवाज़ सुनेंगे और वह प्रकोप से बिफर रही होगी।
\end{hindi}}
\flushright{\begin{Arabic}
\quranayah[67][8]
\end{Arabic}}
\flushleft{\begin{hindi}
ऐसा प्रतीत होगा कि प्रकोप के कारण अभी फट पड़ेगी। हर बार जब भी कोई समूह उसमें डाला जाएगा तो उसके कार्यकर्ता उनसे पूछेंगे, "क्या तुम्हारे पास कोई सावधान करनेवाला नहीं आया?"
\end{hindi}}
\flushright{\begin{Arabic}
\quranayah[67][9]
\end{Arabic}}
\flushleft{\begin{hindi}
वे कहेंगे, "क्यों नहीं, अवश्य हमारे पास आया था, किन्तु हमने झुठला दिया और कहा कि अल्लाह ने कुछ भी नहीं अवतरित किया। तुम तो बस एक बड़ी गुमराही में पड़े हुए हो।"
\end{hindi}}
\flushright{\begin{Arabic}
\quranayah[67][10]
\end{Arabic}}
\flushleft{\begin{hindi}
और वे कहेंगे, "यदि हम सुनते या बुद्धि से काम लेते तो हम दहकती आग में पड़नेवालों में सम्मिलित न होते।"
\end{hindi}}
\flushright{\begin{Arabic}
\quranayah[67][11]
\end{Arabic}}
\flushleft{\begin{hindi}
इस प्रकार वे अपने गुनाहों को स्वीकार करेंगे, तो धिक्कार हो दहकती आगवालों पर!
\end{hindi}}
\flushright{\begin{Arabic}
\quranayah[67][12]
\end{Arabic}}
\flushleft{\begin{hindi}
जो लोग परोक्ष में रहते हुए अपने रब से डरते है, उनके लिए क्षमा और बड़ा बदला है
\end{hindi}}
\flushright{\begin{Arabic}
\quranayah[67][13]
\end{Arabic}}
\flushleft{\begin{hindi}
तुम अपनी बात छिपाओ या उसे व्यक्त करो, वह तो सीनों में छिपी बातों तक को जानता है
\end{hindi}}
\flushright{\begin{Arabic}
\quranayah[67][14]
\end{Arabic}}
\flushleft{\begin{hindi}
क्या वह नहीं जानेगा जिसने पैदा किया? वह सूक्ष्मदर्शी, ख़बर रखनेवाला है
\end{hindi}}
\flushright{\begin{Arabic}
\quranayah[67][15]
\end{Arabic}}
\flushleft{\begin{hindi}
वही तो है जिसने तुम्हारे लिए धरती को वशीभूत किया। अतः तुम उसके (धरती के) कन्धों पर चलो और उसकी रोज़ी में से खाओ, उसी की ओर दोबारा उठकर (जीवित होकर) जाना है
\end{hindi}}
\flushright{\begin{Arabic}
\quranayah[67][16]
\end{Arabic}}
\flushleft{\begin{hindi}
क्या तुम उससे निश्चिन्त हो जो आकाश में है कि तुम्हें धरती में धँसा दे, फिर क्या देखोगे कि वह डाँवाडोल हो रही है?
\end{hindi}}
\flushright{\begin{Arabic}
\quranayah[67][17]
\end{Arabic}}
\flushleft{\begin{hindi}
या तुम उससे निश्चिन्त हो जो आकाश में है कि वह तुमपर पथराव करनेवाली वायु भेज दे? फिर तुम जान लोगे कि मेरी चेतावनी कैसी होती है
\end{hindi}}
\flushright{\begin{Arabic}
\quranayah[67][18]
\end{Arabic}}
\flushleft{\begin{hindi}
उन लोगों ने भी झुठलाया जो उनसे पहले थे, फिर कैसा रहा मेरा इनकार!
\end{hindi}}
\flushright{\begin{Arabic}
\quranayah[67][19]
\end{Arabic}}
\flushleft{\begin{hindi}
क्या उन्होंने अपने ऊपर पक्षियों को पंक्तबन्द्ध पंख फैलाए और उन्हें समेटते नहीं देखा? उन्हें रहमान के सिवा कोई और नहीं थामें रहता। निश्चय ही वह हर चीज़ को देखता है
\end{hindi}}
\flushright{\begin{Arabic}
\quranayah[67][20]
\end{Arabic}}
\flushleft{\begin{hindi}
या वह कौन है जो तुम्हारी सेना बनकर रहमान के मुक़ाबले में तुम्हारी सहायता करे। इनकार करनेवाले तो बस धोखे में पड़े हुए है
\end{hindi}}
\flushright{\begin{Arabic}
\quranayah[67][21]
\end{Arabic}}
\flushleft{\begin{hindi}
या वह कौन है जो तुम्हें रोज़ी दे, यदि वह अपनी रोज़ी रोक ले? नहीं, बल्कि वे तो सरकशी और नफ़रत ही पर अड़े हुए है
\end{hindi}}
\flushright{\begin{Arabic}
\quranayah[67][22]
\end{Arabic}}
\flushleft{\begin{hindi}
तो क्या वह व्यक्ति जो अपने मुँह के बल औंधा चलता हो वह अधिक सीधे मार्ग पर ह या वह जो सीधा होकर सीधे मार्ग पर चल रहा है?
\end{hindi}}
\flushright{\begin{Arabic}
\quranayah[67][23]
\end{Arabic}}
\flushleft{\begin{hindi}
कह दो, "वही है जिसने तुम्हें पैदा किया और तुम्हारे लिए कान और आँखे और दिल बनाए। तुम कृतज्ञता थोड़े ही दिखाते हो।"
\end{hindi}}
\flushright{\begin{Arabic}
\quranayah[67][24]
\end{Arabic}}
\flushleft{\begin{hindi}
कह दो, "वही है जिसने तुम्हें धरती में फैलाया और उसी की ओर तुम एकत्र किए जा रहे हो।"
\end{hindi}}
\flushright{\begin{Arabic}
\quranayah[67][25]
\end{Arabic}}
\flushleft{\begin{hindi}
वे कहते है, "यदि तुम सच्चे हो तो यह वादा कब पूरा होगा?"
\end{hindi}}
\flushright{\begin{Arabic}
\quranayah[67][26]
\end{Arabic}}
\flushleft{\begin{hindi}
कह दो, "इसका ज्ञान तो बस अल्लाह ही के पास है और मैं तो एक स्पष्ट॥ सचेत करनेवाला हूँ।"
\end{hindi}}
\flushright{\begin{Arabic}
\quranayah[67][27]
\end{Arabic}}
\flushleft{\begin{hindi}
फिर जब वे उसे निकट देखेंगे तो उन लोगों के चेहरे बिगड़ जाएँगे जिन्होंने इनकार की नीति अपनाई; और कहा जाएगा, "यही है वह चीज़ जिसकी तुम माँग कर रहे थे।"
\end{hindi}}
\flushright{\begin{Arabic}
\quranayah[67][28]
\end{Arabic}}
\flushleft{\begin{hindi}
कहो, "क्या तुमने यह भी सोचा कि यदि अल्लाह मुझे और उन्हें भी, जो मेरे साथ है, विनष्ट ही कर दे या वह हम पर दया करे, आख़िर इनकार करनेवालों को दुखद यातना से कौन पनाह देगा?"
\end{hindi}}
\flushright{\begin{Arabic}
\quranayah[67][29]
\end{Arabic}}
\flushleft{\begin{hindi}
कह दो, "वह रहमान है। उसी पर हम ईमान लाए है और उसी पर हमने भरोसा किया। तो शीघ्र ही तुम्हें मालूम हो जाएगा कि खुली गुमराही में कौन पड़ा हुआ है।"
\end{hindi}}
\flushright{\begin{Arabic}
\quranayah[67][30]
\end{Arabic}}
\flushleft{\begin{hindi}
कहो, "क्या तुमने यह भी सोचा कि यदि तुम्हारा पानी (धरती में) नीचे उतर जाए तो फिर कौन तुम्हें लाकर देगा निर्मल प्रवाहित जल?"
\end{hindi}}
\chapter{Al-Qalam (The Pen)}
\begin{Arabic}
\Huge{\centerline{\basmalah}}\end{Arabic}
\flushright{\begin{Arabic}
\quranayah[68][1]
\end{Arabic}}
\flushleft{\begin{hindi}
नून॰। गवाह है क़लम और वह चीज़ जो वे लिखते है,
\end{hindi}}
\flushright{\begin{Arabic}
\quranayah[68][2]
\end{Arabic}}
\flushleft{\begin{hindi}
तुम अपने रब की अनुकम्पा से कोई दीवाने नहीं हो
\end{hindi}}
\flushright{\begin{Arabic}
\quranayah[68][3]
\end{Arabic}}
\flushleft{\begin{hindi}
निश्चय ही तुम्हारे लिए ऐसा प्रतिदान है जिसका क्रम कभी टूटनेवाला नहीं
\end{hindi}}
\flushright{\begin{Arabic}
\quranayah[68][4]
\end{Arabic}}
\flushleft{\begin{hindi}
निस्संदेह तुम एक महान नैतिकता के शिखर पर हो
\end{hindi}}
\flushright{\begin{Arabic}
\quranayah[68][5]
\end{Arabic}}
\flushleft{\begin{hindi}
अतः शीघ्र ही तुम भी देख लोगे और वे भी देख लेंगे
\end{hindi}}
\flushright{\begin{Arabic}
\quranayah[68][6]
\end{Arabic}}
\flushleft{\begin{hindi}
कि तुममें से कौन विभ्रमित है
\end{hindi}}
\flushright{\begin{Arabic}
\quranayah[68][7]
\end{Arabic}}
\flushleft{\begin{hindi}
निस्संदेह तुम्हारा रब उसे भली-भाँति जानता है जो उसके मार्ग से भटक गया है, और वही उन लोगों को भी जानता है जो सीधे मार्ग पर हैं
\end{hindi}}
\flushright{\begin{Arabic}
\quranayah[68][8]
\end{Arabic}}
\flushleft{\begin{hindi}
अतः तुम झुठलानेवालों को कहना न मानना
\end{hindi}}
\flushright{\begin{Arabic}
\quranayah[68][9]
\end{Arabic}}
\flushleft{\begin{hindi}
वे चाहते है कि तुम ढीले पड़ो, इस कारण वे चिकनी-चुपड़ी बातें करते है
\end{hindi}}
\flushright{\begin{Arabic}
\quranayah[68][10]
\end{Arabic}}
\flushleft{\begin{hindi}
तुम किसी भी ऐसे व्यक्ति की बात न मानना जो बहुत क़समें खानेवाला, हीन है,
\end{hindi}}
\flushright{\begin{Arabic}
\quranayah[68][11]
\end{Arabic}}
\flushleft{\begin{hindi}
कचोके लगाता, चुग़लियाँ खाता फिरता हैं,
\end{hindi}}
\flushright{\begin{Arabic}
\quranayah[68][12]
\end{Arabic}}
\flushleft{\begin{hindi}
भलाई से रोकता है, सीमा का उल्लंघन करनेवाला, हक़ मारनेवाला है,
\end{hindi}}
\flushright{\begin{Arabic}
\quranayah[68][13]
\end{Arabic}}
\flushleft{\begin{hindi}
क्रूर है फिर अधम भी।
\end{hindi}}
\flushright{\begin{Arabic}
\quranayah[68][14]
\end{Arabic}}
\flushleft{\begin{hindi}
इस कारण कि वह धन और बेटोंवाला है
\end{hindi}}
\flushright{\begin{Arabic}
\quranayah[68][15]
\end{Arabic}}
\flushleft{\begin{hindi}
जब उसे हमारी आयतें सुनाई जाती है तो कहता है, "ये तो पहले लोगों की कहानियाँ हैं!"
\end{hindi}}
\flushright{\begin{Arabic}
\quranayah[68][16]
\end{Arabic}}
\flushleft{\begin{hindi}
शीघ्र ही हम उसकी सूँड पर दाग़ लगाएँगे
\end{hindi}}
\flushright{\begin{Arabic}
\quranayah[68][17]
\end{Arabic}}
\flushleft{\begin{hindi}
हमने उन्हें परीक्षा में डाला है जैसे बाग़वालों को परीक्षा में डाला था, जबकि उन्होंने क़सम खाई कि वे प्रातःकाल अवश्य उस (बाग़) के फल तोड़ लेंगे
\end{hindi}}
\flushright{\begin{Arabic}
\quranayah[68][18]
\end{Arabic}}
\flushleft{\begin{hindi}
और वे इसमें छूट की कोई गुंजाइश नहीं रख रहे थे
\end{hindi}}
\flushright{\begin{Arabic}
\quranayah[68][19]
\end{Arabic}}
\flushleft{\begin{hindi}
अभी वे सो ही रहे थे कि तुम्हारे रब की ओर से गर्दिश का एक झोंका आया
\end{hindi}}
\flushright{\begin{Arabic}
\quranayah[68][20]
\end{Arabic}}
\flushleft{\begin{hindi}
और वह ऐसा हो गया जैसे कटी हुई फ़सल
\end{hindi}}
\flushright{\begin{Arabic}
\quranayah[68][21]
\end{Arabic}}
\flushleft{\begin{hindi}
फिर प्रातःकाल होते ही उन्होंने एक-दूसरे को आवाज़ दी
\end{hindi}}
\flushright{\begin{Arabic}
\quranayah[68][22]
\end{Arabic}}
\flushleft{\begin{hindi}
कि "यदि तुम्हें फल तोड़ना है तो अपनी खेती पर सवेरे ही पहुँचो।"
\end{hindi}}
\flushright{\begin{Arabic}
\quranayah[68][23]
\end{Arabic}}
\flushleft{\begin{hindi}
अतएव वे चुपके-चुपके बातें करते हुए चल पड़े
\end{hindi}}
\flushright{\begin{Arabic}
\quranayah[68][24]
\end{Arabic}}
\flushleft{\begin{hindi}
कि आज वहाँ कोई मुहताज तुम्हारे पास न पहुँचने पाए
\end{hindi}}
\flushright{\begin{Arabic}
\quranayah[68][25]
\end{Arabic}}
\flushleft{\begin{hindi}
और वे आज तेज़ी के साथ चले मानो (मुहताजों को) रोक देने की उन्हें सामर्थ्य प्राप्त है
\end{hindi}}
\flushright{\begin{Arabic}
\quranayah[68][26]
\end{Arabic}}
\flushleft{\begin{hindi}
किन्तु जब उन्होंने उसको देखा, कहने लगे, "निश्चय ही हम भटक गए है।
\end{hindi}}
\flushright{\begin{Arabic}
\quranayah[68][27]
\end{Arabic}}
\flushleft{\begin{hindi}
नहीं, बल्कि हम वंचित होकर रह गए।"
\end{hindi}}
\flushright{\begin{Arabic}
\quranayah[68][28]
\end{Arabic}}
\flushleft{\begin{hindi}
उनमें जो सबसे अच्छा था कहने लगा, "क्या मैंने तुमसे कहा नहीं था? तुम तसबीह क्यों नहीं करते?"
\end{hindi}}
\flushright{\begin{Arabic}
\quranayah[68][29]
\end{Arabic}}
\flushleft{\begin{hindi}
वे पुकार उठे, "महान और उच्च है हमारा रब! निश्चय ही हम ज़ालिम थे।"
\end{hindi}}
\flushright{\begin{Arabic}
\quranayah[68][30]
\end{Arabic}}
\flushleft{\begin{hindi}
फिर वे परस्पर एक-दूसरे की ओर रुख़ करके लगे एक-दूसरे को मलामत करने।
\end{hindi}}
\flushright{\begin{Arabic}
\quranayah[68][31]
\end{Arabic}}
\flushleft{\begin{hindi}
उन्होंने कहा, "अफ़सोस हम पर! निश्चय ही हम सरकश थे।
\end{hindi}}
\flushright{\begin{Arabic}
\quranayah[68][32]
\end{Arabic}}
\flushleft{\begin{hindi}
"आशा है कि हमारा रब बदले में हमें इससे अच्छा प्रदान करे। हम अपने रब की ओर उन्मुख है।"
\end{hindi}}
\flushright{\begin{Arabic}
\quranayah[68][33]
\end{Arabic}}
\flushleft{\begin{hindi}
यातना ऐसी ही होती है, और आख़िरत की यातना तो निश्चय ही इससे भी बड़ी है, काश वे जानते!
\end{hindi}}
\flushright{\begin{Arabic}
\quranayah[68][34]
\end{Arabic}}
\flushleft{\begin{hindi}
निश्चय ही डर रखनेवालों के लिए उनके रब के यहाँ नेमत भरी जन्नतें है
\end{hindi}}
\flushright{\begin{Arabic}
\quranayah[68][35]
\end{Arabic}}
\flushleft{\begin{hindi}
तो क्या हम मुस्लिमों (आज्ञाकारियों) को अपराधियों जैसा कर देंगे?
\end{hindi}}
\flushright{\begin{Arabic}
\quranayah[68][36]
\end{Arabic}}
\flushleft{\begin{hindi}
तुम्हें क्या हो गया है, कैसा फ़ैसला करते हो?
\end{hindi}}
\flushright{\begin{Arabic}
\quranayah[68][37]
\end{Arabic}}
\flushleft{\begin{hindi}
क्या तुम्हारे पास कोई किताब है जिसमें तुम पढ़ते हो
\end{hindi}}
\flushright{\begin{Arabic}
\quranayah[68][38]
\end{Arabic}}
\flushleft{\begin{hindi}
कि उसमें तुम्हारे लिए वह कुछ है जो तुम पसन्द करो?
\end{hindi}}
\flushright{\begin{Arabic}
\quranayah[68][39]
\end{Arabic}}
\flushleft{\begin{hindi}
या तुमने हमसे क़समें ले रखी है जो क़ियामत के दिन तक बाक़ी रहनेवाली है कि तुम्हारे लिए वही कुछ है जो तुम फ़ैसला करो!
\end{hindi}}
\flushright{\begin{Arabic}
\quranayah[68][40]
\end{Arabic}}
\flushleft{\begin{hindi}
उनसे पूछो, "उनमें से कौन इसकी ज़मानत लेता है!
\end{hindi}}
\flushright{\begin{Arabic}
\quranayah[68][41]
\end{Arabic}}
\flushleft{\begin{hindi}
या उनके ठहराए हुए कुछ साझीदार है? फिर तो यह चाहिए कि वे अपने साझीदारों को ले आएँ, यदि वे सच्चे है
\end{hindi}}
\flushright{\begin{Arabic}
\quranayah[68][42]
\end{Arabic}}
\flushleft{\begin{hindi}
जिस दिन पिंडली खुल जाएगी और वे सजदे के लिए बुलाए जाएँगे, तो वे (सजदा) न कर सकेंगे
\end{hindi}}
\flushright{\begin{Arabic}
\quranayah[68][43]
\end{Arabic}}
\flushleft{\begin{hindi}
उनकी निगाहें झुकी हुई होंगी, ज़िल्लत (अपमान) उनपर छा रही होगी। उन्हें उस समय भी सजदा करने के लिए बुलाया जाता था जब वे भले-चंगे थे
\end{hindi}}
\flushright{\begin{Arabic}
\quranayah[68][44]
\end{Arabic}}
\flushleft{\begin{hindi}
अतः तुम मुझे छोड़ दो और उसको जो इस वाणी को झुठलाता है। हम ऐसों को क्रमशः (विनाश की ओर) ले जाएँगे, ऐसे तरीक़े से कि वे नहीं जानते
\end{hindi}}
\flushright{\begin{Arabic}
\quranayah[68][45]
\end{Arabic}}
\flushleft{\begin{hindi}
मैं उन्हें ढील दे रहा हूँ। निश्चय ही मेरी चाल बड़ी मज़बूत है
\end{hindi}}
\flushright{\begin{Arabic}
\quranayah[68][46]
\end{Arabic}}
\flushleft{\begin{hindi}
(क्या वे यातना ही चाहते हैं) या तुम उनसे कोई बदला माँग रहे हो कि वे तावान के बोझ से दबे जाते हों?
\end{hindi}}
\flushright{\begin{Arabic}
\quranayah[68][47]
\end{Arabic}}
\flushleft{\begin{hindi}
या उनके पास परोक्ष का ज्ञान है तो वे लिख रहे हैं?
\end{hindi}}
\flushright{\begin{Arabic}
\quranayah[68][48]
\end{Arabic}}
\flushleft{\begin{hindi}
तो अपने रब के आदेश हेतु धैर्य से काम लो और मछलीवाले (यूनुस अलै॰) की तरह न हो जाना, जबकि उसने पुकारा था इस दशा में कि वह ग़म में घुट रहा था।
\end{hindi}}
\flushright{\begin{Arabic}
\quranayah[68][49]
\end{Arabic}}
\flushleft{\begin{hindi}
यदि उसके रब की अनुकम्पा उसके साथ न हो जाती तो वह अवश्य ही चटियल मैदान में बुरे हाल में डाल दिया जाता।
\end{hindi}}
\flushright{\begin{Arabic}
\quranayah[68][50]
\end{Arabic}}
\flushleft{\begin{hindi}
अन्ततः उसके रब ने उसे चुन लिया और उसे अच्छे लोगों में सम्मिलित कर दिया
\end{hindi}}
\flushright{\begin{Arabic}
\quranayah[68][51]
\end{Arabic}}
\flushleft{\begin{hindi}
जब वे लोग, जिन्होंने इनकार किया, ज़िक्र (क़ुरआन) सुनते है और कहते है, "वह तो दीवाना है!" तो ऐसा लगता है कि वे अपनी निगाहों के ज़ोर से तुम्हें फिसला देंगे
\end{hindi}}
\flushright{\begin{Arabic}
\quranayah[68][52]
\end{Arabic}}
\flushleft{\begin{hindi}
हालाँकि वह सारे संसार के लिए एक अनुस्मृति है
\end{hindi}}
\chapter{Al-Haqqah (The Sure Truth)}
\begin{Arabic}
\Huge{\centerline{\basmalah}}\end{Arabic}
\flushright{\begin{Arabic}
\quranayah[69][1]
\end{Arabic}}
\flushleft{\begin{hindi}
होकर रहनेवाली!
\end{hindi}}
\flushright{\begin{Arabic}
\quranayah[69][2]
\end{Arabic}}
\flushleft{\begin{hindi}
क्या है वह होकर रहनेवाली?
\end{hindi}}
\flushright{\begin{Arabic}
\quranayah[69][3]
\end{Arabic}}
\flushleft{\begin{hindi}
और तुम क्या जानो कि क्या है वह होकर रहनेवाली?
\end{hindi}}
\flushright{\begin{Arabic}
\quranayah[69][4]
\end{Arabic}}
\flushleft{\begin{hindi}
समूद और आद ने उस खड़खड़ा देनेवाली (घटना) को झुठलाया,
\end{hindi}}
\flushright{\begin{Arabic}
\quranayah[69][5]
\end{Arabic}}
\flushleft{\begin{hindi}
फिर समूद तो एक हद से बढ़ जानेवाली आपदा से विनष्ट किए गए
\end{hindi}}
\flushright{\begin{Arabic}
\quranayah[69][6]
\end{Arabic}}
\flushleft{\begin{hindi}
और रहे आद, तो वे एक अनियंत्रित प्रचंड वायु से विनष्ट कर दिए गए
\end{hindi}}
\flushright{\begin{Arabic}
\quranayah[69][7]
\end{Arabic}}
\flushleft{\begin{hindi}
अल्लाह ने उसको सात रात और आठ दिन तक उन्मूलन के उद्देश्य से उनपर लगाए रखा। तो लोगों को तुम देखते कि वे उसमें पछाड़े हुए ऐसे पड़े है मानो वे खजूर के जर्जर तने हों
\end{hindi}}
\flushright{\begin{Arabic}
\quranayah[69][8]
\end{Arabic}}
\flushleft{\begin{hindi}
अब क्या तुम्हें उनमें से कोई शेष दिखाई देता है?
\end{hindi}}
\flushright{\begin{Arabic}
\quranayah[69][9]
\end{Arabic}}
\flushleft{\begin{hindi}
और फ़िरऔन ने और उससे पहले के लोगों ने और तलपट हो जानेवाली बस्तियों ने यह ख़ता की
\end{hindi}}
\flushright{\begin{Arabic}
\quranayah[69][10]
\end{Arabic}}
\flushleft{\begin{hindi}
उन्होंने अपने रब के रसूल की अवज्ञा की तो उसने उन्हें ऐसी पकड़ में ले लिया जो बड़ी कठोर थी
\end{hindi}}
\flushright{\begin{Arabic}
\quranayah[69][11]
\end{Arabic}}
\flushleft{\begin{hindi}
जब पानी उमड़ आया तो हमने तुम्हें प्रवाहित नौका में सवार किया;
\end{hindi}}
\flushright{\begin{Arabic}
\quranayah[69][12]
\end{Arabic}}
\flushleft{\begin{hindi}
ताकि उसे तुम्हारे लिए हम शिक्षाप्रद यादगार बनाएँ और याद रखनेवाले कान उसे सुरक्षित रखें
\end{hindi}}
\flushright{\begin{Arabic}
\quranayah[69][13]
\end{Arabic}}
\flushleft{\begin{hindi}
तो याद रखो जब सूर (नरसिंघा) में एक फूँक मारी जाएगी,
\end{hindi}}
\flushright{\begin{Arabic}
\quranayah[69][14]
\end{Arabic}}
\flushleft{\begin{hindi}
और धरती और पहाड़ों को उठाकर एक ही बार में चूर्ण-विचूर्ण कर दिया जाएगा
\end{hindi}}
\flushright{\begin{Arabic}
\quranayah[69][15]
\end{Arabic}}
\flushleft{\begin{hindi}
तो उस दिन घटित होनेवाली घटना घटित हो जाएगी,
\end{hindi}}
\flushright{\begin{Arabic}
\quranayah[69][16]
\end{Arabic}}
\flushleft{\begin{hindi}
और आकाश फट जाएगा और उस दिन उसका बन्धन ढीला पड़ जाएगा,
\end{hindi}}
\flushright{\begin{Arabic}
\quranayah[69][17]
\end{Arabic}}
\flushleft{\begin{hindi}
और फ़रिश्ते उसके किनारों पर होंगे और उस दिन तुम्हारे रब के सिंहासन को आठ अपने ऊपर उठाए हुए होंगे
\end{hindi}}
\flushright{\begin{Arabic}
\quranayah[69][18]
\end{Arabic}}
\flushleft{\begin{hindi}
उस दिन तुम लोग पेश किए जाओगे, तुम्हारी कोई छिपी बात छिपी न रहेगी
\end{hindi}}
\flushright{\begin{Arabic}
\quranayah[69][19]
\end{Arabic}}
\flushleft{\begin{hindi}
फिर जिस किसी को उसका कर्म-पत्र उसके दाहिने हाथ में दिया गया, तो वह कहेगा, "लो पढ़ो, मेरा कर्म-पत्र!
\end{hindi}}
\flushright{\begin{Arabic}
\quranayah[69][20]
\end{Arabic}}
\flushleft{\begin{hindi}
"मैं तो समझता ही था कि मुझे अपना हिसाब मिलनेवाला है।"
\end{hindi}}
\flushright{\begin{Arabic}
\quranayah[69][21]
\end{Arabic}}
\flushleft{\begin{hindi}
अतः वह सुख और आनन्दमय जीवन में होगा;
\end{hindi}}
\flushright{\begin{Arabic}
\quranayah[69][22]
\end{Arabic}}
\flushleft{\begin{hindi}
उच्च जन्नत में,
\end{hindi}}
\flushright{\begin{Arabic}
\quranayah[69][23]
\end{Arabic}}
\flushleft{\begin{hindi}
जिसके फलों के गुच्छे झुके होंगे
\end{hindi}}
\flushright{\begin{Arabic}
\quranayah[69][24]
\end{Arabic}}
\flushleft{\begin{hindi}
मज़े से खाओ और पियो उन कर्मों के बदले में जो तुमने बीते दिनों में किए है
\end{hindi}}
\flushright{\begin{Arabic}
\quranayah[69][25]
\end{Arabic}}
\flushleft{\begin{hindi}
और रहा वह क्यक्ति जिसका कर्म-पत्र उसके बाएँ हाथ में दिया गया, वह कहेगा, "काश, मेरा कर्म-पत्र मुझे न दिया जाता
\end{hindi}}
\flushright{\begin{Arabic}
\quranayah[69][26]
\end{Arabic}}
\flushleft{\begin{hindi}
और मैं न जानता कि मेरा हिसाब क्या है!
\end{hindi}}
\flushright{\begin{Arabic}
\quranayah[69][27]
\end{Arabic}}
\flushleft{\begin{hindi}
"ऐ काश, वह (मृत्यु) समाप्त करनेवाली होती!
\end{hindi}}
\flushright{\begin{Arabic}
\quranayah[69][28]
\end{Arabic}}
\flushleft{\begin{hindi}
"मेरा माल मेरे कुछ काम न आया,
\end{hindi}}
\flushright{\begin{Arabic}
\quranayah[69][29]
\end{Arabic}}
\flushleft{\begin{hindi}
"मेरा ज़ोर (सत्ता) मुझसे जाता रहा!"
\end{hindi}}
\flushright{\begin{Arabic}
\quranayah[69][30]
\end{Arabic}}
\flushleft{\begin{hindi}
"पकड़ो उसे और उसकी गरदन में तौक़ डाल दो,
\end{hindi}}
\flushright{\begin{Arabic}
\quranayah[69][31]
\end{Arabic}}
\flushleft{\begin{hindi}
"फिर उसे भड़कती हुई आग में झोंक दो,
\end{hindi}}
\flushright{\begin{Arabic}
\quranayah[69][32]
\end{Arabic}}
\flushleft{\begin{hindi}
"फिर उसे एक ऐसी जंजीर में जकड़ दो जिसकी माप सत्तर हाथ है
\end{hindi}}
\flushright{\begin{Arabic}
\quranayah[69][33]
\end{Arabic}}
\flushleft{\begin{hindi}
"वह न तो महिमावान अल्लाह पर ईमान रखता था
\end{hindi}}
\flushright{\begin{Arabic}
\quranayah[69][34]
\end{Arabic}}
\flushleft{\begin{hindi}
और न मुहताज को खाना खिलाने पर उभारता था
\end{hindi}}
\flushright{\begin{Arabic}
\quranayah[69][35]
\end{Arabic}}
\flushleft{\begin{hindi}
"अतः आज उसका यहाँ कोई घनिष्ट मित्र नहीं,
\end{hindi}}
\flushright{\begin{Arabic}
\quranayah[69][36]
\end{Arabic}}
\flushleft{\begin{hindi}
और न ही धोवन के सिवा कोई भोजन है,
\end{hindi}}
\flushright{\begin{Arabic}
\quranayah[69][37]
\end{Arabic}}
\flushleft{\begin{hindi}
"उसे ख़ताकारों (अपराधियों) के अतिरिक्त कोई नहीं खाता।"
\end{hindi}}
\flushright{\begin{Arabic}
\quranayah[69][38]
\end{Arabic}}
\flushleft{\begin{hindi}
अतः कुछ नहीं! मैं क़सम खाता हूँ उन चीज़ों की जो तुम देखते
\end{hindi}}
\flushright{\begin{Arabic}
\quranayah[69][39]
\end{Arabic}}
\flushleft{\begin{hindi}
हो और उन चीज़ों को भी जो तुम नहीं देखते,
\end{hindi}}
\flushright{\begin{Arabic}
\quranayah[69][40]
\end{Arabic}}
\flushleft{\begin{hindi}
निश्चय ही वह एक प्रतिष्ठित रसूल की लाई हुई वाणी है
\end{hindi}}
\flushright{\begin{Arabic}
\quranayah[69][41]
\end{Arabic}}
\flushleft{\begin{hindi}
वह किसी कवि की वाणी नहीं। तुम ईमान थोड़े ही लाते हो
\end{hindi}}
\flushright{\begin{Arabic}
\quranayah[69][42]
\end{Arabic}}
\flushleft{\begin{hindi}
और न वह किसी काहिन का वाणी है। तुम होश से थोड़े ही काम लेते हो
\end{hindi}}
\flushright{\begin{Arabic}
\quranayah[69][43]
\end{Arabic}}
\flushleft{\begin{hindi}
अवतरण है सारे संसार के रब की ओर से,
\end{hindi}}
\flushright{\begin{Arabic}
\quranayah[69][44]
\end{Arabic}}
\flushleft{\begin{hindi}
यदि वह (नबी) हमपर थोपकर कुछ बातें घड़ता,
\end{hindi}}
\flushright{\begin{Arabic}
\quranayah[69][45]
\end{Arabic}}
\flushleft{\begin{hindi}
तो अवश्य हम उसका दाहिना हाथ पकड़ लेते,
\end{hindi}}
\flushright{\begin{Arabic}
\quranayah[69][46]
\end{Arabic}}
\flushleft{\begin{hindi}
फिर उसकी गर्दन की रग काट देते,
\end{hindi}}
\flushright{\begin{Arabic}
\quranayah[69][47]
\end{Arabic}}
\flushleft{\begin{hindi}
और तुममें से कोई भी इससे रोकनेवाला न होता
\end{hindi}}
\flushright{\begin{Arabic}
\quranayah[69][48]
\end{Arabic}}
\flushleft{\begin{hindi}
और निश्चय ही वह एक अनुस्मृति है डर रखनेवालों के लिए
\end{hindi}}
\flushright{\begin{Arabic}
\quranayah[69][49]
\end{Arabic}}
\flushleft{\begin{hindi}
और निश्चय ही हम जानते है कि तुममें कितने ही ऐसे है जो झुठलाते है
\end{hindi}}
\flushright{\begin{Arabic}
\quranayah[69][50]
\end{Arabic}}
\flushleft{\begin{hindi}
निश्चय ही वह इनकार करनेवालों के लिए सर्वथा पछतावा है,
\end{hindi}}
\flushright{\begin{Arabic}
\quranayah[69][51]
\end{Arabic}}
\flushleft{\begin{hindi}
और वह बिल्कुल विश्वसनीय सत्य है।
\end{hindi}}
\flushright{\begin{Arabic}
\quranayah[69][52]
\end{Arabic}}
\flushleft{\begin{hindi}
अतः तुम अपने महिमावान रब के नाम की तसबीह (गुणगान) करो
\end{hindi}}
\chapter{Al-Ma'arij (The Ways of Ascent)}
\begin{Arabic}
\Huge{\centerline{\basmalah}}\end{Arabic}
\flushright{\begin{Arabic}
\quranayah[70][1]
\end{Arabic}}
\flushleft{\begin{hindi}
एक माँगनेवाले ने घटित होनेवाली यातना माँगी,
\end{hindi}}
\flushright{\begin{Arabic}
\quranayah[70][2]
\end{Arabic}}
\flushleft{\begin{hindi}
जो इनकार करनेवालो के लिए होगी, उसे कोई टालनेवाला नहीं,
\end{hindi}}
\flushright{\begin{Arabic}
\quranayah[70][3]
\end{Arabic}}
\flushleft{\begin{hindi}
वह अल्लाह की ओर से होगी, जो चढ़ाव के सोपानों का स्वामी है
\end{hindi}}
\flushright{\begin{Arabic}
\quranayah[70][4]
\end{Arabic}}
\flushleft{\begin{hindi}
फ़रिश्ते और रूह (जिबरील) उसकी ओर चढ़ते है, उस दिन में जिसकी अवधि पचास हज़ार वर्ष है
\end{hindi}}
\flushright{\begin{Arabic}
\quranayah[70][5]
\end{Arabic}}
\flushleft{\begin{hindi}
अतः धैर्य से काम लो, उत्तम धैर्य
\end{hindi}}
\flushright{\begin{Arabic}
\quranayah[70][6]
\end{Arabic}}
\flushleft{\begin{hindi}
वे उसे बहुत दूर देख रहे है,
\end{hindi}}
\flushright{\begin{Arabic}
\quranayah[70][7]
\end{Arabic}}
\flushleft{\begin{hindi}
किन्तु हम उसे निकट देख रहे है
\end{hindi}}
\flushright{\begin{Arabic}
\quranayah[70][8]
\end{Arabic}}
\flushleft{\begin{hindi}
जिस दिन आकाश तेल की तलछट जैसा काला हो जाएगा,
\end{hindi}}
\flushright{\begin{Arabic}
\quranayah[70][9]
\end{Arabic}}
\flushleft{\begin{hindi}
और पर्वत रंग-बिरंगे ऊन के सदृश हो जाएँगे
\end{hindi}}
\flushright{\begin{Arabic}
\quranayah[70][10]
\end{Arabic}}
\flushleft{\begin{hindi}
कोई मित्र किसी मित्र को न पूछेगा,
\end{hindi}}
\flushright{\begin{Arabic}
\quranayah[70][11]
\end{Arabic}}
\flushleft{\begin{hindi}
हालाँकि वे एक-दूसरे को दिखाए जाएँगे। अपराधी चाहेगा कि किसी प्रकार वह उस दिन की यातना से छूटने के लिए अपने बेटों,
\end{hindi}}
\flushright{\begin{Arabic}
\quranayah[70][12]
\end{Arabic}}
\flushleft{\begin{hindi}
अपनी पत्नी , अपने भाई
\end{hindi}}
\flushright{\begin{Arabic}
\quranayah[70][13]
\end{Arabic}}
\flushleft{\begin{hindi}
और अपने उस परिवार को जो उसको आश्रय देता है,
\end{hindi}}
\flushright{\begin{Arabic}
\quranayah[70][14]
\end{Arabic}}
\flushleft{\begin{hindi}
और उन सभी लोगों को जो धरती में रहते है, फ़िदया (मुक्ति-प्रतिदान) के रूप में दे डाले फिर वह उसको छुटकारा दिला दे
\end{hindi}}
\flushright{\begin{Arabic}
\quranayah[70][15]
\end{Arabic}}
\flushleft{\begin{hindi}
कदापि नहीं! वह लपट मारती हुई आग है,
\end{hindi}}
\flushright{\begin{Arabic}
\quranayah[70][16]
\end{Arabic}}
\flushleft{\begin{hindi}
जो मांस और त्वचा को चाट जाएगी,
\end{hindi}}
\flushright{\begin{Arabic}
\quranayah[70][17]
\end{Arabic}}
\flushleft{\begin{hindi}
उस व्यक्ति को बुलाती है जिसने पीठ फेरी और मुँह मोड़ा,
\end{hindi}}
\flushright{\begin{Arabic}
\quranayah[70][18]
\end{Arabic}}
\flushleft{\begin{hindi}
और (धन) एकत्र किया और सैंत कर रखा
\end{hindi}}
\flushright{\begin{Arabic}
\quranayah[70][19]
\end{Arabic}}
\flushleft{\begin{hindi}
निस्संदेह मनुष्य अधीर पैदा हुआ है
\end{hindi}}
\flushright{\begin{Arabic}
\quranayah[70][20]
\end{Arabic}}
\flushleft{\begin{hindi}
जि उसे तकलीफ़ पहुँचती है तो घबरा उठता है,
\end{hindi}}
\flushright{\begin{Arabic}
\quranayah[70][21]
\end{Arabic}}
\flushleft{\begin{hindi}
किन्तु जब उसे सम्पन्नता प्राप्त होती ही तो वह कृपणता दिखाता है
\end{hindi}}
\flushright{\begin{Arabic}
\quranayah[70][22]
\end{Arabic}}
\flushleft{\begin{hindi}
किन्तु नमाज़ अदा करनेवालों की बात और है,
\end{hindi}}
\flushright{\begin{Arabic}
\quranayah[70][23]
\end{Arabic}}
\flushleft{\begin{hindi}
जो अपनी नमाज़ पर सदैव जमें रहते है,
\end{hindi}}
\flushright{\begin{Arabic}
\quranayah[70][24]
\end{Arabic}}
\flushleft{\begin{hindi}
और जिनके मालों में
\end{hindi}}
\flushright{\begin{Arabic}
\quranayah[70][25]
\end{Arabic}}
\flushleft{\begin{hindi}
माँगनेवालों और वंचित का एक ज्ञात और निश्चित हक़ होता है,
\end{hindi}}
\flushright{\begin{Arabic}
\quranayah[70][26]
\end{Arabic}}
\flushleft{\begin{hindi}
जो बदले के दिन को सत्य मानते है,
\end{hindi}}
\flushright{\begin{Arabic}
\quranayah[70][27]
\end{Arabic}}
\flushleft{\begin{hindi}
जो अपने रब की यातना से डरते है -
\end{hindi}}
\flushright{\begin{Arabic}
\quranayah[70][28]
\end{Arabic}}
\flushleft{\begin{hindi}
उनके रब की यातना है ही ऐसी जिससे निश्चिन्त न रहा जाए -
\end{hindi}}
\flushright{\begin{Arabic}
\quranayah[70][29]
\end{Arabic}}
\flushleft{\begin{hindi}
जो अपने गुप्तांगों की रक्षा करते है।
\end{hindi}}
\flushright{\begin{Arabic}
\quranayah[70][30]
\end{Arabic}}
\flushleft{\begin{hindi}
अपनी पत्नि यों या जो उनकी मिल्क में हो उनके अतिरिक्त दूसरों से तो इस बात पर उनकी कोई भर्त्सना नही। -
\end{hindi}}
\flushright{\begin{Arabic}
\quranayah[70][31]
\end{Arabic}}
\flushleft{\begin{hindi}
किन्तु जिस किसी ने इसके अतिरिक्त कुछ और चाहा तो ऐसे ही लोग सीमा का उल्लंघन करनेवाले है।-
\end{hindi}}
\flushright{\begin{Arabic}
\quranayah[70][32]
\end{Arabic}}
\flushleft{\begin{hindi}
जो अपने पास रखी गई अमानतों और अपनी प्रतिज्ञा का निर्वाह करते है,
\end{hindi}}
\flushright{\begin{Arabic}
\quranayah[70][33]
\end{Arabic}}
\flushleft{\begin{hindi}
जो अपनी गवाहियों पर क़़ायम रहते है,
\end{hindi}}
\flushright{\begin{Arabic}
\quranayah[70][34]
\end{Arabic}}
\flushleft{\begin{hindi}
और जो अपनी नमाज़ की रक्षा करते है
\end{hindi}}
\flushright{\begin{Arabic}
\quranayah[70][35]
\end{Arabic}}
\flushleft{\begin{hindi}
वही लोग जन्नतों में सम्मानपूर्वक रहेंगे
\end{hindi}}
\flushright{\begin{Arabic}
\quranayah[70][36]
\end{Arabic}}
\flushleft{\begin{hindi}
फिर उन इनकार करनेवालो को क्या हुआ है कि वे तुम्हारी ओर दौड़े चले आ रहे है?
\end{hindi}}
\flushright{\begin{Arabic}
\quranayah[70][37]
\end{Arabic}}
\flushleft{\begin{hindi}
दाएँ और बाएँ से गिरोह के गिरोह
\end{hindi}}
\flushright{\begin{Arabic}
\quranayah[70][38]
\end{Arabic}}
\flushleft{\begin{hindi}
क्या उनमें से प्रत्येक व्यक्ति इसकी लालसा रखता है कि वह अनुकम्पा से परिपूर्ण जन्नत में प्रविष्ट हो?
\end{hindi}}
\flushright{\begin{Arabic}
\quranayah[70][39]
\end{Arabic}}
\flushleft{\begin{hindi}
कदापि नहीं, हमने उन्हें उस चीज़ से पैदा किया है, जिसे वे भली-भाँति जानते है
\end{hindi}}
\flushright{\begin{Arabic}
\quranayah[70][40]
\end{Arabic}}
\flushleft{\begin{hindi}
अतः कुछ नहीं, मैं क़सम खाता हूँ पूर्वों और पश्चिमों के रब की, हमे इसकी सामर्थ्य प्राप्त है
\end{hindi}}
\flushright{\begin{Arabic}
\quranayah[70][41]
\end{Arabic}}
\flushleft{\begin{hindi}
कि उनकी उनसे अच्छे ले आएँ और हम पिछड़ जानेवाले नहीं है
\end{hindi}}
\flushright{\begin{Arabic}
\quranayah[70][42]
\end{Arabic}}
\flushleft{\begin{hindi}
अतः उन्हें छोड़ो कि वे व्यर्थ बातों में पड़े रहें और खेलते रहे, यहाँ तक कि वे अपने उस दिन से मिलें, जिसका उनसे वादा किया जा रहा है,
\end{hindi}}
\flushright{\begin{Arabic}
\quranayah[70][43]
\end{Arabic}}
\flushleft{\begin{hindi}
जिस दिन वे क़ब्रों से तेज़ी के साथ निकलेंगे जैसे किसी निशान की ओर दौड़े जा रहे है,
\end{hindi}}
\flushright{\begin{Arabic}
\quranayah[70][44]
\end{Arabic}}
\flushleft{\begin{hindi}
उनकी निगाहें झुकी होंगी, ज़िल्लत उनपर छा रही होगी। यह है वह दिन जिससे वह डराए जाते रहे है
\end{hindi}}
\chapter{Nuh (Noah)}
\begin{Arabic}
\Huge{\centerline{\basmalah}}\end{Arabic}
\flushright{\begin{Arabic}
\quranayah[71][1]
\end{Arabic}}
\flushleft{\begin{hindi}
हमने नूह को उसकी कौ़म की ओर भेजा कि "अपनी क़ौम के लोगों को सावधान कर दो, इससे पहले कि उनपर कोई दुखद यातना आ जाए।"
\end{hindi}}
\flushright{\begin{Arabic}
\quranayah[71][2]
\end{Arabic}}
\flushleft{\begin{hindi}
उसने कहा, "ऐ मेरी क़ौम के लोगो! मैं तुम्हारे लिए एक स्पष्ट सचेतकर्ता हूँ
\end{hindi}}
\flushright{\begin{Arabic}
\quranayah[71][3]
\end{Arabic}}
\flushleft{\begin{hindi}
कि अल्लाह की बन्दगी करो और उसका डर रखो और मेरी आज्ञा मानो।-
\end{hindi}}
\flushright{\begin{Arabic}
\quranayah[71][4]
\end{Arabic}}
\flushleft{\begin{hindi}
"वह तुम्हें क्षमा करके तुम्हारे गुनाहों से तुम्हें पाक कर देगा और एक निश्चित समय तक तुम्हे मुहल्लत देगा। निश्चय ही जब अल्लाह का निश्चित समय आ जाता है तो वह टलता नहीं, काश कि तुम जानते!"
\end{hindi}}
\flushright{\begin{Arabic}
\quranayah[71][5]
\end{Arabic}}
\flushleft{\begin{hindi}
उसने कहा, "ऐ मेरे रब! मैंने अपनी क़ौम के लोगों को रात और दिन बुलाया
\end{hindi}}
\flushright{\begin{Arabic}
\quranayah[71][6]
\end{Arabic}}
\flushleft{\begin{hindi}
"किन्तु मेरी पुकार ने उनके पलायन को ही बढ़ाया
\end{hindi}}
\flushright{\begin{Arabic}
\quranayah[71][7]
\end{Arabic}}
\flushleft{\begin{hindi}
"और जब भी मैंने उन्हें बुलाया, ताकि तू उन्हें क्षमा कर दे, तो उन्होंने अपने कानों में अपनी उँगलियाँ दे लीं और अपने कपड़ो से स्वयं को ढाँक लिया और अपनी हठ पर अड़ गए और बड़ा ही घमंड किया
\end{hindi}}
\flushright{\begin{Arabic}
\quranayah[71][8]
\end{Arabic}}
\flushleft{\begin{hindi}
"फिर मैंने उन्हें खुल्लमखुल्ला बुलाया,
\end{hindi}}
\flushright{\begin{Arabic}
\quranayah[71][9]
\end{Arabic}}
\flushleft{\begin{hindi}
"फिर मैंने उनसे खुले तौर पर भी बातें की और उनसे चुपके-चुपके भी बातें की
\end{hindi}}
\flushright{\begin{Arabic}
\quranayah[71][10]
\end{Arabic}}
\flushleft{\begin{hindi}
"और मैंने कहा, अपने रब से क्षमा की प्रार्थना करो। निश्चय ही वह बड़ा क्षमाशील है,
\end{hindi}}
\flushright{\begin{Arabic}
\quranayah[71][11]
\end{Arabic}}
\flushleft{\begin{hindi}
"वह बादल भेजेगा तुमपर ख़ूब बरसनेवाला,
\end{hindi}}
\flushright{\begin{Arabic}
\quranayah[71][12]
\end{Arabic}}
\flushleft{\begin{hindi}
"और वह माल और बेटों से तुम्हें बढ़ोतरी प्रदान करेगा, और तुम्हारे लिए बाग़ पैदा करेगा और तुम्हारे लिए नहरें प्रवाहित करेगा
\end{hindi}}
\flushright{\begin{Arabic}
\quranayah[71][13]
\end{Arabic}}
\flushleft{\begin{hindi}
"तुम्हें क्या हो गया है कि तुम (अपने दिलों में) अल्लाह के लिए किसी गौरव की आशा नहीं रखते?
\end{hindi}}
\flushright{\begin{Arabic}
\quranayah[71][14]
\end{Arabic}}
\flushleft{\begin{hindi}
"हालाँकि उसने तुम्हें विभिन्न अवस्थाओं से गुज़ारते हुए पैदा किया
\end{hindi}}
\flushright{\begin{Arabic}
\quranayah[71][15]
\end{Arabic}}
\flushleft{\begin{hindi}
"क्या तुमने देखा नहीं कि अल्लाह ने किस प्रकार ऊपर तले सात आकाश बनाए,
\end{hindi}}
\flushright{\begin{Arabic}
\quranayah[71][16]
\end{Arabic}}
\flushleft{\begin{hindi}
"और उनमें चन्द्रमा को प्रकाश और सूर्य का प्रदीप बनाया?
\end{hindi}}
\flushright{\begin{Arabic}
\quranayah[71][17]
\end{Arabic}}
\flushleft{\begin{hindi}
"और अल्लाह ने तुम्हें धरती से विशिष्ट प्रकार से विकसित किया,
\end{hindi}}
\flushright{\begin{Arabic}
\quranayah[71][18]
\end{Arabic}}
\flushleft{\begin{hindi}
"फिर वह तुम्हें उसमें लौटाता है और तुम्हें बाहर निकालेगा भी
\end{hindi}}
\flushright{\begin{Arabic}
\quranayah[71][19]
\end{Arabic}}
\flushleft{\begin{hindi}
"और अल्लाह ने तुम्हारे लिए धरती को बिछौना बनाया,
\end{hindi}}
\flushright{\begin{Arabic}
\quranayah[71][20]
\end{Arabic}}
\flushleft{\begin{hindi}
"ताकि तुम उसके विस्तृत मार्गों पर चलो।"
\end{hindi}}
\flushright{\begin{Arabic}
\quranayah[71][21]
\end{Arabic}}
\flushleft{\begin{hindi}
नूह ने कहा, "ऐ मेरे रब! उन्होंने मेरी अवज्ञा की, और उसका अनुसरण किया जिसके धन और जिसकी सन्तान ने उसके घाटे ही मे अभिवृद्धि की
\end{hindi}}
\flushright{\begin{Arabic}
\quranayah[71][22]
\end{Arabic}}
\flushleft{\begin{hindi}
"और वे बहुत बड़ी चाल चले,
\end{hindi}}
\flushright{\begin{Arabic}
\quranayah[71][23]
\end{Arabic}}
\flushleft{\begin{hindi}
"और उन्होंने कहा, अपने इष्ट-पूज्यों के कदापि न छोड़ो और न वह वद्द को छोड़ो और न सुवा को और न यग़ूस और न यऊक़ और नस्र को
\end{hindi}}
\flushright{\begin{Arabic}
\quranayah[71][24]
\end{Arabic}}
\flushleft{\begin{hindi}
"और उन्होंने बहुत-से लोगों को पथभ्रष्ट॥ किया है (तो तू उन्हें मार्ग न दिया) अब, तू भी ज़ालिमों की पथभ्रष्टता ही में अभिवृद्धि कर।"
\end{hindi}}
\flushright{\begin{Arabic}
\quranayah[71][25]
\end{Arabic}}
\flushleft{\begin{hindi}
वे अपनी बड़ी ख़ताओं के कारण पानी में डूबो दिए गए, फिर आग में दाख़िल कर दिए गए, फिर वे अपने और अल्लाह के बीच आड़ बननेवाले सहायक न पा सके
\end{hindi}}
\flushright{\begin{Arabic}
\quranayah[71][26]
\end{Arabic}}
\flushleft{\begin{hindi}
और नूह ने कहा, "ऐ मेरे रब! धरती पर इनकार करनेवालों में से किसी बसनेवाले को न छोड
\end{hindi}}
\flushright{\begin{Arabic}
\quranayah[71][27]
\end{Arabic}}
\flushleft{\begin{hindi}
"यदि तू उन्हें छोड़ देगा तो वे तेरे बन्दों को पथभ्रष्ट कर देंगे और वे दुराचारियों और बड़े अधर्मियों को ही जन्म देंगे
\end{hindi}}
\flushright{\begin{Arabic}
\quranayah[71][28]
\end{Arabic}}
\flushleft{\begin{hindi}
"ऐ मेरे रब! मुझे क्षमा कर दे और मेरे माँ-बाप को भी और हर उस व्यक्ति को भी जो मेरे घर में ईमानवाला बन कर दाख़िल हुआ और (सामान्य) ईमानवाले पुरुषों और ईमानवाली स्त्रियों को भी (क्षमा कर दे), और ज़ालिमों के विनाश को ही बढ़ा।"
\end{hindi}}
\chapter{Al-Jinn (The Jinn)}
\begin{Arabic}
\Huge{\centerline{\basmalah}}\end{Arabic}
\flushright{\begin{Arabic}
\quranayah[72][1]
\end{Arabic}}
\flushleft{\begin{hindi}
कह दो, "मेरी ओर प्रकाशना की गई है कि जिन्नों के एक गिरोह ने सुना, फिर उन्होंने कहा कि हमने एक मनभाता क़ुरआन सुना,
\end{hindi}}
\flushright{\begin{Arabic}
\quranayah[72][2]
\end{Arabic}}
\flushleft{\begin{hindi}
"जो भलाई और सूझ-बूझ का मार्ग दिखाता है, अतः हम उसपर ईमान ले आए, और अब हम कदापि किसी को अपने रब का साझी नहीं ठहराएँगे
\end{hindi}}
\flushright{\begin{Arabic}
\quranayah[72][3]
\end{Arabic}}
\flushleft{\begin{hindi}
"और यह कि हमारे रब का गौरब अत्यन्त उच्च है। उसने अपने लिए न तो कोई पत्नी बनाई और न सन्तान
\end{hindi}}
\flushright{\begin{Arabic}
\quranayah[72][4]
\end{Arabic}}
\flushleft{\begin{hindi}
"और यह कि हममें का मूर्ख व्यक्ति अल्लाह के विषय में सत्य से बिल्कुल हटी हुई बातें कहता रहा है
\end{hindi}}
\flushright{\begin{Arabic}
\quranayah[72][5]
\end{Arabic}}
\flushleft{\begin{hindi}
"और यह कि हमने समझ रखा था कि मनुष्य और जिन्न अल्लाह के विषय में कभी झूठ नहीं बोलते
\end{hindi}}
\flushright{\begin{Arabic}
\quranayah[72][6]
\end{Arabic}}
\flushleft{\begin{hindi}
"और यह कि मनुष्यों में से कितने ही पुरुष ऐसे थे, जो जिन्नों में से कितने ही पुरूषों की शरण माँगा करते थे। इसप्रकार उन्होंने उन्हें (जिन्नों को) और चढ़ा दिया
\end{hindi}}
\flushright{\begin{Arabic}
\quranayah[72][7]
\end{Arabic}}
\flushleft{\begin{hindi}
"और यह कि उन्होंने गुमान किया जैसे कि तुमने गुमान किया कि अल्लाह किसी (नबी) को कदापि न उठाएगा
\end{hindi}}
\flushright{\begin{Arabic}
\quranayah[72][8]
\end{Arabic}}
\flushleft{\begin{hindi}
"और यह कि हमने आकाश को टटोला तो उसे सख़्त पहरेदारों और उल्काओं से भरा हुआ पाया
\end{hindi}}
\flushright{\begin{Arabic}
\quranayah[72][9]
\end{Arabic}}
\flushleft{\begin{hindi}
"और यह कि हम उसमें बैठने के स्थानों में सुनने के लिए बैठा करते थे, किन्तु अब कोई सुनना चाहे तो वह अपने लिए घात में लगा एक उल्का पाएगा
\end{hindi}}
\flushright{\begin{Arabic}
\quranayah[72][10]
\end{Arabic}}
\flushleft{\begin{hindi}
"और यह कि हम नहीं जानते कि उन लोगों के साथ जो धरती में है बुराई का इरादा किया गया है या उनके रब ने उनके लिए भलाई और मार्गदर्शन का इरादा किय है
\end{hindi}}
\flushright{\begin{Arabic}
\quranayah[72][11]
\end{Arabic}}
\flushleft{\begin{hindi}
"और यह कि हममें से कुछ लोग अच्छे है और कुछ लोग उससे निम्नतर है, हम विभिन्न मार्गों पर है
\end{hindi}}
\flushright{\begin{Arabic}
\quranayah[72][12]
\end{Arabic}}
\flushleft{\begin{hindi}
"और यह कि हमने समझ लिया कि हम न धरती में कही जाकर अल्लाह के क़ाबू से निकल सकते है, और न आकाश में कहीं भागकर उसके क़ाबू से निकल सकते है
\end{hindi}}
\flushright{\begin{Arabic}
\quranayah[72][13]
\end{Arabic}}
\flushleft{\begin{hindi}
"और यह कि जब हमने मार्गदर्शन की बात सुनी तो उसपर ईमान ले आए। अब तो कोई अपने रब पर ईमान लाएगा, उसे न तो किसी हक़ के मारे जाने का भय होगा और न किसी ज़ुल्म-ज़्यादती का
\end{hindi}}
\flushright{\begin{Arabic}
\quranayah[72][14]
\end{Arabic}}
\flushleft{\begin{hindi}
"और यह कि हममें से कुछ मुस्लिम (आज्ञाकारी) है और हममें से कुछ हक़ से हटे हुए है। तो जिन्होंने आज्ञापालन का मार्ग ग्रहण कर लिया उन्होंने भलाई और सूझ-बूझ की राह ढूँढ़ ली
\end{hindi}}
\flushright{\begin{Arabic}
\quranayah[72][15]
\end{Arabic}}
\flushleft{\begin{hindi}
"रहे वे लोग जो हक़ से हटे हुए है, तो वे जहन्नम का ईधन होकर रहे।"
\end{hindi}}
\flushright{\begin{Arabic}
\quranayah[72][16]
\end{Arabic}}
\flushleft{\begin{hindi}
और वह प्रकाशना की गई है कि यदि वे सीधे मार्ग पर धैर्यपूर्वक चलते तो हम उन्हें पर्याप्त जल से अभिषिक्त करते,
\end{hindi}}
\flushright{\begin{Arabic}
\quranayah[72][17]
\end{Arabic}}
\flushleft{\begin{hindi}
ताकि हम उसमें उनकी परीक्षा करें। और जो कोई अपने रब की याद से कतराएगा, तो वह उसे कठोर यातना में डाल देगा
\end{hindi}}
\flushright{\begin{Arabic}
\quranayah[72][18]
\end{Arabic}}
\flushleft{\begin{hindi}
और यह कि मस्जिदें अल्लाह के लिए है। अतः अल्लाह के साथ किसी और को न पुकारो
\end{hindi}}
\flushright{\begin{Arabic}
\quranayah[72][19]
\end{Arabic}}
\flushleft{\begin{hindi}
और यह कि "जब अल्लाह का बन्दा उसे पुकारता हुआ खड़ा हुआ तो वे ऐसे लगते है कि उसपर जत्थे बनकर टूट पड़ेगे।"
\end{hindi}}
\flushright{\begin{Arabic}
\quranayah[72][20]
\end{Arabic}}
\flushleft{\begin{hindi}
कह दो, "मैं तो बस अपने रब ही को पुकारता हूँ, और उसके साथ किसी को साझी नहीं ठहराता।"
\end{hindi}}
\flushright{\begin{Arabic}
\quranayah[72][21]
\end{Arabic}}
\flushleft{\begin{hindi}
कह दो, "मैं तो तुम्हारे लिए न किसी हानि का अधिकार रखता हूँ और न किसी भलाई का।"
\end{hindi}}
\flushright{\begin{Arabic}
\quranayah[72][22]
\end{Arabic}}
\flushleft{\begin{hindi}
कहो, "अल्लाह के मुक़ाबले में मुझे कोई पनाह नहीं दे सकता और न मैं उससे बचकर कतराने की कोई जगह पा सकता हूँ। -
\end{hindi}}
\flushright{\begin{Arabic}
\quranayah[72][23]
\end{Arabic}}
\flushleft{\begin{hindi}
"सिवाय अल्लाह की ओर से पहुँचने और उसके संदेश देने के। और जो कोई अल्लाह और उसके रसूल की अवज्ञा करेगा तो उसके लिए जहन्नम की आग है, जिसमें ऐसे लोग सदैव रहेंगे।"
\end{hindi}}
\flushright{\begin{Arabic}
\quranayah[72][24]
\end{Arabic}}
\flushleft{\begin{hindi}
यहाँ तक कि जब वे उस चीज़ को देख लेंगे जिसका उनसे वादा किया जाता है तो वे जान लेंगे कि कौन अपने सहायक की दृष्टि से कमज़ोर और संख्या में न्यूतर है
\end{hindi}}
\flushright{\begin{Arabic}
\quranayah[72][25]
\end{Arabic}}
\flushleft{\begin{hindi}
कह दो, "मैं नहीं जानता कि जिस चीज़ का तुमसे वादा किया जाता है वह निकट है या मेरा रब उसके लिए लम्बी अवधि ठहराता है
\end{hindi}}
\flushright{\begin{Arabic}
\quranayah[72][26]
\end{Arabic}}
\flushleft{\begin{hindi}
"परोक्ष का जाननेवाला वही है और वह अपने परोक्ष को किसी पर प्रकट नहीं करता,
\end{hindi}}
\flushright{\begin{Arabic}
\quranayah[72][27]
\end{Arabic}}
\flushleft{\begin{hindi}
सिवाय उस व्यक्ति के जिसे उसने रसूल की हैसियत से पसन्द कर लिया हो तो उसके आगे से और उसके पीछे से निगरानी की पूर्ण व्यवस्था कर देता है,
\end{hindi}}
\flushright{\begin{Arabic}
\quranayah[72][28]
\end{Arabic}}
\flushleft{\begin{hindi}
ताकि वह यक़ीनी बना दे कि उन्होंने अपने रब के सन्देश पहुँचा दिए और जो कुछ उनके पास है उसे वह घेरे हुए है और हर चीज़ को उसने गिन रखा है।"
\end{hindi}}
\chapter{Al-Muzzammil (The One Covering Himself)}
\begin{Arabic}
\Huge{\centerline{\basmalah}}\end{Arabic}
\flushright{\begin{Arabic}
\quranayah[73][1]
\end{Arabic}}
\flushleft{\begin{hindi}
ऐ कपड़े में लिपटनेवाले!
\end{hindi}}
\flushright{\begin{Arabic}
\quranayah[73][2]
\end{Arabic}}
\flushleft{\begin{hindi}
रात को उठकर (नमाज़ में) खड़े रहा करो - सिवाय थोड़ा हिस्सा -
\end{hindi}}
\flushright{\begin{Arabic}
\quranayah[73][3]
\end{Arabic}}
\flushleft{\begin{hindi}
आधी रात
\end{hindi}}
\flushright{\begin{Arabic}
\quranayah[73][4]
\end{Arabic}}
\flushleft{\begin{hindi}
या उससे कुछ थोड़ा कम कर लो या उससे कुछ अधिक बढ़ा लो और क़ुरआन को भली-भाँति ठहर-ठहरकर पढ़ो। -
\end{hindi}}
\flushright{\begin{Arabic}
\quranayah[73][5]
\end{Arabic}}
\flushleft{\begin{hindi}
निश्चय ही हम तुमपर एक भारी बात डालनेवाले है
\end{hindi}}
\flushright{\begin{Arabic}
\quranayah[73][6]
\end{Arabic}}
\flushleft{\begin{hindi}
निस्संदेह रात का उठना अत्यन्त अनुकूलता रखता है और बात भी उसमें अत्यन्त सधी हुई होती है
\end{hindi}}
\flushright{\begin{Arabic}
\quranayah[73][7]
\end{Arabic}}
\flushleft{\begin{hindi}
निश्चय ही तुम्हार लिए दिन में भी (तसबीह की) बड़ी गुंजाइश है। -
\end{hindi}}
\flushright{\begin{Arabic}
\quranayah[73][8]
\end{Arabic}}
\flushleft{\begin{hindi}
और अपने रब के नाम का ज़िक्र किया करो और सबसे कटकर उसी के हो रहो।
\end{hindi}}
\flushright{\begin{Arabic}
\quranayah[73][9]
\end{Arabic}}
\flushleft{\begin{hindi}
वह पूर्व और पश्चिम का रब है, उसके सिवा कोई इष्ट-पूज्य नहीं, अतः तुम उसी को अपना कार्यसाधक बना लो
\end{hindi}}
\flushright{\begin{Arabic}
\quranayah[73][10]
\end{Arabic}}
\flushleft{\begin{hindi}
और जो कुछ वे कहते है उसपर धैर्य से काम लो और भली रीति से उनसे अलग हो जाओ
\end{hindi}}
\flushright{\begin{Arabic}
\quranayah[73][11]
\end{Arabic}}
\flushleft{\begin{hindi}
और तुम मुझे और झूठलानेवाले सुख-सम्पन्न लोगों को छोड़ दो और उन्हें थोड़ी मुहलत दो
\end{hindi}}
\flushright{\begin{Arabic}
\quranayah[73][12]
\end{Arabic}}
\flushleft{\begin{hindi}
निश्चय ही हमारे पास बेड़ियाँ है और भड़कती हुई आग
\end{hindi}}
\flushright{\begin{Arabic}
\quranayah[73][13]
\end{Arabic}}
\flushleft{\begin{hindi}
और गले में अटकनेवाला भोजन है और दुखद यातना,
\end{hindi}}
\flushright{\begin{Arabic}
\quranayah[73][14]
\end{Arabic}}
\flushleft{\begin{hindi}
जिस दिन धरती और पहाड़ काँप उठेंगे, और पहाड़ रेत के ऐसे ढेर होकर रह जाएगे जो बिखरे जा रहे होंगे
\end{hindi}}
\flushright{\begin{Arabic}
\quranayah[73][15]
\end{Arabic}}
\flushleft{\begin{hindi}
निश्चय ही हुमने तुम्हारी ओर एक रसूल तुमपर गवाह बनाकर भेजा है, जिस प्रकार हमने फ़़िरऔन की ओर एक रसूल भेजा था
\end{hindi}}
\flushright{\begin{Arabic}
\quranayah[73][16]
\end{Arabic}}
\flushleft{\begin{hindi}
किन्तु फ़िरऔन ने रसूल की अवज्ञा कि, तो हमने उसे पकड़ लिया और यह पकड़ सख़्त वबाल थी
\end{hindi}}
\flushright{\begin{Arabic}
\quranayah[73][17]
\end{Arabic}}
\flushleft{\begin{hindi}
यदि तुमने इनकार किया तो उस दिन से कैसे बचोगे जो बच्चों को बूढा कर देगा?
\end{hindi}}
\flushright{\begin{Arabic}
\quranayah[73][18]
\end{Arabic}}
\flushleft{\begin{hindi}
आकाश उसके कारण फटा पड़ रहा है, उसका वादा तो पूरा ही होना है
\end{hindi}}
\flushright{\begin{Arabic}
\quranayah[73][19]
\end{Arabic}}
\flushleft{\begin{hindi}
निश्चय ही यह एक अनुस्मृति है। अब जो चाहे अपने रब की ओर मार्ग ग्रहण कर ले
\end{hindi}}
\flushright{\begin{Arabic}
\quranayah[73][20]
\end{Arabic}}
\flushleft{\begin{hindi}
निस्संदेह तुम्हारा रब जानता है कि तुम लगभग दो तिहाई रात, आधी रात और एक तिहाई रात तक (नमाज़ में) खड़े रहते हो, और एक गिरोंह उन लोगों में से भी जो तुम्हारे साथ है, खड़ा होता है। और अल्लाह रात और दिन की घट-बढ़ नियत करता है। उसे मालूम है कि तुम सब उसका निर्वाह न कर सकोगे, अतः उसने तुमपर दया-दृष्टि की। अब जितना क़ुरआन आसानी से हो सके पढ़ लिया करो। उसे मालूम है कि तुममे से कुछ बीमार भी होंगे, और कुछ दूसरे लोग अल्लाह के उदार अनुग्रह (रोज़ी) को ढूँढ़ते हुए धरती में यात्रा करेंगे, कुछ दूसरे लोग अल्लाह के मार्ग में युद्ध करेंगे। अतः उसमें से जितना आसानी से हो सके पढ़ लिया करो, और नमाज़ क़ायम करो और ज़कात देते रहो, और अल्लाह को ऋण दो, अच्छा ऋण। तुम जो भलाई भी अपने लिए (आगे) भेजोगे उसे अल्लाह के यहाँ अत्युत्तम और प्रतिदान की दृष्टि से बहुत बढ़कर पाओगे। और अल्लाह से माफ़ी माँगते रहो। बेशक अल्लाह अत्यन्त क्षमाशील, दयावान है
\end{hindi}}
\chapter{Al-Muddaththir (The One Wrapping Himself Up)}
\begin{Arabic}
\Huge{\centerline{\basmalah}}\end{Arabic}
\flushright{\begin{Arabic}
\quranayah[74][1]
\end{Arabic}}
\flushleft{\begin{hindi}
ऐ ओढ़ने लपेटनेवाले!
\end{hindi}}
\flushright{\begin{Arabic}
\quranayah[74][2]
\end{Arabic}}
\flushleft{\begin{hindi}
उठो, और सावधान करने में लग जाओ
\end{hindi}}
\flushright{\begin{Arabic}
\quranayah[74][3]
\end{Arabic}}
\flushleft{\begin{hindi}
और अपने रब की बड़ाई ही करो
\end{hindi}}
\flushright{\begin{Arabic}
\quranayah[74][4]
\end{Arabic}}
\flushleft{\begin{hindi}
अपने दामन को पाक रखो
\end{hindi}}
\flushright{\begin{Arabic}
\quranayah[74][5]
\end{Arabic}}
\flushleft{\begin{hindi}
और गन्दगी से दूर ही रहो
\end{hindi}}
\flushright{\begin{Arabic}
\quranayah[74][6]
\end{Arabic}}
\flushleft{\begin{hindi}
अपनी कोशिशों को अधिक समझकर उसके क्रम को भंग न करो
\end{hindi}}
\flushright{\begin{Arabic}
\quranayah[74][7]
\end{Arabic}}
\flushleft{\begin{hindi}
और अपने रब के लिए धैर्य ही से काम लो
\end{hindi}}
\flushright{\begin{Arabic}
\quranayah[74][8]
\end{Arabic}}
\flushleft{\begin{hindi}
जब सूर में फूँक मारी जाएगी
\end{hindi}}
\flushright{\begin{Arabic}
\quranayah[74][9]
\end{Arabic}}
\flushleft{\begin{hindi}
तो जिस दिन ऐसा होगा, वह दिन बड़ा ही कठोर होगा,
\end{hindi}}
\flushright{\begin{Arabic}
\quranayah[74][10]
\end{Arabic}}
\flushleft{\begin{hindi}
इनकार करनेवालो पर आसान न होगा
\end{hindi}}
\flushright{\begin{Arabic}
\quranayah[74][11]
\end{Arabic}}
\flushleft{\begin{hindi}
छोड़ दो मुझे और उसको जिसे मैंने अकेला पैदा किया,
\end{hindi}}
\flushright{\begin{Arabic}
\quranayah[74][12]
\end{Arabic}}
\flushleft{\begin{hindi}
और उसे माल दिया दूर तक फैला हुआ,
\end{hindi}}
\flushright{\begin{Arabic}
\quranayah[74][13]
\end{Arabic}}
\flushleft{\begin{hindi}
और उसके पास उपस्थित रहनेवाले बेटे दिए,
\end{hindi}}
\flushright{\begin{Arabic}
\quranayah[74][14]
\end{Arabic}}
\flushleft{\begin{hindi}
और मैंने उसके लिए अच्छी तरह जीवन-मार्ग समतल किया
\end{hindi}}
\flushright{\begin{Arabic}
\quranayah[74][15]
\end{Arabic}}
\flushleft{\begin{hindi}
फिर वह लोभ रखता है कि मैं उसके लिए और अधिक दूँगा
\end{hindi}}
\flushright{\begin{Arabic}
\quranayah[74][16]
\end{Arabic}}
\flushleft{\begin{hindi}
कदापि नहीं, वह हमारी आयतों का दुश्मन है,
\end{hindi}}
\flushright{\begin{Arabic}
\quranayah[74][17]
\end{Arabic}}
\flushleft{\begin{hindi}
शीघ्र ही मैं उसे घेरकर कठिन चढ़ाई चढ़वाऊँगा
\end{hindi}}
\flushright{\begin{Arabic}
\quranayah[74][18]
\end{Arabic}}
\flushleft{\begin{hindi}
उसने सोचा और अटकल से एक बात बनाई
\end{hindi}}
\flushright{\begin{Arabic}
\quranayah[74][19]
\end{Arabic}}
\flushleft{\begin{hindi}
तो विनष्ट हो, कैसी बात बनाई!
\end{hindi}}
\flushright{\begin{Arabic}
\quranayah[74][20]
\end{Arabic}}
\flushleft{\begin{hindi}
फिर विनष्ट हो, कैसी बात बनाई!
\end{hindi}}
\flushright{\begin{Arabic}
\quranayah[74][21]
\end{Arabic}}
\flushleft{\begin{hindi}
फिर नज़र दौड़ाई,
\end{hindi}}
\flushright{\begin{Arabic}
\quranayah[74][22]
\end{Arabic}}
\flushleft{\begin{hindi}
फिर त्योरी चढ़ाई और मुँह बनाया,
\end{hindi}}
\flushright{\begin{Arabic}
\quranayah[74][23]
\end{Arabic}}
\flushleft{\begin{hindi}
फिर पीठ फेरी और घमंड किया
\end{hindi}}
\flushright{\begin{Arabic}
\quranayah[74][24]
\end{Arabic}}
\flushleft{\begin{hindi}
अन्ततः बोला, "यह तो बस एक जादू है, जो पहले से चला आ रहा है
\end{hindi}}
\flushright{\begin{Arabic}
\quranayah[74][25]
\end{Arabic}}
\flushleft{\begin{hindi}
"यह तो मात्र मनुष्य की वाणी है।"
\end{hindi}}
\flushright{\begin{Arabic}
\quranayah[74][26]
\end{Arabic}}
\flushleft{\begin{hindi}
मैं शीघ्र ही उसे 'सक़र' (जहन्नम की आग) में झोंक दूँगा
\end{hindi}}
\flushright{\begin{Arabic}
\quranayah[74][27]
\end{Arabic}}
\flushleft{\begin{hindi}
और तुम्हें क्या पता की सक़र क्या है?
\end{hindi}}
\flushright{\begin{Arabic}
\quranayah[74][28]
\end{Arabic}}
\flushleft{\begin{hindi}
वह न तरस खाएगी और न छोड़ेगी,
\end{hindi}}
\flushright{\begin{Arabic}
\quranayah[74][29]
\end{Arabic}}
\flushleft{\begin{hindi}
खाल को झुलसा देनेवाली है,
\end{hindi}}
\flushright{\begin{Arabic}
\quranayah[74][30]
\end{Arabic}}
\flushleft{\begin{hindi}
उसपर उन्नीस (कार्यकर्ता) नियुक्त है
\end{hindi}}
\flushright{\begin{Arabic}
\quranayah[74][31]
\end{Arabic}}
\flushleft{\begin{hindi}
और हमने उस आग पर नियुक्त रहनेवालों को फ़रिश्ते ही बनाया है, और हमने उनकी संख्या को इनकार करनेवालों के लिए मुसीबत और आज़माइश ही बनाकर रखा है। ताकि वे लोग जिन्हें किताब प्रदान की गई थी पूर्ण विश्वास प्राप्त करें, और वे लोग जो ईमान ले आए वे ईमान में और आगे बढ़ जाएँ। और जिन लोगों को किताब प्रदान की गई वे और ईमानवाले किसी संशय मे न पड़े, और ताकि जिनके दिलों मे रोग है वे और इनकार करनेवाले कहें, "इस वर्णन से अल्लाह का क्या अभिप्राय है?" इस प्रकार अल्लाह जिसे चाहता है पथभ्रष्ट कर देता है और जिसे चाहता हैं संमार्ग प्रदान करता है। और तुम्हारे रब की सेनाओं को स्वयं उसके सिवा कोई नहीं जानता, और यह तो मनुष्य के लिए मात्र एक शिक्षा-सामग्री है
\end{hindi}}
\flushright{\begin{Arabic}
\quranayah[74][32]
\end{Arabic}}
\flushleft{\begin{hindi}
कुछ नहीं, साक्षी है चाँद
\end{hindi}}
\flushright{\begin{Arabic}
\quranayah[74][33]
\end{Arabic}}
\flushleft{\begin{hindi}
और साक्षी है रात जबकि वह पीठ फेर चुकी,
\end{hindi}}
\flushright{\begin{Arabic}
\quranayah[74][34]
\end{Arabic}}
\flushleft{\begin{hindi}
और प्रातःकाल जबकि वह पूर्णरूपेण प्रकाशित हो जाए।
\end{hindi}}
\flushright{\begin{Arabic}
\quranayah[74][35]
\end{Arabic}}
\flushleft{\begin{hindi}
निश्चय ही वह भारी (भयंकर) चीज़ों में से एक है,
\end{hindi}}
\flushright{\begin{Arabic}
\quranayah[74][36]
\end{Arabic}}
\flushleft{\begin{hindi}
मनुष्यों के लिए सावधानकर्ता के रूप में,
\end{hindi}}
\flushright{\begin{Arabic}
\quranayah[74][37]
\end{Arabic}}
\flushleft{\begin{hindi}
तुममें से उस व्यक्ति के लिए जो आगे बढ़ना या पीछे हटना चाहे
\end{hindi}}
\flushright{\begin{Arabic}
\quranayah[74][38]
\end{Arabic}}
\flushleft{\begin{hindi}
प्रत्येक व्यक्ति जो कुछ उसने कमाया उसके बदले रेहन (गिरवी) है,
\end{hindi}}
\flushright{\begin{Arabic}
\quranayah[74][39]
\end{Arabic}}
\flushleft{\begin{hindi}
सिवाय दाएँवालों के
\end{hindi}}
\flushright{\begin{Arabic}
\quranayah[74][40]
\end{Arabic}}
\flushleft{\begin{hindi}
वे बाग़ों में होंगे, पूछ-ताछ कर रहे होंगे
\end{hindi}}
\flushright{\begin{Arabic}
\quranayah[74][41]
\end{Arabic}}
\flushleft{\begin{hindi}
अपराधियों के विषय में
\end{hindi}}
\flushright{\begin{Arabic}
\quranayah[74][42]
\end{Arabic}}
\flushleft{\begin{hindi}
"तुम्हे क्या चीज़ सकंर (जहन्नम) में ले आई?"
\end{hindi}}
\flushright{\begin{Arabic}
\quranayah[74][43]
\end{Arabic}}
\flushleft{\begin{hindi}
वे कहेंगे, "हम नमाज़ अदा करनेवालों में से न थे।
\end{hindi}}
\flushright{\begin{Arabic}
\quranayah[74][44]
\end{Arabic}}
\flushleft{\begin{hindi}
और न हम मुहताज को खाना खिलाते थे
\end{hindi}}
\flushright{\begin{Arabic}
\quranayah[74][45]
\end{Arabic}}
\flushleft{\begin{hindi}
"और व्यर्थ बात और कठ-हुज्जती में पड़े रहनेवालों के साथ हम भी उसी में लगे रहते थे।
\end{hindi}}
\flushright{\begin{Arabic}
\quranayah[74][46]
\end{Arabic}}
\flushleft{\begin{hindi}
और हम बदला दिए जाने के दिन को झुठलाते थे,
\end{hindi}}
\flushright{\begin{Arabic}
\quranayah[74][47]
\end{Arabic}}
\flushleft{\begin{hindi}
"यहाँ तक कि विश्वसनीय चीज़ (प्रलय-दिवस) में हमें आ लिया।"
\end{hindi}}
\flushright{\begin{Arabic}
\quranayah[74][48]
\end{Arabic}}
\flushleft{\begin{hindi}
अतः सिफ़ारिश करनेवालों को कोई सिफ़ारिश उनको कुछ लाभ न पहुँचा सकेगी
\end{hindi}}
\flushright{\begin{Arabic}
\quranayah[74][49]
\end{Arabic}}
\flushleft{\begin{hindi}
आख़िर उन्हें क्या हुआ है कि वे नसीहत से कतराते है,
\end{hindi}}
\flushright{\begin{Arabic}
\quranayah[74][50]
\end{Arabic}}
\flushleft{\begin{hindi}
मानो वे बिदके हुए जंगली गधे है
\end{hindi}}
\flushright{\begin{Arabic}
\quranayah[74][51]
\end{Arabic}}
\flushleft{\begin{hindi}
जो शेर से (डरकर) भागे है?
\end{hindi}}
\flushright{\begin{Arabic}
\quranayah[74][52]
\end{Arabic}}
\flushleft{\begin{hindi}
नहीं, बल्कि उनमें से प्रत्येक व्यक्ति चाहता है कि उसे खुली किताबें दी जाएँ
\end{hindi}}
\flushright{\begin{Arabic}
\quranayah[74][53]
\end{Arabic}}
\flushleft{\begin{hindi}
कदापि नहीं, बल्कि ले आख़िरत से डरते नहीं
\end{hindi}}
\flushright{\begin{Arabic}
\quranayah[74][54]
\end{Arabic}}
\flushleft{\begin{hindi}
कुछ नहीं, वह तो एक अनुस्मति है
\end{hindi}}
\flushright{\begin{Arabic}
\quranayah[74][55]
\end{Arabic}}
\flushleft{\begin{hindi}
अब जो कोई चाहे इससे नसीहत हासिल करे,
\end{hindi}}
\flushright{\begin{Arabic}
\quranayah[74][56]
\end{Arabic}}
\flushleft{\begin{hindi}
और वे नसीहत हासिल नहीं करेंगे। यह और बात है कि अल्लाह ही ऐसा चाहे। वही इस योग्य है कि उसका डर रखा जाए और इस योग्य भी कि क्षमा करे
\end{hindi}}
\chapter{Al-Qiyamah (The Resurrection)}
\begin{Arabic}
\Huge{\centerline{\basmalah}}\end{Arabic}
\flushright{\begin{Arabic}
\quranayah[75][1]
\end{Arabic}}
\flushleft{\begin{hindi}
नहीं, मैं क़सम खाता हूँ क़ियामत के दिन की,
\end{hindi}}
\flushright{\begin{Arabic}
\quranayah[75][2]
\end{Arabic}}
\flushleft{\begin{hindi}
और नहीं! मैं कसम खाता हूँ मलामत करनेवाली आत्मा की
\end{hindi}}
\flushright{\begin{Arabic}
\quranayah[75][3]
\end{Arabic}}
\flushleft{\begin{hindi}
क्या मनुष्य यह समझता है कि हम कदापि उसकी हड्डियों को एकत्र न करेंगे?
\end{hindi}}
\flushright{\begin{Arabic}
\quranayah[75][4]
\end{Arabic}}
\flushleft{\begin{hindi}
क्यों नहीं, हम उसकी पोरों को ठीक-ठाक करने की सामर्थ्य रखते है
\end{hindi}}
\flushright{\begin{Arabic}
\quranayah[75][5]
\end{Arabic}}
\flushleft{\begin{hindi}
बल्कि मनुष्य चाहता है कि अपने आगे ढिठाई करता रहे
\end{hindi}}
\flushright{\begin{Arabic}
\quranayah[75][6]
\end{Arabic}}
\flushleft{\begin{hindi}
पूछता है, "आख़िर क़ियामत का दिन कब आएगा?"
\end{hindi}}
\flushright{\begin{Arabic}
\quranayah[75][7]
\end{Arabic}}
\flushleft{\begin{hindi}
तो जब निगाह चौंधिया जाएगी,
\end{hindi}}
\flushright{\begin{Arabic}
\quranayah[75][8]
\end{Arabic}}
\flushleft{\begin{hindi}
और चन्द्रमा को ग्रहण लग जाएगा,
\end{hindi}}
\flushright{\begin{Arabic}
\quranayah[75][9]
\end{Arabic}}
\flushleft{\begin{hindi}
और सूर्य और चन्द्रमा इकट्ठे कर दिए जाएँगे,
\end{hindi}}
\flushright{\begin{Arabic}
\quranayah[75][10]
\end{Arabic}}
\flushleft{\begin{hindi}
उस दिन मनुष्य कहेगा, "कहाँ जाऊँ भागकर?"
\end{hindi}}
\flushright{\begin{Arabic}
\quranayah[75][11]
\end{Arabic}}
\flushleft{\begin{hindi}
कुछ नहीं, कोई शरण-स्थल नहीं!
\end{hindi}}
\flushright{\begin{Arabic}
\quranayah[75][12]
\end{Arabic}}
\flushleft{\begin{hindi}
उस दिन तुम्हारे रब ही ओर जाकर ठहरना है
\end{hindi}}
\flushright{\begin{Arabic}
\quranayah[75][13]
\end{Arabic}}
\flushleft{\begin{hindi}
उस दिन मनुष्य को बता दिया जाएगा जो कुछ उसने आगे बढाया और पीछे टाला
\end{hindi}}
\flushright{\begin{Arabic}
\quranayah[75][14]
\end{Arabic}}
\flushleft{\begin{hindi}
नहीं, बल्कि मनुष्य स्वयं अपने हाल पर निगाह रखता है,
\end{hindi}}
\flushright{\begin{Arabic}
\quranayah[75][15]
\end{Arabic}}
\flushleft{\begin{hindi}
यद्यपि उसने अपने कितने ही बहाने पेश किए हो
\end{hindi}}
\flushright{\begin{Arabic}
\quranayah[75][16]
\end{Arabic}}
\flushleft{\begin{hindi}
तू उसे शीघ्र पाने के लिए उसके प्रति अपनी ज़बान को न चला
\end{hindi}}
\flushright{\begin{Arabic}
\quranayah[75][17]
\end{Arabic}}
\flushleft{\begin{hindi}
हमारे ज़िम्मे है उसे एकत्र करना और उसका पढ़ना,
\end{hindi}}
\flushright{\begin{Arabic}
\quranayah[75][18]
\end{Arabic}}
\flushleft{\begin{hindi}
अतः जब हम उसे पढ़े तो उसके पठन का अनुसरण कर,
\end{hindi}}
\flushright{\begin{Arabic}
\quranayah[75][19]
\end{Arabic}}
\flushleft{\begin{hindi}
फिर हमारे ज़िम्मे है उसका स्पष्टीकरण करना
\end{hindi}}
\flushright{\begin{Arabic}
\quranayah[75][20]
\end{Arabic}}
\flushleft{\begin{hindi}
कुछ नहीं, बल्कि तुम लोग शीघ्र मिलनेवाली चीज़ (दुनिया) से प्रेम रखते हो,
\end{hindi}}
\flushright{\begin{Arabic}
\quranayah[75][21]
\end{Arabic}}
\flushleft{\begin{hindi}
और आख़िरत को छोड़ रहे हो
\end{hindi}}
\flushright{\begin{Arabic}
\quranayah[75][22]
\end{Arabic}}
\flushleft{\begin{hindi}
किनते ही चहरे उस दिन तरो ताज़ा और प्रफुल्लित होंगे,
\end{hindi}}
\flushright{\begin{Arabic}
\quranayah[75][23]
\end{Arabic}}
\flushleft{\begin{hindi}
अपने रब की ओर देख रहे होंगे।
\end{hindi}}
\flushright{\begin{Arabic}
\quranayah[75][24]
\end{Arabic}}
\flushleft{\begin{hindi}
और कितने ही चेहरे उस दिन उदास और बिगड़े हुए होंगे,
\end{hindi}}
\flushright{\begin{Arabic}
\quranayah[75][25]
\end{Arabic}}
\flushleft{\begin{hindi}
समझ रहे होंगे कि उनके साथ कमर तोड़ देनेवाला मामला किया जाएगा
\end{hindi}}
\flushright{\begin{Arabic}
\quranayah[75][26]
\end{Arabic}}
\flushleft{\begin{hindi}
कुछ नहीं, जब प्राण कंठ को आ लगेंगे,
\end{hindi}}
\flushright{\begin{Arabic}
\quranayah[75][27]
\end{Arabic}}
\flushleft{\begin{hindi}
और कहा जाएगा, "कौन है झाड़-फूँक करनेवाला?"
\end{hindi}}
\flushright{\begin{Arabic}
\quranayah[75][28]
\end{Arabic}}
\flushleft{\begin{hindi}
और वह समझ लेगा कि वह जुदाई (का समय) है
\end{hindi}}
\flushright{\begin{Arabic}
\quranayah[75][29]
\end{Arabic}}
\flushleft{\begin{hindi}
और पिंडली से पिंडली लिपट जाएगी,
\end{hindi}}
\flushright{\begin{Arabic}
\quranayah[75][30]
\end{Arabic}}
\flushleft{\begin{hindi}
तुम्हारे रब की ओर उस दिन प्रस्थान होगा
\end{hindi}}
\flushright{\begin{Arabic}
\quranayah[75][31]
\end{Arabic}}
\flushleft{\begin{hindi}
किन्तु उसने न तो सत्य माना और न नमाज़ अदा की,
\end{hindi}}
\flushright{\begin{Arabic}
\quranayah[75][32]
\end{Arabic}}
\flushleft{\begin{hindi}
लेकिन झुठलाया और मुँह मोड़ा,
\end{hindi}}
\flushright{\begin{Arabic}
\quranayah[75][33]
\end{Arabic}}
\flushleft{\begin{hindi}
फिर अकड़ता हुआ अपने लोगों की ओर चल दिया
\end{hindi}}
\flushright{\begin{Arabic}
\quranayah[75][34]
\end{Arabic}}
\flushleft{\begin{hindi}
अफ़सोस है तुझपर और अफ़सोस है!
\end{hindi}}
\flushright{\begin{Arabic}
\quranayah[75][35]
\end{Arabic}}
\flushleft{\begin{hindi}
फिर अफ़सोस है तुझपर और अफ़सोस है!
\end{hindi}}
\flushright{\begin{Arabic}
\quranayah[75][36]
\end{Arabic}}
\flushleft{\begin{hindi}
क्या मनुष्य समझता है कि वह यूँ ही स्वतंत्र छोड़ दिया जाएगा?
\end{hindi}}
\flushright{\begin{Arabic}
\quranayah[75][37]
\end{Arabic}}
\flushleft{\begin{hindi}
क्या वह केवल टपकाए हुए वीर्य की एक बूँद न था?
\end{hindi}}
\flushright{\begin{Arabic}
\quranayah[75][38]
\end{Arabic}}
\flushleft{\begin{hindi}
फिर वह रक्त की एक फुटकी हुआ, फिर अल्लाह ने उसे रूप दिया और उसके अंग-प्रत्यंग ठीक-ठाक किए
\end{hindi}}
\flushright{\begin{Arabic}
\quranayah[75][39]
\end{Arabic}}
\flushleft{\begin{hindi}
और उसकी दो जातियाँ बनाई - पुरुष और स्त्री
\end{hindi}}
\flushright{\begin{Arabic}
\quranayah[75][40]
\end{Arabic}}
\flushleft{\begin{hindi}
क्या उसे वह सामर्थ्य प्राप्त- नहीं कि वह मुर्दों को जीवित कर दे?
\end{hindi}}
\chapter{Al-Insan (The Man)}
\begin{Arabic}
\Huge{\centerline{\basmalah}}\end{Arabic}
\flushright{\begin{Arabic}
\quranayah[76][1]
\end{Arabic}}
\flushleft{\begin{hindi}
क्या मनुष्य पर काल-खंड का ऐसा समय भी बीता है कि वह कोई ऐसी चीज़ न था जिसका उल्लेख किया जाता?
\end{hindi}}
\flushright{\begin{Arabic}
\quranayah[76][2]
\end{Arabic}}
\flushleft{\begin{hindi}
हमने मनुष्य को एक मिश्रित वीर्य से पैदा किया, उसे उलटते-पलटते रहे, फिर हमने उसे सुनने और देखनेवाला बना दिया
\end{hindi}}
\flushright{\begin{Arabic}
\quranayah[76][3]
\end{Arabic}}
\flushleft{\begin{hindi}
हमने उसे मार्ग दिखाया, अब चाहे वह कृतज्ञ बने या अकृतज्ञ
\end{hindi}}
\flushright{\begin{Arabic}
\quranayah[76][4]
\end{Arabic}}
\flushleft{\begin{hindi}
हमने इनकार करनेवालों के लिए ज़जीरें और तौक़ और भड़कती हुई आग तैयार कर रखी है
\end{hindi}}
\flushright{\begin{Arabic}
\quranayah[76][5]
\end{Arabic}}
\flushleft{\begin{hindi}
निश्चय ही वफ़ादार लोग ऐसे जाम से पिएँगे जिसमें काफ़ूर का मिश्रण होगा,
\end{hindi}}
\flushright{\begin{Arabic}
\quranayah[76][6]
\end{Arabic}}
\flushleft{\begin{hindi}
उस स्रोत का क्या कहना! जिस पर बैठकर अल्लाह के बन्दे पिएँगे, इस तरह कि उसे बहा-बहाकर (जहाँ चाहेंगे) ले जाएँगे
\end{hindi}}
\flushright{\begin{Arabic}
\quranayah[76][7]
\end{Arabic}}
\flushleft{\begin{hindi}
वे नज़र (मन्नत) पूरी करते है और उस दिन से डरते है जिसकी आपदा व्यापक होगी,
\end{hindi}}
\flushright{\begin{Arabic}
\quranayah[76][8]
\end{Arabic}}
\flushleft{\begin{hindi}
और वे मुहताज, अनाथ और क़ैदी को खाना उसकी चाहत रखते हुए खिलाते है,
\end{hindi}}
\flushright{\begin{Arabic}
\quranayah[76][9]
\end{Arabic}}
\flushleft{\begin{hindi}
"हम तो केवल अल्लाह की प्रसन्नता के लिए तुम्हें खिलाते है, तुमसे न कोई बदला चाहते है और न कृतज्ञता ज्ञापन
\end{hindi}}
\flushright{\begin{Arabic}
\quranayah[76][10]
\end{Arabic}}
\flushleft{\begin{hindi}
"हमें तो अपने रब की ओर से एक ऐसे दिन का भय है जो त्योरी पर बल डाले हुए अत्यन्त क्रूर होगा।"
\end{hindi}}
\flushright{\begin{Arabic}
\quranayah[76][11]
\end{Arabic}}
\flushleft{\begin{hindi}
अतः अल्लाह ने उस दिन की बुराई से बचा लिया और उन्हें ताज़गी और ख़ुशी प्रदान की,
\end{hindi}}
\flushright{\begin{Arabic}
\quranayah[76][12]
\end{Arabic}}
\flushleft{\begin{hindi}
और जो उन्होंने धैर्य से काम लिया, उसके बदले में उन्हें जन्नत और रेशमी वस्त्र प्रदान किया
\end{hindi}}
\flushright{\begin{Arabic}
\quranayah[76][13]
\end{Arabic}}
\flushleft{\begin{hindi}
उसमें वे तख़्तों पर टेक लगाए होंगे, वे उसमें न तो सख़्त धूप देखेंगे औ न सख़्त ठंड़
\end{hindi}}
\flushright{\begin{Arabic}
\quranayah[76][14]
\end{Arabic}}
\flushleft{\begin{hindi}
और उस (बाग़) के साए उनपर झुके होंगे और उसके फलों के गुच्छे बिलकुल उनके वश में होंगे
\end{hindi}}
\flushright{\begin{Arabic}
\quranayah[76][15]
\end{Arabic}}
\flushleft{\begin{hindi}
और उनके पास चाँदी के बरतन ग़र्दिश में होंगे और प्याले
\end{hindi}}
\flushright{\begin{Arabic}
\quranayah[76][16]
\end{Arabic}}
\flushleft{\begin{hindi}
जो बिल्कुल शीशे हो रहे होंगे, शीशे भी चाँदी के जो ठीक अन्दाज़े करके रखे गए होंगे
\end{hindi}}
\flushright{\begin{Arabic}
\quranayah[76][17]
\end{Arabic}}
\flushleft{\begin{hindi}
और वहाँ वे एक और जाम़ पिएँगे जिसमें सोंठ का मिश्रण होगा
\end{hindi}}
\flushright{\begin{Arabic}
\quranayah[76][18]
\end{Arabic}}
\flushleft{\begin{hindi}
क्या कहना उस स्रोत का जो उसमें होगा, जिसका नाम सल-सबील है
\end{hindi}}
\flushright{\begin{Arabic}
\quranayah[76][19]
\end{Arabic}}
\flushleft{\begin{hindi}
उनकी सेवा में ऐसे किशोर दौड़ते रहे होंगे जो सदैव किशोर ही रहेंगे। जब तुम उन्हें देखोगे तो समझोगे कि बिखरे हुए मोती है
\end{hindi}}
\flushright{\begin{Arabic}
\quranayah[76][20]
\end{Arabic}}
\flushleft{\begin{hindi}
जब तुम वहाँ देखोगे तो तुम्हें बड़ी नेमत और विशाल राज्य दिखाई देगा
\end{hindi}}
\flushright{\begin{Arabic}
\quranayah[76][21]
\end{Arabic}}
\flushleft{\begin{hindi}
उनके ऊपर हरे बारीक हरे बारीक रेशमी वस्त्र और गाढ़े रेशमी कपड़े होंगे, और उन्हें चाँदी के कंगन पहनाए जाएँगे और उनका रब उन्हें पवित्र पेय पिलाएगा
\end{hindi}}
\flushright{\begin{Arabic}
\quranayah[76][22]
\end{Arabic}}
\flushleft{\begin{hindi}
"यह है तुम्हारा बदला और तुम्हारा प्रयास क़द्र करने के योग्य है।"
\end{hindi}}
\flushright{\begin{Arabic}
\quranayah[76][23]
\end{Arabic}}
\flushleft{\begin{hindi}
निश्चय ही हमने अत्यन्त व्यवस्थित ढंग से तुमपर क़ुरआन अवतरित किया है;
\end{hindi}}
\flushright{\begin{Arabic}
\quranayah[76][24]
\end{Arabic}}
\flushleft{\begin{hindi}
अतः अपने रब के हुक्म और फ़ैसले के लिए धैर्य से काम लो और उनमें से किसी पापी या कृतघ्न का आज्ञापालन न करना
\end{hindi}}
\flushright{\begin{Arabic}
\quranayah[76][25]
\end{Arabic}}
\flushleft{\begin{hindi}
और प्रातःकाल और संध्या समय अपने रब के नाम का स्मरण करो
\end{hindi}}
\flushright{\begin{Arabic}
\quranayah[76][26]
\end{Arabic}}
\flushleft{\begin{hindi}
और रात के कुछ हिस्से में भी उसे सजदा करो, लम्बी-लम्बी रात तक उसकी तसबीह करते रहो
\end{hindi}}
\flushright{\begin{Arabic}
\quranayah[76][27]
\end{Arabic}}
\flushleft{\begin{hindi}
निस्संदेह ये लोग शीघ्र प्राप्त होनेवाली चीज़ (संसार) से प्रेम रखते है और एक भारी दिन को अपने परे छोड़ रह है
\end{hindi}}
\flushright{\begin{Arabic}
\quranayah[76][28]
\end{Arabic}}
\flushleft{\begin{hindi}
हमने उन्हें पैदा किया और उनके जोड़-बन्द मज़बूत किेए और हम जब चाहे उन जैसों को पूर्णतः बदल दें
\end{hindi}}
\flushright{\begin{Arabic}
\quranayah[76][29]
\end{Arabic}}
\flushleft{\begin{hindi}
निश्चय ही यह एक अनुस्मृति है, अब जो चाहे अपने रब की ओर मार्ग ग्रहण कर ले
\end{hindi}}
\flushright{\begin{Arabic}
\quranayah[76][30]
\end{Arabic}}
\flushleft{\begin{hindi}
और तुम नहीं चाह सकते सिवाय इसके कि अल्लाह चाहे। निस्संदेह अल्लाह सर्वज्ञ, तत्वदर्शी है
\end{hindi}}
\flushright{\begin{Arabic}
\quranayah[76][31]
\end{Arabic}}
\flushleft{\begin{hindi}
वह जिसे चाहता है अपनी दयालुता में दाख़िल करता है। रहे ज़ालिम, तो उनके लिए उसने दुखद यातना तैयार कर रखी है
\end{hindi}}
\chapter{Al-Mursalat (Those Sent Forth)}
\begin{Arabic}
\Huge{\centerline{\basmalah}}\end{Arabic}
\flushright{\begin{Arabic}
\quranayah[77][1]
\end{Arabic}}
\flushleft{\begin{hindi}
साक्षी है वे (हवाएँ) जिनकी चोटी छोड़ दी जाती है
\end{hindi}}
\flushright{\begin{Arabic}
\quranayah[77][2]
\end{Arabic}}
\flushleft{\begin{hindi}
फिर ख़ूब तेज़ हो जाती है,
\end{hindi}}
\flushright{\begin{Arabic}
\quranayah[77][3]
\end{Arabic}}
\flushleft{\begin{hindi}
और (बादलों को) उठाकर फैलाती है,
\end{hindi}}
\flushright{\begin{Arabic}
\quranayah[77][4]
\end{Arabic}}
\flushleft{\begin{hindi}
फिर मामला करती है अलग-अलग,
\end{hindi}}
\flushright{\begin{Arabic}
\quranayah[77][5]
\end{Arabic}}
\flushleft{\begin{hindi}
फिर पेश करती है याददिहानी
\end{hindi}}
\flushright{\begin{Arabic}
\quranayah[77][6]
\end{Arabic}}
\flushleft{\begin{hindi}
इल्ज़ाम उतारने या चेतावनी देने के लिए,
\end{hindi}}
\flushright{\begin{Arabic}
\quranayah[77][7]
\end{Arabic}}
\flushleft{\begin{hindi}
निस्संदेह जिसका वादा तुमसे किया जा रहा है वह निश्चित ही घटित होकर रहेगा
\end{hindi}}
\flushright{\begin{Arabic}
\quranayah[77][8]
\end{Arabic}}
\flushleft{\begin{hindi}
अतः जब तारे विलुप्त (प्रकाशहीन) हो जाएँगे,
\end{hindi}}
\flushright{\begin{Arabic}
\quranayah[77][9]
\end{Arabic}}
\flushleft{\begin{hindi}
और जब आकाश फट जाएगा
\end{hindi}}
\flushright{\begin{Arabic}
\quranayah[77][10]
\end{Arabic}}
\flushleft{\begin{hindi}
और पहाड़ चूर्ण-विचूर्ण होकर बिखर जाएँगे;
\end{hindi}}
\flushright{\begin{Arabic}
\quranayah[77][11]
\end{Arabic}}
\flushleft{\begin{hindi}
और जब रसूलों का हाल यह होगा कि उन का समय नियत कर दिया गया होगा -
\end{hindi}}
\flushright{\begin{Arabic}
\quranayah[77][12]
\end{Arabic}}
\flushleft{\begin{hindi}
किस दिन के लिए वे टाले गए है?
\end{hindi}}
\flushright{\begin{Arabic}
\quranayah[77][13]
\end{Arabic}}
\flushleft{\begin{hindi}
फ़ैसले के दिन के लिए
\end{hindi}}
\flushright{\begin{Arabic}
\quranayah[77][14]
\end{Arabic}}
\flushleft{\begin{hindi}
और तुम्हें क्या मालूम कि वह फ़ैसले का दिन क्या है? -
\end{hindi}}
\flushright{\begin{Arabic}
\quranayah[77][15]
\end{Arabic}}
\flushleft{\begin{hindi}
तबाही है उस दिन झूठलाने-वालों की!
\end{hindi}}
\flushright{\begin{Arabic}
\quranayah[77][16]
\end{Arabic}}
\flushleft{\begin{hindi}
क्या ऐसा नहीं हुआ कि हमने पहलों को विनष्ट किया?
\end{hindi}}
\flushright{\begin{Arabic}
\quranayah[77][17]
\end{Arabic}}
\flushleft{\begin{hindi}
फिर उन्हीं के पीछे बादवालों को भी लगाते रहे?
\end{hindi}}
\flushright{\begin{Arabic}
\quranayah[77][18]
\end{Arabic}}
\flushleft{\begin{hindi}
अपराधियों के साथ हम ऐसा ही करते है
\end{hindi}}
\flushright{\begin{Arabic}
\quranayah[77][19]
\end{Arabic}}
\flushleft{\begin{hindi}
तबाही है उस दिन झुठलानेवालो की!
\end{hindi}}
\flushright{\begin{Arabic}
\quranayah[77][20]
\end{Arabic}}
\flushleft{\begin{hindi}
क्या ऐस नहीं है कि हमने तुम्हे तुच्छ जल से पैदा किया,
\end{hindi}}
\flushright{\begin{Arabic}
\quranayah[77][21]
\end{Arabic}}
\flushleft{\begin{hindi}
फिर हमने उसे एक सुरक्षित टिकने की जगह रखा,
\end{hindi}}
\flushright{\begin{Arabic}
\quranayah[77][22]
\end{Arabic}}
\flushleft{\begin{hindi}
एक ज्ञात और निश्चित अवधि तक?
\end{hindi}}
\flushright{\begin{Arabic}
\quranayah[77][23]
\end{Arabic}}
\flushleft{\begin{hindi}
फिर हमने अन्दाजा ठहराया, तो हम क्या ही अच्छा अन्दाज़ा ठहरानेवाले है
\end{hindi}}
\flushright{\begin{Arabic}
\quranayah[77][24]
\end{Arabic}}
\flushleft{\begin{hindi}
तबाही है उस दिन झूठलानेवालों की!
\end{hindi}}
\flushright{\begin{Arabic}
\quranayah[77][25]
\end{Arabic}}
\flushleft{\begin{hindi}
क्या ऐसा नहीं है कि हमने धरती को समेट रखनेवाली बनाया,
\end{hindi}}
\flushright{\begin{Arabic}
\quranayah[77][26]
\end{Arabic}}
\flushleft{\begin{hindi}
ज़िन्दों को भी और मुर्दों को भी,
\end{hindi}}
\flushright{\begin{Arabic}
\quranayah[77][27]
\end{Arabic}}
\flushleft{\begin{hindi}
और उसमें ऊँचे-ऊँचे पहाड़ जमाए और तुम्हें मीठा पानी पिलाया?
\end{hindi}}
\flushright{\begin{Arabic}
\quranayah[77][28]
\end{Arabic}}
\flushleft{\begin{hindi}
तबाही है उस दिन झुठलानेवालों की!
\end{hindi}}
\flushright{\begin{Arabic}
\quranayah[77][29]
\end{Arabic}}
\flushleft{\begin{hindi}
चलो उस चीज़ की ओर जिसे तुम झुठलाते रहे हो!
\end{hindi}}
\flushright{\begin{Arabic}
\quranayah[77][30]
\end{Arabic}}
\flushleft{\begin{hindi}
चलो तीन शाखाओंवाली छाया की ओर,
\end{hindi}}
\flushright{\begin{Arabic}
\quranayah[77][31]
\end{Arabic}}
\flushleft{\begin{hindi}
जिसमें न छाँव है और न वह अग्नि-ज्वाला से बचा सकती है
\end{hindi}}
\flushright{\begin{Arabic}
\quranayah[77][32]
\end{Arabic}}
\flushleft{\begin{hindi}
निस्संदेह वे (ज्वालाएँ) महल जैसी (ऊँची) चिंगारियाँ फेंकती है
\end{hindi}}
\flushright{\begin{Arabic}
\quranayah[77][33]
\end{Arabic}}
\flushleft{\begin{hindi}
मानो वे पीले ऊँट हैं!
\end{hindi}}
\flushright{\begin{Arabic}
\quranayah[77][34]
\end{Arabic}}
\flushleft{\begin{hindi}
तबाही है उस झुठलानेवालों की!
\end{hindi}}
\flushright{\begin{Arabic}
\quranayah[77][35]
\end{Arabic}}
\flushleft{\begin{hindi}
यह वह दिन है कि वे कुछ बोल नहीं रहे है,
\end{hindi}}
\flushright{\begin{Arabic}
\quranayah[77][36]
\end{Arabic}}
\flushleft{\begin{hindi}
तो कोई उज़ पेश करें, (बात यह है कि) उन्हें बोलने की अनुमति नहीं दी जा रही है
\end{hindi}}
\flushright{\begin{Arabic}
\quranayah[77][37]
\end{Arabic}}
\flushleft{\begin{hindi}
तबाही है उस दिन झुठलानेवालों की
\end{hindi}}
\flushright{\begin{Arabic}
\quranayah[77][38]
\end{Arabic}}
\flushleft{\begin{hindi}
"यह फ़ैसले का दिन है, हमने तुम्हें भी और पहलों को भी इकट्ठा कर दिया
\end{hindi}}
\flushright{\begin{Arabic}
\quranayah[77][39]
\end{Arabic}}
\flushleft{\begin{hindi}
"अब यदि तुम्हारे पास कोई चाल है तो मेरे विरुद्ध चलो।"
\end{hindi}}
\flushright{\begin{Arabic}
\quranayah[77][40]
\end{Arabic}}
\flushleft{\begin{hindi}
तबाही है उस दिन झुठलानेवालो की!
\end{hindi}}
\flushright{\begin{Arabic}
\quranayah[77][41]
\end{Arabic}}
\flushleft{\begin{hindi}
निस्संदेह डर रखनेवाले छाँवों और स्रोतों में है,
\end{hindi}}
\flushright{\begin{Arabic}
\quranayah[77][42]
\end{Arabic}}
\flushleft{\begin{hindi}
और उन फलों के बीच जो वे चाहे
\end{hindi}}
\flushright{\begin{Arabic}
\quranayah[77][43]
\end{Arabic}}
\flushleft{\begin{hindi}
"खाओ-पियो मज़े से, उस कर्मों के बदले में जो तुम करते रहे हो।"
\end{hindi}}
\flushright{\begin{Arabic}
\quranayah[77][44]
\end{Arabic}}
\flushleft{\begin{hindi}
निश्चय ही उत्तमकारों को हम ऐसा ही बदला देते है
\end{hindi}}
\flushright{\begin{Arabic}
\quranayah[77][45]
\end{Arabic}}
\flushleft{\begin{hindi}
तबाही है उस दिन झुठलानेवालों की!
\end{hindi}}
\flushright{\begin{Arabic}
\quranayah[77][46]
\end{Arabic}}
\flushleft{\begin{hindi}
"खा लो और मज़े कर लो थोड़ा-सा, वास्तव में तुम अपराधी हो!"
\end{hindi}}
\flushright{\begin{Arabic}
\quranayah[77][47]
\end{Arabic}}
\flushleft{\begin{hindi}
तबाही है उस दिन झुठलानेवालों की!
\end{hindi}}
\flushright{\begin{Arabic}
\quranayah[77][48]
\end{Arabic}}
\flushleft{\begin{hindi}
जब उनसे कहा जाता है कि "झुको! तो नहीं झुकते।"
\end{hindi}}
\flushright{\begin{Arabic}
\quranayah[77][49]
\end{Arabic}}
\flushleft{\begin{hindi}
तबाही है उस दिन झुठलानेवालों की!
\end{hindi}}
\flushright{\begin{Arabic}
\quranayah[77][50]
\end{Arabic}}
\flushleft{\begin{hindi}
अब आख़िर इसके पश्चात किस वाणी पर वे ईमान लाएँगे?
\end{hindi}}
\chapter{An-Naba' (The Announcement)}
\begin{Arabic}
\Huge{\centerline{\basmalah}}\end{Arabic}
\flushright{\begin{Arabic}
\quranayah[78][1]
\end{Arabic}}
\flushleft{\begin{hindi}
किस चीज़ के विषय में वे आपस में पूछ-गच्छ कर रहे है?
\end{hindi}}
\flushright{\begin{Arabic}
\quranayah[78][2]
\end{Arabic}}
\flushleft{\begin{hindi}
उस बड़ी ख़बर के सम्बन्ध में,
\end{hindi}}
\flushright{\begin{Arabic}
\quranayah[78][3]
\end{Arabic}}
\flushleft{\begin{hindi}
जिसमें वे मतभेद रखते है
\end{hindi}}
\flushright{\begin{Arabic}
\quranayah[78][4]
\end{Arabic}}
\flushleft{\begin{hindi}
कदापि नहीं, शीघ्र ही वे जान लेंगे।
\end{hindi}}
\flushright{\begin{Arabic}
\quranayah[78][5]
\end{Arabic}}
\flushleft{\begin{hindi}
फिर कदापि नहीं, शीघ्र ही वे जान लेंगे।
\end{hindi}}
\flushright{\begin{Arabic}
\quranayah[78][6]
\end{Arabic}}
\flushleft{\begin{hindi}
क्या ऐसा नहीं है कि हमने धरती को बिछौना बनाया
\end{hindi}}
\flushright{\begin{Arabic}
\quranayah[78][7]
\end{Arabic}}
\flushleft{\begin{hindi}
और पहाड़ों को मेख़े?
\end{hindi}}
\flushright{\begin{Arabic}
\quranayah[78][8]
\end{Arabic}}
\flushleft{\begin{hindi}
और हमने तुम्हें जोड़-जोड़े पैदा किया,
\end{hindi}}
\flushright{\begin{Arabic}
\quranayah[78][9]
\end{Arabic}}
\flushleft{\begin{hindi}
और तुम्हारी नींद को थकन दूर करनेवाली बनाया,
\end{hindi}}
\flushright{\begin{Arabic}
\quranayah[78][10]
\end{Arabic}}
\flushleft{\begin{hindi}
रात को आवरण बनाया,
\end{hindi}}
\flushright{\begin{Arabic}
\quranayah[78][11]
\end{Arabic}}
\flushleft{\begin{hindi}
और दिन को जीवन-वृति के लिए बनाया
\end{hindi}}
\flushright{\begin{Arabic}
\quranayah[78][12]
\end{Arabic}}
\flushleft{\begin{hindi}
और तुम्हारे ऊपर सात सुदृढ़ आकाश निर्मित किए,
\end{hindi}}
\flushright{\begin{Arabic}
\quranayah[78][13]
\end{Arabic}}
\flushleft{\begin{hindi}
और एक तप्त और प्रकाशमान प्रदीप बनाया,
\end{hindi}}
\flushright{\begin{Arabic}
\quranayah[78][14]
\end{Arabic}}
\flushleft{\begin{hindi}
और बरस पड़नेवाली घटाओं से हमने मूसलाधार पानी उतारा,
\end{hindi}}
\flushright{\begin{Arabic}
\quranayah[78][15]
\end{Arabic}}
\flushleft{\begin{hindi}
ताकि हम उसके द्वारा अनाज और वनस्पति उत्पादित करें
\end{hindi}}
\flushright{\begin{Arabic}
\quranayah[78][16]
\end{Arabic}}
\flushleft{\begin{hindi}
और सघन बांग़ भी।
\end{hindi}}
\flushright{\begin{Arabic}
\quranayah[78][17]
\end{Arabic}}
\flushleft{\begin{hindi}
निस्संदेह फ़ैसले का दिन एक नियत समय है,
\end{hindi}}
\flushright{\begin{Arabic}
\quranayah[78][18]
\end{Arabic}}
\flushleft{\begin{hindi}
जिस दिन नरसिंघा में फूँक मारी जाएगी, तो तुम गिरोह को गिरोह चले आओगे।
\end{hindi}}
\flushright{\begin{Arabic}
\quranayah[78][19]
\end{Arabic}}
\flushleft{\begin{hindi}
और आकाश खोल दिया जाएगा तो द्वार ही द्वार हो जाएँगे;
\end{hindi}}
\flushright{\begin{Arabic}
\quranayah[78][20]
\end{Arabic}}
\flushleft{\begin{hindi}
और पहाड़ चलाए जाएँगे, तो वे बिल्कुल मरीचिका होकर रह जाएँगे
\end{hindi}}
\flushright{\begin{Arabic}
\quranayah[78][21]
\end{Arabic}}
\flushleft{\begin{hindi}
वास्तव में जहन्नम एक घात-स्थल है;
\end{hindi}}
\flushright{\begin{Arabic}
\quranayah[78][22]
\end{Arabic}}
\flushleft{\begin{hindi}
सरकशों का ठिकाना है
\end{hindi}}
\flushright{\begin{Arabic}
\quranayah[78][23]
\end{Arabic}}
\flushleft{\begin{hindi}
वस्तुस्थिति यह है कि वे उसमें मुद्दत पर मुद्दत बिताते रहेंगे
\end{hindi}}
\flushright{\begin{Arabic}
\quranayah[78][24]
\end{Arabic}}
\flushleft{\begin{hindi}
वे उसमे न किसी शीतलता का मज़ा चखेगे और न किसी पेय का,
\end{hindi}}
\flushright{\begin{Arabic}
\quranayah[78][25]
\end{Arabic}}
\flushleft{\begin{hindi}
सिवाय खौलते पानी और बहती पीप-रक्त के
\end{hindi}}
\flushright{\begin{Arabic}
\quranayah[78][26]
\end{Arabic}}
\flushleft{\begin{hindi}
यह बदले के रूप में उनके कर्मों के ठीक अनुकूल होगा
\end{hindi}}
\flushright{\begin{Arabic}
\quranayah[78][27]
\end{Arabic}}
\flushleft{\begin{hindi}
वास्तव में किसी हिसाब की आशा न रखते थे,
\end{hindi}}
\flushright{\begin{Arabic}
\quranayah[78][28]
\end{Arabic}}
\flushleft{\begin{hindi}
और उन्होंने हमारी आयतों को ख़ूब झुठलाया,
\end{hindi}}
\flushright{\begin{Arabic}
\quranayah[78][29]
\end{Arabic}}
\flushleft{\begin{hindi}
और हमने हर चीज़ लिखकर गिन रखी है
\end{hindi}}
\flushright{\begin{Arabic}
\quranayah[78][30]
\end{Arabic}}
\flushleft{\begin{hindi}
"अब चखो मज़ा कि यातना के अतिरिक्त हम तुम्हारे लिए किसी और चीज़ में बढ़ोत्तरी नहीं करेंगे। "
\end{hindi}}
\flushright{\begin{Arabic}
\quranayah[78][31]
\end{Arabic}}
\flushleft{\begin{hindi}
निस्सदेह डर रखनेवालों के लिए एक बड़ी सफलता है,
\end{hindi}}
\flushright{\begin{Arabic}
\quranayah[78][32]
\end{Arabic}}
\flushleft{\begin{hindi}
बाग़ है और अंगूर,
\end{hindi}}
\flushright{\begin{Arabic}
\quranayah[78][33]
\end{Arabic}}
\flushleft{\begin{hindi}
और नवयौवना समान उम्रवाली,
\end{hindi}}
\flushright{\begin{Arabic}
\quranayah[78][34]
\end{Arabic}}
\flushleft{\begin{hindi}
और छलक़ता जाम
\end{hindi}}
\flushright{\begin{Arabic}
\quranayah[78][35]
\end{Arabic}}
\flushleft{\begin{hindi}
वे उसमें न तो कोई व्यर्थ बात सुनेंगे और न कोई झुठलाने की बात
\end{hindi}}
\flushright{\begin{Arabic}
\quranayah[78][36]
\end{Arabic}}
\flushleft{\begin{hindi}
यह तुम्हारे रब की ओर से बदला होगा, हिसाब के अनुसार प्रदत्त
\end{hindi}}
\flushright{\begin{Arabic}
\quranayah[78][37]
\end{Arabic}}
\flushleft{\begin{hindi}
वह आकाशों और धरती का और जो कुछ उनके बीच है सबका रब है, अत्यन्त कृपाशील है, उसके सामने बात करना उनके बस में नहीं होगा
\end{hindi}}
\flushright{\begin{Arabic}
\quranayah[78][38]
\end{Arabic}}
\flushleft{\begin{hindi}
जिस दिन रूह और फ़रिश्ते पक्तिबद्ध खड़े होंगे, वे बोलेंगे नहीं, सिवाय उस व्यक्ति के जिसे रहमान अनुमति दे और जो ठीक बात कहे
\end{hindi}}
\flushright{\begin{Arabic}
\quranayah[78][39]
\end{Arabic}}
\flushleft{\begin{hindi}
वह दिन सत्य है। अब जो कोई चाहे अपने रब की ओर रुज करे
\end{hindi}}
\flushright{\begin{Arabic}
\quranayah[78][40]
\end{Arabic}}
\flushleft{\begin{hindi}
हमने तुम्हें निकट आ लगी यातना से सावधान कर दिया है। जिस दिन मनुष्य देख लेगा जो कुछ उसके हाथों ने आगे भेजा, और इनकार करनेवाला कहेगा, "ऐ काश! कि मैं मिट्टी होता!"
\end{hindi}}
\chapter{An-Nazi'at (Those Who Yearn)}
\begin{Arabic}
\Huge{\centerline{\basmalah}}\end{Arabic}
\flushright{\begin{Arabic}
\quranayah[79][1]
\end{Arabic}}
\flushleft{\begin{hindi}
गवाह है वे (हवाएँ) जो ज़ोर से उखाड़ फैंके,
\end{hindi}}
\flushright{\begin{Arabic}
\quranayah[79][2]
\end{Arabic}}
\flushleft{\begin{hindi}
और गवाह है वे (हवाएँ) जो नर्मी के साथ चलें,
\end{hindi}}
\flushright{\begin{Arabic}
\quranayah[79][3]
\end{Arabic}}
\flushleft{\begin{hindi}
और गवाह है वे जो वायुमंडल में तैरें,
\end{hindi}}
\flushright{\begin{Arabic}
\quranayah[79][4]
\end{Arabic}}
\flushleft{\begin{hindi}
फिर एक-दूसरे से अग्रसर हों,
\end{hindi}}
\flushright{\begin{Arabic}
\quranayah[79][5]
\end{Arabic}}
\flushleft{\begin{hindi}
और मामले की तदबीर करें
\end{hindi}}
\flushright{\begin{Arabic}
\quranayah[79][6]
\end{Arabic}}
\flushleft{\begin{hindi}
जिस दिन हिला डालेगी हिला डालनेवाले घटना,
\end{hindi}}
\flushright{\begin{Arabic}
\quranayah[79][7]
\end{Arabic}}
\flushleft{\begin{hindi}
उसके पीछ घटित होगी दूसरी (घटना)
\end{hindi}}
\flushright{\begin{Arabic}
\quranayah[79][8]
\end{Arabic}}
\flushleft{\begin{hindi}
कितने ही दिल उस दिन काँप रहे होंगे,
\end{hindi}}
\flushright{\begin{Arabic}
\quranayah[79][9]
\end{Arabic}}
\flushleft{\begin{hindi}
उनकी निगाहें झुकी होंगी
\end{hindi}}
\flushright{\begin{Arabic}
\quranayah[79][10]
\end{Arabic}}
\flushleft{\begin{hindi}
वे कहते है, "क्या वास्तव में हम पहली हालत में फिर लौटाए जाएँगे?
\end{hindi}}
\flushright{\begin{Arabic}
\quranayah[79][11]
\end{Arabic}}
\flushleft{\begin{hindi}
क्या जब हम खोखली गलित हड्डियाँ हो चुके होंगे?"
\end{hindi}}
\flushright{\begin{Arabic}
\quranayah[79][12]
\end{Arabic}}
\flushleft{\begin{hindi}
वे कहते है, "तब तो लौटना बड़े ही घाटे का होगा।"
\end{hindi}}
\flushright{\begin{Arabic}
\quranayah[79][13]
\end{Arabic}}
\flushleft{\begin{hindi}
वह तो बस एक ही झिड़की होगी,
\end{hindi}}
\flushright{\begin{Arabic}
\quranayah[79][14]
\end{Arabic}}
\flushleft{\begin{hindi}
फिर क्या देखेंगे कि वे एक समतल मैदान में उपस्थित है
\end{hindi}}
\flushright{\begin{Arabic}
\quranayah[79][15]
\end{Arabic}}
\flushleft{\begin{hindi}
क्या तुम्हें मूसा की ख़बर पहुँची है?
\end{hindi}}
\flushright{\begin{Arabic}
\quranayah[79][16]
\end{Arabic}}
\flushleft{\begin{hindi}
जबकि उसके रब ने पवित्र घाटी 'तुवा' में उसे पुकारा था
\end{hindi}}
\flushright{\begin{Arabic}
\quranayah[79][17]
\end{Arabic}}
\flushleft{\begin{hindi}
कि "फ़िरऔन के पास जाओ, उसने बहुत सिर उठा रखा है
\end{hindi}}
\flushright{\begin{Arabic}
\quranayah[79][18]
\end{Arabic}}
\flushleft{\begin{hindi}
"और कहो, क्या तू यह चाहता है कि स्वयं को पाक-साफ़ कर ले,
\end{hindi}}
\flushright{\begin{Arabic}
\quranayah[79][19]
\end{Arabic}}
\flushleft{\begin{hindi}
"और मैं तेरे रब की ओर तेरा मार्गदर्शन करूँ कि तु (उससे) डरे?"
\end{hindi}}
\flushright{\begin{Arabic}
\quranayah[79][20]
\end{Arabic}}
\flushleft{\begin{hindi}
फिर उसने (मूसा ने) उसको बड़ी निशानी दिखाई,
\end{hindi}}
\flushright{\begin{Arabic}
\quranayah[79][21]
\end{Arabic}}
\flushleft{\begin{hindi}
किन्तु उसने झुठला दिया और कहा न माना,
\end{hindi}}
\flushright{\begin{Arabic}
\quranayah[79][22]
\end{Arabic}}
\flushleft{\begin{hindi}
फिर सक्रियता दिखाते हुए पलटा,
\end{hindi}}
\flushright{\begin{Arabic}
\quranayah[79][23]
\end{Arabic}}
\flushleft{\begin{hindi}
फिर (लोगों को) एकत्र किया और पुकारकर कहा,
\end{hindi}}
\flushright{\begin{Arabic}
\quranayah[79][24]
\end{Arabic}}
\flushleft{\begin{hindi}
"मैं तुम्हारा उच्चकोटि का स्वामी हूँ!"
\end{hindi}}
\flushright{\begin{Arabic}
\quranayah[79][25]
\end{Arabic}}
\flushleft{\begin{hindi}
अन्ततः अल्लाह ने उसे आख़िरत और दुनिया की शिक्षाप्रद यातना में पकड़ लिया
\end{hindi}}
\flushright{\begin{Arabic}
\quranayah[79][26]
\end{Arabic}}
\flushleft{\begin{hindi}
निस्संदेह इसमें उस व्यक्ति के लिए बड़ी शिक्षा है जो डरे!
\end{hindi}}
\flushright{\begin{Arabic}
\quranayah[79][27]
\end{Arabic}}
\flushleft{\begin{hindi}
क्या तुम्हें पैदा करना अधिक कठिन कार्य है या आकाश को? अल्लाह ने उसे बनाया,
\end{hindi}}
\flushright{\begin{Arabic}
\quranayah[79][28]
\end{Arabic}}
\flushleft{\begin{hindi}
उसकी ऊँचाई को ख़ूब ऊँचा करके उसे ठीक-ठाक किया;
\end{hindi}}
\flushright{\begin{Arabic}
\quranayah[79][29]
\end{Arabic}}
\flushleft{\begin{hindi}
और उसकी रात को अन्धकारमय बनाया और उसका दिवस-प्रकाश प्रकट किया
\end{hindi}}
\flushright{\begin{Arabic}
\quranayah[79][30]
\end{Arabic}}
\flushleft{\begin{hindi}
और धरती को देखो! इसके पश्चात उसे फैलाया;
\end{hindi}}
\flushright{\begin{Arabic}
\quranayah[79][31]
\end{Arabic}}
\flushleft{\begin{hindi}
उसमें से उसका पानी और उसका चारा निकाला
\end{hindi}}
\flushright{\begin{Arabic}
\quranayah[79][32]
\end{Arabic}}
\flushleft{\begin{hindi}
और पहाड़ो को देखो! उन्हें उस (धरती) में जमा दिया,
\end{hindi}}
\flushright{\begin{Arabic}
\quranayah[79][33]
\end{Arabic}}
\flushleft{\begin{hindi}
तुम्हारे लिए और तुम्हारे मवेशियों के लिए जीवन-सामग्री के रूप में
\end{hindi}}
\flushright{\begin{Arabic}
\quranayah[79][34]
\end{Arabic}}
\flushleft{\begin{hindi}
फिर जब वह महाविपदा आएगी,
\end{hindi}}
\flushright{\begin{Arabic}
\quranayah[79][35]
\end{Arabic}}
\flushleft{\begin{hindi}
उस दिन मनुष्य जो कुछ भी उसने प्रयास किया होगा उसे याद करेगा
\end{hindi}}
\flushright{\begin{Arabic}
\quranayah[79][36]
\end{Arabic}}
\flushleft{\begin{hindi}
और भड़कती आग (जहन्नम) देखने वालों के लिए खोल दी जाएगी
\end{hindi}}
\flushright{\begin{Arabic}
\quranayah[79][37]
\end{Arabic}}
\flushleft{\begin{hindi}
तो जिस किसी ने सरकशी की
\end{hindi}}
\flushright{\begin{Arabic}
\quranayah[79][38]
\end{Arabic}}
\flushleft{\begin{hindi}
और सांसारिक जीवन को प्राथमिकता दो होगी,
\end{hindi}}
\flushright{\begin{Arabic}
\quranayah[79][39]
\end{Arabic}}
\flushleft{\begin{hindi}
तो निस्संदेह भड़कती आग ही उसका ठिकाना है
\end{hindi}}
\flushright{\begin{Arabic}
\quranayah[79][40]
\end{Arabic}}
\flushleft{\begin{hindi}
और रहा वह व्यक्ति जिसने अपने रब के सामने खड़े होने का भय रखा और अपने जी को बुरी इच्छा से रोका,
\end{hindi}}
\flushright{\begin{Arabic}
\quranayah[79][41]
\end{Arabic}}
\flushleft{\begin{hindi}
तो जन्नत ही उसका ठिकाना है
\end{hindi}}
\flushright{\begin{Arabic}
\quranayah[79][42]
\end{Arabic}}
\flushleft{\begin{hindi}
वे तुमसे उस घड़ी के विषय में पूछते है कि वह कब आकर ठहरेगी?
\end{hindi}}
\flushright{\begin{Arabic}
\quranayah[79][43]
\end{Arabic}}
\flushleft{\begin{hindi}
उसके बयान करने से तुम्हारा क्या सम्बन्ध?
\end{hindi}}
\flushright{\begin{Arabic}
\quranayah[79][44]
\end{Arabic}}
\flushleft{\begin{hindi}
उसकी अन्तिम पहुँच तो तेरे से ही सम्बन्ध रखती है
\end{hindi}}
\flushright{\begin{Arabic}
\quranayah[79][45]
\end{Arabic}}
\flushleft{\begin{hindi}
तुम तो बस उस व्यक्ति को सावधान करनेवाले हो जो उससे डरे
\end{hindi}}
\flushright{\begin{Arabic}
\quranayah[79][46]
\end{Arabic}}
\flushleft{\begin{hindi}
जिस दिन वे उसे देखेंगे तो (ऐसा लगेगा) मानो वे (दुनिया में) बस एक शाम या उसकी सुबह ही ठहरे है
\end{hindi}}
\chapter{'Abasa (He Frowned)}
\begin{Arabic}
\Huge{\centerline{\basmalah}}\end{Arabic}
\flushright{\begin{Arabic}
\quranayah[80][1]
\end{Arabic}}
\flushleft{\begin{hindi}
उसने त्योरी चढ़ाई और मुँह फेर लिया,
\end{hindi}}
\flushright{\begin{Arabic}
\quranayah[80][2]
\end{Arabic}}
\flushleft{\begin{hindi}
इस कारण कि उसके पास अन्धा आ गया।
\end{hindi}}
\flushright{\begin{Arabic}
\quranayah[80][3]
\end{Arabic}}
\flushleft{\begin{hindi}
और तुझे क्या मालूम शायद वह स्वयं को सँवारता-निखारता हो
\end{hindi}}
\flushright{\begin{Arabic}
\quranayah[80][4]
\end{Arabic}}
\flushleft{\begin{hindi}
या नसीहत हासिल करता हो तो नसीहत उसके लिए लाभदायक हो?
\end{hindi}}
\flushright{\begin{Arabic}
\quranayah[80][5]
\end{Arabic}}
\flushleft{\begin{hindi}
रहा वह व्यक्ति जो धनी हो गया ह
\end{hindi}}
\flushright{\begin{Arabic}
\quranayah[80][6]
\end{Arabic}}
\flushleft{\begin{hindi}
तू उसके पीछे पड़ा है -
\end{hindi}}
\flushright{\begin{Arabic}
\quranayah[80][7]
\end{Arabic}}
\flushleft{\begin{hindi}
हालाँकि वह अपने को न निखारे तो तुझपर कोई ज़िम्मेदारी नहीं आती -
\end{hindi}}
\flushright{\begin{Arabic}
\quranayah[80][8]
\end{Arabic}}
\flushleft{\begin{hindi}
और रहा वह व्यक्ति जो स्वयं ही तेरे पास दौड़ता हुआ आया,
\end{hindi}}
\flushright{\begin{Arabic}
\quranayah[80][9]
\end{Arabic}}
\flushleft{\begin{hindi}
और वह डरता भी है,
\end{hindi}}
\flushright{\begin{Arabic}
\quranayah[80][10]
\end{Arabic}}
\flushleft{\begin{hindi}
तो तू उससे बेपरवाई करता है
\end{hindi}}
\flushright{\begin{Arabic}
\quranayah[80][11]
\end{Arabic}}
\flushleft{\begin{hindi}
कदापि नहीं, वे (आयतें) तो महत्वपूर्ण नसीहत है -
\end{hindi}}
\flushright{\begin{Arabic}
\quranayah[80][12]
\end{Arabic}}
\flushleft{\begin{hindi}
तो जो चाहे उसे याद कर ले -
\end{hindi}}
\flushright{\begin{Arabic}
\quranayah[80][13]
\end{Arabic}}
\flushleft{\begin{hindi}
पवित्र पन्नों में अंकित है,
\end{hindi}}
\flushright{\begin{Arabic}
\quranayah[80][14]
\end{Arabic}}
\flushleft{\begin{hindi}
प्रतिष्ठि्त, उच्च,
\end{hindi}}
\flushright{\begin{Arabic}
\quranayah[80][15]
\end{Arabic}}
\flushleft{\begin{hindi}
ऐसे कातिबों के हाथों में रहा करते है
\end{hindi}}
\flushright{\begin{Arabic}
\quranayah[80][16]
\end{Arabic}}
\flushleft{\begin{hindi}
जो प्रतिष्ठित और नेक है
\end{hindi}}
\flushright{\begin{Arabic}
\quranayah[80][17]
\end{Arabic}}
\flushleft{\begin{hindi}
विनष्ट हुआ मनुष्य! कैसा अकृतज्ञ है!
\end{hindi}}
\flushright{\begin{Arabic}
\quranayah[80][18]
\end{Arabic}}
\flushleft{\begin{hindi}
उसको किस चीज़ से पैदा किया?
\end{hindi}}
\flushright{\begin{Arabic}
\quranayah[80][19]
\end{Arabic}}
\flushleft{\begin{hindi}
तनिक-सी बूँद से उसको पैदा किया, तो उसके लिए एक अंदाजा ठहराया,
\end{hindi}}
\flushright{\begin{Arabic}
\quranayah[80][20]
\end{Arabic}}
\flushleft{\begin{hindi}
फिर मार्ग को देखो, उसे सुगम कर दिया,
\end{hindi}}
\flushright{\begin{Arabic}
\quranayah[80][21]
\end{Arabic}}
\flushleft{\begin{hindi}
फिर उसे मृत्यु दी और क्रब में उसे रखवाया,
\end{hindi}}
\flushright{\begin{Arabic}
\quranayah[80][22]
\end{Arabic}}
\flushleft{\begin{hindi}
फिर जब चाहेगा उसे (जीवित करके) उठा खड़ा करेगा। -
\end{hindi}}
\flushright{\begin{Arabic}
\quranayah[80][23]
\end{Arabic}}
\flushleft{\begin{hindi}
कदापि नहीं, उसने उसको पूरा नहीं किया जिसका आदेश अल्लाह ने उसे दिया है
\end{hindi}}
\flushright{\begin{Arabic}
\quranayah[80][24]
\end{Arabic}}
\flushleft{\begin{hindi}
अतः मनुष्य को चाहिए कि अपने भोजन को देखे,
\end{hindi}}
\flushright{\begin{Arabic}
\quranayah[80][25]
\end{Arabic}}
\flushleft{\begin{hindi}
कि हमने ख़ूब पानी बरसाया,
\end{hindi}}
\flushright{\begin{Arabic}
\quranayah[80][26]
\end{Arabic}}
\flushleft{\begin{hindi}
फिर धरती को विशेष रूप से फाड़ा,
\end{hindi}}
\flushright{\begin{Arabic}
\quranayah[80][27]
\end{Arabic}}
\flushleft{\begin{hindi}
फिर हमने उसमें उगाए अनाज,
\end{hindi}}
\flushright{\begin{Arabic}
\quranayah[80][28]
\end{Arabic}}
\flushleft{\begin{hindi}
और अंगूर और तरकारी,
\end{hindi}}
\flushright{\begin{Arabic}
\quranayah[80][29]
\end{Arabic}}
\flushleft{\begin{hindi}
और ज़ैतून और खजूर,
\end{hindi}}
\flushright{\begin{Arabic}
\quranayah[80][30]
\end{Arabic}}
\flushleft{\begin{hindi}
और घने बाग़,
\end{hindi}}
\flushright{\begin{Arabic}
\quranayah[80][31]
\end{Arabic}}
\flushleft{\begin{hindi}
और मेवे और घास-चारा,
\end{hindi}}
\flushright{\begin{Arabic}
\quranayah[80][32]
\end{Arabic}}
\flushleft{\begin{hindi}
तुम्हारे लिए और तुम्हारे चौपायों के लिेए जीवन-सामग्री के रूप में
\end{hindi}}
\flushright{\begin{Arabic}
\quranayah[80][33]
\end{Arabic}}
\flushleft{\begin{hindi}
फिर जब वह बहरा कर देनेवाली प्रचंड आवाज़ आएगी,
\end{hindi}}
\flushright{\begin{Arabic}
\quranayah[80][34]
\end{Arabic}}
\flushleft{\begin{hindi}
जिस दिन आदमी भागेगा अपने भाई से,
\end{hindi}}
\flushright{\begin{Arabic}
\quranayah[80][35]
\end{Arabic}}
\flushleft{\begin{hindi}
और अपनी माँ और अपने बाप से,
\end{hindi}}
\flushright{\begin{Arabic}
\quranayah[80][36]
\end{Arabic}}
\flushleft{\begin{hindi}
और अपनी पत्नी और अपने बेटों से
\end{hindi}}
\flushright{\begin{Arabic}
\quranayah[80][37]
\end{Arabic}}
\flushleft{\begin{hindi}
उनमें से प्रत्येक व्यक्ति को उस दिन ऐसी पड़ी होगी जो उसे दूसरों से बेपरवाह कर देगी
\end{hindi}}
\flushright{\begin{Arabic}
\quranayah[80][38]
\end{Arabic}}
\flushleft{\begin{hindi}
कितने ही चेहरे उस दिन रौशन होंगे,
\end{hindi}}
\flushright{\begin{Arabic}
\quranayah[80][39]
\end{Arabic}}
\flushleft{\begin{hindi}
हँसते, प्रफुल्लित
\end{hindi}}
\flushright{\begin{Arabic}
\quranayah[80][40]
\end{Arabic}}
\flushleft{\begin{hindi}
और कितने ही चेहरे होंगे जिनपर उस दिन धूल पड़ी होगी,
\end{hindi}}
\flushright{\begin{Arabic}
\quranayah[80][41]
\end{Arabic}}
\flushleft{\begin{hindi}
उनपर कलौंस छा रही होगी
\end{hindi}}
\flushright{\begin{Arabic}
\quranayah[80][42]
\end{Arabic}}
\flushleft{\begin{hindi}
वहीं होंगे इनकार करनेवाले दुराचारी लोग!
\end{hindi}}
\chapter{At-Takwir (The Folding Up)}
\begin{Arabic}
\Huge{\centerline{\basmalah}}\end{Arabic}
\flushright{\begin{Arabic}
\quranayah[81][1]
\end{Arabic}}
\flushleft{\begin{hindi}
जब सूर्य लपेट दिया जाएगा,
\end{hindi}}
\flushright{\begin{Arabic}
\quranayah[81][2]
\end{Arabic}}
\flushleft{\begin{hindi}
सारे तारे मैले हो जाएँगे,
\end{hindi}}
\flushright{\begin{Arabic}
\quranayah[81][3]
\end{Arabic}}
\flushleft{\begin{hindi}
जब पहाड़ चलाए जाएँगे,
\end{hindi}}
\flushright{\begin{Arabic}
\quranayah[81][4]
\end{Arabic}}
\flushleft{\begin{hindi}
जब दस मास की गाभिन ऊँटनियाँ आज़ाद छोड़ दी जाएँगी,
\end{hindi}}
\flushright{\begin{Arabic}
\quranayah[81][5]
\end{Arabic}}
\flushleft{\begin{hindi}
जब जंगली जानवर एकत्र किए जाएँगे,
\end{hindi}}
\flushright{\begin{Arabic}
\quranayah[81][6]
\end{Arabic}}
\flushleft{\begin{hindi}
जब समुद्र भड़का दिया जाएँगे,
\end{hindi}}
\flushright{\begin{Arabic}
\quranayah[81][7]
\end{Arabic}}
\flushleft{\begin{hindi}
जब लोग क़िस्म-क़िस्म कर दिए जाएँगे,
\end{hindi}}
\flushright{\begin{Arabic}
\quranayah[81][8]
\end{Arabic}}
\flushleft{\begin{hindi}
और जब जीवित गाड़ी गई लड़की से पूछा जाएगा,
\end{hindi}}
\flushright{\begin{Arabic}
\quranayah[81][9]
\end{Arabic}}
\flushleft{\begin{hindi}
कि उसकी हत्या किस गुनाह के कारण की गई,
\end{hindi}}
\flushright{\begin{Arabic}
\quranayah[81][10]
\end{Arabic}}
\flushleft{\begin{hindi}
और जब कर्म-पत्र फैला दिए जाएँगे,
\end{hindi}}
\flushright{\begin{Arabic}
\quranayah[81][11]
\end{Arabic}}
\flushleft{\begin{hindi}
और जब आकाश की खाल उतार दी जाएगी,
\end{hindi}}
\flushright{\begin{Arabic}
\quranayah[81][12]
\end{Arabic}}
\flushleft{\begin{hindi}
जब जहन्नम को दहकाया जाएगा,
\end{hindi}}
\flushright{\begin{Arabic}
\quranayah[81][13]
\end{Arabic}}
\flushleft{\begin{hindi}
और जब जन्नत निकट कर दी जाएगी,
\end{hindi}}
\flushright{\begin{Arabic}
\quranayah[81][14]
\end{Arabic}}
\flushleft{\begin{hindi}
तो कोई भी क्यक्ति जान लेगा कि उसने क्या उपस्थित किया है
\end{hindi}}
\flushright{\begin{Arabic}
\quranayah[81][15]
\end{Arabic}}
\flushleft{\begin{hindi}
अतः नहीं! मैं क़सम खाता हूँ पीछे हटनेवालों की,
\end{hindi}}
\flushright{\begin{Arabic}
\quranayah[81][16]
\end{Arabic}}
\flushleft{\begin{hindi}
चलनेवालों, छिपने-दुबकने-वालों की
\end{hindi}}
\flushright{\begin{Arabic}
\quranayah[81][17]
\end{Arabic}}
\flushleft{\begin{hindi}
साक्षी है रात्रि जब वह प्रस्थान करे,
\end{hindi}}
\flushright{\begin{Arabic}
\quranayah[81][18]
\end{Arabic}}
\flushleft{\begin{hindi}
और साक्षी है प्रातः जब वह साँस ले
\end{hindi}}
\flushright{\begin{Arabic}
\quranayah[81][19]
\end{Arabic}}
\flushleft{\begin{hindi}
निश्चय ही वह एक आदरणीय संदेशवाहक की लाई हुई वाणी है,
\end{hindi}}
\flushright{\begin{Arabic}
\quranayah[81][20]
\end{Arabic}}
\flushleft{\begin{hindi}
जो शक्तिवाला है, सिंहासनवाले के यहाँ जिसकी पैठ है
\end{hindi}}
\flushright{\begin{Arabic}
\quranayah[81][21]
\end{Arabic}}
\flushleft{\begin{hindi}
उसका आदेश माना जाता है, वहाँ वह विश्वासपात्र है
\end{hindi}}
\flushright{\begin{Arabic}
\quranayah[81][22]
\end{Arabic}}
\flushleft{\begin{hindi}
तुम्हारा साथी कोई दीवाना नहीं,
\end{hindi}}
\flushright{\begin{Arabic}
\quranayah[81][23]
\end{Arabic}}
\flushleft{\begin{hindi}
उसने तो (पराकाष्ठान के) प्रत्यक्ष क्षितिज पर होकर उस (फ़रिश्ते) को देखा है
\end{hindi}}
\flushright{\begin{Arabic}
\quranayah[81][24]
\end{Arabic}}
\flushleft{\begin{hindi}
और वह परोक्ष के मामले में कृपण नहीं है,
\end{hindi}}
\flushright{\begin{Arabic}
\quranayah[81][25]
\end{Arabic}}
\flushleft{\begin{hindi}
और वह (क़ुरआन) किसी धुतकारे हुए शैतान की लाई हुई वाणी नहीं है
\end{hindi}}
\flushright{\begin{Arabic}
\quranayah[81][26]
\end{Arabic}}
\flushleft{\begin{hindi}
फिर तुम किधर जा रहे हो?
\end{hindi}}
\flushright{\begin{Arabic}
\quranayah[81][27]
\end{Arabic}}
\flushleft{\begin{hindi}
वह तो सारे संसार के लिए बस एक अनुस्मृति है,
\end{hindi}}
\flushright{\begin{Arabic}
\quranayah[81][28]
\end{Arabic}}
\flushleft{\begin{hindi}
उसके लिए तो तुममे से सीधे मार्ग पर चलना चाहे
\end{hindi}}
\flushright{\begin{Arabic}
\quranayah[81][29]
\end{Arabic}}
\flushleft{\begin{hindi}
और तुम नहीं चाह सकते सिवाय इसके कि सारे जहान का रब अल्लाह चाहे
\end{hindi}}
\chapter{Al-Infitar (The Cleaving)}
\begin{Arabic}
\Huge{\centerline{\basmalah}}\end{Arabic}
\flushright{\begin{Arabic}
\quranayah[82][1]
\end{Arabic}}
\flushleft{\begin{hindi}
जबकि आकाश फट जाएगा
\end{hindi}}
\flushright{\begin{Arabic}
\quranayah[82][2]
\end{Arabic}}
\flushleft{\begin{hindi}
और जबकि तारे बिखर जाएँगे
\end{hindi}}
\flushright{\begin{Arabic}
\quranayah[82][3]
\end{Arabic}}
\flushleft{\begin{hindi}
और जबकि समुद्र बह पड़ेंगे
\end{hindi}}
\flushright{\begin{Arabic}
\quranayah[82][4]
\end{Arabic}}
\flushleft{\begin{hindi}
और जबकि क़बें उखेड़ दी जाएँगी
\end{hindi}}
\flushright{\begin{Arabic}
\quranayah[82][5]
\end{Arabic}}
\flushleft{\begin{hindi}
तब हर व्यक्ति जान लेगा जिसे उसने प्राथमिकता दी और पीछे डाला
\end{hindi}}
\flushright{\begin{Arabic}
\quranayah[82][6]
\end{Arabic}}
\flushleft{\begin{hindi}
ऐ मनुष्य! किस चीज़ ने तुझे अपने उदार प्रभु के विषय में धोखे में डाल रखा हैं?
\end{hindi}}
\flushright{\begin{Arabic}
\quranayah[82][7]
\end{Arabic}}
\flushleft{\begin{hindi}
जिसने तेरा प्रारूप बनाया, फिर नख-शिख से तुझे दुरुस्त किया और तुझे संतुलन प्रदान किया
\end{hindi}}
\flushright{\begin{Arabic}
\quranayah[82][8]
\end{Arabic}}
\flushleft{\begin{hindi}
जिस रूप में चाहा उसने तुझे जोड़कर तैयार किया
\end{hindi}}
\flushright{\begin{Arabic}
\quranayah[82][9]
\end{Arabic}}
\flushleft{\begin{hindi}
कुछ नहीं, बल्कि तुम बदला दिए जाने का झुठलाते हो
\end{hindi}}
\flushright{\begin{Arabic}
\quranayah[82][10]
\end{Arabic}}
\flushleft{\begin{hindi}
जबकि तुमपर निगरानी करनेवाले नियुक्त हैं
\end{hindi}}
\flushright{\begin{Arabic}
\quranayah[82][11]
\end{Arabic}}
\flushleft{\begin{hindi}
प्रतिष्ठित लिपिक
\end{hindi}}
\flushright{\begin{Arabic}
\quranayah[82][12]
\end{Arabic}}
\flushleft{\begin{hindi}
वे जान रहे होते है जो कुछ भी तुम लोग करते हो
\end{hindi}}
\flushright{\begin{Arabic}
\quranayah[82][13]
\end{Arabic}}
\flushleft{\begin{hindi}
निस्संदेह वफ़ादार लोग नेमतों में होंगे
\end{hindi}}
\flushright{\begin{Arabic}
\quranayah[82][14]
\end{Arabic}}
\flushleft{\begin{hindi}
और निश्चय ही दुराचारी भड़कती हुई आग में
\end{hindi}}
\flushright{\begin{Arabic}
\quranayah[82][15]
\end{Arabic}}
\flushleft{\begin{hindi}
जिसमें वे बदले के दिन प्रवेश करेंगे
\end{hindi}}
\flushright{\begin{Arabic}
\quranayah[82][16]
\end{Arabic}}
\flushleft{\begin{hindi}
और उससे वे ओझल नहीं होंगे
\end{hindi}}
\flushright{\begin{Arabic}
\quranayah[82][17]
\end{Arabic}}
\flushleft{\begin{hindi}
और तुम्हें क्या मालूम कि बदले का दिन क्या है?
\end{hindi}}
\flushright{\begin{Arabic}
\quranayah[82][18]
\end{Arabic}}
\flushleft{\begin{hindi}
फिर तुम्हें क्या मालूम कि बदले का दिन क्या है?
\end{hindi}}
\flushright{\begin{Arabic}
\quranayah[82][19]
\end{Arabic}}
\flushleft{\begin{hindi}
जिस दिन कोई व्यक्ति किसी व्यक्ति के लिए किसी चीज़ का अधिकारी न होगा, मामला उस दिन अल्लाह ही के हाथ में होगा
\end{hindi}}
\chapter{At-Tatfif (Default in Duty)}
\begin{Arabic}
\Huge{\centerline{\basmalah}}\end{Arabic}
\flushright{\begin{Arabic}
\quranayah[83][1]
\end{Arabic}}
\flushleft{\begin{hindi}
तबाही है घटानेवालों के लिए,
\end{hindi}}
\flushright{\begin{Arabic}
\quranayah[83][2]
\end{Arabic}}
\flushleft{\begin{hindi}
जो नापकर लोगों पर नज़र जमाए हुए लेते हैं तो पूरा-पूरा लेते हैं,
\end{hindi}}
\flushright{\begin{Arabic}
\quranayah[83][3]
\end{Arabic}}
\flushleft{\begin{hindi}
किन्तु जब उन्हें नापकर या तौलकर देते हैं तो घटाकर देते हैं
\end{hindi}}
\flushright{\begin{Arabic}
\quranayah[83][4]
\end{Arabic}}
\flushleft{\begin{hindi}
क्या वे समझते नहीं कि उन्हें (जीवित होकर) उठना है,
\end{hindi}}
\flushright{\begin{Arabic}
\quranayah[83][5]
\end{Arabic}}
\flushleft{\begin{hindi}
एक भारी दिन के लिए,
\end{hindi}}
\flushright{\begin{Arabic}
\quranayah[83][6]
\end{Arabic}}
\flushleft{\begin{hindi}
जिस दिन लोग सारे संसार के रब के सामने खड़े होंगे?
\end{hindi}}
\flushright{\begin{Arabic}
\quranayah[83][7]
\end{Arabic}}
\flushleft{\begin{hindi}
कुछ नहीं, निश्चय ही दुराचारियों का काग़ज 'सिज्जीन' में है
\end{hindi}}
\flushright{\begin{Arabic}
\quranayah[83][8]
\end{Arabic}}
\flushleft{\begin{hindi}
तुम्हें क्या मालूम कि 'सिज्जीन' क्या हैं?
\end{hindi}}
\flushright{\begin{Arabic}
\quranayah[83][9]
\end{Arabic}}
\flushleft{\begin{hindi}
मुहर लगा हुआ काग़ज
\end{hindi}}
\flushright{\begin{Arabic}
\quranayah[83][10]
\end{Arabic}}
\flushleft{\begin{hindi}
तबाही है उस दिन झुठलाने-वालों की,
\end{hindi}}
\flushright{\begin{Arabic}
\quranayah[83][11]
\end{Arabic}}
\flushleft{\begin{hindi}
जो बदले के दिन को झुठलाते है
\end{hindi}}
\flushright{\begin{Arabic}
\quranayah[83][12]
\end{Arabic}}
\flushleft{\begin{hindi}
और उसे तो बस प्रत्येक वह क्यक्ति ही झूठलाता है जो सीमा का उल्लंघन करनेवाला, पापी है
\end{hindi}}
\flushright{\begin{Arabic}
\quranayah[83][13]
\end{Arabic}}
\flushleft{\begin{hindi}
जब हमारी आयतें उसे सुनाई जाती है तो कहता है, "ये तो पहले की कहानियाँ है।"
\end{hindi}}
\flushright{\begin{Arabic}
\quranayah[83][14]
\end{Arabic}}
\flushleft{\begin{hindi}
कुछ नहीं, बल्कि जो कुछ वे कमाते रहे है वह उनके दिलों पर चढ़ गया है
\end{hindi}}
\flushright{\begin{Arabic}
\quranayah[83][15]
\end{Arabic}}
\flushleft{\begin{hindi}
कुछ नहीं, अवश्य ही वे उस दिन अपने रब से ओट में होंगे,
\end{hindi}}
\flushright{\begin{Arabic}
\quranayah[83][16]
\end{Arabic}}
\flushleft{\begin{hindi}
फिर वे भड़कती आग में जा पड़ेगे
\end{hindi}}
\flushright{\begin{Arabic}
\quranayah[83][17]
\end{Arabic}}
\flushleft{\begin{hindi}
फिर कहा जाएगा, "यह वही है जिस तुम झुठलाते थे"
\end{hindi}}
\flushright{\begin{Arabic}
\quranayah[83][18]
\end{Arabic}}
\flushleft{\begin{hindi}
कुछ नही, निस्संदेह वफ़ादार लोगों का काग़ज़ 'इल्लीयीन' (उच्च श्रेणी के लोगों) में है।-
\end{hindi}}
\flushright{\begin{Arabic}
\quranayah[83][19]
\end{Arabic}}
\flushleft{\begin{hindi}
और तुम क्या जानो कि 'इल्लीयीन' क्या है? -
\end{hindi}}
\flushright{\begin{Arabic}
\quranayah[83][20]
\end{Arabic}}
\flushleft{\begin{hindi}
लिखा हुआ रजिस्टर
\end{hindi}}
\flushright{\begin{Arabic}
\quranayah[83][21]
\end{Arabic}}
\flushleft{\begin{hindi}
जिसे देखने के लिए सामीप्य प्राप्त लोग उपस्थित होंगे,
\end{hindi}}
\flushright{\begin{Arabic}
\quranayah[83][22]
\end{Arabic}}
\flushleft{\begin{hindi}
निस्संदेह अच्छे लोग नेमतों में होंगे,
\end{hindi}}
\flushright{\begin{Arabic}
\quranayah[83][23]
\end{Arabic}}
\flushleft{\begin{hindi}
ऊँची मसनदों पर से देख रहे होंगे
\end{hindi}}
\flushright{\begin{Arabic}
\quranayah[83][24]
\end{Arabic}}
\flushleft{\begin{hindi}
उनके चहरों से तुम्हें नेमतों की ताज़गी और आभा को बोध हो रहा होगा,
\end{hindi}}
\flushright{\begin{Arabic}
\quranayah[83][25]
\end{Arabic}}
\flushleft{\begin{hindi}
उन्हें मुहरबंद विशुद्ध पेय पिलाया जाएगा,
\end{hindi}}
\flushright{\begin{Arabic}
\quranayah[83][26]
\end{Arabic}}
\flushleft{\begin{hindi}
मुहर उसकी मुश्क ही होगी - जो लोग दूसरी पर बाज़ी ले जाना चाहते हो वे इस चीज़ को प्राप्त करने में बाज़ी ले जाने का प्रयास करे -
\end{hindi}}
\flushright{\begin{Arabic}
\quranayah[83][27]
\end{Arabic}}
\flushleft{\begin{hindi}
और उसमें 'तसनीम' का मिश्रण होगा,
\end{hindi}}
\flushright{\begin{Arabic}
\quranayah[83][28]
\end{Arabic}}
\flushleft{\begin{hindi}
हाल यह है कि वह एक स्रोत है, जिसपर बैठकर सामीप्य प्राप्त लोग पिएँगे
\end{hindi}}
\flushright{\begin{Arabic}
\quranayah[83][29]
\end{Arabic}}
\flushleft{\begin{hindi}
जो अपराधी है वे ईमान लानेवालों पर हँसते थे,
\end{hindi}}
\flushright{\begin{Arabic}
\quranayah[83][30]
\end{Arabic}}
\flushleft{\begin{hindi}
और जब उनके पास से गुज़रते तो आपस में आँखों और भौंहों से इशारे करते थे,
\end{hindi}}
\flushright{\begin{Arabic}
\quranayah[83][31]
\end{Arabic}}
\flushleft{\begin{hindi}
और जब अपने लोगों की ओर पलटते है तो चहकते, इतराते हुए पलटते थे,
\end{hindi}}
\flushright{\begin{Arabic}
\quranayah[83][32]
\end{Arabic}}
\flushleft{\begin{hindi}
और जब उन्हें देखते तो कहते, "ये तो भटके हुए है।"
\end{hindi}}
\flushright{\begin{Arabic}
\quranayah[83][33]
\end{Arabic}}
\flushleft{\begin{hindi}
हालाँकि वे उनपर कोई निगरानी करनेवाले बनाकर नहीं भेजे गए थे
\end{hindi}}
\flushright{\begin{Arabic}
\quranayah[83][34]
\end{Arabic}}
\flushleft{\begin{hindi}
तो आज ईमान लानेवाले, इनकार करनेवालों पर हँस रहे हैं,
\end{hindi}}
\flushright{\begin{Arabic}
\quranayah[83][35]
\end{Arabic}}
\flushleft{\begin{hindi}
ऊँची मसनदों पर से देख रहे है
\end{hindi}}
\flushright{\begin{Arabic}
\quranayah[83][36]
\end{Arabic}}
\flushleft{\begin{hindi}
क्या मिल गया बदला इनकार करनेवालों को उसका जो कुछ वे करते रहे है?
\end{hindi}}
\chapter{Al-Inshiqaq (The Bursting Asunder)}
\begin{Arabic}
\Huge{\centerline{\basmalah}}\end{Arabic}
\flushright{\begin{Arabic}
\quranayah[84][1]
\end{Arabic}}
\flushleft{\begin{hindi}
जबकि आकाश फट जाएगा,
\end{hindi}}
\flushright{\begin{Arabic}
\quranayah[84][2]
\end{Arabic}}
\flushleft{\begin{hindi}
और वह अपने रब की सुनेगा, और उसे यही चाहिए भी
\end{hindi}}
\flushright{\begin{Arabic}
\quranayah[84][3]
\end{Arabic}}
\flushleft{\begin{hindi}
जब धरती फैला दी जाएगी
\end{hindi}}
\flushright{\begin{Arabic}
\quranayah[84][4]
\end{Arabic}}
\flushleft{\begin{hindi}
और जो कुछ उसके भीतर है उसे बाहर डालकर खाली हो जाएगी
\end{hindi}}
\flushright{\begin{Arabic}
\quranayah[84][5]
\end{Arabic}}
\flushleft{\begin{hindi}
और वह अपने रब की सुनेगी, और उसे यही चाहिए भी
\end{hindi}}
\flushright{\begin{Arabic}
\quranayah[84][6]
\end{Arabic}}
\flushleft{\begin{hindi}
ऐ मनुष्य! तू मशक़्क़त करता हुआ अपने रब ही की ओर खिंचा चला जा रहा है और अन्ततः उससे मिलने वाला है
\end{hindi}}
\flushright{\begin{Arabic}
\quranayah[84][7]
\end{Arabic}}
\flushleft{\begin{hindi}
फिर जिस किसी को उसका कर्म-पत्र उसके दाहिने हाथ में दिया गया,
\end{hindi}}
\flushright{\begin{Arabic}
\quranayah[84][8]
\end{Arabic}}
\flushleft{\begin{hindi}
तो उससे आसान, सरसरी हिसाब लिया जाएगा
\end{hindi}}
\flushright{\begin{Arabic}
\quranayah[84][9]
\end{Arabic}}
\flushleft{\begin{hindi}
और वह अपने लोगों की ओर ख़ुश-ख़ुश पलटेगा
\end{hindi}}
\flushright{\begin{Arabic}
\quranayah[84][10]
\end{Arabic}}
\flushleft{\begin{hindi}
और रह वह व्यक्ति जिसका कर्म-पत्र (उसके बाएँ हाथ में) उसकी पीठ के पीछे से दिया गया,
\end{hindi}}
\flushright{\begin{Arabic}
\quranayah[84][11]
\end{Arabic}}
\flushleft{\begin{hindi}
तो वह विनाश (मृत्यु) को पुकारेगा,
\end{hindi}}
\flushright{\begin{Arabic}
\quranayah[84][12]
\end{Arabic}}
\flushleft{\begin{hindi}
और दहकती आग में जा पड़ेगा
\end{hindi}}
\flushright{\begin{Arabic}
\quranayah[84][13]
\end{Arabic}}
\flushleft{\begin{hindi}
वह अपने लोगों में मग्न था,
\end{hindi}}
\flushright{\begin{Arabic}
\quranayah[84][14]
\end{Arabic}}
\flushleft{\begin{hindi}
उसने यह समझ रखा था कि उसे कभी पलटना नहीं है
\end{hindi}}
\flushright{\begin{Arabic}
\quranayah[84][15]
\end{Arabic}}
\flushleft{\begin{hindi}
क्यों नहीं, निश्चय ही उसका रब तो उसे देख रहा था!
\end{hindi}}
\flushright{\begin{Arabic}
\quranayah[84][16]
\end{Arabic}}
\flushleft{\begin{hindi}
अतः कुछ नहीं, मैं क़सम खाता हूँ सांध्य-लालिमा की,
\end{hindi}}
\flushright{\begin{Arabic}
\quranayah[84][17]
\end{Arabic}}
\flushleft{\begin{hindi}
और रात की और उसके समेट लेने की,
\end{hindi}}
\flushright{\begin{Arabic}
\quranayah[84][18]
\end{Arabic}}
\flushleft{\begin{hindi}
और चन्द्रमा की जबकि वह पूर्ण हो जाता है,
\end{hindi}}
\flushright{\begin{Arabic}
\quranayah[84][19]
\end{Arabic}}
\flushleft{\begin{hindi}
निश्चय ही तुम्हें मंजिल पर मंजिल चढ़ना है
\end{hindi}}
\flushright{\begin{Arabic}
\quranayah[84][20]
\end{Arabic}}
\flushleft{\begin{hindi}
फिर उन्हें क्या हो गया है कि ईमान नहीं लाते?
\end{hindi}}
\flushright{\begin{Arabic}
\quranayah[84][21]
\end{Arabic}}
\flushleft{\begin{hindi}
और जब उन्हें कुरआन पढ़कर सुनाया जाता है तो सजदे में नहीं गिर पड़ते?
\end{hindi}}
\flushright{\begin{Arabic}
\quranayah[84][22]
\end{Arabic}}
\flushleft{\begin{hindi}
नहीं, बल्कि इनकार करनेवाले तो झुठलाते है,
\end{hindi}}
\flushright{\begin{Arabic}
\quranayah[84][23]
\end{Arabic}}
\flushleft{\begin{hindi}
हालाँकि जो कुछ वे अपने अन्दर एकत्र कर रहे है, अल्लाह उसे भली-भाँति जानता है
\end{hindi}}
\flushright{\begin{Arabic}
\quranayah[84][24]
\end{Arabic}}
\flushleft{\begin{hindi}
अतः उन्हें दुखद यातना की मंगल सूचना दे दो
\end{hindi}}
\flushright{\begin{Arabic}
\quranayah[84][25]
\end{Arabic}}
\flushleft{\begin{hindi}
अलबत्ता जो लोग ईमान लाए और उन्होंने अच्छे कर्म किए उनके लिए कभी न समाप्त॥ होनेवाला प्रतिदान है
\end{hindi}}
\chapter{Al-Buruj (The Stars)}
\begin{Arabic}
\Huge{\centerline{\basmalah}}\end{Arabic}
\flushright{\begin{Arabic}
\quranayah[85][1]
\end{Arabic}}
\flushleft{\begin{hindi}
साक्षी है बुर्जोंवाला आकाश,
\end{hindi}}
\flushright{\begin{Arabic}
\quranayah[85][2]
\end{Arabic}}
\flushleft{\begin{hindi}
और वह दिन जिसका वादा किया गया है,
\end{hindi}}
\flushright{\begin{Arabic}
\quranayah[85][3]
\end{Arabic}}
\flushleft{\begin{hindi}
और देखनेवाला, और जो देखा गया
\end{hindi}}
\flushright{\begin{Arabic}
\quranayah[85][4]
\end{Arabic}}
\flushleft{\begin{hindi}
विनष्ट हों खाईवाले,
\end{hindi}}
\flushright{\begin{Arabic}
\quranayah[85][5]
\end{Arabic}}
\flushleft{\begin{hindi}
ईधन भरी आगवाले,
\end{hindi}}
\flushright{\begin{Arabic}
\quranayah[85][6]
\end{Arabic}}
\flushleft{\begin{hindi}
जबकि वे वहाँ बैठे होंगे
\end{hindi}}
\flushright{\begin{Arabic}
\quranayah[85][7]
\end{Arabic}}
\flushleft{\begin{hindi}
और वे जो कुछ ईमानवालों के साथ करते रहे, उसे देखेंगे
\end{hindi}}
\flushright{\begin{Arabic}
\quranayah[85][8]
\end{Arabic}}
\flushleft{\begin{hindi}
उन्होंने उन (ईमानवालों) से केवल इस कारण बदला लिया और शत्रुता की कि वे उस अल्लाह पर ईमान रखते थे जो अत्यन्त प्रभुत्वशाली, प्रशंसनीय है,
\end{hindi}}
\flushright{\begin{Arabic}
\quranayah[85][9]
\end{Arabic}}
\flushleft{\begin{hindi}
जिसके लिए आकाशों और धरती की बादशाही है। और अल्लाह हर चीज़ का साक्षी है
\end{hindi}}
\flushright{\begin{Arabic}
\quranayah[85][10]
\end{Arabic}}
\flushleft{\begin{hindi}
जिन लोगों ने ईमानवाले पुरुषों और ईमानवाली स्त्रियों को सताया और आज़माईश में डाला, फिर तौबा न की, निश्चय ही उनके लिए जहन्नम की यातना है और उनके लिए जलने की यातना है
\end{hindi}}
\flushright{\begin{Arabic}
\quranayah[85][11]
\end{Arabic}}
\flushleft{\begin{hindi}
निश्चय ही जो लोग ईमान लाए और उन्होंने अच्छे कर्म किए उनके लिए बाग़ है, जिनके नीचे नहरें बह रही होगी। वही है बड़ी सफलता
\end{hindi}}
\flushright{\begin{Arabic}
\quranayah[85][12]
\end{Arabic}}
\flushleft{\begin{hindi}
वास्तव में तुम्हारे रब की पकड़ बड़ी ही सख़्त है
\end{hindi}}
\flushright{\begin{Arabic}
\quranayah[85][13]
\end{Arabic}}
\flushleft{\begin{hindi}
वही आरम्भ करता है और वही पुनरावृत्ति करता है,
\end{hindi}}
\flushright{\begin{Arabic}
\quranayah[85][14]
\end{Arabic}}
\flushleft{\begin{hindi}
वह बड़ा क्षमाशील, बहुत प्रेम करनेवाला है,
\end{hindi}}
\flushright{\begin{Arabic}
\quranayah[85][15]
\end{Arabic}}
\flushleft{\begin{hindi}
सिंहासन का स्वामी है, बडा गौरवशाली,
\end{hindi}}
\flushright{\begin{Arabic}
\quranayah[85][16]
\end{Arabic}}
\flushleft{\begin{hindi}
जो चाहे उसे कर डालनेवाला
\end{hindi}}
\flushright{\begin{Arabic}
\quranayah[85][17]
\end{Arabic}}
\flushleft{\begin{hindi}
क्या तुम्हें उन सेनाओं की भी ख़बर पहुँची हैं,
\end{hindi}}
\flushright{\begin{Arabic}
\quranayah[85][18]
\end{Arabic}}
\flushleft{\begin{hindi}
फ़िरऔन और समूद की?
\end{hindi}}
\flushright{\begin{Arabic}
\quranayah[85][19]
\end{Arabic}}
\flushleft{\begin{hindi}
नहीं, बल्कि जिन लोगों ने इनकार किया है, वे झुठलाने में लगे हुए है;
\end{hindi}}
\flushright{\begin{Arabic}
\quranayah[85][20]
\end{Arabic}}
\flushleft{\begin{hindi}
हालाँकि अल्लाह उन्हें घेरे हुए है, उनके आगे-पीछे से
\end{hindi}}
\flushright{\begin{Arabic}
\quranayah[85][21]
\end{Arabic}}
\flushleft{\begin{hindi}
नहीं, बल्कि वह तो गौरव क़ुरआन है,
\end{hindi}}
\flushright{\begin{Arabic}
\quranayah[85][22]
\end{Arabic}}
\flushleft{\begin{hindi}
सुरक्षित पट्टिका में अंकित है
\end{hindi}}
\chapter{At-Tariq (The Comer by Night)}
\begin{Arabic}
\Huge{\centerline{\basmalah}}\end{Arabic}
\flushright{\begin{Arabic}
\quranayah[86][1]
\end{Arabic}}
\flushleft{\begin{hindi}
साक्षी है आकाश, और रात में प्रकट होनेवाला -
\end{hindi}}
\flushright{\begin{Arabic}
\quranayah[86][2]
\end{Arabic}}
\flushleft{\begin{hindi}
और तुम क्या जानो कि रात में प्रकट होनेवाला क्या है?
\end{hindi}}
\flushright{\begin{Arabic}
\quranayah[86][3]
\end{Arabic}}
\flushleft{\begin{hindi}
दमकता हुआ तारा! -
\end{hindi}}
\flushright{\begin{Arabic}
\quranayah[86][4]
\end{Arabic}}
\flushleft{\begin{hindi}
कि हर एक व्यक्ति पर एक निगरानी करनेवाला नियुक्त है
\end{hindi}}
\flushright{\begin{Arabic}
\quranayah[86][5]
\end{Arabic}}
\flushleft{\begin{hindi}
अतः मनुष्य को चाहिए कि देखे कि वह किस चीज़ से पैदा किया गया है
\end{hindi}}
\flushright{\begin{Arabic}
\quranayah[86][6]
\end{Arabic}}
\flushleft{\begin{hindi}
एक उछलते पानी से पैदा किया गया है,
\end{hindi}}
\flushright{\begin{Arabic}
\quranayah[86][7]
\end{Arabic}}
\flushleft{\begin{hindi}
जो पीठ और पसलियों के मध्य से निकलता है
\end{hindi}}
\flushright{\begin{Arabic}
\quranayah[86][8]
\end{Arabic}}
\flushleft{\begin{hindi}
निश्चय ही वह उसके लौटा देने की सामर्थ्य रखता है
\end{hindi}}
\flushright{\begin{Arabic}
\quranayah[86][9]
\end{Arabic}}
\flushleft{\begin{hindi}
जिस दिन छिपी चीज़ें परखी जाएँगी,
\end{hindi}}
\flushright{\begin{Arabic}
\quranayah[86][10]
\end{Arabic}}
\flushleft{\begin{hindi}
तो उस समय उसके पास न तो अपनी कोई शक्ति होगी और न कोई सहायक
\end{hindi}}
\flushright{\begin{Arabic}
\quranayah[86][11]
\end{Arabic}}
\flushleft{\begin{hindi}
साक्षी है आवर्तन (उलट-फेर) वाला आकाश,
\end{hindi}}
\flushright{\begin{Arabic}
\quranayah[86][12]
\end{Arabic}}
\flushleft{\begin{hindi}
और फट जानेवाली धरती
\end{hindi}}
\flushright{\begin{Arabic}
\quranayah[86][13]
\end{Arabic}}
\flushleft{\begin{hindi}
वह दो-टूक बात है,
\end{hindi}}
\flushright{\begin{Arabic}
\quranayah[86][14]
\end{Arabic}}
\flushleft{\begin{hindi}
वह कोई हँसी-मज़ाक नही है
\end{hindi}}
\flushright{\begin{Arabic}
\quranayah[86][15]
\end{Arabic}}
\flushleft{\begin{hindi}
वे एक चाल चल रहे है,
\end{hindi}}
\flushright{\begin{Arabic}
\quranayah[86][16]
\end{Arabic}}
\flushleft{\begin{hindi}
और मैं भी एक चाल चल रहा हूँ
\end{hindi}}
\flushright{\begin{Arabic}
\quranayah[86][17]
\end{Arabic}}
\flushleft{\begin{hindi}
अत मुहलत दे दो उन इनकार करनेवालों को; मुहलत दे दो उन्हें थोड़ी-सी
\end{hindi}}
\chapter{Al-A'la (The Most High)}
\begin{Arabic}
\Huge{\centerline{\basmalah}}\end{Arabic}
\flushright{\begin{Arabic}
\quranayah[87][1]
\end{Arabic}}
\flushleft{\begin{hindi}
तसबीह करो, अपने सर्वाच्च रब के नाम की,
\end{hindi}}
\flushright{\begin{Arabic}
\quranayah[87][2]
\end{Arabic}}
\flushleft{\begin{hindi}
जिसने पैदा किया, फिर ठीक-ठाक किया,
\end{hindi}}
\flushright{\begin{Arabic}
\quranayah[87][3]
\end{Arabic}}
\flushleft{\begin{hindi}
जिसने निर्धारित किया, फिर मार्ग दिखाया,
\end{hindi}}
\flushright{\begin{Arabic}
\quranayah[87][4]
\end{Arabic}}
\flushleft{\begin{hindi}
जिसने वनस्पति उगाई,
\end{hindi}}
\flushright{\begin{Arabic}
\quranayah[87][5]
\end{Arabic}}
\flushleft{\begin{hindi}
फिर उसे ख़ूब घना और हरा-भरा कर दिया
\end{hindi}}
\flushright{\begin{Arabic}
\quranayah[87][6]
\end{Arabic}}
\flushleft{\begin{hindi}
हम तुम्हें पढ़ा देंगे, फिर तुम भूलोगे नहीं
\end{hindi}}
\flushright{\begin{Arabic}
\quranayah[87][7]
\end{Arabic}}
\flushleft{\begin{hindi}
बात यह है कि अल्लाह की इच्छा ही क्रियान्वित है। निश्चय ही वह जानता है खुले को भी और उसे भी जो छिपा रहे
\end{hindi}}
\flushright{\begin{Arabic}
\quranayah[87][8]
\end{Arabic}}
\flushleft{\begin{hindi}
हम तुम्हें सहज ढंग से उस चीज़ की पात्र बना देंगे जो सहज एवं मृदुल (आरामदायक) है
\end{hindi}}
\flushright{\begin{Arabic}
\quranayah[87][9]
\end{Arabic}}
\flushleft{\begin{hindi}
अतः नसीहत करो, यदि नसीहत लाभप्रद हो!
\end{hindi}}
\flushright{\begin{Arabic}
\quranayah[87][10]
\end{Arabic}}
\flushleft{\begin{hindi}
नसीहत हासिल कर लेगा जिसको डर होगा,
\end{hindi}}
\flushright{\begin{Arabic}
\quranayah[87][11]
\end{Arabic}}
\flushleft{\begin{hindi}
किन्तु उससे कतराएगा वह अत्यन्त दुर्भाग्यवाला,
\end{hindi}}
\flushright{\begin{Arabic}
\quranayah[87][12]
\end{Arabic}}
\flushleft{\begin{hindi}
जो बड़ी आग में पड़ेगा,
\end{hindi}}
\flushright{\begin{Arabic}
\quranayah[87][13]
\end{Arabic}}
\flushleft{\begin{hindi}
फिर वह उसमें न मरेगा न जिएगा
\end{hindi}}
\flushright{\begin{Arabic}
\quranayah[87][14]
\end{Arabic}}
\flushleft{\begin{hindi}
सफल हो गया वह जिसने अपने आपको निखार लिया,
\end{hindi}}
\flushright{\begin{Arabic}
\quranayah[87][15]
\end{Arabic}}
\flushleft{\begin{hindi}
और अपने रब के नाम का स्मरण किया, अतः नमाज़ अदा की
\end{hindi}}
\flushright{\begin{Arabic}
\quranayah[87][16]
\end{Arabic}}
\flushleft{\begin{hindi}
नहीं, बल्कि तुम तो सांसारिक जीवन को प्राथमिकता देते हो,
\end{hindi}}
\flushright{\begin{Arabic}
\quranayah[87][17]
\end{Arabic}}
\flushleft{\begin{hindi}
हालाँकि आख़िरत अधिक उत्तम और शेष रहनेवाली है
\end{hindi}}
\flushright{\begin{Arabic}
\quranayah[87][18]
\end{Arabic}}
\flushleft{\begin{hindi}
निस्संदेह यही बात पहले की किताबों में भी है;
\end{hindi}}
\flushright{\begin{Arabic}
\quranayah[87][19]
\end{Arabic}}
\flushleft{\begin{hindi}
इबराईम और मूसा की किताबों में
\end{hindi}}
\chapter{Al-Ghashiyah (The Overwhelming Event)}
\begin{Arabic}
\Huge{\centerline{\basmalah}}\end{Arabic}
\flushright{\begin{Arabic}
\quranayah[88][1]
\end{Arabic}}
\flushleft{\begin{hindi}
क्या तुम्हें उस छा जानेवाली की ख़बर पहुँची है?
\end{hindi}}
\flushright{\begin{Arabic}
\quranayah[88][2]
\end{Arabic}}
\flushleft{\begin{hindi}
उस दिन कितने ही चेहरे गिरे हुए होंगे,
\end{hindi}}
\flushright{\begin{Arabic}
\quranayah[88][3]
\end{Arabic}}
\flushleft{\begin{hindi}
कठिन परिश्रम में पड़े, थके-हारे
\end{hindi}}
\flushright{\begin{Arabic}
\quranayah[88][4]
\end{Arabic}}
\flushleft{\begin{hindi}
दहकती आग में प्रवेश करेंगे
\end{hindi}}
\flushright{\begin{Arabic}
\quranayah[88][5]
\end{Arabic}}
\flushleft{\begin{hindi}
खौलते हुए स्रोत से पिएँगे,
\end{hindi}}
\flushright{\begin{Arabic}
\quranayah[88][6]
\end{Arabic}}
\flushleft{\begin{hindi}
उनके लिए कोई खाना न होगा सिवाय एक प्रकार के ज़री के,
\end{hindi}}
\flushright{\begin{Arabic}
\quranayah[88][7]
\end{Arabic}}
\flushleft{\begin{hindi}
जो न पुष्ट करे और न भूख मिटाए
\end{hindi}}
\flushright{\begin{Arabic}
\quranayah[88][8]
\end{Arabic}}
\flushleft{\begin{hindi}
उस दिन कितने ही चेहरे प्रफुल्लित और सौम्य होंगे,
\end{hindi}}
\flushright{\begin{Arabic}
\quranayah[88][9]
\end{Arabic}}
\flushleft{\begin{hindi}
अपने प्रयास पर प्रसन्न,
\end{hindi}}
\flushright{\begin{Arabic}
\quranayah[88][10]
\end{Arabic}}
\flushleft{\begin{hindi}
उच्च जन्नत में,
\end{hindi}}
\flushright{\begin{Arabic}
\quranayah[88][11]
\end{Arabic}}
\flushleft{\begin{hindi}
जिसमें कोई व्यर्थ बात न सुनेंगे
\end{hindi}}
\flushright{\begin{Arabic}
\quranayah[88][12]
\end{Arabic}}
\flushleft{\begin{hindi}
उसमें स्रोत प्रवाहित होगा,
\end{hindi}}
\flushright{\begin{Arabic}
\quranayah[88][13]
\end{Arabic}}
\flushleft{\begin{hindi}
उसमें ऊँची-ऊँची मसनदें होगी,
\end{hindi}}
\flushright{\begin{Arabic}
\quranayah[88][14]
\end{Arabic}}
\flushleft{\begin{hindi}
प्याले ढंग से रखे होंगे,
\end{hindi}}
\flushright{\begin{Arabic}
\quranayah[88][15]
\end{Arabic}}
\flushleft{\begin{hindi}
क्रम से गाव तकिए लगे होंगे,
\end{hindi}}
\flushright{\begin{Arabic}
\quranayah[88][16]
\end{Arabic}}
\flushleft{\begin{hindi}
और हर ओर क़ालीने बिछी होंगी
\end{hindi}}
\flushright{\begin{Arabic}
\quranayah[88][17]
\end{Arabic}}
\flushleft{\begin{hindi}
फिर क्या वे ऊँट की ओर नहीं देखते कि कैसा बनाया गया?
\end{hindi}}
\flushright{\begin{Arabic}
\quranayah[88][18]
\end{Arabic}}
\flushleft{\begin{hindi}
और आकाश की ओर कि कैसा ऊँचा किया गया?
\end{hindi}}
\flushright{\begin{Arabic}
\quranayah[88][19]
\end{Arabic}}
\flushleft{\begin{hindi}
और पहाड़ो की ओर कि कैसे खड़े किए गए?
\end{hindi}}
\flushright{\begin{Arabic}
\quranayah[88][20]
\end{Arabic}}
\flushleft{\begin{hindi}
और धरती की ओर कि कैसी बिछाई गई?
\end{hindi}}
\flushright{\begin{Arabic}
\quranayah[88][21]
\end{Arabic}}
\flushleft{\begin{hindi}
अच्छा तो नसीहत करो! तुम तो बस एक नसीहत करनेवाले हो
\end{hindi}}
\flushright{\begin{Arabic}
\quranayah[88][22]
\end{Arabic}}
\flushleft{\begin{hindi}
तुम उनपर कोई दरोग़ा नही हो
\end{hindi}}
\flushright{\begin{Arabic}
\quranayah[88][23]
\end{Arabic}}
\flushleft{\begin{hindi}
किन्तु जिस किसी ने मुँह फेरा और इनकार किया,
\end{hindi}}
\flushright{\begin{Arabic}
\quranayah[88][24]
\end{Arabic}}
\flushleft{\begin{hindi}
तो अल्लाह उसे बड़ी यातना देगा
\end{hindi}}
\flushright{\begin{Arabic}
\quranayah[88][25]
\end{Arabic}}
\flushleft{\begin{hindi}
निस्संदेह हमारी ओर ही है उनका लौटना,
\end{hindi}}
\flushright{\begin{Arabic}
\quranayah[88][26]
\end{Arabic}}
\flushleft{\begin{hindi}
फिर हमारे ही ज़िम्मे है उनका हिसाब लेना
\end{hindi}}
\chapter{Al-Fajr (The Daybreak)}
\begin{Arabic}
\Huge{\centerline{\basmalah}}\end{Arabic}
\flushright{\begin{Arabic}
\quranayah[89][1]
\end{Arabic}}
\flushleft{\begin{hindi}
साक्षी है उषाकाल,
\end{hindi}}
\flushright{\begin{Arabic}
\quranayah[89][2]
\end{Arabic}}
\flushleft{\begin{hindi}
साक्षी है दस रातें,
\end{hindi}}
\flushright{\begin{Arabic}
\quranayah[89][3]
\end{Arabic}}
\flushleft{\begin{hindi}
साक्षी है युग्म और अयुग्म,
\end{hindi}}
\flushright{\begin{Arabic}
\quranayah[89][4]
\end{Arabic}}
\flushleft{\begin{hindi}
साक्षी है रात जब वह विदा हो रही हो
\end{hindi}}
\flushright{\begin{Arabic}
\quranayah[89][5]
\end{Arabic}}
\flushleft{\begin{hindi}
क्या इसमें बुद्धिमान के लिए बड़ी गवाही है?
\end{hindi}}
\flushright{\begin{Arabic}
\quranayah[89][6]
\end{Arabic}}
\flushleft{\begin{hindi}
क्या तुमने देखा नहीं कि तुम्हारे रब ने क्या किया आद के साथ,
\end{hindi}}
\flushright{\begin{Arabic}
\quranayah[89][7]
\end{Arabic}}
\flushleft{\begin{hindi}
स्तम्भों वाले 'इरम' के साथ?
\end{hindi}}
\flushright{\begin{Arabic}
\quranayah[89][8]
\end{Arabic}}
\flushleft{\begin{hindi}
वे ऐसे थे जिनके सदृश बस्तियों में पैदा नहीं हुए
\end{hindi}}
\flushright{\begin{Arabic}
\quranayah[89][9]
\end{Arabic}}
\flushleft{\begin{hindi}
और समूद के साथ, जिन्होंने घाटी में चट्टाने तराशी थी,
\end{hindi}}
\flushright{\begin{Arabic}
\quranayah[89][10]
\end{Arabic}}
\flushleft{\begin{hindi}
और मेखोवाले फ़िरऔन के साथ?
\end{hindi}}
\flushright{\begin{Arabic}
\quranayah[89][11]
\end{Arabic}}
\flushleft{\begin{hindi}
वे लोग कि जिन्होंने देशो में सरकशी की,
\end{hindi}}
\flushright{\begin{Arabic}
\quranayah[89][12]
\end{Arabic}}
\flushleft{\begin{hindi}
और उनमें बहुत बिगाड़ पैदा किया
\end{hindi}}
\flushright{\begin{Arabic}
\quranayah[89][13]
\end{Arabic}}
\flushleft{\begin{hindi}
अततः तुम्हारे रब ने उनपर यातना का कोड़ा बरसा दिया
\end{hindi}}
\flushright{\begin{Arabic}
\quranayah[89][14]
\end{Arabic}}
\flushleft{\begin{hindi}
निस्संदेह तुम्हारा रब घात में रहता है
\end{hindi}}
\flushright{\begin{Arabic}
\quranayah[89][15]
\end{Arabic}}
\flushleft{\begin{hindi}
किन्तु मनुष्य का हाल यह है कि जब उसका रब इस प्रकार उसकी परीक्षा करता है कि उसे प्रतिष्ठा और नेमत प्रदान करता है, तो वह कहता है, "मेरे रब ने मुझे प्रतिष्ठित किया।"
\end{hindi}}
\flushright{\begin{Arabic}
\quranayah[89][16]
\end{Arabic}}
\flushleft{\begin{hindi}
किन्तु जब कभी वह उसकी परीक्षा इस प्रकार करता है कि उसकी रोज़ी नपी-तुली कर देता है, तो वह कहता है, "मेरे रब ने मेरा अपमान किया।"
\end{hindi}}
\flushright{\begin{Arabic}
\quranayah[89][17]
\end{Arabic}}
\flushleft{\begin{hindi}
कदापि नहीं, बल्कि तुम अनाथ का सम्मान नहीं करते,
\end{hindi}}
\flushright{\begin{Arabic}
\quranayah[89][18]
\end{Arabic}}
\flushleft{\begin{hindi}
और न मुहताज को खिलान पर एक-दूसरे को उभारते हो,
\end{hindi}}
\flushright{\begin{Arabic}
\quranayah[89][19]
\end{Arabic}}
\flushleft{\begin{hindi}
और सारी मीरास समेटकर खा जाते हो,
\end{hindi}}
\flushright{\begin{Arabic}
\quranayah[89][20]
\end{Arabic}}
\flushleft{\begin{hindi}
और धन से उत्कट प्रेम रखते हो
\end{hindi}}
\flushright{\begin{Arabic}
\quranayah[89][21]
\end{Arabic}}
\flushleft{\begin{hindi}
कुछ नहीं, जब धरती कूट-कूटकर चुर्ण-विचुर्ण कर दी जाएगी,
\end{hindi}}
\flushright{\begin{Arabic}
\quranayah[89][22]
\end{Arabic}}
\flushleft{\begin{hindi}
और तुम्हारा रब और फ़रिश्ता (बन्दों की) एक-एक पंक्ति के पास आएगा,
\end{hindi}}
\flushright{\begin{Arabic}
\quranayah[89][23]
\end{Arabic}}
\flushleft{\begin{hindi}
और जहन्नम को उस दिन लाया जाएगा, उस दिन मनुष्य चेतेगा, किन्तु कहाँ है उसके लिए लाभप्रद उस समय का चेतना?
\end{hindi}}
\flushright{\begin{Arabic}
\quranayah[89][24]
\end{Arabic}}
\flushleft{\begin{hindi}
वह कहेगा, "ऐ काश! मैंने अपने जीवन के लिए कुछ करके आगे भेजा होता।"
\end{hindi}}
\flushright{\begin{Arabic}
\quranayah[89][25]
\end{Arabic}}
\flushleft{\begin{hindi}
फिर उस दिन कोई नहीं जो उसकी जैसी यातना दे,
\end{hindi}}
\flushright{\begin{Arabic}
\quranayah[89][26]
\end{Arabic}}
\flushleft{\begin{hindi}
और कोई नहीं जो उसकी जकड़बन्द की तरह बाँधे
\end{hindi}}
\flushright{\begin{Arabic}
\quranayah[89][27]
\end{Arabic}}
\flushleft{\begin{hindi}
"ऐ संतुष्ट आत्मा!
\end{hindi}}
\flushright{\begin{Arabic}
\quranayah[89][28]
\end{Arabic}}
\flushleft{\begin{hindi}
लौट अपने रब की ओर, इस तरह कि तू उससे राज़ी है वह तुझसे राज़ी है। अतः मेरे बन्दों में सम्मिलित हो जा। -
\end{hindi}}
\flushright{\begin{Arabic}
\quranayah[89][29]
\end{Arabic}}
\flushleft{\begin{hindi}
अतः मेरे बन्दों में सम्मिलित हो जा
\end{hindi}}
\flushright{\begin{Arabic}
\quranayah[89][30]
\end{Arabic}}
\flushleft{\begin{hindi}
और प्रवेश कर मेरी जन्नत में।"
\end{hindi}}
\chapter{Al-Balad (The City)}
\begin{Arabic}
\Huge{\centerline{\basmalah}}\end{Arabic}
\flushright{\begin{Arabic}
\quranayah[90][1]
\end{Arabic}}
\flushleft{\begin{hindi}
सुनो! मैं क़सम खाता हूँ इस नगर (मक्का) की -
\end{hindi}}
\flushright{\begin{Arabic}
\quranayah[90][2]
\end{Arabic}}
\flushleft{\begin{hindi}
हाल यह है कि तुम इसी नगर में रह रहे हो -
\end{hindi}}
\flushright{\begin{Arabic}
\quranayah[90][3]
\end{Arabic}}
\flushleft{\begin{hindi}
और बाप और उसकी सन्तान की,
\end{hindi}}
\flushright{\begin{Arabic}
\quranayah[90][4]
\end{Arabic}}
\flushleft{\begin{hindi}
निस्संदेह हमने मनुष्य को पूर्ण मशक़्क़त (अनुकूलता और सन्तुलन) के साथ पैदा किया
\end{hindi}}
\flushright{\begin{Arabic}
\quranayah[90][5]
\end{Arabic}}
\flushleft{\begin{hindi}
क्या वह समझता है कि उसपर किसी का बस न चलेगा?
\end{hindi}}
\flushright{\begin{Arabic}
\quranayah[90][6]
\end{Arabic}}
\flushleft{\begin{hindi}
कहता है कि "मैंने ढेरो माल उड़ा दिया।"
\end{hindi}}
\flushright{\begin{Arabic}
\quranayah[90][7]
\end{Arabic}}
\flushleft{\begin{hindi}
क्या वह समझता है कि किसी ने उसे देखा नहीं?
\end{hindi}}
\flushright{\begin{Arabic}
\quranayah[90][8]
\end{Arabic}}
\flushleft{\begin{hindi}
क्या हमने उसे नहीं दी दो आँखें,
\end{hindi}}
\flushright{\begin{Arabic}
\quranayah[90][9]
\end{Arabic}}
\flushleft{\begin{hindi}
और एक ज़बान और दो होंठ?
\end{hindi}}
\flushright{\begin{Arabic}
\quranayah[90][10]
\end{Arabic}}
\flushleft{\begin{hindi}
और क्या ऐसा नहीं है कि हमने दिखाई उसे दो ऊँचाइयाँ?
\end{hindi}}
\flushright{\begin{Arabic}
\quranayah[90][11]
\end{Arabic}}
\flushleft{\begin{hindi}
किन्तु वह तो हुमककर घाटी में से गुजंरा ही नहीं और (न उसने मुक्ति का मार्ग पाया)
\end{hindi}}
\flushright{\begin{Arabic}
\quranayah[90][12]
\end{Arabic}}
\flushleft{\begin{hindi}
और तुम्हें क्या मालूम कि वह घाटी क्या है!
\end{hindi}}
\flushright{\begin{Arabic}
\quranayah[90][13]
\end{Arabic}}
\flushleft{\begin{hindi}
किसी गरदन का छुड़ाना
\end{hindi}}
\flushright{\begin{Arabic}
\quranayah[90][14]
\end{Arabic}}
\flushleft{\begin{hindi}
या भूख के दिन खाना खिलाना
\end{hindi}}
\flushright{\begin{Arabic}
\quranayah[90][15]
\end{Arabic}}
\flushleft{\begin{hindi}
किसी निकटवर्ती अनाथ को,
\end{hindi}}
\flushright{\begin{Arabic}
\quranayah[90][16]
\end{Arabic}}
\flushleft{\begin{hindi}
या धूल-धूसरित मुहताज को;
\end{hindi}}
\flushright{\begin{Arabic}
\quranayah[90][17]
\end{Arabic}}
\flushleft{\begin{hindi}
फिर यह कि वह उन लोगों में से हो जो ईमान लाए और जिन्होंने एक-दूसरे को धैर्य की ताकीद की , और एक-दूसरे को दया की ताकीद की
\end{hindi}}
\flushright{\begin{Arabic}
\quranayah[90][18]
\end{Arabic}}
\flushleft{\begin{hindi}
वही लोग है सौभाग्यशाली
\end{hindi}}
\flushright{\begin{Arabic}
\quranayah[90][19]
\end{Arabic}}
\flushleft{\begin{hindi}
रहे वे लोग जिन्होंने हमारी आयातों का इनकार किया, वे दुर्भाग्यशाली लोग है
\end{hindi}}
\flushright{\begin{Arabic}
\quranayah[90][20]
\end{Arabic}}
\flushleft{\begin{hindi}
उनपर आग होगी, जिसे बन्द कर दिया गया होगा
\end{hindi}}
\chapter{Ash-Shams (The Sun)}
\begin{Arabic}
\Huge{\centerline{\basmalah}}\end{Arabic}
\flushright{\begin{Arabic}
\quranayah[91][1]
\end{Arabic}}
\flushleft{\begin{hindi}
साक्षी है सूर्य और उसकी प्रभा,
\end{hindi}}
\flushright{\begin{Arabic}
\quranayah[91][2]
\end{Arabic}}
\flushleft{\begin{hindi}
और चन्द्रमा जबकि वह उनके पीछे आए,
\end{hindi}}
\flushright{\begin{Arabic}
\quranayah[91][3]
\end{Arabic}}
\flushleft{\begin{hindi}
और दिन, जबकि वह उसे प्रकट कर दे,
\end{hindi}}
\flushright{\begin{Arabic}
\quranayah[91][4]
\end{Arabic}}
\flushleft{\begin{hindi}
और रात, जबकि वह उसको ढाँक ले
\end{hindi}}
\flushright{\begin{Arabic}
\quranayah[91][5]
\end{Arabic}}
\flushleft{\begin{hindi}
और आकाश और जैसा कुछ उसे उठाया,
\end{hindi}}
\flushright{\begin{Arabic}
\quranayah[91][6]
\end{Arabic}}
\flushleft{\begin{hindi}
और धरती और जैसा कुछ उसे बिछाया
\end{hindi}}
\flushright{\begin{Arabic}
\quranayah[91][7]
\end{Arabic}}
\flushleft{\begin{hindi}
और आत्मा और जैसा कुछ उसे सँवारा
\end{hindi}}
\flushright{\begin{Arabic}
\quranayah[91][8]
\end{Arabic}}
\flushleft{\begin{hindi}
फिर उसके दिल में डाली उसकी बुराई और उसकी परहेज़गारी
\end{hindi}}
\flushright{\begin{Arabic}
\quranayah[91][9]
\end{Arabic}}
\flushleft{\begin{hindi}
सफल हो गया जिसने उसे विकसित किया
\end{hindi}}
\flushright{\begin{Arabic}
\quranayah[91][10]
\end{Arabic}}
\flushleft{\begin{hindi}
और असफल हुआ जिसने उसे दबा दिया
\end{hindi}}
\flushright{\begin{Arabic}
\quranayah[91][11]
\end{Arabic}}
\flushleft{\begin{hindi}
समूद ने अपनी सरकशी से झुठलाया,
\end{hindi}}
\flushright{\begin{Arabic}
\quranayah[91][12]
\end{Arabic}}
\flushleft{\begin{hindi}
जब उनमें का सबसे बड़ा दुर्भाग्यशाली उठ खड़ा हुआ,
\end{hindi}}
\flushright{\begin{Arabic}
\quranayah[91][13]
\end{Arabic}}
\flushleft{\begin{hindi}
तो अल्लाह के रसूल ने उनसे कहा, "सावधान, अल्लाह की ऊँटनी और उसके पिलाने (की बारी) से।"
\end{hindi}}
\flushright{\begin{Arabic}
\quranayah[91][14]
\end{Arabic}}
\flushleft{\begin{hindi}
किन्तु उन्होंने उसे झुठलाया और उस ऊँटनी की कूचें काट डाली। अन्ततः उनके रब ने उनके गुनाह के कारण उनपर तबाही डाल दी और उन्हें बराबर कर दिया
\end{hindi}}
\flushright{\begin{Arabic}
\quranayah[91][15]
\end{Arabic}}
\flushleft{\begin{hindi}
और उसे उसके परिणाम का कोई भय नहीं
\end{hindi}}
\chapter{Al-Lail (The Night)}
\begin{Arabic}
\Huge{\centerline{\basmalah}}\end{Arabic}
\flushright{\begin{Arabic}
\quranayah[92][1]
\end{Arabic}}
\flushleft{\begin{hindi}
साक्षी है रात जबकि वह छा जाए,
\end{hindi}}
\flushright{\begin{Arabic}
\quranayah[92][2]
\end{Arabic}}
\flushleft{\begin{hindi}
और दिन जबकि वह प्रकाशमान हो,
\end{hindi}}
\flushright{\begin{Arabic}
\quranayah[92][3]
\end{Arabic}}
\flushleft{\begin{hindi}
और नर और मादा का पैदा करना,
\end{hindi}}
\flushright{\begin{Arabic}
\quranayah[92][4]
\end{Arabic}}
\flushleft{\begin{hindi}
कि तुम्हारा प्रयास भिन्न-भिन्न है
\end{hindi}}
\flushright{\begin{Arabic}
\quranayah[92][5]
\end{Arabic}}
\flushleft{\begin{hindi}
तो जिस किसी ने दिया और डर रखा,
\end{hindi}}
\flushright{\begin{Arabic}
\quranayah[92][6]
\end{Arabic}}
\flushleft{\begin{hindi}
और अच्छी चीज़ की पुष्टि की,
\end{hindi}}
\flushright{\begin{Arabic}
\quranayah[92][7]
\end{Arabic}}
\flushleft{\begin{hindi}
हम उस सहज ढंग से उस चीज का पात्र बना देंगे, जो सहज और मृदुल (सुख-साध्य) है
\end{hindi}}
\flushright{\begin{Arabic}
\quranayah[92][8]
\end{Arabic}}
\flushleft{\begin{hindi}
रहा वह व्यक्ति जिसने कंजूसी की और बेपरवाही बरती,
\end{hindi}}
\flushright{\begin{Arabic}
\quranayah[92][9]
\end{Arabic}}
\flushleft{\begin{hindi}
और अच्छी चीज़ को झुठला दिया,
\end{hindi}}
\flushright{\begin{Arabic}
\quranayah[92][10]
\end{Arabic}}
\flushleft{\begin{hindi}
हम उसे सहज ढंग से उस चीज़ का पात्र बना देंगे, जो कठिन चीज़ (कष्ट-साध्य) है
\end{hindi}}
\flushright{\begin{Arabic}
\quranayah[92][11]
\end{Arabic}}
\flushleft{\begin{hindi}
और उसका माल उसके कुछ काम न आएगा, जब वह (सिर के बल) खड्ड में गिरेगा
\end{hindi}}
\flushright{\begin{Arabic}
\quranayah[92][12]
\end{Arabic}}
\flushleft{\begin{hindi}
निस्संदेह हमारे ज़िम्मे है मार्ग दिखाना
\end{hindi}}
\flushright{\begin{Arabic}
\quranayah[92][13]
\end{Arabic}}
\flushleft{\begin{hindi}
और वास्तव में हमारे अधिकार में है आख़िरत और दुनिया भी
\end{hindi}}
\flushright{\begin{Arabic}
\quranayah[92][14]
\end{Arabic}}
\flushleft{\begin{hindi}
अतः मैंने तुम्हें दहकती आग से सावधान कर दिया
\end{hindi}}
\flushright{\begin{Arabic}
\quranayah[92][15]
\end{Arabic}}
\flushleft{\begin{hindi}
इसमें बस वही पड़ेगा जो बड़ा ही अभागा होगा,
\end{hindi}}
\flushright{\begin{Arabic}
\quranayah[92][16]
\end{Arabic}}
\flushleft{\begin{hindi}
जिसने झुठलाया और मुँह फेरा
\end{hindi}}
\flushright{\begin{Arabic}
\quranayah[92][17]
\end{Arabic}}
\flushleft{\begin{hindi}
और उससे बच जाएगा वह अत्यन्त परहेज़गार व्यक्ति,
\end{hindi}}
\flushright{\begin{Arabic}
\quranayah[92][18]
\end{Arabic}}
\flushleft{\begin{hindi}
जो अपना माल देकर अपने आपको निखारता है
\end{hindi}}
\flushright{\begin{Arabic}
\quranayah[92][19]
\end{Arabic}}
\flushleft{\begin{hindi}
और हाल यह है कि किसी का उसपर उपकार नहीं कि उसका बदला दिया जा रहा हो,
\end{hindi}}
\flushright{\begin{Arabic}
\quranayah[92][20]
\end{Arabic}}
\flushleft{\begin{hindi}
बल्कि इससे अभीष्ट केवल उसके अपने उच्च रब के मुख (प्रसन्नता) की चाह है
\end{hindi}}
\flushright{\begin{Arabic}
\quranayah[92][21]
\end{Arabic}}
\flushleft{\begin{hindi}
और वह शीघ्र ही राज़ी हो जाएगा
\end{hindi}}
\chapter{Ad-Duha (The Brightness of the Day)}
\begin{Arabic}
\Huge{\centerline{\basmalah}}\end{Arabic}
\flushright{\begin{Arabic}
\quranayah[93][1]
\end{Arabic}}
\flushleft{\begin{hindi}
साक्षी है चढ़ता दिन,
\end{hindi}}
\flushright{\begin{Arabic}
\quranayah[93][2]
\end{Arabic}}
\flushleft{\begin{hindi}
और रात जबकि उसका सन्नाटा छा जाए
\end{hindi}}
\flushright{\begin{Arabic}
\quranayah[93][3]
\end{Arabic}}
\flushleft{\begin{hindi}
तुम्हारे रब ने तुम्हें न तो विदा किया और न वह बेज़ार (अप्रसन्न) हुआ
\end{hindi}}
\flushright{\begin{Arabic}
\quranayah[93][4]
\end{Arabic}}
\flushleft{\begin{hindi}
और निश्चय ही बाद में आनेवाली (अवधि) तुम्हारे लिए पहलेवाली से उत्तम है
\end{hindi}}
\flushright{\begin{Arabic}
\quranayah[93][5]
\end{Arabic}}
\flushleft{\begin{hindi}
और शीघ्र ही तुम्हारा रब तुम्हें प्रदान करेगा कि तुम प्रसन्न हो जाओगे
\end{hindi}}
\flushright{\begin{Arabic}
\quranayah[93][6]
\end{Arabic}}
\flushleft{\begin{hindi}
क्या ऐसा नहीं कि उसने तुम्हें अनाथ पाया तो ठिकाना दिया?
\end{hindi}}
\flushright{\begin{Arabic}
\quranayah[93][7]
\end{Arabic}}
\flushleft{\begin{hindi}
और तुम्हें मार्ग से अपरिचित पाया तो मार्ग दिखाया?
\end{hindi}}
\flushright{\begin{Arabic}
\quranayah[93][8]
\end{Arabic}}
\flushleft{\begin{hindi}
और तुम्हें निर्धन पाया तो समृद्ध कर दिया?
\end{hindi}}
\flushright{\begin{Arabic}
\quranayah[93][9]
\end{Arabic}}
\flushleft{\begin{hindi}
अतः जो अनाथ हो उसे मत दबाना,
\end{hindi}}
\flushright{\begin{Arabic}
\quranayah[93][10]
\end{Arabic}}
\flushleft{\begin{hindi}
और जो माँगता हो उसे न झिझकना,
\end{hindi}}
\flushright{\begin{Arabic}
\quranayah[93][11]
\end{Arabic}}
\flushleft{\begin{hindi}
और जो तुम्हें रब की अनुकम्पा है, उसे बयान करते रहो
\end{hindi}}
\chapter{Al-Inshirah (The Expansion)}
\begin{Arabic}
\Huge{\centerline{\basmalah}}\end{Arabic}
\flushright{\begin{Arabic}
\quranayah[94][1]
\end{Arabic}}
\flushleft{\begin{hindi}
क्या ऐसा नहीं कि हमने तुम्हारा सीना तुम्हारे लिए खोल दिया?
\end{hindi}}
\flushright{\begin{Arabic}
\quranayah[94][2]
\end{Arabic}}
\flushleft{\begin{hindi}
और तुमपर से तुम्हारा बोझ उतार दिया,
\end{hindi}}
\flushright{\begin{Arabic}
\quranayah[94][3]
\end{Arabic}}
\flushleft{\begin{hindi}
जो तुम्हारी कमर तोड़े डाल रहा था?
\end{hindi}}
\flushright{\begin{Arabic}
\quranayah[94][4]
\end{Arabic}}
\flushleft{\begin{hindi}
और तुम्हारे लिए तुम्हारे ज़िक्र को ऊँचा कर दिया?
\end{hindi}}
\flushright{\begin{Arabic}
\quranayah[94][5]
\end{Arabic}}
\flushleft{\begin{hindi}
अतः निस्संदेह कठिनाई के साथ आसानी भी है
\end{hindi}}
\flushright{\begin{Arabic}
\quranayah[94][6]
\end{Arabic}}
\flushleft{\begin{hindi}
निस्संदेह कठिनाई के साथ आसानी भी है
\end{hindi}}
\flushright{\begin{Arabic}
\quranayah[94][7]
\end{Arabic}}
\flushleft{\begin{hindi}
अतः जब निवृत हो तो परिश्रम में लग जाओ,
\end{hindi}}
\flushright{\begin{Arabic}
\quranayah[94][8]
\end{Arabic}}
\flushleft{\begin{hindi}
और अपने रब से लौ लगाओ
\end{hindi}}
\chapter{At-Tin (The Fig)}
\begin{Arabic}
\Huge{\centerline{\basmalah}}\end{Arabic}
\flushright{\begin{Arabic}
\quranayah[95][1]
\end{Arabic}}
\flushleft{\begin{hindi}
साक्षी है तीन और ज़ैतून
\end{hindi}}
\flushright{\begin{Arabic}
\quranayah[95][2]
\end{Arabic}}
\flushleft{\begin{hindi}
और तूर सीनीन,
\end{hindi}}
\flushright{\begin{Arabic}
\quranayah[95][3]
\end{Arabic}}
\flushleft{\begin{hindi}
और यह शान्तिपूर्ण भूमि (मक्का)
\end{hindi}}
\flushright{\begin{Arabic}
\quranayah[95][4]
\end{Arabic}}
\flushleft{\begin{hindi}
निस्संदेह हमने मनुष्य को सर्वोत्तम संरचना के साथ पैदा किया
\end{hindi}}
\flushright{\begin{Arabic}
\quranayah[95][5]
\end{Arabic}}
\flushleft{\begin{hindi}
फिर हमने उसे निकृष्टतम दशा की ओर लौटा दिया, जबकि वह स्वयं गिरनेवाला बना
\end{hindi}}
\flushright{\begin{Arabic}
\quranayah[95][6]
\end{Arabic}}
\flushleft{\begin{hindi}
सिवाय उन लोगों के जो ईमान लाए और जिन्होंने अच्छे कर्म किए, तो उनके लिए कभी न समाप्त होनेवाला बदला है
\end{hindi}}
\flushright{\begin{Arabic}
\quranayah[95][7]
\end{Arabic}}
\flushleft{\begin{hindi}
अब इसके बाद क्या है, जो बदले के विषय में तुम्हें झुठलाए?
\end{hindi}}
\flushright{\begin{Arabic}
\quranayah[95][8]
\end{Arabic}}
\flushleft{\begin{hindi}
क्या अल्लाह सब हाकिमों से बड़ा हाकिम नहीं हैं?
\end{hindi}}
\chapter{Al-'Alaq (The Clot)}
\begin{Arabic}
\Huge{\centerline{\basmalah}}\end{Arabic}
\flushright{\begin{Arabic}
\quranayah[96][1]
\end{Arabic}}
\flushleft{\begin{hindi}
पढ़ो, अपने रब के नाम के साथ जिसने पैदा किया,
\end{hindi}}
\flushright{\begin{Arabic}
\quranayah[96][2]
\end{Arabic}}
\flushleft{\begin{hindi}
पैदा किया मनुष्य को जमे हुए ख़ून के एक लोथड़े से
\end{hindi}}
\flushright{\begin{Arabic}
\quranayah[96][3]
\end{Arabic}}
\flushleft{\begin{hindi}
पढ़ो, हाल यह है कि तुम्हारा रब बड़ा ही उदार है,
\end{hindi}}
\flushright{\begin{Arabic}
\quranayah[96][4]
\end{Arabic}}
\flushleft{\begin{hindi}
जिसने क़लम के द्वारा शिक्षा दी,
\end{hindi}}
\flushright{\begin{Arabic}
\quranayah[96][5]
\end{Arabic}}
\flushleft{\begin{hindi}
मनुष्य को वह ज्ञान प्रदान किया जिस वह न जानता था
\end{hindi}}
\flushright{\begin{Arabic}
\quranayah[96][6]
\end{Arabic}}
\flushleft{\begin{hindi}
कदापि नहीं, मनुष्य सरकशी करता है,
\end{hindi}}
\flushright{\begin{Arabic}
\quranayah[96][7]
\end{Arabic}}
\flushleft{\begin{hindi}
इसलिए कि वह अपने आपको आत्मनिर्भर देखता है
\end{hindi}}
\flushright{\begin{Arabic}
\quranayah[96][8]
\end{Arabic}}
\flushleft{\begin{hindi}
निश्चय ही तुम्हारे रब ही की ओर पलटना है
\end{hindi}}
\flushright{\begin{Arabic}
\quranayah[96][9]
\end{Arabic}}
\flushleft{\begin{hindi}
क्या तुमने देखा उस व्यक्ति को
\end{hindi}}
\flushright{\begin{Arabic}
\quranayah[96][10]
\end{Arabic}}
\flushleft{\begin{hindi}
जो एक बन्दे को रोकता है, जब वह नमाज़ अदा करता है? -
\end{hindi}}
\flushright{\begin{Arabic}
\quranayah[96][11]
\end{Arabic}}
\flushleft{\begin{hindi}
तुम्हारा क्या विचार है? यदि वह सीधे मार्ग पर हो,
\end{hindi}}
\flushright{\begin{Arabic}
\quranayah[96][12]
\end{Arabic}}
\flushleft{\begin{hindi}
या परहेज़गारी का हुक्म दे (उसके अच्छा होने में क्या संदेह है)
\end{hindi}}
\flushright{\begin{Arabic}
\quranayah[96][13]
\end{Arabic}}
\flushleft{\begin{hindi}
तुम्हारा क्या विचार है? यदि उस (रोकनेवाले) ने झुठलाया और मुँह मोड़ा (तो उसके बुरा होने में क्या संदेह है) -
\end{hindi}}
\flushright{\begin{Arabic}
\quranayah[96][14]
\end{Arabic}}
\flushleft{\begin{hindi}
क्या उसने नहीं जाना कि अल्लाह देख रहा है?
\end{hindi}}
\flushright{\begin{Arabic}
\quranayah[96][15]
\end{Arabic}}
\flushleft{\begin{hindi}
कदापि नहीं, यदि वह बाज़ न आया तो हम चोटी पकड़कर घसीटेंगे,
\end{hindi}}
\flushright{\begin{Arabic}
\quranayah[96][16]
\end{Arabic}}
\flushleft{\begin{hindi}
झूठी, ख़ताकार चोटी
\end{hindi}}
\flushright{\begin{Arabic}
\quranayah[96][17]
\end{Arabic}}
\flushleft{\begin{hindi}
अब बुला ले वह अपनी मजलिस को!
\end{hindi}}
\flushright{\begin{Arabic}
\quranayah[96][18]
\end{Arabic}}
\flushleft{\begin{hindi}
हम भी बुलाए लेते है सिपाहियों को
\end{hindi}}
\flushright{\begin{Arabic}
\quranayah[96][19]
\end{Arabic}}
\flushleft{\begin{hindi}
कदापि नहीं, उसकी बात न मानो और सजदे करते और क़रीब होते रहो
\end{hindi}}
\chapter{Al-Qadr (The Majesty)}
\begin{Arabic}
\Huge{\centerline{\basmalah}}\end{Arabic}
\flushright{\begin{Arabic}
\quranayah[97][1]
\end{Arabic}}
\flushleft{\begin{hindi}
हमने इसे क़द्र की रात में अवतरित किया
\end{hindi}}
\flushright{\begin{Arabic}
\quranayah[97][2]
\end{Arabic}}
\flushleft{\begin{hindi}
और तुम्हें क्या मालूम कि क़द्र की रात क्या है?
\end{hindi}}
\flushright{\begin{Arabic}
\quranayah[97][3]
\end{Arabic}}
\flushleft{\begin{hindi}
क़द्र की रात उत्तम है हज़ार महीनों से,
\end{hindi}}
\flushright{\begin{Arabic}
\quranayah[97][4]
\end{Arabic}}
\flushleft{\begin{hindi}
उसमें फ़रिश्तें और रूह हर महत्वपूर्ण मामलें में अपने रब की अनुमति से उतरते है
\end{hindi}}
\flushright{\begin{Arabic}
\quranayah[97][5]
\end{Arabic}}
\flushleft{\begin{hindi}
वह रात पूर्णतः शान्ति और सलामती है, उषाकाल के उदय होने तक
\end{hindi}}
\chapter{Al-Bayyinah (The Clear Evidence)}
\begin{Arabic}
\Huge{\centerline{\basmalah}}\end{Arabic}
\flushright{\begin{Arabic}
\quranayah[98][1]
\end{Arabic}}
\flushleft{\begin{hindi}
किताबवालों और मुशरिकों (बहुदेववादियों) में से जिन लोगों ने इनकार किया वे कुफ़्र (इनकार) से अलग होनेवाले नहीं जब तक कि उनके पास स्पष्ट प्रमाण न आ जाए;
\end{hindi}}
\flushright{\begin{Arabic}
\quranayah[98][2]
\end{Arabic}}
\flushleft{\begin{hindi}
अल्लाह की ओर से एक रसूल पवित्र पृष्ठों को पढ़ता हुआ;
\end{hindi}}
\flushright{\begin{Arabic}
\quranayah[98][3]
\end{Arabic}}
\flushleft{\begin{hindi}
जिनमें ठोस और ठीक आदेश अंकित हों,
\end{hindi}}
\flushright{\begin{Arabic}
\quranayah[98][4]
\end{Arabic}}
\flushleft{\begin{hindi}
हालाँकि जिन्हें किताब दी गई थी। वे इसके पश्चात फूट में पड़े कि उनके पास स्पष्ट प्रमाण आ चुका था
\end{hindi}}
\flushright{\begin{Arabic}
\quranayah[98][5]
\end{Arabic}}
\flushleft{\begin{hindi}
और उन्हें आदेश भी बस यही दिया गया था कि वे अल्लाह की बन्दगी करे निष्ठा एवं विनयशीलता को उसके लिए विशिष्ट करके, बिलकुल एकाग्र होकर, और नमाज़ की पाबन्दी करें और ज़कात दे। और यही है सत्यवादी समुदाय का धर्म
\end{hindi}}
\flushright{\begin{Arabic}
\quranayah[98][6]
\end{Arabic}}
\flushleft{\begin{hindi}
निस्संदेह किताबवालों और मुशरिकों (बहुदेववादियों) में से जिन लोगों ने इनकार किया है, वे जहन्नम की आग में पड़ेगे; उसमें सदैव रहने के लिए। वही पैदा किए गए प्राणियों में सबसे बुरे है
\end{hindi}}
\flushright{\begin{Arabic}
\quranayah[98][7]
\end{Arabic}}
\flushleft{\begin{hindi}
किन्तु निश्चय ही वे लोग, जो ईमान लाए और उन्होंने अच्छे कर्म किए पैदा किए गए प्राणियों में सबसे अच्छे है
\end{hindi}}
\flushright{\begin{Arabic}
\quranayah[98][8]
\end{Arabic}}
\flushleft{\begin{hindi}
उनका बदला उनके अपने रब के पास सदाबहार बाग़ है, जिनके नीचे नहरें बह रही होंगी। उनमें वे सदैव रहेंगे। अल्लाह उनसे राज़ी हुआ और वे उससे राज़ी हुए। यह कुछ उसके लिए है, जो अपने रब से डरा
\end{hindi}}
\chapter{Al-Zilzal (The Shaking)}
\begin{Arabic}
\Huge{\centerline{\basmalah}}\end{Arabic}
\flushright{\begin{Arabic}
\quranayah[99][1]
\end{Arabic}}
\flushleft{\begin{hindi}
जब धरती इस प्रकार हिला डाली जाएगी जैसा उसे हिलाया जाना है,
\end{hindi}}
\flushright{\begin{Arabic}
\quranayah[99][2]
\end{Arabic}}
\flushleft{\begin{hindi}
और धरती अपने बोझ बाहर निकाल देगी,
\end{hindi}}
\flushright{\begin{Arabic}
\quranayah[99][3]
\end{Arabic}}
\flushleft{\begin{hindi}
और मनुष्य कहेगा, "उसे क्या हो गया है?"
\end{hindi}}
\flushright{\begin{Arabic}
\quranayah[99][4]
\end{Arabic}}
\flushleft{\begin{hindi}
उस दिन वह अपना वृत्तांत सुनाएगी,
\end{hindi}}
\flushright{\begin{Arabic}
\quranayah[99][5]
\end{Arabic}}
\flushleft{\begin{hindi}
इस कारण कि तुम्हारे रब ने उसे यही संकेत किया होगा
\end{hindi}}
\flushright{\begin{Arabic}
\quranayah[99][6]
\end{Arabic}}
\flushleft{\begin{hindi}
उस दिन लोग अलग-अलग निकलेंगे, ताकि उन्हें कर्म दिखाए जाएँ
\end{hindi}}
\flushright{\begin{Arabic}
\quranayah[99][7]
\end{Arabic}}
\flushleft{\begin{hindi}
अतः जो कोई कणभर भी नेकी करेगा, वह उसे देख लेगा,
\end{hindi}}
\flushright{\begin{Arabic}
\quranayah[99][8]
\end{Arabic}}
\flushleft{\begin{hindi}
और जो कोई कणभर भी बुराई करेगा, वह भी उसे देख लेगा
\end{hindi}}
\chapter{Al-'Adiyat (The Assaulters)}
\begin{Arabic}
\Huge{\centerline{\basmalah}}\end{Arabic}
\flushright{\begin{Arabic}
\quranayah[100][1]
\end{Arabic}}
\flushleft{\begin{hindi}
साक्षी है जो हाँफते-फुँकार मारते हुए दौड़ते है,
\end{hindi}}
\flushright{\begin{Arabic}
\quranayah[100][2]
\end{Arabic}}
\flushleft{\begin{hindi}
फिर ठोकरों से चिनगारियाँ निकालते है,
\end{hindi}}
\flushright{\begin{Arabic}
\quranayah[100][3]
\end{Arabic}}
\flushleft{\begin{hindi}
फिर सुबह सवेरे धावा मारते होते है,
\end{hindi}}
\flushright{\begin{Arabic}
\quranayah[100][4]
\end{Arabic}}
\flushleft{\begin{hindi}
उसमें उठाया उन्होंने गर्द-गुबार
\end{hindi}}
\flushright{\begin{Arabic}
\quranayah[100][5]
\end{Arabic}}
\flushleft{\begin{hindi}
और इसी हाल में वे दल में जा घुसे
\end{hindi}}
\flushright{\begin{Arabic}
\quranayah[100][6]
\end{Arabic}}
\flushleft{\begin{hindi}
निस्संदेह मनुष्य अपने रब का बड़ा अकृतज्ञ हैं,
\end{hindi}}
\flushright{\begin{Arabic}
\quranayah[100][7]
\end{Arabic}}
\flushleft{\begin{hindi}
और निश्चय ही वह स्वयं इसपर गवाह है!
\end{hindi}}
\flushright{\begin{Arabic}
\quranayah[100][8]
\end{Arabic}}
\flushleft{\begin{hindi}
और निश्चय ही वह धन के मोह में बड़ा दृढ़ है
\end{hindi}}
\flushright{\begin{Arabic}
\quranayah[100][9]
\end{Arabic}}
\flushleft{\begin{hindi}
तो क्या वह जानता नहीं जब उगवला लिया जाएगा तो क़ब्रों में है
\end{hindi}}
\flushright{\begin{Arabic}
\quranayah[100][10]
\end{Arabic}}
\flushleft{\begin{hindi}
और स्पष्ट अनावृत्त कर दिया जाएगा तो कुछ सीनों में है
\end{hindi}}
\flushright{\begin{Arabic}
\quranayah[100][11]
\end{Arabic}}
\flushleft{\begin{hindi}
निस्संदेह उनका रब उस दिन उनकी पूरी ख़बर रखता होगा
\end{hindi}}
\chapter{Al-Qari'ah (The Calamity)}
\begin{Arabic}
\Huge{\centerline{\basmalah}}\end{Arabic}
\flushright{\begin{Arabic}
\quranayah[101][1]
\end{Arabic}}
\flushleft{\begin{hindi}
वह खड़खड़ानेवाली!
\end{hindi}}
\flushright{\begin{Arabic}
\quranayah[101][2]
\end{Arabic}}
\flushleft{\begin{hindi}
क्या है वह खड़खड़ानेवाली?
\end{hindi}}
\flushright{\begin{Arabic}
\quranayah[101][3]
\end{Arabic}}
\flushleft{\begin{hindi}
और तुम्हें क्या मालूम कि क्या है वह खड़खड़ानेवाली?
\end{hindi}}
\flushright{\begin{Arabic}
\quranayah[101][4]
\end{Arabic}}
\flushleft{\begin{hindi}
जिस दिन लोग बिखरे हुए पतंगों के सदृश हो जाएँगें,
\end{hindi}}
\flushright{\begin{Arabic}
\quranayah[101][5]
\end{Arabic}}
\flushleft{\begin{hindi}
और पहाड़ के धुन के हुए रंग-बिरंग के ऊन जैसे हो जाएँगे
\end{hindi}}
\flushright{\begin{Arabic}
\quranayah[101][6]
\end{Arabic}}
\flushleft{\begin{hindi}
फिर जिस किसी के वज़न भारी होंगे,
\end{hindi}}
\flushright{\begin{Arabic}
\quranayah[101][7]
\end{Arabic}}
\flushleft{\begin{hindi}
वह मनभाते जीवन में रहेगा
\end{hindi}}
\flushright{\begin{Arabic}
\quranayah[101][8]
\end{Arabic}}
\flushleft{\begin{hindi}
और रहा वह व्यक्ति जिसके वज़न हलके होंगे,
\end{hindi}}
\flushright{\begin{Arabic}
\quranayah[101][9]
\end{Arabic}}
\flushleft{\begin{hindi}
उसकी माँ होगी गहरा खड्ड
\end{hindi}}
\flushright{\begin{Arabic}
\quranayah[101][10]
\end{Arabic}}
\flushleft{\begin{hindi}
और तुम्हें क्या मालूम कि वह क्या है?
\end{hindi}}
\flushright{\begin{Arabic}
\quranayah[101][11]
\end{Arabic}}
\flushleft{\begin{hindi}
आग है दहकती हुई
\end{hindi}}
\chapter{At-Takathur (The Abundance of Wealth)}
\begin{Arabic}
\Huge{\centerline{\basmalah}}\end{Arabic}
\flushright{\begin{Arabic}
\quranayah[102][1]
\end{Arabic}}
\flushleft{\begin{hindi}
तुम्हें एक-दूसरे के मुक़ाबले में बहुतायत के प्रदर्शन और घमंड ने ग़फ़़लत में डाल रखा है,
\end{hindi}}
\flushright{\begin{Arabic}
\quranayah[102][2]
\end{Arabic}}
\flushleft{\begin{hindi}
यहाँ तक कि तुम क़ब्रिस्तानों में पहुँच गए
\end{hindi}}
\flushright{\begin{Arabic}
\quranayah[102][3]
\end{Arabic}}
\flushleft{\begin{hindi}
कुछ नहीं, तुम शीघ्र ही जान लोगे
\end{hindi}}
\flushright{\begin{Arabic}
\quranayah[102][4]
\end{Arabic}}
\flushleft{\begin{hindi}
फिर, कुछ नहीं, तुम्हें शीघ्र ही मालूम हो जाएगा -
\end{hindi}}
\flushright{\begin{Arabic}
\quranayah[102][5]
\end{Arabic}}
\flushleft{\begin{hindi}
कुछ नहीं, अगर तुम विश्वसनीय ज्ञान के रूप में जान लो! (तो तुम धन-दौलत के पुजारी न बनो) -
\end{hindi}}
\flushright{\begin{Arabic}
\quranayah[102][6]
\end{Arabic}}
\flushleft{\begin{hindi}
अवश्य ही तुम भड़कती आग से दो-चार होगे
\end{hindi}}
\flushright{\begin{Arabic}
\quranayah[102][7]
\end{Arabic}}
\flushleft{\begin{hindi}
फिर सुनो, उसे अवश्य देखोगे इस दशा में कि वह यथावत विश्वास होगा
\end{hindi}}
\flushright{\begin{Arabic}
\quranayah[102][8]
\end{Arabic}}
\flushleft{\begin{hindi}
फिर निश्चय ही उस दिन तुमसे नेमतों के बारे में पूछा जाएगा
\end{hindi}}
\chapter{Al-'Asr (The Time)}
\begin{Arabic}
\Huge{\centerline{\basmalah}}\end{Arabic}
\flushright{\begin{Arabic}
\quranayah[103][1]
\end{Arabic}}
\flushleft{\begin{hindi}
गवाह है गुज़रता समय,
\end{hindi}}
\flushright{\begin{Arabic}
\quranayah[103][2]
\end{Arabic}}
\flushleft{\begin{hindi}
कि वास्तव में मनुष्य घाटे में है,
\end{hindi}}
\flushright{\begin{Arabic}
\quranayah[103][3]
\end{Arabic}}
\flushleft{\begin{hindi}
सिवाय उन लोगों के जो ईमान लाए और अच्छे कर्म किए और एक-दूसरे को हक़ की ताकीद की, और एक-दूसरे को धैर्य की ताकीद की
\end{hindi}}
\chapter{Al-Humazah (The Slanderer)}
\begin{Arabic}
\Huge{\centerline{\basmalah}}\end{Arabic}
\flushright{\begin{Arabic}
\quranayah[104][1]
\end{Arabic}}
\flushleft{\begin{hindi}
तबाही है हर कचो के लगानेवाले, ऐब निकालनेवाले के लिए,
\end{hindi}}
\flushright{\begin{Arabic}
\quranayah[104][2]
\end{Arabic}}
\flushleft{\begin{hindi}
जो माल इकट्ठा करता और उसे गिनता रहा
\end{hindi}}
\flushright{\begin{Arabic}
\quranayah[104][3]
\end{Arabic}}
\flushleft{\begin{hindi}
समझता है कि उसके माल ने उसे अमर कर दिया
\end{hindi}}
\flushright{\begin{Arabic}
\quranayah[104][4]
\end{Arabic}}
\flushleft{\begin{hindi}
कदापि नहीं, वह चूर-चूर कर देनेवाली में फेंक दिया जाएगा,
\end{hindi}}
\flushright{\begin{Arabic}
\quranayah[104][5]
\end{Arabic}}
\flushleft{\begin{hindi}
और तुम्हें क्या मालूम कि वह चूर-चूर कर देनेवाली क्या है?
\end{hindi}}
\flushright{\begin{Arabic}
\quranayah[104][6]
\end{Arabic}}
\flushleft{\begin{hindi}
वह अल्लाह की दहकाई हुई आग है,
\end{hindi}}
\flushright{\begin{Arabic}
\quranayah[104][7]
\end{Arabic}}
\flushleft{\begin{hindi}
जो झाँक लेती है दिलों को
\end{hindi}}
\flushright{\begin{Arabic}
\quranayah[104][8]
\end{Arabic}}
\flushleft{\begin{hindi}
वह उनपर ढाँककर बन्द कर दी गई होगी,
\end{hindi}}
\flushright{\begin{Arabic}
\quranayah[104][9]
\end{Arabic}}
\flushleft{\begin{hindi}
लम्बे-लम्बे स्तम्भों में
\end{hindi}}
\chapter{Al-Fil (The Elephant)}
\begin{Arabic}
\Huge{\centerline{\basmalah}}\end{Arabic}
\flushright{\begin{Arabic}
\quranayah[105][1]
\end{Arabic}}
\flushleft{\begin{hindi}
क्या तुमने देखा नहीं कि तुम्हारे रब ने हाथीवालों के साथ कैसा बरताव किया?
\end{hindi}}
\flushright{\begin{Arabic}
\quranayah[105][2]
\end{Arabic}}
\flushleft{\begin{hindi}
क्या उसने उनकी चाल को अकारथ नहीं कर दिया?
\end{hindi}}
\flushright{\begin{Arabic}
\quranayah[105][3]
\end{Arabic}}
\flushleft{\begin{hindi}
और उनपर नियुक्त होने को झुंड के झुंड पक्षी भेजे,
\end{hindi}}
\flushright{\begin{Arabic}
\quranayah[105][4]
\end{Arabic}}
\flushleft{\begin{hindi}
उनपर कंकरीले पत्थर मार रहे थे
\end{hindi}}
\flushright{\begin{Arabic}
\quranayah[105][5]
\end{Arabic}}
\flushleft{\begin{hindi}
अन्ततः उन्हें ऐसा कर दिया, जैसे खाने का भूसा हो
\end{hindi}}
\chapter{Al-Quraish (The Quraish)}
\begin{Arabic}
\Huge{\centerline{\basmalah}}\end{Arabic}
\flushright{\begin{Arabic}
\quranayah[106][1]
\end{Arabic}}
\flushleft{\begin{hindi}
कितना है क़ुरैश को लगाए और परचाए रखना,
\end{hindi}}
\flushright{\begin{Arabic}
\quranayah[106][2]
\end{Arabic}}
\flushleft{\begin{hindi}
लगाए और परचाए रखना उन्हें जाड़े और गर्मी की यात्रा से
\end{hindi}}
\flushright{\begin{Arabic}
\quranayah[106][3]
\end{Arabic}}
\flushleft{\begin{hindi}
अतः उन्हें चाहिए कि इस घर (काबा) के रब की बन्दगी करे,
\end{hindi}}
\flushright{\begin{Arabic}
\quranayah[106][4]
\end{Arabic}}
\flushleft{\begin{hindi}
जिसने उन्हें खिलाकर भूख से बचाया और निश्चिन्तता प्रदान करके भय से बचाया
\end{hindi}}
\chapter{Al-Ma'un (Acts of Kindness)}
\begin{Arabic}
\Huge{\centerline{\basmalah}}\end{Arabic}
\flushright{\begin{Arabic}
\quranayah[107][1]
\end{Arabic}}
\flushleft{\begin{hindi}
क्या तुमने उसे देखा जो दीन को झुठलाता है?
\end{hindi}}
\flushright{\begin{Arabic}
\quranayah[107][2]
\end{Arabic}}
\flushleft{\begin{hindi}
वही तो है जो अनाथ को धक्के देता है,
\end{hindi}}
\flushright{\begin{Arabic}
\quranayah[107][3]
\end{Arabic}}
\flushleft{\begin{hindi}
और मुहताज के खिलाने पर नहीं उकसाता
\end{hindi}}
\flushright{\begin{Arabic}
\quranayah[107][4]
\end{Arabic}}
\flushleft{\begin{hindi}
अतः तबाही है उन नमाज़ियों के लिए,
\end{hindi}}
\flushright{\begin{Arabic}
\quranayah[107][5]
\end{Arabic}}
\flushleft{\begin{hindi}
जो अपनी नमाज़ से ग़ाफिल (असावधान) हैं,
\end{hindi}}
\flushright{\begin{Arabic}
\quranayah[107][6]
\end{Arabic}}
\flushleft{\begin{hindi}
जो दिखावे के लिए कार्य करते हैं,
\end{hindi}}
\flushright{\begin{Arabic}
\quranayah[107][7]
\end{Arabic}}
\flushleft{\begin{hindi}
और साधारण बरतने की चीज़ भी किसी को नहीं देते
\end{hindi}}
\chapter{Al-Kauthar (The Abundance of Good)}
\begin{Arabic}
\Huge{\centerline{\basmalah}}\end{Arabic}
\flushright{\begin{Arabic}
\quranayah[108][1]
\end{Arabic}}
\flushleft{\begin{hindi}
निश्चय ही हमने तुम्हें कौसर प्रदान किया,
\end{hindi}}
\flushright{\begin{Arabic}
\quranayah[108][2]
\end{Arabic}}
\flushleft{\begin{hindi}
अतः तुम अपने रब ही के लिए नमाज़ पढ़ो और (उसी के दिन) क़़ुरबानी करो
\end{hindi}}
\flushright{\begin{Arabic}
\quranayah[108][3]
\end{Arabic}}
\flushleft{\begin{hindi}
निस्संदेह तुम्हारा जो वैरी है वही जड़कटा है
\end{hindi}}
\chapter{Al-Kafirun (The Disbelievers)}
\begin{Arabic}
\Huge{\centerline{\basmalah}}\end{Arabic}
\flushright{\begin{Arabic}
\quranayah[109][1]
\end{Arabic}}
\flushleft{\begin{hindi}
कह दो, "ऐ इनकार करनेवालो!"
\end{hindi}}
\flushright{\begin{Arabic}
\quranayah[109][2]
\end{Arabic}}
\flushleft{\begin{hindi}
मैं वैसी बन्दगी नहीं करूँगा जैसी बन्दगी तुम करते हो,
\end{hindi}}
\flushright{\begin{Arabic}
\quranayah[109][3]
\end{Arabic}}
\flushleft{\begin{hindi}
और न तुम वैसी बन्दगी करनेवाले हो जैसी बन्दगी में करता हूँ
\end{hindi}}
\flushright{\begin{Arabic}
\quranayah[109][4]
\end{Arabic}}
\flushleft{\begin{hindi}
और न मैं वैसी बन्दगी करनेवाला हूँ जैसी बन्दगी तुमने की है
\end{hindi}}
\flushright{\begin{Arabic}
\quranayah[109][5]
\end{Arabic}}
\flushleft{\begin{hindi}
और न तुम वैसी बन्दगी करनेवाला हुए जैसी बन्दगी मैं करता हूँ
\end{hindi}}
\flushright{\begin{Arabic}
\quranayah[109][6]
\end{Arabic}}
\flushleft{\begin{hindi}
तुम्हारे लिए तूम्हारा धर्म है और मेरे लिए मेरा धर्म!"
\end{hindi}}
\chapter{An-Nasr (The Help)}
\begin{Arabic}
\Huge{\centerline{\basmalah}}\end{Arabic}
\flushright{\begin{Arabic}
\quranayah[110][1]
\end{Arabic}}
\flushleft{\begin{hindi}
जब अल्लाह की सहायता आ जाए और विजय प्राप्त हो,
\end{hindi}}
\flushright{\begin{Arabic}
\quranayah[110][2]
\end{Arabic}}
\flushleft{\begin{hindi}
और तुम लोगों को देखो कि वे अल्लाह के दीन (धर्म) में गिरोह के गिरोह प्रवेश कर रहे है,
\end{hindi}}
\flushright{\begin{Arabic}
\quranayah[110][3]
\end{Arabic}}
\flushleft{\begin{hindi}
तो अपने रब की प्रशंसा करो और उससे क्षमा चाहो। निस्संदेह वह बड़ा तौबा क़बूल करनेवाला है
\end{hindi}}
\chapter{Al-Lahab (The Flame)}
\begin{Arabic}
\Huge{\centerline{\basmalah}}\end{Arabic}
\flushright{\begin{Arabic}
\quranayah[111][1]
\end{Arabic}}
\flushleft{\begin{hindi}
टूट गए अबू लहब के दोनों हाथ और वह स्वयं भी विनष्ट हो गया!
\end{hindi}}
\flushright{\begin{Arabic}
\quranayah[111][2]
\end{Arabic}}
\flushleft{\begin{hindi}
न उसका माल उसके काम आया और न वह कुछ जो उसने कमाया
\end{hindi}}
\flushright{\begin{Arabic}
\quranayah[111][3]
\end{Arabic}}
\flushleft{\begin{hindi}
वह शीघ्र ही प्रज्वलित भड़कती आग में पड़ेगा,
\end{hindi}}
\flushright{\begin{Arabic}
\quranayah[111][4]
\end{Arabic}}
\flushleft{\begin{hindi}
और उसकी स्त्री भी ईधन लादनेवाली,
\end{hindi}}
\flushright{\begin{Arabic}
\quranayah[111][5]
\end{Arabic}}
\flushleft{\begin{hindi}
उसकी गरदन में खजूर के रेसों की बटी हुई रस्सी पड़ी है
\end{hindi}}
\chapter{Al-Ikhlas (The Unity)}
\begin{Arabic}
\Huge{\centerline{\basmalah}}\end{Arabic}
\flushright{\begin{Arabic}
\quranayah[112][1]
\end{Arabic}}
\flushleft{\begin{hindi}
कहो, "वह अल्लाह यकता है,
\end{hindi}}
\flushright{\begin{Arabic}
\quranayah[112][2]
\end{Arabic}}
\flushleft{\begin{hindi}
अल्लाह निरपेक्ष (और सर्वाधार) है,
\end{hindi}}
\flushright{\begin{Arabic}
\quranayah[112][3]
\end{Arabic}}
\flushleft{\begin{hindi}
न वह जनिता है और न जन्य,
\end{hindi}}
\flushright{\begin{Arabic}
\quranayah[112][4]
\end{Arabic}}
\flushleft{\begin{hindi}
और न कोई उसका समकक्ष है।"
\end{hindi}}
\chapter{Al-Falaq (The Dawn)}
\begin{Arabic}
\Huge{\centerline{\basmalah}}\end{Arabic}
\flushright{\begin{Arabic}
\quranayah[113][1]
\end{Arabic}}
\flushleft{\begin{hindi}
कहो, "मैं शरण लेता हूँ, प्रकट करनेवाले रब की,
\end{hindi}}
\flushright{\begin{Arabic}
\quranayah[113][2]
\end{Arabic}}
\flushleft{\begin{hindi}
जो कुछ भी उसने पैदा किया उसकी बुराई से,
\end{hindi}}
\flushright{\begin{Arabic}
\quranayah[113][3]
\end{Arabic}}
\flushleft{\begin{hindi}
और अँधेरे की बुराई से जबकि वह घुस आए,
\end{hindi}}
\flushright{\begin{Arabic}
\quranayah[113][4]
\end{Arabic}}
\flushleft{\begin{hindi}
और गाँठो में फूँक मारने-वालों की बुराई से,
\end{hindi}}
\flushright{\begin{Arabic}
\quranayah[113][5]
\end{Arabic}}
\flushleft{\begin{hindi}
और ईर्ष्यालु की बुराई से, जब वह ईर्ष्या करे।"
\end{hindi}}
\chapter{An-Nas (The Men)}
\begin{Arabic}
\Huge{\centerline{\basmalah}}\end{Arabic}
\flushright{\begin{Arabic}
\quranayah[114][1]
\end{Arabic}}
\flushleft{\begin{hindi}
कहो, "मैं शरण लेता हूँ मनुष्यों के रब की
\end{hindi}}
\flushright{\begin{Arabic}
\quranayah[114][2]
\end{Arabic}}
\flushleft{\begin{hindi}
मनुष्यों के सम्राट की
\end{hindi}}
\flushright{\begin{Arabic}
\quranayah[114][3]
\end{Arabic}}
\flushleft{\begin{hindi}
मनुष्यों के उपास्य की
\end{hindi}}
\flushright{\begin{Arabic}
\quranayah[114][4]
\end{Arabic}}
\flushleft{\begin{hindi}
वसवसा डालनेवाले, खिसक जानेवाले की बुराई से
\end{hindi}}
\flushright{\begin{Arabic}
\quranayah[114][5]
\end{Arabic}}
\flushleft{\begin{hindi}
जो मनुष्यों के सीनों में वसवसा डालता हैं
\end{hindi}}
\flushright{\begin{Arabic}
\quranayah[114][6]
\end{Arabic}}
\flushleft{\begin{hindi}
जो जिन्नों में से भी होता हैं और मनुष्यों में से भी
\end{hindi}}
